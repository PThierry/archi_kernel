%%
%%
%% archi_macroscopique.tex for  in /doctorat/ece/partenariat/cours/archi_kernel
%%
%% Made by Philippe THIERRY
%% Login   <Philippe THIERRYreseau-libre.net>
%%
%% Started on  Mon Sep  6 16:18:47 2010 Philippe THIERRY
%% Last update Wed Sep 29 16:07:10 2010 Philippe THIERRY
%%

\chapter{Architecture macroscopique}

\section{La d�composition des sources}

\paragraph{}

\section{Linux, un noyau modulaire}

\subsection{Principe d'un module}
\paragraph{}

\subsection{Le syst�me de gestion des modules et des d�pendances}
\paragraph{}

\subsection{Les modules et le boot : l'initramfs}
\paragraph{}

\section{Les diff�rents �l�ments constituants le noyau Linux}

\subsection{Le support de l'architecture}
\paragraph{}

\subsection{La gestion m�moire}
\paragraph{}

\subsection{Le gestionnaire d'interruptions}
\paragraph{}

\subsection{Les syscalls}
\paragraph{}

\subsection{Les IPC}
\paragraph{}

\subsection{L'ordonnanceur est la gestion des taches}
\paragraph{}

\subsection{Le syst�me de fichiers}
\paragraph{}

\subsection{La pile r�seau}
\paragraph{}

\subsection{La pile USB}
\paragraph{}

\subsection{La s�curit� et la cryptographie}
\paragraph{}

\subsection{Les pilotes de p�riph�riques}
\paragraph{}

\section{Coder dans le noyau Linux}

\subsection{Configurer et compiler un noyau Linux}
\paragraph{}

\subsection{Le syst�me de production du noyau}
\paragraph{}

