%%
%%
%% nf_hooks.tex for  in /doctorat/ece/partenariat/cours/archi_kernel
%%
%% Made by Philippe THIERRY
%% Login   <Philippe THIERRYreseau-libre.net>
%%
%% Started on  Fri Oct  1 13:35:49 2010 Philippe THIERRY
%% Last update lun. 30 mai 2011 11:45:48 CEST Philippe THIERRY
%%

\begin{pdfpic}
\scalebox{0.75}{
\begin{pspicture}(-1,-1)(20,8)
\definecolor{NF}{rgb}{0.7,1,1}
% incoming NIC
\psframe[linewidth=0.04cm](0,0)(1,0.7)
\psframe[linewidth=0.03cm](0.2,0.2)(0.6,0.5)
\psline[linewidth=0.06cm](1,0)(1,1)(1.1,1)
\rput[rt](0.5,-0.2){\text NIC}
% ISR and poller
\psframe[linewidth=0.04cm](2,2)(3,2.5)
\rput(2.5,2.25){\text ISR}
\psframe[linewidth=0.04cm](2,3.5)(3,4)
\rput(2.5,3.75){\text Poller}
% link from NIC to ISR & poller
\psline[linewidth=0.04cm]{->}(0.5,0.7)(2,2.25)
\psline[linewidth=0.04cm]{->}(0.5,0.7)(2,3.75)
% first temporal break
\psline[linewidth=0.04cm]{->}(3,2.25)(4,2.25)
\psline[linewidth=0.04cm]{->}(3,3.75)(4,3.75)
\psline[linewidth=0.04cm](4,1.5)(4,4.5)
\rput(4.25,3){\rotatebox{90}{\text first temporal break}}
\psline[linewidth=0.04cm](4.5,1.5)(4.5,4.5)
% L2 first pass
\psframe[linewidth=0.04cm](4.5,2)(6.5,3)
\rput(5.5,2.6){\text L2}
\rput(5.5,2.3){\text prerouting}
% In
\psline[linewidth=0.04cm](9,6)(8,6)(8,7)(9,7)
\rput(8.5,6.5){\text In}
% ip prerouting
\psframe[linewidth=0.04cm](7.5,4)(9.5,5)
\rput(8.5,4.6){\text IP}
\rput(8.5,4.3){\text Prerouting}
% ip forwarder
\psframe[linewidth=0.04cm](10,4)(12,5)
\rput(11,4.6){\text IP}
\rput(11,4.3){\text Forwarder}
% briding
\psframe[linewidth=0.04cm](7.5,2)(14.5,3)
\rput(11,2.5){\text Bridging}
% links from L2 first pass
\psline[linewidth=0.04cm]{->}(6.5,2.5)(7.5,2.5)
\psline[linewidth=0.04cm]{->}(6.5,3)(7.5,4.5)
\psline[linewidth=0.04cm]{->}(9.5,4.5)(10,4.5)
\psline[linewidth=0.04cm]{->}(8.5,5)(8.5,6)
% out
\psline[linewidth=0.04cm](13,6)(14,6)(14,7)(13,7)
\rput(13.5,6.5){\text Out}
\psline[linewidth=0.04cm]{->}(13.5,6)(13.5,5)
% ip postrouting
\psframe[linewidth=0.04cm](12.5,4)(14.5,5)
\rput(13.5,4.6){\text IP}
\rput(13.5,4.3){\text postrouting}
\psline[linewidth=0.04cm]{->}(12,4.5)(12.5,4.5)
% br postrouting
\psframe[linewidth=0.04cm](15.5,2)(17.5,3)
\rput(16.5,2.6){\text L2}
\rput(16.5,2.3){\text postrouting}
\psline[linewidth=0.04cm]{->}(14.5,4)(16.5,3)
% second temporal break
\psline[linewidth=0.04cm](17.5,1.5)(17.5,4.5)
\rput(17.75,3){\rotatebox{90}{\text second temporal break}}
\psline[linewidth=0.04cm](18,1.5)(18,4.5)
% output queue 1
\psframe[linewidth=0.04cm](18.3,2)(18.7,3)
\psline[linewidth=0.04cm](18.3,2.2)(18.7,2.2)
\psline[linewidth=0.04cm](18.3,2.4)(18.7,2.4)
\psline[linewidth=0.04cm](18.3,2.6)(18.7,2.6)
\psline[linewidth=0.04cm](18.3,2.8)(18.7,2.8)
% output queue 2
\psframe[linewidth=0.04cm](18.9,2)(19.3,2.8)
\psline[linewidth=0.04cm](18.9,2.2)(19.3,2.2)
\psline[linewidth=0.04cm](18.9,2.4)(19.3,2.4)
\psline[linewidth=0.04cm](18.9,2.6)(19.3,2.6)
% output NIC
\psframe[linewidth=0.04cm](18.3,0)(19.3,0.7)
\psframe[linewidth=0.03cm](18.5,0.2)(18.9,0.5)
\psline[linewidth=0.06cm](19.3,0)(19.3,1)(19.4,1)
\rput(18.8,-0.2){\text NIC}
% Let's mark the NF hooks now
\psline[linewidth=0.06cm,linecolor=red]{<-}(5.5,2)(5.5,0)
\rput[t](5.2,1.7){\text {\rotatebox{90}{\color{red}{NF\_BR\_PREROUTING}}}}
\psline[linewidth=0.06cm,linecolor=red]{<-}(8.5,4)(8.5,0)
\rput[t](8.2,1.7){\text {\rotatebox{90}{\color{red}{NF\_IP\_PREROUTING}}}}
\psline[linewidth=0.06cm,linecolor=red]{<-}(11,4)(11,0)
\rput[t](10.7,1.7){\text {\rotatebox{90}{\color{red}{NF\_IP\_FORWARD}}}}
\psline[linewidth=0.06cm,linecolor=red]{<-}(13.5,4)(13.5,0)
\rput[t](13.2,1.7){\text {\rotatebox{90}{\color{red}{NF\_IP\_POSTROUTING}}}}
\psline[linewidth=0.06cm,linecolor=red]{<-}(16.5,2)(16.5,0)
\rput[t](16.2,1.7){\text {\rotatebox{90}{\color{red}{NF\_BR\_POSTROUTING}}}}
\psline[linewidth=0.06cm,linecolor=red]{<-}(8,6.5)(7,6)(7,0)
\rput[t](6.7,1.7){\text {\rotatebox{90}{\color{red}{NF\_IP\_IN}}}}
\psline[linewidth=0.06cm,linecolor=red]{<-}(14,6.5)(15,6)(15,0)
\rput[t](14.7,1.7){\text {\rotatebox{90}{\color{red}{NF\_IP\_OUT}}}}
\end{pspicture}
}
\end{pdfpic}
