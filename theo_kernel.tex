%%
%%
%% theo_kernel.tex for  in /doctorat/ece/partenariat/cours/archi_kernel
%%
%% Made by Philippe THIERRY
%% Login   <Philippe THIERRYreseau-libre.net>
%%
%% Started on  Wed Sep 22 14:45:23 2010 Philippe THIERRY
%% Last update Fri Sep 24 16:10:04 2010 Philippe THIERRY
%%

\chapter{Un petit historique du noyau}

\section{Des premiers noyaux aux exokernels}

\subsection{L'invention du noyau}

\subsection{Les noyaux monolithiques}

\paragraph{}
Historiquement le plus ancien, le noyau monolithique est la cons�quence des ajouts incessants de
nouvelles fonctionnalit�s dans les noyaux au fur et � mesure de l'int�gration de nouvelle
technologies.\\
En effet, les premiers noyaux ne poss�daient pas de pile r�seau et avait un nombre tr�s faible de
drivers. Avec la cr�ation du r�seau d'une part et l'h�t�rog�n�isation du mat�riel d'autre part,
beaucoup de code fut ajout� aux noyaux.

\begin{figure}[ht]
%%
%%
%% kernel_monolithic.tex for  in /doctorat/ece/partenariat/cours/archi_kernel
%%
%% Made by Philippe THIERRY
%% Login   <Philippe THIERRYreseau-libre.net>
%%
%% Started on  Thu Sep 23 14:19:58 2010 Philippe THIERRY
%% Last update Thu Sep 23 15:45:50 2010 Philippe THIERRY
%%

\begin{pdfpic}
\scalebox{1}{
\begin{pspicture}(0,0)(13,8)
% first line seems not to be considered by pstricks... strange...
\psline{-}(0,0)(0,0)
% kernel block
\definecolor{KernBase}{rgb}{0,0,0.7}
\definecolor{KernLow}{rgb}{0,0,0.6}
\psframe[linewidth=0.04cm,fillstyle=solid,fillcolor=KernBase](1,3)(11.5,0)
\psframe[linewidth=0.04cm,linestyle=dashed,fillstyle=solid,fillcolor=KernLow](1,1.6)(11.5,0)
\definecolor{UserBase}{rgb}{0.7,0.7,0.7}
\psframe[linewidth=0.0cm,fillstyle=solid,fillcolor=UserBase](1,3)(11.5,7.5)
%----- base kernel modules
\definecolor{KernModule}{rgb}{0,0,0.4}
\definecolor{KernShadow}{rgb}{0,0,0.1}
% memory module
\psframe[linewidth=0.00cm,fillstyle=solid,fillcolor=KernShadow,framearc=.3](1.6,0.4)(3.6,1.4) % shadow
\psframe[linewidth=0.04cm,fillstyle=solid,fillcolor=KernModule,framearc=.3](1.5,0.5)(3.5,1.5)
\rput(2.5,1){\text \color{white}{memory}}
% IRQ module
\psframe[linewidth=0.00cm,fillstyle=solid,fillcolor=KernShadow,framearc=.3](4.1,0.4)(6.1,1.4) % shadow
\psframe[linewidth=0.04cm,fillstyle=solid,fillcolor=KernModule,framearc=.3](4,0.5)(6,1.5)
\rput(5,1){\text \color{white}{interrupt}}
% sched module
\psframe[linewidth=0.00cm,fillstyle=solid,fillcolor=KernShadow,framearc=.3](6.6,0.4)(8.6,1.4) % shadow
\psframe[linewidth=0.04cm,fillstyle=solid,fillcolor=KernModule,framearc=.3](6.5,0.5)(8.5,1.5)
\rput(7.5,1){\text \color{white}{scheduler}}
% CPU module
\psframe[linewidth=0.00cm,fillstyle=solid,fillcolor=KernShadow,framearc=.3](9.1,0.4)(11.1,1.4) % shadow
\psframe[linewidth=0.04cm,fillstyle=solid,fillcolor=KernModule,framearc=.3](9,0.5)(11,1.5)
\rput(10,1){\text \color{white}{processor}}
%----- high kernel modules
\definecolor{KernModule2}{rgb}{0.3,0,0.3}
\definecolor{KernShadow2}{rgb}{0.1,0,0.1}
% driver modules
\psframe[linewidth=0.00cm,fillstyle=solid,fillcolor=KernShadow2,framearc=.3](1.6,1.7)(3.6,2.7) % shadow
\psframe[linewidth=0.04cm,fillstyle=solid,fillcolor=KernModule2,framearc=.3](1.5,1.8)(3.5,2.8)
\rput(2.5,2.3){\text \color{white}{drivers}}
% network module
\psframe[linewidth=0.00cm,fillstyle=solid,fillcolor=KernShadow2,framearc=.3](4.1,1.7)(6.1,2.7) % shadow
\psframe[linewidth=0.04cm,fillstyle=solid,fillcolor=KernModule2,framearc=.3](4,1.8)(6,2.8)
\rput(5,2.3){\text \color{white}{network}}
% fs module
\psframe[linewidth=0.00cm,fillstyle=solid,fillcolor=KernShadow2,framearc=.3](6.6,1.7)(8.6,2.7) % shadow
\psframe[linewidth=0.04cm,fillstyle=solid,fillcolor=KernModule2,framearc=.3](6.5,1.8)(8.5,2.8)
\rput(7.5,2.3){\text \color{white}{filesystem}}
% syscall, IPC module
\psframe[linewidth=0.00cm,fillstyle=solid,fillcolor=KernShadow2,framearc=.3](9.1,1.7)(11.1,2.7) % shadow
\psframe[linewidth=0.04cm,fillstyle=solid,fillcolor=KernModule2,framearc=.3](9,1.8)(11,2.8)
\rput(10,2.3){\text \color{white}{syscalls}}
% US/KS separation line
\psline[linewidth=0.05cm,linestyle=dashed]{-}(0.5,3)(12,3)
\rput{90}(12,1.5){\text kernelspace}
%--- library
\definecolor{Library}{rgb}{0,0.3,0}
\definecolor{LibraryShadow}{rgb}{0,0.1,0}
% lib 1
\psframe[linewidth=0.00cm,fillstyle=solid,fillcolor=LibraryShadow,framearc=.3](1.3,3.1)(4.3,3.9)
\psframe[linewidth=0.04cm,fillstyle=solid,fillcolor=Library,framearc=.3](1.2,3.2)(4.2,4)
\rput(2.7,3.6){\text \color{white}{Library}}
% lib 2
\psframe[linewidth=0.00cm,fillstyle=solid,fillcolor=LibraryShadow,framearc=.3](4.6,3.1)(7.6,3.9)
\psframe[linewidth=0.04cm,fillstyle=solid,fillcolor=Library,framearc=.3](4.5,3.2)(7.5,4)
\rput(6.1,3.6){\text \color{white}{Library}}
% lib 3
\psframe[linewidth=0.00cm,fillstyle=solid,fillcolor=LibraryShadow,framearc=.3](7.9,3.1)(10.9,3.9)
\psframe[linewidth=0.04cm,fillstyle=solid,fillcolor=Library,framearc=.3](7.8,3.2)(10.8,4)
\rput(9.4,3.6){\text \color{white}{Library}}
%--- tasks
\definecolor{Task}{rgb}{0.3,0,0}
\definecolor{TaskShadow}{rgb}{0.1,0,0}
% first task
\psframe[linewidth=0.0cm,fillstyle=solid,fillcolor=TaskShadow,framearc=.3](1.6,4.1)(3.1,7.1)
\psframe[linewidth=0.04cm,fillstyle=solid,fillcolor=Task,framearc=.3](1.5,4.2)(3,7.2)
\rput{60}(2.25,5.7){\text \color{white}{first task}}
% second task
\psframe[linewidth=0.0cm,fillstyle=solid,fillcolor=TaskShadow,framearc=.3](3.6,4.1)(5.1,7.1)
\psframe[linewidth=0.04cm,fillstyle=solid,fillcolor=Task,framearc=.3](3.5,4.2)(5,7.2)
\rput{60}(4.25,5.7){\text \color{white}{second task}}
% etc line
\psline[linewidth=0.04cm,linestyle=dotted](5.4,5.7)(6.6,5.7)
% third task
\psframe[linewidth=0.0cm,fillstyle=solid,fillcolor=TaskShadow,framearc=.3](7.1,4.1)(8.6,7.1)
\psframe[linewidth=0.04cm,fillstyle=solid,fillcolor=Task,framearc=.3](7,4.2)(8.5,7.2)
\rput{60}(7.75,5.7){\text \color{white}{{\it N-1}\textsuperscript{th} task}}
% fourth task
\psframe[linewidth=0.0cm,fillstyle=solid,fillcolor=TaskShadow,framearc=.3](9.1,4.1)(10.6,7.1)
\psframe[linewidth=0.04cm,fillstyle=solid,fillcolor=Task,framearc=.3](9,4.2)(10.5,7.2)
\rput{60}(9.75,5.7){\text \color{white}{{\it N}\textsuperscript{th} task}}
\rput{90}(12,5.25){\text userspace}
\rput{90}(0.6,0.75){\text bases}
\rput{90}(0.6,2.25){\text add-ons}
\end{pspicture}
}
\end{pdfpic}

\label{pic:kern_monolithic}
\caption{R�partition des services dans un noyau monolithique}
\end{figure}

\subsection{Les noyaux hybrides}

\paragraph{}
Exemple de GNU Hurd, noyau hybride �crit en C++, d�portant une partie de ses modules dans un
domaine d'ex�cution dit de {\it service}, hors de l'espace d'adressage du noyau. Un plantage dans
cette zone de service ne provoque pas de plantage du noyau, du fait de cette s�paration.

\begin{figure}[ht]
\input{pictures/kernel_hybrid.tex}
\label{pic:kern_hybrid}
\caption{R�partition des services dans un noyau hybride}
\end{figure}


\subsection{Les micro-noyaux}

\subsection{Les exo-noyaux}

\chapter{Les noyaux UNIX}

\section{Introduction}
Afin de fournir aux applicatifs une interface normalis�e leur permettant de s'abstraire des sp�cificit�s
du mat�riel mais �galement des diff�rents noyaux existant, une norme fut cr��e au d�but des ann�es 70 :
la norme POSIX.\\
Le but est que l'ensemble des noyaux existant soit compatible de cette norme, ceci afin de rendre
les applicatifs portable d'un noyau � un autre. Ce fut le cas de tous les noyaux monolithiques (hors
recherche)... jusqu'� l'arriv�e de MS-DOS).

\paragraph{}
Arbre g�n�alogique de la famille UNIX
\begin{figure}[ht]
%LaTeX with PSTricks extensions
%%Creator: 0.48.0
%%Please note this file requires PSTricks extensions

\begin{pdfpic}
\psset{xunit=.5pt,yunit=.5pt,runit=.5pt}
\begin{pspicture}(1031.98046875,1706.73706055)
{
\newrgbcolor{curcolor}{1 1 1}
\pscustom[linestyle=none,fillstyle=solid,fillcolor=curcolor]
{
\newpath
\moveto(0.31753251,1618.95450592)
\lineto(1031.66299149,1618.95450592)
\lineto(1031.66299149,1574.68345642)
\lineto(0.31753251,1574.68345642)
\closepath
}
}
{
\newrgbcolor{curcolor}{0 0 0}
\pscustom[linewidth=0.63506502,linecolor=curcolor,strokeopacity=0.19607802]
{
\newpath
\moveto(0.31753251,1618.95450592)
\lineto(1031.66299149,1618.95450592)
\lineto(1031.66299149,1574.68345642)
\lineto(0.31753251,1574.68345642)
\closepath
}
}
{
\newrgbcolor{curcolor}{1 1 1}
\pscustom[linestyle=none,fillstyle=solid,fillcolor=curcolor]
{
\newpath
\moveto(0.31753251,1706.41934764)
\lineto(1031.66299149,1706.41934764)
\lineto(1031.66299149,1662.14829814)
\lineto(0.31753251,1662.14829814)
\closepath
}
}
{
\newrgbcolor{curcolor}{0 0 0}
\pscustom[linewidth=0.63506502,linecolor=curcolor,strokeopacity=0.19607802]
{
\newpath
\moveto(0.31753251,1706.41934764)
\lineto(1031.66299149,1706.41934764)
\lineto(1031.66299149,1662.14829814)
\lineto(0.31753251,1662.14829814)
\closepath
}
}
{
\newrgbcolor{curcolor}{1 1 1}
\pscustom[linestyle=none,fillstyle=solid,fillcolor=curcolor]
{
\newpath
\moveto(0.31753251,1531.48981476)
\lineto(1031.66299149,1531.48981476)
\lineto(1031.66299149,1487.21876526)
\lineto(0.31753251,1487.21876526)
\closepath
}
}
{
\newrgbcolor{curcolor}{0 0 0}
\pscustom[linewidth=0.63506502,linecolor=curcolor,strokeopacity=0.19607802]
{
\newpath
\moveto(0.31753251,1531.48981476)
\lineto(1031.66299149,1531.48981476)
\lineto(1031.66299149,1487.21876526)
\lineto(0.31753251,1487.21876526)
\closepath
}
}
{
\newrgbcolor{curcolor}{1 1 1}
\pscustom[linestyle=none,fillstyle=solid,fillcolor=curcolor]
{
\newpath
\moveto(0.31753251,1444.02498627)
\lineto(1031.66299149,1444.02498627)
\lineto(1031.66299149,1399.75393677)
\lineto(0.31753251,1399.75393677)
\closepath
}
}
{
\newrgbcolor{curcolor}{0 0 0}
\pscustom[linewidth=0.63506502,linecolor=curcolor,strokeopacity=0.19607802]
{
\newpath
\moveto(0.31753251,1444.02498627)
\lineto(1031.66299149,1444.02498627)
\lineto(1031.66299149,1399.75393677)
\lineto(0.31753251,1399.75393677)
\closepath
}
}
{
\newrgbcolor{curcolor}{1 1 1}
\pscustom[linestyle=none,fillstyle=solid,fillcolor=curcolor]
{
\newpath
\moveto(0.31753251,1356.56020355)
\lineto(1031.66299149,1356.56020355)
\lineto(1031.66299149,1312.28915405)
\lineto(0.31753251,1312.28915405)
\closepath
}
}
{
\newrgbcolor{curcolor}{0 0 0}
\pscustom[linewidth=0.63506502,linecolor=curcolor,strokeopacity=0.19607802]
{
\newpath
\moveto(0.31753251,1356.56020355)
\lineto(1031.66299149,1356.56020355)
\lineto(1031.66299149,1312.28915405)
\lineto(0.31753251,1312.28915405)
\closepath
}
}
{
\newrgbcolor{curcolor}{1 1 1}
\pscustom[linestyle=none,fillstyle=solid,fillcolor=curcolor]
{
\newpath
\moveto(0.31753251,1269.09545136)
\lineto(1031.66299149,1269.09545136)
\lineto(1031.66299149,1224.82440186)
\lineto(0.31753251,1224.82440186)
\closepath
}
}
{
\newrgbcolor{curcolor}{0 0 0}
\pscustom[linewidth=0.63506502,linecolor=curcolor,strokeopacity=0.19607802]
{
\newpath
\moveto(0.31753251,1269.09545136)
\lineto(1031.66299149,1269.09545136)
\lineto(1031.66299149,1224.82440186)
\lineto(0.31753251,1224.82440186)
\closepath
}
}
{
\newrgbcolor{curcolor}{1 1 1}
\pscustom[linestyle=none,fillstyle=solid,fillcolor=curcolor]
{
\newpath
\moveto(0.31753251,1181.63057709)
\lineto(1031.66299149,1181.63057709)
\lineto(1031.66299149,1137.35952759)
\lineto(0.31753251,1137.35952759)
\closepath
}
}
{
\newrgbcolor{curcolor}{0 0 0}
\pscustom[linewidth=0.63506502,linecolor=curcolor,strokeopacity=0.19607802]
{
\newpath
\moveto(0.31753251,1181.63057709)
\lineto(1031.66299149,1181.63057709)
\lineto(1031.66299149,1137.35952759)
\lineto(0.31753251,1137.35952759)
\closepath
}
}
{
\newrgbcolor{curcolor}{1 1 1}
\pscustom[linestyle=none,fillstyle=solid,fillcolor=curcolor]
{
\newpath
\moveto(0.31753251,1094.16585541)
\lineto(1031.66299149,1094.16585541)
\lineto(1031.66299149,1049.89480591)
\lineto(0.31753251,1049.89480591)
\closepath
}
}
{
\newrgbcolor{curcolor}{0 0 0}
\pscustom[linewidth=0.63506502,linecolor=curcolor,strokeopacity=0.19607802]
{
\newpath
\moveto(0.31753251,1094.16585541)
\lineto(1031.66299149,1094.16585541)
\lineto(1031.66299149,1049.89480591)
\lineto(0.31753251,1049.89480591)
\closepath
}
}
{
\newrgbcolor{curcolor}{1 1 1}
\pscustom[linestyle=none,fillstyle=solid,fillcolor=curcolor]
{
\newpath
\moveto(0.31753251,1006.70113373)
\lineto(1031.66299149,1006.70113373)
\lineto(1031.66299149,962.43008423)
\lineto(0.31753251,962.43008423)
\closepath
}
}
{
\newrgbcolor{curcolor}{0 0 0}
\pscustom[linewidth=0.63506502,linecolor=curcolor,strokeopacity=0.19607802]
{
\newpath
\moveto(0.31753251,1006.70113373)
\lineto(1031.66299149,1006.70113373)
\lineto(1031.66299149,962.43008423)
\lineto(0.31753251,962.43008423)
\closepath
}
}
{
\newrgbcolor{curcolor}{1 1 1}
\pscustom[linestyle=none,fillstyle=solid,fillcolor=curcolor]
{
\newpath
\moveto(0.31753251,919.23635101)
\lineto(1031.66299149,919.23635101)
\lineto(1031.66299149,874.96530151)
\lineto(0.31753251,874.96530151)
\closepath
}
}
{
\newrgbcolor{curcolor}{0 0 0}
\pscustom[linewidth=0.63506502,linecolor=curcolor,strokeopacity=0.19607802]
{
\newpath
\moveto(0.31753251,919.23635101)
\lineto(1031.66299149,919.23635101)
\lineto(1031.66299149,874.96530151)
\lineto(0.31753251,874.96530151)
\closepath
}
}
{
\newrgbcolor{curcolor}{1 1 1}
\pscustom[linestyle=none,fillstyle=solid,fillcolor=curcolor]
{
\newpath
\moveto(0.31753251,831.77162933)
\lineto(1031.66299149,831.77162933)
\lineto(1031.66299149,787.50057983)
\lineto(0.31753251,787.50057983)
\closepath
}
}
{
\newrgbcolor{curcolor}{0 0 0}
\pscustom[linewidth=0.63506502,linecolor=curcolor,strokeopacity=0.19607802]
{
\newpath
\moveto(0.31753251,831.77162933)
\lineto(1031.66299149,831.77162933)
\lineto(1031.66299149,787.50057983)
\lineto(0.31753251,787.50057983)
\closepath
}
}
{
\newrgbcolor{curcolor}{1 1 1}
\pscustom[linestyle=none,fillstyle=solid,fillcolor=curcolor]
{
\newpath
\moveto(0.31753251,744.30684662)
\lineto(1031.66299149,744.30684662)
\lineto(1031.66299149,700.03579712)
\lineto(0.31753251,700.03579712)
\closepath
}
}
{
\newrgbcolor{curcolor}{0 0 0}
\pscustom[linewidth=0.63506502,linecolor=curcolor,strokeopacity=0.19607802]
{
\newpath
\moveto(0.31753251,744.30684662)
\lineto(1031.66299149,744.30684662)
\lineto(1031.66299149,700.03579712)
\lineto(0.31753251,700.03579712)
\closepath
}
}
{
\newrgbcolor{curcolor}{1 1 1}
\pscustom[linestyle=none,fillstyle=solid,fillcolor=curcolor]
{
\newpath
\moveto(0.31753251,656.84212494)
\lineto(1031.66299149,656.84212494)
\lineto(1031.66299149,612.57107544)
\lineto(0.31753251,612.57107544)
\closepath
}
}
{
\newrgbcolor{curcolor}{0 0 0}
\pscustom[linewidth=0.63506502,linecolor=curcolor,strokeopacity=0.19607802]
{
\newpath
\moveto(0.31753251,656.84212494)
\lineto(1031.66299149,656.84212494)
\lineto(1031.66299149,612.57107544)
\lineto(0.31753251,612.57107544)
\closepath
}
}
{
\newrgbcolor{curcolor}{1 1 1}
\pscustom[linestyle=none,fillstyle=solid,fillcolor=curcolor]
{
\newpath
\moveto(0.31753251,569.37728119)
\lineto(1031.66299149,569.37728119)
\lineto(1031.66299149,525.10623169)
\lineto(0.31753251,525.10623169)
\closepath
}
}
{
\newrgbcolor{curcolor}{0 0 0}
\pscustom[linewidth=0.63506502,linecolor=curcolor,strokeopacity=0.19607802]
{
\newpath
\moveto(0.31753251,569.37728119)
\lineto(1031.66299149,569.37728119)
\lineto(1031.66299149,525.10623169)
\lineto(0.31753251,525.10623169)
\closepath
}
}
{
\newrgbcolor{curcolor}{1 1 1}
\pscustom[linestyle=none,fillstyle=solid,fillcolor=curcolor]
{
\newpath
\moveto(0.31753251,481.91243744)
\lineto(1031.66299149,481.91243744)
\lineto(1031.66299149,437.64138794)
\lineto(0.31753251,437.64138794)
\closepath
}
}
{
\newrgbcolor{curcolor}{0 0 0}
\pscustom[linewidth=0.63506502,linecolor=curcolor,strokeopacity=0.19607802]
{
\newpath
\moveto(0.31753251,481.91243744)
\lineto(1031.66299149,481.91243744)
\lineto(1031.66299149,437.64138794)
\lineto(0.31753251,437.64138794)
\closepath
}
}
{
\newrgbcolor{curcolor}{1 1 1}
\pscustom[linestyle=none,fillstyle=solid,fillcolor=curcolor]
{
\newpath
\moveto(0.31753251,394.44759369)
\lineto(1031.66299149,394.44759369)
\lineto(1031.66299149,350.17654419)
\lineto(0.31753251,350.17654419)
\closepath
}
}
{
\newrgbcolor{curcolor}{0 0 0}
\pscustom[linewidth=0.63506502,linecolor=curcolor,strokeopacity=0.19607802]
{
\newpath
\moveto(0.31753251,394.44759369)
\lineto(1031.66299149,394.44759369)
\lineto(1031.66299149,350.17654419)
\lineto(0.31753251,350.17654419)
\closepath
}
}
{
\newrgbcolor{curcolor}{1 1 1}
\pscustom[linestyle=none,fillstyle=solid,fillcolor=curcolor]
{
\newpath
\moveto(0.31753251,132.05342865)
\lineto(1031.66299149,132.05342865)
\lineto(1031.66299149,87.78237915)
\lineto(0.31753251,87.78237915)
\closepath
}
}
{
\newrgbcolor{curcolor}{0 0 0}
\pscustom[linewidth=0.63506502,linecolor=curcolor,strokeopacity=0.19607802]
{
\newpath
\moveto(0.31753251,132.05342865)
\lineto(1031.66299149,132.05342865)
\lineto(1031.66299149,87.78237915)
\lineto(0.31753251,87.78237915)
\closepath
}
}
{
\newrgbcolor{curcolor}{1 1 1}
\pscustom[linestyle=none,fillstyle=solid,fillcolor=curcolor]
{
\newpath
\moveto(0.31753251,219.51790619)
\lineto(1031.66299149,219.51790619)
\lineto(1031.66299149,175.24685669)
\lineto(0.31753251,175.24685669)
\closepath
}
}
{
\newrgbcolor{curcolor}{0 0 0}
\pscustom[linewidth=0.63506502,linecolor=curcolor,strokeopacity=0.19607802]
{
\newpath
\moveto(0.31753251,219.51790619)
\lineto(1031.66299149,219.51790619)
\lineto(1031.66299149,175.24685669)
\lineto(0.31753251,175.24685669)
\closepath
}
}
{
\newrgbcolor{curcolor}{1 1 1}
\pscustom[linestyle=none,fillstyle=solid,fillcolor=curcolor]
{
\newpath
\moveto(0.31753251,306.98274994)
\lineto(1031.66299149,306.98274994)
\lineto(1031.66299149,262.71170044)
\lineto(0.31753251,262.71170044)
\closepath
}
}
{
\newrgbcolor{curcolor}{0 0 0}
\pscustom[linewidth=0.63506502,linecolor=curcolor,strokeopacity=0.19607802]
{
\newpath
\moveto(0.31753251,306.98274994)
\lineto(1031.66299149,306.98274994)
\lineto(1031.66299149,262.71170044)
\lineto(0.31753251,262.71170044)
\closepath
}
}
{
\newrgbcolor{curcolor}{1 1 1}
\pscustom[linestyle=none,fillstyle=solid,fillcolor=curcolor]
{
\newpath
\moveto(0.31753251,44.5885849)
\lineto(1031.66299149,44.5885849)
\lineto(1031.66299149,0.3175354)
\lineto(0.31753251,0.3175354)
\closepath
}
}
{
\newrgbcolor{curcolor}{0 0 0}
\pscustom[linewidth=0.63506502,linecolor=curcolor,strokeopacity=0.19607802]
{
\newpath
\moveto(0.31753251,44.5885849)
\lineto(1031.66299149,44.5885849)
\lineto(1031.66299149,0.3175354)
\lineto(0.31753251,0.3175354)
\closepath
}
}
{
\newrgbcolor{curcolor}{0.45882353 0.45882353 0.45882353}
\pscustom[linestyle=none,fillstyle=solid,fillcolor=curcolor]
{
\newpath
\moveto(108.60991955,1696.6648798)
\lineto(170.40585041,1696.6648798)
\curveto(173.81540009,1696.6648798)(176.56027222,1694.16571183)(176.56027222,1691.06136417)
\lineto(176.56027222,1680.78874302)
\curveto(176.56027222,1677.68439536)(173.81540009,1675.18522739)(170.40585041,1675.18522739)
\lineto(108.60991955,1675.18522739)
\curveto(105.20036987,1675.18522739)(102.45549774,1677.68439536)(102.45549774,1680.78874302)
\lineto(102.45549774,1691.06136417)
\curveto(102.45549774,1694.16571183)(105.20036987,1696.6648798)(108.60991955,1696.6648798)
\closepath
}
}
{
\newrgbcolor{curcolor}{0.45882353 0.45882353 0.45882353}
\pscustom[linestyle=none,fillstyle=solid,fillcolor=curcolor]
{
\newpath
\moveto(602.50122356,1303.33061981)
\lineto(638.47687244,1303.33061981)
\curveto(641.88642212,1303.33061981)(644.63129425,1300.83145184)(644.63129425,1297.72710419)
\lineto(644.63129425,1287.45449066)
\curveto(644.63129425,1284.35014301)(641.88642212,1281.85097504)(638.47687244,1281.85097504)
\lineto(602.50122356,1281.85097504)
\curveto(599.09167388,1281.85097504)(596.34680176,1284.35014301)(596.34680176,1287.45449066)
\lineto(596.34680176,1297.72710419)
\curveto(596.34680176,1300.83145184)(599.09167388,1303.33061981)(602.50122356,1303.33061981)
\closepath
}
}
{
\newrgbcolor{curcolor}{0.45882353 0.45882353 0.45882353}
\pscustom[linestyle=none,fillstyle=solid,fillcolor=curcolor]
{
\newpath
\moveto(456.5861845,1259.54006195)
\lineto(510.02851582,1259.54006195)
\curveto(513.4380655,1259.54006195)(516.18293762,1257.04089398)(516.18293762,1253.93654633)
\lineto(516.18293762,1243.6639328)
\curveto(516.18293762,1240.55958514)(513.4380655,1238.06041718)(510.02851582,1238.06041718)
\lineto(456.5861845,1238.06041718)
\curveto(453.17663482,1238.06041718)(450.4317627,1240.55958514)(450.4317627,1243.6639328)
\lineto(450.4317627,1253.93654633)
\curveto(450.4317627,1257.04089398)(453.17663482,1259.54006195)(456.5861845,1259.54006195)
\closepath
}
}
{
\newrgbcolor{curcolor}{0.45882353 0.45882353 0.45882353}
\pscustom[linestyle=none,fillstyle=solid,fillcolor=curcolor]
{
\newpath
\moveto(748.94238567,1215.02120209)
\lineto(802.38470173,1215.02120209)
\curveto(805.79425141,1215.02120209)(808.53912354,1212.52203412)(808.53912354,1209.41768646)
\lineto(808.53912354,1199.14507294)
\curveto(808.53912354,1196.04072528)(805.79425141,1193.54155731)(802.38470173,1193.54155731)
\lineto(748.94238567,1193.54155731)
\curveto(745.53283599,1193.54155731)(742.78796387,1196.04072528)(742.78796387,1199.14507294)
\lineto(742.78796387,1209.41768646)
\curveto(742.78796387,1212.52203412)(745.53283599,1215.02120209)(748.94238567,1215.02120209)
\closepath
}
}
{
\newrgbcolor{curcolor}{0.45882353 0.45882353 0.45882353}
\pscustom[linestyle=none,fillstyle=solid,fillcolor=curcolor]
{
\newpath
\moveto(597.94470501,1171.67928314)
\lineto(643.03339863,1171.67928314)
\curveto(646.44294831,1171.67928314)(649.18782043,1169.18011517)(649.18782043,1166.07576752)
\lineto(649.18782043,1155.80315399)
\curveto(649.18782043,1152.69880634)(646.44294831,1150.19963837)(643.03339863,1150.19963837)
\lineto(597.94470501,1150.19963837)
\curveto(594.53515533,1150.19963837)(591.7902832,1152.69880634)(591.7902832,1155.80315399)
\lineto(591.7902832,1166.07576752)
\curveto(591.7902832,1169.18011517)(594.53515533,1171.67928314)(597.94470501,1171.67928314)
\closepath
}
}
{
\newrgbcolor{curcolor}{0.45882353 0.45882353 0.45882353}
\pscustom[linestyle=none,fillstyle=solid,fillcolor=curcolor]
{
\newpath
\moveto(91.08538342,1127.96425629)
\lineto(179.46101856,1127.96425629)
\curveto(182.87056824,1127.96425629)(185.61544037,1125.46508832)(185.61544037,1122.36074066)
\lineto(185.61544037,1112.08812714)
\curveto(185.61544037,1108.98377948)(182.87056824,1106.48461151)(179.46101856,1106.48461151)
\lineto(91.08538342,1106.48461151)
\curveto(87.67583373,1106.48461151)(84.93096161,1108.98377948)(84.93096161,1112.08812714)
\lineto(84.93096161,1122.36074066)
\curveto(84.93096161,1125.46508832)(87.67583373,1127.96425629)(91.08538342,1127.96425629)
\closepath
}
}
{
\newrgbcolor{curcolor}{0.45882353 0.45882353 0.45882353}
\pscustom[linestyle=none,fillstyle=solid,fillcolor=curcolor]
{
\newpath
\moveto(596.42584515,1127.96425629)
\lineto(644.55222034,1127.96425629)
\curveto(647.96177003,1127.96425629)(650.70664215,1125.46508832)(650.70664215,1122.36074066)
\lineto(650.70664215,1112.08812714)
\curveto(650.70664215,1108.98377948)(647.96177003,1106.48461151)(644.55222034,1106.48461151)
\lineto(596.42584515,1106.48461151)
\curveto(593.01629547,1106.48461151)(590.27142334,1108.98377948)(590.27142334,1112.08812714)
\lineto(590.27142334,1122.36074066)
\curveto(590.27142334,1125.46508832)(593.01629547,1127.96425629)(596.42584515,1127.96425629)
\closepath
}
}
{
\newrgbcolor{curcolor}{0.45882353 0.45882353 0.45882353}
\pscustom[linestyle=none,fillstyle=solid,fillcolor=curcolor]
{
\newpath
\moveto(91.84479046,1083.03078461)
\lineto(178.701581,1083.03078461)
\curveto(182.11113068,1083.03078461)(184.85600281,1080.53161664)(184.85600281,1077.42726898)
\lineto(184.85600281,1067.15465546)
\curveto(184.85600281,1064.0503078)(182.11113068,1061.55113983)(178.701581,1061.55113983)
\lineto(91.84479046,1061.55113983)
\curveto(88.43524078,1061.55113983)(85.69036865,1064.0503078)(85.69036865,1067.15465546)
\lineto(85.69036865,1077.42726898)
\curveto(85.69036865,1080.53161664)(88.43524078,1083.03078461)(91.84479046,1083.03078461)
\closepath
}
}
{
\newrgbcolor{curcolor}{0.45882353 0.45882353 0.45882353}
\pscustom[linestyle=none,fillstyle=solid,fillcolor=curcolor]
{
\newpath
\moveto(942.26001263,1083.03078461)
\lineto(975.95740604,1083.03078461)
\curveto(979.36695572,1083.03078461)(982.11182785,1080.53161664)(982.11182785,1077.42726898)
\lineto(982.11182785,1067.15465546)
\curveto(982.11182785,1064.0503078)(979.36695572,1061.55113983)(975.95740604,1061.55113983)
\lineto(942.26001263,1061.55113983)
\curveto(938.85046295,1061.55113983)(936.10559082,1064.0503078)(936.10559082,1067.15465546)
\lineto(936.10559082,1077.42726898)
\curveto(936.10559082,1080.53161664)(938.85046295,1083.03078461)(942.26001263,1083.03078461)
\closepath
}
}
{
\newrgbcolor{curcolor}{0.45882353 0.45882353 0.45882353}
\pscustom[linestyle=none,fillstyle=solid,fillcolor=curcolor]
{
\newpath
\moveto(104.37523174,1477.25043488)
\lineto(166.17116261,1477.25043488)
\curveto(169.58071229,1477.25043488)(172.32558441,1474.75126691)(172.32558441,1471.64691925)
\lineto(172.32558441,1461.3742981)
\curveto(172.32558441,1458.26995044)(169.58071229,1455.77078247)(166.17116261,1455.77078247)
\lineto(104.37523174,1455.77078247)
\curveto(100.96568206,1455.77078247)(98.22080994,1458.26995044)(98.22080994,1461.3742981)
\lineto(98.22080994,1471.64691925)
\curveto(98.22080994,1474.75126691)(100.96568206,1477.25043488)(104.37523174,1477.25043488)
\closepath
}
}
{
\newrgbcolor{curcolor}{0.45882353 0.45882353 0.45882353}
\pscustom[linestyle=none,fillstyle=solid,fillcolor=curcolor]
{
\newpath
\moveto(59.1283083,1607.70168304)
\lineto(219.88746166,1607.70168304)
\curveto(223.41233115,1607.70168304)(226.25004196,1605.20251508)(226.25004196,1602.09816742)
\lineto(226.25004196,1591.82554626)
\curveto(226.25004196,1588.72119861)(223.41233115,1586.22203064)(219.88746166,1586.22203064)
\lineto(59.1283083,1586.22203064)
\curveto(55.60343881,1586.22203064)(52.765728,1588.72119861)(52.765728,1591.82554626)
\lineto(52.765728,1602.09816742)
\curveto(52.765728,1605.20251508)(55.60343881,1607.70168304)(59.1283083,1607.70168304)
\closepath
}
}
{
\newrgbcolor{curcolor}{0.45882353 0.45882353 0.45882353}
\pscustom[linestyle=none,fillstyle=solid,fillcolor=curcolor]
{
\newpath
\moveto(59.1283083,1562.6554184)
\lineto(219.88746166,1562.6554184)
\curveto(223.41233115,1562.6554184)(226.25004196,1560.15625043)(226.25004196,1557.05190277)
\lineto(226.25004196,1546.77928162)
\curveto(226.25004196,1543.67493396)(223.41233115,1541.17576599)(219.88746166,1541.17576599)
\lineto(59.1283083,1541.17576599)
\curveto(55.60343881,1541.17576599)(52.765728,1543.67493396)(52.765728,1546.77928162)
\lineto(52.765728,1557.05190277)
\curveto(52.765728,1560.15625043)(55.60343881,1562.6554184)(59.1283083,1562.6554184)
\closepath
}
}
{
\newrgbcolor{curcolor}{0.45882353 0.45882353 0.45882353}
\pscustom[linestyle=none,fillstyle=solid,fillcolor=curcolor]
{
\newpath
\moveto(59.1283083,1519.96704865)
\lineto(219.88746166,1519.96704865)
\curveto(223.41233115,1519.96704865)(226.25004196,1517.46788068)(226.25004196,1514.36353302)
\lineto(226.25004196,1504.09091187)
\curveto(226.25004196,1500.98656421)(223.41233115,1498.48739624)(219.88746166,1498.48739624)
\lineto(59.1283083,1498.48739624)
\curveto(55.60343881,1498.48739624)(52.765728,1500.98656421)(52.765728,1504.09091187)
\lineto(52.765728,1514.36353302)
\curveto(52.765728,1517.46788068)(55.60343881,1519.96704865)(59.1283083,1519.96704865)
\closepath
}
}
{
\newrgbcolor{curcolor}{0.45882353 0.45882353 0.45882353}
\pscustom[linestyle=none,fillstyle=solid,fillcolor=curcolor]
{
\newpath
\moveto(251.3253336,1519.96704865)
\lineto(412.08448696,1519.96704865)
\curveto(415.60935645,1519.96704865)(418.44706726,1517.46788068)(418.44706726,1514.36353302)
\lineto(418.44706726,1504.09091187)
\curveto(418.44706726,1500.98656421)(415.60935645,1498.48739624)(412.08448696,1498.48739624)
\lineto(251.3253336,1498.48739624)
\curveto(247.80046411,1498.48739624)(244.9627533,1500.98656421)(244.9627533,1504.09091187)
\lineto(244.9627533,1514.36353302)
\curveto(244.9627533,1517.46788068)(247.80046411,1519.96704865)(251.3253336,1519.96704865)
\closepath
}
}
{
\newrgbcolor{curcolor}{0.45882353 0.45882353 0.45882353}
\pscustom[linestyle=none,fillstyle=solid,fillcolor=curcolor]
{
\newpath
\moveto(251.3253336,1477.25043488)
\lineto(412.08448696,1477.25043488)
\curveto(415.60935645,1477.25043488)(418.44706726,1474.75126691)(418.44706726,1471.64691925)
\lineto(418.44706726,1461.3742981)
\curveto(418.44706726,1458.26995044)(415.60935645,1455.77078247)(412.08448696,1455.77078247)
\lineto(251.3253336,1455.77078247)
\curveto(247.80046411,1455.77078247)(244.9627533,1458.26995044)(244.9627533,1461.3742981)
\lineto(244.9627533,1471.64691925)
\curveto(244.9627533,1474.75126691)(247.80046411,1477.25043488)(251.3253336,1477.25043488)
\closepath
}
}
{
\newrgbcolor{curcolor}{0.45882353 0.45882353 0.45882353}
\pscustom[linestyle=none,fillstyle=solid,fillcolor=curcolor]
{
\newpath
\moveto(251.3253336,1431.90050507)
\lineto(412.08448696,1431.90050507)
\curveto(415.60935645,1431.90050507)(418.44706726,1429.4013371)(418.44706726,1426.29698944)
\lineto(418.44706726,1416.02436829)
\curveto(418.44706726,1412.92002063)(415.60935645,1410.42085266)(412.08448696,1410.42085266)
\lineto(251.3253336,1410.42085266)
\curveto(247.80046411,1410.42085266)(244.9627533,1412.92002063)(244.9627533,1416.02436829)
\lineto(244.9627533,1426.29698944)
\curveto(244.9627533,1429.4013371)(247.80046411,1431.90050507)(251.3253336,1431.90050507)
\closepath
}
}
{
\newrgbcolor{curcolor}{0.45882353 0.45882353 0.45882353}
\pscustom[linestyle=none,fillstyle=solid,fillcolor=curcolor]
{
\newpath
\moveto(251.3253336,1259.54006195)
\lineto(412.08448696,1259.54006195)
\curveto(415.60935645,1259.54006195)(418.44706726,1257.04089398)(418.44706726,1253.93654633)
\lineto(418.44706726,1243.66392517)
\curveto(418.44706726,1240.55957751)(415.60935645,1238.06040955)(412.08448696,1238.06040955)
\lineto(251.3253336,1238.06040955)
\curveto(247.80046411,1238.06040955)(244.9627533,1240.55957751)(244.9627533,1243.66392517)
\lineto(244.9627533,1253.93654633)
\curveto(244.9627533,1257.04089398)(247.80046411,1259.54006195)(251.3253336,1259.54006195)
\closepath
}
}
{
\newrgbcolor{curcolor}{0.45882353 0.45882353 0.45882353}
\pscustom[linestyle=none,fillstyle=solid,fillcolor=curcolor]
{
\newpath
\moveto(404.93286228,994.70528412)
\lineto(565.69201565,994.70528412)
\curveto(569.21688513,994.70528412)(572.05459595,992.20611615)(572.05459595,989.10176849)
\lineto(572.05459595,978.82914734)
\curveto(572.05459595,975.72479968)(569.21688513,973.22563171)(565.69201565,973.22563171)
\lineto(404.93286228,973.22563171)
\curveto(401.4079928,973.22563171)(398.57028198,975.72479968)(398.57028198,978.82914734)
\lineto(398.57028198,989.10176849)
\curveto(398.57028198,992.20611615)(401.4079928,994.70528412)(404.93286228,994.70528412)
\closepath
}
}
{
\newrgbcolor{curcolor}{0.45882353 0.45882353 0.45882353}
\pscustom[linestyle=none,fillstyle=solid,fillcolor=curcolor]
{
\newpath
\moveto(404.93286228,949.63832855)
\lineto(565.69201565,949.63832855)
\curveto(569.21688513,949.63832855)(572.05459595,947.13916058)(572.05459595,944.03481293)
\lineto(572.05459595,933.76219177)
\curveto(572.05459595,930.65784412)(569.21688513,928.15867615)(565.69201565,928.15867615)
\lineto(404.93286228,928.15867615)
\curveto(401.4079928,928.15867615)(398.57028198,930.65784412)(398.57028198,933.76219177)
\lineto(398.57028198,944.03481293)
\curveto(398.57028198,947.13916058)(401.4079928,949.63832855)(404.93286228,949.63832855)
\closepath
}
}
{
\newrgbcolor{curcolor}{0.45882353 0.45882353 0.45882353}
\pscustom[linestyle=none,fillstyle=solid,fillcolor=curcolor]
{
\newpath
\moveto(404.297822,819.3286972)
\lineto(565.05697536,819.3286972)
\curveto(568.58184485,819.3286972)(571.41955566,816.82952924)(571.41955566,813.72518158)
\lineto(571.41955566,803.45256043)
\curveto(571.41955566,800.34821277)(568.58184485,797.8490448)(565.05697536,797.8490448)
\lineto(404.297822,797.8490448)
\curveto(400.77295251,797.8490448)(397.9352417,800.34821277)(397.9352417,803.45256043)
\lineto(397.9352417,813.72518158)
\curveto(397.9352417,816.82952924)(400.77295251,819.3286972)(404.297822,819.3286972)
\closepath
}
}
{
\newrgbcolor{curcolor}{0.45882353 0.45882353 0.45882353}
\pscustom[linestyle=none,fillstyle=solid,fillcolor=curcolor]
{
\newpath
\moveto(684.96338177,1127.96425629)
\lineto(746.75931263,1127.96425629)
\curveto(750.16886231,1127.96425629)(752.91373444,1125.46508832)(752.91373444,1122.36074066)
\lineto(752.91373444,1112.08811951)
\curveto(752.91373444,1108.98377185)(750.16886231,1106.48460388)(746.75931263,1106.48460388)
\lineto(684.96338177,1106.48460388)
\curveto(681.55383209,1106.48460388)(678.80895996,1108.98377185)(678.80895996,1112.08811951)
\lineto(678.80895996,1122.36074066)
\curveto(678.80895996,1125.46508832)(681.55383209,1127.96425629)(684.96338177,1127.96425629)
\closepath
}
}
{
\newrgbcolor{curcolor}{0.45882353 0.45882353 0.45882353}
\pscustom[linestyle=none,fillstyle=solid,fillcolor=curcolor]
{
\newpath
\moveto(598.32440472,949.63832855)
\lineto(642.65367985,949.63832855)
\curveto(646.06322953,949.63832855)(648.80810165,947.13916058)(648.80810165,944.03481293)
\lineto(648.80810165,933.7621994)
\curveto(648.80810165,930.65785175)(646.06322953,928.15868378)(642.65367985,928.15868378)
\lineto(598.32440472,928.15868378)
\curveto(594.91485504,928.15868378)(592.16998291,930.65785175)(592.16998291,933.7621994)
\lineto(592.16998291,944.03481293)
\curveto(592.16998291,947.13916058)(594.91485504,949.63832855)(598.32440472,949.63832855)
\closepath
}
}
{
\newrgbcolor{curcolor}{0.45882353 0.45882353 0.45882353}
\pscustom[linestyle=none,fillstyle=solid,fillcolor=curcolor]
{
\newpath
\moveto(684.96338177,908.62740326)
\lineto(746.75931263,908.62740326)
\curveto(750.16886231,908.62740326)(752.91373444,906.12823529)(752.91373444,903.02388763)
\lineto(752.91373444,892.75126648)
\curveto(752.91373444,889.64691882)(750.16886231,887.14775085)(746.75931263,887.14775085)
\lineto(684.96338177,887.14775085)
\curveto(681.55383209,887.14775085)(678.80895996,889.64691882)(678.80895996,892.75126648)
\lineto(678.80895996,903.02388763)
\curveto(678.80895996,906.12823529)(681.55383209,908.62740326)(684.96338177,908.62740326)
\closepath
}
}
{
\newrgbcolor{curcolor}{0.45882353 0.45882353 0.45882353}
\pscustom[linestyle=none,fillstyle=solid,fillcolor=curcolor]
{
\newpath
\moveto(818.87829876,908.62740326)
\lineto(880.67422962,908.62740326)
\curveto(884.0837793,908.62740326)(886.82865143,906.12823529)(886.82865143,903.02388763)
\lineto(886.82865143,892.75126648)
\curveto(886.82865143,889.64691882)(884.0837793,887.14775085)(880.67422962,887.14775085)
\lineto(818.87829876,887.14775085)
\curveto(815.46874908,887.14775085)(812.72387695,889.64691882)(812.72387695,892.75126648)
\lineto(812.72387695,903.02388763)
\curveto(812.72387695,906.12823529)(815.46874908,908.62740326)(818.87829876,908.62740326)
\closepath
}
}
{
\newrgbcolor{curcolor}{0.45882353 0.45882353 0.45882353}
\pscustom[linestyle=none,fillstyle=solid,fillcolor=curcolor]
{
\newpath
\moveto(67.5433569,866.38332367)
\lineto(203.00303745,866.38332367)
\curveto(206.41258713,866.38332367)(209.15745926,863.8841557)(209.15745926,860.77980804)
\lineto(209.15745926,850.50719452)
\curveto(209.15745926,847.40284686)(206.41258713,844.90367889)(203.00303745,844.90367889)
\lineto(67.5433569,844.90367889)
\curveto(64.13380721,844.90367889)(61.38893509,847.40284686)(61.38893509,850.50719452)
\lineto(61.38893509,860.77980804)
\curveto(61.38893509,863.8841557)(64.13380721,866.38332367)(67.5433569,866.38332367)
\closepath
}
}
{
\newrgbcolor{curcolor}{0.45882353 0.45882353 0.45882353}
\pscustom[linestyle=none,fillstyle=solid,fillcolor=curcolor]
{
\newpath
\moveto(582.88067913,863.52553558)
\lineto(656.82731342,863.52553558)
\curveto(660.2368631,863.52553558)(662.98173523,861.02636762)(662.98173523,857.92201996)
\lineto(662.98173523,847.64940643)
\curveto(662.98173523,844.54505878)(660.2368631,842.04589081)(656.82731342,842.04589081)
\lineto(582.88067913,842.04589081)
\curveto(579.47112945,842.04589081)(576.72625732,844.54505878)(576.72625732,847.64940643)
\lineto(576.72625732,857.92201996)
\curveto(576.72625732,861.02636762)(579.47112945,863.52553558)(582.88067913,863.52553558)
\closepath
}
}
{
\newrgbcolor{curcolor}{0.45882353 0.45882353 0.45882353}
\pscustom[linestyle=none,fillstyle=solid,fillcolor=curcolor]
{
\newpath
\moveto(588.95599651,775.65938568)
\lineto(650.75192738,775.65938568)
\curveto(654.16147706,775.65938568)(656.90634918,773.16021771)(656.90634918,770.05587006)
\lineto(656.90634918,759.7832489)
\curveto(656.90634918,756.67890125)(654.16147706,754.17973328)(650.75192738,754.17973328)
\lineto(588.95599651,754.17973328)
\curveto(585.54644683,754.17973328)(582.80157471,756.67890125)(582.80157471,759.7832489)
\lineto(582.80157471,770.05587006)
\curveto(582.80157471,773.16021771)(585.54644683,775.65938568)(588.95599651,775.65938568)
\closepath
}
}
{
\newrgbcolor{curcolor}{0.45882353 0.45882353 0.45882353}
\pscustom[linestyle=none,fillstyle=solid,fillcolor=curcolor]
{
\newpath
\moveto(139.43901348,733.04085541)
\lineto(201.23494434,733.04085541)
\curveto(204.64449402,733.04085541)(207.38936615,730.54168744)(207.38936615,727.43733978)
\lineto(207.38936615,717.16471863)
\curveto(207.38936615,714.06037097)(204.64449402,711.561203)(201.23494434,711.561203)
\lineto(139.43901348,711.561203)
\curveto(136.0294638,711.561203)(133.28459167,714.06037097)(133.28459167,717.16471863)
\lineto(133.28459167,727.43733978)
\curveto(133.28459167,730.54168744)(136.0294638,733.04085541)(139.43901348,733.04085541)
\closepath
}
}
{
\newrgbcolor{curcolor}{0.45882353 0.45882353 0.45882353}
\pscustom[linestyle=none,fillstyle=solid,fillcolor=curcolor]
{
\newpath
\moveto(270.90985394,733.04085541)
\lineto(332.7057848,733.04085541)
\curveto(336.11533448,733.04085541)(338.8602066,730.54168744)(338.8602066,727.43733978)
\lineto(338.8602066,717.16471863)
\curveto(338.8602066,714.06037097)(336.11533448,711.561203)(332.7057848,711.561203)
\lineto(270.90985394,711.561203)
\curveto(267.50030425,711.561203)(264.75543213,714.06037097)(264.75543213,717.16471863)
\lineto(264.75543213,727.43733978)
\curveto(264.75543213,730.54168744)(267.50030425,733.04085541)(270.90985394,733.04085541)
\closepath
}
}
{
\newrgbcolor{curcolor}{0.45882353 0.45882353 0.45882353}
\pscustom[linestyle=none,fillstyle=solid,fillcolor=curcolor]
{
\newpath
\moveto(870.49615765,733.67592621)
\lineto(932.29208851,733.67592621)
\curveto(935.70163819,733.67592621)(938.44651031,731.17675824)(938.44651031,728.07241058)
\lineto(938.44651031,717.79978943)
\curveto(938.44651031,714.69544177)(935.70163819,712.1962738)(932.29208851,712.1962738)
\lineto(870.49615765,712.1962738)
\curveto(867.08660797,712.1962738)(864.34173584,714.69544177)(864.34173584,717.79978943)
\lineto(864.34173584,728.07241058)
\curveto(864.34173584,731.17675824)(867.08660797,733.67592621)(870.49615765,733.67592621)
\closepath
}
}
{
\newrgbcolor{curcolor}{0.45882353 0.45882353 0.45882353}
\pscustom[linestyle=none,fillstyle=solid,fillcolor=curcolor]
{
\newpath
\moveto(67.13604259,689.91970062)
\lineto(138.8044405,689.91970062)
\curveto(142.21399018,689.91970062)(144.9588623,687.42053265)(144.9588623,684.316185)
\lineto(144.9588623,674.04357147)
\curveto(144.9588623,670.93922382)(142.21399018,668.44005585)(138.8044405,668.44005585)
\lineto(67.13604259,668.44005585)
\curveto(63.72649291,668.44005585)(60.98162079,670.93922382)(60.98162079,674.04357147)
\lineto(60.98162079,684.316185)
\curveto(60.98162079,687.42053265)(63.72649291,689.91970062)(67.13604259,689.91970062)
\closepath
}
}
{
\newrgbcolor{curcolor}{0.45882353 0.45882353 0.45882353}
\pscustom[linestyle=none,fillstyle=solid,fillcolor=curcolor]
{
\newpath
\moveto(212.51471233,689.91970062)
\lineto(288.73962116,689.91970062)
\curveto(292.14917084,689.91970062)(294.89404297,687.42053265)(294.89404297,684.316185)
\lineto(294.89404297,674.04357147)
\curveto(294.89404297,670.93922382)(292.14917084,668.44005585)(288.73962116,668.44005585)
\lineto(212.51471233,668.44005585)
\curveto(209.10516265,668.44005585)(206.36029053,670.93922382)(206.36029053,674.04357147)
\lineto(206.36029053,684.316185)
\curveto(206.36029053,687.42053265)(209.10516265,689.91970062)(212.51471233,689.91970062)
\closepath
}
}
{
\newrgbcolor{curcolor}{0.45882353 0.45882353 0.45882353}
\pscustom[linestyle=none,fillstyle=solid,fillcolor=curcolor]
{
\newpath
\moveto(311.23932171,689.91970062)
\lineto(373.03525257,689.91970062)
\curveto(376.44480225,689.91970062)(379.18967438,687.42053265)(379.18967438,684.316185)
\lineto(379.18967438,674.04356384)
\curveto(379.18967438,670.93921619)(376.44480225,668.44004822)(373.03525257,668.44004822)
\lineto(311.23932171,668.44004822)
\curveto(307.82977203,668.44004822)(305.0848999,670.93921619)(305.0848999,674.04356384)
\lineto(305.0848999,684.316185)
\curveto(305.0848999,687.42053265)(307.82977203,689.91970062)(311.23932171,689.91970062)
\closepath
}
}
{
\newrgbcolor{curcolor}{0.45882353 0.45882353 0.45882353}
\pscustom[linestyle=none,fillstyle=solid,fillcolor=curcolor]
{
\newpath
\moveto(403.12915325,689.91970062)
\lineto(464.92508411,689.91970062)
\curveto(468.33463379,689.91970062)(471.07950592,687.42053265)(471.07950592,684.316185)
\lineto(471.07950592,674.04356384)
\curveto(471.07950592,670.93921619)(468.33463379,668.44004822)(464.92508411,668.44004822)
\lineto(403.12915325,668.44004822)
\curveto(399.71960357,668.44004822)(396.97473145,670.93921619)(396.97473145,674.04356384)
\lineto(396.97473145,684.316185)
\curveto(396.97473145,687.42053265)(399.71960357,689.91970062)(403.12915325,689.91970062)
\closepath
}
}
{
\newrgbcolor{curcolor}{0.45882353 0.45882353 0.45882353}
\pscustom[linestyle=none,fillstyle=solid,fillcolor=curcolor]
{
\newpath
\moveto(928.21075726,685.1568222)
\lineto(990.00668812,685.1568222)
\curveto(993.4162378,685.1568222)(996.16110992,682.65765424)(996.16110992,679.55330658)
\lineto(996.16110992,669.28068543)
\curveto(996.16110992,666.17633777)(993.4162378,663.6771698)(990.00668812,663.6771698)
\lineto(928.21075726,663.6771698)
\curveto(924.80120757,663.6771698)(922.05633545,666.17633777)(922.05633545,669.28068543)
\lineto(922.05633545,679.55330658)
\curveto(922.05633545,682.65765424)(924.80120757,685.1568222)(928.21075726,685.1568222)
\closepath
}
}
{
\newrgbcolor{curcolor}{0.45882353 0.45882353 0.45882353}
\pscustom[linestyle=none,fillstyle=solid,fillcolor=curcolor]
{
\newpath
\moveto(467.40451336,644.97029877)
\lineto(529.20044422,644.97029877)
\curveto(532.6099939,644.97029877)(535.35486603,642.4711308)(535.35486603,639.36678314)
\lineto(535.35486603,629.09416199)
\curveto(535.35486603,625.98981433)(532.6099939,623.49064636)(529.20044422,623.49064636)
\lineto(467.40451336,623.49064636)
\curveto(463.99496368,623.49064636)(461.25009155,625.98981433)(461.25009155,629.09416199)
\lineto(461.25009155,639.36678314)
\curveto(461.25009155,642.4711308)(463.99496368,644.97029877)(467.40451336,644.97029877)
\closepath
}
}
{
\newrgbcolor{curcolor}{0.45882353 0.45882353 0.45882353}
\pscustom[linestyle=none,fillstyle=solid,fillcolor=curcolor]
{
\newpath
\moveto(216.83635235,644.97029877)
\lineto(283.94822407,644.97029877)
\curveto(287.35777375,644.97029877)(290.10264587,642.4711308)(290.10264587,639.36678314)
\lineto(290.10264587,629.09416962)
\curveto(290.10264587,625.98982196)(287.35777375,623.49065399)(283.94822407,623.49065399)
\lineto(216.83635235,623.49065399)
\curveto(213.42680267,623.49065399)(210.68193054,625.98982196)(210.68193054,629.09416962)
\lineto(210.68193054,639.36678314)
\curveto(210.68193054,642.4711308)(213.42680267,644.97029877)(216.83635235,644.97029877)
\closepath
}
}
{
\newrgbcolor{curcolor}{0.45882353 0.45882353 0.45882353}
\pscustom[linestyle=none,fillstyle=solid,fillcolor=curcolor]
{
\newpath
\moveto(467.40451336,601.1733017)
\lineto(529.20044422,601.1733017)
\curveto(532.6099939,601.1733017)(535.35486603,598.67413373)(535.35486603,595.56978607)
\lineto(535.35486603,585.29716492)
\curveto(535.35486603,582.19281726)(532.6099939,579.69364929)(529.20044422,579.69364929)
\lineto(467.40451336,579.69364929)
\curveto(463.99496368,579.69364929)(461.25009155,582.19281726)(461.25009155,585.29716492)
\lineto(461.25009155,595.56978607)
\curveto(461.25009155,598.67413373)(463.99496368,601.1733017)(467.40451336,601.1733017)
\closepath
}
}
{
\newrgbcolor{curcolor}{0.45882353 0.45882353 0.45882353}
\pscustom[linestyle=none,fillstyle=solid,fillcolor=curcolor]
{
\newpath
\moveto(335.49472332,601.1733017)
\lineto(397.29065418,601.1733017)
\curveto(400.70020386,601.1733017)(403.44507599,598.67413373)(403.44507599,595.56978607)
\lineto(403.44507599,585.29716492)
\curveto(403.44507599,582.19281726)(400.70020386,579.69364929)(397.29065418,579.69364929)
\lineto(335.49472332,579.69364929)
\curveto(332.08517364,579.69364929)(329.34030151,582.19281726)(329.34030151,585.29716492)
\lineto(329.34030151,595.56978607)
\curveto(329.34030151,598.67413373)(332.08517364,601.1733017)(335.49472332,601.1733017)
\closepath
}
}
{
\newrgbcolor{curcolor}{0.45882353 0.45882353 0.45882353}
\pscustom[linestyle=none,fillstyle=solid,fillcolor=curcolor]
{
\newpath
\moveto(588.95599651,601.1733017)
\lineto(650.75192738,601.1733017)
\curveto(654.16147706,601.1733017)(656.90634918,598.67413373)(656.90634918,595.56978607)
\lineto(656.90634918,585.29716492)
\curveto(656.90634918,582.19281726)(654.16147706,579.69364929)(650.75192738,579.69364929)
\lineto(588.95599651,579.69364929)
\curveto(585.54644683,579.69364929)(582.80157471,582.19281726)(582.80157471,585.29716492)
\lineto(582.80157471,595.56978607)
\curveto(582.80157471,598.67413373)(585.54644683,601.1733017)(588.95599651,601.1733017)
\closepath
}
}
{
\newrgbcolor{curcolor}{0.45882353 0.45882353 0.45882353}
\pscustom[linestyle=none,fillstyle=solid,fillcolor=curcolor]
{
\newpath
\moveto(679.6474638,601.1733017)
\lineto(752.07526875,601.1733017)
\curveto(755.48481843,601.1733017)(758.22969055,598.67413373)(758.22969055,595.56978607)
\lineto(758.22969055,585.29717255)
\curveto(758.22969055,582.19282489)(755.48481843,579.69365692)(752.07526875,579.69365692)
\lineto(679.6474638,579.69365692)
\curveto(676.23791412,579.69365692)(673.49304199,582.19282489)(673.49304199,585.29717255)
\lineto(673.49304199,595.56978607)
\curveto(673.49304199,598.67413373)(676.23791412,601.1733017)(679.6474638,601.1733017)
\closepath
}
}
{
\newrgbcolor{curcolor}{0.45882353 0.45882353 0.45882353}
\pscustom[linestyle=none,fillstyle=solid,fillcolor=curcolor]
{
\newpath
\moveto(558.57922649,556.62300873)
\lineto(681.1287508,556.62300873)
\curveto(684.53830048,556.62300873)(687.28317261,554.12384076)(687.28317261,551.0194931)
\lineto(687.28317261,540.74687958)
\curveto(687.28317261,537.64253192)(684.53830048,535.14336395)(681.1287508,535.14336395)
\lineto(558.57922649,535.14336395)
\curveto(555.16967681,535.14336395)(552.42480469,537.64253192)(552.42480469,540.74687958)
\lineto(552.42480469,551.0194931)
\curveto(552.42480469,554.12384076)(555.16967681,556.62300873)(558.57922649,556.62300873)
\closepath
}
}
{
\newrgbcolor{curcolor}{0.45882353 0.45882353 0.45882353}
\pscustom[linestyle=none,fillstyle=solid,fillcolor=curcolor]
{
\newpath
\moveto(219.49432659,556.62300873)
\lineto(281.29025745,556.62300873)
\curveto(284.69980713,556.62300873)(287.44467926,554.12384076)(287.44467926,551.0194931)
\lineto(287.44467926,540.74687195)
\curveto(287.44467926,537.64252429)(284.69980713,535.14335632)(281.29025745,535.14335632)
\lineto(219.49432659,535.14335632)
\curveto(216.08477691,535.14335632)(213.33990479,537.64252429)(213.33990479,540.74687195)
\lineto(213.33990479,551.0194931)
\curveto(213.33990479,554.12384076)(216.08477691,556.62300873)(219.49432659,556.62300873)
\closepath
}
}
{
\newrgbcolor{curcolor}{0.45882353 0.45882353 0.45882353}
\pscustom[linestyle=none,fillstyle=solid,fillcolor=curcolor]
{
\newpath
\moveto(467.40451336,556.62300873)
\lineto(529.20044422,556.62300873)
\curveto(532.6099939,556.62300873)(535.35486603,554.12384076)(535.35486603,551.0194931)
\lineto(535.35486603,540.74687195)
\curveto(535.35486603,537.64252429)(532.6099939,535.14335632)(529.20044422,535.14335632)
\lineto(467.40451336,535.14335632)
\curveto(463.99496368,535.14335632)(461.25009155,537.64252429)(461.25009155,540.74687195)
\lineto(461.25009155,551.0194931)
\curveto(461.25009155,554.12384076)(463.99496368,556.62300873)(467.40451336,556.62300873)
\closepath
}
}
{
\newrgbcolor{curcolor}{0.45882353 0.45882353 0.45882353}
\pscustom[linestyle=none,fillstyle=solid,fillcolor=curcolor]
{
\newpath
\moveto(335.64044476,513.94564056)
\lineto(405.78999805,513.94564056)
\curveto(409.19954774,513.94564056)(411.94441986,511.4464726)(411.94441986,508.34212494)
\lineto(411.94441986,498.06951141)
\curveto(411.94441986,494.96516376)(409.19954774,492.46599579)(405.78999805,492.46599579)
\lineto(335.64044476,492.46599579)
\curveto(332.23089507,492.46599579)(329.48602295,494.96516376)(329.48602295,498.06951141)
\lineto(329.48602295,508.34212494)
\curveto(329.48602295,511.4464726)(332.23089507,513.94564056)(335.64044476,513.94564056)
\closepath
}
}
{
\newrgbcolor{curcolor}{0.45882353 0.45882353 0.45882353}
\pscustom[linestyle=none,fillstyle=solid,fillcolor=curcolor]
{
\newpath
\moveto(723.50177288,513.94564056)
\lineto(827.82522297,513.94564056)
\curveto(831.23477265,513.94564056)(833.97964478,511.4464726)(833.97964478,508.34212494)
\lineto(833.97964478,498.06951141)
\curveto(833.97964478,494.96516376)(831.23477265,492.46599579)(827.82522297,492.46599579)
\lineto(723.50177288,492.46599579)
\curveto(720.0922232,492.46599579)(717.34735107,494.96516376)(717.34735107,498.06951141)
\lineto(717.34735107,508.34212494)
\curveto(717.34735107,511.4464726)(720.0922232,513.94564056)(723.50177288,513.94564056)
\closepath
}
}
{
\newrgbcolor{curcolor}{0.45882353 0.45882353 0.45882353}
\pscustom[linestyle=none,fillstyle=solid,fillcolor=curcolor]
{
\newpath
\moveto(467.40451336,427.26985931)
\lineto(529.20044422,427.26985931)
\curveto(532.6099939,427.26985931)(535.35486603,424.77069135)(535.35486603,421.66634369)
\lineto(535.35486603,411.39372253)
\curveto(535.35486603,408.28937488)(532.6099939,405.79020691)(529.20044422,405.79020691)
\lineto(467.40451336,405.79020691)
\curveto(463.99496368,405.79020691)(461.25009155,408.28937488)(461.25009155,411.39372253)
\lineto(461.25009155,421.66634369)
\curveto(461.25009155,424.77069135)(463.99496368,427.26985931)(467.40451336,427.26985931)
\closepath
}
}
{
\newrgbcolor{curcolor}{0.45882353 0.45882353 0.45882353}
\pscustom[linestyle=none,fillstyle=solid,fillcolor=curcolor]
{
\newpath
\moveto(336.02014446,427.26985931)
\lineto(405.41027546,427.26985931)
\curveto(408.81982514,427.26985931)(411.56469727,424.77069135)(411.56469727,421.66634369)
\lineto(411.56469727,411.39373016)
\curveto(411.56469727,408.28938251)(408.81982514,405.79021454)(405.41027546,405.79021454)
\lineto(336.02014446,405.79021454)
\curveto(332.61059478,405.79021454)(329.86572266,408.28938251)(329.86572266,411.39373016)
\lineto(329.86572266,421.66634369)
\curveto(329.86572266,424.77069135)(332.61059478,427.26985931)(336.02014446,427.26985931)
\closepath
}
}
{
\newrgbcolor{curcolor}{0.45882353 0.45882353 0.45882353}
\pscustom[linestyle=none,fillstyle=solid,fillcolor=curcolor]
{
\newpath
\moveto(217.21606731,427.26985931)
\lineto(283.56852436,427.26985931)
\curveto(286.97807404,427.26985931)(289.72294617,424.77069135)(289.72294617,421.66634369)
\lineto(289.72294617,411.39373016)
\curveto(289.72294617,408.28938251)(286.97807404,405.79021454)(283.56852436,405.79021454)
\lineto(217.21606731,405.79021454)
\curveto(213.80651763,405.79021454)(211.06164551,408.28938251)(211.06164551,411.39373016)
\lineto(211.06164551,421.66634369)
\curveto(211.06164551,424.77069135)(213.80651763,427.26985931)(217.21606731,427.26985931)
\closepath
}
}
{
\newrgbcolor{curcolor}{0.45882353 0.45882353 0.45882353}
\pscustom[linestyle=none,fillstyle=solid,fillcolor=curcolor]
{
\newpath
\moveto(722.74237347,338.80562592)
\lineto(828.58468342,338.80562592)
\curveto(831.9942331,338.80562592)(834.73910522,336.30645795)(834.73910522,333.20211029)
\lineto(834.73910522,322.92949677)
\curveto(834.73910522,319.82514911)(831.9942331,317.32598114)(828.58468342,317.32598114)
\lineto(722.74237347,317.32598114)
\curveto(719.33282379,317.32598114)(716.58795166,319.82514911)(716.58795166,322.92949677)
\lineto(716.58795166,333.20211029)
\curveto(716.58795166,336.30645795)(719.33282379,338.80562592)(722.74237347,338.80562592)
\closepath
}
}
{
\newrgbcolor{curcolor}{0.45882353 0.45882353 0.45882353}
\pscustom[linestyle=none,fillstyle=solid,fillcolor=curcolor]
{
\newpath
\moveto(63.35546398,163.48079681)
\lineto(145.65575123,163.48079681)
\curveto(149.06530091,163.48079681)(151.81017303,160.98162885)(151.81017303,157.87728119)
\lineto(151.81017303,147.60466766)
\curveto(151.81017303,144.50032001)(149.06530091,142.00115204)(145.65575123,142.00115204)
\lineto(63.35546398,142.00115204)
\curveto(59.9459143,142.00115204)(57.20104218,144.50032001)(57.20104218,147.60466766)
\lineto(57.20104218,157.87728119)
\curveto(57.20104218,160.98162885)(59.9459143,163.48079681)(63.35546398,163.48079681)
\closepath
}
}
{
\newrgbcolor{curcolor}{0.45882353 0.45882353 0.45882353}
\pscustom[linestyle=none,fillstyle=solid,fillcolor=curcolor]
{
\newpath
\moveto(621.53442669,163.48079681)
\lineto(739.52741718,163.48079681)
\curveto(742.93696686,163.48079681)(745.68183899,160.98162885)(745.68183899,157.87728119)
\lineto(745.68183899,147.60466766)
\curveto(745.68183899,144.50032001)(742.93696686,142.00115204)(739.52741718,142.00115204)
\lineto(621.53442669,142.00115204)
\curveto(618.12487701,142.00115204)(615.38000488,144.50032001)(615.38000488,147.60466766)
\lineto(615.38000488,157.87728119)
\curveto(615.38000488,160.98162885)(618.12487701,163.48079681)(621.53442669,163.48079681)
\closepath
}
}
{
\newrgbcolor{curcolor}{0.45882353 0.45882353 0.45882353}
\pscustom[linestyle=none,fillstyle=solid,fillcolor=curcolor]
{
\newpath
\moveto(138.92496014,119.20796967)
\lineto(200.720891,119.20796967)
\curveto(204.13044068,119.20796967)(206.87531281,116.7088017)(206.87531281,113.60445404)
\lineto(206.87531281,103.33183289)
\curveto(206.87531281,100.22748523)(204.13044068,97.72831726)(200.720891,97.72831726)
\lineto(138.92496014,97.72831726)
\curveto(135.51541046,97.72831726)(132.77053833,100.22748523)(132.77053833,103.33183289)
\lineto(132.77053833,113.60445404)
\curveto(132.77053833,116.7088017)(135.51541046,119.20796967)(138.92496014,119.20796967)
\closepath
}
}
{
\newrgbcolor{curcolor}{0.45882353 0.45882353 0.45882353}
\pscustom[linestyle=none,fillstyle=solid,fillcolor=curcolor]
{
\newpath
\moveto(469.0610075,119.20796967)
\lineto(533.89461994,119.20796967)
\curveto(537.30416962,119.20796967)(540.04904175,116.7088017)(540.04904175,113.60445404)
\lineto(540.04904175,103.33184052)
\curveto(540.04904175,100.22749286)(537.30416962,97.72832489)(533.89461994,97.72832489)
\lineto(469.0610075,97.72832489)
\curveto(465.65145782,97.72832489)(462.90658569,100.22749286)(462.90658569,103.33184052)
\lineto(462.90658569,113.60445404)
\curveto(462.90658569,116.7088017)(465.65145782,119.20796967)(469.0610075,119.20796967)
\closepath
}
}
{
\newrgbcolor{curcolor}{0.45882353 0.45882353 0.45882353}
\pscustom[linestyle=none,fillstyle=solid,fillcolor=curcolor]
{
\newpath
\moveto(334.5012846,76.67147064)
\lineto(406.92908955,76.67147064)
\curveto(410.33863923,76.67147064)(413.08351135,74.17230267)(413.08351135,71.06795502)
\lineto(413.08351135,60.79534149)
\curveto(413.08351135,57.69099384)(410.33863923,55.19182587)(406.92908955,55.19182587)
\lineto(334.5012846,55.19182587)
\curveto(331.09173492,55.19182587)(328.34686279,57.69099384)(328.34686279,60.79534149)
\lineto(328.34686279,71.06795502)
\curveto(328.34686279,74.17230267)(331.09173492,76.67147064)(334.5012846,76.67147064)
\closepath
}
}
{
\newrgbcolor{curcolor}{0.45882353 0.45882353 0.45882353}
\pscustom[linestyle=none,fillstyle=solid,fillcolor=curcolor]
{
\newpath
\moveto(216.07693768,76.67147064)
\lineto(284.70764637,76.67147064)
\curveto(288.11719605,76.67147064)(290.86206818,74.17230267)(290.86206818,71.06795502)
\lineto(290.86206818,60.79534149)
\curveto(290.86206818,57.69099384)(288.11719605,55.19182587)(284.70764637,55.19182587)
\lineto(216.07693768,55.19182587)
\curveto(212.66738799,55.19182587)(209.92251587,57.69099384)(209.92251587,60.79534149)
\lineto(209.92251587,71.06795502)
\curveto(209.92251587,74.17230267)(212.66738799,76.67147064)(216.07693768,76.67147064)
\closepath
}
}
{
\newrgbcolor{curcolor}{0.45882353 0.45882353 0.45882353}
\pscustom[linestyle=none,fillstyle=solid,fillcolor=curcolor]
{
\newpath
\moveto(816.97973919,76.67147064)
\lineto(882.57277393,76.67147064)
\curveto(885.98232361,76.67147064)(888.72719574,74.17230267)(888.72719574,71.06795502)
\lineto(888.72719574,60.79534149)
\curveto(888.72719574,57.69099384)(885.98232361,55.19182587)(882.57277393,55.19182587)
\lineto(816.97973919,55.19182587)
\curveto(813.57018951,55.19182587)(810.82531738,57.69099384)(810.82531738,60.79534149)
\lineto(810.82531738,71.06795502)
\curveto(810.82531738,74.17230267)(813.57018951,76.67147064)(816.97973919,76.67147064)
\closepath
}
}
{
\newrgbcolor{curcolor}{0.45882353 0.45882353 0.45882353}
\pscustom[linestyle=none,fillstyle=solid,fillcolor=curcolor]
{
\newpath
\moveto(909.22528362,76.67147064)
\lineto(1008.99223042,76.67147064)
\curveto(1012.4017801,76.67147064)(1015.14665222,74.17230267)(1015.14665222,71.06795502)
\lineto(1015.14665222,60.79534149)
\curveto(1015.14665222,57.69099384)(1012.4017801,55.19182587)(1008.99223042,55.19182587)
\lineto(909.22528362,55.19182587)
\curveto(905.81573394,55.19182587)(903.07086182,57.69099384)(903.07086182,60.79534149)
\lineto(903.07086182,71.06795502)
\curveto(903.07086182,74.17230267)(905.81573394,76.67147064)(909.22528362,76.67147064)
\closepath
}
}
{
\newrgbcolor{curcolor}{0.45882353 0.45882353 0.45882353}
\pscustom[linestyle=none,fillstyle=solid,fillcolor=curcolor]
{
\newpath
\moveto(602.50122356,1259.54006195)
\lineto(638.47687244,1259.54006195)
\curveto(641.88642212,1259.54006195)(644.63129425,1257.04089398)(644.63129425,1253.93654633)
\lineto(644.63129425,1243.6639328)
\curveto(644.63129425,1240.55958514)(641.88642212,1238.06041718)(638.47687244,1238.06041718)
\lineto(602.50122356,1238.06041718)
\curveto(599.09167388,1238.06041718)(596.34680176,1240.55958514)(596.34680176,1243.6639328)
\lineto(596.34680176,1253.93654633)
\curveto(596.34680176,1257.04089398)(599.09167388,1259.54006195)(602.50122356,1259.54006195)
\closepath
}
}
{
\newrgbcolor{curcolor}{0.45882353 0.45882353 0.45882353}
\pscustom[linestyle=none,fillstyle=solid,fillcolor=curcolor]
{
\newpath
\moveto(602.50122356,1215.02120209)
\lineto(638.47687244,1215.02120209)
\curveto(641.88642212,1215.02120209)(644.63129425,1212.52203412)(644.63129425,1209.41768646)
\lineto(644.63129425,1199.14507294)
\curveto(644.63129425,1196.04072528)(641.88642212,1193.54155731)(638.47687244,1193.54155731)
\lineto(602.50122356,1193.54155731)
\curveto(599.09167388,1193.54155731)(596.34680176,1196.04072528)(596.34680176,1199.14507294)
\lineto(596.34680176,1209.41768646)
\curveto(596.34680176,1212.52203412)(599.09167388,1215.02120209)(602.50122356,1215.02120209)
\closepath
}
}
{
\newrgbcolor{curcolor}{0 0 0}
\pscustom[linestyle=none,fillstyle=solid,fillcolor=curcolor]
{
\newpath
\moveto(137.59375,1676.76815033)
\lineto(137.59375,1609.29940033)
}
}
{
\newrgbcolor{curcolor}{0 0 0}
\pscustom[linewidth=0.63506401,linecolor=curcolor]
{
\newpath
\moveto(137.59375,1676.76815033)
\lineto(137.59375,1609.29940033)
}
}
{
\newrgbcolor{curcolor}{0 0 0}
\pscustom[linestyle=none,fillstyle=solid,fillcolor=curcolor]
{
\newpath
\moveto(137.59375,1674.27869943)
\curveto(136.19152868,1674.27869943)(135.05349398,1675.41673413)(135.05349398,1676.81895545)
\curveto(135.05349398,1678.22117678)(136.19152868,1679.35921147)(137.59375,1679.35921147)
\curveto(138.99597132,1679.35921147)(140.13400602,1678.22117678)(140.13400602,1676.81895545)
\curveto(140.13400602,1675.41673413)(138.99597132,1674.27869943)(137.59375,1674.27869943)
\closepath
}
}
{
\newrgbcolor{curcolor}{0 0 0}
\pscustom[linewidth=0.63506401,linecolor=curcolor]
{
\newpath
\moveto(137.59375,1674.27869943)
\curveto(136.19152868,1674.27869943)(135.05349398,1675.41673413)(135.05349398,1676.81895545)
\curveto(135.05349398,1678.22117678)(136.19152868,1679.35921147)(137.59375,1679.35921147)
\curveto(138.99597132,1679.35921147)(140.13400602,1678.22117678)(140.13400602,1676.81895545)
\curveto(140.13400602,1675.41673413)(138.99597132,1674.27869943)(137.59375,1674.27869943)
\closepath
}
}
{
\newrgbcolor{curcolor}{0 0 0}
\pscustom[linestyle=none,fillstyle=solid,fillcolor=curcolor]
{
\newpath
\moveto(140.41159804,1616.08851816)
\lineto(137.60493639,1608.45602361)
\lineto(134.79827536,1616.08851858)
\curveto(136.4553269,1614.86916505)(138.72246101,1614.87619099)(140.41159804,1616.08851816)
\closepath
}
}
{
\newrgbcolor{curcolor}{0 0 0}
\pscustom[linestyle=none,fillstyle=solid,fillcolor=curcolor]
{
\newpath
\moveto(137.59375,1587.79940033)
\lineto(137.59375,1564.23690033)
}
}
{
\newrgbcolor{curcolor}{0 0 0}
\pscustom[linewidth=0.63506401,linecolor=curcolor]
{
\newpath
\moveto(137.59375,1587.79940033)
\lineto(137.59375,1564.23690033)
}
}
{
\newrgbcolor{curcolor}{0 0 0}
\pscustom[linestyle=none,fillstyle=solid,fillcolor=curcolor]
{
\newpath
\moveto(137.59375,1585.30994943)
\curveto(136.19152868,1585.30994943)(135.05349398,1586.44798413)(135.05349398,1587.85020545)
\curveto(135.05349398,1589.25242678)(136.19152868,1590.39046147)(137.59375,1590.39046147)
\curveto(138.99597132,1590.39046147)(140.13400602,1589.25242678)(140.13400602,1587.85020545)
\curveto(140.13400602,1586.44798413)(138.99597132,1585.30994943)(137.59375,1585.30994943)
\closepath
}
}
{
\newrgbcolor{curcolor}{0 0 0}
\pscustom[linewidth=0.63506401,linecolor=curcolor]
{
\newpath
\moveto(137.59375,1585.30994943)
\curveto(136.19152868,1585.30994943)(135.05349398,1586.44798413)(135.05349398,1587.85020545)
\curveto(135.05349398,1589.25242678)(136.19152868,1590.39046147)(137.59375,1590.39046147)
\curveto(138.99597132,1590.39046147)(140.13400602,1589.25242678)(140.13400602,1587.85020545)
\curveto(140.13400602,1586.44798413)(138.99597132,1585.30994943)(137.59375,1585.30994943)
\closepath
}
}
{
\newrgbcolor{curcolor}{0 0 0}
\pscustom[linestyle=none,fillstyle=solid,fillcolor=curcolor]
{
\newpath
\moveto(140.41159804,1571.02601816)
\lineto(137.60493639,1563.39352361)
\lineto(134.79827536,1571.02601858)
\curveto(136.4553269,1569.80666505)(138.72246101,1569.81369099)(140.41159804,1571.02601816)
\closepath
}
}
{
\newrgbcolor{curcolor}{0 0 0}
\pscustom[linestyle=none,fillstyle=solid,fillcolor=curcolor]
{
\newpath
\moveto(137.59375,1542.76815033)
\lineto(137.59375,1521.54940033)
}
}
{
\newrgbcolor{curcolor}{0 0 0}
\pscustom[linewidth=0.63506401,linecolor=curcolor]
{
\newpath
\moveto(137.59375,1542.76815033)
\lineto(137.59375,1521.54940033)
}
}
{
\newrgbcolor{curcolor}{0 0 0}
\pscustom[linestyle=none,fillstyle=solid,fillcolor=curcolor]
{
\newpath
\moveto(137.59375,1540.27869943)
\curveto(136.19152868,1540.27869943)(135.05349398,1541.41673413)(135.05349398,1542.81895545)
\curveto(135.05349398,1544.22117678)(136.19152868,1545.35921147)(137.59375,1545.35921147)
\curveto(138.99597132,1545.35921147)(140.13400602,1544.22117678)(140.13400602,1542.81895545)
\curveto(140.13400602,1541.41673413)(138.99597132,1540.27869943)(137.59375,1540.27869943)
\closepath
}
}
{
\newrgbcolor{curcolor}{0 0 0}
\pscustom[linewidth=0.63506401,linecolor=curcolor]
{
\newpath
\moveto(137.59375,1540.27869943)
\curveto(136.19152868,1540.27869943)(135.05349398,1541.41673413)(135.05349398,1542.81895545)
\curveto(135.05349398,1544.22117678)(136.19152868,1545.35921147)(137.59375,1545.35921147)
\curveto(138.99597132,1545.35921147)(140.13400602,1544.22117678)(140.13400602,1542.81895545)
\curveto(140.13400602,1541.41673413)(138.99597132,1540.27869943)(137.59375,1540.27869943)
\closepath
}
}
{
\newrgbcolor{curcolor}{0 0 0}
\pscustom[linestyle=none,fillstyle=solid,fillcolor=curcolor]
{
\newpath
\moveto(140.41159804,1528.33851816)
\lineto(137.60493639,1520.70602361)
\lineto(134.79827536,1528.33851858)
\curveto(136.4553269,1527.11916505)(138.72246101,1527.12619099)(140.41159804,1528.33851816)
\closepath
}
}
{
\newrgbcolor{curcolor}{0 0 0}
\pscustom[linestyle=none,fillstyle=solid,fillcolor=curcolor]
{
\newpath
\moveto(224.65625,1510.48690033)
\lineto(243.375,1510.48690033)
}
}
{
\newrgbcolor{curcolor}{0 0 0}
\pscustom[linewidth=0.63506401,linecolor=curcolor]
{
\newpath
\moveto(224.65625,1510.48690033)
\lineto(243.375,1510.48690033)
}
}
{
\newrgbcolor{curcolor}{0 0 0}
\pscustom[linestyle=none,fillstyle=solid,fillcolor=curcolor]
{
\newpath
\moveto(227.1457009,1510.48690033)
\curveto(227.1457009,1509.084679)(226.0076662,1507.94664431)(224.60544488,1507.94664431)
\curveto(223.20322355,1507.94664431)(222.06518886,1509.084679)(222.06518886,1510.48690033)
\curveto(222.06518886,1511.88912165)(223.20322355,1513.02715635)(224.60544488,1513.02715635)
\curveto(226.0076662,1513.02715635)(227.1457009,1511.88912165)(227.1457009,1510.48690033)
\closepath
}
}
{
\newrgbcolor{curcolor}{0 0 0}
\pscustom[linewidth=0.63506401,linecolor=curcolor]
{
\newpath
\moveto(227.1457009,1510.48690033)
\curveto(227.1457009,1509.084679)(226.0076662,1507.94664431)(224.60544488,1507.94664431)
\curveto(223.20322355,1507.94664431)(222.06518886,1509.084679)(222.06518886,1510.48690033)
\curveto(222.06518886,1511.88912165)(223.20322355,1513.02715635)(224.60544488,1513.02715635)
\curveto(226.0076662,1513.02715635)(227.1457009,1511.88912165)(227.1457009,1510.48690033)
\closepath
}
}
{
\newrgbcolor{curcolor}{0 0 0}
\pscustom[linestyle=none,fillstyle=solid,fillcolor=curcolor]
{
\newpath
\moveto(236.58588217,1513.30474837)
\lineto(244.21837672,1510.49808672)
\lineto(236.58588175,1507.69142569)
\curveto(237.80523528,1509.34847723)(237.79820934,1511.61561134)(236.58588217,1513.30474837)
\closepath
}
}
{
\newrgbcolor{curcolor}{0 0 0}
\pscustom[linestyle=none,fillstyle=solid,fillcolor=curcolor]
{
\newpath
\moveto(280.65625,1500.39315033)
\lineto(170.65625,1476.45565033)
}
}
{
\newrgbcolor{curcolor}{0 0 0}
\pscustom[linewidth=0.63506401,linecolor=curcolor]
{
\newpath
\moveto(280.65625,1500.39315033)
\lineto(170.65625,1476.45565033)
}
}
{
\newrgbcolor{curcolor}{0 0 0}
\pscustom[linestyle=none,fillstyle=solid,fillcolor=curcolor]
{
\newpath
\moveto(278.22372974,1499.86380075)
\curveto(277.92556549,1501.23395502)(278.79558637,1502.58795266)(280.16574064,1502.88611691)
\curveto(281.53589491,1503.18428117)(282.88989255,1502.31426028)(283.1880568,1500.94410601)
\curveto(283.48622106,1499.57395175)(282.61620017,1498.2199541)(281.2460459,1497.92178985)
\curveto(279.87589163,1497.6236256)(278.52189399,1498.49364648)(278.22372974,1499.86380075)
\closepath
}
}
{
\newrgbcolor{curcolor}{0 0 0}
\pscustom[linewidth=0.63506401,linecolor=curcolor]
{
\newpath
\moveto(278.22372974,1499.86380075)
\curveto(277.92556549,1501.23395502)(278.79558637,1502.58795266)(280.16574064,1502.88611691)
\curveto(281.53589491,1503.18428117)(282.88989255,1502.31426028)(283.1880568,1500.94410601)
\curveto(283.48622106,1499.57395175)(282.61620017,1498.2199541)(281.2460459,1497.92178985)
\curveto(279.87589163,1497.6236256)(278.52189399,1498.49364648)(278.22372974,1499.86380075)
\closepath
}
}
{
\newrgbcolor{curcolor}{0 0 0}
\pscustom[linestyle=none,fillstyle=solid,fillcolor=curcolor]
{
\newpath
\moveto(177.88928815,1475.14586118)
\lineto(169.8345389,1476.2653866)
\lineto(176.69568799,1480.63081425)
\curveto(175.85657017,1478.75237764)(176.34551222,1476.53858403)(177.88928815,1475.14586118)
\closepath
}
}
{
\newrgbcolor{curcolor}{0 0 0}
\pscustom[linestyle=none,fillstyle=solid,fillcolor=curcolor]
{
\newpath
\moveto(329.8125,1500.08065033)
\lineto(329.8125,1478.83065033)
}
}
{
\newrgbcolor{curcolor}{0 0 0}
\pscustom[linewidth=0.63506401,linecolor=curcolor]
{
\newpath
\moveto(329.8125,1500.08065033)
\lineto(329.8125,1478.83065033)
}
}
{
\newrgbcolor{curcolor}{0 0 0}
\pscustom[linestyle=none,fillstyle=solid,fillcolor=curcolor]
{
\newpath
\moveto(329.8125,1497.59119943)
\curveto(328.41027868,1497.59119943)(327.27224398,1498.72923413)(327.27224398,1500.13145545)
\curveto(327.27224398,1501.53367678)(328.41027868,1502.67171147)(329.8125,1502.67171147)
\curveto(331.21472132,1502.67171147)(332.35275602,1501.53367678)(332.35275602,1500.13145545)
\curveto(332.35275602,1498.72923413)(331.21472132,1497.59119943)(329.8125,1497.59119943)
\closepath
}
}
{
\newrgbcolor{curcolor}{0 0 0}
\pscustom[linewidth=0.63506401,linecolor=curcolor]
{
\newpath
\moveto(329.8125,1497.59119943)
\curveto(328.41027868,1497.59119943)(327.27224398,1498.72923413)(327.27224398,1500.13145545)
\curveto(327.27224398,1501.53367678)(328.41027868,1502.67171147)(329.8125,1502.67171147)
\curveto(331.21472132,1502.67171147)(332.35275602,1501.53367678)(332.35275602,1500.13145545)
\curveto(332.35275602,1498.72923413)(331.21472132,1497.59119943)(329.8125,1497.59119943)
\closepath
}
}
{
\newrgbcolor{curcolor}{0 0 0}
\pscustom[linestyle=none,fillstyle=solid,fillcolor=curcolor]
{
\newpath
\moveto(332.63034804,1485.61976816)
\lineto(329.82368639,1477.98727361)
\lineto(327.01702536,1485.61976858)
\curveto(328.6740769,1484.40041505)(330.94121101,1484.40744099)(332.63034804,1485.61976816)
\closepath
}
}
{
\newrgbcolor{curcolor}{0 0 0}
\pscustom[linestyle=none,fillstyle=solid,fillcolor=curcolor]
{
\newpath
\moveto(329.8125,1457.36190033)
\lineto(329.8125,1433.48690033)
}
}
{
\newrgbcolor{curcolor}{0 0 0}
\pscustom[linewidth=0.63506401,linecolor=curcolor]
{
\newpath
\moveto(329.8125,1457.36190033)
\lineto(329.8125,1433.48690033)
}
}
{
\newrgbcolor{curcolor}{0 0 0}
\pscustom[linestyle=none,fillstyle=solid,fillcolor=curcolor]
{
\newpath
\moveto(329.8125,1454.87244943)
\curveto(328.41027868,1454.87244943)(327.27224398,1456.01048413)(327.27224398,1457.41270545)
\curveto(327.27224398,1458.81492678)(328.41027868,1459.95296147)(329.8125,1459.95296147)
\curveto(331.21472132,1459.95296147)(332.35275602,1458.81492678)(332.35275602,1457.41270545)
\curveto(332.35275602,1456.01048413)(331.21472132,1454.87244943)(329.8125,1454.87244943)
\closepath
}
}
{
\newrgbcolor{curcolor}{0 0 0}
\pscustom[linewidth=0.63506401,linecolor=curcolor]
{
\newpath
\moveto(329.8125,1454.87244943)
\curveto(328.41027868,1454.87244943)(327.27224398,1456.01048413)(327.27224398,1457.41270545)
\curveto(327.27224398,1458.81492678)(328.41027868,1459.95296147)(329.8125,1459.95296147)
\curveto(331.21472132,1459.95296147)(332.35275602,1458.81492678)(332.35275602,1457.41270545)
\curveto(332.35275602,1456.01048413)(331.21472132,1454.87244943)(329.8125,1454.87244943)
\closepath
}
}
{
\newrgbcolor{curcolor}{0 0 0}
\pscustom[linestyle=none,fillstyle=solid,fillcolor=curcolor]
{
\newpath
\moveto(332.63034804,1440.27601816)
\lineto(329.82368639,1432.64352361)
\lineto(327.01702536,1440.27601858)
\curveto(328.6740769,1439.05666505)(330.94121101,1439.06369099)(332.63034804,1440.27601816)
\closepath
}
}
{
\newrgbcolor{curcolor}{0 0 0}
\pscustom[linestyle=none,fillstyle=solid,fillcolor=curcolor]
{
\newpath
\moveto(354.125,1411.70565033)
\lineto(594.9375,1304.61190033)
}
}
{
\newrgbcolor{curcolor}{0 0 0}
\pscustom[linewidth=0.63506401,linecolor=curcolor]
{
\newpath
\moveto(354.125,1411.70565033)
\lineto(594.9375,1304.61190033)
}
}
{
\newrgbcolor{curcolor}{0 0 0}
\pscustom[linestyle=none,fillstyle=solid,fillcolor=curcolor]
{
\newpath
\moveto(356.39965765,1410.69406822)
\curveto(355.82986854,1409.41283248)(354.32758751,1408.83542652)(353.04635177,1409.40521563)
\curveto(351.76511603,1409.97500474)(351.18771007,1411.47728577)(351.75749918,1412.75852151)
\curveto(352.32728829,1414.03975725)(353.82956932,1414.61716321)(355.11080506,1414.0473741)
\curveto(356.3920408,1413.47758499)(356.96944676,1411.97530395)(356.39965765,1410.69406822)
\closepath
}
}
{
\newrgbcolor{curcolor}{0 0 0}
\pscustom[linewidth=0.63506401,linecolor=curcolor]
{
\newpath
\moveto(356.39965765,1410.69406822)
\curveto(355.82986854,1409.41283248)(354.32758751,1408.83542652)(353.04635177,1409.40521563)
\curveto(351.76511603,1409.97500474)(351.18771007,1411.47728577)(351.75749918,1412.75852151)
\curveto(352.32728829,1414.03975725)(353.82956932,1414.61716321)(355.11080506,1414.0473741)
\curveto(356.3920408,1413.47758499)(356.96944676,1411.97530395)(356.39965765,1410.69406822)
\closepath
}
}
{
\newrgbcolor{curcolor}{0 0 0}
\pscustom[linestyle=none,fillstyle=solid,fillcolor=curcolor]
{
\newpath
\moveto(589.87918207,1309.94536151)
\lineto(595.71265457,1304.27941753)
\lineto(587.59822213,1304.81636421)
\curveto(589.38570689,1305.83496183)(590.3005314,1307.90933946)(589.87918207,1309.94536151)
\closepath
}
}
{
\newrgbcolor{curcolor}{0 0 0}
\pscustom[linestyle=none,fillstyle=solid,fillcolor=curcolor]
{
\newpath
\moveto(329.8125,1412.01815033)
\lineto(329.8125,1261.14315033)
}
}
{
\newrgbcolor{curcolor}{0 0 0}
\pscustom[linewidth=0.63506401,linecolor=curcolor]
{
\newpath
\moveto(329.8125,1412.01815033)
\lineto(329.8125,1261.14315033)
}
}
{
\newrgbcolor{curcolor}{0 0 0}
\pscustom[linestyle=none,fillstyle=solid,fillcolor=curcolor]
{
\newpath
\moveto(329.8125,1409.52869943)
\curveto(328.41027868,1409.52869943)(327.27224398,1410.66673413)(327.27224398,1412.06895545)
\curveto(327.27224398,1413.47117678)(328.41027868,1414.60921147)(329.8125,1414.60921147)
\curveto(331.21472132,1414.60921147)(332.35275602,1413.47117678)(332.35275602,1412.06895545)
\curveto(332.35275602,1410.66673413)(331.21472132,1409.52869943)(329.8125,1409.52869943)
\closepath
}
}
{
\newrgbcolor{curcolor}{0 0 0}
\pscustom[linewidth=0.63506401,linecolor=curcolor]
{
\newpath
\moveto(329.8125,1409.52869943)
\curveto(328.41027868,1409.52869943)(327.27224398,1410.66673413)(327.27224398,1412.06895545)
\curveto(327.27224398,1413.47117678)(328.41027868,1414.60921147)(329.8125,1414.60921147)
\curveto(331.21472132,1414.60921147)(332.35275602,1413.47117678)(332.35275602,1412.06895545)
\curveto(332.35275602,1410.66673413)(331.21472132,1409.52869943)(329.8125,1409.52869943)
\closepath
}
}
{
\newrgbcolor{curcolor}{0 0 0}
\pscustom[linestyle=none,fillstyle=solid,fillcolor=curcolor]
{
\newpath
\moveto(332.63034804,1267.93226816)
\lineto(329.82368639,1260.29977361)
\lineto(327.01702536,1267.93226858)
\curveto(328.6740769,1266.71291505)(330.94121101,1266.71994099)(332.63034804,1267.93226816)
\closepath
}
}
{
\newrgbcolor{curcolor}{0 0 0}
\pscustom[linestyle=none,fillstyle=solid,fillcolor=curcolor]
{
\newpath
\moveto(618.90625,1283.42440033)
\lineto(618.90625,1261.14315033)
}
}
{
\newrgbcolor{curcolor}{0 0 0}
\pscustom[linewidth=0.63506401,linecolor=curcolor]
{
\newpath
\moveto(618.90625,1283.42440033)
\lineto(618.90625,1261.14315033)
}
}
{
\newrgbcolor{curcolor}{0 0 0}
\pscustom[linestyle=none,fillstyle=solid,fillcolor=curcolor]
{
\newpath
\moveto(618.90625,1280.93494943)
\curveto(617.50402868,1280.93494943)(616.36599398,1282.07298413)(616.36599398,1283.47520545)
\curveto(616.36599398,1284.87742678)(617.50402868,1286.01546147)(618.90625,1286.01546147)
\curveto(620.30847132,1286.01546147)(621.44650602,1284.87742678)(621.44650602,1283.47520545)
\curveto(621.44650602,1282.07298413)(620.30847132,1280.93494943)(618.90625,1280.93494943)
\closepath
}
}
{
\newrgbcolor{curcolor}{0 0 0}
\pscustom[linewidth=0.63506401,linecolor=curcolor]
{
\newpath
\moveto(618.90625,1280.93494943)
\curveto(617.50402868,1280.93494943)(616.36599398,1282.07298413)(616.36599398,1283.47520545)
\curveto(616.36599398,1284.87742678)(617.50402868,1286.01546147)(618.90625,1286.01546147)
\curveto(620.30847132,1286.01546147)(621.44650602,1284.87742678)(621.44650602,1283.47520545)
\curveto(621.44650602,1282.07298413)(620.30847132,1280.93494943)(618.90625,1280.93494943)
\closepath
}
}
{
\newrgbcolor{curcolor}{0 0 0}
\pscustom[linestyle=none,fillstyle=solid,fillcolor=curcolor]
{
\newpath
\moveto(621.72409804,1267.93226816)
\lineto(618.91743639,1260.29977361)
\lineto(616.11077536,1267.93226858)
\curveto(617.7678269,1266.71291505)(620.03496101,1266.71994099)(621.72409804,1267.93226816)
\closepath
}
}
{
\newrgbcolor{curcolor}{0 0 0}
\pscustom[linestyle=none,fillstyle=solid,fillcolor=curcolor]
{
\newpath
\moveto(618.90625,1239.64315033)
\lineto(618.90625,1216.61190033)
}
}
{
\newrgbcolor{curcolor}{0 0 0}
\pscustom[linewidth=0.63506401,linecolor=curcolor]
{
\newpath
\moveto(618.90625,1239.64315033)
\lineto(618.90625,1216.61190033)
}
}
{
\newrgbcolor{curcolor}{0 0 0}
\pscustom[linestyle=none,fillstyle=solid,fillcolor=curcolor]
{
\newpath
\moveto(618.90625,1237.15369943)
\curveto(617.50402868,1237.15369943)(616.36599398,1238.29173413)(616.36599398,1239.69395545)
\curveto(616.36599398,1241.09617678)(617.50402868,1242.23421147)(618.90625,1242.23421147)
\curveto(620.30847132,1242.23421147)(621.44650602,1241.09617678)(621.44650602,1239.69395545)
\curveto(621.44650602,1238.29173413)(620.30847132,1237.15369943)(618.90625,1237.15369943)
\closepath
}
}
{
\newrgbcolor{curcolor}{0 0 0}
\pscustom[linewidth=0.63506401,linecolor=curcolor]
{
\newpath
\moveto(618.90625,1237.15369943)
\curveto(617.50402868,1237.15369943)(616.36599398,1238.29173413)(616.36599398,1239.69395545)
\curveto(616.36599398,1241.09617678)(617.50402868,1242.23421147)(618.90625,1242.23421147)
\curveto(620.30847132,1242.23421147)(621.44650602,1241.09617678)(621.44650602,1239.69395545)
\curveto(621.44650602,1238.29173413)(620.30847132,1237.15369943)(618.90625,1237.15369943)
\closepath
}
}
{
\newrgbcolor{curcolor}{0 0 0}
\pscustom[linestyle=none,fillstyle=solid,fillcolor=curcolor]
{
\newpath
\moveto(621.72409804,1223.40101816)
\lineto(618.91743639,1215.76852361)
\lineto(616.11077536,1223.40101858)
\curveto(617.7678269,1222.18166505)(620.03496101,1222.18869099)(621.72409804,1223.40101816)
\closepath
}
}
{
\newrgbcolor{curcolor}{0 0 0}
\pscustom[linestyle=none,fillstyle=solid,fillcolor=curcolor]
{
\newpath
\moveto(133.375,1457.36190033)
\lineto(133.375,1129.54940033)
}
}
{
\newrgbcolor{curcolor}{0 0 0}
\pscustom[linewidth=0.63506401,linecolor=curcolor]
{
\newpath
\moveto(133.375,1457.36190033)
\lineto(133.375,1129.54940033)
}
}
{
\newrgbcolor{curcolor}{0 0 0}
\pscustom[linestyle=none,fillstyle=solid,fillcolor=curcolor]
{
\newpath
\moveto(133.375,1454.87244943)
\curveto(131.97277868,1454.87244943)(130.83474398,1456.01048413)(130.83474398,1457.41270545)
\curveto(130.83474398,1458.81492678)(131.97277868,1459.95296147)(133.375,1459.95296147)
\curveto(134.77722132,1459.95296147)(135.91525602,1458.81492678)(135.91525602,1457.41270545)
\curveto(135.91525602,1456.01048413)(134.77722132,1454.87244943)(133.375,1454.87244943)
\closepath
}
}
{
\newrgbcolor{curcolor}{0 0 0}
\pscustom[linewidth=0.63506401,linecolor=curcolor]
{
\newpath
\moveto(133.375,1454.87244943)
\curveto(131.97277868,1454.87244943)(130.83474398,1456.01048413)(130.83474398,1457.41270545)
\curveto(130.83474398,1458.81492678)(131.97277868,1459.95296147)(133.375,1459.95296147)
\curveto(134.77722132,1459.95296147)(135.91525602,1458.81492678)(135.91525602,1457.41270545)
\curveto(135.91525602,1456.01048413)(134.77722132,1454.87244943)(133.375,1454.87244943)
\closepath
}
}
{
\newrgbcolor{curcolor}{0 0 0}
\pscustom[linestyle=none,fillstyle=solid,fillcolor=curcolor]
{
\newpath
\moveto(136.19284804,1136.33851816)
\lineto(133.38618639,1128.70602361)
\lineto(130.57952536,1136.33851858)
\curveto(132.2365769,1135.11916505)(134.50371101,1135.12619099)(136.19284804,1136.33851816)
\closepath
}
}
{
\newrgbcolor{curcolor}{0 0 0}
\pscustom[linestyle=none,fillstyle=solid,fillcolor=curcolor]
{
\newpath
\moveto(482.125,1239.64315033)
\lineto(482.125,996.29940033)
}
}
{
\newrgbcolor{curcolor}{0 0 0}
\pscustom[linewidth=0.63506401,linecolor=curcolor]
{
\newpath
\moveto(482.125,1239.64315033)
\lineto(482.125,996.29940033)
}
}
{
\newrgbcolor{curcolor}{0 0 0}
\pscustom[linestyle=none,fillstyle=solid,fillcolor=curcolor]
{
\newpath
\moveto(482.125,1237.15369943)
\curveto(480.72277868,1237.15369943)(479.58474398,1238.29173413)(479.58474398,1239.69395545)
\curveto(479.58474398,1241.09617678)(480.72277868,1242.23421147)(482.125,1242.23421147)
\curveto(483.52722132,1242.23421147)(484.66525602,1241.09617678)(484.66525602,1239.69395545)
\curveto(484.66525602,1238.29173413)(483.52722132,1237.15369943)(482.125,1237.15369943)
\closepath
}
}
{
\newrgbcolor{curcolor}{0 0 0}
\pscustom[linewidth=0.63506401,linecolor=curcolor]
{
\newpath
\moveto(482.125,1237.15369943)
\curveto(480.72277868,1237.15369943)(479.58474398,1238.29173413)(479.58474398,1239.69395545)
\curveto(479.58474398,1241.09617678)(480.72277868,1242.23421147)(482.125,1242.23421147)
\curveto(483.52722132,1242.23421147)(484.66525602,1241.09617678)(484.66525602,1239.69395545)
\curveto(484.66525602,1238.29173413)(483.52722132,1237.15369943)(482.125,1237.15369943)
\closepath
}
}
{
\newrgbcolor{curcolor}{0 0 0}
\pscustom[linestyle=none,fillstyle=solid,fillcolor=curcolor]
{
\newpath
\moveto(484.94284804,1003.08851816)
\lineto(482.13618639,995.45602361)
\lineto(479.32952536,1003.08851858)
\curveto(480.9865769,1001.86916505)(483.25371101,1001.87619099)(484.94284804,1003.08851816)
\closepath
}
}
{
\newrgbcolor{curcolor}{0 0 0}
\pscustom[linestyle=none,fillstyle=solid,fillcolor=curcolor]
{
\newpath
\moveto(514.5,1240.04940033)
\lineto(595.1875,1213.98690033)
}
}
{
\newrgbcolor{curcolor}{0 0 0}
\pscustom[linewidth=0.63506401,linecolor=curcolor]
{
\newpath
\moveto(514.5,1240.04940033)
\lineto(595.1875,1213.98690033)
}
}
{
\newrgbcolor{curcolor}{0 0 0}
\pscustom[linestyle=none,fillstyle=solid,fillcolor=curcolor]
{
\newpath
\moveto(516.86893769,1239.28422061)
\curveto(516.4379385,1237.9498802)(515.00519912,1237.21673369)(513.6708587,1237.64773288)
\curveto(512.33651829,1238.07873207)(511.60337178,1239.51147146)(512.03437097,1240.84581187)
\curveto(512.46537016,1242.18015229)(513.89810955,1242.91329879)(515.23244997,1242.4822996)
\curveto(516.56679038,1242.05130041)(517.29993688,1240.61856103)(516.86893769,1239.28422061)
\closepath
}
}
{
\newrgbcolor{curcolor}{0 0 0}
\pscustom[linewidth=0.63506401,linecolor=curcolor]
{
\newpath
\moveto(516.86893769,1239.28422061)
\curveto(516.4379385,1237.9498802)(515.00519912,1237.21673369)(513.6708587,1237.64773288)
\curveto(512.33651829,1238.07873207)(511.60337178,1239.51147146)(512.03437097,1240.84581187)
\curveto(512.46537016,1242.18015229)(513.89810955,1242.91329879)(515.23244997,1242.4822996)
\curveto(516.56679038,1242.05130041)(517.29993688,1240.61856103)(516.86893769,1239.28422061)
\closepath
}
}
{
\newrgbcolor{curcolor}{0 0 0}
\pscustom[linestyle=none,fillstyle=solid,fillcolor=curcolor]
{
\newpath
\moveto(589.59315912,1218.7551011)
\lineto(595.99348757,1213.73831744)
\lineto(587.86779805,1213.413517)
\curveto(589.53744928,1214.6155601)(590.22760992,1216.77510281)(589.59315912,1218.7551011)
\closepath
}
}
{
\newrgbcolor{curcolor}{0 0 0}
\pscustom[linestyle=none,fillstyle=solid,fillcolor=curcolor]
{
\newpath
\moveto(618.90625,1195.14315033)
\lineto(618.90625,1173.26815033)
}
}
{
\newrgbcolor{curcolor}{0 0 0}
\pscustom[linewidth=0.63506401,linecolor=curcolor]
{
\newpath
\moveto(618.90625,1195.14315033)
\lineto(618.90625,1173.26815033)
}
}
{
\newrgbcolor{curcolor}{0 0 0}
\pscustom[linestyle=none,fillstyle=solid,fillcolor=curcolor]
{
\newpath
\moveto(618.90625,1192.65369943)
\curveto(617.50402868,1192.65369943)(616.36599398,1193.79173413)(616.36599398,1195.19395545)
\curveto(616.36599398,1196.59617678)(617.50402868,1197.73421147)(618.90625,1197.73421147)
\curveto(620.30847132,1197.73421147)(621.44650602,1196.59617678)(621.44650602,1195.19395545)
\curveto(621.44650602,1193.79173413)(620.30847132,1192.65369943)(618.90625,1192.65369943)
\closepath
}
}
{
\newrgbcolor{curcolor}{0 0 0}
\pscustom[linewidth=0.63506401,linecolor=curcolor]
{
\newpath
\moveto(618.90625,1192.65369943)
\curveto(617.50402868,1192.65369943)(616.36599398,1193.79173413)(616.36599398,1195.19395545)
\curveto(616.36599398,1196.59617678)(617.50402868,1197.73421147)(618.90625,1197.73421147)
\curveto(620.30847132,1197.73421147)(621.44650602,1196.59617678)(621.44650602,1195.19395545)
\curveto(621.44650602,1193.79173413)(620.30847132,1192.65369943)(618.90625,1192.65369943)
\closepath
}
}
{
\newrgbcolor{curcolor}{0 0 0}
\pscustom[linestyle=none,fillstyle=solid,fillcolor=curcolor]
{
\newpath
\moveto(621.72409804,1180.05726816)
\lineto(618.91743639,1172.42477361)
\lineto(616.11077536,1180.05726858)
\curveto(617.7678269,1178.83791505)(620.03496101,1178.84494099)(621.72409804,1180.05726816)
\closepath
}
}
{
\newrgbcolor{curcolor}{0 0 0}
\pscustom[linestyle=none,fillstyle=solid,fillcolor=curcolor]
{
\newpath
\moveto(618.90625,1151.79940033)
\lineto(618.90625,1129.54940033)
}
}
{
\newrgbcolor{curcolor}{0 0 0}
\pscustom[linewidth=0.63506401,linecolor=curcolor]
{
\newpath
\moveto(618.90625,1151.79940033)
\lineto(618.90625,1129.54940033)
}
}
{
\newrgbcolor{curcolor}{0 0 0}
\pscustom[linestyle=none,fillstyle=solid,fillcolor=curcolor]
{
\newpath
\moveto(618.90625,1149.30994943)
\curveto(617.50402868,1149.30994943)(616.36599398,1150.44798413)(616.36599398,1151.85020545)
\curveto(616.36599398,1153.25242678)(617.50402868,1154.39046147)(618.90625,1154.39046147)
\curveto(620.30847132,1154.39046147)(621.44650602,1153.25242678)(621.44650602,1151.85020545)
\curveto(621.44650602,1150.44798413)(620.30847132,1149.30994943)(618.90625,1149.30994943)
\closepath
}
}
{
\newrgbcolor{curcolor}{0 0 0}
\pscustom[linewidth=0.63506401,linecolor=curcolor]
{
\newpath
\moveto(618.90625,1149.30994943)
\curveto(617.50402868,1149.30994943)(616.36599398,1150.44798413)(616.36599398,1151.85020545)
\curveto(616.36599398,1153.25242678)(617.50402868,1154.39046147)(618.90625,1154.39046147)
\curveto(620.30847132,1154.39046147)(621.44650602,1153.25242678)(621.44650602,1151.85020545)
\curveto(621.44650602,1150.44798413)(620.30847132,1149.30994943)(618.90625,1149.30994943)
\closepath
}
}
{
\newrgbcolor{curcolor}{0 0 0}
\pscustom[linestyle=none,fillstyle=solid,fillcolor=curcolor]
{
\newpath
\moveto(621.72409804,1136.33851816)
\lineto(618.91743639,1128.70602361)
\lineto(616.11077536,1136.33851858)
\curveto(617.7678269,1135.11916505)(620.03496101,1135.12619099)(621.72409804,1136.33851816)
\closepath
}
}
{
\newrgbcolor{curcolor}{0 0 0}
\pscustom[linestyle=none,fillstyle=solid,fillcolor=curcolor]
{
\newpath
\moveto(134,1108.08065033)
\lineto(134,1084.61190033)
}
}
{
\newrgbcolor{curcolor}{0 0 0}
\pscustom[linewidth=0.63506401,linecolor=curcolor]
{
\newpath
\moveto(134,1108.08065033)
\lineto(134,1084.61190033)
}
}
{
\newrgbcolor{curcolor}{0 0 0}
\pscustom[linestyle=none,fillstyle=solid,fillcolor=curcolor]
{
\newpath
\moveto(134,1105.59119943)
\curveto(132.59777868,1105.59119943)(131.45974398,1106.72923413)(131.45974398,1108.13145545)
\curveto(131.45974398,1109.53367678)(132.59777868,1110.67171147)(134,1110.67171147)
\curveto(135.40222132,1110.67171147)(136.54025602,1109.53367678)(136.54025602,1108.13145545)
\curveto(136.54025602,1106.72923413)(135.40222132,1105.59119943)(134,1105.59119943)
\closepath
}
}
{
\newrgbcolor{curcolor}{0 0 0}
\pscustom[linewidth=0.63506401,linecolor=curcolor]
{
\newpath
\moveto(134,1105.59119943)
\curveto(132.59777868,1105.59119943)(131.45974398,1106.72923413)(131.45974398,1108.13145545)
\curveto(131.45974398,1109.53367678)(132.59777868,1110.67171147)(134,1110.67171147)
\curveto(135.40222132,1110.67171147)(136.54025602,1109.53367678)(136.54025602,1108.13145545)
\curveto(136.54025602,1106.72923413)(135.40222132,1105.59119943)(134,1105.59119943)
\closepath
}
}
{
\newrgbcolor{curcolor}{0 0 0}
\pscustom[linestyle=none,fillstyle=solid,fillcolor=curcolor]
{
\newpath
\moveto(136.81784804,1091.40101816)
\lineto(134.01118639,1083.76852361)
\lineto(131.20452536,1091.40101858)
\curveto(132.8615769,1090.18166505)(135.12871101,1090.18869099)(136.81784804,1091.40101816)
\closepath
}
}
{
\newrgbcolor{curcolor}{0 0 0}
\pscustom[linestyle=none,fillstyle=solid,fillcolor=curcolor]
{
\newpath
\moveto(134,1063.14315033)
\lineto(134,867.95565033)
}
}
{
\newrgbcolor{curcolor}{0 0 0}
\pscustom[linewidth=0.63506401,linecolor=curcolor]
{
\newpath
\moveto(134,1063.14315033)
\lineto(134,867.95565033)
}
}
{
\newrgbcolor{curcolor}{0 0 0}
\pscustom[linestyle=none,fillstyle=solid,fillcolor=curcolor]
{
\newpath
\moveto(134,1060.65369943)
\curveto(132.59777868,1060.65369943)(131.45974398,1061.79173413)(131.45974398,1063.19395545)
\curveto(131.45974398,1064.59617678)(132.59777868,1065.73421147)(134,1065.73421147)
\curveto(135.40222132,1065.73421147)(136.54025602,1064.59617678)(136.54025602,1063.19395545)
\curveto(136.54025602,1061.79173413)(135.40222132,1060.65369943)(134,1060.65369943)
\closepath
}
}
{
\newrgbcolor{curcolor}{0 0 0}
\pscustom[linewidth=0.63506401,linecolor=curcolor]
{
\newpath
\moveto(134,1060.65369943)
\curveto(132.59777868,1060.65369943)(131.45974398,1061.79173413)(131.45974398,1063.19395545)
\curveto(131.45974398,1064.59617678)(132.59777868,1065.73421147)(134,1065.73421147)
\curveto(135.40222132,1065.73421147)(136.54025602,1064.59617678)(136.54025602,1063.19395545)
\curveto(136.54025602,1061.79173413)(135.40222132,1060.65369943)(134,1060.65369943)
\closepath
}
}
{
\newrgbcolor{curcolor}{0 0 0}
\pscustom[linestyle=none,fillstyle=solid,fillcolor=curcolor]
{
\newpath
\moveto(136.81784804,874.74476816)
\lineto(134.01118639,867.11227361)
\lineto(131.20452536,874.74476858)
\curveto(132.8615769,873.52541505)(135.12871101,873.53244099)(136.81784804,874.74476816)
\closepath
}
}
{
\newrgbcolor{curcolor}{0 0 0}
\pscustom[linestyle=none,fillstyle=solid,fillcolor=curcolor]
{
\newpath
\moveto(484.03125,974.79940033)
\lineto(484.03125,951.23690033)
}
}
{
\newrgbcolor{curcolor}{0 0 0}
\pscustom[linewidth=0.63506401,linecolor=curcolor]
{
\newpath
\moveto(484.03125,974.79940033)
\lineto(484.03125,951.23690033)
}
}
{
\newrgbcolor{curcolor}{0 0 0}
\pscustom[linestyle=none,fillstyle=solid,fillcolor=curcolor]
{
\newpath
\moveto(484.03125,972.30994943)
\curveto(482.62902868,972.30994943)(481.49099398,973.44798413)(481.49099398,974.85020545)
\curveto(481.49099398,976.25242678)(482.62902868,977.39046147)(484.03125,977.39046147)
\curveto(485.43347132,977.39046147)(486.57150602,976.25242678)(486.57150602,974.85020545)
\curveto(486.57150602,973.44798413)(485.43347132,972.30994943)(484.03125,972.30994943)
\closepath
}
}
{
\newrgbcolor{curcolor}{0 0 0}
\pscustom[linewidth=0.63506401,linecolor=curcolor]
{
\newpath
\moveto(484.03125,972.30994943)
\curveto(482.62902868,972.30994943)(481.49099398,973.44798413)(481.49099398,974.85020545)
\curveto(481.49099398,976.25242678)(482.62902868,977.39046147)(484.03125,977.39046147)
\curveto(485.43347132,977.39046147)(486.57150602,976.25242678)(486.57150602,974.85020545)
\curveto(486.57150602,973.44798413)(485.43347132,972.30994943)(484.03125,972.30994943)
\closepath
}
}
{
\newrgbcolor{curcolor}{0 0 0}
\pscustom[linestyle=none,fillstyle=solid,fillcolor=curcolor]
{
\newpath
\moveto(486.84909804,958.02601816)
\lineto(484.04243639,950.39352361)
\lineto(481.23577536,958.02601858)
\curveto(482.8928269,956.80666505)(485.15996101,956.81369099)(486.84909804,958.02601816)
\closepath
}
}
{
\newrgbcolor{curcolor}{0 0 0}
\pscustom[linestyle=none,fillstyle=solid,fillcolor=curcolor]
{
\newpath
\moveto(176.3125,1063.45565033)
\lineto(441.25,995.98690033)
}
}
{
\newrgbcolor{curcolor}{0 0 0}
\pscustom[linewidth=0.63506401,linecolor=curcolor]
{
\newpath
\moveto(176.3125,1063.45565033)
\lineto(441.25,995.98690033)
}
}
{
\newrgbcolor{curcolor}{0 0 0}
\pscustom[linestyle=none,fillstyle=solid,fillcolor=curcolor]
{
\newpath
\moveto(178.72495427,1062.84129685)
\curveto(178.37891027,1061.48244506)(176.99522673,1060.66045613)(175.63637494,1061.00650013)
\curveto(174.27752314,1061.35254412)(173.45553421,1062.73622766)(173.80157821,1064.09507946)
\curveto(174.14762221,1065.45393125)(175.53130575,1066.27592019)(176.89015754,1065.92987619)
\curveto(178.24900933,1065.58383219)(179.07099827,1064.20014865)(178.72495427,1062.84129685)
\closepath
}
}
{
\newrgbcolor{curcolor}{0 0 0}
\pscustom[linewidth=0.63506401,linecolor=curcolor]
{
\newpath
\moveto(178.72495427,1062.84129685)
\curveto(178.37891027,1061.48244506)(176.99522673,1060.66045613)(175.63637494,1061.00650013)
\curveto(174.27752314,1061.35254412)(173.45553421,1062.73622766)(173.80157821,1064.09507946)
\curveto(174.14762221,1065.45393125)(175.53130575,1066.27592019)(176.89015754,1065.92987619)
\curveto(178.24900933,1065.58383219)(179.07099827,1064.20014865)(178.72495427,1062.84129685)
\closepath
}
}
{
\newrgbcolor{curcolor}{0 0 0}
\pscustom[linestyle=none,fillstyle=solid,fillcolor=curcolor]
{
\newpath
\moveto(435.36626012,1000.39303168)
\lineto(442.07005239,995.78960993)
\lineto(433.98098865,994.95332447)
\curveto(435.57156029,996.25820942)(436.12424118,998.45695685)(435.36626012,1000.39303168)
\closepath
}
}
{
\newrgbcolor{curcolor}{0 0 0}
\pscustom[linestyle=none,fillstyle=solid,fillcolor=curcolor]
{
\newpath
\moveto(608.0625,1108.29940033)
\lineto(494.875,996.08065033)
}
}
{
\newrgbcolor{curcolor}{0 0 0}
\pscustom[linewidth=0.63506401,linecolor=curcolor]
{
\newpath
\moveto(608.0625,1108.29940033)
\lineto(494.875,996.08065033)
}
}
{
\newrgbcolor{curcolor}{0 0 0}
\pscustom[linestyle=none,fillstyle=solid,fillcolor=curcolor]
{
\newpath
\moveto(606.29464327,1106.54667434)
\curveto(605.30739353,1107.54244671)(605.31431044,1109.15185595)(606.31008281,1110.13910569)
\curveto(607.30585517,1111.12635543)(608.91526441,1111.11943851)(609.90251415,1110.12366615)
\curveto(610.88976389,1109.12789379)(610.88284697,1107.51848455)(609.88707461,1106.53123481)
\curveto(608.89130225,1105.54398507)(607.28189301,1105.55090198)(606.29464327,1106.54667434)
\closepath
}
}
{
\newrgbcolor{curcolor}{0 0 0}
\pscustom[linewidth=0.63506401,linecolor=curcolor]
{
\newpath
\moveto(606.29464327,1106.54667434)
\curveto(605.30739353,1107.54244671)(605.31431044,1109.15185595)(606.31008281,1110.13910569)
\curveto(607.30585517,1111.12635543)(608.91526441,1111.11943851)(609.90251415,1110.12366615)
\curveto(610.88976389,1109.12789379)(610.88284697,1107.51848455)(609.88707461,1106.53123481)
\curveto(608.89130225,1105.54398507)(607.28189301,1105.55090198)(606.29464327,1106.54667434)
\closepath
}
}
{
\newrgbcolor{curcolor}{0 0 0}
\pscustom[linestyle=none,fillstyle=solid,fillcolor=curcolor]
{
\newpath
\moveto(501.68015655,998.85954089)
\lineto(494.28396102,995.47891755)
\lineto(497.72803371,1002.8457818)
\curveto(498.02878877,1000.81054489)(499.62997951,999.20551073)(501.68015655,998.85954089)
\closepath
}
}
{
\newrgbcolor{curcolor}{0 0 0}
\pscustom[linestyle=none,fillstyle=solid,fillcolor=curcolor]
{
\newpath
\moveto(618.90625,1108.08065033)
\lineto(618.90625,951.23690033)
}
}
{
\newrgbcolor{curcolor}{0 0 0}
\pscustom[linewidth=0.63506401,linecolor=curcolor]
{
\newpath
\moveto(618.90625,1108.08065033)
\lineto(618.90625,951.23690033)
}
}
{
\newrgbcolor{curcolor}{0 0 0}
\pscustom[linestyle=none,fillstyle=solid,fillcolor=curcolor]
{
\newpath
\moveto(618.90625,1105.59119943)
\curveto(617.50402868,1105.59119943)(616.36599398,1106.72923413)(616.36599398,1108.13145545)
\curveto(616.36599398,1109.53367678)(617.50402868,1110.67171147)(618.90625,1110.67171147)
\curveto(620.30847132,1110.67171147)(621.44650602,1109.53367678)(621.44650602,1108.13145545)
\curveto(621.44650602,1106.72923413)(620.30847132,1105.59119943)(618.90625,1105.59119943)
\closepath
}
}
{
\newrgbcolor{curcolor}{0 0 0}
\pscustom[linewidth=0.63506401,linecolor=curcolor]
{
\newpath
\moveto(618.90625,1105.59119943)
\curveto(617.50402868,1105.59119943)(616.36599398,1106.72923413)(616.36599398,1108.13145545)
\curveto(616.36599398,1109.53367678)(617.50402868,1110.67171147)(618.90625,1110.67171147)
\curveto(620.30847132,1110.67171147)(621.44650602,1109.53367678)(621.44650602,1108.13145545)
\curveto(621.44650602,1106.72923413)(620.30847132,1105.59119943)(618.90625,1105.59119943)
\closepath
}
}
{
\newrgbcolor{curcolor}{0 0 0}
\pscustom[linestyle=none,fillstyle=solid,fillcolor=curcolor]
{
\newpath
\moveto(621.72409804,958.02601816)
\lineto(618.91743639,950.39352361)
\lineto(616.11077536,958.02601858)
\curveto(617.7678269,956.80666505)(620.03496101,956.81369099)(621.72409804,958.02601816)
\closepath
}
}
{
\newrgbcolor{curcolor}{0 0 0}
\pscustom[linestyle=none,fillstyle=solid,fillcolor=curcolor]
{
\newpath
\moveto(590.90625,940.17440033)
\lineto(572.8125,940.17440033)
}
}
{
\newrgbcolor{curcolor}{0 0 0}
\pscustom[linewidth=0.63506401,linecolor=curcolor]
{
\newpath
\moveto(590.90625,940.17440033)
\lineto(572.8125,940.17440033)
}
}
{
\newrgbcolor{curcolor}{0 0 0}
\pscustom[linestyle=none,fillstyle=solid,fillcolor=curcolor]
{
\newpath
\moveto(588.4167991,940.17440033)
\curveto(588.4167991,941.57662165)(589.5548338,942.71465635)(590.95705512,942.71465635)
\curveto(592.35927645,942.71465635)(593.49731114,941.57662165)(593.49731114,940.17440033)
\curveto(593.49731114,938.772179)(592.35927645,937.63414431)(590.95705512,937.63414431)
\curveto(589.5548338,937.63414431)(588.4167991,938.772179)(588.4167991,940.17440033)
\closepath
}
}
{
\newrgbcolor{curcolor}{0 0 0}
\pscustom[linewidth=0.63506401,linecolor=curcolor]
{
\newpath
\moveto(588.4167991,940.17440033)
\curveto(588.4167991,941.57662165)(589.5548338,942.71465635)(590.95705512,942.71465635)
\curveto(592.35927645,942.71465635)(593.49731114,941.57662165)(593.49731114,940.17440033)
\curveto(593.49731114,938.772179)(592.35927645,937.63414431)(590.95705512,937.63414431)
\curveto(589.5548338,937.63414431)(588.4167991,938.772179)(588.4167991,940.17440033)
\closepath
}
}
{
\newrgbcolor{curcolor}{0 0 0}
\pscustom[linestyle=none,fillstyle=solid,fillcolor=curcolor]
{
\newpath
\moveto(579.60161783,937.35655229)
\lineto(571.96912328,940.16321394)
\lineto(579.60161825,942.96987497)
\curveto(578.38226472,941.31282343)(578.38929066,939.04568932)(579.60161783,937.35655229)
\closepath
}
}
{
\newrgbcolor{curcolor}{0 0 0}
\pscustom[linestyle=none,fillstyle=solid,fillcolor=curcolor]
{
\newpath
\moveto(618.90625,929.73690033)
\lineto(618.90625,865.11190033)
}
}
{
\newrgbcolor{curcolor}{0 0 0}
\pscustom[linewidth=0.63506401,linecolor=curcolor]
{
\newpath
\moveto(618.90625,929.73690033)
\lineto(618.90625,865.11190033)
}
}
{
\newrgbcolor{curcolor}{0 0 0}
\pscustom[linestyle=none,fillstyle=solid,fillcolor=curcolor]
{
\newpath
\moveto(618.90625,927.24744943)
\curveto(617.50402868,927.24744943)(616.36599398,928.38548413)(616.36599398,929.78770545)
\curveto(616.36599398,931.18992678)(617.50402868,932.32796147)(618.90625,932.32796147)
\curveto(620.30847132,932.32796147)(621.44650602,931.18992678)(621.44650602,929.78770545)
\curveto(621.44650602,928.38548413)(620.30847132,927.24744943)(618.90625,927.24744943)
\closepath
}
}
{
\newrgbcolor{curcolor}{0 0 0}
\pscustom[linewidth=0.63506401,linecolor=curcolor]
{
\newpath
\moveto(618.90625,927.24744943)
\curveto(617.50402868,927.24744943)(616.36599398,928.38548413)(616.36599398,929.78770545)
\curveto(616.36599398,931.18992678)(617.50402868,932.32796147)(618.90625,932.32796147)
\curveto(620.30847132,932.32796147)(621.44650602,931.18992678)(621.44650602,929.78770545)
\curveto(621.44650602,928.38548413)(620.30847132,927.24744943)(618.90625,927.24744943)
\closepath
}
}
{
\newrgbcolor{curcolor}{0 0 0}
\pscustom[linestyle=none,fillstyle=solid,fillcolor=curcolor]
{
\newpath
\moveto(621.72409804,871.90101816)
\lineto(618.91743639,864.26852361)
\lineto(616.11077536,871.90101858)
\curveto(617.7678269,870.68166505)(620.03496101,870.68869099)(621.72409804,871.90101816)
\closepath
}
}
{
\newrgbcolor{curcolor}{0 0 0}
\pscustom[linestyle=none,fillstyle=solid,fillcolor=curcolor]
{
\newpath
\moveto(483.75,929.73690033)
\lineto(483.75,820.92440033)
}
}
{
\newrgbcolor{curcolor}{0 0 0}
\pscustom[linewidth=0.63506401,linecolor=curcolor]
{
\newpath
\moveto(483.75,929.73690033)
\lineto(483.75,820.92440033)
}
}
{
\newrgbcolor{curcolor}{0 0 0}
\pscustom[linestyle=none,fillstyle=solid,fillcolor=curcolor]
{
\newpath
\moveto(483.75,927.24744943)
\curveto(482.34777868,927.24744943)(481.20974398,928.38548413)(481.20974398,929.78770545)
\curveto(481.20974398,931.18992678)(482.34777868,932.32796147)(483.75,932.32796147)
\curveto(485.15222132,932.32796147)(486.29025602,931.18992678)(486.29025602,929.78770545)
\curveto(486.29025602,928.38548413)(485.15222132,927.24744943)(483.75,927.24744943)
\closepath
}
}
{
\newrgbcolor{curcolor}{0 0 0}
\pscustom[linewidth=0.63506401,linecolor=curcolor]
{
\newpath
\moveto(483.75,927.24744943)
\curveto(482.34777868,927.24744943)(481.20974398,928.38548413)(481.20974398,929.78770545)
\curveto(481.20974398,931.18992678)(482.34777868,932.32796147)(483.75,932.32796147)
\curveto(485.15222132,932.32796147)(486.29025602,931.18992678)(486.29025602,929.78770545)
\curveto(486.29025602,928.38548413)(485.15222132,927.24744943)(483.75,927.24744943)
\closepath
}
}
{
\newrgbcolor{curcolor}{0 0 0}
\pscustom[linestyle=none,fillstyle=solid,fillcolor=curcolor]
{
\newpath
\moveto(486.56784804,827.71351816)
\lineto(483.76118639,820.08102361)
\lineto(480.95452536,827.71351858)
\curveto(482.6115769,826.49416505)(484.87871101,826.50119099)(486.56784804,827.71351816)
\closepath
}
}
{
\newrgbcolor{curcolor}{0 0 0}
\pscustom[linestyle=none,fillstyle=solid,fillcolor=curcolor]
{
\newpath
\moveto(642.53125,1151.48690033)
\lineto(691.03125,1129.26815033)
}
}
{
\newrgbcolor{curcolor}{0 0 0}
\pscustom[linewidth=0.63506401,linecolor=curcolor]
{
\newpath
\moveto(642.53125,1151.48690033)
\lineto(691.03125,1129.26815033)
}
}
{
\newrgbcolor{curcolor}{0 0 0}
\pscustom[linestyle=none,fillstyle=solid,fillcolor=curcolor]
{
\newpath
\moveto(644.79450632,1150.45006061)
\curveto(644.21049048,1149.17524684)(642.70187515,1148.61459925)(641.42706138,1149.19861509)
\curveto(640.15224761,1149.78263094)(639.59160002,1151.29124627)(640.17561587,1152.56606004)
\curveto(640.75963171,1153.84087381)(642.26824704,1154.4015214)(643.54306081,1153.81750555)
\curveto(644.81787458,1153.23348971)(645.37852217,1151.72487438)(644.79450632,1150.45006061)
\closepath
}
}
{
\newrgbcolor{curcolor}{0 0 0}
\pscustom[linewidth=0.63506401,linecolor=curcolor]
{
\newpath
\moveto(644.79450632,1150.45006061)
\curveto(644.21049048,1149.17524684)(642.70187515,1148.61459925)(641.42706138,1149.19861509)
\curveto(640.15224761,1149.78263094)(639.59160002,1151.29124627)(640.17561587,1152.56606004)
\curveto(640.75963171,1153.84087381)(642.26824704,1154.4015214)(643.54306081,1153.81750555)
\curveto(644.81787458,1153.23348971)(645.37852217,1151.72487438)(644.79450632,1150.45006061)
\closepath
}
}
{
\newrgbcolor{curcolor}{0 0 0}
\pscustom[linestyle=none,fillstyle=solid,fillcolor=curcolor]
{
\newpath
\moveto(686.03261473,1134.65758758)
\lineto(691.80265552,1128.92705952)
\lineto(683.69470283,1129.55429849)
\curveto(685.49341533,1130.55293567)(686.43127403,1132.61700144)(686.03261473,1134.65758758)
\closepath
}
}
{
\newrgbcolor{curcolor}{0 0 0}
\pscustom[linestyle=none,fillstyle=solid,fillcolor=curcolor]
{
\newpath
\moveto(714.28125,1108.08065033)
\lineto(714.28125,910.20565033)
}
}
{
\newrgbcolor{curcolor}{0 0 0}
\pscustom[linewidth=0.63506401,linecolor=curcolor]
{
\newpath
\moveto(714.28125,1108.08065033)
\lineto(714.28125,910.20565033)
}
}
{
\newrgbcolor{curcolor}{0 0 0}
\pscustom[linestyle=none,fillstyle=solid,fillcolor=curcolor]
{
\newpath
\moveto(714.28125,1105.59119943)
\curveto(712.87902868,1105.59119943)(711.74099398,1106.72923413)(711.74099398,1108.13145545)
\curveto(711.74099398,1109.53367678)(712.87902868,1110.67171147)(714.28125,1110.67171147)
\curveto(715.68347132,1110.67171147)(716.82150602,1109.53367678)(716.82150602,1108.13145545)
\curveto(716.82150602,1106.72923413)(715.68347132,1105.59119943)(714.28125,1105.59119943)
\closepath
}
}
{
\newrgbcolor{curcolor}{0 0 0}
\pscustom[linewidth=0.63506401,linecolor=curcolor]
{
\newpath
\moveto(714.28125,1105.59119943)
\curveto(712.87902868,1105.59119943)(711.74099398,1106.72923413)(711.74099398,1108.13145545)
\curveto(711.74099398,1109.53367678)(712.87902868,1110.67171147)(714.28125,1110.67171147)
\curveto(715.68347132,1110.67171147)(716.82150602,1109.53367678)(716.82150602,1108.13145545)
\curveto(716.82150602,1106.72923413)(715.68347132,1105.59119943)(714.28125,1105.59119943)
\closepath
}
}
{
\newrgbcolor{curcolor}{0 0 0}
\pscustom[linestyle=none,fillstyle=solid,fillcolor=curcolor]
{
\newpath
\moveto(717.09909804,916.99476816)
\lineto(714.29243639,909.36227361)
\lineto(711.48577536,916.99476858)
\curveto(713.1428269,915.77541505)(715.40996101,915.78244099)(717.09909804,916.99476816)
\closepath
}
}
{
\newrgbcolor{curcolor}{0 0 0}
\pscustom[linestyle=none,fillstyle=solid,fillcolor=curcolor]
{
\newpath
\moveto(957.21875,1063.14315033)
\lineto(957.21875,686.73690033)
}
}
{
\newrgbcolor{curcolor}{0 0 0}
\pscustom[linewidth=0.63506401,linecolor=curcolor]
{
\newpath
\moveto(957.21875,1063.14315033)
\lineto(957.21875,686.73690033)
}
}
{
\newrgbcolor{curcolor}{0 0 0}
\pscustom[linestyle=none,fillstyle=solid,fillcolor=curcolor]
{
\newpath
\moveto(957.21875,1060.65369943)
\curveto(955.81652868,1060.65369943)(954.67849398,1061.79173413)(954.67849398,1063.19395545)
\curveto(954.67849398,1064.59617678)(955.81652868,1065.73421147)(957.21875,1065.73421147)
\curveto(958.62097132,1065.73421147)(959.75900602,1064.59617678)(959.75900602,1063.19395545)
\curveto(959.75900602,1061.79173413)(958.62097132,1060.65369943)(957.21875,1060.65369943)
\closepath
}
}
{
\newrgbcolor{curcolor}{0 0 0}
\pscustom[linewidth=0.63506401,linecolor=curcolor]
{
\newpath
\moveto(957.21875,1060.65369943)
\curveto(955.81652868,1060.65369943)(954.67849398,1061.79173413)(954.67849398,1063.19395545)
\curveto(954.67849398,1064.59617678)(955.81652868,1065.73421147)(957.21875,1065.73421147)
\curveto(958.62097132,1065.73421147)(959.75900602,1064.59617678)(959.75900602,1063.19395545)
\curveto(959.75900602,1061.79173413)(958.62097132,1060.65369943)(957.21875,1060.65369943)
\closepath
}
}
{
\newrgbcolor{curcolor}{0 0 0}
\pscustom[linestyle=none,fillstyle=solid,fillcolor=curcolor]
{
\newpath
\moveto(960.03659804,693.52601816)
\lineto(957.22993639,685.89352361)
\lineto(954.42327536,693.52601858)
\curveto(956.0803269,692.30666505)(958.34746101,692.31369099)(960.03659804,693.52601816)
\closepath
}
}
{
\newrgbcolor{curcolor}{0 0 0}
\pscustom[linestyle=none,fillstyle=solid,fillcolor=curcolor]
{
\newpath
\moveto(912.375,713.54940033)
\lineto(944.53125,686.48690033)
}
}
{
\newrgbcolor{curcolor}{0 0 0}
\pscustom[linewidth=0.63506401,linecolor=curcolor]
{
\newpath
\moveto(912.375,713.54940033)
\lineto(944.53125,686.48690033)
}
}
{
\newrgbcolor{curcolor}{0 0 0}
\pscustom[linestyle=none,fillstyle=solid,fillcolor=curcolor]
{
\newpath
\moveto(914.27968879,711.94642609)
\curveto(913.37678901,710.87358098)(911.77328592,710.73565432)(910.7004408,711.6385541)
\curveto(909.62759569,712.54145388)(909.48966903,714.14495697)(910.39256881,715.21780209)
\curveto(911.29546859,716.2906472)(912.89897169,716.42857386)(913.9718168,715.52567408)
\curveto(915.04466191,714.6227743)(915.18258857,713.01927121)(914.27968879,711.94642609)
\closepath
}
}
{
\newrgbcolor{curcolor}{0 0 0}
\pscustom[linewidth=0.63506401,linecolor=curcolor]
{
\newpath
\moveto(914.27968879,711.94642609)
\curveto(913.37678901,710.87358098)(911.77328592,710.73565432)(910.7004408,711.6385541)
\curveto(909.62759569,712.54145388)(909.48966903,714.14495697)(910.39256881,715.21780209)
\curveto(911.29546859,716.2906472)(912.89897169,716.42857386)(913.9718168,715.52567408)
\curveto(915.04466191,714.6227743)(915.18258857,713.01927121)(914.27968879,711.94642609)
\closepath
}
}
{
\newrgbcolor{curcolor}{0 0 0}
\pscustom[linestyle=none,fillstyle=solid,fillcolor=curcolor]
{
\newpath
\moveto(941.15130031,693.01440587)
\lineto(945.18372387,685.95240312)
\lineto(937.53684363,688.71963062)
\curveto(939.53676254,689.20229739)(940.99120992,690.94141477)(941.15130031,693.01440587)
\closepath
}
}
{
\newrgbcolor{curcolor}{0 0 0}
\pscustom[linestyle=none,fillstyle=solid,fillcolor=curcolor]
{
\newpath
\moveto(848.5,888.73690033)
\lineto(848.5,78.26810033)
}
}
{
\newrgbcolor{curcolor}{0 0 0}
\pscustom[linewidth=0.63506401,linecolor=curcolor]
{
\newpath
\moveto(848.5,888.73690033)
\lineto(848.5,78.26810033)
}
}
{
\newrgbcolor{curcolor}{0 0 0}
\pscustom[linestyle=none,fillstyle=solid,fillcolor=curcolor]
{
\newpath
\moveto(848.5,886.24744943)
\curveto(847.09777868,886.24744943)(845.95974398,887.38548413)(845.95974398,888.78770545)
\curveto(845.95974398,890.18992678)(847.09777868,891.32796147)(848.5,891.32796147)
\curveto(849.90222132,891.32796147)(851.04025602,890.18992678)(851.04025602,888.78770545)
\curveto(851.04025602,887.38548413)(849.90222132,886.24744943)(848.5,886.24744943)
\closepath
}
}
{
\newrgbcolor{curcolor}{0 0 0}
\pscustom[linewidth=0.63506401,linecolor=curcolor]
{
\newpath
\moveto(848.5,886.24744943)
\curveto(847.09777868,886.24744943)(845.95974398,887.38548413)(845.95974398,888.78770545)
\curveto(845.95974398,890.18992678)(847.09777868,891.32796147)(848.5,891.32796147)
\curveto(849.90222132,891.32796147)(851.04025602,890.18992678)(851.04025602,888.78770545)
\curveto(851.04025602,887.38548413)(849.90222132,886.24744943)(848.5,886.24744943)
\closepath
}
}
{
\newrgbcolor{curcolor}{0 0 0}
\pscustom[linestyle=none,fillstyle=solid,fillcolor=curcolor]
{
\newpath
\moveto(851.31784804,85.05721816)
\lineto(848.51118639,77.42472361)
\lineto(845.70452536,85.05721858)
\curveto(847.3615769,83.83786505)(849.62871101,83.84489099)(851.31784804,85.05721816)
\closepath
}
}
{
\newrgbcolor{curcolor}{0 0 0}
\pscustom[linestyle=none,fillstyle=solid,fillcolor=curcolor]
{
\newpath
\moveto(136.1875,846.42440033)
\lineto(165.625,734.54940033)
}
}
{
\newrgbcolor{curcolor}{0 0 0}
\pscustom[linewidth=0.63506401,linecolor=curcolor]
{
\newpath
\moveto(136.1875,846.42440033)
\lineto(165.625,734.54940033)
}
}
{
\newrgbcolor{curcolor}{0 0 0}
\pscustom[linestyle=none,fillstyle=solid,fillcolor=curcolor]
{
\newpath
\moveto(136.82098237,844.01689833)
\curveto(135.46492002,843.66007969)(134.07475573,844.47106009)(133.71793709,845.82712244)
\curveto(133.36111845,847.18318479)(134.17209886,848.57334908)(135.52816121,848.93016772)
\curveto(136.88422356,849.28698636)(138.27438784,848.47600596)(138.63120648,847.11994361)
\curveto(138.98802512,845.76388125)(138.17704472,844.37371697)(136.82098237,844.01689833)
\closepath
}
}
{
\newrgbcolor{curcolor}{0 0 0}
\pscustom[linewidth=0.63506401,linecolor=curcolor]
{
\newpath
\moveto(136.82098237,844.01689833)
\curveto(135.46492002,843.66007969)(134.07475573,844.47106009)(133.71793709,845.82712244)
\curveto(133.36111845,847.18318479)(134.17209886,848.57334908)(135.52816121,848.93016772)
\curveto(136.88422356,849.28698636)(138.27438784,848.47600596)(138.63120648,847.11994361)
\curveto(138.98802512,845.76388125)(138.17704472,844.37371697)(136.82098237,844.01689833)
\closepath
}
}
{
\newrgbcolor{curcolor}{0 0 0}
\pscustom[linestyle=none,fillstyle=solid,fillcolor=curcolor]
{
\newpath
\moveto(166.62248436,741.83207933)
\lineto(165.85042944,733.73663284)
\lineto(161.19394353,740.403676)
\curveto(163.10673236,739.64612617)(165.29744803,740.22983098)(166.62248436,741.83207933)
\closepath
}
}
{
\newrgbcolor{curcolor}{0 0 0}
\pscustom[linestyle=none,fillstyle=solid,fillcolor=curcolor]
{
\newpath
\moveto(582.125,761.23690033)
\lineto(337.21875,729.14315033)
}
}
{
\newrgbcolor{curcolor}{0 0 0}
\pscustom[linewidth=0.63506401,linecolor=curcolor]
{
\newpath
\moveto(582.125,761.23690033)
\lineto(337.21875,729.14315033)
}
}
{
\newrgbcolor{curcolor}{0 0 0}
\pscustom[linestyle=none,fillstyle=solid,fillcolor=curcolor]
{
\newpath
\moveto(579.6566531,760.9134357)
\curveto(579.4744567,762.30376988)(580.45497431,763.58002658)(581.84530848,763.76222298)
\curveto(583.23564266,763.94441938)(584.51189936,762.96390177)(584.69409576,761.57356759)
\curveto(584.87629216,760.18323342)(583.89577454,758.90697672)(582.50544037,758.72478032)
\curveto(581.1151062,758.54258392)(579.8388495,759.52310153)(579.6566531,760.9134357)
\closepath
}
}
{
\newrgbcolor{curcolor}{0 0 0}
\pscustom[linewidth=0.63506401,linecolor=curcolor]
{
\newpath
\moveto(579.6566531,760.9134357)
\curveto(579.4744567,762.30376988)(580.45497431,763.58002658)(581.84530848,763.76222298)
\curveto(583.23564266,763.94441938)(584.51189936,762.96390177)(584.69409576,761.57356759)
\curveto(584.87629216,760.18323342)(583.89577454,758.90697672)(582.50544037,758.72478032)
\curveto(581.1151062,758.54258392)(579.8388495,759.52310153)(579.6566531,760.9134357)
\closepath
}
}
{
\newrgbcolor{curcolor}{0 0 0}
\pscustom[linestyle=none,fillstyle=solid,fillcolor=curcolor]
{
\newpath
\moveto(344.31644858,727.23132832)
\lineto(336.38397639,729.02247536)
\lineto(343.58708682,732.79706484)
\curveto(342.59337775,730.9956251)(342.89492222,728.74862324)(344.31644858,727.23132832)
\closepath
}
}
{
\newrgbcolor{curcolor}{0 0 0}
\pscustom[linestyle=none,fillstyle=solid,fillcolor=curcolor]
{
\newpath
\moveto(287.6875,713.36190033)
\lineto(262,691.73690033)
}
}
{
\newrgbcolor{curcolor}{0 0 0}
\pscustom[linewidth=0.63506401,linecolor=curcolor]
{
\newpath
\moveto(287.6875,713.36190033)
\lineto(262,691.73690033)
}
}
{
\newrgbcolor{curcolor}{0 0 0}
\pscustom[linestyle=none,fillstyle=solid,fillcolor=curcolor]
{
\newpath
\moveto(285.78305083,711.75864142)
\curveto(284.8799907,712.83135156)(285.01767768,714.43487526)(286.09038783,715.33793538)
\curveto(287.16309797,716.24099551)(288.76662167,716.10330853)(289.66968179,715.03059838)
\curveto(290.57274192,713.95788824)(290.43505494,712.35436454)(289.36234479,711.45130442)
\curveto(288.28963465,710.54824429)(286.68611095,710.68593127)(285.78305083,711.75864142)
\closepath
}
}
{
\newrgbcolor{curcolor}{0 0 0}
\pscustom[linewidth=0.63506401,linecolor=curcolor]
{
\newpath
\moveto(285.78305083,711.75864142)
\curveto(284.8799907,712.83135156)(285.01767768,714.43487526)(286.09038783,715.33793538)
\curveto(287.16309797,716.24099551)(288.76662167,716.10330853)(289.66968179,715.03059838)
\curveto(290.57274192,713.95788824)(290.43505494,712.35436454)(289.36234479,711.45130442)
\curveto(288.28963465,710.54824429)(286.68611095,710.68593127)(285.78305083,711.75864142)
\closepath
}
}
{
\newrgbcolor{curcolor}{0 0 0}
\pscustom[linestyle=none,fillstyle=solid,fillcolor=curcolor]
{
\newpath
\moveto(269.00848119,693.95355996)
\lineto(261.36201457,691.18519025)
\lineto(265.39338325,698.24779545)
\curveto(265.52774454,696.19484881)(266.99320165,694.46499856)(269.00848119,693.95355996)
\closepath
}
}
{
\newrgbcolor{curcolor}{0 0 0}
\pscustom[linestyle=none,fillstyle=solid,fillcolor=curcolor]
{
\newpath
\moveto(310.5,713.36190033)
\lineto(330.75,691.73690033)
}
}
{
\newrgbcolor{curcolor}{0 0 0}
\pscustom[linewidth=0.63506401,linecolor=curcolor]
{
\newpath
\moveto(310.5,713.36190033)
\lineto(330.75,691.73690033)
}
}
{
\newrgbcolor{curcolor}{0 0 0}
\pscustom[linestyle=none,fillstyle=solid,fillcolor=curcolor]
{
\newpath
\moveto(312.2015894,711.54477091)
\curveto(311.17806344,710.58632464)(309.56950627,710.63914293)(308.61105999,711.66266889)
\curveto(307.65261372,712.68619485)(307.70543202,714.29475202)(308.72895797,715.2531983)
\curveto(309.75248393,716.21164457)(311.3610411,716.15882627)(312.31948738,715.13530032)
\curveto(313.27793365,714.11177436)(313.22511536,712.50321718)(312.2015894,711.54477091)
\closepath
}
}
{
\newrgbcolor{curcolor}{0 0 0}
\pscustom[linewidth=0.63506401,linecolor=curcolor]
{
\newpath
\moveto(312.2015894,711.54477091)
\curveto(311.17806344,710.58632464)(309.56950627,710.63914293)(308.61105999,711.66266889)
\curveto(307.65261372,712.68619485)(307.70543202,714.29475202)(308.72895797,715.2531983)
\curveto(309.75248393,716.21164457)(311.3610411,716.15882627)(312.31948738,715.13530032)
\curveto(313.27793365,714.11177436)(313.22511536,712.50321718)(312.2015894,711.54477091)
\closepath
}
}
{
\newrgbcolor{curcolor}{0 0 0}
\pscustom[linestyle=none,fillstyle=solid,fillcolor=curcolor]
{
\newpath
\moveto(328.16633935,698.61854883)
\lineto(331.33463012,691.12893895)
\lineto(324.06899624,694.78173101)
\curveto(326.11198337,695.02431389)(327.76203431,696.57907378)(328.16633935,698.61854883)
\closepath
}
}
{
\newrgbcolor{curcolor}{0 0 0}
\pscustom[linestyle=none,fillstyle=solid,fillcolor=curcolor]
{
\newpath
\moveto(377.59375,681.08060033)
\lineto(395.375,680.45560033)
}
}
{
\newrgbcolor{curcolor}{0 0 0}
\pscustom[linewidth=0.63506401,linecolor=curcolor]
{
\newpath
\moveto(377.59375,681.08060033)
\lineto(395.375,680.45560033)
}
}
{
\newrgbcolor{curcolor}{0 0 0}
\pscustom[linestyle=none,fillstyle=solid,fillcolor=curcolor]
{
\newpath
\moveto(380.08166449,680.99315167)
\curveto(380.0324077,679.59179575)(378.85509882,678.49443994)(377.4537429,678.54369674)
\curveto(376.05238699,678.59295354)(374.95503118,679.77026241)(375.00428798,681.17161833)
\curveto(375.05354477,682.57297425)(376.23085365,683.67033005)(377.63220957,683.62107326)
\curveto(379.03356548,683.57181646)(380.13092129,682.39450758)(380.08166449,680.99315167)
\closepath
}
}
{
\newrgbcolor{curcolor}{0 0 0}
\pscustom[linewidth=0.63506401,linecolor=curcolor]
{
\newpath
\moveto(380.08166449,680.99315167)
\curveto(380.0324077,679.59179575)(378.85509882,678.49443994)(377.4537429,678.54369674)
\curveto(376.05238699,678.59295354)(374.95503118,679.77026241)(375.00428798,681.17161833)
\curveto(375.05354477,682.57297425)(376.23085365,683.67033005)(377.63220957,683.62107326)
\curveto(379.03356548,683.57181646)(380.13092129,682.39450758)(380.08166449,680.99315167)
\closepath
}
}
{
\newrgbcolor{curcolor}{0 0 0}
\pscustom[linestyle=none,fillstyle=solid,fillcolor=curcolor]
{
\newpath
\moveto(388.68905669,683.51019532)
\lineto(396.21824916,680.43715394)
\lineto(388.4918732,677.90033702)
\curveto(389.76868258,679.51353281)(389.84130016,681.77951452)(388.68905669,683.51019532)
\closepath
}
}
{
\newrgbcolor{curcolor}{0 0 0}
\pscustom[linestyle=none,fillstyle=solid,fillcolor=curcolor]
{
\newpath
\moveto(395.3125,672.01810033)
\lineto(288.4375,645.23690033)
}
}
{
\newrgbcolor{curcolor}{0 0 0}
\pscustom[linewidth=0.63506401,linecolor=curcolor]
{
\newpath
\moveto(395.3125,672.01810033)
\lineto(288.4375,645.23690033)
}
}
{
\newrgbcolor{curcolor}{0 0 0}
\pscustom[linestyle=none,fillstyle=solid,fillcolor=curcolor]
{
\newpath
\moveto(392.89771025,671.41299186)
\curveto(392.55687364,672.77315915)(393.38415652,674.15368405)(394.74432381,674.49452066)
\curveto(396.10449109,674.83535726)(397.48501599,674.00807439)(397.8258526,672.6479071)
\curveto(398.1666892,671.28773981)(397.33940633,669.90721491)(395.97923904,669.56637831)
\curveto(394.61907175,669.2255417)(393.23854685,670.05282458)(392.89771025,671.41299186)
\closepath
}
}
{
\newrgbcolor{curcolor}{0 0 0}
\pscustom[linewidth=0.63506401,linecolor=curcolor]
{
\newpath
\moveto(392.89771025,671.41299186)
\curveto(392.55687364,672.77315915)(393.38415652,674.15368405)(394.74432381,674.49452066)
\curveto(396.10449109,674.83535726)(397.48501599,674.00807439)(397.8258526,672.6479071)
\curveto(398.1666892,671.28773981)(397.33940633,669.90721491)(395.97923904,669.56637831)
\curveto(394.61907175,669.2255417)(393.23854685,670.05282458)(392.89771025,671.41299186)
\closepath
}
}
{
\newrgbcolor{curcolor}{0 0 0}
\pscustom[linestyle=none,fillstyle=solid,fillcolor=curcolor]
{
\newpath
\moveto(295.70793696,644.15378683)
\lineto(287.62213607,645.02105066)
\lineto(294.34351236,649.59876038)
\curveto(293.56350642,647.69501839)(294.12139178,645.49758571)(295.70793696,644.15378683)
\closepath
}
}
{
\newrgbcolor{curcolor}{0 0 0}
\pscustom[linestyle=none,fillstyle=solid,fillcolor=curcolor]
{
\newpath
\moveto(447.625,670.29940033)
\lineto(481.1875,646.29940033)
}
}
{
\newrgbcolor{curcolor}{0 0 0}
\pscustom[linewidth=0.63506401,linecolor=curcolor]
{
\newpath
\moveto(447.625,670.29940033)
\lineto(481.1875,646.29940033)
}
}
{
\newrgbcolor{curcolor}{0 0 0}
\pscustom[linestyle=none,fillstyle=solid,fillcolor=curcolor]
{
\newpath
\moveto(449.64998487,668.85136646)
\curveto(448.83435762,667.71076274)(447.24669192,667.44701371)(446.1060882,668.26264095)
\curveto(444.96548448,669.0782682)(444.70173545,670.6659339)(445.51736269,671.80653762)
\curveto(446.33298993,672.94714134)(447.92065564,673.21089037)(449.06125936,672.39526313)
\curveto(450.20186308,671.57963589)(450.46561211,669.99197018)(449.64998487,668.85136646)
\closepath
}
}
{
\newrgbcolor{curcolor}{0 0 0}
\pscustom[linewidth=0.63506401,linecolor=curcolor]
{
\newpath
\moveto(449.64998487,668.85136646)
\curveto(448.83435762,667.71076274)(447.24669192,667.44701371)(446.1060882,668.26264095)
\curveto(444.96548448,669.0782682)(444.70173545,670.6659339)(445.51736269,671.80653762)
\curveto(446.33298993,672.94714134)(447.92065564,673.21089037)(449.06125936,672.39526313)
\curveto(450.20186308,671.57963589)(450.46561211,669.99197018)(449.64998487,668.85136646)
\closepath
}
}
{
\newrgbcolor{curcolor}{0 0 0}
\pscustom[linestyle=none,fillstyle=solid,fillcolor=curcolor]
{
\newpath
\moveto(477.30410489,652.5405245)
\lineto(481.88003157,645.8179344)
\lineto(474.0390145,647.97450038)
\curveto(475.99472257,648.61313081)(477.30772679,650.46136411)(477.30410489,652.5405245)
\closepath
}
}
{
\newrgbcolor{curcolor}{0 0 0}
\pscustom[linestyle=none,fillstyle=solid,fillcolor=curcolor]
{
\newpath
\moveto(424,670.20560033)
\lineto(372.75,602.95560033)
}
}
{
\newrgbcolor{curcolor}{0 0 0}
\pscustom[linewidth=0.63506401,linecolor=curcolor]
{
\newpath
\moveto(424,670.20560033)
\lineto(372.75,602.95560033)
}
}
{
\newrgbcolor{curcolor}{0 0 0}
\pscustom[linestyle=none,fillstyle=solid,fillcolor=curcolor]
{
\newpath
\moveto(422.49106341,668.2255811)
\curveto(421.37578727,669.07551273)(421.1604352,670.67046396)(422.01036684,671.7857401)
\curveto(422.86029847,672.90101624)(424.4552497,673.11636831)(425.57052584,672.26643667)
\curveto(426.68580198,671.41650504)(426.90115404,669.82155381)(426.05122241,668.70627767)
\curveto(425.20129078,667.59100153)(423.60633955,667.37564947)(422.49106341,668.2255811)
\closepath
}
}
{
\newrgbcolor{curcolor}{0 0 0}
\pscustom[linewidth=0.63506401,linecolor=curcolor]
{
\newpath
\moveto(422.49106341,668.2255811)
\curveto(421.37578727,669.07551273)(421.1604352,670.67046396)(422.01036684,671.7857401)
\curveto(422.86029847,672.90101624)(424.4552497,673.11636831)(425.57052584,672.26643667)
\curveto(426.68580198,671.41650504)(426.90115404,669.82155381)(426.05122241,668.70627767)
\curveto(425.20129078,667.59100153)(423.60633955,667.37564947)(422.49106341,668.2255811)
\closepath
}
}
{
\newrgbcolor{curcolor}{0 0 0}
\pscustom[linestyle=none,fillstyle=solid,fillcolor=curcolor]
{
\newpath
\moveto(379.10631801,606.64743047)
\lineto(372.24769938,602.27802855)
\lineto(374.64168438,610.04984696)
\curveto(375.22055371,608.0756248)(377.02800885,606.70702978)(379.10631801,606.64743047)
\closepath
}
}
{
\newrgbcolor{curcolor}{0 0 0}
\pscustom[linestyle=none,fillstyle=solid,fillcolor=curcolor]
{
\newpath
\moveto(496.40625,625.08060033)
\lineto(496.40625,602.76810033)
}
}
{
\newrgbcolor{curcolor}{0 0 0}
\pscustom[linewidth=0.63506401,linecolor=curcolor]
{
\newpath
\moveto(496.40625,625.08060033)
\lineto(496.40625,602.76810033)
}
}
{
\newrgbcolor{curcolor}{0 0 0}
\pscustom[linestyle=none,fillstyle=solid,fillcolor=curcolor]
{
\newpath
\moveto(496.40625,622.59114943)
\curveto(495.00402868,622.59114943)(493.86599398,623.72918413)(493.86599398,625.13140545)
\curveto(493.86599398,626.53362678)(495.00402868,627.67166147)(496.40625,627.67166147)
\curveto(497.80847132,627.67166147)(498.94650602,626.53362678)(498.94650602,625.13140545)
\curveto(498.94650602,623.72918413)(497.80847132,622.59114943)(496.40625,622.59114943)
\closepath
}
}
{
\newrgbcolor{curcolor}{0 0 0}
\pscustom[linewidth=0.63506401,linecolor=curcolor]
{
\newpath
\moveto(496.40625,622.59114943)
\curveto(495.00402868,622.59114943)(493.86599398,623.72918413)(493.86599398,625.13140545)
\curveto(493.86599398,626.53362678)(495.00402868,627.67166147)(496.40625,627.67166147)
\curveto(497.80847132,627.67166147)(498.94650602,626.53362678)(498.94650602,625.13140545)
\curveto(498.94650602,623.72918413)(497.80847132,622.59114943)(496.40625,622.59114943)
\closepath
}
}
{
\newrgbcolor{curcolor}{0 0 0}
\pscustom[linestyle=none,fillstyle=solid,fillcolor=curcolor]
{
\newpath
\moveto(499.22409804,609.55721816)
\lineto(496.41743639,601.92472361)
\lineto(493.61077536,609.55721858)
\curveto(495.2678269,608.33786505)(497.53496101,608.34489099)(499.22409804,609.55721816)
\closepath
}
}
{
\newrgbcolor{curcolor}{0 0 0}
\pscustom[linestyle=none,fillstyle=solid,fillcolor=curcolor]
{
\newpath
\moveto(496.40625,581.26810033)
\lineto(496.40625,558.20560033)
}
}
{
\newrgbcolor{curcolor}{0 0 0}
\pscustom[linewidth=0.63506401,linecolor=curcolor]
{
\newpath
\moveto(496.40625,581.26810033)
\lineto(496.40625,558.20560033)
}
}
{
\newrgbcolor{curcolor}{0 0 0}
\pscustom[linestyle=none,fillstyle=solid,fillcolor=curcolor]
{
\newpath
\moveto(496.40625,578.77864943)
\curveto(495.00402868,578.77864943)(493.86599398,579.91668413)(493.86599398,581.31890545)
\curveto(493.86599398,582.72112678)(495.00402868,583.85916147)(496.40625,583.85916147)
\curveto(497.80847132,583.85916147)(498.94650602,582.72112678)(498.94650602,581.31890545)
\curveto(498.94650602,579.91668413)(497.80847132,578.77864943)(496.40625,578.77864943)
\closepath
}
}
{
\newrgbcolor{curcolor}{0 0 0}
\pscustom[linewidth=0.63506401,linecolor=curcolor]
{
\newpath
\moveto(496.40625,578.77864943)
\curveto(495.00402868,578.77864943)(493.86599398,579.91668413)(493.86599398,581.31890545)
\curveto(493.86599398,582.72112678)(495.00402868,583.85916147)(496.40625,583.85916147)
\curveto(497.80847132,583.85916147)(498.94650602,582.72112678)(498.94650602,581.31890545)
\curveto(498.94650602,579.91668413)(497.80847132,578.77864943)(496.40625,578.77864943)
\closepath
}
}
{
\newrgbcolor{curcolor}{0 0 0}
\pscustom[linestyle=none,fillstyle=solid,fillcolor=curcolor]
{
\newpath
\moveto(499.22409804,564.99471816)
\lineto(496.41743639,557.36222361)
\lineto(493.61077536,564.99471858)
\curveto(495.2678269,563.77536505)(497.53496101,563.78239099)(499.22409804,564.99471816)
\closepath
}
}
{
\newrgbcolor{curcolor}{0 0 0}
\pscustom[linestyle=none,fillstyle=solid,fillcolor=curcolor]
{
\newpath
\moveto(496.40625,536.73690033)
\lineto(496.40625,428.86190033)
}
}
{
\newrgbcolor{curcolor}{0 0 0}
\pscustom[linewidth=0.63506401,linecolor=curcolor]
{
\newpath
\moveto(496.40625,536.73690033)
\lineto(496.40625,428.86190033)
}
}
{
\newrgbcolor{curcolor}{0 0 0}
\pscustom[linestyle=none,fillstyle=solid,fillcolor=curcolor]
{
\newpath
\moveto(496.40625,534.24744943)
\curveto(495.00402868,534.24744943)(493.86599398,535.38548413)(493.86599398,536.78770545)
\curveto(493.86599398,538.18992678)(495.00402868,539.32796147)(496.40625,539.32796147)
\curveto(497.80847132,539.32796147)(498.94650602,538.18992678)(498.94650602,536.78770545)
\curveto(498.94650602,535.38548413)(497.80847132,534.24744943)(496.40625,534.24744943)
\closepath
}
}
{
\newrgbcolor{curcolor}{0 0 0}
\pscustom[linewidth=0.63506401,linecolor=curcolor]
{
\newpath
\moveto(496.40625,534.24744943)
\curveto(495.00402868,534.24744943)(493.86599398,535.38548413)(493.86599398,536.78770545)
\curveto(493.86599398,538.18992678)(495.00402868,539.32796147)(496.40625,539.32796147)
\curveto(497.80847132,539.32796147)(498.94650602,538.18992678)(498.94650602,536.78770545)
\curveto(498.94650602,535.38548413)(497.80847132,534.24744943)(496.40625,534.24744943)
\closepath
}
}
{
\newrgbcolor{curcolor}{0 0 0}
\pscustom[linestyle=none,fillstyle=solid,fillcolor=curcolor]
{
\newpath
\moveto(499.22409804,435.65101816)
\lineto(496.41743639,428.01852361)
\lineto(493.61077536,435.65101858)
\curveto(495.2678269,434.43166505)(497.53496101,434.43869099)(499.22409804,435.65101816)
\closepath
}
}
{
\newrgbcolor{curcolor}{0 0 0}
\pscustom[linestyle=none,fillstyle=solid,fillcolor=curcolor]
{
\newpath
\moveto(464.5,536.42440033)
\lineto(401.125,515.23690033)
}
}
{
\newrgbcolor{curcolor}{0 0 0}
\pscustom[linewidth=0.63506401,linecolor=curcolor]
{
\newpath
\moveto(464.5,536.42440033)
\lineto(401.125,515.23690033)
}
}
{
\newrgbcolor{curcolor}{0 0 0}
\pscustom[linestyle=none,fillstyle=solid,fillcolor=curcolor]
{
\newpath
\moveto(462.13899896,535.63507158)
\curveto(461.69439746,536.96494155)(462.41287622,538.40509232)(463.74274619,538.84969382)
\curveto(465.07261617,539.29429532)(466.51276693,538.57581656)(466.95736843,537.24594658)
\curveto(467.40196993,535.91607661)(466.68349117,534.47592584)(465.3536212,534.03132434)
\curveto(464.02375122,533.58672284)(462.58360046,534.3052016)(462.13899896,535.63507158)
\closepath
}
}
{
\newrgbcolor{curcolor}{0 0 0}
\pscustom[linewidth=0.63506401,linecolor=curcolor]
{
\newpath
\moveto(462.13899896,535.63507158)
\curveto(461.69439746,536.96494155)(462.41287622,538.40509232)(463.74274619,538.84969382)
\curveto(465.07261617,539.29429532)(466.51276693,538.57581656)(466.95736843,537.24594658)
\curveto(467.40196993,535.91607661)(466.68349117,534.47592584)(465.3536212,534.03132434)
\curveto(464.02375122,533.58672284)(462.58360046,534.3052016)(462.13899896,535.63507158)
\closepath
}
}
{
\newrgbcolor{curcolor}{0 0 0}
\pscustom[linestyle=none,fillstyle=solid,fillcolor=curcolor]
{
\newpath
\moveto(408.45726862,514.71706832)
\lineto(400.32868642,514.95888217)
\lineto(406.67745606,520.04075678)
\curveto(406.0464187,518.08258551)(406.77192102,515.93465795)(408.45726862,514.71706832)
\closepath
}
}
{
\newrgbcolor{curcolor}{0 0 0}
\pscustom[linestyle=none,fillstyle=solid,fillcolor=curcolor]
{
\newpath
\moveto(368.8125,494.04940033)
\lineto(368.8125,428.86190033)
}
}
{
\newrgbcolor{curcolor}{0 0 0}
\pscustom[linewidth=0.63506401,linecolor=curcolor]
{
\newpath
\moveto(368.8125,494.04940033)
\lineto(368.8125,428.86190033)
}
}
{
\newrgbcolor{curcolor}{0 0 0}
\pscustom[linestyle=none,fillstyle=solid,fillcolor=curcolor]
{
\newpath
\moveto(368.8125,491.55994943)
\curveto(367.41027868,491.55994943)(366.27224398,492.69798413)(366.27224398,494.10020545)
\curveto(366.27224398,495.50242678)(367.41027868,496.64046147)(368.8125,496.64046147)
\curveto(370.21472132,496.64046147)(371.35275602,495.50242678)(371.35275602,494.10020545)
\curveto(371.35275602,492.69798413)(370.21472132,491.55994943)(368.8125,491.55994943)
\closepath
}
}
{
\newrgbcolor{curcolor}{0 0 0}
\pscustom[linewidth=0.63506401,linecolor=curcolor]
{
\newpath
\moveto(368.8125,491.55994943)
\curveto(367.41027868,491.55994943)(366.27224398,492.69798413)(366.27224398,494.10020545)
\curveto(366.27224398,495.50242678)(367.41027868,496.64046147)(368.8125,496.64046147)
\curveto(370.21472132,496.64046147)(371.35275602,495.50242678)(371.35275602,494.10020545)
\curveto(371.35275602,492.69798413)(370.21472132,491.55994943)(368.8125,491.55994943)
\closepath
}
}
{
\newrgbcolor{curcolor}{0 0 0}
\pscustom[linestyle=none,fillstyle=solid,fillcolor=curcolor]
{
\newpath
\moveto(371.63034804,435.65101816)
\lineto(368.82368639,428.01852361)
\lineto(366.01702536,435.65101858)
\curveto(367.6740769,434.43166505)(369.94121101,434.43869099)(371.63034804,435.65101816)
\closepath
}
}
{
\newrgbcolor{curcolor}{0 0 0}
\pscustom[linestyle=none,fillstyle=solid,fillcolor=curcolor]
{
\newpath
\moveto(248.5,625.08060033)
\lineto(248.5,558.20560033)
}
}
{
\newrgbcolor{curcolor}{0 0 0}
\pscustom[linewidth=0.63506401,linecolor=curcolor]
{
\newpath
\moveto(248.5,625.08060033)
\lineto(248.5,558.20560033)
}
}
{
\newrgbcolor{curcolor}{0 0 0}
\pscustom[linestyle=none,fillstyle=solid,fillcolor=curcolor]
{
\newpath
\moveto(248.5,622.59114943)
\curveto(247.09777868,622.59114943)(245.95974398,623.72918413)(245.95974398,625.13140545)
\curveto(245.95974398,626.53362678)(247.09777868,627.67166147)(248.5,627.67166147)
\curveto(249.90222132,627.67166147)(251.04025602,626.53362678)(251.04025602,625.13140545)
\curveto(251.04025602,623.72918413)(249.90222132,622.59114943)(248.5,622.59114943)
\closepath
}
}
{
\newrgbcolor{curcolor}{0 0 0}
\pscustom[linewidth=0.63506401,linecolor=curcolor]
{
\newpath
\moveto(248.5,622.59114943)
\curveto(247.09777868,622.59114943)(245.95974398,623.72918413)(245.95974398,625.13140545)
\curveto(245.95974398,626.53362678)(247.09777868,627.67166147)(248.5,627.67166147)
\curveto(249.90222132,627.67166147)(251.04025602,626.53362678)(251.04025602,625.13140545)
\curveto(251.04025602,623.72918413)(249.90222132,622.59114943)(248.5,622.59114943)
\closepath
}
}
{
\newrgbcolor{curcolor}{0 0 0}
\pscustom[linestyle=none,fillstyle=solid,fillcolor=curcolor]
{
\newpath
\moveto(251.31784804,564.99471816)
\lineto(248.51118639,557.36222361)
\lineto(245.70452536,564.99471858)
\curveto(247.3615769,563.77536505)(249.62871101,563.78239099)(251.31784804,564.99471816)
\closepath
}
}
{
\newrgbcolor{curcolor}{0 0 0}
\pscustom[linestyle=none,fillstyle=solid,fillcolor=curcolor]
{
\newpath
\moveto(248.5,536.73690033)
\lineto(248.5,428.86190033)
}
}
{
\newrgbcolor{curcolor}{0 0 0}
\pscustom[linewidth=0.63506401,linecolor=curcolor]
{
\newpath
\moveto(248.5,536.73690033)
\lineto(248.5,428.86190033)
}
}
{
\newrgbcolor{curcolor}{0 0 0}
\pscustom[linestyle=none,fillstyle=solid,fillcolor=curcolor]
{
\newpath
\moveto(248.5,534.24744943)
\curveto(247.09777868,534.24744943)(245.95974398,535.38548413)(245.95974398,536.78770545)
\curveto(245.95974398,538.18992678)(247.09777868,539.32796147)(248.5,539.32796147)
\curveto(249.90222132,539.32796147)(251.04025602,538.18992678)(251.04025602,536.78770545)
\curveto(251.04025602,535.38548413)(249.90222132,534.24744943)(248.5,534.24744943)
\closepath
}
}
{
\newrgbcolor{curcolor}{0 0 0}
\pscustom[linewidth=0.63506401,linecolor=curcolor]
{
\newpath
\moveto(248.5,534.24744943)
\curveto(247.09777868,534.24744943)(245.95974398,535.38548413)(245.95974398,536.78770545)
\curveto(245.95974398,538.18992678)(247.09777868,539.32796147)(248.5,539.32796147)
\curveto(249.90222132,539.32796147)(251.04025602,538.18992678)(251.04025602,536.78770545)
\curveto(251.04025602,535.38548413)(249.90222132,534.24744943)(248.5,534.24744943)
\closepath
}
}
{
\newrgbcolor{curcolor}{0 0 0}
\pscustom[linestyle=none,fillstyle=solid,fillcolor=curcolor]
{
\newpath
\moveto(251.31784804,435.65101816)
\lineto(248.51118639,428.01852361)
\lineto(245.70452536,435.65101858)
\curveto(247.3615769,434.43166505)(249.62871101,434.43869099)(251.31784804,435.65101816)
\closepath
}
}
{
\newrgbcolor{curcolor}{0 0 0}
\pscustom[linestyle=none,fillstyle=solid,fillcolor=curcolor]
{
\newpath
\moveto(618.90625,843.64315033)
\lineto(618.90625,777.23690033)
}
}
{
\newrgbcolor{curcolor}{0 0 0}
\pscustom[linewidth=0.63506401,linecolor=curcolor]
{
\newpath
\moveto(618.90625,843.64315033)
\lineto(618.90625,777.23690033)
}
}
{
\newrgbcolor{curcolor}{0 0 0}
\pscustom[linestyle=none,fillstyle=solid,fillcolor=curcolor]
{
\newpath
\moveto(618.90625,841.15369943)
\curveto(617.50402868,841.15369943)(616.36599398,842.29173413)(616.36599398,843.69395545)
\curveto(616.36599398,845.09617678)(617.50402868,846.23421147)(618.90625,846.23421147)
\curveto(620.30847132,846.23421147)(621.44650602,845.09617678)(621.44650602,843.69395545)
\curveto(621.44650602,842.29173413)(620.30847132,841.15369943)(618.90625,841.15369943)
\closepath
}
}
{
\newrgbcolor{curcolor}{0 0 0}
\pscustom[linewidth=0.63506401,linecolor=curcolor]
{
\newpath
\moveto(618.90625,841.15369943)
\curveto(617.50402868,841.15369943)(616.36599398,842.29173413)(616.36599398,843.69395545)
\curveto(616.36599398,845.09617678)(617.50402868,846.23421147)(618.90625,846.23421147)
\curveto(620.30847132,846.23421147)(621.44650602,845.09617678)(621.44650602,843.69395545)
\curveto(621.44650602,842.29173413)(620.30847132,841.15369943)(618.90625,841.15369943)
\closepath
}
}
{
\newrgbcolor{curcolor}{0 0 0}
\pscustom[linestyle=none,fillstyle=solid,fillcolor=curcolor]
{
\newpath
\moveto(621.72409804,784.02601816)
\lineto(618.91743639,776.39352361)
\lineto(616.11077536,784.02601858)
\curveto(617.7678269,782.80666505)(620.03496101,782.81369099)(621.72409804,784.02601816)
\closepath
}
}
{
\newrgbcolor{curcolor}{0 0 0}
\pscustom[linestyle=none,fillstyle=solid,fillcolor=curcolor]
{
\newpath
\moveto(618.90625,755.76815033)
\lineto(618.90625,602.76810033)
}
}
{
\newrgbcolor{curcolor}{0 0 0}
\pscustom[linewidth=0.63506401,linecolor=curcolor]
{
\newpath
\moveto(618.90625,755.76815033)
\lineto(618.90625,602.76810033)
}
}
{
\newrgbcolor{curcolor}{0 0 0}
\pscustom[linestyle=none,fillstyle=solid,fillcolor=curcolor]
{
\newpath
\moveto(618.90625,753.27869943)
\curveto(617.50402868,753.27869943)(616.36599398,754.41673413)(616.36599398,755.81895545)
\curveto(616.36599398,757.22117678)(617.50402868,758.35921147)(618.90625,758.35921147)
\curveto(620.30847132,758.35921147)(621.44650602,757.22117678)(621.44650602,755.81895545)
\curveto(621.44650602,754.41673413)(620.30847132,753.27869943)(618.90625,753.27869943)
\closepath
}
}
{
\newrgbcolor{curcolor}{0 0 0}
\pscustom[linewidth=0.63506401,linecolor=curcolor]
{
\newpath
\moveto(618.90625,753.27869943)
\curveto(617.50402868,753.27869943)(616.36599398,754.41673413)(616.36599398,755.81895545)
\curveto(616.36599398,757.22117678)(617.50402868,758.35921147)(618.90625,758.35921147)
\curveto(620.30847132,758.35921147)(621.44650602,757.22117678)(621.44650602,755.81895545)
\curveto(621.44650602,754.41673413)(620.30847132,753.27869943)(618.90625,753.27869943)
\closepath
}
}
{
\newrgbcolor{curcolor}{0 0 0}
\pscustom[linestyle=none,fillstyle=solid,fillcolor=curcolor]
{
\newpath
\moveto(621.72409804,609.55721816)
\lineto(618.91743639,601.92472361)
\lineto(616.11077536,609.55721858)
\curveto(617.7678269,608.33786505)(620.03496101,608.34489099)(621.72409804,609.55721816)
\closepath
}
}
{
\newrgbcolor{curcolor}{0 0 0}
\pscustom[linestyle=none,fillstyle=solid,fillcolor=curcolor]
{
\newpath
\moveto(618.90625,581.26810033)
\lineto(618.90625,558.20560033)
}
}
{
\newrgbcolor{curcolor}{0 0 0}
\pscustom[linewidth=0.63506401,linecolor=curcolor]
{
\newpath
\moveto(618.90625,581.26810033)
\lineto(618.90625,558.20560033)
}
}
{
\newrgbcolor{curcolor}{0 0 0}
\pscustom[linestyle=none,fillstyle=solid,fillcolor=curcolor]
{
\newpath
\moveto(618.90625,578.77864943)
\curveto(617.50402868,578.77864943)(616.36599398,579.91668413)(616.36599398,581.31890545)
\curveto(616.36599398,582.72112678)(617.50402868,583.85916147)(618.90625,583.85916147)
\curveto(620.30847132,583.85916147)(621.44650602,582.72112678)(621.44650602,581.31890545)
\curveto(621.44650602,579.91668413)(620.30847132,578.77864943)(618.90625,578.77864943)
\closepath
}
}
{
\newrgbcolor{curcolor}{0 0 0}
\pscustom[linewidth=0.63506401,linecolor=curcolor]
{
\newpath
\moveto(618.90625,578.77864943)
\curveto(617.50402868,578.77864943)(616.36599398,579.91668413)(616.36599398,581.31890545)
\curveto(616.36599398,582.72112678)(617.50402868,583.85916147)(618.90625,583.85916147)
\curveto(620.30847132,583.85916147)(621.44650602,582.72112678)(621.44650602,581.31890545)
\curveto(621.44650602,579.91668413)(620.30847132,578.77864943)(618.90625,578.77864943)
\closepath
}
}
{
\newrgbcolor{curcolor}{0 0 0}
\pscustom[linestyle=none,fillstyle=solid,fillcolor=curcolor]
{
\newpath
\moveto(621.72409804,564.99471816)
\lineto(618.91743639,557.36222361)
\lineto(616.11077536,564.99471858)
\curveto(617.7678269,563.77536505)(620.03496101,563.78239099)(621.72409804,564.99471816)
\closepath
}
}
{
\newrgbcolor{curcolor}{0 0 0}
\pscustom[linestyle=none,fillstyle=solid,fillcolor=curcolor]
{
\newpath
\moveto(608.8125,536.95560033)
\lineto(507.125,428.64310033)
}
}
{
\newrgbcolor{curcolor}{0 0 0}
\pscustom[linewidth=0.63506401,linecolor=curcolor]
{
\newpath
\moveto(608.8125,536.95560033)
\lineto(507.125,428.64310033)
}
}
{
\newrgbcolor{curcolor}{0 0 0}
\pscustom[linestyle=none,fillstyle=solid,fillcolor=curcolor]
{
\newpath
\moveto(607.10857361,535.14066214)
\curveto(606.08628189,536.10042476)(606.03553364,537.70904857)(606.99529626,538.73134029)
\curveto(607.95505888,539.753632)(609.56368269,539.80438025)(610.58597441,538.84461763)
\curveto(611.60826613,537.88485501)(611.65901438,536.2762312)(610.69925176,535.25393948)
\curveto(609.73948913,534.23164777)(608.13086532,534.18089952)(607.10857361,535.14066214)
\closepath
}
}
{
\newrgbcolor{curcolor}{0 0 0}
\pscustom[linewidth=0.63506401,linecolor=curcolor]
{
\newpath
\moveto(607.10857361,535.14066214)
\curveto(606.08628189,536.10042476)(606.03553364,537.70904857)(606.99529626,538.73134029)
\curveto(607.95505888,539.753632)(609.56368269,539.80438025)(610.58597441,538.84461763)
\curveto(611.60826613,537.88485501)(611.65901438,536.2762312)(610.69925176,535.25393948)
\curveto(609.73948913,534.23164777)(608.13086532,534.18089952)(607.10857361,535.14066214)
\closepath
}
}
{
\newrgbcolor{curcolor}{0 0 0}
\pscustom[linestyle=none,fillstyle=solid,fillcolor=curcolor]
{
\newpath
\moveto(513.82622759,431.66401695)
\lineto(506.5558989,428.02057856)
\lineto(509.73382593,435.50610494)
\curveto(510.10730487,433.4829499)(511.76497162,431.93631245)(513.82622759,431.66401695)
\closepath
}
}
{
\newrgbcolor{curcolor}{0 0 0}
\pscustom[linestyle=none,fillstyle=solid,fillcolor=curcolor]
{
\newpath
\moveto(598.3125,537.01810033)
\lineto(389.75,429.14310033)
}
}
{
\newrgbcolor{curcolor}{0 0 0}
\pscustom[linewidth=0.63506401,linecolor=curcolor]
{
\newpath
\moveto(598.3125,537.01810033)
\lineto(389.75,429.14310033)
}
}
{
\newrgbcolor{curcolor}{0 0 0}
\pscustom[linestyle=none,fillstyle=solid,fillcolor=curcolor]
{
\newpath
\moveto(596.10131646,535.87440756)
\curveto(595.457114,537.11989053)(595.94510978,538.65354827)(597.19059275,539.29775073)
\curveto(598.43607572,539.94195319)(599.96973347,539.45395741)(600.61393593,538.20847444)
\curveto(601.25813839,536.96299147)(600.77014261,535.42933373)(599.52465964,534.78513127)
\curveto(598.27917667,534.14092881)(596.74551892,534.62892458)(596.10131646,535.87440756)
\closepath
}
}
{
\newrgbcolor{curcolor}{0 0 0}
\pscustom[linewidth=0.63506401,linecolor=curcolor]
{
\newpath
\moveto(596.10131646,535.87440756)
\curveto(595.457114,537.11989053)(595.94510978,538.65354827)(597.19059275,539.29775073)
\curveto(598.43607572,539.94195319)(599.96973347,539.45395741)(600.61393593,538.20847444)
\curveto(601.25813839,536.96299147)(600.77014261,535.42933373)(599.52465964,534.78513127)
\curveto(598.27917667,534.14092881)(596.74551892,534.62892458)(596.10131646,535.87440756)
\closepath
}
}
{
\newrgbcolor{curcolor}{0 0 0}
\pscustom[linestyle=none,fillstyle=solid,fillcolor=curcolor]
{
\newpath
\moveto(397.07480323,429.7592546)
\lineto(389.00603396,428.74570386)
\lineto(394.49595516,434.74512806)
\curveto(394.17417474,432.71310931)(395.22197228,430.70262013)(397.07480323,429.7592546)
\closepath
}
}
{
\newrgbcolor{curcolor}{0 0 0}
\pscustom[linestyle=none,fillstyle=solid,fillcolor=curcolor]
{
\newpath
\moveto(714.59375,888.73690033)
\lineto(714.59375,602.76810033)
}
}
{
\newrgbcolor{curcolor}{0 0 0}
\pscustom[linewidth=0.63506401,linecolor=curcolor]
{
\newpath
\moveto(714.59375,888.73690033)
\lineto(714.59375,602.76810033)
}
}
{
\newrgbcolor{curcolor}{0 0 0}
\pscustom[linestyle=none,fillstyle=solid,fillcolor=curcolor]
{
\newpath
\moveto(714.59375,886.24744943)
\curveto(713.19152868,886.24744943)(712.05349398,887.38548413)(712.05349398,888.78770545)
\curveto(712.05349398,890.18992678)(713.19152868,891.32796147)(714.59375,891.32796147)
\curveto(715.99597132,891.32796147)(717.13400602,890.18992678)(717.13400602,888.78770545)
\curveto(717.13400602,887.38548413)(715.99597132,886.24744943)(714.59375,886.24744943)
\closepath
}
}
{
\newrgbcolor{curcolor}{0 0 0}
\pscustom[linewidth=0.63506401,linecolor=curcolor]
{
\newpath
\moveto(714.59375,886.24744943)
\curveto(713.19152868,886.24744943)(712.05349398,887.38548413)(712.05349398,888.78770545)
\curveto(712.05349398,890.18992678)(713.19152868,891.32796147)(714.59375,891.32796147)
\curveto(715.99597132,891.32796147)(717.13400602,890.18992678)(717.13400602,888.78770545)
\curveto(717.13400602,887.38548413)(715.99597132,886.24744943)(714.59375,886.24744943)
\closepath
}
}
{
\newrgbcolor{curcolor}{0 0 0}
\pscustom[linestyle=none,fillstyle=solid,fillcolor=curcolor]
{
\newpath
\moveto(717.41159804,609.55721816)
\lineto(714.60493639,601.92472361)
\lineto(711.79827536,609.55721858)
\curveto(713.4553269,608.33786505)(715.72246101,608.34489099)(717.41159804,609.55721816)
\closepath
}
}
{
\newrgbcolor{curcolor}{0 0 0}
\pscustom[linestyle=none,fillstyle=solid,fillcolor=curcolor]
{
\newpath
\moveto(658.28125,536.42440033)
\lineto(735.03125,515.23690033)
}
}
{
\newrgbcolor{curcolor}{0 0 0}
\pscustom[linewidth=0.63506401,linecolor=curcolor]
{
\newpath
\moveto(658.28125,536.42440033)
\lineto(735.03125,515.23690033)
}
}
{
\newrgbcolor{curcolor}{0 0 0}
\pscustom[linestyle=none,fillstyle=solid,fillcolor=curcolor]
{
\newpath
\moveto(660.68094133,535.76194482)
\curveto(660.30780313,534.41028195)(658.90796457,533.616117)(657.5563017,533.9892552)
\curveto(656.20463883,534.36239341)(655.41047388,535.76223196)(655.78361208,537.11389484)
\curveto(656.15675028,538.46555771)(657.55658884,539.25972266)(658.90825171,538.88658446)
\curveto(660.25991459,538.51344625)(661.05407954,537.1136077)(660.68094133,535.76194482)
\closepath
}
}
{
\newrgbcolor{curcolor}{0 0 0}
\pscustom[linewidth=0.63506401,linecolor=curcolor]
{
\newpath
\moveto(660.68094133,535.76194482)
\curveto(660.30780313,534.41028195)(658.90796457,533.616117)(657.5563017,533.9892552)
\curveto(656.20463883,534.36239341)(655.41047388,535.76223196)(655.78361208,537.11389484)
\curveto(656.15675028,538.46555771)(657.55658884,539.25972266)(658.90825171,538.88658446)
\curveto(660.25991459,538.51344625)(661.05407954,537.1136077)(660.68094133,535.76194482)
\closepath
}
}
{
\newrgbcolor{curcolor}{0 0 0}
\pscustom[linestyle=none,fillstyle=solid,fillcolor=curcolor]
{
\newpath
\moveto(729.23676405,519.7597668)
\lineto(735.8471947,515.02325656)
\lineto(727.74303002,514.34883802)
\curveto(729.35936839,515.62166681)(729.95589166,517.80892684)(729.23676405,519.7597668)
\closepath
}
}
{
\newrgbcolor{curcolor}{0 0 0}
\pscustom[linestyle=none,fillstyle=solid,fillcolor=curcolor]
{
\newpath
\moveto(773.75,494.04940033)
\lineto(773.75,340.39310033)
}
}
{
\newrgbcolor{curcolor}{0 0 0}
\pscustom[linewidth=0.63506401,linecolor=curcolor]
{
\newpath
\moveto(773.75,494.04940033)
\lineto(773.75,340.39310033)
}
}
{
\newrgbcolor{curcolor}{0 0 0}
\pscustom[linestyle=none,fillstyle=solid,fillcolor=curcolor]
{
\newpath
\moveto(773.75,491.55994943)
\curveto(772.34777868,491.55994943)(771.20974398,492.69798413)(771.20974398,494.10020545)
\curveto(771.20974398,495.50242678)(772.34777868,496.64046147)(773.75,496.64046147)
\curveto(775.15222132,496.64046147)(776.29025602,495.50242678)(776.29025602,494.10020545)
\curveto(776.29025602,492.69798413)(775.15222132,491.55994943)(773.75,491.55994943)
\closepath
}
}
{
\newrgbcolor{curcolor}{0 0 0}
\pscustom[linewidth=0.63506401,linecolor=curcolor]
{
\newpath
\moveto(773.75,491.55994943)
\curveto(772.34777868,491.55994943)(771.20974398,492.69798413)(771.20974398,494.10020545)
\curveto(771.20974398,495.50242678)(772.34777868,496.64046147)(773.75,496.64046147)
\curveto(775.15222132,496.64046147)(776.29025602,495.50242678)(776.29025602,494.10020545)
\curveto(776.29025602,492.69798413)(775.15222132,491.55994943)(773.75,491.55994943)
\closepath
}
}
{
\newrgbcolor{curcolor}{0 0 0}
\pscustom[linestyle=none,fillstyle=solid,fillcolor=curcolor]
{
\newpath
\moveto(776.56784804,347.18221816)
\lineto(773.76118639,339.54972361)
\lineto(770.95452536,347.18221858)
\curveto(772.6115769,345.96286505)(774.87871101,345.96989099)(776.56784804,347.18221816)
\closepath
}
}
{
\newrgbcolor{curcolor}{0 0 0}
\pscustom[linestyle=none,fillstyle=solid,fillcolor=curcolor]
{
\newpath
\moveto(619.8125,843.61190033)
\lineto(677.71875,165.04940033)
}
}
{
\newrgbcolor{curcolor}{0 0 0}
\pscustom[linewidth=0.63506401,linecolor=curcolor]
{
\newpath
\moveto(619.8125,843.61190033)
\lineto(677.71875,165.04940033)
}
}
{
\newrgbcolor{curcolor}{0 0 0}
\pscustom[linestyle=none,fillstyle=solid,fillcolor=curcolor]
{
\newpath
\moveto(620.02417206,841.13146474)
\curveto(618.62702875,841.01223721)(617.39635096,842.04938625)(617.27712343,843.44652956)
\curveto(617.1578959,844.84367288)(618.19504495,846.07435066)(619.59218826,846.19357819)
\curveto(620.98933158,846.31280572)(622.22000936,845.27565668)(622.33923689,843.87851336)
\curveto(622.45846442,842.48137005)(621.42131538,841.25069226)(620.02417206,841.13146474)
\closepath
}
}
{
\newrgbcolor{curcolor}{0 0 0}
\pscustom[linewidth=0.63506401,linecolor=curcolor]
{
\newpath
\moveto(620.02417206,841.13146474)
\curveto(618.62702875,841.01223721)(617.39635096,842.04938625)(617.27712343,843.44652956)
\curveto(617.1578959,844.84367288)(618.19504495,846.07435066)(619.59218826,846.19357819)
\curveto(620.98933158,846.31280572)(622.22000936,845.27565668)(622.33923689,843.87851336)
\curveto(622.45846442,842.48137005)(621.42131538,841.25069226)(620.02417206,841.13146474)
\closepath
}
}
{
\newrgbcolor{curcolor}{0 0 0}
\pscustom[linestyle=none,fillstyle=solid,fillcolor=curcolor]
{
\newpath
\moveto(679.949131,172.0535269)
\lineto(677.80160618,164.21002897)
\lineto(674.3561364,171.57623991)
\curveto(676.11086581,170.50219728)(678.36919231,170.70196678)(679.949131,172.0535269)
\closepath
}
}
{
\newrgbcolor{curcolor}{0 0 0}
\pscustom[linestyle=none,fillstyle=solid,fillcolor=curcolor]
{
\newpath
\moveto(957.21875,665.26810033)
\lineto(957.21875,78.26810033)
}
}
{
\newrgbcolor{curcolor}{0 0 0}
\pscustom[linewidth=0.63506401,linecolor=curcolor]
{
\newpath
\moveto(957.21875,665.26810033)
\lineto(957.21875,78.26810033)
}
}
{
\newrgbcolor{curcolor}{0 0 0}
\pscustom[linestyle=none,fillstyle=solid,fillcolor=curcolor]
{
\newpath
\moveto(957.21875,662.77864943)
\curveto(955.81652868,662.77864943)(954.67849398,663.91668413)(954.67849398,665.31890545)
\curveto(954.67849398,666.72112678)(955.81652868,667.85916147)(957.21875,667.85916147)
\curveto(958.62097132,667.85916147)(959.75900602,666.72112678)(959.75900602,665.31890545)
\curveto(959.75900602,663.91668413)(958.62097132,662.77864943)(957.21875,662.77864943)
\closepath
}
}
{
\newrgbcolor{curcolor}{0 0 0}
\pscustom[linewidth=0.63506401,linecolor=curcolor]
{
\newpath
\moveto(957.21875,662.77864943)
\curveto(955.81652868,662.77864943)(954.67849398,663.91668413)(954.67849398,665.31890545)
\curveto(954.67849398,666.72112678)(955.81652868,667.85916147)(957.21875,667.85916147)
\curveto(958.62097132,667.85916147)(959.75900602,666.72112678)(959.75900602,665.31890545)
\curveto(959.75900602,663.91668413)(958.62097132,662.77864943)(957.21875,662.77864943)
\closepath
}
}
{
\newrgbcolor{curcolor}{0 0 0}
\pscustom[linestyle=none,fillstyle=solid,fillcolor=curcolor]
{
\newpath
\moveto(960.03659804,85.05721816)
\lineto(957.22993639,77.42472361)
\lineto(954.42327536,85.05721858)
\curveto(956.0803269,83.83786505)(958.34746101,83.84489099)(960.03659804,85.05721816)
\closepath
}
}
{
\newrgbcolor{curcolor}{0 0 0}
\pscustom[linestyle=none,fillstyle=solid,fillcolor=curcolor]
{
\newpath
\moveto(101.09375,670.01810033)
\lineto(102.5625,165.08060033)
}
}
{
\newrgbcolor{curcolor}{0 0 0}
\pscustom[linewidth=0.63506401,linecolor=curcolor]
{
\newpath
\moveto(101.09375,670.01810033)
\lineto(102.5625,165.08060033)
}
}
{
\newrgbcolor{curcolor}{0 0 0}
\pscustom[linestyle=none,fillstyle=solid,fillcolor=curcolor]
{
\newpath
\moveto(101.10099122,667.52865996)
\curveto(99.69877583,667.52458123)(98.55743567,668.65930084)(98.55335694,670.06151623)
\curveto(98.54927821,671.46373162)(99.68399782,672.60507178)(101.08621322,672.60915051)
\curveto(102.48842861,672.61322924)(103.62976877,671.47850963)(103.6338475,670.07629424)
\curveto(103.63792623,668.67407885)(102.50320662,667.53273869)(101.10099122,667.52865996)
\closepath
}
}
{
\newrgbcolor{curcolor}{0 0 0}
\pscustom[linewidth=0.63506401,linecolor=curcolor]
{
\newpath
\moveto(101.10099122,667.52865996)
\curveto(99.69877583,667.52458123)(98.55743567,668.65930084)(98.55335694,670.06151623)
\curveto(98.54927821,671.46373162)(99.68399782,672.60507178)(101.08621322,672.60915051)
\curveto(102.48842861,672.61322924)(103.62976877,671.47850963)(103.6338475,670.07629424)
\curveto(103.63792623,668.67407885)(102.50320662,667.53273869)(101.10099122,667.52865996)
\closepath
}
}
{
\newrgbcolor{curcolor}{0 0 0}
\pscustom[linestyle=none,fillstyle=solid,fillcolor=curcolor]
{
\newpath
\moveto(105.36058818,171.87788589)
\lineto(102.57613952,164.23725972)
\lineto(99.74728924,171.86155848)
\curveto(101.40788059,170.64703009)(103.67498467,170.66065055)(105.36058818,171.87788589)
\closepath
}
}
{
\newrgbcolor{curcolor}{0 0 0}
\pscustom[linestyle=none,fillstyle=solid,fillcolor=curcolor]
{
\newpath
\moveto(168.5625,713.79940033)
\lineto(168.5625,120.79940033)
}
}
{
\newrgbcolor{curcolor}{0 0 0}
\pscustom[linewidth=0.63506401,linecolor=curcolor]
{
\newpath
\moveto(168.5625,713.79940033)
\lineto(168.5625,120.79940033)
}
}
{
\newrgbcolor{curcolor}{0 0 0}
\pscustom[linestyle=none,fillstyle=solid,fillcolor=curcolor]
{
\newpath
\moveto(168.5625,711.30994943)
\curveto(167.16027868,711.30994943)(166.02224398,712.44798413)(166.02224398,713.85020545)
\curveto(166.02224398,715.25242678)(167.16027868,716.39046147)(168.5625,716.39046147)
\curveto(169.96472132,716.39046147)(171.10275602,715.25242678)(171.10275602,713.85020545)
\curveto(171.10275602,712.44798413)(169.96472132,711.30994943)(168.5625,711.30994943)
\closepath
}
}
{
\newrgbcolor{curcolor}{0 0 0}
\pscustom[linewidth=0.63506401,linecolor=curcolor]
{
\newpath
\moveto(168.5625,711.30994943)
\curveto(167.16027868,711.30994943)(166.02224398,712.44798413)(166.02224398,713.85020545)
\curveto(166.02224398,715.25242678)(167.16027868,716.39046147)(168.5625,716.39046147)
\curveto(169.96472132,716.39046147)(171.10275602,715.25242678)(171.10275602,713.85020545)
\curveto(171.10275602,712.44798413)(169.96472132,711.30994943)(168.5625,711.30994943)
\closepath
}
}
{
\newrgbcolor{curcolor}{0 0 0}
\pscustom[linestyle=none,fillstyle=solid,fillcolor=curcolor]
{
\newpath
\moveto(171.38034804,127.58851816)
\lineto(168.57368639,119.95602361)
\lineto(165.76702536,127.58851858)
\curveto(167.4240769,126.36916505)(169.69121101,126.37619099)(171.38034804,127.58851816)
\closepath
}
}
{
\newrgbcolor{curcolor}{0 0 0}
\pscustom[linestyle=none,fillstyle=solid,fillcolor=curcolor]
{
\newpath
\moveto(248.5,407.39310033)
\lineto(248.5,78.26810033)
}
}
{
\newrgbcolor{curcolor}{0 0 0}
\pscustom[linewidth=0.63506401,linecolor=curcolor]
{
\newpath
\moveto(248.5,407.39310033)
\lineto(248.5,78.26810033)
}
}
{
\newrgbcolor{curcolor}{0 0 0}
\pscustom[linestyle=none,fillstyle=solid,fillcolor=curcolor]
{
\newpath
\moveto(248.5,404.90364943)
\curveto(247.09777868,404.90364943)(245.95974398,406.04168413)(245.95974398,407.44390545)
\curveto(245.95974398,408.84612678)(247.09777868,409.98416147)(248.5,409.98416147)
\curveto(249.90222132,409.98416147)(251.04025602,408.84612678)(251.04025602,407.44390545)
\curveto(251.04025602,406.04168413)(249.90222132,404.90364943)(248.5,404.90364943)
\closepath
}
}
{
\newrgbcolor{curcolor}{0 0 0}
\pscustom[linewidth=0.63506401,linecolor=curcolor]
{
\newpath
\moveto(248.5,404.90364943)
\curveto(247.09777868,404.90364943)(245.95974398,406.04168413)(245.95974398,407.44390545)
\curveto(245.95974398,408.84612678)(247.09777868,409.98416147)(248.5,409.98416147)
\curveto(249.90222132,409.98416147)(251.04025602,408.84612678)(251.04025602,407.44390545)
\curveto(251.04025602,406.04168413)(249.90222132,404.90364943)(248.5,404.90364943)
\closepath
}
}
{
\newrgbcolor{curcolor}{0 0 0}
\pscustom[linestyle=none,fillstyle=solid,fillcolor=curcolor]
{
\newpath
\moveto(251.31784804,85.05721816)
\lineto(248.51118639,77.42472361)
\lineto(245.70452536,85.05721858)
\curveto(247.3615769,83.83786505)(249.62871101,83.84489099)(251.31784804,85.05721816)
\closepath
}
}
{
\newrgbcolor{curcolor}{0 0 0}
\pscustom[linestyle=none,fillstyle=solid,fillcolor=curcolor]
{
\newpath
\moveto(496.5,407.36190033)
\lineto(500.09375,120.79940033)
}
}
{
\newrgbcolor{curcolor}{0 0 0}
\pscustom[linewidth=0.63506401,linecolor=curcolor]
{
\newpath
\moveto(496.5,407.36190033)
\lineto(500.09375,120.79940033)
}
}
{
\newrgbcolor{curcolor}{0 0 0}
\pscustom[linestyle=none,fillstyle=solid,fillcolor=curcolor]
{
\newpath
\moveto(496.53121749,404.87264517)
\curveto(495.12910641,404.85506144)(493.97689035,405.97873581)(493.95930662,407.38084688)
\curveto(493.94172289,408.78295795)(495.06539726,409.93517402)(496.46750833,409.95275774)
\curveto(497.8696194,409.97034147)(499.02183547,408.8466671)(499.0394192,407.44455603)
\curveto(499.05700292,406.04244496)(497.93332856,404.89022889)(496.53121749,404.87264517)
\closepath
}
}
{
\newrgbcolor{curcolor}{0 0 0}
\pscustom[linewidth=0.63506401,linecolor=curcolor]
{
\newpath
\moveto(496.53121749,404.87264517)
\curveto(495.12910641,404.85506144)(493.97689035,405.97873581)(493.95930662,407.38084688)
\curveto(493.94172289,408.78295795)(495.06539726,409.93517402)(496.46750833,409.95275774)
\curveto(497.8696194,409.97034147)(499.02183547,408.8466671)(499.0394192,407.44455603)
\curveto(499.05700292,406.04244496)(497.93332856,404.89022889)(496.53121749,404.87264517)
\closepath
}
}
{
\newrgbcolor{curcolor}{0 0 0}
\pscustom[linestyle=none,fillstyle=solid,fillcolor=curcolor]
{
\newpath
\moveto(502.82624156,127.6233199)
\lineto(500.11551137,119.9562302)
\lineto(497.21336024,127.55292977)
\curveto(498.88557208,126.35445139)(501.15243982,126.38990644)(502.82624156,127.6233199)
\closepath
}
}
{
\newrgbcolor{curcolor}{0 0 0}
\pscustom[linestyle=none,fillstyle=solid,fillcolor=curcolor]
{
\newpath
\moveto(416.875,1250.70565033)
\lineto(448.84375,1250.70565033)
}
}
{
\newrgbcolor{curcolor}{0 0 0}
\pscustom[linewidth=0.63506401,linecolor=curcolor]
{
\newpath
\moveto(416.875,1250.70565033)
\lineto(448.84375,1250.70565033)
}
}
{
\newrgbcolor{curcolor}{0 0 0}
\pscustom[linestyle=none,fillstyle=solid,fillcolor=curcolor]
{
\newpath
\moveto(419.3644509,1250.70565033)
\curveto(419.3644509,1249.303429)(418.2264162,1248.16539431)(416.82419488,1248.16539431)
\curveto(415.42197355,1248.16539431)(414.28393886,1249.303429)(414.28393886,1250.70565033)
\curveto(414.28393886,1252.10787165)(415.42197355,1253.24590635)(416.82419488,1253.24590635)
\curveto(418.2264162,1253.24590635)(419.3644509,1252.10787165)(419.3644509,1250.70565033)
\closepath
}
}
{
\newrgbcolor{curcolor}{0 0 0}
\pscustom[linewidth=0.63506401,linecolor=curcolor]
{
\newpath
\moveto(419.3644509,1250.70565033)
\curveto(419.3644509,1249.303429)(418.2264162,1248.16539431)(416.82419488,1248.16539431)
\curveto(415.42197355,1248.16539431)(414.28393886,1249.303429)(414.28393886,1250.70565033)
\curveto(414.28393886,1252.10787165)(415.42197355,1253.24590635)(416.82419488,1253.24590635)
\curveto(418.2264162,1253.24590635)(419.3644509,1252.10787165)(419.3644509,1250.70565033)
\closepath
}
}
{
\newrgbcolor{curcolor}{0 0 0}
\pscustom[linestyle=none,fillstyle=solid,fillcolor=curcolor]
{
\newpath
\moveto(442.05463217,1253.52349837)
\lineto(449.68712672,1250.71683672)
\lineto(442.05463175,1247.91017569)
\curveto(443.27398528,1249.56722723)(443.26695934,1251.83436134)(442.05463217,1253.52349837)
\closepath
}
}
{
\newrgbcolor{curcolor}{0 0 0}
\pscustom[linestyle=none,fillstyle=solid,fillcolor=curcolor]
{
\newpath
\moveto(643.4375,1243.76815033)
\lineto(741.125,1215.61190033)
}
}
{
\newrgbcolor{curcolor}{0 0 0}
\pscustom[linewidth=0.63506401,linecolor=curcolor]
{
\newpath
\moveto(643.4375,1243.76815033)
\lineto(741.125,1215.61190033)
}
}
{
\newrgbcolor{curcolor}{0 0 0}
\pscustom[linestyle=none,fillstyle=solid,fillcolor=curcolor]
{
\newpath
\moveto(645.82957202,1243.07868875)
\curveto(645.44122223,1241.73131758)(644.03252116,1240.95298138)(642.68514998,1241.34133116)
\curveto(641.33777881,1241.72968095)(640.55944261,1243.13838202)(640.94779239,1244.48575319)
\curveto(641.33614218,1245.83312437)(642.74484325,1246.61146057)(644.09221442,1246.22311079)
\curveto(645.4395856,1245.834761)(646.2179218,1244.42605993)(645.82957202,1243.07868875)
\closepath
}
}
{
\newrgbcolor{curcolor}{0 0 0}
\pscustom[linewidth=0.63506401,linecolor=curcolor]
{
\newpath
\moveto(645.82957202,1243.07868875)
\curveto(645.44122223,1241.73131758)(644.03252116,1240.95298138)(642.68514998,1241.34133116)
\curveto(641.33777881,1241.72968095)(640.55944261,1243.13838202)(640.94779239,1244.48575319)
\curveto(641.33614218,1245.83312437)(642.74484325,1246.61146057)(644.09221442,1246.22311079)
\curveto(645.4395856,1245.834761)(646.2179218,1244.42605993)(645.82957202,1243.07868875)
\closepath
}
}
{
\newrgbcolor{curcolor}{0 0 0}
\pscustom[linestyle=none,fillstyle=solid,fillcolor=curcolor]
{
\newpath
\moveto(735.3818617,1220.1997921)
\lineto(741.93848478,1215.3890732)
\lineto(733.82723321,1214.80604371)
\curveto(735.4578156,1216.06057306)(736.07895471,1218.24097042)(735.3818617,1220.1997921)
\closepath
}
}
{
\newrgbcolor{curcolor}{0 0 0}
\pscustom[linestyle=none,fillstyle=solid,fillcolor=curcolor]
{
\newpath
\moveto(131.40625,846.54940033)
\lineto(103.03125,691.54940033)
}
}
{
\newrgbcolor{curcolor}{0 0 0}
\pscustom[linewidth=0.63506401,linecolor=curcolor]
{
\newpath
\moveto(131.40625,846.54940033)
\lineto(103.03125,691.54940033)
}
}
{
\newrgbcolor{curcolor}{0 0 0}
\pscustom[linestyle=none,fillstyle=solid,fillcolor=curcolor]
{
\newpath
\moveto(130.9579695,844.10064342)
\curveto(129.57866969,844.35314427)(128.66416619,845.67750423)(128.91666704,847.05680404)
\curveto(129.1691679,848.43610385)(130.49352785,849.35060735)(131.87282766,849.0981065)
\curveto(133.25212747,848.84560565)(134.16663097,847.52124569)(133.91413012,846.14194588)
\curveto(133.66162927,844.76264607)(132.33726931,843.84814257)(130.9579695,844.10064342)
\closepath
}
}
{
\newrgbcolor{curcolor}{0 0 0}
\pscustom[linewidth=0.63506401,linecolor=curcolor]
{
\newpath
\moveto(130.9579695,844.10064342)
\curveto(129.57866969,844.35314427)(128.66416619,845.67750423)(128.91666704,847.05680404)
\curveto(129.1691679,848.43610385)(130.49352785,849.35060735)(131.87282766,849.0981065)
\curveto(133.25212747,848.84560565)(134.16663097,847.52124569)(133.91413012,846.14194588)
\curveto(133.66162927,844.76264607)(132.33726931,843.84814257)(130.9579695,844.10064342)
\closepath
}
}
{
\newrgbcolor{curcolor}{0 0 0}
\pscustom[linestyle=none,fillstyle=solid,fillcolor=curcolor]
{
\newpath
\moveto(107.02556616,697.7201237)
\lineto(102.89038496,690.71779558)
\lineto(101.50400215,698.73092657)
\curveto(102.91439511,697.23311664)(105.14573452,696.83178028)(107.02556616,697.7201237)
\closepath
}
}
{
\newrgbcolor{curcolor}{0 0 0}
\pscustom[linestyle=none,fillstyle=solid,fillcolor=curcolor]
{
\newpath
\moveto(368.8125,407.39310033)
\lineto(368.8125,78.26810033)
}
}
{
\newrgbcolor{curcolor}{0 0 0}
\pscustom[linewidth=0.63506401,linecolor=curcolor]
{
\newpath
\moveto(368.8125,407.39310033)
\lineto(368.8125,78.26810033)
}
}
{
\newrgbcolor{curcolor}{0 0 0}
\pscustom[linestyle=none,fillstyle=solid,fillcolor=curcolor]
{
\newpath
\moveto(368.8125,404.90364943)
\curveto(367.41027868,404.90364943)(366.27224398,406.04168413)(366.27224398,407.44390545)
\curveto(366.27224398,408.84612678)(367.41027868,409.98416147)(368.8125,409.98416147)
\curveto(370.21472132,409.98416147)(371.35275602,408.84612678)(371.35275602,407.44390545)
\curveto(371.35275602,406.04168413)(370.21472132,404.90364943)(368.8125,404.90364943)
\closepath
}
}
{
\newrgbcolor{curcolor}{0 0 0}
\pscustom[linewidth=0.63506401,linecolor=curcolor]
{
\newpath
\moveto(368.8125,404.90364943)
\curveto(367.41027868,404.90364943)(366.27224398,406.04168413)(366.27224398,407.44390545)
\curveto(366.27224398,408.84612678)(367.41027868,409.98416147)(368.8125,409.98416147)
\curveto(370.21472132,409.98416147)(371.35275602,408.84612678)(371.35275602,407.44390545)
\curveto(371.35275602,406.04168413)(370.21472132,404.90364943)(368.8125,404.90364943)
\closepath
}
}
{
\newrgbcolor{curcolor}{0 0 0}
\pscustom[linestyle=none,fillstyle=solid,fillcolor=curcolor]
{
\newpath
\moveto(371.63034804,85.05721816)
\lineto(368.82368639,77.42472361)
\lineto(366.01702536,85.05721858)
\curveto(367.6740769,83.83786505)(369.94121101,83.84489099)(371.63034804,85.05721816)
\closepath
}
}
{
\newrgbcolor{curcolor}{0 0 0}
\pscustom[linestyle=none,fillstyle=solid,fillcolor=curcolor]
{
\newpath
\moveto(773.75,1195.14315033)
\lineto(773.75,515.51810033)
}
}
{
\newrgbcolor{curcolor}{0 0 0}
\pscustom[linewidth=0.63506401,linecolor=curcolor]
{
\newpath
\moveto(773.75,1195.14315033)
\lineto(773.75,515.51810033)
}
}
{
\newrgbcolor{curcolor}{0 0 0}
\pscustom[linestyle=none,fillstyle=solid,fillcolor=curcolor]
{
\newpath
\moveto(773.75,1192.65369943)
\curveto(772.34777868,1192.65369943)(771.20974398,1193.79173413)(771.20974398,1195.19395545)
\curveto(771.20974398,1196.59617678)(772.34777868,1197.73421147)(773.75,1197.73421147)
\curveto(775.15222132,1197.73421147)(776.29025602,1196.59617678)(776.29025602,1195.19395545)
\curveto(776.29025602,1193.79173413)(775.15222132,1192.65369943)(773.75,1192.65369943)
\closepath
}
}
{
\newrgbcolor{curcolor}{0 0 0}
\pscustom[linewidth=0.63506401,linecolor=curcolor]
{
\newpath
\moveto(773.75,1192.65369943)
\curveto(772.34777868,1192.65369943)(771.20974398,1193.79173413)(771.20974398,1195.19395545)
\curveto(771.20974398,1196.59617678)(772.34777868,1197.73421147)(773.75,1197.73421147)
\curveto(775.15222132,1197.73421147)(776.29025602,1196.59617678)(776.29025602,1195.19395545)
\curveto(776.29025602,1193.79173413)(775.15222132,1192.65369943)(773.75,1192.65369943)
\closepath
}
}
{
\newrgbcolor{curcolor}{0 0 0}
\pscustom[linestyle=none,fillstyle=solid,fillcolor=curcolor]
{
\newpath
\moveto(776.56784804,522.30721816)
\lineto(773.76118639,514.67472361)
\lineto(770.95452536,522.30721858)
\curveto(772.6115769,521.08786505)(774.87871101,521.09489099)(776.56784804,522.30721816)
\closepath
}
}
{
\newrgbcolor{curcolor}{0 0 0}
\pscustom[linestyle=none,fillstyle=solid,fillcolor=curcolor]
{
\newpath
\moveto(453.1875,1239.36190033)
\lineto(161.96875,1129.86190033)
}
}
{
\newrgbcolor{curcolor}{0 0 0}
\pscustom[linewidth=0.63506401,linecolor=curcolor]
{
\newpath
\moveto(453.1875,1239.36190033)
\lineto(161.96875,1129.86190033)
}
}
{
\newrgbcolor{curcolor}{0 0 0}
\pscustom[linestyle=none,fillstyle=solid,fillcolor=curcolor]
{
\newpath
\moveto(450.85732608,1238.48574093)
\curveto(450.36381589,1239.79824706)(451.02850824,1241.26399943)(452.34101437,1241.75750963)
\curveto(453.6535205,1242.25101982)(455.11927287,1241.58632746)(455.61278306,1240.27382134)
\curveto(456.10629325,1238.96131521)(455.4416009,1237.49556283)(454.12909477,1237.00205264)
\curveto(452.81658864,1236.50854245)(451.35083627,1237.1732348)(450.85732608,1238.48574093)
\closepath
}
}
{
\newrgbcolor{curcolor}{0 0 0}
\pscustom[linewidth=0.63506401,linecolor=curcolor]
{
\newpath
\moveto(450.85732608,1238.48574093)
\curveto(450.36381589,1239.79824706)(451.02850824,1241.26399943)(452.34101437,1241.75750963)
\curveto(453.6535205,1242.25101982)(455.11927287,1241.58632746)(455.61278306,1240.27382134)
\curveto(456.10629325,1238.96131521)(455.4416009,1237.49556283)(454.12909477,1237.00205264)
\curveto(452.81658864,1236.50854245)(451.35083627,1237.1732348)(450.85732608,1238.48574093)
\closepath
}
}
{
\newrgbcolor{curcolor}{0 0 0}
\pscustom[linestyle=none,fillstyle=solid,fillcolor=curcolor]
{
\newpath
\moveto(169.31523325,1129.61376263)
\lineto(161.18327021,1129.55460419)
\lineto(167.33963115,1134.86794076)
\curveto(166.78149019,1132.88775857)(167.58598186,1130.76815021)(169.31523325,1129.61376263)
\closepath
}
}
{
\newrgbcolor{curcolor}{0.85882354 0.85882354 0.85882354}
\pscustom[linestyle=none,fillstyle=solid,fillcolor=curcolor]
{
\newpath
\moveto(107.02226543,1698.25254154)
\lineto(168.8181963,1698.25254154)
\curveto(172.22774598,1698.25254154)(174.9726181,1695.75337357)(174.9726181,1692.64902592)
\lineto(174.9726181,1682.37640476)
\curveto(174.9726181,1679.27205711)(172.22774598,1676.77288914)(168.8181963,1676.77288914)
\lineto(107.02226543,1676.77288914)
\curveto(103.61271575,1676.77288914)(100.86784363,1679.27205711)(100.86784363,1682.37640476)
\lineto(100.86784363,1692.64902592)
\curveto(100.86784363,1695.75337357)(103.61271575,1698.25254154)(107.02226543,1698.25254154)
\closepath
}
}
{
\newrgbcolor{curcolor}{0.85882354 0.85882354 0.85882354}
\pscustom[linestyle=none,fillstyle=solid,fillcolor=curcolor]
{
\newpath
\moveto(601.23108196,1304.9182663)
\lineto(637.20673084,1304.9182663)
\curveto(640.61628052,1304.9182663)(643.36115265,1302.41909833)(643.36115265,1299.31475067)
\lineto(643.36115265,1289.04213715)
\curveto(643.36115265,1285.93778949)(640.61628052,1283.43862152)(637.20673084,1283.43862152)
\lineto(601.23108196,1283.43862152)
\curveto(597.82153228,1283.43862152)(595.07666016,1285.93778949)(595.07666016,1289.04213715)
\lineto(595.07666016,1299.31475067)
\curveto(595.07666016,1302.41909833)(597.82153228,1304.9182663)(601.23108196,1304.9182663)
\closepath
}
}
{
\newrgbcolor{curcolor}{0.85882354 0.85882354 0.85882354}
\pscustom[linestyle=none,fillstyle=solid,fillcolor=curcolor]
{
\newpath
\moveto(454.99853802,1261.12770844)
\lineto(508.44086933,1261.12770844)
\curveto(511.85041901,1261.12770844)(514.59529114,1258.62854047)(514.59529114,1255.52419281)
\lineto(514.59529114,1245.25157928)
\curveto(514.59529114,1242.14723163)(511.85041901,1239.64806366)(508.44086933,1239.64806366)
\lineto(454.99853802,1239.64806366)
\curveto(451.58898834,1239.64806366)(448.84411621,1242.14723163)(448.84411621,1245.25157928)
\lineto(448.84411621,1255.52419281)
\curveto(448.84411621,1258.62854047)(451.58898834,1261.12770844)(454.99853802,1261.12770844)
\closepath
}
}
{
\newrgbcolor{curcolor}{0.85882354 0.85882354 0.85882354}
\pscustom[linestyle=none,fillstyle=solid,fillcolor=curcolor]
{
\newpath
\moveto(747.35473919,1216.60884857)
\lineto(800.79705524,1216.60884857)
\curveto(804.20660493,1216.60884857)(806.95147705,1214.1096806)(806.95147705,1211.00533295)
\lineto(806.95147705,1200.73271942)
\curveto(806.95147705,1197.62837177)(804.20660493,1195.1292038)(800.79705524,1195.1292038)
\lineto(747.35473919,1195.1292038)
\curveto(743.94518951,1195.1292038)(741.20031738,1197.62837177)(741.20031738,1200.73271942)
\lineto(741.20031738,1211.00533295)
\curveto(741.20031738,1214.1096806)(743.94518951,1216.60884857)(747.35473919,1216.60884857)
\closepath
}
}
{
\newrgbcolor{curcolor}{0.85882354 0.85882354 0.85882354}
\pscustom[linestyle=none,fillstyle=solid,fillcolor=curcolor]
{
\newpath
\moveto(596.67456341,1173.26692963)
\lineto(641.76325703,1173.26692963)
\curveto(645.17280671,1173.26692963)(647.91767883,1170.76776166)(647.91767883,1167.663414)
\lineto(647.91767883,1157.39080048)
\curveto(647.91767883,1154.28645282)(645.17280671,1151.78728485)(641.76325703,1151.78728485)
\lineto(596.67456341,1151.78728485)
\curveto(593.26501373,1151.78728485)(590.5201416,1154.28645282)(590.5201416,1157.39080048)
\lineto(590.5201416,1167.663414)
\curveto(590.5201416,1170.76776166)(593.26501373,1173.26692963)(596.67456341,1173.26692963)
\closepath
}
}
{
\newrgbcolor{curcolor}{0.85882354 0.85882354 0.85882354}
\pscustom[linestyle=none,fillstyle=solid,fillcolor=curcolor]
{
\newpath
\moveto(89.49772167,1129.55190277)
\lineto(177.87335682,1129.55190277)
\curveto(181.2829065,1129.55190277)(184.02777863,1127.0527348)(184.02777863,1123.94838715)
\lineto(184.02777863,1113.67577362)
\curveto(184.02777863,1110.57142596)(181.2829065,1108.072258)(177.87335682,1108.072258)
\lineto(89.49772167,1108.072258)
\curveto(86.08817199,1108.072258)(83.34329987,1110.57142596)(83.34329987,1113.67577362)
\lineto(83.34329987,1123.94838715)
\curveto(83.34329987,1127.0527348)(86.08817199,1129.55190277)(89.49772167,1129.55190277)
\closepath
}
}
{
\newrgbcolor{curcolor}{0.85882354 0.85882354 0.85882354}
\pscustom[linestyle=none,fillstyle=solid,fillcolor=curcolor]
{
\newpath
\moveto(595.15570354,1129.55190277)
\lineto(643.28207874,1129.55190277)
\curveto(646.69162842,1129.55190277)(649.43650055,1127.0527348)(649.43650055,1123.94838715)
\lineto(649.43650055,1113.67577362)
\curveto(649.43650055,1110.57142596)(646.69162842,1108.072258)(643.28207874,1108.072258)
\lineto(595.15570354,1108.072258)
\curveto(591.74615386,1108.072258)(589.00128174,1110.57142596)(589.00128174,1113.67577362)
\lineto(589.00128174,1123.94838715)
\curveto(589.00128174,1127.0527348)(591.74615386,1129.55190277)(595.15570354,1129.55190277)
\closepath
}
}
{
\newrgbcolor{curcolor}{0.85882354 0.85882354 0.85882354}
\pscustom[linestyle=none,fillstyle=solid,fillcolor=curcolor]
{
\newpath
\moveto(90.25712872,1084.61849213)
\lineto(177.11391926,1084.61849213)
\curveto(180.52346894,1084.61849213)(183.26834106,1082.11932416)(183.26834106,1079.0149765)
\lineto(183.26834106,1068.74236298)
\curveto(183.26834106,1065.63801532)(180.52346894,1063.13884735)(177.11391926,1063.13884735)
\lineto(90.25712872,1063.13884735)
\curveto(86.84757903,1063.13884735)(84.10270691,1065.63801532)(84.10270691,1068.74236298)
\lineto(84.10270691,1079.0149765)
\curveto(84.10270691,1082.11932416)(86.84757903,1084.61849213)(90.25712872,1084.61849213)
\closepath
}
}
{
\newrgbcolor{curcolor}{0.85882354 0.85882354 0.85882354}
\pscustom[linestyle=none,fillstyle=solid,fillcolor=curcolor]
{
\newpath
\moveto(940.67236614,1084.61849213)
\lineto(974.36975956,1084.61849213)
\curveto(977.77930924,1084.61849213)(980.52418137,1082.11932416)(980.52418137,1079.0149765)
\lineto(980.52418137,1068.74236298)
\curveto(980.52418137,1065.63801532)(977.77930924,1063.13884735)(974.36975956,1063.13884735)
\lineto(940.67236614,1063.13884735)
\curveto(937.26281646,1063.13884735)(934.51794434,1065.63801532)(934.51794434,1068.74236298)
\lineto(934.51794434,1079.0149765)
\curveto(934.51794434,1082.11932416)(937.26281646,1084.61849213)(940.67236614,1084.61849213)
\closepath
}
}
{
\newrgbcolor{curcolor}{0.85882354 0.85882354 0.85882354}
\pscustom[linestyle=none,fillstyle=solid,fillcolor=curcolor]
{
\newpath
\moveto(102.78757,1478.83809662)
\lineto(164.58350086,1478.83809662)
\curveto(167.99305054,1478.83809662)(170.73792267,1476.33892865)(170.73792267,1473.23458099)
\lineto(170.73792267,1462.96195984)
\curveto(170.73792267,1459.85761218)(167.99305054,1457.35844421)(164.58350086,1457.35844421)
\lineto(102.78757,1457.35844421)
\curveto(99.37802032,1457.35844421)(96.63314819,1459.85761218)(96.63314819,1462.96195984)
\lineto(96.63314819,1473.23458099)
\curveto(96.63314819,1476.33892865)(99.37802032,1478.83809662)(102.78757,1478.83809662)
\closepath
}
}
{
\newrgbcolor{curcolor}{0.85882354 0.85882354 0.85882354}
\pscustom[linestyle=none,fillstyle=solid,fillcolor=curcolor]
{
\newpath
\moveto(57.54064655,1609.28933716)
\lineto(218.29979992,1609.28933716)
\curveto(221.8246694,1609.28933716)(224.66238022,1606.79016919)(224.66238022,1603.68582153)
\lineto(224.66238022,1593.41320038)
\curveto(224.66238022,1590.30885272)(221.8246694,1587.80968475)(218.29979992,1587.80968475)
\lineto(57.54064655,1587.80968475)
\curveto(54.01577707,1587.80968475)(51.17806625,1590.30885272)(51.17806625,1593.41320038)
\lineto(51.17806625,1603.68582153)
\curveto(51.17806625,1606.79016919)(54.01577707,1609.28933716)(57.54064655,1609.28933716)
\closepath
}
}
{
\newrgbcolor{curcolor}{0.85882354 0.85882354 0.85882354}
\pscustom[linestyle=none,fillstyle=solid,fillcolor=curcolor]
{
\newpath
\moveto(57.54064655,1564.24308014)
\lineto(218.29979992,1564.24308014)
\curveto(221.8246694,1564.24308014)(224.66238022,1561.74391217)(224.66238022,1558.63956451)
\lineto(224.66238022,1548.36694336)
\curveto(224.66238022,1545.2625957)(221.8246694,1542.76342773)(218.29979992,1542.76342773)
\lineto(57.54064655,1542.76342773)
\curveto(54.01577707,1542.76342773)(51.17806625,1545.2625957)(51.17806625,1548.36694336)
\lineto(51.17806625,1558.63956451)
\curveto(51.17806625,1561.74391217)(54.01577707,1564.24308014)(57.54064655,1564.24308014)
\closepath
}
}
{
\newrgbcolor{curcolor}{0.85882354 0.85882354 0.85882354}
\pscustom[linestyle=none,fillstyle=solid,fillcolor=curcolor]
{
\newpath
\moveto(57.54064655,1521.55471039)
\lineto(218.29979992,1521.55471039)
\curveto(221.8246694,1521.55471039)(224.66238022,1519.05554242)(224.66238022,1515.95119476)
\lineto(224.66238022,1505.67857361)
\curveto(224.66238022,1502.57422595)(221.8246694,1500.07505798)(218.29979992,1500.07505798)
\lineto(57.54064655,1500.07505798)
\curveto(54.01577707,1500.07505798)(51.17806625,1502.57422595)(51.17806625,1505.67857361)
\lineto(51.17806625,1515.95119476)
\curveto(51.17806625,1519.05554242)(54.01577707,1521.55471039)(57.54064655,1521.55471039)
\closepath
}
}
{
\newrgbcolor{curcolor}{0.85882354 0.85882354 0.85882354}
\pscustom[linestyle=none,fillstyle=solid,fillcolor=curcolor]
{
\newpath
\moveto(249.73765659,1521.55471039)
\lineto(410.49680996,1521.55471039)
\curveto(414.02167945,1521.55471039)(416.85939026,1519.05554242)(416.85939026,1515.95119476)
\lineto(416.85939026,1505.67857361)
\curveto(416.85939026,1502.57422595)(414.02167945,1500.07505798)(410.49680996,1500.07505798)
\lineto(249.73765659,1500.07505798)
\curveto(246.21278711,1500.07505798)(243.37507629,1502.57422595)(243.37507629,1505.67857361)
\lineto(243.37507629,1515.95119476)
\curveto(243.37507629,1519.05554242)(246.21278711,1521.55471039)(249.73765659,1521.55471039)
\closepath
}
}
{
\newrgbcolor{curcolor}{0.85882354 0.85882354 0.85882354}
\pscustom[linestyle=none,fillstyle=solid,fillcolor=curcolor]
{
\newpath
\moveto(249.73765659,1478.83809662)
\lineto(410.49680996,1478.83809662)
\curveto(414.02167945,1478.83809662)(416.85939026,1476.33892865)(416.85939026,1473.23458099)
\lineto(416.85939026,1462.96195984)
\curveto(416.85939026,1459.85761218)(414.02167945,1457.35844421)(410.49680996,1457.35844421)
\lineto(249.73765659,1457.35844421)
\curveto(246.21278711,1457.35844421)(243.37507629,1459.85761218)(243.37507629,1462.96195984)
\lineto(243.37507629,1473.23458099)
\curveto(243.37507629,1476.33892865)(246.21278711,1478.83809662)(249.73765659,1478.83809662)
\closepath
}
}
{
\newrgbcolor{curcolor}{0.85882354 0.85882354 0.85882354}
\pscustom[linestyle=none,fillstyle=solid,fillcolor=curcolor]
{
\newpath
\moveto(249.73765659,1433.48818207)
\lineto(410.49680996,1433.48818207)
\curveto(414.02167945,1433.48818207)(416.85939026,1430.9890141)(416.85939026,1427.88466644)
\lineto(416.85939026,1417.61204529)
\curveto(416.85939026,1414.50769763)(414.02167945,1412.00852966)(410.49680996,1412.00852966)
\lineto(249.73765659,1412.00852966)
\curveto(246.21278711,1412.00852966)(243.37507629,1414.50769763)(243.37507629,1417.61204529)
\lineto(243.37507629,1427.88466644)
\curveto(243.37507629,1430.9890141)(246.21278711,1433.48818207)(249.73765659,1433.48818207)
\closepath
}
}
{
\newrgbcolor{curcolor}{0.85882354 0.85882354 0.85882354}
\pscustom[linestyle=none,fillstyle=solid,fillcolor=curcolor]
{
\newpath
\moveto(249.73765659,1261.12770844)
\lineto(410.49680996,1261.12770844)
\curveto(414.02167945,1261.12770844)(416.85939026,1258.62854047)(416.85939026,1255.52419281)
\lineto(416.85939026,1245.25157166)
\curveto(416.85939026,1242.147224)(414.02167945,1239.64805603)(410.49680996,1239.64805603)
\lineto(249.73765659,1239.64805603)
\curveto(246.21278711,1239.64805603)(243.37507629,1242.147224)(243.37507629,1245.25157166)
\lineto(243.37507629,1255.52419281)
\curveto(243.37507629,1258.62854047)(246.21278711,1261.12770844)(249.73765659,1261.12770844)
\closepath
}
}
{
\newrgbcolor{curcolor}{0.85882354 0.85882354 0.85882354}
\pscustom[linestyle=none,fillstyle=solid,fillcolor=curcolor]
{
\newpath
\moveto(403.3452158,996.29299164)
\lineto(564.10436916,996.29299164)
\curveto(567.62923865,996.29299164)(570.46694946,993.79382367)(570.46694946,990.68947601)
\lineto(570.46694946,980.41685486)
\curveto(570.46694946,977.3125072)(567.62923865,974.81333923)(564.10436916,974.81333923)
\lineto(403.3452158,974.81333923)
\curveto(399.82034631,974.81333923)(396.9826355,977.3125072)(396.9826355,980.41685486)
\lineto(396.9826355,990.68947601)
\curveto(396.9826355,993.79382367)(399.82034631,996.29299164)(403.3452158,996.29299164)
\closepath
}
}
{
\newrgbcolor{curcolor}{0.85882354 0.85882354 0.85882354}
\pscustom[linestyle=none,fillstyle=solid,fillcolor=curcolor]
{
\newpath
\moveto(403.3452158,951.22597504)
\lineto(564.10436916,951.22597504)
\curveto(567.62923865,951.22597504)(570.46694946,948.72680707)(570.46694946,945.62245941)
\lineto(570.46694946,935.34983826)
\curveto(570.46694946,932.2454906)(567.62923865,929.74632263)(564.10436916,929.74632263)
\lineto(403.3452158,929.74632263)
\curveto(399.82034631,929.74632263)(396.9826355,932.2454906)(396.9826355,935.34983826)
\lineto(396.9826355,945.62245941)
\curveto(396.9826355,948.72680707)(399.82034631,951.22597504)(403.3452158,951.22597504)
\closepath
}
}
{
\newrgbcolor{curcolor}{0.85882354 0.85882354 0.85882354}
\pscustom[linestyle=none,fillstyle=solid,fillcolor=curcolor]
{
\newpath
\moveto(403.3452158,820.91634369)
\lineto(564.10436916,820.91634369)
\curveto(567.62923865,820.91634369)(570.46694946,818.41717572)(570.46694946,815.31282806)
\lineto(570.46694946,805.04020691)
\curveto(570.46694946,801.93585925)(567.62923865,799.43669128)(564.10436916,799.43669128)
\lineto(403.3452158,799.43669128)
\curveto(399.82034631,799.43669128)(396.9826355,801.93585925)(396.9826355,805.04020691)
\lineto(396.9826355,815.31282806)
\curveto(396.9826355,818.41717572)(399.82034631,820.91634369)(403.3452158,820.91634369)
\closepath
}
}
{
\newrgbcolor{curcolor}{0.85882354 0.85882354 0.85882354}
\pscustom[linestyle=none,fillstyle=solid,fillcolor=curcolor]
{
\newpath
\moveto(683.69324017,1129.55190277)
\lineto(745.48917103,1129.55190277)
\curveto(748.89872071,1129.55190277)(751.64359283,1127.0527348)(751.64359283,1123.94838715)
\lineto(751.64359283,1113.67576599)
\curveto(751.64359283,1110.57141834)(748.89872071,1108.07225037)(745.48917103,1108.07225037)
\lineto(683.69324017,1108.07225037)
\curveto(680.28369049,1108.07225037)(677.53881836,1110.57141834)(677.53881836,1113.67576599)
\lineto(677.53881836,1123.94838715)
\curveto(677.53881836,1127.0527348)(680.28369049,1129.55190277)(683.69324017,1129.55190277)
\closepath
}
}
{
\newrgbcolor{curcolor}{0.85882354 0.85882354 0.85882354}
\pscustom[linestyle=none,fillstyle=solid,fillcolor=curcolor]
{
\newpath
\moveto(597.05426311,951.22597504)
\lineto(641.38353825,951.22597504)
\curveto(644.79308793,951.22597504)(647.53796005,948.72680707)(647.53796005,945.62245941)
\lineto(647.53796005,935.34984589)
\curveto(647.53796005,932.24549823)(644.79308793,929.74633026)(641.38353825,929.74633026)
\lineto(597.05426311,929.74633026)
\curveto(593.64471343,929.74633026)(590.89984131,932.24549823)(590.89984131,935.34984589)
\lineto(590.89984131,945.62245941)
\curveto(590.89984131,948.72680707)(593.64471343,951.22597504)(597.05426311,951.22597504)
\closepath
}
}
{
\newrgbcolor{curcolor}{0.85882354 0.85882354 0.85882354}
\pscustom[linestyle=none,fillstyle=solid,fillcolor=curcolor]
{
\newpath
\moveto(683.69324017,910.21504974)
\lineto(745.48917103,910.21504974)
\curveto(748.89872071,910.21504974)(751.64359283,907.71588178)(751.64359283,904.61153412)
\lineto(751.64359283,894.33891296)
\curveto(751.64359283,891.23456531)(748.89872071,888.73539734)(745.48917103,888.73539734)
\lineto(683.69324017,888.73539734)
\curveto(680.28369049,888.73539734)(677.53881836,891.23456531)(677.53881836,894.33891296)
\lineto(677.53881836,904.61153412)
\curveto(677.53881836,907.71588178)(680.28369049,910.21504974)(683.69324017,910.21504974)
\closepath
}
}
{
\newrgbcolor{curcolor}{0.85882354 0.85882354 0.85882354}
\pscustom[linestyle=none,fillstyle=solid,fillcolor=curcolor]
{
\newpath
\moveto(817.29065228,910.21504974)
\lineto(879.08658314,910.21504974)
\curveto(882.49613282,910.21504974)(885.24100494,907.71588178)(885.24100494,904.61153412)
\lineto(885.24100494,894.33891296)
\curveto(885.24100494,891.23456531)(882.49613282,888.73539734)(879.08658314,888.73539734)
\lineto(817.29065228,888.73539734)
\curveto(813.88110259,888.73539734)(811.13623047,891.23456531)(811.13623047,894.33891296)
\lineto(811.13623047,904.61153412)
\curveto(811.13623047,907.71588178)(813.88110259,910.21504974)(817.29065228,910.21504974)
\closepath
}
}
{
\newrgbcolor{curcolor}{0.85882354 0.85882354 0.85882354}
\pscustom[linestyle=none,fillstyle=solid,fillcolor=curcolor]
{
\newpath
\moveto(65.95569515,867.97103119)
\lineto(201.41537571,867.97103119)
\curveto(204.82492539,867.97103119)(207.56979752,865.47186322)(207.56979752,862.36751556)
\lineto(207.56979752,852.09490204)
\curveto(207.56979752,848.99055438)(204.82492539,846.49138641)(201.41537571,846.49138641)
\lineto(65.95569515,846.49138641)
\curveto(62.54614547,846.49138641)(59.80127335,848.99055438)(59.80127335,852.09490204)
\lineto(59.80127335,862.36751556)
\curveto(59.80127335,865.47186322)(62.54614547,867.97103119)(65.95569515,867.97103119)
\closepath
}
}
{
\newrgbcolor{curcolor}{0.85882354 0.85882354 0.85882354}
\pscustom[linestyle=none,fillstyle=solid,fillcolor=curcolor]
{
\newpath
\moveto(582.24560833,865.1132431)
\lineto(656.19224262,865.1132431)
\curveto(659.6017923,865.1132431)(662.34666443,862.61407513)(662.34666443,859.50972748)
\lineto(662.34666443,849.23711395)
\curveto(662.34666443,846.1327663)(659.6017923,843.63359833)(656.19224262,843.63359833)
\lineto(582.24560833,843.63359833)
\curveto(578.83605865,843.63359833)(576.09118652,846.1327663)(576.09118652,849.23711395)
\lineto(576.09118652,859.50972748)
\curveto(576.09118652,862.61407513)(578.83605865,865.1132431)(582.24560833,865.1132431)
\closepath
}
}
{
\newrgbcolor{curcolor}{0.85882354 0.85882354 0.85882354}
\pscustom[linestyle=none,fillstyle=solid,fillcolor=curcolor]
{
\newpath
\moveto(588.32098675,777.24703217)
\lineto(650.11691761,777.24703217)
\curveto(653.52646729,777.24703217)(656.27133942,774.7478642)(656.27133942,771.64351654)
\lineto(656.27133942,761.37089539)
\curveto(656.27133942,758.26654773)(653.52646729,755.76737976)(650.11691761,755.76737976)
\lineto(588.32098675,755.76737976)
\curveto(584.91143707,755.76737976)(582.16656494,758.26654773)(582.16656494,761.37089539)
\lineto(582.16656494,771.64351654)
\curveto(582.16656494,774.7478642)(584.91143707,777.24703217)(588.32098675,777.24703217)
\closepath
}
}
{
\newrgbcolor{curcolor}{0.85882354 0.85882354 0.85882354}
\pscustom[linestyle=none,fillstyle=solid,fillcolor=curcolor]
{
\newpath
\moveto(137.85135174,734.62856293)
\lineto(199.6472826,734.62856293)
\curveto(203.05683228,734.62856293)(205.80170441,732.12939496)(205.80170441,729.0250473)
\lineto(205.80170441,718.75242615)
\curveto(205.80170441,715.64807849)(203.05683228,713.14891052)(199.6472826,713.14891052)
\lineto(137.85135174,713.14891052)
\curveto(134.44180206,713.14891052)(131.69692993,715.64807849)(131.69692993,718.75242615)
\lineto(131.69692993,729.0250473)
\curveto(131.69692993,732.12939496)(134.44180206,734.62856293)(137.85135174,734.62856293)
\closepath
}
}
{
\newrgbcolor{curcolor}{0.85882354 0.85882354 0.85882354}
\pscustom[linestyle=none,fillstyle=solid,fillcolor=curcolor]
{
\newpath
\moveto(269.32220745,734.62856293)
\lineto(331.11813831,734.62856293)
\curveto(334.52768799,734.62856293)(337.27256012,732.12939496)(337.27256012,729.0250473)
\lineto(337.27256012,718.75242615)
\curveto(337.27256012,715.64807849)(334.52768799,713.14891052)(331.11813831,713.14891052)
\lineto(269.32220745,713.14891052)
\curveto(265.91265777,713.14891052)(263.16778564,715.64807849)(263.16778564,718.75242615)
\lineto(263.16778564,729.0250473)
\curveto(263.16778564,732.12939496)(265.91265777,734.62856293)(269.32220745,734.62856293)
\closepath
}
}
{
\newrgbcolor{curcolor}{0.85882354 0.85882354 0.85882354}
\pscustom[linestyle=none,fillstyle=solid,fillcolor=curcolor]
{
\newpath
\moveto(868.90851116,735.26363373)
\lineto(930.70444202,735.26363373)
\curveto(934.1139917,735.26363373)(936.85886383,732.76446576)(936.85886383,729.6601181)
\lineto(936.85886383,719.38749695)
\curveto(936.85886383,716.28314929)(934.1139917,713.78398132)(930.70444202,713.78398132)
\lineto(868.90851116,713.78398132)
\curveto(865.49896148,713.78398132)(862.75408936,716.28314929)(862.75408936,719.38749695)
\lineto(862.75408936,729.6601181)
\curveto(862.75408936,732.76446576)(865.49896148,735.26363373)(868.90851116,735.26363373)
\closepath
}
}
{
\newrgbcolor{curcolor}{0.85882354 0.85882354 0.85882354}
\pscustom[linestyle=none,fillstyle=solid,fillcolor=curcolor]
{
\newpath
\moveto(65.54838085,691.50740814)
\lineto(137.21677876,691.50740814)
\curveto(140.62632844,691.50740814)(143.37120056,689.00824017)(143.37120056,685.90389252)
\lineto(143.37120056,675.63127899)
\curveto(143.37120056,672.52693134)(140.62632844,670.02776337)(137.21677876,670.02776337)
\lineto(65.54838085,670.02776337)
\curveto(62.13883117,670.02776337)(59.39395905,672.52693134)(59.39395905,675.63127899)
\lineto(59.39395905,685.90389252)
\curveto(59.39395905,689.00824017)(62.13883117,691.50740814)(65.54838085,691.50740814)
\closepath
}
}
{
\newrgbcolor{curcolor}{0.85882354 0.85882354 0.85882354}
\pscustom[linestyle=none,fillstyle=solid,fillcolor=curcolor]
{
\newpath
\moveto(210.92705059,691.50740814)
\lineto(287.15195942,691.50740814)
\curveto(290.5615091,691.50740814)(293.30638123,689.00824017)(293.30638123,685.90389252)
\lineto(293.30638123,675.63127899)
\curveto(293.30638123,672.52693134)(290.5615091,670.02776337)(287.15195942,670.02776337)
\lineto(210.92705059,670.02776337)
\curveto(207.51750091,670.02776337)(204.77262878,672.52693134)(204.77262878,675.63127899)
\lineto(204.77262878,685.90389252)
\curveto(204.77262878,689.00824017)(207.51750091,691.50740814)(210.92705059,691.50740814)
\closepath
}
}
{
\newrgbcolor{curcolor}{0.85882354 0.85882354 0.85882354}
\pscustom[linestyle=none,fillstyle=solid,fillcolor=curcolor]
{
\newpath
\moveto(309.65164471,691.50740814)
\lineto(371.44757557,691.50740814)
\curveto(374.85712525,691.50740814)(377.60199738,689.00824017)(377.60199738,685.90389252)
\lineto(377.60199738,675.63127136)
\curveto(377.60199738,672.52692371)(374.85712525,670.02775574)(371.44757557,670.02775574)
\lineto(309.65164471,670.02775574)
\curveto(306.24209503,670.02775574)(303.4972229,672.52692371)(303.4972229,675.63127136)
\lineto(303.4972229,685.90389252)
\curveto(303.4972229,689.00824017)(306.24209503,691.50740814)(309.65164471,691.50740814)
\closepath
}
}
{
\newrgbcolor{curcolor}{0.85882354 0.85882354 0.85882354}
\pscustom[linestyle=none,fillstyle=solid,fillcolor=curcolor]
{
\newpath
\moveto(401.54147625,691.50740814)
\lineto(463.33740711,691.50740814)
\curveto(466.74695679,691.50740814)(469.49182892,689.00824017)(469.49182892,685.90389252)
\lineto(469.49182892,675.63127136)
\curveto(469.49182892,672.52692371)(466.74695679,670.02775574)(463.33740711,670.02775574)
\lineto(401.54147625,670.02775574)
\curveto(398.13192657,670.02775574)(395.38705444,672.52692371)(395.38705444,675.63127136)
\lineto(395.38705444,685.90389252)
\curveto(395.38705444,689.00824017)(398.13192657,691.50740814)(401.54147625,691.50740814)
\closepath
}
}
{
\newrgbcolor{curcolor}{0.85882354 0.85882354 0.85882354}
\pscustom[linestyle=none,fillstyle=solid,fillcolor=curcolor]
{
\newpath
\moveto(926.62311077,686.74440765)
\lineto(988.41904163,686.74440765)
\curveto(991.82859131,686.74440765)(994.57346344,684.24523969)(994.57346344,681.14089203)
\lineto(994.57346344,670.86827087)
\curveto(994.57346344,667.76392322)(991.82859131,665.26475525)(988.41904163,665.26475525)
\lineto(926.62311077,665.26475525)
\curveto(923.21356109,665.26475525)(920.46868896,667.76392322)(920.46868896,670.86827087)
\lineto(920.46868896,681.14089203)
\curveto(920.46868896,684.24523969)(923.21356109,686.74440765)(926.62311077,686.74440765)
\closepath
}
}
{
\newrgbcolor{curcolor}{0.85882354 0.85882354 0.85882354}
\pscustom[linestyle=none,fillstyle=solid,fillcolor=curcolor]
{
\newpath
\moveto(465.81683636,646.55794525)
\lineto(527.61276722,646.55794525)
\curveto(531.0223169,646.55794525)(533.76718903,644.05877728)(533.76718903,640.95442963)
\lineto(533.76718903,630.68180847)
\curveto(533.76718903,627.57746082)(531.0223169,625.07829285)(527.61276722,625.07829285)
\lineto(465.81683636,625.07829285)
\curveto(462.40728668,625.07829285)(459.66241455,627.57746082)(459.66241455,630.68180847)
\lineto(459.66241455,640.95442963)
\curveto(459.66241455,644.05877728)(462.40728668,646.55794525)(465.81683636,646.55794525)
\closepath
}
}
{
\newrgbcolor{curcolor}{0.85882354 0.85882354 0.85882354}
\pscustom[linestyle=none,fillstyle=solid,fillcolor=curcolor]
{
\newpath
\moveto(215.24869061,646.55794525)
\lineto(282.36056232,646.55794525)
\curveto(285.77011201,646.55794525)(288.51498413,644.05877728)(288.51498413,640.95442963)
\lineto(288.51498413,630.6818161)
\curveto(288.51498413,627.57746845)(285.77011201,625.07830048)(282.36056232,625.07830048)
\lineto(215.24869061,625.07830048)
\curveto(211.83914092,625.07830048)(209.0942688,627.57746845)(209.0942688,630.6818161)
\lineto(209.0942688,640.95442963)
\curveto(209.0942688,644.05877728)(211.83914092,646.55794525)(215.24869061,646.55794525)
\closepath
}
}
{
\newrgbcolor{curcolor}{0.85882354 0.85882354 0.85882354}
\pscustom[linestyle=none,fillstyle=solid,fillcolor=curcolor]
{
\newpath
\moveto(465.81683636,602.76094818)
\lineto(527.61276722,602.76094818)
\curveto(531.0223169,602.76094818)(533.76718903,600.26178021)(533.76718903,597.15743256)
\lineto(533.76718903,586.8848114)
\curveto(533.76718903,583.78046375)(531.0223169,581.28129578)(527.61276722,581.28129578)
\lineto(465.81683636,581.28129578)
\curveto(462.40728668,581.28129578)(459.66241455,583.78046375)(459.66241455,586.8848114)
\lineto(459.66241455,597.15743256)
\curveto(459.66241455,600.26178021)(462.40728668,602.76094818)(465.81683636,602.76094818)
\closepath
}
}
{
\newrgbcolor{curcolor}{0.85882354 0.85882354 0.85882354}
\pscustom[linestyle=none,fillstyle=solid,fillcolor=curcolor]
{
\newpath
\moveto(333.90707684,602.76094818)
\lineto(395.7030077,602.76094818)
\curveto(399.11255738,602.76094818)(401.8574295,600.26178021)(401.8574295,597.15743256)
\lineto(401.8574295,586.8848114)
\curveto(401.8574295,583.78046375)(399.11255738,581.28129578)(395.7030077,581.28129578)
\lineto(333.90707684,581.28129578)
\curveto(330.49752715,581.28129578)(327.75265503,583.78046375)(327.75265503,586.8848114)
\lineto(327.75265503,597.15743256)
\curveto(327.75265503,600.26178021)(330.49752715,602.76094818)(333.90707684,602.76094818)
\closepath
}
}
{
\newrgbcolor{curcolor}{0.85882354 0.85882354 0.85882354}
\pscustom[linestyle=none,fillstyle=solid,fillcolor=curcolor]
{
\newpath
\moveto(588.32098675,602.76094818)
\lineto(650.11691761,602.76094818)
\curveto(653.52646729,602.76094818)(656.27133942,600.26178021)(656.27133942,597.15743256)
\lineto(656.27133942,586.8848114)
\curveto(656.27133942,583.78046375)(653.52646729,581.28129578)(650.11691761,581.28129578)
\lineto(588.32098675,581.28129578)
\curveto(584.91143707,581.28129578)(582.16656494,583.78046375)(582.16656494,586.8848114)
\lineto(582.16656494,597.15743256)
\curveto(582.16656494,600.26178021)(584.91143707,602.76094818)(588.32098675,602.76094818)
\closepath
}
}
{
\newrgbcolor{curcolor}{0.85882354 0.85882354 0.85882354}
\pscustom[linestyle=none,fillstyle=solid,fillcolor=curcolor]
{
\newpath
\moveto(678.3773222,602.76094818)
\lineto(750.80512714,602.76094818)
\curveto(754.21467682,602.76094818)(756.95954895,600.26178021)(756.95954895,597.15743256)
\lineto(756.95954895,586.88481903)
\curveto(756.95954895,583.78047137)(754.21467682,581.28130341)(750.80512714,581.28130341)
\lineto(678.3773222,581.28130341)
\curveto(674.96777252,581.28130341)(672.22290039,583.78047137)(672.22290039,586.88481903)
\lineto(672.22290039,597.15743256)
\curveto(672.22290039,600.26178021)(674.96777252,602.76094818)(678.3773222,602.76094818)
\closepath
}
}
{
\newrgbcolor{curcolor}{0.85882354 0.85882354 0.85882354}
\pscustom[linestyle=none,fillstyle=solid,fillcolor=curcolor]
{
\newpath
\moveto(557.94415569,558.21065521)
\lineto(680.49368,558.21065521)
\curveto(683.90322968,558.21065521)(686.64810181,555.71148724)(686.64810181,552.60713959)
\lineto(686.64810181,542.33452606)
\curveto(686.64810181,539.23017841)(683.90322968,536.73101044)(680.49368,536.73101044)
\lineto(557.94415569,536.73101044)
\curveto(554.53460601,536.73101044)(551.78973389,539.23017841)(551.78973389,542.33452606)
\lineto(551.78973389,552.60713959)
\curveto(551.78973389,555.71148724)(554.53460601,558.21065521)(557.94415569,558.21065521)
\closepath
}
}
{
\newrgbcolor{curcolor}{0.85882354 0.85882354 0.85882354}
\pscustom[linestyle=none,fillstyle=solid,fillcolor=curcolor]
{
\newpath
\moveto(217.90666485,558.21065521)
\lineto(279.70259571,558.21065521)
\curveto(283.11214539,558.21065521)(285.85701752,555.71148724)(285.85701752,552.60713959)
\lineto(285.85701752,542.33451843)
\curveto(285.85701752,539.23017078)(283.11214539,536.73100281)(279.70259571,536.73100281)
\lineto(217.90666485,536.73100281)
\curveto(214.49711517,536.73100281)(211.75224304,539.23017078)(211.75224304,542.33451843)
\lineto(211.75224304,552.60713959)
\curveto(211.75224304,555.71148724)(214.49711517,558.21065521)(217.90666485,558.21065521)
\closepath
}
}
{
\newrgbcolor{curcolor}{0.85882354 0.85882354 0.85882354}
\pscustom[linestyle=none,fillstyle=solid,fillcolor=curcolor]
{
\newpath
\moveto(465.81683636,558.21065521)
\lineto(527.61276722,558.21065521)
\curveto(531.0223169,558.21065521)(533.76718903,555.71148724)(533.76718903,552.60713959)
\lineto(533.76718903,542.33451843)
\curveto(533.76718903,539.23017078)(531.0223169,536.73100281)(527.61276722,536.73100281)
\lineto(465.81683636,536.73100281)
\curveto(462.40728668,536.73100281)(459.66241455,539.23017078)(459.66241455,542.33451843)
\lineto(459.66241455,552.60713959)
\curveto(459.66241455,555.71148724)(462.40728668,558.21065521)(465.81683636,558.21065521)
\closepath
}
}
{
\newrgbcolor{curcolor}{0.85882354 0.85882354 0.85882354}
\pscustom[linestyle=none,fillstyle=solid,fillcolor=curcolor]
{
\newpath
\moveto(334.05276775,515.53328705)
\lineto(404.20232105,515.53328705)
\curveto(407.61187073,515.53328705)(410.35674286,513.03411908)(410.35674286,509.92977142)
\lineto(410.35674286,499.6571579)
\curveto(410.35674286,496.55281024)(407.61187073,494.05364227)(404.20232105,494.05364227)
\lineto(334.05276775,494.05364227)
\curveto(330.64321807,494.05364227)(327.89834595,496.55281024)(327.89834595,499.6571579)
\lineto(327.89834595,509.92977142)
\curveto(327.89834595,513.03411908)(330.64321807,515.53328705)(334.05276775,515.53328705)
\closepath
}
}
{
\newrgbcolor{curcolor}{0.85882354 0.85882354 0.85882354}
\pscustom[linestyle=none,fillstyle=solid,fillcolor=curcolor]
{
\newpath
\moveto(721.9141264,515.53328705)
\lineto(826.23757648,515.53328705)
\curveto(829.64712617,515.53328705)(832.39199829,513.03411908)(832.39199829,509.92977142)
\lineto(832.39199829,499.6571579)
\curveto(832.39199829,496.55281024)(829.64712617,494.05364227)(826.23757648,494.05364227)
\lineto(721.9141264,494.05364227)
\curveto(718.50457672,494.05364227)(715.75970459,496.55281024)(715.75970459,499.6571579)
\lineto(715.75970459,509.92977142)
\curveto(715.75970459,513.03411908)(718.50457672,515.53328705)(721.9141264,515.53328705)
\closepath
}
}
{
\newrgbcolor{curcolor}{0.85882354 0.85882354 0.85882354}
\pscustom[linestyle=none,fillstyle=solid,fillcolor=curcolor]
{
\newpath
\moveto(465.81683636,428.85762787)
\lineto(527.61276722,428.85762787)
\curveto(531.0223169,428.85762787)(533.76718903,426.3584599)(533.76718903,423.25411224)
\lineto(533.76718903,412.98149109)
\curveto(533.76718903,409.87714343)(531.0223169,407.37797546)(527.61276722,407.37797546)
\lineto(465.81683636,407.37797546)
\curveto(462.40728668,407.37797546)(459.66241455,409.87714343)(459.66241455,412.98149109)
\lineto(459.66241455,423.25411224)
\curveto(459.66241455,426.3584599)(462.40728668,428.85762787)(465.81683636,428.85762787)
\closepath
}
}
{
\newrgbcolor{curcolor}{0.85882354 0.85882354 0.85882354}
\pscustom[linestyle=none,fillstyle=solid,fillcolor=curcolor]
{
\newpath
\moveto(334.43246746,428.85762787)
\lineto(403.82259846,428.85762787)
\curveto(407.23214814,428.85762787)(409.97702026,426.3584599)(409.97702026,423.25411224)
\lineto(409.97702026,412.98149872)
\curveto(409.97702026,409.87715106)(407.23214814,407.37798309)(403.82259846,407.37798309)
\lineto(334.43246746,407.37798309)
\curveto(331.02291778,407.37798309)(328.27804565,409.87715106)(328.27804565,412.98149872)
\lineto(328.27804565,423.25411224)
\curveto(328.27804565,426.3584599)(331.02291778,428.85762787)(334.43246746,428.85762787)
\closepath
}
}
{
\newrgbcolor{curcolor}{0.85882354 0.85882354 0.85882354}
\pscustom[linestyle=none,fillstyle=solid,fillcolor=curcolor]
{
\newpath
\moveto(215.62840557,428.85762787)
\lineto(281.98086262,428.85762787)
\curveto(285.3904123,428.85762787)(288.13528442,426.3584599)(288.13528442,423.25411224)
\lineto(288.13528442,412.98149872)
\curveto(288.13528442,409.87715106)(285.3904123,407.37798309)(281.98086262,407.37798309)
\lineto(215.62840557,407.37798309)
\curveto(212.21885589,407.37798309)(209.47398376,409.87715106)(209.47398376,412.98149872)
\lineto(209.47398376,423.25411224)
\curveto(209.47398376,426.3584599)(212.21885589,428.85762787)(215.62840557,428.85762787)
\closepath
}
}
{
\newrgbcolor{curcolor}{0.85882354 0.85882354 0.85882354}
\pscustom[linestyle=none,fillstyle=solid,fillcolor=curcolor]
{
\newpath
\moveto(721.15472698,340.3932724)
\lineto(826.99703693,340.3932724)
\curveto(830.40658661,340.3932724)(833.15145874,337.89410443)(833.15145874,334.78975678)
\lineto(833.15145874,324.51714325)
\curveto(833.15145874,321.41279559)(830.40658661,318.91362762)(826.99703693,318.91362762)
\lineto(721.15472698,318.91362762)
\curveto(717.7451773,318.91362762)(715.00030518,321.41279559)(715.00030518,324.51714325)
\lineto(715.00030518,334.78975678)
\curveto(715.00030518,337.89410443)(717.7451773,340.3932724)(721.15472698,340.3932724)
\closepath
}
}
{
\newrgbcolor{curcolor}{0.85882354 0.85882354 0.85882354}
\pscustom[linestyle=none,fillstyle=solid,fillcolor=curcolor]
{
\newpath
\moveto(61.76780224,165.0684433)
\lineto(144.06808949,165.0684433)
\curveto(147.47763917,165.0684433)(150.22251129,162.56927533)(150.22251129,159.46492767)
\lineto(150.22251129,149.19231415)
\curveto(150.22251129,146.08796649)(147.47763917,143.58879852)(144.06808949,143.58879852)
\lineto(61.76780224,143.58879852)
\curveto(58.35825256,143.58879852)(55.61338043,146.08796649)(55.61338043,149.19231415)
\lineto(55.61338043,159.46492767)
\curveto(55.61338043,162.56927533)(58.35825256,165.0684433)(61.76780224,165.0684433)
\closepath
}
}
{
\newrgbcolor{curcolor}{0.85882354 0.85882354 0.85882354}
\pscustom[linestyle=none,fillstyle=solid,fillcolor=curcolor]
{
\newpath
\moveto(619.9467802,165.0684433)
\lineto(737.9397707,165.0684433)
\curveto(741.34932038,165.0684433)(744.0941925,162.56927533)(744.0941925,159.46492767)
\lineto(744.0941925,149.19231415)
\curveto(744.0941925,146.08796649)(741.34932038,143.58879852)(737.9397707,143.58879852)
\lineto(619.9467802,143.58879852)
\curveto(616.53723052,143.58879852)(613.7923584,146.08796649)(613.7923584,149.19231415)
\lineto(613.7923584,159.46492767)
\curveto(613.7923584,162.56927533)(616.53723052,165.0684433)(619.9467802,165.0684433)
\closepath
}
}
{
\newrgbcolor{curcolor}{0.85882354 0.85882354 0.85882354}
\pscustom[linestyle=none,fillstyle=solid,fillcolor=curcolor]
{
\newpath
\moveto(137.33729839,120.79561615)
\lineto(199.13322926,120.79561615)
\curveto(202.54277894,120.79561615)(205.28765106,118.29644818)(205.28765106,115.19210053)
\lineto(205.28765106,104.91947937)
\curveto(205.28765106,101.81513171)(202.54277894,99.31596375)(199.13322926,99.31596375)
\lineto(137.33729839,99.31596375)
\curveto(133.92774871,99.31596375)(131.18287659,101.81513171)(131.18287659,104.91947937)
\lineto(131.18287659,115.19210053)
\curveto(131.18287659,118.29644818)(133.92774871,120.79561615)(137.33729839,120.79561615)
\closepath
}
}
{
\newrgbcolor{curcolor}{0.85882354 0.85882354 0.85882354}
\pscustom[linestyle=none,fillstyle=solid,fillcolor=curcolor]
{
\newpath
\moveto(467.47336102,120.79561615)
\lineto(532.30697346,120.79561615)
\curveto(535.71652314,120.79561615)(538.46139526,118.29644818)(538.46139526,115.19210053)
\lineto(538.46139526,104.919487)
\curveto(538.46139526,101.81513934)(535.71652314,99.31597137)(532.30697346,99.31597137)
\lineto(467.47336102,99.31597137)
\curveto(464.06381133,99.31597137)(461.31893921,101.81513934)(461.31893921,104.919487)
\lineto(461.31893921,115.19210053)
\curveto(461.31893921,118.29644818)(464.06381133,120.79561615)(467.47336102,120.79561615)
\closepath
}
}
{
\newrgbcolor{curcolor}{0.85882354 0.85882354 0.85882354}
\pscustom[linestyle=none,fillstyle=solid,fillcolor=curcolor]
{
\newpath
\moveto(332.9136076,78.25911713)
\lineto(405.34141254,78.25911713)
\curveto(408.75096222,78.25911713)(411.49583435,75.75994916)(411.49583435,72.6556015)
\lineto(411.49583435,62.38298798)
\curveto(411.49583435,59.27864032)(408.75096222,56.77947235)(405.34141254,56.77947235)
\lineto(332.9136076,56.77947235)
\curveto(329.50405792,56.77947235)(326.75918579,59.27864032)(326.75918579,62.38298798)
\lineto(326.75918579,72.6556015)
\curveto(326.75918579,75.75994916)(329.50405792,78.25911713)(332.9136076,78.25911713)
\closepath
}
}
{
\newrgbcolor{curcolor}{0.85882354 0.85882354 0.85882354}
\pscustom[linestyle=none,fillstyle=solid,fillcolor=curcolor]
{
\newpath
\moveto(214.48927593,78.25911713)
\lineto(283.11998463,78.25911713)
\curveto(286.52953431,78.25911713)(289.27440643,75.75994916)(289.27440643,72.6556015)
\lineto(289.27440643,62.38298798)
\curveto(289.27440643,59.27864032)(286.52953431,56.77947235)(283.11998463,56.77947235)
\lineto(214.48927593,56.77947235)
\curveto(211.07972625,56.77947235)(208.33485413,59.27864032)(208.33485413,62.38298798)
\lineto(208.33485413,72.6556015)
\curveto(208.33485413,75.75994916)(211.07972625,78.25911713)(214.48927593,78.25911713)
\closepath
}
}
{
\newrgbcolor{curcolor}{0.85882354 0.85882354 0.85882354}
\pscustom[linestyle=none,fillstyle=solid,fillcolor=curcolor]
{
\newpath
\moveto(815.3920927,78.25911713)
\lineto(880.98512745,78.25911713)
\curveto(884.39467713,78.25911713)(887.13954926,75.75994916)(887.13954926,72.6556015)
\lineto(887.13954926,62.38298798)
\curveto(887.13954926,59.27864032)(884.39467713,56.77947235)(880.98512745,56.77947235)
\lineto(815.3920927,56.77947235)
\curveto(811.98254302,56.77947235)(809.2376709,59.27864032)(809.2376709,62.38298798)
\lineto(809.2376709,72.6556015)
\curveto(809.2376709,75.75994916)(811.98254302,78.25911713)(815.3920927,78.25911713)
\closepath
}
}
{
\newrgbcolor{curcolor}{0.85882354 0.85882354 0.85882354}
\pscustom[linestyle=none,fillstyle=solid,fillcolor=curcolor]
{
\newpath
\moveto(907.6375761,78.25911713)
\lineto(1007.4045229,78.25911713)
\curveto(1010.81407258,78.25911713)(1013.5589447,75.75994916)(1013.5589447,72.6556015)
\lineto(1013.5589447,62.38298798)
\curveto(1013.5589447,59.27864032)(1010.81407258,56.77947235)(1007.4045229,56.77947235)
\lineto(907.6375761,56.77947235)
\curveto(904.22802642,56.77947235)(901.4831543,59.27864032)(901.4831543,62.38298798)
\lineto(901.4831543,72.6556015)
\curveto(901.4831543,75.75994916)(904.22802642,78.25911713)(907.6375761,78.25911713)
\closepath
}
}
{
\newrgbcolor{curcolor}{0.85882354 0.85882354 0.85882354}
\pscustom[linestyle=none,fillstyle=solid,fillcolor=curcolor]
{
\newpath
\moveto(601.23108196,1261.12770844)
\lineto(637.20673084,1261.12770844)
\curveto(640.61628052,1261.12770844)(643.36115265,1258.62854047)(643.36115265,1255.52419281)
\lineto(643.36115265,1245.25157928)
\curveto(643.36115265,1242.14723163)(640.61628052,1239.64806366)(637.20673084,1239.64806366)
\lineto(601.23108196,1239.64806366)
\curveto(597.82153228,1239.64806366)(595.07666016,1242.14723163)(595.07666016,1245.25157928)
\lineto(595.07666016,1255.52419281)
\curveto(595.07666016,1258.62854047)(597.82153228,1261.12770844)(601.23108196,1261.12770844)
\closepath
}
}
{
\newrgbcolor{curcolor}{0.85882354 0.85882354 0.85882354}
\pscustom[linestyle=none,fillstyle=solid,fillcolor=curcolor]
{
\newpath
\moveto(601.23108196,1216.60884857)
\lineto(637.20673084,1216.60884857)
\curveto(640.61628052,1216.60884857)(643.36115265,1214.1096806)(643.36115265,1211.00533295)
\lineto(643.36115265,1200.73271942)
\curveto(643.36115265,1197.62837177)(640.61628052,1195.1292038)(637.20673084,1195.1292038)
\lineto(601.23108196,1195.1292038)
\curveto(597.82153228,1195.1292038)(595.07666016,1197.62837177)(595.07666016,1200.73271942)
\lineto(595.07666016,1211.00533295)
\curveto(595.07666016,1214.1096806)(597.82153228,1216.60884857)(601.23108196,1216.60884857)
\closepath
}
}
{
\newrgbcolor{curcolor}{0 0 0}
\pscustom[linestyle=none,fillstyle=solid,fillcolor=curcolor]
{
\newpath
\moveto(109.5206106,1691.60958604)
\lineto(110.60344888,1691.60958604)
\lineto(110.60344888,1686.88193642)
\curveto(110.60344155,1686.05957207)(110.5104143,1685.40652076)(110.32436685,1684.92278056)
\curveto(110.1383053,1684.43903735)(109.80247692,1684.04553208)(109.31688072,1683.74226356)
\curveto(108.83127242,1683.43899441)(108.19403575,1683.28735999)(107.4051688,1683.28735985)
\curveto(106.63862011,1683.28735999)(106.01161644,1683.41945868)(105.5241559,1683.68365633)
\curveto(105.03669085,1683.94785347)(104.68876894,1684.33019547)(104.4803891,1684.83068348)
\curveto(104.27200685,1685.33116869)(104.16781633,1686.01491899)(104.16781723,1686.88193642)
\lineto(104.16781723,1691.60958604)
\lineto(105.25065551,1691.60958604)
\lineto(105.25065551,1686.88751806)
\curveto(105.25065353,1686.1767864)(105.31670288,1685.65304298)(105.44880375,1685.31628622)
\curveto(105.58090027,1684.97952568)(105.80788676,1684.71997965)(106.12976391,1684.53764735)
\curveto(106.45163534,1684.35531283)(106.84514061,1684.26414612)(107.31028091,1684.26414696)
\curveto(108.10659013,1684.26414612)(108.67405637,1684.44461899)(109.0126813,1684.8055661)
\curveto(109.35129475,1685.16651046)(109.52060435,1685.86049375)(109.5206106,1686.88751806)
\closepath
}
}
{
\newrgbcolor{curcolor}{0 0 0}
\pscustom[linestyle=none,fillstyle=solid,fillcolor=curcolor]
{
\newpath
\moveto(112.28352259,1683.42690086)
\lineto(112.28352259,1689.35460322)
\lineto(113.18774837,1689.35460322)
\lineto(113.18774837,1688.51177549)
\curveto(113.62311425,1689.16296116)(114.25197847,1689.48855654)(115.07434291,1689.4885626)
\curveto(115.43156401,1689.48855654)(115.7599502,1689.42436773)(116.05950248,1689.295996)
\curveto(116.3590457,1689.16761252)(116.58324138,1688.9992332)(116.73209018,1688.79085752)
\curveto(116.88092858,1688.58247111)(116.9851191,1688.33501862)(117.04466205,1688.04849931)
\curveto(117.08186744,1687.86244019)(117.10047289,1687.53684481)(117.10047846,1687.0717122)
\lineto(117.10047846,1683.42690086)
\lineto(116.09578315,1683.42690086)
\lineto(116.09578315,1687.03264072)
\curveto(116.09577858,1687.44195701)(116.05670713,1687.74801667)(115.97856869,1687.9508206)
\curveto(115.90042135,1688.15361548)(115.76181075,1688.3154829)(115.56273647,1688.43642334)
\curveto(115.36365412,1688.55735375)(115.13015572,1688.61782147)(114.86224057,1688.61782666)
\curveto(114.43431188,1688.61782147)(114.06499369,1688.48200168)(113.7542849,1688.21036689)
\curveto(113.44357166,1687.93872253)(113.28821615,1687.42335156)(113.2882179,1686.66425243)
\lineto(113.2882179,1683.42690086)
\closepath
}
}
{
\newrgbcolor{curcolor}{0 0 0}
\pscustom[linestyle=none,fillstyle=solid,fillcolor=curcolor]
{
\newpath
\moveto(118.65217472,1690.45418643)
\lineto(118.65217472,1691.60958604)
\lineto(119.65687003,1691.60958604)
\lineto(119.65687003,1690.45418643)
\closepath
\moveto(118.65217472,1683.42690086)
\lineto(118.65217472,1689.35460322)
\lineto(119.65687003,1689.35460322)
\lineto(119.65687003,1683.42690086)
\closepath
}
}
{
\newrgbcolor{curcolor}{0 0 0}
\pscustom[linestyle=none,fillstyle=solid,fillcolor=curcolor]
{
\newpath
\moveto(125.05989804,1685.59815907)
\lineto(126.04784844,1685.46978134)
\curveto(125.93993122,1684.78881982)(125.66364028,1684.25577367)(125.2189748,1683.87064129)
\curveto(124.77429976,1683.48550803)(124.2282298,1683.29294162)(123.58076328,1683.29294149)
\curveto(122.76956251,1683.29294162)(122.11744148,1683.55806929)(121.62439823,1684.08832528)
\curveto(121.13135262,1684.61857995)(120.8848304,1685.37861259)(120.88483085,1686.36842548)
\curveto(120.8848304,1687.00845003)(120.99088147,1687.56847408)(121.20298436,1688.04849931)
\curveto(121.41508573,1688.52851531)(121.73789029,1688.88853077)(122.17139901,1689.12854678)
\curveto(122.60490427,1689.36855138)(123.07655243,1689.48855654)(123.58634492,1689.4885626)
\curveto(124.23009034,1689.48855654)(124.75662459,1689.32575885)(125.16594922,1689.00016904)
\curveto(125.57526439,1688.67456809)(125.83760124,1688.21222265)(125.95296055,1687.61313134)
\lineto(124.97617344,1687.46242705)
\curveto(124.88314165,1687.86057964)(124.71848341,1688.16012739)(124.48219824,1688.36107119)
\curveto(124.24590498,1688.56200512)(123.96031132,1688.66247455)(123.6254164,1688.66247978)
\curveto(123.11934497,1688.66247455)(122.70816452,1688.48107141)(122.39187382,1688.11826982)
\curveto(122.07557921,1687.75545885)(121.91743289,1687.18148071)(121.91743437,1686.39633368)
\curveto(121.91743289,1685.60001745)(122.06999758,1685.02138795)(122.3751289,1684.66044345)
\curveto(122.68025634,1684.29949648)(123.07841298,1684.11902361)(123.56959999,1684.1190243)
\curveto(123.96403241,1684.11902361)(124.29334888,1684.23995904)(124.55755039,1684.48183094)
\curveto(124.82174366,1684.72370074)(124.98919271,1685.09580975)(125.05989804,1685.59815907)
\closepath
}
}
{
\newrgbcolor{curcolor}{0 0 0}
\pscustom[linestyle=none,fillstyle=solid,fillcolor=curcolor]
{
\newpath
\moveto(126.50554392,1685.19628095)
\lineto(127.49907596,1685.35256688)
\curveto(127.55489096,1684.95440832)(127.71024647,1684.64927894)(127.96514295,1684.43717782)
\curveto(128.22003581,1684.22507468)(128.57633018,1684.11902361)(129.03402713,1684.1190243)
\curveto(129.49543942,1684.11902361)(129.8377797,1684.21298113)(130.06104901,1684.40089716)
\curveto(130.28431051,1684.58881123)(130.39594321,1684.80928581)(130.39594745,1685.06232157)
\curveto(130.39594321,1685.28930643)(130.29733432,1685.46791875)(130.1001205,1685.59815907)
\curveto(129.96243622,1685.68746306)(129.62009593,1685.80095631)(129.07309862,1685.93863915)
\curveto(128.33631987,1686.12469114)(127.82560026,1686.28562829)(127.54093826,1686.42145107)
\curveto(127.25627349,1686.55726786)(127.04045027,1686.74518291)(126.89346795,1686.98519677)
\curveto(126.74648415,1687.22520352)(126.67299262,1687.49033119)(126.67299314,1687.78058056)
\curveto(126.67299262,1688.0447736)(126.73346034,1688.28943527)(126.85439646,1688.51456631)
\curveto(126.97533119,1688.73968717)(127.13998942,1688.92667194)(127.34837166,1689.07552119)
\curveto(127.50465625,1689.19086933)(127.71768865,1689.28854795)(127.98746951,1689.36855732)
\curveto(128.25724671,1689.44855482)(128.54656146,1689.48855654)(128.85541463,1689.4885626)
\curveto(129.32054819,1689.48855654)(129.72893782,1689.42157691)(130.08058475,1689.28762353)
\curveto(130.43222383,1689.15365843)(130.69176986,1688.97225529)(130.85922362,1688.74341357)
\curveto(131.02666797,1688.51456122)(131.14202176,1688.20850156)(131.20528534,1687.82523369)
\lineto(130.22291659,1687.69127431)
\curveto(130.17825944,1687.99639943)(130.04895156,1688.23454919)(129.83499256,1688.40572431)
\curveto(129.62102621,1688.57688948)(129.31868764,1688.66247455)(128.92797596,1688.66247978)
\curveto(128.46655802,1688.66247455)(128.13724155,1688.5861922)(127.94002557,1688.43363252)
\curveto(127.74280601,1688.28106282)(127.64419712,1688.1024505)(127.64419861,1687.89779502)
\curveto(127.64419712,1687.76755239)(127.68512911,1687.65033806)(127.76699471,1687.54615166)
\curveto(127.84885708,1687.43823592)(127.97723468,1687.34892976)(128.15212791,1687.2782329)
\curveto(128.25259535,1687.24101815)(128.548422,1687.15543308)(129.03960877,1687.02147744)
\curveto(129.75033409,1686.83169825)(130.24616933,1686.67634274)(130.527116,1686.55541044)
\curveto(130.80805393,1686.43447189)(131.02852851,1686.25865038)(131.18854042,1686.0279454)
\curveto(131.34854226,1685.79723522)(131.42854569,1685.51071129)(131.42855097,1685.16837274)
\curveto(131.42854569,1684.8334729)(131.33086708,1684.51811052)(131.13551483,1684.22228465)
\curveto(130.94015262,1683.9264572)(130.65828005,1683.69761016)(130.28989628,1683.53574286)
\curveto(129.92150423,1683.37387533)(129.50474214,1683.29294162)(129.03960877,1683.29294149)
\curveto(128.26934025,1683.29294162)(127.6823383,1683.45294849)(127.27860115,1683.77296258)
\curveto(126.87486176,1684.09297598)(126.61717627,1684.56741496)(126.50554392,1685.19628095)
\closepath
}
}
{
\newrgbcolor{curcolor}{0 0 0}
\pscustom[linestyle=none,fillstyle=solid,fillcolor=curcolor]
{
\newpath
\moveto(137.7134775,1681.02121375)
\curveto(137.15903241,1681.72078108)(136.69017506,1682.53942089)(136.30690406,1683.47713563)
\curveto(135.92363052,1684.41485027)(135.73199438,1685.38605477)(135.73199507,1686.39075204)
\curveto(135.73199438,1687.27636851)(135.87525635,1688.12477704)(136.1617814,1688.93598017)
\curveto(136.49667838,1689.87741044)(137.0139099,1690.81512513)(137.7134775,1691.74912706)
\lineto(138.43350914,1691.74912706)
\curveto(137.98325385,1690.97513201)(137.68556665,1690.42255014)(137.54044664,1690.09137979)
\curveto(137.31345765,1689.5778627)(137.13484532,1689.04202573)(137.00460914,1688.48386728)
\curveto(136.8446003,1687.78801839)(136.76459687,1687.08845346)(136.76459859,1686.3851704)
\curveto(136.76459687,1684.59532314)(137.32089983,1682.80733937)(138.43350914,1681.02121375)
\closepath
}
}
{
\newrgbcolor{curcolor}{0 0 0}
\pscustom[linestyle=none,fillstyle=solid,fillcolor=curcolor]
{
\newpath
\moveto(139.73961143,1683.42690086)
\lineto(139.73961143,1691.60958604)
\lineto(142.8262587,1691.60958604)
\curveto(143.36953388,1691.60957786)(143.78443542,1691.58353023)(144.07096457,1691.53144307)
\curveto(144.47283708,1691.46445535)(144.80959572,1691.33700801)(145.08124152,1691.14910069)
\curveto(145.35287487,1690.96117792)(145.57148891,1690.6979108)(145.7370843,1690.35929854)
\curveto(145.90266592,1690.02067241)(145.98546018,1689.64856341)(145.9854673,1689.24297041)
\curveto(145.98546018,1688.54712076)(145.76405532,1687.95825826)(145.32125207,1687.47638115)
\curveto(144.87843589,1686.99449594)(144.07840153,1686.75355536)(142.92114659,1686.75355868)
\lineto(140.82244971,1686.75355868)
\lineto(140.82244971,1683.42690086)
\closepath
\moveto(140.82244971,1687.71918251)
\lineto(142.93789152,1687.71918251)
\curveto(143.63745236,1687.71917822)(144.13421788,1687.84941637)(144.42818957,1688.10989736)
\curveto(144.72215011,1688.37036898)(144.86913317,1688.73689635)(144.86913918,1689.20948057)
\curveto(144.86913317,1689.55181507)(144.78261782,1689.84485091)(144.60959289,1690.08858897)
\curveto(144.43655645,1690.3323137)(144.20863968,1690.49325085)(143.92584191,1690.57140088)
\curveto(143.74350343,1690.61976791)(143.40674478,1690.64395499)(142.91556495,1690.64396221)
\lineto(140.82244971,1690.64396221)
\closepath
}
}
{
\newrgbcolor{curcolor}{0 0 0}
\pscustom[linestyle=none,fillstyle=solid,fillcolor=curcolor]
{
\newpath
\moveto(147.3752964,1683.42690086)
\lineto(147.3752964,1691.60958604)
\lineto(150.19402492,1691.60958604)
\curveto(150.83032762,1691.60957786)(151.31592987,1691.57050641)(151.65083313,1691.49237159)
\curveto(152.11968531,1691.38445191)(152.51970249,1691.18909468)(152.85088586,1690.90629932)
\curveto(153.28252595,1690.54162502)(153.60533051,1690.07555849)(153.81930051,1689.50809834)
\curveto(154.03325587,1688.94062603)(154.1402372,1688.29222609)(154.14024485,1687.56289658)
\curveto(154.1402372,1686.9414704)(154.06767595,1686.39074908)(153.92256087,1685.91073095)
\curveto(153.77743093,1685.43070785)(153.59137642,1685.03348149)(153.3643968,1684.71905067)
\curveto(153.13740344,1684.40461727)(152.88902068,1684.15716478)(152.61924778,1683.97669247)
\curveto(152.34946262,1683.79621905)(152.02386725,1683.65946899)(151.64246067,1683.56644188)
\curveto(151.26104379,1683.47341449)(150.82288544,1683.42690086)(150.32798429,1683.42690086)
\closepath
\moveto(148.45813468,1684.39252469)
\lineto(150.2051882,1684.39252469)
\curveto(150.74474254,1684.39252373)(151.16801654,1684.44275844)(151.47501145,1684.54322899)
\curveto(151.78199639,1684.64369731)(152.02665806,1684.78509873)(152.20899719,1684.96743368)
\curveto(152.46574669,1685.22418735)(152.66575528,1685.56931845)(152.80902356,1686.00282802)
\curveto(152.95227921,1686.43633243)(153.02391019,1686.9619364)(153.02391672,1687.5796415)
\curveto(153.02391019,1688.43548805)(152.88343904,1689.09319072)(152.60250286,1689.55275146)
\curveto(152.32155445,1690.01229996)(151.98014444,1690.32022016)(151.5782718,1690.47651299)
\curveto(151.28802169,1690.58813864)(150.82102489,1690.64395499)(150.17728,1690.64396221)
\lineto(148.45813468,1690.64396221)
\closepath
}
}
{
\newrgbcolor{curcolor}{0 0 0}
\pscustom[linestyle=none,fillstyle=solid,fillcolor=curcolor]
{
\newpath
\moveto(155.63612632,1683.42690086)
\lineto(155.63612632,1691.60958604)
\lineto(158.7227736,1691.60958604)
\curveto(159.26604877,1691.60957786)(159.68095031,1691.58353023)(159.96747946,1691.53144307)
\curveto(160.36935197,1691.46445535)(160.70611062,1691.33700801)(160.97775641,1691.14910069)
\curveto(161.24938976,1690.96117792)(161.4680038,1690.6979108)(161.63359919,1690.35929854)
\curveto(161.79918082,1690.02067241)(161.88197507,1689.64856341)(161.8819822,1689.24297041)
\curveto(161.88197507,1688.54712076)(161.66057021,1687.95825826)(161.21776696,1687.47638115)
\curveto(160.77495078,1686.99449594)(159.97491642,1686.75355536)(158.81766149,1686.75355868)
\lineto(156.71896461,1686.75355868)
\lineto(156.71896461,1683.42690086)
\closepath
\moveto(156.71896461,1687.71918251)
\lineto(158.83440641,1687.71918251)
\curveto(159.53396726,1687.71917822)(160.03073278,1687.84941637)(160.32470446,1688.10989736)
\curveto(160.618665,1688.37036898)(160.76564806,1688.73689635)(160.76565407,1689.20948057)
\curveto(160.76564806,1689.55181507)(160.67913271,1689.84485091)(160.50610778,1690.08858897)
\curveto(160.33307134,1690.3323137)(160.10515458,1690.49325085)(159.8223568,1690.57140088)
\curveto(159.64001832,1690.61976791)(159.30325967,1690.64395499)(158.81207985,1690.64396221)
\lineto(156.71896461,1690.64396221)
\closepath
}
}
{
\newrgbcolor{curcolor}{0 0 0}
\pscustom[linestyle=none,fillstyle=solid,fillcolor=curcolor]
{
\newpath
\moveto(162.93133121,1690.53791104)
\lineto(162.93133121,1691.50353487)
\lineto(168.22830818,1691.50353487)
\lineto(168.22830818,1690.72210518)
\curveto(167.70734973,1690.16765547)(167.19104849,1689.43087964)(166.6794029,1688.51177549)
\curveto(166.16774873,1687.59266116)(165.77238291,1686.64750429)(165.49330426,1685.67630204)
\curveto(165.2923623,1684.99161922)(165.16398469,1684.24181958)(165.10817106,1683.42690086)
\lineto(164.07556754,1683.42690086)
\curveto(164.08672913,1684.07064944)(164.21324619,1684.84835726)(164.45511911,1685.76002665)
\curveto(164.69698789,1686.67169138)(165.04397954,1687.5507989)(165.49609508,1688.39735185)
\curveto(165.94820442,1689.24389487)(166.4291553,1689.95741388)(166.93894919,1690.53791104)
\closepath
}
}
{
\newrgbcolor{curcolor}{0 0 0}
\pscustom[linestyle=none,fillstyle=solid,fillcolor=curcolor]
{
\newpath
\moveto(170.16513859,1681.02121375)
\lineto(169.44510695,1681.02121375)
\curveto(170.55771218,1682.80733937)(171.11401514,1684.59532314)(171.1140175,1686.3851704)
\curveto(171.11401514,1687.08473237)(171.0340117,1687.77871566)(170.87400695,1688.46712236)
\curveto(170.74748777,1689.02528083)(170.57073599,1689.56111779)(170.34375109,1690.07463486)
\curveto(170.19862699,1690.40952632)(169.89907924,1690.96768983)(169.44510695,1691.74912706)
\lineto(170.16513859,1691.74912706)
\curveto(170.86470211,1690.81512513)(171.38193362,1689.87741044)(171.71683469,1688.93598017)
\curveto(172.00335566,1688.12477704)(172.14661763,1687.27636851)(172.14662102,1686.39075204)
\curveto(172.14661763,1685.38605477)(171.95405122,1684.41485027)(171.56892121,1683.47713563)
\curveto(171.18378558,1682.53942089)(170.71585851,1681.72078108)(170.16513859,1681.02121375)
\closepath
}
}
{
\newrgbcolor{curcolor}{0 0 0}
\pscustom[linestyle=none,fillstyle=solid,fillcolor=curcolor]
{
\newpath
\moveto(103.9445516,1595.39589691)
\lineto(100.77417972,1603.57858209)
\lineto(101.94632425,1603.57858209)
\lineto(104.07292934,1597.63413481)
\curveto(104.24409613,1597.15783304)(104.3873581,1596.71130224)(104.50271567,1596.29454105)
\curveto(104.62922895,1596.74107096)(104.77621201,1597.18760177)(104.94366528,1597.63413481)
\lineto(107.15399497,1603.57858209)
\lineto(108.25915982,1603.57858209)
\lineto(105.05529809,1595.39589691)
\closepath
}
}
{
\newrgbcolor{curcolor}{0 0 0}
\pscustom[linestyle=none,fillstyle=solid,fillcolor=curcolor]
{
\newpath
\moveto(113.17100368,1597.30481801)
\lineto(114.20918883,1597.17644028)
\curveto(114.04545502,1596.56990082)(113.74218618,1596.09918293)(113.29938141,1595.76428519)
\curveto(112.85656676,1595.42938672)(112.29096107,1595.26193767)(111.60256266,1595.26193754)
\curveto(110.73554543,1595.26193767)(110.04807405,1595.52892588)(109.54014644,1596.06290297)
\curveto(109.03221647,1596.59687872)(108.77825207,1597.34574809)(108.77825249,1598.30951333)
\curveto(108.77825207,1599.30676254)(109.03500729,1600.08074927)(109.5485189,1600.63147583)
\curveto(110.06202814,1601.18219192)(110.72810325,1601.45755258)(111.54674625,1601.45755865)
\curveto(112.33933524,1601.45755258)(112.98680491,1601.18777356)(113.48915719,1600.64822075)
\curveto(113.99149922,1600.10865744)(114.24267279,1599.34955508)(114.24267868,1598.37091137)
\curveto(114.24267279,1598.31137096)(114.24081225,1598.2220648)(114.23709704,1598.10299262)
\lineto(109.81643765,1598.10299262)
\curveto(109.85364709,1597.45179916)(110.03784105,1596.95317309)(110.36902007,1596.60711293)
\curveto(110.70019508,1596.26105035)(111.11323607,1596.08801966)(111.6081443,1596.08802035)
\curveto(111.97652896,1596.08801966)(112.29096107,1596.184768)(112.55144156,1596.37826566)
\curveto(112.81191368,1596.57176136)(113.01843417,1596.88061184)(113.17100368,1597.30481801)
\closepath
\moveto(109.87225406,1598.92907544)
\lineto(113.18216696,1598.92907544)
\curveto(113.13750905,1599.42769797)(113.01099199,1599.80166752)(112.80261539,1600.0509852)
\curveto(112.48259721,1600.43797391)(112.06769567,1600.6314706)(111.55790953,1600.63147583)
\curveto(111.09649117,1600.6314706)(110.70856753,1600.47704536)(110.39413746,1600.16819966)
\curveto(110.07970331,1599.85934441)(109.90574236,1599.44630342)(109.87225406,1598.92907544)
\closepath
}
}
{
\newrgbcolor{curcolor}{0 0 0}
\pscustom[linestyle=none,fillstyle=solid,fillcolor=curcolor]
{
\newpath
\moveto(115.46505766,1595.39589691)
\lineto(115.46505766,1601.32359927)
\lineto(116.36928344,1601.32359927)
\lineto(116.36928344,1600.42495513)
\curveto(116.59998938,1600.84543327)(116.81302178,1601.12265448)(117.0083813,1601.25661958)
\curveto(117.20373624,1601.39057296)(117.41862919,1601.45755258)(117.65306079,1601.45755865)
\curveto(117.99167705,1601.45755258)(118.33587788,1601.34964097)(118.68566431,1601.13382349)
\lineto(118.33960259,1600.2016895)
\curveto(118.09400703,1600.34680721)(117.84841509,1600.41936846)(117.60282603,1600.41937349)
\curveto(117.38327883,1600.41936846)(117.18606106,1600.35331911)(117.01117212,1600.22122524)
\curveto(116.8362786,1600.08912172)(116.71162208,1599.90585804)(116.63720219,1599.67143364)
\curveto(116.52556758,1599.31420472)(116.46975123,1598.92349027)(116.46975298,1598.49928911)
\lineto(116.46975298,1595.39589691)
\closepath
}
}
{
\newrgbcolor{curcolor}{0 0 0}
\pscustom[linestyle=none,fillstyle=solid,fillcolor=curcolor]
{
\newpath
\moveto(118.8921853,1597.16527699)
\lineto(119.88571733,1597.32156293)
\curveto(119.94153234,1596.92340437)(120.09688785,1596.61827499)(120.35178433,1596.40617387)
\curveto(120.60667718,1596.19407073)(120.96297155,1596.08801966)(121.42066851,1596.08802035)
\curveto(121.88208079,1596.08801966)(122.22442108,1596.18197718)(122.44769039,1596.3698932)
\curveto(122.67095188,1596.55780728)(122.78258458,1596.77828186)(122.78258882,1597.03131762)
\curveto(122.78258458,1597.25830248)(122.6839757,1597.4369148)(122.48676187,1597.56715512)
\curveto(122.34907759,1597.65645911)(122.00673731,1597.76995236)(121.45973999,1597.9076352)
\curveto(120.72296125,1598.09368719)(120.21224164,1598.25462433)(119.92757964,1598.39044711)
\curveto(119.64291486,1598.52626391)(119.42709164,1598.71417895)(119.28010932,1598.95419282)
\curveto(119.13312553,1599.19419957)(119.059634,1599.45932723)(119.05963452,1599.74957661)
\curveto(119.059634,1600.01376965)(119.12010171,1600.25843132)(119.24103784,1600.48356235)
\curveto(119.36197257,1600.70868321)(119.5266308,1600.89566799)(119.73501304,1601.04451724)
\curveto(119.89129762,1601.15986538)(120.10433003,1601.25754399)(120.37411089,1601.33755337)
\curveto(120.64388808,1601.41755087)(120.93320283,1601.45755258)(121.24205601,1601.45755865)
\curveto(121.70718956,1601.45755258)(122.11557919,1601.39057296)(122.46722613,1601.25661958)
\curveto(122.81886521,1601.12265448)(123.07841124,1600.94125134)(123.245865,1600.71240962)
\curveto(123.41330934,1600.48355727)(123.52866314,1600.17749761)(123.59192672,1599.79422973)
\lineto(122.60955796,1599.66027036)
\curveto(122.56490082,1599.96539548)(122.43559294,1600.20354524)(122.22163394,1600.37472036)
\curveto(122.00766758,1600.54588552)(121.70532902,1600.6314706)(121.31461734,1600.63147583)
\curveto(120.8531994,1600.6314706)(120.52388293,1600.55518825)(120.32666694,1600.40262856)
\curveto(120.12944739,1600.25005887)(120.0308385,1600.07144654)(120.03083999,1599.86679106)
\curveto(120.0308385,1599.73654844)(120.07177049,1599.6193341)(120.15363608,1599.5151477)
\curveto(120.23549845,1599.40723197)(120.36387606,1599.31792581)(120.53876929,1599.24722895)
\curveto(120.63923672,1599.2100142)(120.93506338,1599.12442913)(121.42625015,1598.99047348)
\curveto(122.13697546,1598.8006943)(122.63281071,1598.64533879)(122.91375738,1598.52440649)
\curveto(123.1946953,1598.40346793)(123.41516989,1598.22764643)(123.5751818,1597.99694145)
\curveto(123.73518363,1597.76623127)(123.81518707,1597.47970733)(123.81519234,1597.13736879)
\curveto(123.81518707,1596.80246895)(123.71750845,1596.48710657)(123.52215621,1596.1912807)
\curveto(123.326794,1595.89545325)(123.04492143,1595.66660621)(122.67653765,1595.5047389)
\curveto(122.3081456,1595.34287138)(121.89138352,1595.26193767)(121.42625015,1595.26193754)
\curveto(120.65598163,1595.26193767)(120.06897967,1595.42194454)(119.66524253,1595.74195863)
\curveto(119.26150313,1596.06197203)(119.00381765,1596.53641101)(118.8921853,1597.16527699)
\closepath
}
}
{
\newrgbcolor{curcolor}{0 0 0}
\pscustom[linestyle=none,fillstyle=solid,fillcolor=curcolor]
{
\newpath
\moveto(125.01524603,1602.42318248)
\lineto(125.01524603,1603.57858209)
\lineto(126.01994135,1603.57858209)
\lineto(126.01994135,1602.42318248)
\closepath
\moveto(125.01524603,1595.39589691)
\lineto(125.01524603,1601.32359927)
\lineto(126.01994135,1601.32359927)
\lineto(126.01994135,1595.39589691)
\closepath
}
}
{
\newrgbcolor{curcolor}{0 0 0}
\pscustom[linestyle=none,fillstyle=solid,fillcolor=curcolor]
{
\newpath
\moveto(127.18092152,1598.35974809)
\curveto(127.18092114,1599.45746669)(127.48605052,1600.27052486)(128.09631059,1600.79892505)
\curveto(128.60609863,1601.23800827)(129.22752066,1601.45755258)(129.96057856,1601.45755865)
\curveto(130.77549412,1601.45755258)(131.44156923,1601.19056437)(131.95880591,1600.65659321)
\curveto(132.47603227,1600.12261153)(132.73464802,1599.38490543)(132.73465396,1598.4434727)
\curveto(132.73464802,1597.6806462)(132.6202245,1597.08062043)(132.39138306,1596.64339359)
\curveto(132.16253043,1596.20616427)(131.82949287,1595.8666148)(131.39226938,1595.62474418)
\curveto(130.95503671,1595.3828731)(130.47780692,1595.26193767)(129.96057856,1595.26193754)
\curveto(129.13077232,1595.26193767)(128.46004584,1595.52799561)(127.94839711,1596.06011215)
\curveto(127.43674608,1596.59222736)(127.18092114,1597.35877191)(127.18092152,1598.35974809)
\closepath
\moveto(128.21352504,1598.35974809)
\curveto(128.21352363,1597.60064276)(128.37911213,1597.03224626)(128.71029106,1596.65455688)
\curveto(129.04146616,1596.27686498)(129.45822824,1596.08801966)(129.96057856,1596.08802035)
\curveto(130.45920146,1596.08801966)(130.874103,1596.27779525)(131.20528442,1596.6573477)
\curveto(131.53645703,1597.03689762)(131.70204554,1597.61552712)(131.70205044,1598.39323793)
\curveto(131.70204554,1599.12628967)(131.53552676,1599.68166236)(131.2024936,1600.05935767)
\curveto(130.86945164,1600.43704364)(130.45548037,1600.62588896)(129.96057856,1600.62589419)
\curveto(129.45822824,1600.62588896)(129.04146616,1600.43797391)(128.71029106,1600.06214849)
\curveto(128.37911213,1599.68631373)(128.21352363,1599.11884749)(128.21352504,1598.35974809)
\closepath
}
}
{
\newrgbcolor{curcolor}{0 0 0}
\pscustom[linestyle=none,fillstyle=solid,fillcolor=curcolor]
{
\newpath
\moveto(133.91796098,1595.39589691)
\lineto(133.91796098,1601.32359927)
\lineto(134.82218676,1601.32359927)
\lineto(134.82218676,1600.48077153)
\curveto(135.25755264,1601.13195721)(135.88641685,1601.45755258)(136.7087813,1601.45755865)
\curveto(137.0660024,1601.45755258)(137.39438859,1601.39336378)(137.69394087,1601.26499204)
\curveto(137.99348409,1601.13660857)(138.21767976,1600.96822924)(138.36652857,1600.75985357)
\curveto(138.51536697,1600.55146716)(138.61955749,1600.30401467)(138.67910044,1600.01749536)
\curveto(138.71630583,1599.83143624)(138.73491128,1599.50584086)(138.73491685,1599.04070825)
\lineto(138.73491685,1595.39589691)
\lineto(137.73022153,1595.39589691)
\lineto(137.73022153,1599.00163676)
\curveto(137.73021697,1599.41095306)(137.69114552,1599.71701272)(137.61300708,1599.91981665)
\curveto(137.53485974,1600.12261153)(137.39624914,1600.28447895)(137.19717485,1600.40541938)
\curveto(136.9980925,1600.5263498)(136.7645941,1600.58681751)(136.49667895,1600.58682271)
\curveto(136.06875027,1600.58681751)(135.69943208,1600.45099773)(135.38872329,1600.17936294)
\curveto(135.07801004,1599.90771858)(134.92265453,1599.39234761)(134.92265629,1598.63324848)
\lineto(134.92265629,1595.39589691)
\closepath
}
}
{
\newrgbcolor{curcolor}{0 0 0}
\pscustom[linestyle=none,fillstyle=solid,fillcolor=curcolor]
{
\newpath
\moveto(146.95667475,1595.39589691)
\lineto(145.95197944,1595.39589691)
\lineto(145.95197944,1601.79803872)
\curveto(145.71010533,1601.56732474)(145.3928824,1601.33661716)(145.00030971,1601.10591529)
\curveto(144.6077324,1600.87520199)(144.25515912,1600.70217131)(143.94258881,1600.58682271)
\lineto(143.94258881,1601.55802818)
\curveto(144.50447216,1601.82221941)(144.99565604,1602.14223315)(145.41614193,1602.51807037)
\curveto(145.83662239,1602.89389334)(146.13430959,1603.25856016)(146.30920444,1603.61207193)
\lineto(146.95667475,1603.61207193)
\closepath
}
}
{
\newrgbcolor{curcolor}{0 0 0}
\pscustom[linestyle=none,fillstyle=solid,fillcolor=curcolor]
{
\newpath
\moveto(158.48276391,1603.57858209)
\lineto(159.56560219,1603.57858209)
\lineto(159.56560219,1598.85093247)
\curveto(159.56559486,1598.02856811)(159.47256761,1597.37551681)(159.28652016,1596.8917766)
\curveto(159.1004586,1596.4080334)(158.76463023,1596.01452813)(158.27903403,1595.71125961)
\curveto(157.79342573,1595.40799045)(157.15618906,1595.25635604)(156.36732211,1595.2563559)
\curveto(155.60077342,1595.25635604)(154.97376975,1595.38845473)(154.48630921,1595.65265238)
\curveto(153.99884416,1595.91684952)(153.65092224,1596.29919152)(153.44254241,1596.79967953)
\curveto(153.23416016,1597.30016474)(153.12996964,1597.98391503)(153.12997054,1598.85093247)
\lineto(153.12997054,1603.57858209)
\lineto(154.21280882,1603.57858209)
\lineto(154.21280882,1598.85651411)
\curveto(154.21280684,1598.14578245)(154.27885619,1597.62203903)(154.41095706,1597.28528227)
\curveto(154.54305358,1596.94852173)(154.77004007,1596.6889757)(155.09191722,1596.5066434)
\curveto(155.41378865,1596.32430888)(155.80729392,1596.23314217)(156.27243422,1596.23314301)
\curveto(157.06874344,1596.23314217)(157.63620967,1596.41361504)(157.97483461,1596.77456215)
\curveto(158.31344806,1597.1355065)(158.48275766,1597.8294898)(158.48276391,1598.85651411)
\closepath
}
}
{
\newrgbcolor{curcolor}{0 0 0}
\pscustom[linestyle=none,fillstyle=solid,fillcolor=curcolor]
{
\newpath
\moveto(161.24567399,1595.39589691)
\lineto(161.24567399,1601.32359927)
\lineto(162.14989977,1601.32359927)
\lineto(162.14989977,1600.48077153)
\curveto(162.58526565,1601.13195721)(163.21412987,1601.45755258)(164.03649431,1601.45755865)
\curveto(164.39371541,1601.45755258)(164.72210161,1601.39336378)(165.02165388,1601.26499204)
\curveto(165.3211971,1601.13660857)(165.54539278,1600.96822924)(165.69424158,1600.75985357)
\curveto(165.84307998,1600.55146716)(165.9472705,1600.30401467)(166.00681346,1600.01749536)
\curveto(166.04401884,1599.83143624)(166.06262429,1599.50584086)(166.06262986,1599.04070825)
\lineto(166.06262986,1595.39589691)
\lineto(165.05793455,1595.39589691)
\lineto(165.05793455,1599.00163676)
\curveto(165.05792998,1599.41095306)(165.01885854,1599.71701272)(164.94072009,1599.91981665)
\curveto(164.86257275,1600.12261153)(164.72396215,1600.28447895)(164.52488787,1600.40541938)
\curveto(164.32580552,1600.5263498)(164.09230712,1600.58681751)(163.82439197,1600.58682271)
\curveto(163.39646328,1600.58681751)(163.02714509,1600.45099773)(162.7164363,1600.17936294)
\curveto(162.40572306,1599.90771858)(162.25036755,1599.39234761)(162.25036931,1598.63324848)
\lineto(162.25036931,1595.39589691)
\closepath
}
}
{
\newrgbcolor{curcolor}{0 0 0}
\pscustom[linestyle=none,fillstyle=solid,fillcolor=curcolor]
{
\newpath
\moveto(167.61432707,1602.42318248)
\lineto(167.61432707,1603.57858209)
\lineto(168.61902239,1603.57858209)
\lineto(168.61902239,1602.42318248)
\closepath
\moveto(167.61432707,1595.39589691)
\lineto(167.61432707,1601.32359927)
\lineto(168.61902239,1601.32359927)
\lineto(168.61902239,1595.39589691)
\closepath
}
}
{
\newrgbcolor{curcolor}{0 0 0}
\pscustom[linestyle=none,fillstyle=solid,fillcolor=curcolor]
{
\newpath
\moveto(169.48417561,1595.39589691)
\lineto(171.64985217,1598.47696254)
\lineto(169.64604319,1601.32359927)
\lineto(170.90191233,1601.32359927)
\lineto(171.81171975,1599.93377075)
\curveto(171.98288748,1599.66956882)(172.12056781,1599.44816396)(172.22476116,1599.26955551)
\curveto(172.3884863,1599.51514358)(172.53919044,1599.73282735)(172.67687405,1599.92260747)
\lineto(173.67598773,1601.32359927)
\lineto(174.87604046,1601.32359927)
\lineto(172.82757835,1598.53277895)
\lineto(175.0323264,1595.39589691)
\lineto(173.79878382,1595.39589691)
\lineto(172.58198616,1597.23783832)
\lineto(172.258251,1597.73460434)
\lineto(170.70097327,1595.39589691)
\closepath
}
}
{
\newrgbcolor{curcolor}{0 0 0}
\pscustom[linestyle=none,fillstyle=solid,fillcolor=curcolor]
{
\newpath
\moveto(103.9445516,1550.34990692)
\lineto(100.77417972,1558.5325921)
\lineto(101.94632425,1558.5325921)
\lineto(104.07292934,1552.58814482)
\curveto(104.24409613,1552.11184305)(104.3873581,1551.66531225)(104.50271567,1551.24855106)
\curveto(104.62922895,1551.69508097)(104.77621201,1552.14161177)(104.94366528,1552.58814482)
\lineto(107.15399497,1558.5325921)
\lineto(108.25915982,1558.5325921)
\lineto(105.05529809,1550.34990692)
\closepath
}
}
{
\newrgbcolor{curcolor}{0 0 0}
\pscustom[linestyle=none,fillstyle=solid,fillcolor=curcolor]
{
\newpath
\moveto(113.17100368,1552.25882802)
\lineto(114.20918883,1552.13045029)
\curveto(114.04545502,1551.52391083)(113.74218618,1551.05319294)(113.29938141,1550.7182952)
\curveto(112.85656676,1550.38339673)(112.29096107,1550.21594768)(111.60256266,1550.21594755)
\curveto(110.73554543,1550.21594768)(110.04807405,1550.48293589)(109.54014644,1551.01691298)
\curveto(109.03221647,1551.55088873)(108.77825207,1552.2997581)(108.77825249,1553.26352333)
\curveto(108.77825207,1554.26077255)(109.03500729,1555.03475928)(109.5485189,1555.58548584)
\curveto(110.06202814,1556.13620193)(110.72810325,1556.41156259)(111.54674625,1556.41156865)
\curveto(112.33933524,1556.41156259)(112.98680491,1556.14178357)(113.48915719,1555.60223076)
\curveto(113.99149922,1555.06266745)(114.24267279,1554.30356509)(114.24267868,1553.32492138)
\curveto(114.24267279,1553.26538097)(114.24081225,1553.17607481)(114.23709704,1553.05700263)
\lineto(109.81643765,1553.05700263)
\curveto(109.85364709,1552.40580917)(110.03784105,1551.9071831)(110.36902007,1551.56112294)
\curveto(110.70019508,1551.21506036)(111.11323607,1551.04202967)(111.6081443,1551.04203036)
\curveto(111.97652896,1551.04202967)(112.29096107,1551.13877801)(112.55144156,1551.33227567)
\curveto(112.81191368,1551.52577137)(113.01843417,1551.83462185)(113.17100368,1552.25882802)
\closepath
\moveto(109.87225406,1553.88308545)
\lineto(113.18216696,1553.88308545)
\curveto(113.13750905,1554.38170798)(113.01099199,1554.75567753)(112.80261539,1555.00499521)
\curveto(112.48259721,1555.39198392)(112.06769567,1555.5854806)(111.55790953,1555.58548584)
\curveto(111.09649117,1555.5854806)(110.70856753,1555.43105537)(110.39413746,1555.12220967)
\curveto(110.07970331,1554.81335442)(109.90574236,1554.40031343)(109.87225406,1553.88308545)
\closepath
}
}
{
\newrgbcolor{curcolor}{0 0 0}
\pscustom[linestyle=none,fillstyle=solid,fillcolor=curcolor]
{
\newpath
\moveto(115.46505766,1550.34990692)
\lineto(115.46505766,1556.27760928)
\lineto(116.36928344,1556.27760928)
\lineto(116.36928344,1555.37896514)
\curveto(116.59998938,1555.79944328)(116.81302178,1556.07666449)(117.0083813,1556.21062959)
\curveto(117.20373624,1556.34458297)(117.41862919,1556.41156259)(117.65306079,1556.41156865)
\curveto(117.99167705,1556.41156259)(118.33587788,1556.30365098)(118.68566431,1556.0878335)
\lineto(118.33960259,1555.15569951)
\curveto(118.09400703,1555.30081722)(117.84841509,1555.37337847)(117.60282603,1555.3733835)
\curveto(117.38327883,1555.37337847)(117.18606106,1555.30732912)(117.01117212,1555.17523525)
\curveto(116.8362786,1555.04313173)(116.71162208,1554.85986805)(116.63720219,1554.62544365)
\curveto(116.52556758,1554.26821473)(116.46975123,1553.87750028)(116.46975298,1553.45329912)
\lineto(116.46975298,1550.34990692)
\closepath
}
}
{
\newrgbcolor{curcolor}{0 0 0}
\pscustom[linestyle=none,fillstyle=solid,fillcolor=curcolor]
{
\newpath
\moveto(118.8921853,1552.119287)
\lineto(119.88571733,1552.27557294)
\curveto(119.94153234,1551.87741438)(120.09688785,1551.572285)(120.35178433,1551.36018388)
\curveto(120.60667718,1551.14808073)(120.96297155,1551.04202967)(121.42066851,1551.04203036)
\curveto(121.88208079,1551.04202967)(122.22442108,1551.13598719)(122.44769039,1551.32390321)
\curveto(122.67095188,1551.51181729)(122.78258458,1551.73229187)(122.78258882,1551.98532763)
\curveto(122.78258458,1552.21231249)(122.6839757,1552.39092481)(122.48676187,1552.52116513)
\curveto(122.34907759,1552.61046912)(122.00673731,1552.72396237)(121.45973999,1552.86164521)
\curveto(120.72296125,1553.0476972)(120.21224164,1553.20863434)(119.92757964,1553.34445712)
\curveto(119.64291486,1553.48027392)(119.42709164,1553.66818896)(119.28010932,1553.90820283)
\curveto(119.13312553,1554.14820958)(119.059634,1554.41333724)(119.05963452,1554.70358662)
\curveto(119.059634,1554.96777966)(119.12010171,1555.21244133)(119.24103784,1555.43757236)
\curveto(119.36197257,1555.66269322)(119.5266308,1555.849678)(119.73501304,1555.99852725)
\curveto(119.89129762,1556.11387539)(120.10433003,1556.211554)(120.37411089,1556.29156338)
\curveto(120.64388808,1556.37156088)(120.93320283,1556.41156259)(121.24205601,1556.41156865)
\curveto(121.70718956,1556.41156259)(122.11557919,1556.34458297)(122.46722613,1556.21062959)
\curveto(122.81886521,1556.07666449)(123.07841124,1555.89526135)(123.245865,1555.66641963)
\curveto(123.41330934,1555.43756728)(123.52866314,1555.13150762)(123.59192672,1554.74823974)
\lineto(122.60955796,1554.61428037)
\curveto(122.56490082,1554.91940549)(122.43559294,1555.15755525)(122.22163394,1555.32873037)
\curveto(122.00766758,1555.49989553)(121.70532902,1555.5854806)(121.31461734,1555.58548584)
\curveto(120.8531994,1555.5854806)(120.52388293,1555.50919826)(120.32666694,1555.35663857)
\curveto(120.12944739,1555.20406888)(120.0308385,1555.02545655)(120.03083999,1554.82080107)
\curveto(120.0308385,1554.69055845)(120.07177049,1554.57334411)(120.15363608,1554.46915771)
\curveto(120.23549845,1554.36124198)(120.36387606,1554.27193582)(120.53876929,1554.20123896)
\curveto(120.63923672,1554.16402421)(120.93506338,1554.07843914)(121.42625015,1553.94448349)
\curveto(122.13697546,1553.75470431)(122.63281071,1553.5993488)(122.91375738,1553.4784165)
\curveto(123.1946953,1553.35747794)(123.41516989,1553.18165644)(123.5751818,1552.95095146)
\curveto(123.73518363,1552.72024128)(123.81518707,1552.43371734)(123.81519234,1552.0913788)
\curveto(123.81518707,1551.75647896)(123.71750845,1551.44111658)(123.52215621,1551.14529071)
\curveto(123.326794,1550.84946326)(123.04492143,1550.62061622)(122.67653765,1550.45874891)
\curveto(122.3081456,1550.29688139)(121.89138352,1550.21594768)(121.42625015,1550.21594755)
\curveto(120.65598163,1550.21594768)(120.06897967,1550.37595455)(119.66524253,1550.69596864)
\curveto(119.26150313,1551.01598204)(119.00381765,1551.49042102)(118.8921853,1552.119287)
\closepath
}
}
{
\newrgbcolor{curcolor}{0 0 0}
\pscustom[linestyle=none,fillstyle=solid,fillcolor=curcolor]
{
\newpath
\moveto(125.01524603,1557.37719249)
\lineto(125.01524603,1558.5325921)
\lineto(126.01994135,1558.5325921)
\lineto(126.01994135,1557.37719249)
\closepath
\moveto(125.01524603,1550.34990692)
\lineto(125.01524603,1556.27760928)
\lineto(126.01994135,1556.27760928)
\lineto(126.01994135,1550.34990692)
\closepath
}
}
{
\newrgbcolor{curcolor}{0 0 0}
\pscustom[linestyle=none,fillstyle=solid,fillcolor=curcolor]
{
\newpath
\moveto(127.18092152,1553.3137581)
\curveto(127.18092114,1554.4114767)(127.48605052,1555.22453487)(128.09631059,1555.75293506)
\curveto(128.60609863,1556.19201828)(129.22752066,1556.41156259)(129.96057856,1556.41156865)
\curveto(130.77549412,1556.41156259)(131.44156923,1556.14457438)(131.95880591,1555.61060322)
\curveto(132.47603227,1555.07662154)(132.73464802,1554.33891544)(132.73465396,1553.39748271)
\curveto(132.73464802,1552.6346562)(132.6202245,1552.03463044)(132.39138306,1551.5974036)
\curveto(132.16253043,1551.16017428)(131.82949287,1550.82062481)(131.39226938,1550.57875419)
\curveto(130.95503671,1550.33688311)(130.47780692,1550.21594768)(129.96057856,1550.21594755)
\curveto(129.13077232,1550.21594768)(128.46004584,1550.48200562)(127.94839711,1551.01412216)
\curveto(127.43674608,1551.54623737)(127.18092114,1552.31278192)(127.18092152,1553.3137581)
\closepath
\moveto(128.21352504,1553.3137581)
\curveto(128.21352363,1552.55465277)(128.37911213,1551.98625627)(128.71029106,1551.60856689)
\curveto(129.04146616,1551.23087499)(129.45822824,1551.04202967)(129.96057856,1551.04203036)
\curveto(130.45920146,1551.04202967)(130.874103,1551.23180526)(131.20528442,1551.61135771)
\curveto(131.53645703,1551.99090763)(131.70204554,1552.56953713)(131.70205044,1553.34724794)
\curveto(131.70204554,1554.08029968)(131.53552676,1554.63567237)(131.2024936,1555.01336768)
\curveto(130.86945164,1555.39105365)(130.45548037,1555.57989897)(129.96057856,1555.5799042)
\curveto(129.45822824,1555.57989897)(129.04146616,1555.39198392)(128.71029106,1555.0161585)
\curveto(128.37911213,1554.64032374)(128.21352363,1554.0728575)(128.21352504,1553.3137581)
\closepath
}
}
{
\newrgbcolor{curcolor}{0 0 0}
\pscustom[linestyle=none,fillstyle=solid,fillcolor=curcolor]
{
\newpath
\moveto(133.91796098,1550.34990692)
\lineto(133.91796098,1556.27760928)
\lineto(134.82218676,1556.27760928)
\lineto(134.82218676,1555.43478154)
\curveto(135.25755264,1556.08596722)(135.88641685,1556.41156259)(136.7087813,1556.41156865)
\curveto(137.0660024,1556.41156259)(137.39438859,1556.34737379)(137.69394087,1556.21900205)
\curveto(137.99348409,1556.09061858)(138.21767976,1555.92223925)(138.36652857,1555.71386358)
\curveto(138.51536697,1555.50547717)(138.61955749,1555.25802468)(138.67910044,1554.97150537)
\curveto(138.71630583,1554.78544625)(138.73491128,1554.45985087)(138.73491685,1553.99471826)
\lineto(138.73491685,1550.34990692)
\lineto(137.73022153,1550.34990692)
\lineto(137.73022153,1553.95564677)
\curveto(137.73021697,1554.36496307)(137.69114552,1554.67102273)(137.61300708,1554.87382666)
\curveto(137.53485974,1555.07662154)(137.39624914,1555.23848896)(137.19717485,1555.35942939)
\curveto(136.9980925,1555.48035981)(136.7645941,1555.54082752)(136.49667895,1555.54083272)
\curveto(136.06875027,1555.54082752)(135.69943208,1555.40500774)(135.38872329,1555.13337295)
\curveto(135.07801004,1554.86172859)(134.92265453,1554.34635762)(134.92265629,1553.58725849)
\lineto(134.92265629,1550.34990692)
\closepath
}
}
{
\newrgbcolor{curcolor}{0 0 0}
\pscustom[linestyle=none,fillstyle=solid,fillcolor=curcolor]
{
\newpath
\moveto(148.45255444,1551.31553075)
\lineto(148.45255444,1550.34990692)
\lineto(143.04394466,1550.34990692)
\curveto(143.03650214,1550.59177777)(143.07557358,1550.8243459)(143.16115912,1551.047612)
\curveto(143.29883898,1551.41599922)(143.51931357,1551.7788055)(143.82258353,1552.13603193)
\curveto(144.12585124,1552.49325478)(144.5640096,1552.90629578)(145.1370599,1553.37515615)
\curveto(146.02639798,1554.10448677)(146.62735402,1554.682186)(146.93992983,1555.10825557)
\curveto(147.25249715,1555.53431562)(147.40878293,1555.93712361)(147.40878764,1556.31668076)
\curveto(147.40878293,1556.71483143)(147.26645124,1557.05065981)(146.98179213,1557.3241669)
\curveto(146.69712446,1557.59766004)(146.32594573,1557.7344101)(145.86825483,1557.73441749)
\curveto(145.38450995,1557.7344101)(144.99751659,1557.58928759)(144.70727357,1557.29904952)
\curveto(144.41702654,1557.00879754)(144.27004348,1556.60691982)(144.26632396,1556.09341514)
\lineto(143.23372044,1556.19946631)
\curveto(143.30442062,1556.9697261)(143.57047856,1557.55672805)(144.03189506,1557.96047393)
\curveto(144.49330889,1558.36420459)(145.11287038,1558.56607373)(145.89058139,1558.56608194)
\curveto(146.67572819,1558.56607373)(147.29715023,1558.34838996)(147.75484936,1557.91302999)
\curveto(148.21253838,1557.47765489)(148.44138542,1556.93809683)(148.44139116,1556.2943542)
\curveto(148.44138542,1555.96689233)(148.3744058,1555.64501805)(148.2404521,1555.32873037)
\curveto(148.10648731,1555.01243274)(147.88415218,1554.67939518)(147.57344604,1554.3296167)
\curveto(147.26273015,1553.97983025)(146.7464289,1553.49980964)(146.02454076,1552.88955341)
\curveto(145.42172085,1552.38348263)(145.03472749,1552.04021207)(144.86355951,1551.85974071)
\curveto(144.6923872,1551.67926634)(144.55098578,1551.4978632)(144.43935482,1551.31553075)
\closepath
}
}
{
\newrgbcolor{curcolor}{0 0 0}
\pscustom[linestyle=none,fillstyle=solid,fillcolor=curcolor]
{
\newpath
\moveto(158.48276391,1558.5325921)
\lineto(159.56560219,1558.5325921)
\lineto(159.56560219,1553.80494248)
\curveto(159.56559486,1552.98257812)(159.47256761,1552.32952682)(159.28652016,1551.84578661)
\curveto(159.1004586,1551.36204341)(158.76463023,1550.96853814)(158.27903403,1550.66526962)
\curveto(157.79342573,1550.36200046)(157.15618906,1550.21036605)(156.36732211,1550.21036591)
\curveto(155.60077342,1550.21036605)(154.97376975,1550.34246474)(154.48630921,1550.60666239)
\curveto(153.99884416,1550.87085953)(153.65092224,1551.25320153)(153.44254241,1551.75368954)
\curveto(153.23416016,1552.25417475)(153.12996964,1552.93792504)(153.12997054,1553.80494248)
\lineto(153.12997054,1558.5325921)
\lineto(154.21280882,1558.5325921)
\lineto(154.21280882,1553.81052412)
\curveto(154.21280684,1553.09979246)(154.27885619,1552.57604904)(154.41095706,1552.23929228)
\curveto(154.54305358,1551.90253174)(154.77004007,1551.64298571)(155.09191722,1551.46065341)
\curveto(155.41378865,1551.27831889)(155.80729392,1551.18715218)(156.27243422,1551.18715302)
\curveto(157.06874344,1551.18715218)(157.63620967,1551.36762505)(157.97483461,1551.72857216)
\curveto(158.31344806,1552.08951651)(158.48275766,1552.78349981)(158.48276391,1553.81052412)
\closepath
}
}
{
\newrgbcolor{curcolor}{0 0 0}
\pscustom[linestyle=none,fillstyle=solid,fillcolor=curcolor]
{
\newpath
\moveto(161.24567399,1550.34990692)
\lineto(161.24567399,1556.27760928)
\lineto(162.14989977,1556.27760928)
\lineto(162.14989977,1555.43478154)
\curveto(162.58526565,1556.08596722)(163.21412987,1556.41156259)(164.03649431,1556.41156865)
\curveto(164.39371541,1556.41156259)(164.72210161,1556.34737379)(165.02165388,1556.21900205)
\curveto(165.3211971,1556.09061858)(165.54539278,1555.92223925)(165.69424158,1555.71386358)
\curveto(165.84307998,1555.50547717)(165.9472705,1555.25802468)(166.00681346,1554.97150537)
\curveto(166.04401884,1554.78544625)(166.06262429,1554.45985087)(166.06262986,1553.99471826)
\lineto(166.06262986,1550.34990692)
\lineto(165.05793455,1550.34990692)
\lineto(165.05793455,1553.95564677)
\curveto(165.05792998,1554.36496307)(165.01885854,1554.67102273)(164.94072009,1554.87382666)
\curveto(164.86257275,1555.07662154)(164.72396215,1555.23848896)(164.52488787,1555.35942939)
\curveto(164.32580552,1555.48035981)(164.09230712,1555.54082752)(163.82439197,1555.54083272)
\curveto(163.39646328,1555.54082752)(163.02714509,1555.40500774)(162.7164363,1555.13337295)
\curveto(162.40572306,1554.86172859)(162.25036755,1554.34635762)(162.25036931,1553.58725849)
\lineto(162.25036931,1550.34990692)
\closepath
}
}
{
\newrgbcolor{curcolor}{0 0 0}
\pscustom[linestyle=none,fillstyle=solid,fillcolor=curcolor]
{
\newpath
\moveto(167.61432707,1557.37719249)
\lineto(167.61432707,1558.5325921)
\lineto(168.61902239,1558.5325921)
\lineto(168.61902239,1557.37719249)
\closepath
\moveto(167.61432707,1550.34990692)
\lineto(167.61432707,1556.27760928)
\lineto(168.61902239,1556.27760928)
\lineto(168.61902239,1550.34990692)
\closepath
}
}
{
\newrgbcolor{curcolor}{0 0 0}
\pscustom[linestyle=none,fillstyle=solid,fillcolor=curcolor]
{
\newpath
\moveto(169.48417561,1550.34990692)
\lineto(171.64985217,1553.43097255)
\lineto(169.64604319,1556.27760928)
\lineto(170.90191233,1556.27760928)
\lineto(171.81171975,1554.88778076)
\curveto(171.98288748,1554.62357883)(172.12056781,1554.40217397)(172.22476116,1554.22356552)
\curveto(172.3884863,1554.46915359)(172.53919044,1554.68683736)(172.67687405,1554.87661748)
\lineto(173.67598773,1556.27760928)
\lineto(174.87604046,1556.27760928)
\lineto(172.82757835,1553.48678896)
\lineto(175.0323264,1550.34990692)
\lineto(173.79878382,1550.34990692)
\lineto(172.58198616,1552.19184833)
\lineto(172.258251,1552.68861435)
\lineto(170.70097327,1550.34990692)
\closepath
}
}
{
\newrgbcolor{curcolor}{0 0 0}
\pscustom[linestyle=none,fillstyle=solid,fillcolor=curcolor]
{
\newpath
\moveto(103.9445516,1507.6608963)
\lineto(100.77417972,1515.84358148)
\lineto(101.94632425,1515.84358148)
\lineto(104.07292934,1509.8991342)
\curveto(104.24409613,1509.42283243)(104.3873581,1508.97630163)(104.50271567,1508.55954044)
\curveto(104.62922895,1509.00607035)(104.77621201,1509.45260115)(104.94366528,1509.8991342)
\lineto(107.15399497,1515.84358148)
\lineto(108.25915982,1515.84358148)
\lineto(105.05529809,1507.6608963)
\closepath
}
}
{
\newrgbcolor{curcolor}{0 0 0}
\pscustom[linestyle=none,fillstyle=solid,fillcolor=curcolor]
{
\newpath
\moveto(113.17100368,1509.5698174)
\lineto(114.20918883,1509.44143967)
\curveto(114.04545502,1508.83490021)(113.74218618,1508.36418232)(113.29938141,1508.02928458)
\curveto(112.85656676,1507.69438611)(112.29096107,1507.52693706)(111.60256266,1507.52693693)
\curveto(110.73554543,1507.52693706)(110.04807405,1507.79392527)(109.54014644,1508.32790236)
\curveto(109.03221647,1508.86187811)(108.77825207,1509.61074748)(108.77825249,1510.57451271)
\curveto(108.77825207,1511.57176193)(109.03500729,1512.34574866)(109.5485189,1512.89647522)
\curveto(110.06202814,1513.44719131)(110.72810325,1513.72255197)(111.54674625,1513.72255803)
\curveto(112.33933524,1513.72255197)(112.98680491,1513.45277295)(113.48915719,1512.91322014)
\curveto(113.99149922,1512.37365683)(114.24267279,1511.61455447)(114.24267868,1510.63591076)
\curveto(114.24267279,1510.57637035)(114.24081225,1510.48706419)(114.23709704,1510.36799201)
\lineto(109.81643765,1510.36799201)
\curveto(109.85364709,1509.71679855)(110.03784105,1509.21817248)(110.36902007,1508.87211232)
\curveto(110.70019508,1508.52604974)(111.11323607,1508.35301905)(111.6081443,1508.35301974)
\curveto(111.97652896,1508.35301905)(112.29096107,1508.44976739)(112.55144156,1508.64326505)
\curveto(112.81191368,1508.83676075)(113.01843417,1509.14561123)(113.17100368,1509.5698174)
\closepath
\moveto(109.87225406,1511.19407483)
\lineto(113.18216696,1511.19407483)
\curveto(113.13750905,1511.69269736)(113.01099199,1512.06666691)(112.80261539,1512.31598459)
\curveto(112.48259721,1512.7029733)(112.06769567,1512.89646998)(111.55790953,1512.89647522)
\curveto(111.09649117,1512.89646998)(110.70856753,1512.74204475)(110.39413746,1512.43319905)
\curveto(110.07970331,1512.1243438)(109.90574236,1511.71130281)(109.87225406,1511.19407483)
\closepath
}
}
{
\newrgbcolor{curcolor}{0 0 0}
\pscustom[linestyle=none,fillstyle=solid,fillcolor=curcolor]
{
\newpath
\moveto(115.46505766,1507.6608963)
\lineto(115.46505766,1513.58859866)
\lineto(116.36928344,1513.58859866)
\lineto(116.36928344,1512.68995452)
\curveto(116.59998938,1513.11043266)(116.81302178,1513.38765387)(117.0083813,1513.52161897)
\curveto(117.20373624,1513.65557235)(117.41862919,1513.72255197)(117.65306079,1513.72255803)
\curveto(117.99167705,1513.72255197)(118.33587788,1513.61464036)(118.68566431,1513.39882288)
\lineto(118.33960259,1512.46668889)
\curveto(118.09400703,1512.6118066)(117.84841509,1512.68436785)(117.60282603,1512.68437288)
\curveto(117.38327883,1512.68436785)(117.18606106,1512.6183185)(117.01117212,1512.48622463)
\curveto(116.8362786,1512.35412111)(116.71162208,1512.17085743)(116.63720219,1511.93643303)
\curveto(116.52556758,1511.57920411)(116.46975123,1511.18848966)(116.46975298,1510.7642885)
\lineto(116.46975298,1507.6608963)
\closepath
}
}
{
\newrgbcolor{curcolor}{0 0 0}
\pscustom[linestyle=none,fillstyle=solid,fillcolor=curcolor]
{
\newpath
\moveto(118.8921853,1509.43027638)
\lineto(119.88571733,1509.58656232)
\curveto(119.94153234,1509.18840376)(120.09688785,1508.88327438)(120.35178433,1508.67117326)
\curveto(120.60667718,1508.45907011)(120.96297155,1508.35301905)(121.42066851,1508.35301974)
\curveto(121.88208079,1508.35301905)(122.22442108,1508.44697657)(122.44769039,1508.63489259)
\curveto(122.67095188,1508.82280667)(122.78258458,1509.04328125)(122.78258882,1509.29631701)
\curveto(122.78258458,1509.52330187)(122.6839757,1509.70191419)(122.48676187,1509.83215451)
\curveto(122.34907759,1509.9214585)(122.00673731,1510.03495175)(121.45973999,1510.17263459)
\curveto(120.72296125,1510.35868658)(120.21224164,1510.51962372)(119.92757964,1510.6554465)
\curveto(119.64291486,1510.7912633)(119.42709164,1510.97917834)(119.28010932,1511.21919221)
\curveto(119.13312553,1511.45919896)(119.059634,1511.72432662)(119.05963452,1512.014576)
\curveto(119.059634,1512.27876904)(119.12010171,1512.52343071)(119.24103784,1512.74856174)
\curveto(119.36197257,1512.9736826)(119.5266308,1513.16066738)(119.73501304,1513.30951663)
\curveto(119.89129762,1513.42486477)(120.10433003,1513.52254338)(120.37411089,1513.60255276)
\curveto(120.64388808,1513.68255026)(120.93320283,1513.72255197)(121.24205601,1513.72255803)
\curveto(121.70718956,1513.72255197)(122.11557919,1513.65557235)(122.46722613,1513.52161897)
\curveto(122.81886521,1513.38765387)(123.07841124,1513.20625073)(123.245865,1512.97740901)
\curveto(123.41330934,1512.74855666)(123.52866314,1512.442497)(123.59192672,1512.05922912)
\lineto(122.60955796,1511.92526975)
\curveto(122.56490082,1512.23039487)(122.43559294,1512.46854463)(122.22163394,1512.63971975)
\curveto(122.00766758,1512.81088491)(121.70532902,1512.89646998)(121.31461734,1512.89647522)
\curveto(120.8531994,1512.89646998)(120.52388293,1512.82018764)(120.32666694,1512.66762795)
\curveto(120.12944739,1512.51505826)(120.0308385,1512.33644593)(120.03083999,1512.13179045)
\curveto(120.0308385,1512.00154783)(120.07177049,1511.88433349)(120.15363608,1511.78014709)
\curveto(120.23549845,1511.67223136)(120.36387606,1511.5829252)(120.53876929,1511.51222834)
\curveto(120.63923672,1511.47501359)(120.93506338,1511.38942852)(121.42625015,1511.25547287)
\curveto(122.13697546,1511.06569369)(122.63281071,1510.91033818)(122.91375738,1510.78940588)
\curveto(123.1946953,1510.66846732)(123.41516989,1510.49264582)(123.5751818,1510.26194084)
\curveto(123.73518363,1510.03123066)(123.81518707,1509.74470672)(123.81519234,1509.40236818)
\curveto(123.81518707,1509.06746834)(123.71750845,1508.75210596)(123.52215621,1508.45628009)
\curveto(123.326794,1508.16045264)(123.04492143,1507.9316056)(122.67653765,1507.76973829)
\curveto(122.3081456,1507.60787077)(121.89138352,1507.52693706)(121.42625015,1507.52693693)
\curveto(120.65598163,1507.52693706)(120.06897967,1507.68694393)(119.66524253,1508.00695802)
\curveto(119.26150313,1508.32697142)(119.00381765,1508.8014104)(118.8921853,1509.43027638)
\closepath
}
}
{
\newrgbcolor{curcolor}{0 0 0}
\pscustom[linestyle=none,fillstyle=solid,fillcolor=curcolor]
{
\newpath
\moveto(125.01524603,1514.68818187)
\lineto(125.01524603,1515.84358148)
\lineto(126.01994135,1515.84358148)
\lineto(126.01994135,1514.68818187)
\closepath
\moveto(125.01524603,1507.6608963)
\lineto(125.01524603,1513.58859866)
\lineto(126.01994135,1513.58859866)
\lineto(126.01994135,1507.6608963)
\closepath
}
}
{
\newrgbcolor{curcolor}{0 0 0}
\pscustom[linestyle=none,fillstyle=solid,fillcolor=curcolor]
{
\newpath
\moveto(127.18092152,1510.62474748)
\curveto(127.18092114,1511.72246608)(127.48605052,1512.53552425)(128.09631059,1513.06392444)
\curveto(128.60609863,1513.50300766)(129.22752066,1513.72255197)(129.96057856,1513.72255803)
\curveto(130.77549412,1513.72255197)(131.44156923,1513.45556376)(131.95880591,1512.9215926)
\curveto(132.47603227,1512.38761092)(132.73464802,1511.64990482)(132.73465396,1510.70847209)
\curveto(132.73464802,1509.94564558)(132.6202245,1509.34561982)(132.39138306,1508.90839298)
\curveto(132.16253043,1508.47116366)(131.82949287,1508.13161419)(131.39226938,1507.88974357)
\curveto(130.95503671,1507.64787249)(130.47780692,1507.52693706)(129.96057856,1507.52693693)
\curveto(129.13077232,1507.52693706)(128.46004584,1507.792995)(127.94839711,1508.32511154)
\curveto(127.43674608,1508.85722675)(127.18092114,1509.6237713)(127.18092152,1510.62474748)
\closepath
\moveto(128.21352504,1510.62474748)
\curveto(128.21352363,1509.86564215)(128.37911213,1509.29724565)(128.71029106,1508.91955627)
\curveto(129.04146616,1508.54186437)(129.45822824,1508.35301905)(129.96057856,1508.35301974)
\curveto(130.45920146,1508.35301905)(130.874103,1508.54279464)(131.20528442,1508.92234709)
\curveto(131.53645703,1509.30189701)(131.70204554,1509.88052651)(131.70205044,1510.65823732)
\curveto(131.70204554,1511.39128906)(131.53552676,1511.94666175)(131.2024936,1512.32435705)
\curveto(130.86945164,1512.70204303)(130.45548037,1512.89088835)(129.96057856,1512.89089358)
\curveto(129.45822824,1512.89088835)(129.04146616,1512.7029733)(128.71029106,1512.32714788)
\curveto(128.37911213,1511.95131312)(128.21352363,1511.38384688)(128.21352504,1510.62474748)
\closepath
}
}
{
\newrgbcolor{curcolor}{0 0 0}
\pscustom[linestyle=none,fillstyle=solid,fillcolor=curcolor]
{
\newpath
\moveto(133.91796098,1507.6608963)
\lineto(133.91796098,1513.58859866)
\lineto(134.82218676,1513.58859866)
\lineto(134.82218676,1512.74577092)
\curveto(135.25755264,1513.39695659)(135.88641685,1513.72255197)(136.7087813,1513.72255803)
\curveto(137.0660024,1513.72255197)(137.39438859,1513.65836317)(137.69394087,1513.52999143)
\curveto(137.99348409,1513.40160796)(138.21767976,1513.23322863)(138.36652857,1513.02485296)
\curveto(138.51536697,1512.81646655)(138.61955749,1512.56901406)(138.67910044,1512.28249475)
\curveto(138.71630583,1512.09643563)(138.73491128,1511.77084025)(138.73491685,1511.30570764)
\lineto(138.73491685,1507.6608963)
\lineto(137.73022153,1507.6608963)
\lineto(137.73022153,1511.26663615)
\curveto(137.73021697,1511.67595245)(137.69114552,1511.98201211)(137.61300708,1512.18481604)
\curveto(137.53485974,1512.38761092)(137.39624914,1512.54947834)(137.19717485,1512.67041877)
\curveto(136.9980925,1512.79134919)(136.7645941,1512.8518169)(136.49667895,1512.8518221)
\curveto(136.06875027,1512.8518169)(135.69943208,1512.71599712)(135.38872329,1512.44436233)
\curveto(135.07801004,1512.17271797)(134.92265453,1511.657347)(134.92265629,1510.89824787)
\lineto(134.92265629,1507.6608963)
\closepath
}
}
{
\newrgbcolor{curcolor}{0 0 0}
\pscustom[linestyle=none,fillstyle=solid,fillcolor=curcolor]
{
\newpath
\moveto(143.17790404,1509.82099123)
\lineto(144.18259935,1509.9549506)
\curveto(144.29795166,1509.38562153)(144.49423916,1508.97537136)(144.77146244,1508.72419884)
\curveto(145.04868157,1508.4730242)(145.3863705,1508.34743741)(145.78453022,1508.3474381)
\curveto(146.25710556,1508.34743741)(146.65619247,1508.51116538)(146.98179213,1508.83862248)
\curveto(147.30738323,1509.16607722)(147.47018092,1509.57167604)(147.47018569,1510.05542014)
\curveto(147.47018092,1510.51683291)(147.31947677,1510.89731436)(147.0180728,1511.19686565)
\curveto(146.71666018,1511.49640986)(146.33338791,1511.64618373)(145.86825483,1511.64618772)
\curveto(145.67847606,1511.64618373)(145.44218685,1511.60897283)(145.15938646,1511.5345549)
\lineto(145.27101928,1512.41645413)
\curveto(145.33799632,1512.40900719)(145.39195213,1512.4052861)(145.43288686,1512.40529084)
\curveto(145.86080948,1512.4052861)(146.24594229,1512.5169188)(146.58828647,1512.74018928)
\curveto(146.93062286,1512.96344961)(147.101793,1513.30765043)(147.10179741,1513.7727928)
\curveto(147.101793,1514.1411746)(146.97713649,1514.44630399)(146.72782748,1514.68818187)
\curveto(146.47851042,1514.93004569)(146.15663613,1515.05098112)(145.76220365,1515.05098851)
\curveto(145.37148614,1515.05098112)(145.04589076,1514.92818515)(144.78541654,1514.68260022)
\curveto(144.52493815,1514.43700126)(144.3574891,1514.06861335)(144.28306888,1513.57743538)
\lineto(143.27837357,1513.75604788)
\curveto(143.40116896,1514.42955908)(143.68025071,1514.95144196)(144.11561967,1515.32169808)
\curveto(144.55098578,1515.69193888)(145.09240438,1515.87706311)(145.73987709,1515.87707132)
\curveto(146.18640485,1515.87706311)(146.5975853,1515.78124504)(146.97341967,1515.58961683)
\curveto(147.34924549,1515.39797276)(147.6366997,1515.13656619)(147.83578315,1514.80539632)
\curveto(148.03485633,1514.47421216)(148.13439549,1514.12256915)(148.13440092,1513.75046624)
\curveto(148.13439549,1513.39695659)(148.03950769,1513.07508231)(147.84973725,1512.78484241)
\curveto(147.65995651,1512.49459226)(147.37901421,1512.26388468)(147.00690952,1512.09271897)
\curveto(147.49064691,1511.98108184)(147.86647701,1511.74944398)(148.13440092,1511.39780471)
\curveto(148.40231397,1511.04615796)(148.53627321,1510.60613907)(148.53627905,1510.0777467)
\curveto(148.53627321,1509.36329499)(148.27579691,1508.75768759)(147.75484936,1508.26092267)
\curveto(147.2338917,1507.76415655)(146.57525876,1507.51577379)(145.77894858,1507.51577364)
\curveto(145.06077512,1507.51577379)(144.46447044,1507.72973647)(143.99003275,1508.15766232)
\curveto(143.51559248,1508.58558718)(143.24488318,1509.14002959)(143.17790404,1509.82099123)
\closepath
}
}
{
\newrgbcolor{curcolor}{0 0 0}
\pscustom[linestyle=none,fillstyle=solid,fillcolor=curcolor]
{
\newpath
\moveto(158.48276391,1515.84358148)
\lineto(159.56560219,1515.84358148)
\lineto(159.56560219,1511.11593186)
\curveto(159.56559486,1510.2935675)(159.47256761,1509.6405162)(159.28652016,1509.15677599)
\curveto(159.1004586,1508.67303279)(158.76463023,1508.27952752)(158.27903403,1507.976259)
\curveto(157.79342573,1507.67298984)(157.15618906,1507.52135543)(156.36732211,1507.52135529)
\curveto(155.60077342,1507.52135543)(154.97376975,1507.65345412)(154.48630921,1507.91765177)
\curveto(153.99884416,1508.18184891)(153.65092224,1508.56419091)(153.44254241,1509.06467892)
\curveto(153.23416016,1509.56516413)(153.12996964,1510.24891442)(153.12997054,1511.11593186)
\lineto(153.12997054,1515.84358148)
\lineto(154.21280882,1515.84358148)
\lineto(154.21280882,1511.1215135)
\curveto(154.21280684,1510.41078184)(154.27885619,1509.88703842)(154.41095706,1509.55028166)
\curveto(154.54305358,1509.21352112)(154.77004007,1508.95397509)(155.09191722,1508.77164279)
\curveto(155.41378865,1508.58930827)(155.80729392,1508.49814156)(156.27243422,1508.4981424)
\curveto(157.06874344,1508.49814156)(157.63620967,1508.67861443)(157.97483461,1509.03956154)
\curveto(158.31344806,1509.40050589)(158.48275766,1510.09448919)(158.48276391,1511.1215135)
\closepath
}
}
{
\newrgbcolor{curcolor}{0 0 0}
\pscustom[linestyle=none,fillstyle=solid,fillcolor=curcolor]
{
\newpath
\moveto(161.24567399,1507.6608963)
\lineto(161.24567399,1513.58859866)
\lineto(162.14989977,1513.58859866)
\lineto(162.14989977,1512.74577092)
\curveto(162.58526565,1513.39695659)(163.21412987,1513.72255197)(164.03649431,1513.72255803)
\curveto(164.39371541,1513.72255197)(164.72210161,1513.65836317)(165.02165388,1513.52999143)
\curveto(165.3211971,1513.40160796)(165.54539278,1513.23322863)(165.69424158,1513.02485296)
\curveto(165.84307998,1512.81646655)(165.9472705,1512.56901406)(166.00681346,1512.28249475)
\curveto(166.04401884,1512.09643563)(166.06262429,1511.77084025)(166.06262986,1511.30570764)
\lineto(166.06262986,1507.6608963)
\lineto(165.05793455,1507.6608963)
\lineto(165.05793455,1511.26663615)
\curveto(165.05792998,1511.67595245)(165.01885854,1511.98201211)(164.94072009,1512.18481604)
\curveto(164.86257275,1512.38761092)(164.72396215,1512.54947834)(164.52488787,1512.67041877)
\curveto(164.32580552,1512.79134919)(164.09230712,1512.8518169)(163.82439197,1512.8518221)
\curveto(163.39646328,1512.8518169)(163.02714509,1512.71599712)(162.7164363,1512.44436233)
\curveto(162.40572306,1512.17271797)(162.25036755,1511.657347)(162.25036931,1510.89824787)
\lineto(162.25036931,1507.6608963)
\closepath
}
}
{
\newrgbcolor{curcolor}{0 0 0}
\pscustom[linestyle=none,fillstyle=solid,fillcolor=curcolor]
{
\newpath
\moveto(167.61432707,1514.68818187)
\lineto(167.61432707,1515.84358148)
\lineto(168.61902239,1515.84358148)
\lineto(168.61902239,1514.68818187)
\closepath
\moveto(167.61432707,1507.6608963)
\lineto(167.61432707,1513.58859866)
\lineto(168.61902239,1513.58859866)
\lineto(168.61902239,1507.6608963)
\closepath
}
}
{
\newrgbcolor{curcolor}{0 0 0}
\pscustom[linestyle=none,fillstyle=solid,fillcolor=curcolor]
{
\newpath
\moveto(169.48417561,1507.6608963)
\lineto(171.64985217,1510.74196193)
\lineto(169.64604319,1513.58859866)
\lineto(170.90191233,1513.58859866)
\lineto(171.81171975,1512.19877014)
\curveto(171.98288748,1511.93456821)(172.12056781,1511.71316335)(172.22476116,1511.5345549)
\curveto(172.3884863,1511.78014297)(172.53919044,1511.99782674)(172.67687405,1512.18760686)
\lineto(173.67598773,1513.58859866)
\lineto(174.87604046,1513.58859866)
\lineto(172.82757835,1510.79777834)
\lineto(175.0323264,1507.6608963)
\lineto(173.79878382,1507.6608963)
\lineto(172.58198616,1509.50283771)
\lineto(172.258251,1509.99960373)
\lineto(170.70097327,1507.6608963)
\closepath
}
}
{
\newrgbcolor{curcolor}{0 0 0}
\pscustom[linestyle=none,fillstyle=solid,fillcolor=curcolor]
{
\newpath
\moveto(296.14155783,1507.6608963)
\lineto(292.97118595,1515.84358148)
\lineto(294.14333048,1515.84358148)
\lineto(296.26993556,1509.8991342)
\curveto(296.44110236,1509.42283243)(296.58436432,1508.97630163)(296.69972189,1508.55954044)
\curveto(296.82623517,1509.00607035)(296.97321823,1509.45260115)(297.1406715,1509.8991342)
\lineto(299.3510012,1515.84358148)
\lineto(300.45616604,1515.84358148)
\lineto(297.25230432,1507.6608963)
\closepath
}
}
{
\newrgbcolor{curcolor}{0 0 0}
\pscustom[linestyle=none,fillstyle=solid,fillcolor=curcolor]
{
\newpath
\moveto(305.3680099,1509.5698174)
\lineto(306.40619506,1509.44143967)
\curveto(306.24246125,1508.83490021)(305.93919241,1508.36418232)(305.49638764,1508.02928458)
\curveto(305.05357298,1507.69438611)(304.4879673,1507.52693706)(303.79956888,1507.52693693)
\curveto(302.93255166,1507.52693706)(302.24508028,1507.79392527)(301.73715267,1508.32790236)
\curveto(301.22922269,1508.86187811)(300.9752583,1509.61074748)(300.97525872,1510.57451271)
\curveto(300.9752583,1511.57176193)(301.23201351,1512.34574866)(301.74552513,1512.89647522)
\curveto(302.25903436,1513.44719131)(302.92510948,1513.72255197)(303.74375247,1513.72255803)
\curveto(304.53634147,1513.72255197)(305.18381113,1513.45277295)(305.68616342,1512.91322014)
\curveto(306.18850544,1512.37365683)(306.43967902,1511.61455447)(306.4396849,1510.63591076)
\curveto(306.43967902,1510.57637035)(306.43781848,1510.48706419)(306.43410326,1510.36799201)
\lineto(302.01344388,1510.36799201)
\curveto(302.05065332,1509.71679855)(302.23484728,1509.21817248)(302.5660263,1508.87211232)
\curveto(302.8972013,1508.52604974)(303.3102423,1508.35301905)(303.80515052,1508.35301974)
\curveto(304.17353519,1508.35301905)(304.4879673,1508.44976739)(304.74844779,1508.64326505)
\curveto(305.0089199,1508.83676075)(305.2154404,1509.14561123)(305.3680099,1509.5698174)
\closepath
\moveto(302.06926028,1511.19407483)
\lineto(305.37917318,1511.19407483)
\curveto(305.33451528,1511.69269736)(305.20799822,1512.06666691)(304.99962162,1512.31598459)
\curveto(304.67960343,1512.7029733)(304.26470189,1512.89646998)(303.75491576,1512.89647522)
\curveto(303.29349739,1512.89646998)(302.90557376,1512.74204475)(302.59114368,1512.43319905)
\curveto(302.27670954,1512.1243438)(302.10274858,1511.71130281)(302.06926028,1511.19407483)
\closepath
}
}
{
\newrgbcolor{curcolor}{0 0 0}
\pscustom[linestyle=none,fillstyle=solid,fillcolor=curcolor]
{
\newpath
\moveto(307.66206389,1507.6608963)
\lineto(307.66206389,1513.58859866)
\lineto(308.56628967,1513.58859866)
\lineto(308.56628967,1512.68995452)
\curveto(308.79699561,1513.11043266)(309.01002801,1513.38765387)(309.20538752,1513.52161897)
\curveto(309.40074246,1513.65557235)(309.61563541,1513.72255197)(309.85006702,1513.72255803)
\curveto(310.18868328,1513.72255197)(310.53288411,1513.61464036)(310.88267053,1513.39882288)
\lineto(310.53660882,1512.46668889)
\curveto(310.29101326,1512.6118066)(310.04542131,1512.68436785)(309.79983225,1512.68437288)
\curveto(309.58028506,1512.68436785)(309.38306729,1512.6183185)(309.20817834,1512.48622463)
\curveto(309.03328482,1512.35412111)(308.90862831,1512.17085743)(308.83420842,1511.93643303)
\curveto(308.7225738,1511.57920411)(308.66675745,1511.18848966)(308.6667592,1510.7642885)
\lineto(308.6667592,1507.6608963)
\closepath
}
}
{
\newrgbcolor{curcolor}{0 0 0}
\pscustom[linestyle=none,fillstyle=solid,fillcolor=curcolor]
{
\newpath
\moveto(311.08919153,1509.43027638)
\lineto(312.08272356,1509.58656232)
\curveto(312.13853856,1509.18840376)(312.29389407,1508.88327438)(312.54879055,1508.67117326)
\curveto(312.80368341,1508.45907011)(313.15997778,1508.35301905)(313.61767473,1508.35301974)
\curveto(314.07908702,1508.35301905)(314.4214273,1508.44697657)(314.64469661,1508.63489259)
\curveto(314.86795811,1508.82280667)(314.97959081,1509.04328125)(314.97959505,1509.29631701)
\curveto(314.97959081,1509.52330187)(314.88098192,1509.70191419)(314.6837681,1509.83215451)
\curveto(314.54608382,1509.9214585)(314.20374354,1510.03495175)(313.65674622,1510.17263459)
\curveto(312.91996747,1510.35868658)(312.40924786,1510.51962372)(312.12458586,1510.6554465)
\curveto(311.83992109,1510.7912633)(311.62409787,1510.97917834)(311.47711555,1511.21919221)
\curveto(311.33013175,1511.45919896)(311.25664023,1511.72432662)(311.25664074,1512.014576)
\curveto(311.25664023,1512.27876904)(311.31710794,1512.52343071)(311.43804407,1512.74856174)
\curveto(311.55897879,1512.9736826)(311.72363703,1513.16066738)(311.93201926,1513.30951663)
\curveto(312.08830385,1513.42486477)(312.30133625,1513.52254338)(312.57111711,1513.60255276)
\curveto(312.84089431,1513.68255026)(313.13020906,1513.72255197)(313.43906223,1513.72255803)
\curveto(313.90419579,1513.72255197)(314.31258542,1513.65557235)(314.66423235,1513.52161897)
\curveto(315.01587144,1513.38765387)(315.27541747,1513.20625073)(315.44287122,1512.97740901)
\curveto(315.61031557,1512.74855666)(315.72566936,1512.442497)(315.78893294,1512.05922912)
\lineto(314.80656419,1511.92526975)
\curveto(314.76190704,1512.23039487)(314.63259916,1512.46854463)(314.41864017,1512.63971975)
\curveto(314.20467381,1512.81088491)(313.90233524,1512.89646998)(313.51162356,1512.89647522)
\curveto(313.05020562,1512.89646998)(312.72088916,1512.82018764)(312.52367317,1512.66762795)
\curveto(312.32645361,1512.51505826)(312.22784473,1512.33644593)(312.22784622,1512.13179045)
\curveto(312.22784473,1512.00154783)(312.26877672,1511.88433349)(312.35064231,1511.78014709)
\curveto(312.43250468,1511.67223136)(312.56088228,1511.5829252)(312.73577551,1511.51222834)
\curveto(312.83624295,1511.47501359)(313.1320696,1511.38942852)(313.62325638,1511.25547287)
\curveto(314.33398169,1511.06569369)(314.82981693,1510.91033818)(315.11076361,1510.78940588)
\curveto(315.39170153,1510.66846732)(315.61217611,1510.49264582)(315.77218802,1510.26194084)
\curveto(315.93218986,1510.03123066)(316.01219329,1509.74470672)(316.01219857,1509.40236818)
\curveto(316.01219329,1509.06746834)(315.91451468,1508.75210596)(315.71916243,1508.45628009)
\curveto(315.52380023,1508.16045264)(315.24192766,1507.9316056)(314.87354388,1507.76973829)
\curveto(314.50515183,1507.60787077)(314.08838974,1507.52693706)(313.62325638,1507.52693693)
\curveto(312.85298785,1507.52693706)(312.2659859,1507.68694393)(311.86224875,1508.00695802)
\curveto(311.45850936,1508.32697142)(311.20082387,1508.8014104)(311.08919153,1509.43027638)
\closepath
}
}
{
\newrgbcolor{curcolor}{0 0 0}
\pscustom[linestyle=none,fillstyle=solid,fillcolor=curcolor]
{
\newpath
\moveto(317.21225226,1514.68818187)
\lineto(317.21225226,1515.84358148)
\lineto(318.21694757,1515.84358148)
\lineto(318.21694757,1514.68818187)
\closepath
\moveto(317.21225226,1507.6608963)
\lineto(317.21225226,1513.58859866)
\lineto(318.21694757,1513.58859866)
\lineto(318.21694757,1507.6608963)
\closepath
}
}
{
\newrgbcolor{curcolor}{0 0 0}
\pscustom[linestyle=none,fillstyle=solid,fillcolor=curcolor]
{
\newpath
\moveto(319.37792775,1510.62474748)
\curveto(319.37792737,1511.72246608)(319.68305675,1512.53552425)(320.29331681,1513.06392444)
\curveto(320.80310485,1513.50300766)(321.42452689,1513.72255197)(322.15758478,1513.72255803)
\curveto(322.97250034,1513.72255197)(323.63857546,1513.45556376)(324.15581213,1512.9215926)
\curveto(324.67303849,1512.38761092)(324.93165425,1511.64990482)(324.93166018,1510.70847209)
\curveto(324.93165425,1509.94564558)(324.81723073,1509.34561982)(324.58838928,1508.90839298)
\curveto(324.35953666,1508.47116366)(324.0264991,1508.13161419)(323.58927561,1507.88974357)
\curveto(323.15204294,1507.64787249)(322.67481314,1507.52693706)(322.15758478,1507.52693693)
\curveto(321.32777855,1507.52693706)(320.65705207,1507.792995)(320.14540333,1508.32511154)
\curveto(319.63375231,1508.85722675)(319.37792737,1509.6237713)(319.37792775,1510.62474748)
\closepath
\moveto(320.41053126,1510.62474748)
\curveto(320.41052985,1509.86564215)(320.57611836,1509.29724565)(320.90729728,1508.91955627)
\curveto(321.23847239,1508.54186437)(321.65523447,1508.35301905)(322.15758478,1508.35301974)
\curveto(322.65620769,1508.35301905)(323.07110923,1508.54279464)(323.40229065,1508.92234709)
\curveto(323.73346326,1509.30189701)(323.89905176,1509.88052651)(323.89905666,1510.65823732)
\curveto(323.89905176,1511.39128906)(323.73253298,1511.94666175)(323.39949983,1512.32435705)
\curveto(323.06645787,1512.70204303)(322.6524866,1512.89088835)(322.15758478,1512.89089358)
\curveto(321.65523447,1512.89088835)(321.23847239,1512.7029733)(320.90729728,1512.32714788)
\curveto(320.57611836,1511.95131312)(320.41052985,1511.38384688)(320.41053126,1510.62474748)
\closepath
}
}
{
\newrgbcolor{curcolor}{0 0 0}
\pscustom[linestyle=none,fillstyle=solid,fillcolor=curcolor]
{
\newpath
\moveto(326.1149672,1507.6608963)
\lineto(326.1149672,1513.58859866)
\lineto(327.01919299,1513.58859866)
\lineto(327.01919299,1512.74577092)
\curveto(327.45455886,1513.39695659)(328.08342308,1513.72255197)(328.90578752,1513.72255803)
\curveto(329.26300862,1513.72255197)(329.59139482,1513.65836317)(329.8909471,1513.52999143)
\curveto(330.19049031,1513.40160796)(330.41468599,1513.23322863)(330.56353479,1513.02485296)
\curveto(330.71237319,1512.81646655)(330.81656371,1512.56901406)(330.87610667,1512.28249475)
\curveto(330.91331205,1512.09643563)(330.9319175,1511.77084025)(330.93192307,1511.30570764)
\lineto(330.93192307,1507.6608963)
\lineto(329.92722776,1507.6608963)
\lineto(329.92722776,1511.26663615)
\curveto(329.92722319,1511.67595245)(329.88815175,1511.98201211)(329.81001331,1512.18481604)
\curveto(329.73186597,1512.38761092)(329.59325536,1512.54947834)(329.39418108,1512.67041877)
\curveto(329.19509873,1512.79134919)(328.96160033,1512.8518169)(328.69368518,1512.8518221)
\curveto(328.26575649,1512.8518169)(327.89643831,1512.71599712)(327.58572951,1512.44436233)
\curveto(327.27501627,1512.17271797)(327.11966076,1511.657347)(327.11966252,1510.89824787)
\lineto(327.11966252,1507.6608963)
\closepath
}
}
{
\newrgbcolor{curcolor}{0 0 0}
\pscustom[linestyle=none,fillstyle=solid,fillcolor=curcolor]
{
\newpath
\moveto(338.58993527,1507.6608963)
\lineto(338.58993527,1509.62005217)
\lineto(335.04001183,1509.62005217)
\lineto(335.04001183,1510.54102287)
\lineto(338.77412941,1515.84358148)
\lineto(339.59463059,1515.84358148)
\lineto(339.59463059,1510.54102287)
\lineto(340.69979543,1510.54102287)
\lineto(340.69979543,1509.62005217)
\lineto(339.59463059,1509.62005217)
\lineto(339.59463059,1507.6608963)
\closepath
\moveto(338.58993527,1510.54102287)
\lineto(338.58993527,1514.23048733)
\lineto(336.02796222,1510.54102287)
\closepath
}
}
{
\newrgbcolor{curcolor}{0 0 0}
\pscustom[linestyle=none,fillstyle=solid,fillcolor=curcolor]
{
\newpath
\moveto(350.67977013,1515.84358148)
\lineto(351.76260842,1515.84358148)
\lineto(351.76260842,1511.11593186)
\curveto(351.76260108,1510.2935675)(351.66957383,1509.6405162)(351.48352639,1509.15677599)
\curveto(351.29746483,1508.67303279)(350.96163645,1508.27952752)(350.47604025,1507.976259)
\curveto(349.99043195,1507.67298984)(349.35319528,1507.52135543)(348.56432833,1507.52135529)
\curveto(347.79777965,1507.52135543)(347.17077598,1507.65345412)(346.68331544,1507.91765177)
\curveto(346.19585039,1508.18184891)(345.84792847,1508.56419091)(345.63954864,1509.06467892)
\curveto(345.43116638,1509.56516413)(345.32697586,1510.24891442)(345.32697676,1511.11593186)
\lineto(345.32697676,1515.84358148)
\lineto(346.40981505,1515.84358148)
\lineto(346.40981505,1511.1215135)
\curveto(346.40981306,1510.41078184)(346.47586241,1509.88703842)(346.60796329,1509.55028166)
\curveto(346.74005981,1509.21352112)(346.9670463,1508.95397509)(347.28892345,1508.77164279)
\curveto(347.61079487,1508.58930827)(348.00430015,1508.49814156)(348.46944044,1508.4981424)
\curveto(349.26574967,1508.49814156)(349.8332159,1508.67861443)(350.17184084,1509.03956154)
\curveto(350.51045429,1509.40050589)(350.67976388,1510.09448919)(350.67977013,1511.1215135)
\closepath
}
}
{
\newrgbcolor{curcolor}{0 0 0}
\pscustom[linestyle=none,fillstyle=solid,fillcolor=curcolor]
{
\newpath
\moveto(353.44268022,1507.6608963)
\lineto(353.44268022,1513.58859866)
\lineto(354.346906,1513.58859866)
\lineto(354.346906,1512.74577092)
\curveto(354.78227188,1513.39695659)(355.41113609,1513.72255197)(356.23350054,1513.72255803)
\curveto(356.59072163,1513.72255197)(356.91910783,1513.65836317)(357.21866011,1513.52999143)
\curveto(357.51820333,1513.40160796)(357.742399,1513.23322863)(357.89124781,1513.02485296)
\curveto(358.0400862,1512.81646655)(358.14427673,1512.56901406)(358.20381968,1512.28249475)
\curveto(358.24102507,1512.09643563)(358.25963052,1511.77084025)(358.25963609,1511.30570764)
\lineto(358.25963609,1507.6608963)
\lineto(357.25494077,1507.6608963)
\lineto(357.25494077,1511.26663615)
\curveto(357.25493621,1511.67595245)(357.21586476,1511.98201211)(357.13772632,1512.18481604)
\curveto(357.05957898,1512.38761092)(356.92096838,1512.54947834)(356.72189409,1512.67041877)
\curveto(356.52281174,1512.79134919)(356.28931334,1512.8518169)(356.02139819,1512.8518221)
\curveto(355.5934695,1512.8518169)(355.22415132,1512.71599712)(354.91344252,1512.44436233)
\curveto(354.60272928,1512.17271797)(354.44737377,1511.657347)(354.44737553,1510.89824787)
\lineto(354.44737553,1507.6608963)
\closepath
}
}
{
\newrgbcolor{curcolor}{0 0 0}
\pscustom[linestyle=none,fillstyle=solid,fillcolor=curcolor]
{
\newpath
\moveto(359.8113333,1514.68818187)
\lineto(359.8113333,1515.84358148)
\lineto(360.81602861,1515.84358148)
\lineto(360.81602861,1514.68818187)
\closepath
\moveto(359.8113333,1507.6608963)
\lineto(359.8113333,1513.58859866)
\lineto(360.81602861,1513.58859866)
\lineto(360.81602861,1507.6608963)
\closepath
}
}
{
\newrgbcolor{curcolor}{0 0 0}
\pscustom[linestyle=none,fillstyle=solid,fillcolor=curcolor]
{
\newpath
\moveto(361.68118183,1507.6608963)
\lineto(363.8468584,1510.74196193)
\lineto(361.84304941,1513.58859866)
\lineto(363.09891855,1513.58859866)
\lineto(364.00872598,1512.19877014)
\curveto(364.17989371,1511.93456821)(364.31757404,1511.71316335)(364.42176739,1511.5345549)
\curveto(364.58549252,1511.78014297)(364.73619667,1511.99782674)(364.87388028,1512.18760686)
\lineto(365.87299395,1513.58859866)
\lineto(367.07304669,1513.58859866)
\lineto(365.02458457,1510.79777834)
\lineto(367.22933263,1507.6608963)
\lineto(365.99579005,1507.6608963)
\lineto(364.77899239,1509.50283771)
\lineto(364.45525723,1509.99960373)
\lineto(362.89797949,1507.6608963)
\closepath
}
}
{
\newrgbcolor{curcolor}{0 0 0}
\pscustom[linestyle=none,fillstyle=solid,fillcolor=curcolor]
{
\newpath
\moveto(296.14155783,1464.94489288)
\lineto(292.97118595,1473.12757806)
\lineto(294.14333048,1473.12757806)
\lineto(296.26993556,1467.18313078)
\curveto(296.44110236,1466.70682902)(296.58436432,1466.26029821)(296.69972189,1465.84353703)
\curveto(296.82623517,1466.29006693)(296.97321823,1466.73659774)(297.1406715,1467.18313078)
\lineto(299.3510012,1473.12757806)
\lineto(300.45616604,1473.12757806)
\lineto(297.25230432,1464.94489288)
\closepath
}
}
{
\newrgbcolor{curcolor}{0 0 0}
\pscustom[linestyle=none,fillstyle=solid,fillcolor=curcolor]
{
\newpath
\moveto(305.3680099,1466.85381398)
\lineto(306.40619506,1466.72543625)
\curveto(306.24246125,1466.11889679)(305.93919241,1465.6481789)(305.49638764,1465.31328117)
\curveto(305.05357298,1464.97838269)(304.4879673,1464.81093364)(303.79956888,1464.81093351)
\curveto(302.93255166,1464.81093364)(302.24508028,1465.07792185)(301.73715267,1465.61189894)
\curveto(301.22922269,1466.14587469)(300.9752583,1466.89474406)(300.97525872,1467.8585093)
\curveto(300.9752583,1468.85575851)(301.23201351,1469.62974524)(301.74552513,1470.1804718)
\curveto(302.25903436,1470.73118789)(302.92510948,1471.00654856)(303.74375247,1471.00655462)
\curveto(304.53634147,1471.00654856)(305.18381113,1470.73676953)(305.68616342,1470.19721672)
\curveto(306.18850544,1469.65765342)(306.43967902,1468.89855105)(306.4396849,1467.91990734)
\curveto(306.43967902,1467.86036693)(306.43781848,1467.77106077)(306.43410326,1467.65198859)
\lineto(302.01344388,1467.65198859)
\curveto(302.05065332,1467.00079513)(302.23484728,1466.50216906)(302.5660263,1466.1561089)
\curveto(302.8972013,1465.81004632)(303.3102423,1465.63701563)(303.80515052,1465.63701632)
\curveto(304.17353519,1465.63701563)(304.4879673,1465.73376397)(304.74844779,1465.92726164)
\curveto(305.0089199,1466.12075734)(305.2154404,1466.42960781)(305.3680099,1466.85381398)
\closepath
\moveto(302.06926028,1468.47807141)
\lineto(305.37917318,1468.47807141)
\curveto(305.33451528,1468.97669394)(305.20799822,1469.35066349)(304.99962162,1469.59998118)
\curveto(304.67960343,1469.98696988)(304.26470189,1470.18046657)(303.75491576,1470.1804718)
\curveto(303.29349739,1470.18046657)(302.90557376,1470.02604133)(302.59114368,1469.71719563)
\curveto(302.27670954,1469.40834038)(302.10274858,1468.99529939)(302.06926028,1468.47807141)
\closepath
}
}
{
\newrgbcolor{curcolor}{0 0 0}
\pscustom[linestyle=none,fillstyle=solid,fillcolor=curcolor]
{
\newpath
\moveto(307.66206389,1464.94489288)
\lineto(307.66206389,1470.87259524)
\lineto(308.56628967,1470.87259524)
\lineto(308.56628967,1469.9739511)
\curveto(308.79699561,1470.39442924)(309.01002801,1470.67165045)(309.20538752,1470.80561555)
\curveto(309.40074246,1470.93956893)(309.61563541,1471.00654856)(309.85006702,1471.00655462)
\curveto(310.18868328,1471.00654856)(310.53288411,1470.89863694)(310.88267053,1470.68281946)
\lineto(310.53660882,1469.75068547)
\curveto(310.29101326,1469.89580318)(310.04542131,1469.96836443)(309.79983225,1469.96836946)
\curveto(309.58028506,1469.96836443)(309.38306729,1469.90231509)(309.20817834,1469.77022122)
\curveto(309.03328482,1469.63811769)(308.90862831,1469.45485401)(308.83420842,1469.22042961)
\curveto(308.7225738,1468.86320069)(308.66675745,1468.47248624)(308.6667592,1468.04828508)
\lineto(308.6667592,1464.94489288)
\closepath
}
}
{
\newrgbcolor{curcolor}{0 0 0}
\pscustom[linestyle=none,fillstyle=solid,fillcolor=curcolor]
{
\newpath
\moveto(311.08919153,1466.71427297)
\lineto(312.08272356,1466.8705589)
\curveto(312.13853856,1466.47240034)(312.29389407,1466.16727096)(312.54879055,1465.95516984)
\curveto(312.80368341,1465.7430667)(313.15997778,1465.63701563)(313.61767473,1465.63701632)
\curveto(314.07908702,1465.63701563)(314.4214273,1465.73097315)(314.64469661,1465.91888917)
\curveto(314.86795811,1466.10680325)(314.97959081,1466.32727783)(314.97959505,1466.58031359)
\curveto(314.97959081,1466.80729845)(314.88098192,1466.98591077)(314.6837681,1467.11615109)
\curveto(314.54608382,1467.20545508)(314.20374354,1467.31894833)(313.65674622,1467.45663117)
\curveto(312.91996747,1467.64268316)(312.40924786,1467.80362031)(312.12458586,1467.93944309)
\curveto(311.83992109,1468.07525988)(311.62409787,1468.26317492)(311.47711555,1468.50318879)
\curveto(311.33013175,1468.74319554)(311.25664023,1469.0083232)(311.25664074,1469.29857258)
\curveto(311.25664023,1469.56276562)(311.31710794,1469.80742729)(311.43804407,1470.03255833)
\curveto(311.55897879,1470.25767919)(311.72363703,1470.44466396)(311.93201926,1470.59351321)
\curveto(312.08830385,1470.70886135)(312.30133625,1470.80653997)(312.57111711,1470.88654934)
\curveto(312.84089431,1470.96654684)(313.13020906,1471.00654856)(313.43906223,1471.00655462)
\curveto(313.90419579,1471.00654856)(314.31258542,1470.93956893)(314.66423235,1470.80561555)
\curveto(315.01587144,1470.67165045)(315.27541747,1470.49024731)(315.44287122,1470.26140559)
\curveto(315.61031557,1470.03255324)(315.72566936,1469.72649358)(315.78893294,1469.34322571)
\lineto(314.80656419,1469.20926633)
\curveto(314.76190704,1469.51439145)(314.63259916,1469.75254121)(314.41864017,1469.92371633)
\curveto(314.20467381,1470.0948815)(313.90233524,1470.18046657)(313.51162356,1470.1804718)
\curveto(313.05020562,1470.18046657)(312.72088916,1470.10418422)(312.52367317,1469.95162454)
\curveto(312.32645361,1469.79905484)(312.22784473,1469.62044252)(312.22784622,1469.41578703)
\curveto(312.22784473,1469.28554441)(312.26877672,1469.16833008)(312.35064231,1469.06414367)
\curveto(312.43250468,1468.95622794)(312.56088228,1468.86692178)(312.73577551,1468.79622492)
\curveto(312.83624295,1468.75901017)(313.1320696,1468.6734251)(313.62325638,1468.53946945)
\curveto(314.33398169,1468.34969027)(314.82981693,1468.19433476)(315.11076361,1468.07340246)
\curveto(315.39170153,1467.95246391)(315.61217611,1467.7766424)(315.77218802,1467.54593742)
\curveto(315.93218986,1467.31522724)(316.01219329,1467.0287033)(316.01219857,1466.68636476)
\curveto(316.01219329,1466.35146492)(315.91451468,1466.03610254)(315.71916243,1465.74027667)
\curveto(315.52380023,1465.44444922)(315.24192766,1465.21560218)(314.87354388,1465.05373488)
\curveto(314.50515183,1464.89186735)(314.08838974,1464.81093364)(313.62325638,1464.81093351)
\curveto(312.85298785,1464.81093364)(312.2659859,1464.97094051)(311.86224875,1465.2909546)
\curveto(311.45850936,1465.610968)(311.20082387,1466.08540698)(311.08919153,1466.71427297)
\closepath
}
}
{
\newrgbcolor{curcolor}{0 0 0}
\pscustom[linestyle=none,fillstyle=solid,fillcolor=curcolor]
{
\newpath
\moveto(317.21225226,1471.97217845)
\lineto(317.21225226,1473.12757806)
\lineto(318.21694757,1473.12757806)
\lineto(318.21694757,1471.97217845)
\closepath
\moveto(317.21225226,1464.94489288)
\lineto(317.21225226,1470.87259524)
\lineto(318.21694757,1470.87259524)
\lineto(318.21694757,1464.94489288)
\closepath
}
}
{
\newrgbcolor{curcolor}{0 0 0}
\pscustom[linestyle=none,fillstyle=solid,fillcolor=curcolor]
{
\newpath
\moveto(319.37792775,1467.90874406)
\curveto(319.37792737,1469.00646266)(319.68305675,1469.81952083)(320.29331681,1470.34792102)
\curveto(320.80310485,1470.78700424)(321.42452689,1471.00654856)(322.15758478,1471.00655462)
\curveto(322.97250034,1471.00654856)(323.63857546,1470.73956035)(324.15581213,1470.20558919)
\curveto(324.67303849,1469.6716075)(324.93165425,1468.9339014)(324.93166018,1467.99246867)
\curveto(324.93165425,1467.22964217)(324.81723073,1466.6296164)(324.58838928,1466.19238957)
\curveto(324.35953666,1465.75516024)(324.0264991,1465.41561077)(323.58927561,1465.17374015)
\curveto(323.15204294,1464.93186907)(322.67481314,1464.81093364)(322.15758478,1464.81093351)
\curveto(321.32777855,1464.81093364)(320.65705207,1465.07699158)(320.14540333,1465.60910812)
\curveto(319.63375231,1466.14122333)(319.37792737,1466.90776788)(319.37792775,1467.90874406)
\closepath
\moveto(320.41053126,1467.90874406)
\curveto(320.41052985,1467.14963873)(320.57611836,1466.58124223)(320.90729728,1466.20355285)
\curveto(321.23847239,1465.82586095)(321.65523447,1465.63701563)(322.15758478,1465.63701632)
\curveto(322.65620769,1465.63701563)(323.07110923,1465.82679122)(323.40229065,1466.20634367)
\curveto(323.73346326,1466.58589359)(323.89905176,1467.16452309)(323.89905666,1467.94223391)
\curveto(323.89905176,1468.67528565)(323.73253298,1469.23065833)(323.39949983,1469.60835364)
\curveto(323.06645787,1469.98603961)(322.6524866,1470.17488493)(322.15758478,1470.17489016)
\curveto(321.65523447,1470.17488493)(321.23847239,1469.98696988)(320.90729728,1469.61114446)
\curveto(320.57611836,1469.2353097)(320.41052985,1468.66784347)(320.41053126,1467.90874406)
\closepath
}
}
{
\newrgbcolor{curcolor}{0 0 0}
\pscustom[linestyle=none,fillstyle=solid,fillcolor=curcolor]
{
\newpath
\moveto(326.1149672,1464.94489288)
\lineto(326.1149672,1470.87259524)
\lineto(327.01919299,1470.87259524)
\lineto(327.01919299,1470.02976751)
\curveto(327.45455886,1470.68095318)(328.08342308,1471.00654856)(328.90578752,1471.00655462)
\curveto(329.26300862,1471.00654856)(329.59139482,1470.94235975)(329.8909471,1470.81398801)
\curveto(330.19049031,1470.68560454)(330.41468599,1470.51722522)(330.56353479,1470.30884954)
\curveto(330.71237319,1470.10046313)(330.81656371,1469.85301064)(330.87610667,1469.56649133)
\curveto(330.91331205,1469.38043221)(330.9319175,1469.05483683)(330.93192307,1468.58970422)
\lineto(330.93192307,1464.94489288)
\lineto(329.92722776,1464.94489288)
\lineto(329.92722776,1468.55063274)
\curveto(329.92722319,1468.95994903)(329.88815175,1469.26600869)(329.81001331,1469.46881262)
\curveto(329.73186597,1469.6716075)(329.59325536,1469.83347492)(329.39418108,1469.95441536)
\curveto(329.19509873,1470.07534577)(328.96160033,1470.13581349)(328.69368518,1470.13581868)
\curveto(328.26575649,1470.13581349)(327.89643831,1469.9999937)(327.58572951,1469.72835891)
\curveto(327.27501627,1469.45671455)(327.11966076,1468.94134358)(327.11966252,1468.18224445)
\lineto(327.11966252,1464.94489288)
\closepath
}
}
{
\newrgbcolor{curcolor}{0 0 0}
\pscustom[linestyle=none,fillstyle=solid,fillcolor=curcolor]
{
\newpath
\moveto(335.36932862,1467.08824289)
\lineto(336.4242587,1467.17754914)
\curveto(336.50240006,1466.66403648)(336.6838032,1466.27797339)(336.96846867,1466.01935871)
\curveto(337.25312998,1465.76074187)(337.59640054,1465.631434)(337.99828136,1465.63143468)
\curveto(338.48201996,1465.631434)(338.89133987,1465.81376741)(339.2262423,1466.17843546)
\curveto(339.56113608,1466.54310105)(339.72858513,1467.02684276)(339.72858996,1467.62966203)
\curveto(339.72858513,1468.20270721)(339.56764798,1468.65481965)(339.24577805,1468.98600071)
\curveto(338.92389941,1469.31717368)(338.50248596,1469.48276218)(337.98153644,1469.48276672)
\curveto(337.65779852,1469.48276218)(337.36569295,1469.40927066)(337.10521886,1469.26229192)
\curveto(336.84474035,1469.11530454)(336.6400804,1468.92459868)(336.49123839,1468.69017375)
\lineto(335.54794112,1468.81296985)
\lineto(336.34053409,1473.01594525)
\lineto(340.40955012,1473.01594525)
\lineto(340.40955012,1472.05590306)
\lineto(337.14429035,1472.05590306)
\lineto(336.70334074,1469.85673665)
\curveto(337.19452281,1470.19907202)(337.70989378,1470.37024216)(338.24945519,1470.37024758)
\curveto(338.96390112,1470.37024216)(339.56671771,1470.12278967)(340.05790676,1469.62788938)
\curveto(340.54908548,1469.13297972)(340.79467742,1468.49667332)(340.79468332,1467.71896828)
\curveto(340.79467742,1466.97846859)(340.5788542,1466.3384411)(340.14721301,1465.7988839)
\curveto(339.62253406,1465.13652902)(338.90622423,1464.80535201)(337.99828136,1464.80535187)
\curveto(337.25406025,1464.80535201)(336.6465923,1465.01373305)(336.17587569,1465.43049562)
\curveto(335.70515652,1465.84725722)(335.43630777,1466.39983909)(335.36932862,1467.08824289)
\closepath
}
}
{
\newrgbcolor{curcolor}{0 0 0}
\pscustom[linestyle=none,fillstyle=solid,fillcolor=curcolor]
{
\newpath
\moveto(350.67977013,1473.12757806)
\lineto(351.76260842,1473.12757806)
\lineto(351.76260842,1468.39992844)
\curveto(351.76260108,1467.57756409)(351.66957383,1466.92451278)(351.48352639,1466.44077257)
\curveto(351.29746483,1465.95702937)(350.96163645,1465.5635241)(350.47604025,1465.26025558)
\curveto(349.99043195,1464.95698643)(349.35319528,1464.80535201)(348.56432833,1464.80535187)
\curveto(347.79777965,1464.80535201)(347.17077598,1464.9374507)(346.68331544,1465.20164835)
\curveto(346.19585039,1465.46584549)(345.84792847,1465.84818749)(345.63954864,1466.3486755)
\curveto(345.43116638,1466.84916071)(345.32697586,1467.532911)(345.32697676,1468.39992844)
\lineto(345.32697676,1473.12757806)
\lineto(346.40981505,1473.12757806)
\lineto(346.40981505,1468.40551008)
\curveto(346.40981306,1467.69477842)(346.47586241,1467.171035)(346.60796329,1466.83427824)
\curveto(346.74005981,1466.4975177)(346.9670463,1466.23797167)(347.28892345,1466.05563937)
\curveto(347.61079487,1465.87330485)(348.00430015,1465.78213814)(348.46944044,1465.78213898)
\curveto(349.26574967,1465.78213814)(349.8332159,1465.96261101)(350.17184084,1466.32355812)
\curveto(350.51045429,1466.68450248)(350.67976388,1467.37848577)(350.67977013,1468.40551008)
\closepath
}
}
{
\newrgbcolor{curcolor}{0 0 0}
\pscustom[linestyle=none,fillstyle=solid,fillcolor=curcolor]
{
\newpath
\moveto(353.44268022,1464.94489288)
\lineto(353.44268022,1470.87259524)
\lineto(354.346906,1470.87259524)
\lineto(354.346906,1470.02976751)
\curveto(354.78227188,1470.68095318)(355.41113609,1471.00654856)(356.23350054,1471.00655462)
\curveto(356.59072163,1471.00654856)(356.91910783,1470.94235975)(357.21866011,1470.81398801)
\curveto(357.51820333,1470.68560454)(357.742399,1470.51722522)(357.89124781,1470.30884954)
\curveto(358.0400862,1470.10046313)(358.14427673,1469.85301064)(358.20381968,1469.56649133)
\curveto(358.24102507,1469.38043221)(358.25963052,1469.05483683)(358.25963609,1468.58970422)
\lineto(358.25963609,1464.94489288)
\lineto(357.25494077,1464.94489288)
\lineto(357.25494077,1468.55063274)
\curveto(357.25493621,1468.95994903)(357.21586476,1469.26600869)(357.13772632,1469.46881262)
\curveto(357.05957898,1469.6716075)(356.92096838,1469.83347492)(356.72189409,1469.95441536)
\curveto(356.52281174,1470.07534577)(356.28931334,1470.13581349)(356.02139819,1470.13581868)
\curveto(355.5934695,1470.13581349)(355.22415132,1469.9999937)(354.91344252,1469.72835891)
\curveto(354.60272928,1469.45671455)(354.44737377,1468.94134358)(354.44737553,1468.18224445)
\lineto(354.44737553,1464.94489288)
\closepath
}
}
{
\newrgbcolor{curcolor}{0 0 0}
\pscustom[linestyle=none,fillstyle=solid,fillcolor=curcolor]
{
\newpath
\moveto(359.8113333,1471.97217845)
\lineto(359.8113333,1473.12757806)
\lineto(360.81602861,1473.12757806)
\lineto(360.81602861,1471.97217845)
\closepath
\moveto(359.8113333,1464.94489288)
\lineto(359.8113333,1470.87259524)
\lineto(360.81602861,1470.87259524)
\lineto(360.81602861,1464.94489288)
\closepath
}
}
{
\newrgbcolor{curcolor}{0 0 0}
\pscustom[linestyle=none,fillstyle=solid,fillcolor=curcolor]
{
\newpath
\moveto(361.68118183,1464.94489288)
\lineto(363.8468584,1468.02595852)
\lineto(361.84304941,1470.87259524)
\lineto(363.09891855,1470.87259524)
\lineto(364.00872598,1469.48276672)
\curveto(364.17989371,1469.21856479)(364.31757404,1468.99715993)(364.42176739,1468.81855149)
\curveto(364.58549252,1469.06413956)(364.73619667,1469.28182332)(364.87388028,1469.47160344)
\lineto(365.87299395,1470.87259524)
\lineto(367.07304669,1470.87259524)
\lineto(365.02458457,1468.08177492)
\lineto(367.22933263,1464.94489288)
\lineto(365.99579005,1464.94489288)
\lineto(364.77899239,1466.78683429)
\lineto(364.45525723,1467.28360031)
\lineto(362.89797949,1464.94489288)
\closepath
}
}
{
\newrgbcolor{curcolor}{0 0 0}
\pscustom[linestyle=none,fillstyle=solid,fillcolor=curcolor]
{
\newpath
\moveto(296.14155783,1419.59490204)
\lineto(292.97118595,1427.77758721)
\lineto(294.14333048,1427.77758721)
\lineto(296.26993556,1421.83313993)
\curveto(296.44110236,1421.35683817)(296.58436432,1420.91030737)(296.69972189,1420.49354618)
\curveto(296.82623517,1420.94007609)(296.97321823,1421.38660689)(297.1406715,1421.83313993)
\lineto(299.3510012,1427.77758721)
\lineto(300.45616604,1427.77758721)
\lineto(297.25230432,1419.59490204)
\closepath
}
}
{
\newrgbcolor{curcolor}{0 0 0}
\pscustom[linestyle=none,fillstyle=solid,fillcolor=curcolor]
{
\newpath
\moveto(305.3680099,1421.50382314)
\lineto(306.40619506,1421.3754454)
\curveto(306.24246125,1420.76890595)(305.93919241,1420.29818806)(305.49638764,1419.96329032)
\curveto(305.05357298,1419.62839185)(304.4879673,1419.4609428)(303.79956888,1419.46094266)
\curveto(302.93255166,1419.4609428)(302.24508028,1419.72793101)(301.73715267,1420.2619081)
\curveto(301.22922269,1420.79588385)(300.9752583,1421.54475322)(300.97525872,1422.50851845)
\curveto(300.9752583,1423.50576767)(301.23201351,1424.2797544)(301.74552513,1424.83048096)
\curveto(302.25903436,1425.38119705)(302.92510948,1425.65655771)(303.74375247,1425.65656377)
\curveto(304.53634147,1425.65655771)(305.18381113,1425.38677868)(305.68616342,1424.84722588)
\curveto(306.18850544,1424.30766257)(306.43967902,1423.5485602)(306.4396849,1422.5699165)
\curveto(306.43967902,1422.51037608)(306.43781848,1422.42106992)(306.43410326,1422.30199775)
\lineto(302.01344388,1422.30199775)
\curveto(302.05065332,1421.65080428)(302.23484728,1421.15217822)(302.5660263,1420.80611806)
\curveto(302.8972013,1420.46005547)(303.3102423,1420.28702479)(303.80515052,1420.28702548)
\curveto(304.17353519,1420.28702479)(304.4879673,1420.38377313)(304.74844779,1420.57727079)
\curveto(305.0089199,1420.77076649)(305.2154404,1421.07961696)(305.3680099,1421.50382314)
\closepath
\moveto(302.06926028,1423.12808056)
\lineto(305.37917318,1423.12808056)
\curveto(305.33451528,1423.62670309)(305.20799822,1424.00067264)(304.99962162,1424.24999033)
\curveto(304.67960343,1424.63697904)(304.26470189,1424.83047572)(303.75491576,1424.83048096)
\curveto(303.29349739,1424.83047572)(302.90557376,1424.67605049)(302.59114368,1424.36720478)
\curveto(302.27670954,1424.05834954)(302.10274858,1423.64530855)(302.06926028,1423.12808056)
\closepath
}
}
{
\newrgbcolor{curcolor}{0 0 0}
\pscustom[linestyle=none,fillstyle=solid,fillcolor=curcolor]
{
\newpath
\moveto(307.66206389,1419.59490204)
\lineto(307.66206389,1425.5226044)
\lineto(308.56628967,1425.5226044)
\lineto(308.56628967,1424.62396025)
\curveto(308.79699561,1425.0444384)(309.01002801,1425.32165961)(309.20538752,1425.45562471)
\curveto(309.40074246,1425.58957809)(309.61563541,1425.65655771)(309.85006702,1425.65656377)
\curveto(310.18868328,1425.65655771)(310.53288411,1425.5486461)(310.88267053,1425.33282862)
\lineto(310.53660882,1424.40069463)
\curveto(310.29101326,1424.54581233)(310.04542131,1424.61837359)(309.79983225,1424.61837861)
\curveto(309.58028506,1424.61837359)(309.38306729,1424.55232424)(309.20817834,1424.42023037)
\curveto(309.03328482,1424.28812685)(308.90862831,1424.10486316)(308.83420842,1423.87043877)
\curveto(308.7225738,1423.51320985)(308.66675745,1423.12249539)(308.6667592,1422.69829423)
\lineto(308.6667592,1419.59490204)
\closepath
}
}
{
\newrgbcolor{curcolor}{0 0 0}
\pscustom[linestyle=none,fillstyle=solid,fillcolor=curcolor]
{
\newpath
\moveto(311.08919153,1421.36428212)
\lineto(312.08272356,1421.52056806)
\curveto(312.13853856,1421.1224095)(312.29389407,1420.81728012)(312.54879055,1420.60517899)
\curveto(312.80368341,1420.39307585)(313.15997778,1420.28702479)(313.61767473,1420.28702548)
\curveto(314.07908702,1420.28702479)(314.4214273,1420.38098231)(314.64469661,1420.56889833)
\curveto(314.86795811,1420.7568124)(314.97959081,1420.97728699)(314.97959505,1421.23032275)
\curveto(314.97959081,1421.4573076)(314.88098192,1421.63591992)(314.6837681,1421.76616025)
\curveto(314.54608382,1421.85546424)(314.20374354,1421.96895748)(313.65674622,1422.10664033)
\curveto(312.91996747,1422.29269232)(312.40924786,1422.45362946)(312.12458586,1422.58945224)
\curveto(311.83992109,1422.72526903)(311.62409787,1422.91318408)(311.47711555,1423.15319795)
\curveto(311.33013175,1423.39320469)(311.25664023,1423.65833236)(311.25664074,1423.94858174)
\curveto(311.25664023,1424.21277478)(311.31710794,1424.45743645)(311.43804407,1424.68256748)
\curveto(311.55897879,1424.90768834)(311.72363703,1425.09467311)(311.93201926,1425.24352236)
\curveto(312.08830385,1425.35887051)(312.30133625,1425.45654912)(312.57111711,1425.5365585)
\curveto(312.84089431,1425.61655599)(313.13020906,1425.65655771)(313.43906223,1425.65656377)
\curveto(313.90419579,1425.65655771)(314.31258542,1425.58957809)(314.66423235,1425.45562471)
\curveto(315.01587144,1425.32165961)(315.27541747,1425.14025647)(315.44287122,1424.91141475)
\curveto(315.61031557,1424.68256239)(315.72566936,1424.37650274)(315.78893294,1423.99323486)
\lineto(314.80656419,1423.85927549)
\curveto(314.76190704,1424.16440061)(314.63259916,1424.40255037)(314.41864017,1424.57372549)
\curveto(314.20467381,1424.74489065)(313.90233524,1424.83047572)(313.51162356,1424.83048096)
\curveto(313.05020562,1424.83047572)(312.72088916,1424.75419338)(312.52367317,1424.60163369)
\curveto(312.32645361,1424.44906399)(312.22784473,1424.27045167)(312.22784622,1424.06579619)
\curveto(312.22784473,1423.93555357)(312.26877672,1423.81833923)(312.35064231,1423.71415283)
\curveto(312.43250468,1423.6062371)(312.56088228,1423.51693094)(312.73577551,1423.44623408)
\curveto(312.83624295,1423.40901933)(313.1320696,1423.32343426)(313.62325638,1423.18947861)
\curveto(314.33398169,1422.99969942)(314.82981693,1422.84434391)(315.11076361,1422.72341162)
\curveto(315.39170153,1422.60247306)(315.61217611,1422.42665156)(315.77218802,1422.19594658)
\curveto(315.93218986,1421.96523639)(316.01219329,1421.67871246)(316.01219857,1421.33637392)
\curveto(316.01219329,1421.00147407)(315.91451468,1420.68611169)(315.71916243,1420.39028583)
\curveto(315.52380023,1420.09445838)(315.24192766,1419.86561134)(314.87354388,1419.70374403)
\curveto(314.50515183,1419.54187651)(314.08838974,1419.4609428)(313.62325638,1419.46094266)
\curveto(312.85298785,1419.4609428)(312.2659859,1419.62094967)(311.86224875,1419.94096376)
\curveto(311.45850936,1420.26097716)(311.20082387,1420.73541614)(311.08919153,1421.36428212)
\closepath
}
}
{
\newrgbcolor{curcolor}{0 0 0}
\pscustom[linestyle=none,fillstyle=solid,fillcolor=curcolor]
{
\newpath
\moveto(317.21225226,1426.6221876)
\lineto(317.21225226,1427.77758721)
\lineto(318.21694757,1427.77758721)
\lineto(318.21694757,1426.6221876)
\closepath
\moveto(317.21225226,1419.59490204)
\lineto(317.21225226,1425.5226044)
\lineto(318.21694757,1425.5226044)
\lineto(318.21694757,1419.59490204)
\closepath
}
}
{
\newrgbcolor{curcolor}{0 0 0}
\pscustom[linestyle=none,fillstyle=solid,fillcolor=curcolor]
{
\newpath
\moveto(319.37792775,1422.55875322)
\curveto(319.37792737,1423.65647182)(319.68305675,1424.46952999)(320.29331681,1424.99793018)
\curveto(320.80310485,1425.4370134)(321.42452689,1425.65655771)(322.15758478,1425.65656377)
\curveto(322.97250034,1425.65655771)(323.63857546,1425.3895695)(324.15581213,1424.85559834)
\curveto(324.67303849,1424.32161666)(324.93165425,1423.58391056)(324.93166018,1422.64247783)
\curveto(324.93165425,1421.87965132)(324.81723073,1421.27962555)(324.58838928,1420.84239872)
\curveto(324.35953666,1420.40516939)(324.0264991,1420.06561993)(323.58927561,1419.82374931)
\curveto(323.15204294,1419.58187822)(322.67481314,1419.4609428)(322.15758478,1419.46094266)
\curveto(321.32777855,1419.4609428)(320.65705207,1419.72700074)(320.14540333,1420.25911727)
\curveto(319.63375231,1420.79123249)(319.37792737,1421.55777703)(319.37792775,1422.55875322)
\closepath
\moveto(320.41053126,1422.55875322)
\curveto(320.41052985,1421.79964789)(320.57611836,1421.23125138)(320.90729728,1420.853562)
\curveto(321.23847239,1420.47587011)(321.65523447,1420.28702479)(322.15758478,1420.28702548)
\curveto(322.65620769,1420.28702479)(323.07110923,1420.47680038)(323.40229065,1420.85635282)
\curveto(323.73346326,1421.23590275)(323.89905176,1421.81453225)(323.89905666,1422.59224306)
\curveto(323.89905176,1423.3252948)(323.73253298,1423.88066749)(323.39949983,1424.25836279)
\curveto(323.06645787,1424.63604877)(322.6524866,1424.82489409)(322.15758478,1424.82489932)
\curveto(321.65523447,1424.82489409)(321.23847239,1424.63697904)(320.90729728,1424.26115361)
\curveto(320.57611836,1423.88531885)(320.41052985,1423.31785262)(320.41053126,1422.55875322)
\closepath
}
}
{
\newrgbcolor{curcolor}{0 0 0}
\pscustom[linestyle=none,fillstyle=solid,fillcolor=curcolor]
{
\newpath
\moveto(326.1149672,1419.59490204)
\lineto(326.1149672,1425.5226044)
\lineto(327.01919299,1425.5226044)
\lineto(327.01919299,1424.67977666)
\curveto(327.45455886,1425.33096233)(328.08342308,1425.65655771)(328.90578752,1425.65656377)
\curveto(329.26300862,1425.65655771)(329.59139482,1425.59236891)(329.8909471,1425.46399717)
\curveto(330.19049031,1425.33561369)(330.41468599,1425.16723437)(330.56353479,1424.95885869)
\curveto(330.71237319,1424.75047229)(330.81656371,1424.5030198)(330.87610667,1424.21650049)
\curveto(330.91331205,1424.03044136)(330.9319175,1423.70484599)(330.93192307,1423.23971338)
\lineto(330.93192307,1419.59490204)
\lineto(329.92722776,1419.59490204)
\lineto(329.92722776,1423.20064189)
\curveto(329.92722319,1423.60995819)(329.88815175,1423.91601785)(329.81001331,1424.11882178)
\curveto(329.73186597,1424.32161666)(329.59325536,1424.48348408)(329.39418108,1424.60442451)
\curveto(329.19509873,1424.72535493)(328.96160033,1424.78582264)(328.69368518,1424.78582783)
\curveto(328.26575649,1424.78582264)(327.89643831,1424.65000286)(327.58572951,1424.37836807)
\curveto(327.27501627,1424.10672371)(327.11966076,1423.59135274)(327.11966252,1422.83225361)
\lineto(327.11966252,1419.59490204)
\closepath
}
}
{
\newrgbcolor{curcolor}{0 0 0}
\pscustom[linestyle=none,fillstyle=solid,fillcolor=curcolor]
{
\newpath
\moveto(340.58258098,1425.77377823)
\lineto(339.5834673,1425.69563526)
\curveto(339.49415646,1426.0900647)(339.36763939,1426.37658863)(339.20391574,1426.55520791)
\curveto(338.93227186,1426.84172489)(338.59737376,1426.98498685)(338.19922043,1426.98499424)
\curveto(337.87920338,1426.98498685)(337.59826108,1426.89568069)(337.35639269,1426.71707549)
\curveto(337.04009758,1426.48636079)(336.79078454,1426.14960214)(336.60845284,1425.70679854)
\curveto(336.42611772,1425.26398271)(336.33122992,1424.63325795)(336.32378917,1423.81462236)
\curveto(336.5656586,1424.18300606)(336.86148525,1424.45650617)(337.21127003,1424.63512354)
\curveto(337.56105018,1424.81373082)(337.92757755,1424.90303698)(338.31085324,1424.90304229)
\curveto(338.98064603,1424.90303698)(339.55090308,1424.65651476)(340.02162609,1424.1634749)
\curveto(340.49233886,1423.6704259)(340.7276978,1423.03318923)(340.72770364,1422.25176298)
\curveto(340.7276978,1421.7382499)(340.61699537,1421.2610201)(340.39559602,1420.82007216)
\curveto(340.17418566,1420.37912176)(339.86998655,1420.04143284)(339.48299777,1419.80700438)
\curveto(339.09599982,1419.5725755)(338.6569112,1419.45536116)(338.16573058,1419.45536102)
\curveto(337.32848205,1419.45536116)(336.64566203,1419.76328136)(336.11726847,1420.37912255)
\curveto(335.58887246,1420.99496217)(335.32467507,1422.00988947)(335.3246755,1423.42390752)
\curveto(335.32467507,1425.00536695)(335.61678064,1426.15518378)(336.20099308,1426.87336143)
\curveto(336.71078111,1427.49849728)(337.39732222,1427.81106884)(338.26061847,1427.81107706)
\curveto(338.90436368,1427.81106884)(339.4318282,1427.63059598)(339.84301359,1427.26965792)
\curveto(340.2541891,1426.90870451)(340.50071131,1426.41007844)(340.58258098,1425.77377823)
\closepath
\moveto(336.48007511,1422.24618134)
\curveto(336.48007352,1421.90011732)(336.55356505,1421.5689403)(336.70054991,1421.25264931)
\curveto(336.84753117,1420.936355)(337.05312139,1420.69541442)(337.31732121,1420.52982685)
\curveto(337.58151618,1420.3642374)(337.85873738,1420.28144315)(338.14898566,1420.28144384)
\curveto(338.57318667,1420.28144315)(338.93785349,1420.45261329)(339.24298723,1420.79495478)
\curveto(339.54811226,1421.13729386)(339.70067695,1421.60243011)(339.70068176,1422.19036494)
\curveto(339.70067695,1422.75596803)(339.54997281,1423.20156856)(339.24856887,1423.52716787)
\curveto(338.94715622,1423.85275931)(338.56760504,1424.015557)(338.10991418,1424.01556142)
\curveto(337.65593798,1424.015557)(337.27080516,1423.85275931)(336.95451456,1423.52716787)
\curveto(336.63821985,1423.20156856)(336.48007352,1422.77457348)(336.48007511,1422.24618134)
\closepath
}
}
{
\newrgbcolor{curcolor}{0 0 0}
\pscustom[linestyle=none,fillstyle=solid,fillcolor=curcolor]
{
\newpath
\moveto(350.67977013,1427.77758721)
\lineto(351.76260842,1427.77758721)
\lineto(351.76260842,1423.04993759)
\curveto(351.76260108,1422.22757324)(351.66957383,1421.57452194)(351.48352639,1421.09078173)
\curveto(351.29746483,1420.60703853)(350.96163645,1420.21353326)(350.47604025,1419.91026473)
\curveto(349.99043195,1419.60699558)(349.35319528,1419.45536116)(348.56432833,1419.45536102)
\curveto(347.79777965,1419.45536116)(347.17077598,1419.58745986)(346.68331544,1419.85165751)
\curveto(346.19585039,1420.11585464)(345.84792847,1420.49819665)(345.63954864,1420.99868466)
\curveto(345.43116638,1421.49916987)(345.32697586,1422.18292016)(345.32697676,1423.04993759)
\lineto(345.32697676,1427.77758721)
\lineto(346.40981505,1427.77758721)
\lineto(346.40981505,1423.05551923)
\curveto(346.40981306,1422.34478758)(346.47586241,1421.82104415)(346.60796329,1421.48428739)
\curveto(346.74005981,1421.14752686)(346.9670463,1420.88798083)(347.28892345,1420.70564853)
\curveto(347.61079487,1420.523314)(348.00430015,1420.4321473)(348.46944044,1420.43214813)
\curveto(349.26574967,1420.4321473)(349.8332159,1420.61262016)(350.17184084,1420.97356728)
\curveto(350.51045429,1421.33451163)(350.67976388,1422.02849492)(350.67977013,1423.05551923)
\closepath
}
}
{
\newrgbcolor{curcolor}{0 0 0}
\pscustom[linestyle=none,fillstyle=solid,fillcolor=curcolor]
{
\newpath
\moveto(353.44268022,1419.59490204)
\lineto(353.44268022,1425.5226044)
\lineto(354.346906,1425.5226044)
\lineto(354.346906,1424.67977666)
\curveto(354.78227188,1425.33096233)(355.41113609,1425.65655771)(356.23350054,1425.65656377)
\curveto(356.59072163,1425.65655771)(356.91910783,1425.59236891)(357.21866011,1425.46399717)
\curveto(357.51820333,1425.33561369)(357.742399,1425.16723437)(357.89124781,1424.95885869)
\curveto(358.0400862,1424.75047229)(358.14427673,1424.5030198)(358.20381968,1424.21650049)
\curveto(358.24102507,1424.03044136)(358.25963052,1423.70484599)(358.25963609,1423.23971338)
\lineto(358.25963609,1419.59490204)
\lineto(357.25494077,1419.59490204)
\lineto(357.25494077,1423.20064189)
\curveto(357.25493621,1423.60995819)(357.21586476,1423.91601785)(357.13772632,1424.11882178)
\curveto(357.05957898,1424.32161666)(356.92096838,1424.48348408)(356.72189409,1424.60442451)
\curveto(356.52281174,1424.72535493)(356.28931334,1424.78582264)(356.02139819,1424.78582783)
\curveto(355.5934695,1424.78582264)(355.22415132,1424.65000286)(354.91344252,1424.37836807)
\curveto(354.60272928,1424.10672371)(354.44737377,1423.59135274)(354.44737553,1422.83225361)
\lineto(354.44737553,1419.59490204)
\closepath
}
}
{
\newrgbcolor{curcolor}{0 0 0}
\pscustom[linestyle=none,fillstyle=solid,fillcolor=curcolor]
{
\newpath
\moveto(359.8113333,1426.6221876)
\lineto(359.8113333,1427.77758721)
\lineto(360.81602861,1427.77758721)
\lineto(360.81602861,1426.6221876)
\closepath
\moveto(359.8113333,1419.59490204)
\lineto(359.8113333,1425.5226044)
\lineto(360.81602861,1425.5226044)
\lineto(360.81602861,1419.59490204)
\closepath
}
}
{
\newrgbcolor{curcolor}{0 0 0}
\pscustom[linestyle=none,fillstyle=solid,fillcolor=curcolor]
{
\newpath
\moveto(361.68118183,1419.59490204)
\lineto(363.8468584,1422.67596767)
\lineto(361.84304941,1425.5226044)
\lineto(363.09891855,1425.5226044)
\lineto(364.00872598,1424.13277588)
\curveto(364.17989371,1423.86857395)(364.31757404,1423.64716909)(364.42176739,1423.46856064)
\curveto(364.58549252,1423.71414871)(364.73619667,1423.93183248)(364.87388028,1424.1216126)
\lineto(365.87299395,1425.5226044)
\lineto(367.07304669,1425.5226044)
\lineto(365.02458457,1422.73178408)
\lineto(367.22933263,1419.59490204)
\lineto(365.99579005,1419.59490204)
\lineto(364.77899239,1421.43684345)
\lineto(364.45525723,1421.93360947)
\lineto(362.89797949,1419.59490204)
\closepath
}
}
{
\newrgbcolor{curcolor}{0 0 0}
\pscustom[linestyle=none,fillstyle=solid,fillcolor=curcolor]
{
\newpath
\moveto(296.14155783,1247.23390961)
\lineto(292.97118595,1255.41659478)
\lineto(294.14333048,1255.41659478)
\lineto(296.26993556,1249.4721475)
\curveto(296.44110236,1248.99584574)(296.58436432,1248.54931494)(296.69972189,1248.13255375)
\curveto(296.82623517,1248.57908366)(296.97321823,1249.02561446)(297.1406715,1249.4721475)
\lineto(299.3510012,1255.41659478)
\lineto(300.45616604,1255.41659478)
\lineto(297.25230432,1247.23390961)
\closepath
}
}
{
\newrgbcolor{curcolor}{0 0 0}
\pscustom[linestyle=none,fillstyle=solid,fillcolor=curcolor]
{
\newpath
\moveto(305.3680099,1249.14283071)
\lineto(306.40619506,1249.01445297)
\curveto(306.24246125,1248.40791351)(305.93919241,1247.93719562)(305.49638764,1247.60229789)
\curveto(305.05357298,1247.26739942)(304.4879673,1247.09995037)(303.79956888,1247.09995023)
\curveto(302.93255166,1247.09995037)(302.24508028,1247.36693858)(301.73715267,1247.90091566)
\curveto(301.22922269,1248.43489142)(300.9752583,1249.18376079)(300.97525872,1250.14752602)
\curveto(300.9752583,1251.14477524)(301.23201351,1251.91876196)(301.74552513,1252.46948853)
\curveto(302.25903436,1253.02020462)(302.92510948,1253.29556528)(303.74375247,1253.29557134)
\curveto(304.53634147,1253.29556528)(305.18381113,1253.02578625)(305.68616342,1252.48623345)
\curveto(306.18850544,1251.94667014)(306.43967902,1251.18756777)(306.4396849,1250.20892407)
\curveto(306.43967902,1250.14938365)(306.43781848,1250.06007749)(306.43410326,1249.94100532)
\lineto(302.01344388,1249.94100532)
\curveto(302.05065332,1249.28981185)(302.23484728,1248.79118579)(302.5660263,1248.44512563)
\curveto(302.8972013,1248.09906304)(303.3102423,1247.92603235)(303.80515052,1247.92603305)
\curveto(304.17353519,1247.92603235)(304.4879673,1248.0227807)(304.74844779,1248.21627836)
\curveto(305.0089199,1248.40977406)(305.2154404,1248.71862453)(305.3680099,1249.14283071)
\closepath
\moveto(302.06926028,1250.76708813)
\lineto(305.37917318,1250.76708813)
\curveto(305.33451528,1251.26571066)(305.20799822,1251.63968021)(304.99962162,1251.8889979)
\curveto(304.67960343,1252.27598661)(304.26470189,1252.46948329)(303.75491576,1252.46948853)
\curveto(303.29349739,1252.46948329)(302.90557376,1252.31505805)(302.59114368,1252.00621235)
\curveto(302.27670954,1251.69735711)(302.10274858,1251.28431611)(302.06926028,1250.76708813)
\closepath
}
}
{
\newrgbcolor{curcolor}{0 0 0}
\pscustom[linestyle=none,fillstyle=solid,fillcolor=curcolor]
{
\newpath
\moveto(307.66206389,1247.23390961)
\lineto(307.66206389,1253.16161197)
\lineto(308.56628967,1253.16161197)
\lineto(308.56628967,1252.26296782)
\curveto(308.79699561,1252.68344597)(309.01002801,1252.96066718)(309.20538752,1253.09463228)
\curveto(309.40074246,1253.22858566)(309.61563541,1253.29556528)(309.85006702,1253.29557134)
\curveto(310.18868328,1253.29556528)(310.53288411,1253.18765367)(310.88267053,1252.97183618)
\lineto(310.53660882,1252.0397022)
\curveto(310.29101326,1252.1848199)(310.04542131,1252.25738116)(309.79983225,1252.25738618)
\curveto(309.58028506,1252.25738116)(309.38306729,1252.19133181)(309.20817834,1252.05923794)
\curveto(309.03328482,1251.92713442)(308.90862831,1251.74387073)(308.83420842,1251.50944634)
\curveto(308.7225738,1251.15221742)(308.66675745,1250.76150296)(308.6667592,1250.3373018)
\lineto(308.6667592,1247.23390961)
\closepath
}
}
{
\newrgbcolor{curcolor}{0 0 0}
\pscustom[linestyle=none,fillstyle=solid,fillcolor=curcolor]
{
\newpath
\moveto(311.08919153,1249.00328969)
\lineto(312.08272356,1249.15957563)
\curveto(312.13853856,1248.76141707)(312.29389407,1248.45628768)(312.54879055,1248.24418656)
\curveto(312.80368341,1248.03208342)(313.15997778,1247.92603235)(313.61767473,1247.92603305)
\curveto(314.07908702,1247.92603235)(314.4214273,1248.01998988)(314.64469661,1248.2079059)
\curveto(314.86795811,1248.39581997)(314.97959081,1248.61629456)(314.97959505,1248.86933031)
\curveto(314.97959081,1249.09631517)(314.88098192,1249.27492749)(314.6837681,1249.40516782)
\curveto(314.54608382,1249.49447181)(314.20374354,1249.60796505)(313.65674622,1249.74564789)
\curveto(312.91996747,1249.93169988)(312.40924786,1250.09263703)(312.12458586,1250.22845981)
\curveto(311.83992109,1250.3642766)(311.62409787,1250.55219165)(311.47711555,1250.79220551)
\curveto(311.33013175,1251.03221226)(311.25664023,1251.29733993)(311.25664074,1251.58758931)
\curveto(311.25664023,1251.85178234)(311.31710794,1252.09644401)(311.43804407,1252.32157505)
\curveto(311.55897879,1252.54669591)(311.72363703,1252.73368068)(311.93201926,1252.88252993)
\curveto(312.08830385,1252.99787808)(312.30133625,1253.09555669)(312.57111711,1253.17556607)
\curveto(312.84089431,1253.25556356)(313.13020906,1253.29556528)(313.43906223,1253.29557134)
\curveto(313.90419579,1253.29556528)(314.31258542,1253.22858566)(314.66423235,1253.09463228)
\curveto(315.01587144,1252.96066718)(315.27541747,1252.77926404)(315.44287122,1252.55042232)
\curveto(315.61031557,1252.32156996)(315.72566936,1252.01551031)(315.78893294,1251.63224243)
\lineto(314.80656419,1251.49828305)
\curveto(314.76190704,1251.80340817)(314.63259916,1252.04155794)(314.41864017,1252.21273306)
\curveto(314.20467381,1252.38389822)(313.90233524,1252.46948329)(313.51162356,1252.46948853)
\curveto(313.05020562,1252.46948329)(312.72088916,1252.39320094)(312.52367317,1252.24064126)
\curveto(312.32645361,1252.08807156)(312.22784473,1251.90945924)(312.22784622,1251.70480376)
\curveto(312.22784473,1251.57456114)(312.26877672,1251.4573468)(312.35064231,1251.3531604)
\curveto(312.43250468,1251.24524467)(312.56088228,1251.15593851)(312.73577551,1251.08524165)
\curveto(312.83624295,1251.0480269)(313.1320696,1250.96244183)(313.62325638,1250.82848618)
\curveto(314.33398169,1250.63870699)(314.82981693,1250.48335148)(315.11076361,1250.36241919)
\curveto(315.39170153,1250.24148063)(315.61217611,1250.06565913)(315.77218802,1249.83495414)
\curveto(315.93218986,1249.60424396)(316.01219329,1249.31772003)(316.01219857,1248.97538149)
\curveto(316.01219329,1248.64048164)(315.91451468,1248.32511926)(315.71916243,1248.0292934)
\curveto(315.52380023,1247.73346594)(315.24192766,1247.50461891)(314.87354388,1247.3427516)
\curveto(314.50515183,1247.18088407)(314.08838974,1247.09995037)(313.62325638,1247.09995023)
\curveto(312.85298785,1247.09995037)(312.2659859,1247.25995724)(311.86224875,1247.57997133)
\curveto(311.45850936,1247.89998472)(311.20082387,1248.3744237)(311.08919153,1249.00328969)
\closepath
}
}
{
\newrgbcolor{curcolor}{0 0 0}
\pscustom[linestyle=none,fillstyle=solid,fillcolor=curcolor]
{
\newpath
\moveto(317.21225226,1254.26119517)
\lineto(317.21225226,1255.41659478)
\lineto(318.21694757,1255.41659478)
\lineto(318.21694757,1254.26119517)
\closepath
\moveto(317.21225226,1247.23390961)
\lineto(317.21225226,1253.16161197)
\lineto(318.21694757,1253.16161197)
\lineto(318.21694757,1247.23390961)
\closepath
}
}
{
\newrgbcolor{curcolor}{0 0 0}
\pscustom[linestyle=none,fillstyle=solid,fillcolor=curcolor]
{
\newpath
\moveto(319.37792775,1250.19776079)
\curveto(319.37792737,1251.29547938)(319.68305675,1252.10853756)(320.29331681,1252.63693775)
\curveto(320.80310485,1253.07602097)(321.42452689,1253.29556528)(322.15758478,1253.29557134)
\curveto(322.97250034,1253.29556528)(323.63857546,1253.02857707)(324.15581213,1252.49460591)
\curveto(324.67303849,1251.96062423)(324.93165425,1251.22291813)(324.93166018,1250.2814854)
\curveto(324.93165425,1249.51865889)(324.81723073,1248.91863312)(324.58838928,1248.48140629)
\curveto(324.35953666,1248.04417696)(324.0264991,1247.7046275)(323.58927561,1247.46275687)
\curveto(323.15204294,1247.22088579)(322.67481314,1247.09995037)(322.15758478,1247.09995023)
\curveto(321.32777855,1247.09995037)(320.65705207,1247.3660083)(320.14540333,1247.89812484)
\curveto(319.63375231,1248.43024005)(319.37792737,1249.1967846)(319.37792775,1250.19776079)
\closepath
\moveto(320.41053126,1250.19776079)
\curveto(320.41052985,1249.43865545)(320.57611836,1248.87025895)(320.90729728,1248.49256957)
\curveto(321.23847239,1248.11487767)(321.65523447,1247.92603235)(322.15758478,1247.92603305)
\curveto(322.65620769,1247.92603235)(323.07110923,1248.11580795)(323.40229065,1248.49536039)
\curveto(323.73346326,1248.87491031)(323.89905176,1249.45353981)(323.89905666,1250.23125063)
\curveto(323.89905176,1250.96430237)(323.73253298,1251.51967506)(323.39949983,1251.89737036)
\curveto(323.06645787,1252.27505634)(322.6524866,1252.46390166)(322.15758478,1252.46390689)
\curveto(321.65523447,1252.46390166)(321.23847239,1252.27598661)(320.90729728,1251.90016118)
\curveto(320.57611836,1251.52432642)(320.41052985,1250.95686019)(320.41053126,1250.19776079)
\closepath
}
}
{
\newrgbcolor{curcolor}{0 0 0}
\pscustom[linestyle=none,fillstyle=solid,fillcolor=curcolor]
{
\newpath
\moveto(326.1149672,1247.23390961)
\lineto(326.1149672,1253.16161197)
\lineto(327.01919299,1253.16161197)
\lineto(327.01919299,1252.31878423)
\curveto(327.45455886,1252.9699699)(328.08342308,1253.29556528)(328.90578752,1253.29557134)
\curveto(329.26300862,1253.29556528)(329.59139482,1253.23137648)(329.8909471,1253.10300474)
\curveto(330.19049031,1252.97462126)(330.41468599,1252.80624194)(330.56353479,1252.59786626)
\curveto(330.71237319,1252.38947985)(330.81656371,1252.14202737)(330.87610667,1251.85550806)
\curveto(330.91331205,1251.66944893)(330.9319175,1251.34385355)(330.93192307,1250.87872094)
\lineto(330.93192307,1247.23390961)
\lineto(329.92722776,1247.23390961)
\lineto(329.92722776,1250.83964946)
\curveto(329.92722319,1251.24896576)(329.88815175,1251.55502541)(329.81001331,1251.75782934)
\curveto(329.73186597,1251.96062423)(329.59325536,1252.12249164)(329.39418108,1252.24343208)
\curveto(329.19509873,1252.3643625)(328.96160033,1252.42483021)(328.69368518,1252.4248354)
\curveto(328.26575649,1252.42483021)(327.89643831,1252.28901042)(327.58572951,1252.01737563)
\curveto(327.27501627,1251.74573128)(327.11966076,1251.23036031)(327.11966252,1250.47126118)
\lineto(327.11966252,1247.23390961)
\closepath
}
}
{
\newrgbcolor{curcolor}{0 0 0}
\pscustom[linestyle=none,fillstyle=solid,fillcolor=curcolor]
{
\newpath
\moveto(335.43630831,1254.34491978)
\lineto(335.43630831,1255.31054361)
\lineto(340.73328528,1255.31054361)
\lineto(340.73328528,1254.52911392)
\curveto(340.21232683,1253.97466421)(339.69602559,1253.23788838)(339.18438,1252.31878423)
\curveto(338.67272583,1251.3996699)(338.27736001,1250.45451303)(337.99828136,1249.48331078)
\curveto(337.7973394,1248.79862797)(337.66896179,1248.04882833)(337.61314816,1247.23390961)
\lineto(336.58054464,1247.23390961)
\curveto(336.59170623,1247.87765818)(336.71822329,1248.655366)(336.9600962,1249.56703539)
\curveto(337.20196499,1250.47870012)(337.54895664,1251.35780764)(338.00107218,1252.2043606)
\curveto(338.45318152,1253.05090361)(338.9341324,1253.76442262)(339.44392629,1254.34491978)
\closepath
}
}
{
\newrgbcolor{curcolor}{0 0 0}
\pscustom[linestyle=none,fillstyle=solid,fillcolor=curcolor]
{
\newpath
\moveto(350.67977013,1255.41659478)
\lineto(351.76260842,1255.41659478)
\lineto(351.76260842,1250.68894516)
\curveto(351.76260108,1249.86658081)(351.66957383,1249.21352951)(351.48352639,1248.7297893)
\curveto(351.29746483,1248.2460461)(350.96163645,1247.85254083)(350.47604025,1247.5492723)
\curveto(349.99043195,1247.24600315)(349.35319528,1247.09436873)(348.56432833,1247.09436859)
\curveto(347.79777965,1247.09436873)(347.17077598,1247.22646743)(346.68331544,1247.49066508)
\curveto(346.19585039,1247.75486221)(345.84792847,1248.13720421)(345.63954864,1248.63769223)
\curveto(345.43116638,1249.13817743)(345.32697586,1249.82192773)(345.32697676,1250.68894516)
\lineto(345.32697676,1255.41659478)
\lineto(346.40981505,1255.41659478)
\lineto(346.40981505,1250.6945268)
\curveto(346.40981306,1249.98379515)(346.47586241,1249.46005172)(346.60796329,1249.12329496)
\curveto(346.74005981,1248.78653443)(346.9670463,1248.5269884)(347.28892345,1248.34465609)
\curveto(347.61079487,1248.16232157)(348.00430015,1248.07115487)(348.46944044,1248.0711557)
\curveto(349.26574967,1248.07115487)(349.8332159,1248.25162773)(350.17184084,1248.61257484)
\curveto(350.51045429,1248.9735192)(350.67976388,1249.66750249)(350.67977013,1250.6945268)
\closepath
}
}
{
\newrgbcolor{curcolor}{0 0 0}
\pscustom[linestyle=none,fillstyle=solid,fillcolor=curcolor]
{
\newpath
\moveto(353.44268022,1247.23390961)
\lineto(353.44268022,1253.16161197)
\lineto(354.346906,1253.16161197)
\lineto(354.346906,1252.31878423)
\curveto(354.78227188,1252.9699699)(355.41113609,1253.29556528)(356.23350054,1253.29557134)
\curveto(356.59072163,1253.29556528)(356.91910783,1253.23137648)(357.21866011,1253.10300474)
\curveto(357.51820333,1252.97462126)(357.742399,1252.80624194)(357.89124781,1252.59786626)
\curveto(358.0400862,1252.38947985)(358.14427673,1252.14202737)(358.20381968,1251.85550806)
\curveto(358.24102507,1251.66944893)(358.25963052,1251.34385355)(358.25963609,1250.87872094)
\lineto(358.25963609,1247.23390961)
\lineto(357.25494077,1247.23390961)
\lineto(357.25494077,1250.83964946)
\curveto(357.25493621,1251.24896576)(357.21586476,1251.55502541)(357.13772632,1251.75782934)
\curveto(357.05957898,1251.96062423)(356.92096838,1252.12249164)(356.72189409,1252.24343208)
\curveto(356.52281174,1252.3643625)(356.28931334,1252.42483021)(356.02139819,1252.4248354)
\curveto(355.5934695,1252.42483021)(355.22415132,1252.28901042)(354.91344252,1252.01737563)
\curveto(354.60272928,1251.74573128)(354.44737377,1251.23036031)(354.44737553,1250.47126118)
\lineto(354.44737553,1247.23390961)
\closepath
}
}
{
\newrgbcolor{curcolor}{0 0 0}
\pscustom[linestyle=none,fillstyle=solid,fillcolor=curcolor]
{
\newpath
\moveto(359.8113333,1254.26119517)
\lineto(359.8113333,1255.41659478)
\lineto(360.81602861,1255.41659478)
\lineto(360.81602861,1254.26119517)
\closepath
\moveto(359.8113333,1247.23390961)
\lineto(359.8113333,1253.16161197)
\lineto(360.81602861,1253.16161197)
\lineto(360.81602861,1247.23390961)
\closepath
}
}
{
\newrgbcolor{curcolor}{0 0 0}
\pscustom[linestyle=none,fillstyle=solid,fillcolor=curcolor]
{
\newpath
\moveto(361.68118183,1247.23390961)
\lineto(363.8468584,1250.31497524)
\lineto(361.84304941,1253.16161197)
\lineto(363.09891855,1253.16161197)
\lineto(364.00872598,1251.77178345)
\curveto(364.17989371,1251.50758152)(364.31757404,1251.28617666)(364.42176739,1251.10756821)
\curveto(364.58549252,1251.35315628)(364.73619667,1251.57084005)(364.87388028,1251.76062016)
\lineto(365.87299395,1253.16161197)
\lineto(367.07304669,1253.16161197)
\lineto(365.02458457,1250.37079165)
\lineto(367.22933263,1247.23390961)
\lineto(365.99579005,1247.23390961)
\lineto(364.77899239,1249.07585102)
\lineto(364.45525723,1249.57261703)
\lineto(362.89797949,1247.23390961)
\closepath
}
}
{
\newrgbcolor{curcolor}{0 0 0}
\pscustom[linestyle=none,fillstyle=solid,fillcolor=curcolor]
{
\newpath
\moveto(449.74955954,982.39992523)
\lineto(446.57918765,990.58261041)
\lineto(447.75133219,990.58261041)
\lineto(449.87793727,984.63816313)
\curveto(450.04910406,984.16186137)(450.19236603,983.71533056)(450.3077236,983.29856938)
\curveto(450.43423688,983.74509928)(450.58121994,984.19163009)(450.74867321,984.63816313)
\lineto(452.9590029,990.58261041)
\lineto(454.06416775,990.58261041)
\lineto(450.86030602,982.39992523)
\closepath
}
}
{
\newrgbcolor{curcolor}{0 0 0}
\pscustom[linestyle=none,fillstyle=solid,fillcolor=curcolor]
{
\newpath
\moveto(458.97601161,984.30884633)
\lineto(460.01419677,984.1804686)
\curveto(459.85046296,983.57392914)(459.54719412,983.10321125)(459.10438934,982.76831351)
\curveto(458.66157469,982.43341504)(458.095969,982.26596599)(457.40757059,982.26596586)
\curveto(456.54055337,982.26596599)(455.85308198,982.5329542)(455.34515437,983.06693129)
\curveto(454.8372244,983.60090704)(454.58326001,984.34977641)(454.58326043,985.31354165)
\curveto(454.58326001,986.31079086)(454.84001522,987.08477759)(455.35352684,987.63550415)
\curveto(455.86703607,988.18622024)(456.53311119,988.4615809)(457.35175418,988.46158697)
\curveto(458.14434317,988.4615809)(458.79181284,988.19180188)(459.29416513,987.65224907)
\curveto(459.79650715,987.11268577)(460.04768073,986.3535834)(460.04768661,985.37493969)
\curveto(460.04768073,985.31539928)(460.04582018,985.22609312)(460.04210497,985.10702094)
\lineto(455.62144559,985.10702094)
\curveto(455.65865503,984.45582748)(455.84284899,983.95720141)(456.17402801,983.61114125)
\curveto(456.50520301,983.26507867)(456.91824401,983.09204798)(457.41315223,983.09204867)
\curveto(457.7815369,983.09204798)(458.095969,983.18879632)(458.3564495,983.38229398)
\curveto(458.61692161,983.57578968)(458.82344211,983.88464016)(458.97601161,984.30884633)
\closepath
\moveto(455.67726199,985.93310376)
\lineto(458.98717489,985.93310376)
\curveto(458.94251699,986.43172629)(458.81599993,986.80569584)(458.60762333,987.05501352)
\curveto(458.28760514,987.44200223)(457.8727036,987.63549892)(457.36291747,987.63550415)
\curveto(456.9014991,987.63549892)(456.51357547,987.48107368)(456.19914539,987.17222798)
\curveto(455.88471125,986.86337273)(455.71075029,986.45033174)(455.67726199,985.93310376)
\closepath
}
}
{
\newrgbcolor{curcolor}{0 0 0}
\pscustom[linestyle=none,fillstyle=solid,fillcolor=curcolor]
{
\newpath
\moveto(461.2700656,982.39992523)
\lineto(461.2700656,988.32762759)
\lineto(462.17429138,988.32762759)
\lineto(462.17429138,987.42898345)
\curveto(462.40499731,987.84946159)(462.61802972,988.1266828)(462.81338923,988.2606479)
\curveto(463.00874417,988.39460128)(463.22363712,988.4615809)(463.45806873,988.46158697)
\curveto(463.79668499,988.4615809)(464.14088582,988.35366929)(464.49067224,988.13785181)
\lineto(464.14461052,987.20571782)
\curveto(463.89901496,987.35083553)(463.65342302,987.42339678)(463.40783396,987.42340181)
\curveto(463.18828677,987.42339678)(462.991069,987.35734744)(462.81618005,987.22525356)
\curveto(462.64128653,987.09315004)(462.51663002,986.90988636)(462.44221013,986.67546196)
\curveto(462.33057551,986.31823304)(462.27475916,985.92751859)(462.27476091,985.50331743)
\lineto(462.27476091,982.39992523)
\closepath
}
}
{
\newrgbcolor{curcolor}{0 0 0}
\pscustom[linestyle=none,fillstyle=solid,fillcolor=curcolor]
{
\newpath
\moveto(464.69719323,984.16930531)
\lineto(465.69072527,984.32559125)
\curveto(465.74654027,983.92743269)(465.90189578,983.62230331)(466.15679226,983.41020219)
\curveto(466.41168512,983.19809905)(466.76797949,983.09204798)(467.22567644,983.09204867)
\curveto(467.68708873,983.09204798)(468.02942901,983.1860055)(468.25269832,983.37392152)
\curveto(468.47595982,983.5618356)(468.58759252,983.78231018)(468.58759676,984.03534594)
\curveto(468.58759252,984.2623308)(468.48898363,984.44094312)(468.29176981,984.57118344)
\curveto(468.15408553,984.66048743)(467.81174524,984.77398068)(467.26474793,984.91166352)
\curveto(466.52796918,985.09771551)(466.01724957,985.25865265)(465.73258757,985.39447543)
\curveto(465.4479228,985.53029223)(465.23209958,985.71820727)(465.08511726,985.95822114)
\curveto(464.93813346,986.19822789)(464.86464193,986.46335555)(464.86464245,986.75360493)
\curveto(464.86464193,987.01779797)(464.92510965,987.26245964)(465.04604577,987.48759067)
\curveto(465.1669805,987.71271153)(465.33163873,987.89969631)(465.54002097,988.04854556)
\curveto(465.69630556,988.1638937)(465.90933796,988.26157231)(466.17911882,988.34158169)
\curveto(466.44889602,988.42157919)(466.73821077,988.4615809)(467.04706394,988.46158697)
\curveto(467.5121975,988.4615809)(467.92058713,988.39460128)(468.27223406,988.2606479)
\curveto(468.62387315,988.1266828)(468.88341918,987.94527966)(469.05087293,987.71643794)
\curveto(469.21831728,987.48758559)(469.33367107,987.18152593)(469.39693465,986.79825806)
\lineto(468.4145659,986.66429868)
\curveto(468.36990875,986.9694238)(468.24060087,987.20757356)(468.02664188,987.37874868)
\curveto(467.81267552,987.54991384)(467.51033695,987.63549892)(467.11962527,987.63550415)
\curveto(466.65820733,987.63549892)(466.32889086,987.55921657)(466.13167488,987.40665688)
\curveto(465.93445532,987.25408719)(465.83584643,987.07547486)(465.83584792,986.87081938)
\curveto(465.83584643,986.74057676)(465.87677842,986.62336243)(465.95864402,986.51917602)
\curveto(466.04050639,986.41126029)(466.16888399,986.32195413)(466.34377722,986.25125727)
\curveto(466.44424466,986.21404252)(466.74007131,986.12845745)(467.23125808,985.9945018)
\curveto(467.9419834,985.80472262)(468.43781864,985.64936711)(468.71876531,985.52843481)
\curveto(468.99970324,985.40749626)(469.22017782,985.23167475)(469.38018973,985.00096977)
\curveto(469.54019157,984.77025959)(469.620195,984.48373565)(469.62020028,984.14139711)
\curveto(469.620195,983.80649727)(469.52251639,983.49113489)(469.32716414,983.19530902)
\curveto(469.13180194,982.89948157)(468.84992936,982.67063453)(468.48154559,982.50876722)
\curveto(468.11315354,982.3468997)(467.69639145,982.26596599)(467.23125808,982.26596586)
\curveto(466.46098956,982.26596599)(465.87398761,982.42597286)(465.47025046,982.74598695)
\curveto(465.06651107,983.06600035)(464.80882558,983.54043933)(464.69719323,984.16930531)
\closepath
}
}
{
\newrgbcolor{curcolor}{0 0 0}
\pscustom[linestyle=none,fillstyle=solid,fillcolor=curcolor]
{
\newpath
\moveto(470.82025397,989.4272108)
\lineto(470.82025397,990.58261041)
\lineto(471.82494928,990.58261041)
\lineto(471.82494928,989.4272108)
\closepath
\moveto(470.82025397,982.39992523)
\lineto(470.82025397,988.32762759)
\lineto(471.82494928,988.32762759)
\lineto(471.82494928,982.39992523)
\closepath
}
}
{
\newrgbcolor{curcolor}{0 0 0}
\pscustom[linestyle=none,fillstyle=solid,fillcolor=curcolor]
{
\newpath
\moveto(472.98592946,985.36377641)
\curveto(472.98592908,986.46149501)(473.29105846,987.27455318)(473.90131852,987.80295337)
\curveto(474.41110656,988.24203659)(475.0325286,988.4615809)(475.76558649,988.46158697)
\curveto(476.58050205,988.4615809)(477.24657717,988.19459269)(477.76381384,987.66062153)
\curveto(478.2810402,987.12663985)(478.53965596,986.38893375)(478.53966189,985.44750102)
\curveto(478.53965596,984.68467452)(478.42523244,984.08464875)(478.19639099,983.64742191)
\curveto(477.96753836,983.21019259)(477.63450081,982.87064312)(477.19727732,982.6287725)
\curveto(476.76004465,982.38690142)(476.28281485,982.26596599)(475.76558649,982.26596586)
\curveto(474.93578026,982.26596599)(474.26505378,982.53202393)(473.75340504,983.06414047)
\curveto(473.24175402,983.59625568)(472.98592908,984.36280023)(472.98592946,985.36377641)
\closepath
\moveto(474.01853297,985.36377641)
\curveto(474.01853156,984.60467108)(474.18412007,984.03627458)(474.51529899,983.6585852)
\curveto(474.84647409,983.2808933)(475.26323618,983.09204798)(475.76558649,983.09204867)
\curveto(476.2642094,983.09204798)(476.67911094,983.28182357)(477.01029236,983.66137602)
\curveto(477.34146497,984.04092594)(477.50705347,984.61955544)(477.50705837,985.39726626)
\curveto(477.50705347,986.130318)(477.34053469,986.68569068)(477.00750154,987.06338599)
\curveto(476.67445958,987.44107196)(476.26048831,987.62991728)(475.76558649,987.62992251)
\curveto(475.26323618,987.62991728)(474.84647409,987.44200223)(474.51529899,987.06617681)
\curveto(474.18412007,986.69034205)(474.01853156,986.12287582)(474.01853297,985.36377641)
\closepath
}
}
{
\newrgbcolor{curcolor}{0 0 0}
\pscustom[linestyle=none,fillstyle=solid,fillcolor=curcolor]
{
\newpath
\moveto(479.72296891,982.39992523)
\lineto(479.72296891,988.32762759)
\lineto(480.6271947,988.32762759)
\lineto(480.6271947,987.48479985)
\curveto(481.06256057,988.13598553)(481.69142479,988.4615809)(482.51378923,988.46158697)
\curveto(482.87101033,988.4615809)(483.19939653,988.3973921)(483.4989488,988.26902036)
\curveto(483.79849202,988.14063689)(484.0226877,987.97225756)(484.1715365,987.76388189)
\curveto(484.3203749,987.55549548)(484.42456542,987.30804299)(484.48410838,987.02152368)
\curveto(484.52131376,986.83546456)(484.53991921,986.50986918)(484.53992478,986.04473657)
\lineto(484.53992478,982.39992523)
\lineto(483.53522947,982.39992523)
\lineto(483.53522947,986.00566508)
\curveto(483.5352249,986.41498138)(483.49615346,986.72104104)(483.41801502,986.92384497)
\curveto(483.33986768,987.12663985)(483.20125707,987.28850727)(483.00218279,987.40944771)
\curveto(482.80310044,987.53037812)(482.56960204,987.59084584)(482.30168689,987.59085103)
\curveto(481.8737582,987.59084584)(481.50444001,987.45502605)(481.19373122,987.18339126)
\curveto(480.88301798,986.9117469)(480.72766247,986.39637593)(480.72766423,985.6372768)
\lineto(480.72766423,982.39992523)
\closepath
}
}
{
\newrgbcolor{curcolor}{0 0 0}
\pscustom[linestyle=none,fillstyle=solid,fillcolor=curcolor]
{
\newpath
\moveto(490.52344479,986.83732954)
\curveto(490.10668068,986.98988979)(489.79783021,987.20757356)(489.59689244,987.49038149)
\curveto(489.39595249,987.77317925)(489.29548306,988.11179844)(489.29548385,988.50624009)
\curveto(489.29548306,989.10160839)(489.50944573,989.602095)(489.93737252,990.00770142)
\curveto(490.36529644,990.41329263)(490.93462322,990.61609204)(491.64535456,990.61610025)
\curveto(492.3598007,990.61609204)(492.93470911,990.40864127)(493.37008151,989.99374732)
\curveto(493.80544418,989.57883819)(494.02312795,989.07370022)(494.02313347,988.47833189)
\curveto(494.02312795,988.09877463)(493.92358879,987.76852788)(493.72451569,987.48759067)
\curveto(493.52543216,987.20664329)(493.22309359,986.98988979)(492.81749909,986.83732954)
\curveto(493.31984193,986.67359714)(493.70218393,986.40939975)(493.96452624,986.04473657)
\curveto(494.22685763,985.6800661)(494.35802605,985.24469857)(494.35803191,984.73863266)
\curveto(494.35802605,984.03906539)(494.11057356,983.45113317)(493.6156737,982.97483422)
\curveto(493.12076361,982.49853412)(492.46957286,982.26038436)(491.66209948,982.26038422)
\curveto(490.85461978,982.26038436)(490.20342902,982.49946439)(489.70852526,982.97762504)
\curveto(489.21361907,983.45578453)(488.96616659,984.05208921)(488.96616705,984.76654086)
\curveto(488.96616659,985.29865437)(489.1010561,985.7442549)(489.370836,986.1033438)
\curveto(489.64061416,986.46242528)(490.0248167,986.70708695)(490.52344479,986.83732954)
\closepath
\moveto(490.32250573,988.53972993)
\curveto(490.32250391,988.15273043)(490.44716042,987.83643778)(490.69647565,987.59085103)
\curveto(490.94578649,987.34525389)(491.26952132,987.22245792)(491.66768112,987.22246274)
\curveto(492.05467132,987.22245792)(492.37189424,987.34432362)(492.61935085,987.58806021)
\curveto(492.86679922,987.83178641)(492.99052546,988.13040389)(492.99052995,988.48391353)
\curveto(492.99052546,988.85229536)(492.86307813,989.1620761)(492.60818757,989.41325669)
\curveto(492.35328879,989.66442326)(492.03606587,989.79001005)(491.65651784,989.79001744)
\curveto(491.27324241,989.79001005)(490.95508921,989.66721408)(490.70205729,989.42162916)
\curveto(490.44902097,989.17603019)(490.32250391,988.88206408)(490.32250573,988.53972993)
\closepath
\moveto(489.99877057,984.76095922)
\curveto(489.99876907,984.47443293)(490.06667897,984.19721172)(490.20250045,983.92929477)
\curveto(490.33831854,983.66137476)(490.54018767,983.45392399)(490.80810846,983.30694184)
\curveto(491.07602464,983.15995787)(491.36440912,983.08646634)(491.67326276,983.08646703)
\curveto(492.1532802,983.08646634)(492.54957629,983.24089158)(492.86215222,983.5497432)
\curveto(493.17471942,983.85859253)(493.3310052,984.25116753)(493.33101003,984.72746938)
\curveto(493.3310052,985.21120876)(493.17006806,985.61122593)(492.84819811,985.92752212)
\curveto(492.52631948,986.24381124)(492.12351148,986.40195757)(491.63977292,986.40196157)
\curveto(491.16719134,986.40195757)(490.77554662,986.24567179)(490.46483756,985.93310376)
\curveto(490.15412458,985.62052866)(489.99876907,985.22981421)(489.99877057,984.76095922)
\closepath
}
}
{
\newrgbcolor{curcolor}{0 0 0}
\pscustom[linestyle=none,fillstyle=solid,fillcolor=curcolor]
{
\newpath
\moveto(504.28777184,990.58261041)
\lineto(505.37061013,990.58261041)
\lineto(505.37061013,985.85496079)
\curveto(505.37060279,985.03259643)(505.27757554,984.37954513)(505.0915281,983.89580492)
\curveto(504.90546654,983.41206172)(504.56963816,983.01855645)(504.08404196,982.71528793)
\curveto(503.59843366,982.41201877)(502.96119699,982.26038436)(502.17233004,982.26038422)
\curveto(501.40578136,982.26038436)(500.77877769,982.39248305)(500.29131715,982.6566807)
\curveto(499.8038521,982.92087784)(499.45593018,983.30321984)(499.24755035,983.80370785)
\curveto(499.03916809,984.30419306)(498.93497757,984.98794335)(498.93497847,985.85496079)
\lineto(498.93497847,990.58261041)
\lineto(500.01781675,990.58261041)
\lineto(500.01781675,985.86054243)
\curveto(500.01781477,985.14981077)(500.08386412,984.62606735)(500.215965,984.28931059)
\curveto(500.34806151,983.95255005)(500.57504801,983.69300402)(500.89692516,983.51067172)
\curveto(501.21879658,983.3283372)(501.61230185,983.23717049)(502.07744215,983.23717133)
\curveto(502.87375138,983.23717049)(503.44121761,983.41764336)(503.77984255,983.77859047)
\curveto(504.118456,984.13953482)(504.28776559,984.83351812)(504.28777184,985.86054243)
\closepath
}
}
{
\newrgbcolor{curcolor}{0 0 0}
\pscustom[linestyle=none,fillstyle=solid,fillcolor=curcolor]
{
\newpath
\moveto(507.05068193,982.39992523)
\lineto(507.05068193,988.32762759)
\lineto(507.95490771,988.32762759)
\lineto(507.95490771,987.48479985)
\curveto(508.39027359,988.13598553)(509.0191378,988.4615809)(509.84150224,988.46158697)
\curveto(510.19872334,988.4615809)(510.52710954,988.3973921)(510.82666182,988.26902036)
\curveto(511.12620504,988.14063689)(511.35040071,987.97225756)(511.49924951,987.76388189)
\curveto(511.64808791,987.55549548)(511.75227843,987.30804299)(511.81182139,987.02152368)
\curveto(511.84902678,986.83546456)(511.86763223,986.50986918)(511.8676378,986.04473657)
\lineto(511.8676378,982.39992523)
\lineto(510.86294248,982.39992523)
\lineto(510.86294248,986.00566508)
\curveto(510.86293792,986.41498138)(510.82386647,986.72104104)(510.74572803,986.92384497)
\curveto(510.66758069,987.12663985)(510.52897008,987.28850727)(510.3298958,987.40944771)
\curveto(510.13081345,987.53037812)(509.89731505,987.59084584)(509.6293999,987.59085103)
\curveto(509.20147121,987.59084584)(508.83215303,987.45502605)(508.52144423,987.18339126)
\curveto(508.21073099,986.9117469)(508.05537548,986.39637593)(508.05537724,985.6372768)
\lineto(508.05537724,982.39992523)
\closepath
}
}
{
\newrgbcolor{curcolor}{0 0 0}
\pscustom[linestyle=none,fillstyle=solid,fillcolor=curcolor]
{
\newpath
\moveto(513.41933501,989.4272108)
\lineto(513.41933501,990.58261041)
\lineto(514.42403032,990.58261041)
\lineto(514.42403032,989.4272108)
\closepath
\moveto(513.41933501,982.39992523)
\lineto(513.41933501,988.32762759)
\lineto(514.42403032,988.32762759)
\lineto(514.42403032,982.39992523)
\closepath
}
}
{
\newrgbcolor{curcolor}{0 0 0}
\pscustom[linestyle=none,fillstyle=solid,fillcolor=curcolor]
{
\newpath
\moveto(515.28918354,982.39992523)
\lineto(517.45486011,985.48099086)
\lineto(515.45105112,988.32762759)
\lineto(516.70692026,988.32762759)
\lineto(517.61672769,986.93779907)
\curveto(517.78789542,986.67359714)(517.92557575,986.45219228)(518.02976909,986.27358384)
\curveto(518.19349423,986.5191719)(518.34419838,986.73685567)(518.48188199,986.92663579)
\lineto(519.48099566,988.32762759)
\lineto(520.6810484,988.32762759)
\lineto(518.63258628,985.53680727)
\lineto(520.83733434,982.39992523)
\lineto(519.60379175,982.39992523)
\lineto(518.3869941,984.24186664)
\lineto(518.06325894,984.73863266)
\lineto(516.5059812,982.39992523)
\closepath
}
}
{
\newrgbcolor{curcolor}{0 0 0}
\pscustom[linestyle=none,fillstyle=solid,fillcolor=curcolor]
{
\newpath
\moveto(449.74955954,937.33290863)
\lineto(446.57918765,945.51559381)
\lineto(447.75133219,945.51559381)
\lineto(449.87793727,939.57114653)
\curveto(450.04910406,939.09484476)(450.19236603,938.64831396)(450.3077236,938.23155277)
\curveto(450.43423688,938.67808268)(450.58121994,939.12461348)(450.74867321,939.57114653)
\lineto(452.9590029,945.51559381)
\lineto(454.06416775,945.51559381)
\lineto(450.86030602,937.33290863)
\closepath
}
}
{
\newrgbcolor{curcolor}{0 0 0}
\pscustom[linestyle=none,fillstyle=solid,fillcolor=curcolor]
{
\newpath
\moveto(458.97601161,939.24182973)
\lineto(460.01419677,939.11345199)
\curveto(459.85046296,938.50691254)(459.54719412,938.03619465)(459.10438934,937.70129691)
\curveto(458.66157469,937.36639844)(458.095969,937.19894939)(457.40757059,937.19894926)
\curveto(456.54055337,937.19894939)(455.85308198,937.4659376)(455.34515437,937.99991469)
\curveto(454.8372244,938.53389044)(454.58326001,939.28275981)(454.58326043,940.24652504)
\curveto(454.58326001,941.24377426)(454.84001522,942.01776099)(455.35352684,942.56848755)
\curveto(455.86703607,943.11920364)(456.53311119,943.3945643)(457.35175418,943.39457036)
\curveto(458.14434317,943.3945643)(458.79181284,943.12478527)(459.29416513,942.58523247)
\curveto(459.79650715,942.04566916)(460.04768073,941.2865668)(460.04768661,940.30792309)
\curveto(460.04768073,940.24838268)(460.04582018,940.15907651)(460.04210497,940.04000434)
\lineto(455.62144559,940.04000434)
\curveto(455.65865503,939.38881088)(455.84284899,938.89018481)(456.17402801,938.54412465)
\curveto(456.50520301,938.19806206)(456.91824401,938.02503138)(457.41315223,938.02503207)
\curveto(457.7815369,938.02503138)(458.095969,938.12177972)(458.3564495,938.31527738)
\curveto(458.61692161,938.50877308)(458.82344211,938.81762356)(458.97601161,939.24182973)
\closepath
\moveto(455.67726199,940.86608715)
\lineto(458.98717489,940.86608715)
\curveto(458.94251699,941.36470969)(458.81599993,941.73867924)(458.60762333,941.98799692)
\curveto(458.28760514,942.37498563)(457.8727036,942.56848231)(457.36291747,942.56848755)
\curveto(456.9014991,942.56848231)(456.51357547,942.41405708)(456.19914539,942.10521138)
\curveto(455.88471125,941.79635613)(455.71075029,941.38331514)(455.67726199,940.86608715)
\closepath
}
}
{
\newrgbcolor{curcolor}{0 0 0}
\pscustom[linestyle=none,fillstyle=solid,fillcolor=curcolor]
{
\newpath
\moveto(461.2700656,937.33290863)
\lineto(461.2700656,943.26061099)
\lineto(462.17429138,943.26061099)
\lineto(462.17429138,942.36196685)
\curveto(462.40499731,942.78244499)(462.61802972,943.0596662)(462.81338923,943.1936313)
\curveto(463.00874417,943.32758468)(463.22363712,943.3945643)(463.45806873,943.39457036)
\curveto(463.79668499,943.3945643)(464.14088582,943.28665269)(464.49067224,943.07083521)
\lineto(464.14461052,942.13870122)
\curveto(463.89901496,942.28381893)(463.65342302,942.35638018)(463.40783396,942.35638521)
\curveto(463.18828677,942.35638018)(462.991069,942.29033083)(462.81618005,942.15823696)
\curveto(462.64128653,942.02613344)(462.51663002,941.84286976)(462.44221013,941.60844536)
\curveto(462.33057551,941.25121644)(462.27475916,940.86050199)(462.27476091,940.43630083)
\lineto(462.27476091,937.33290863)
\closepath
}
}
{
\newrgbcolor{curcolor}{0 0 0}
\pscustom[linestyle=none,fillstyle=solid,fillcolor=curcolor]
{
\newpath
\moveto(464.69719323,939.10228871)
\lineto(465.69072527,939.25857465)
\curveto(465.74654027,938.86041609)(465.90189578,938.55528671)(466.15679226,938.34318559)
\curveto(466.41168512,938.13108244)(466.76797949,938.02503138)(467.22567644,938.02503207)
\curveto(467.68708873,938.02503138)(468.02942901,938.1189889)(468.25269832,938.30690492)
\curveto(468.47595982,938.49481899)(468.58759252,938.71529358)(468.58759676,938.96832934)
\curveto(468.58759252,939.19531419)(468.48898363,939.37392652)(468.29176981,939.50416684)
\curveto(468.15408553,939.59347083)(467.81174524,939.70696407)(467.26474793,939.84464692)
\curveto(466.52796918,940.03069891)(466.01724957,940.19163605)(465.73258757,940.32745883)
\curveto(465.4479228,940.46327562)(465.23209958,940.65119067)(465.08511726,940.89120454)
\curveto(464.93813346,941.13121129)(464.86464193,941.39633895)(464.86464245,941.68658833)
\curveto(464.86464193,941.95078137)(464.92510965,942.19544304)(465.04604577,942.42057407)
\curveto(465.1669805,942.64569493)(465.33163873,942.83267971)(465.54002097,942.98152896)
\curveto(465.69630556,943.0968771)(465.90933796,943.19455571)(466.17911882,943.27456509)
\curveto(466.44889602,943.35456258)(466.73821077,943.3945643)(467.04706394,943.39457036)
\curveto(467.5121975,943.3945643)(467.92058713,943.32758468)(468.27223406,943.1936313)
\curveto(468.62387315,943.0596662)(468.88341918,942.87826306)(469.05087293,942.64942134)
\curveto(469.21831728,942.42056898)(469.33367107,942.11450933)(469.39693465,941.73124145)
\lineto(468.4145659,941.59728208)
\curveto(468.36990875,941.9024072)(468.24060087,942.14055696)(468.02664188,942.31173208)
\curveto(467.81267552,942.48289724)(467.51033695,942.56848231)(467.11962527,942.56848755)
\curveto(466.65820733,942.56848231)(466.32889086,942.49219997)(466.13167488,942.33964028)
\curveto(465.93445532,942.18707059)(465.83584643,942.00845826)(465.83584792,941.80380278)
\curveto(465.83584643,941.67356016)(465.87677842,941.55634582)(465.95864402,941.45215942)
\curveto(466.04050639,941.34424369)(466.16888399,941.25493753)(466.34377722,941.18424067)
\curveto(466.44424466,941.14702592)(466.74007131,941.06144085)(467.23125808,940.9274852)
\curveto(467.9419834,940.73770602)(468.43781864,940.58235051)(468.71876531,940.46141821)
\curveto(468.99970324,940.34047965)(469.22017782,940.16465815)(469.38018973,939.93395317)
\curveto(469.54019157,939.70324298)(469.620195,939.41671905)(469.62020028,939.07438051)
\curveto(469.620195,938.73948066)(469.52251639,938.42411828)(469.32716414,938.12829242)
\curveto(469.13180194,937.83246497)(468.84992936,937.60361793)(468.48154559,937.44175062)
\curveto(468.11315354,937.2798831)(467.69639145,937.19894939)(467.23125808,937.19894926)
\curveto(466.46098956,937.19894939)(465.87398761,937.35895626)(465.47025046,937.67897035)
\curveto(465.06651107,937.99898375)(464.80882558,938.47342273)(464.69719323,939.10228871)
\closepath
}
}
{
\newrgbcolor{curcolor}{0 0 0}
\pscustom[linestyle=none,fillstyle=solid,fillcolor=curcolor]
{
\newpath
\moveto(470.82025397,944.36019419)
\lineto(470.82025397,945.51559381)
\lineto(471.82494928,945.51559381)
\lineto(471.82494928,944.36019419)
\closepath
\moveto(470.82025397,937.33290863)
\lineto(470.82025397,943.26061099)
\lineto(471.82494928,943.26061099)
\lineto(471.82494928,937.33290863)
\closepath
}
}
{
\newrgbcolor{curcolor}{0 0 0}
\pscustom[linestyle=none,fillstyle=solid,fillcolor=curcolor]
{
\newpath
\moveto(472.98592946,940.29675981)
\curveto(472.98592908,941.39447841)(473.29105846,942.20753658)(473.90131852,942.73593677)
\curveto(474.41110656,943.17501999)(475.0325286,943.3945643)(475.76558649,943.39457036)
\curveto(476.58050205,943.3945643)(477.24657717,943.12757609)(477.76381384,942.59360493)
\curveto(478.2810402,942.05962325)(478.53965596,941.32191715)(478.53966189,940.38048442)
\curveto(478.53965596,939.61765791)(478.42523244,939.01763215)(478.19639099,938.58040531)
\curveto(477.96753836,938.14317599)(477.63450081,937.80362652)(477.19727732,937.5617559)
\curveto(476.76004465,937.31988482)(476.28281485,937.19894939)(475.76558649,937.19894926)
\curveto(474.93578026,937.19894939)(474.26505378,937.46500733)(473.75340504,937.99712387)
\curveto(473.24175402,938.52923908)(472.98592908,939.29578363)(472.98592946,940.29675981)
\closepath
\moveto(474.01853297,940.29675981)
\curveto(474.01853156,939.53765448)(474.18412007,938.96925797)(474.51529899,938.59156859)
\curveto(474.84647409,938.2138767)(475.26323618,938.02503138)(475.76558649,938.02503207)
\curveto(476.2642094,938.02503138)(476.67911094,938.21480697)(477.01029236,938.59435941)
\curveto(477.34146497,938.97390934)(477.50705347,939.55253884)(477.50705837,940.33024965)
\curveto(477.50705347,941.06330139)(477.34053469,941.61867408)(477.00750154,941.99636938)
\curveto(476.67445958,942.37405536)(476.26048831,942.56290068)(475.76558649,942.56290591)
\curveto(475.26323618,942.56290068)(474.84647409,942.37498563)(474.51529899,941.9991602)
\curveto(474.18412007,941.62332544)(474.01853156,941.05585921)(474.01853297,940.29675981)
\closepath
}
}
{
\newrgbcolor{curcolor}{0 0 0}
\pscustom[linestyle=none,fillstyle=solid,fillcolor=curcolor]
{
\newpath
\moveto(479.72296891,937.33290863)
\lineto(479.72296891,943.26061099)
\lineto(480.6271947,943.26061099)
\lineto(480.6271947,942.41778325)
\curveto(481.06256057,943.06896892)(481.69142479,943.3945643)(482.51378923,943.39457036)
\curveto(482.87101033,943.3945643)(483.19939653,943.3303755)(483.4989488,943.20200376)
\curveto(483.79849202,943.07362029)(484.0226877,942.90524096)(484.1715365,942.69686528)
\curveto(484.3203749,942.48847888)(484.42456542,942.24102639)(484.48410838,941.95450708)
\curveto(484.52131376,941.76844796)(484.53991921,941.44285258)(484.53992478,940.97771997)
\lineto(484.53992478,937.33290863)
\lineto(483.53522947,937.33290863)
\lineto(483.53522947,940.93864848)
\curveto(483.5352249,941.34796478)(483.49615346,941.65402444)(483.41801502,941.85682837)
\curveto(483.33986768,942.05962325)(483.20125707,942.22149067)(483.00218279,942.3424311)
\curveto(482.80310044,942.46336152)(482.56960204,942.52382923)(482.30168689,942.52383442)
\curveto(481.8737582,942.52382923)(481.50444001,942.38800945)(481.19373122,942.11637466)
\curveto(480.88301798,941.8447303)(480.72766247,941.32935933)(480.72766423,940.5702602)
\lineto(480.72766423,937.33290863)
\closepath
}
}
{
\newrgbcolor{curcolor}{0 0 0}
\pscustom[linestyle=none,fillstyle=solid,fillcolor=curcolor]
{
\newpath
\moveto(489.12803463,939.22508481)
\lineto(490.09365846,939.31439106)
\curveto(490.17552085,938.86041609)(490.33180663,938.53109962)(490.56251627,938.32644066)
\curveto(490.7932218,938.12177972)(491.08904845,938.01944974)(491.44999713,938.01945043)
\curveto(491.75884466,938.01944974)(492.02955396,938.09015045)(492.26212585,938.23155277)
\curveto(492.49469022,938.3729533)(492.68539608,938.56179862)(492.83424401,938.7980893)
\curveto(492.98308328,939.03437705)(493.1077398,939.35346052)(493.20821394,939.75534067)
\curveto(493.30867866,940.15721597)(493.35891338,940.56653587)(493.35891823,940.98330161)
\curveto(493.35891338,941.02795104)(493.35705283,941.09493066)(493.35333659,941.18424067)
\curveto(493.15239288,940.86422308)(492.87796249,940.60467705)(492.5300446,940.4056018)
\curveto(492.18211865,940.20652041)(491.80535829,940.10698125)(491.39976237,940.10698403)
\curveto(490.72252109,940.10698125)(490.14947322,940.3525732)(489.68061705,940.84376059)
\curveto(489.21175853,941.33494097)(488.97732986,941.98241063)(488.97733033,942.78617153)
\curveto(488.97732986,943.61596916)(489.22199153,944.28390482)(489.71131608,944.78998052)
\curveto(490.20063821,945.29604131)(490.81368779,945.54907543)(491.55046667,945.54908365)
\curveto(492.08257949,945.54907543)(492.56911202,945.40581347)(493.01006569,945.11929732)
\curveto(493.45101036,944.8327656)(493.78590846,944.42437597)(494.01476101,943.8941272)
\curveto(494.24360253,943.36386531)(494.35802605,942.59639049)(494.35803191,941.59170044)
\curveto(494.35802605,940.54606988)(494.24453281,939.71347598)(494.01755183,939.09391625)
\curveto(493.79055982,938.474353)(493.4528709,938.00270484)(493.00448405,937.67897035)
\curveto(492.5560882,937.35523517)(492.03048423,937.19336775)(491.42767057,937.19336761)
\curveto(490.78764016,937.19336775)(490.26482701,937.3710498)(489.85922955,937.7264143)
\curveto(489.45362938,938.081778)(489.20989798,938.58133434)(489.12803463,939.22508481)
\closepath
\moveto(493.24170378,942.8364063)
\curveto(493.24169904,943.41316975)(493.08820408,943.87086383)(492.78121843,944.2094899)
\curveto(492.47422422,944.54810221)(492.10490603,944.71741181)(491.67326276,944.7174192)
\curveto(491.22672879,944.71741181)(490.83787488,944.5350784)(490.50669987,944.17041841)
\curveto(490.17552085,943.80574475)(490.00993234,943.33316632)(490.00993385,942.75268169)
\curveto(490.00993234,942.23172367)(490.1671484,941.80844967)(490.48158248,941.48285845)
\curveto(490.79601261,941.15725892)(491.18393625,940.99446123)(491.64535456,940.99446489)
\curveto(492.11048767,940.99446123)(492.49282967,941.15725892)(492.79238171,941.48285845)
\curveto(493.09192517,941.80844967)(493.24169904,942.25963184)(493.24170378,942.8364063)
\closepath
}
}
{
\newrgbcolor{curcolor}{0 0 0}
\pscustom[linestyle=none,fillstyle=solid,fillcolor=curcolor]
{
\newpath
\moveto(504.28777184,945.51559381)
\lineto(505.37061013,945.51559381)
\lineto(505.37061013,940.78794419)
\curveto(505.37060279,939.96557983)(505.27757554,939.31252853)(505.0915281,938.82878832)
\curveto(504.90546654,938.34504512)(504.56963816,937.95153985)(504.08404196,937.64827133)
\curveto(503.59843366,937.34500217)(502.96119699,937.19336775)(502.17233004,937.19336761)
\curveto(501.40578136,937.19336775)(500.77877769,937.32546645)(500.29131715,937.5896641)
\curveto(499.8038521,937.85386124)(499.45593018,938.23620324)(499.24755035,938.73669125)
\curveto(499.03916809,939.23717646)(498.93497757,939.92092675)(498.93497847,940.78794419)
\lineto(498.93497847,945.51559381)
\lineto(500.01781675,945.51559381)
\lineto(500.01781675,940.79352583)
\curveto(500.01781477,940.08279417)(500.08386412,939.55905075)(500.215965,939.22229399)
\curveto(500.34806151,938.88553345)(500.57504801,938.62598742)(500.89692516,938.44365512)
\curveto(501.21879658,938.2613206)(501.61230185,938.17015389)(502.07744215,938.17015473)
\curveto(502.87375138,938.17015389)(503.44121761,938.35062676)(503.77984255,938.71157387)
\curveto(504.118456,939.07251822)(504.28776559,939.76650152)(504.28777184,940.79352583)
\closepath
}
}
{
\newrgbcolor{curcolor}{0 0 0}
\pscustom[linestyle=none,fillstyle=solid,fillcolor=curcolor]
{
\newpath
\moveto(507.05068193,937.33290863)
\lineto(507.05068193,943.26061099)
\lineto(507.95490771,943.26061099)
\lineto(507.95490771,942.41778325)
\curveto(508.39027359,943.06896892)(509.0191378,943.3945643)(509.84150224,943.39457036)
\curveto(510.19872334,943.3945643)(510.52710954,943.3303755)(510.82666182,943.20200376)
\curveto(511.12620504,943.07362029)(511.35040071,942.90524096)(511.49924951,942.69686528)
\curveto(511.64808791,942.48847888)(511.75227843,942.24102639)(511.81182139,941.95450708)
\curveto(511.84902678,941.76844796)(511.86763223,941.44285258)(511.8676378,940.97771997)
\lineto(511.8676378,937.33290863)
\lineto(510.86294248,937.33290863)
\lineto(510.86294248,940.93864848)
\curveto(510.86293792,941.34796478)(510.82386647,941.65402444)(510.74572803,941.85682837)
\curveto(510.66758069,942.05962325)(510.52897008,942.22149067)(510.3298958,942.3424311)
\curveto(510.13081345,942.46336152)(509.89731505,942.52382923)(509.6293999,942.52383442)
\curveto(509.20147121,942.52382923)(508.83215303,942.38800945)(508.52144423,942.11637466)
\curveto(508.21073099,941.8447303)(508.05537548,941.32935933)(508.05537724,940.5702602)
\lineto(508.05537724,937.33290863)
\closepath
}
}
{
\newrgbcolor{curcolor}{0 0 0}
\pscustom[linestyle=none,fillstyle=solid,fillcolor=curcolor]
{
\newpath
\moveto(513.41933501,944.36019419)
\lineto(513.41933501,945.51559381)
\lineto(514.42403032,945.51559381)
\lineto(514.42403032,944.36019419)
\closepath
\moveto(513.41933501,937.33290863)
\lineto(513.41933501,943.26061099)
\lineto(514.42403032,943.26061099)
\lineto(514.42403032,937.33290863)
\closepath
}
}
{
\newrgbcolor{curcolor}{0 0 0}
\pscustom[linestyle=none,fillstyle=solid,fillcolor=curcolor]
{
\newpath
\moveto(515.28918354,937.33290863)
\lineto(517.45486011,940.41397426)
\lineto(515.45105112,943.26061099)
\lineto(516.70692026,943.26061099)
\lineto(517.61672769,941.87078247)
\curveto(517.78789542,941.60658054)(517.92557575,941.38517568)(518.02976909,941.20656723)
\curveto(518.19349423,941.4521553)(518.34419838,941.66983907)(518.48188199,941.85961919)
\lineto(519.48099566,943.26061099)
\lineto(520.6810484,943.26061099)
\lineto(518.63258628,940.46979067)
\lineto(520.83733434,937.33290863)
\lineto(519.60379175,937.33290863)
\lineto(518.3869941,939.17485004)
\lineto(518.06325894,939.67161606)
\lineto(516.5059812,937.33290863)
\closepath
}
}
{
\newrgbcolor{curcolor}{0 0 0}
\pscustom[linestyle=none,fillstyle=solid,fillcolor=curcolor]
{
\newpath
\moveto(446.56802437,807.02291107)
\lineto(443.39765249,815.20559625)
\lineto(444.56979703,815.20559625)
\lineto(446.69640211,809.26114897)
\curveto(446.8675689,808.7848472)(447.01083087,808.3383164)(447.12618844,807.92155521)
\curveto(447.25270172,808.36808512)(447.39968478,808.81461593)(447.56713805,809.26114897)
\lineto(449.77746774,815.20559625)
\lineto(450.88263259,815.20559625)
\lineto(447.67877086,807.02291107)
\closepath
}
}
{
\newrgbcolor{curcolor}{0 0 0}
\pscustom[linestyle=none,fillstyle=solid,fillcolor=curcolor]
{
\newpath
\moveto(455.79447645,808.93183217)
\lineto(456.8326616,808.80345444)
\curveto(456.66892779,808.19691498)(456.36565896,807.72619709)(455.92285418,807.39129935)
\curveto(455.48003953,807.05640088)(454.91443384,806.88895183)(454.22603543,806.8889517)
\curveto(453.3590182,806.88895183)(452.67154682,807.15594004)(452.16361921,807.68991713)
\curveto(451.65568924,808.22389288)(451.40172484,808.97276225)(451.40172526,809.93652749)
\curveto(451.40172484,810.9337767)(451.65848006,811.70776343)(452.17199167,812.25848999)
\curveto(452.68550091,812.80920608)(453.35157602,813.08456674)(454.17021902,813.08457281)
\curveto(454.96280801,813.08456674)(455.61027768,812.81478772)(456.11262996,812.27523491)
\curveto(456.61497199,811.7356716)(456.86614557,810.97656924)(456.86615145,809.99792553)
\curveto(456.86614557,809.93838512)(456.86428502,809.84907896)(456.86056981,809.73000678)
\lineto(452.43991042,809.73000678)
\curveto(452.47711987,809.07881332)(452.66131382,808.58018725)(452.99249285,808.23412709)
\curveto(453.32366785,807.88806451)(453.73670884,807.71503382)(454.23161707,807.71503451)
\curveto(454.60000173,807.71503382)(454.91443384,807.81178216)(455.17491433,808.00527982)
\curveto(455.43538645,808.19877552)(455.64190694,808.507626)(455.79447645,808.93183217)
\closepath
\moveto(452.49572683,810.5560896)
\lineto(455.80563973,810.5560896)
\curveto(455.76098182,811.05471213)(455.63446476,811.42868168)(455.42608816,811.67799936)
\curveto(455.10606998,812.06498807)(454.69116844,812.25848476)(454.1813823,812.25848999)
\curveto(453.71996394,812.25848476)(453.3320403,812.10405952)(453.01761023,811.79521382)
\curveto(452.70317609,811.48635857)(452.52921513,811.07331758)(452.49572683,810.5560896)
\closepath
}
}
{
\newrgbcolor{curcolor}{0 0 0}
\pscustom[linestyle=none,fillstyle=solid,fillcolor=curcolor]
{
\newpath
\moveto(458.08853043,807.02291107)
\lineto(458.08853043,812.95061343)
\lineto(458.99275621,812.95061343)
\lineto(458.99275621,812.05196929)
\curveto(459.22346215,812.47244743)(459.43649456,812.74966864)(459.63185407,812.88363374)
\curveto(459.82720901,813.01758712)(460.04210196,813.08456674)(460.27653356,813.08457281)
\curveto(460.61514982,813.08456674)(460.95935065,812.97665513)(461.30913708,812.76083765)
\lineto(460.96307536,811.82870366)
\curveto(460.7174798,811.97382137)(460.47188786,812.04638262)(460.2262988,812.04638765)
\curveto(460.0067516,812.04638262)(459.80953383,811.98033327)(459.63464489,811.8482394)
\curveto(459.45975137,811.71613588)(459.33509485,811.5328722)(459.26067497,811.2984478)
\curveto(459.14904035,810.94121888)(459.093224,810.55050443)(459.09322575,810.12630327)
\lineto(459.09322575,807.02291107)
\closepath
}
}
{
\newrgbcolor{curcolor}{0 0 0}
\pscustom[linestyle=none,fillstyle=solid,fillcolor=curcolor]
{
\newpath
\moveto(461.51565807,808.79229115)
\lineto(462.5091901,808.94857709)
\curveto(462.56500511,808.55041853)(462.72036062,808.24528915)(462.9752571,808.03318803)
\curveto(463.23014995,807.82108489)(463.58644432,807.71503382)(464.04414128,807.71503451)
\curveto(464.50555356,807.71503382)(464.84789385,807.80899134)(465.07116316,807.99690736)
\curveto(465.29442465,808.18482144)(465.40605735,808.40529602)(465.4060616,808.65833178)
\curveto(465.40605735,808.88531664)(465.30744847,809.06392896)(465.11023464,809.19416928)
\curveto(464.97255036,809.28347327)(464.63021008,809.39696652)(464.08321276,809.53464936)
\curveto(463.34643402,809.72070135)(462.83571441,809.88163849)(462.55105241,810.01746127)
\curveto(462.26638763,810.15327807)(462.05056441,810.34119311)(461.90358209,810.58120698)
\curveto(461.7565983,810.82121373)(461.68310677,811.08634139)(461.68310729,811.37659077)
\curveto(461.68310677,811.64078381)(461.74357448,811.88544548)(461.86451061,812.11057651)
\curveto(461.98544534,812.33569737)(462.15010357,812.52268215)(462.35848581,812.6715314)
\curveto(462.51477039,812.78687954)(462.7278028,812.88455815)(462.99758366,812.96456753)
\curveto(463.26736085,813.04456503)(463.5566756,813.08456674)(463.86552878,813.08457281)
\curveto(464.33066233,813.08456674)(464.73905196,813.01758712)(465.0906989,812.88363374)
\curveto(465.44233798,812.74966864)(465.70188401,812.5682655)(465.86933777,812.33942378)
\curveto(466.03678211,812.11057143)(466.15213591,811.80451177)(466.21539949,811.4212439)
\lineto(465.23303074,811.28728452)
\curveto(465.18837359,811.59240964)(465.05906571,811.8305594)(464.84510671,812.00173452)
\curveto(464.63114035,812.17289968)(464.32880179,812.25848476)(463.93809011,812.25848999)
\curveto(463.47667217,812.25848476)(463.1473557,812.18220241)(462.95013971,812.02964272)
\curveto(462.75292016,811.87707303)(462.65431127,811.6984607)(462.65431276,811.49380522)
\curveto(462.65431127,811.3635626)(462.69524326,811.24634827)(462.77710885,811.14216186)
\curveto(462.85897122,811.03424613)(462.98734883,810.94493997)(463.16224206,810.87424311)
\curveto(463.26270949,810.83702836)(463.55853615,810.75144329)(464.04972292,810.61748764)
\curveto(464.76044823,810.42770846)(465.25628348,810.27235295)(465.53723015,810.15142065)
\curveto(465.81816807,810.0304821)(466.03864266,809.85466059)(466.19865457,809.62395561)
\curveto(466.3586564,809.39324543)(466.43865984,809.10672149)(466.43866511,808.76438295)
\curveto(466.43865984,808.42948311)(466.34098123,808.11412073)(466.14562898,807.81829486)
\curveto(465.95026677,807.52246741)(465.6683942,807.29362037)(465.30001042,807.13175306)
\curveto(464.93161837,806.96988554)(464.51485629,806.88895183)(464.04972292,806.8889517)
\curveto(463.2794544,806.88895183)(462.69245244,807.0489587)(462.2887153,807.36897279)
\curveto(461.8849759,807.68898619)(461.62729042,808.16342517)(461.51565807,808.79229115)
\closepath
}
}
{
\newrgbcolor{curcolor}{0 0 0}
\pscustom[linestyle=none,fillstyle=solid,fillcolor=curcolor]
{
\newpath
\moveto(467.6387188,814.05019664)
\lineto(467.6387188,815.20559625)
\lineto(468.64341412,815.20559625)
\lineto(468.64341412,814.05019664)
\closepath
\moveto(467.6387188,807.02291107)
\lineto(467.6387188,812.95061343)
\lineto(468.64341412,812.95061343)
\lineto(468.64341412,807.02291107)
\closepath
}
}
{
\newrgbcolor{curcolor}{0 0 0}
\pscustom[linestyle=none,fillstyle=solid,fillcolor=curcolor]
{
\newpath
\moveto(469.80439429,809.98676225)
\curveto(469.80439391,811.08448085)(470.1095233,811.89753902)(470.71978336,812.42593921)
\curveto(471.2295714,812.86502243)(471.85099343,813.08456674)(472.58405133,813.08457281)
\curveto(473.39896689,813.08456674)(474.06504201,812.81757853)(474.58227868,812.28360737)
\curveto(475.09950504,811.74962569)(475.35812079,811.01191959)(475.35812673,810.07048686)
\curveto(475.35812079,809.30766036)(475.24369727,808.70763459)(475.01485583,808.27040775)
\curveto(474.7860032,807.83317843)(474.45296564,807.49362896)(474.01574215,807.25175834)
\curveto(473.57850948,807.00988726)(473.10127969,806.88895183)(472.58405133,806.8889517)
\curveto(471.75424509,806.88895183)(471.08351861,807.15500977)(470.57186988,807.68712631)
\curveto(470.06021885,808.21924152)(469.80439391,808.98578607)(469.80439429,809.98676225)
\closepath
\moveto(470.83699781,809.98676225)
\curveto(470.8369964,809.22765692)(471.0025849,808.65926042)(471.33376383,808.28157104)
\curveto(471.66493893,807.90387914)(472.08170102,807.71503382)(472.58405133,807.71503451)
\curveto(473.08267424,807.71503382)(473.49757577,807.90480941)(473.82875719,808.28436186)
\curveto(474.1599298,808.66391178)(474.32551831,809.24254128)(474.32552321,810.02025209)
\curveto(474.32551831,810.75330384)(474.15899953,811.30867652)(473.82596637,811.68637183)
\curveto(473.49292441,812.0640578)(473.07895315,812.25290312)(472.58405133,812.25290835)
\curveto(472.08170102,812.25290312)(471.66493893,812.06498807)(471.33376383,811.68916265)
\curveto(471.0025849,811.31332789)(470.8369964,810.74586165)(470.83699781,809.98676225)
\closepath
}
}
{
\newrgbcolor{curcolor}{0 0 0}
\pscustom[linestyle=none,fillstyle=solid,fillcolor=curcolor]
{
\newpath
\moveto(476.54143375,807.02291107)
\lineto(476.54143375,812.95061343)
\lineto(477.44565953,812.95061343)
\lineto(477.44565953,812.10778569)
\curveto(477.88102541,812.75897137)(478.50988963,813.08456674)(479.33225407,813.08457281)
\curveto(479.68947517,813.08456674)(480.01786136,813.02037794)(480.31741364,812.8920062)
\curveto(480.61695686,812.76362273)(480.84115253,812.5952434)(480.99000134,812.38686773)
\curveto(481.13883974,812.17848132)(481.24303026,811.93102883)(481.30257321,811.64450952)
\curveto(481.3397786,811.4584504)(481.35838405,811.13285502)(481.35838962,810.66772241)
\lineto(481.35838962,807.02291107)
\lineto(480.3536943,807.02291107)
\lineto(480.3536943,810.62865092)
\curveto(480.35368974,811.03796722)(480.31461829,811.34402688)(480.23647985,811.54683081)
\curveto(480.15833251,811.74962569)(480.01972191,811.91149311)(479.82064762,812.03243355)
\curveto(479.62156527,812.15336396)(479.38806687,812.21383167)(479.12015172,812.21383687)
\curveto(478.69222304,812.21383167)(478.32290485,812.07801189)(478.01219606,811.8063771)
\curveto(477.70148281,811.53473274)(477.54612731,811.01936177)(477.54612906,810.26026264)
\lineto(477.54612906,807.02291107)
\closepath
}
}
{
\newrgbcolor{curcolor}{0 0 0}
\pscustom[linestyle=none,fillstyle=solid,fillcolor=curcolor]
{
\newpath
\moveto(489.58014752,807.02291107)
\lineto(488.57545221,807.02291107)
\lineto(488.57545221,813.42505288)
\curveto(488.3335781,813.1943389)(488.01635517,812.96363132)(487.62378248,812.73292945)
\curveto(487.23120518,812.50221615)(486.87863189,812.32918547)(486.56606158,812.21383687)
\lineto(486.56606158,813.18504234)
\curveto(487.12794493,813.44923357)(487.61912881,813.76924731)(488.0396147,814.14508453)
\curveto(488.46009516,814.5209075)(488.75778236,814.88557432)(488.93267721,815.23908609)
\lineto(489.58014752,815.23908609)
\closepath
}
}
{
\newrgbcolor{curcolor}{0 0 0}
\pscustom[linestyle=none,fillstyle=solid,fillcolor=curcolor]
{
\newpath
\moveto(492.15886661,811.05843725)
\curveto(492.15886614,812.02591663)(492.25840529,812.80455472)(492.45748438,813.39435386)
\curveto(492.65656193,813.98414026)(492.95238859,814.43904352)(493.34496525,814.759065)
\curveto(493.73753858,815.079071)(494.23151329,815.23907788)(494.82689083,815.23908609)
\curveto(495.26597632,815.23907788)(495.65110914,815.15070199)(495.98229045,814.97395816)
\curveto(496.31346316,814.79719843)(496.58696328,814.54230377)(496.80279162,814.20927339)
\curveto(497.01860972,813.87622865)(497.18791932,813.47062984)(497.31072092,812.99247573)
\curveto(497.43351126,812.5143097)(497.49490925,811.86963085)(497.49491506,811.05843725)
\curveto(497.49490925,810.09839199)(497.39630036,809.32347499)(497.19908811,808.73368393)
\curveto(497.00186482,808.14388945)(496.70696843,807.68805592)(496.31439806,807.36618197)
\curveto(495.92181844,807.04430734)(495.42598319,806.8833702)(494.82689083,806.88337006)
\curveto(494.0380166,806.8833702)(493.41845511,807.16617304)(492.9682045,807.73177943)
\curveto(492.42864516,808.4127382)(492.15886614,809.52162303)(492.15886661,811.05843725)
\closepath
\moveto(493.19147013,811.05843725)
\curveto(493.19146862,809.71511971)(493.34868467,808.82112783)(493.66311876,808.37645893)
\curveto(493.97754889,807.93178731)(494.36547253,807.70945218)(494.82689083,807.70945287)
\curveto(495.28830286,807.70945218)(495.67622649,807.93271759)(495.99066291,808.37924975)
\curveto(496.30509071,808.8257792)(496.46230676,809.7188408)(496.46231154,811.05843725)
\curveto(496.46230676,812.40546781)(496.30509071,813.30038997)(495.99066291,813.7432064)
\curveto(495.67622649,814.18600939)(495.28458177,814.40741425)(494.81572755,814.40742164)
\curveto(494.35430926,814.40741425)(493.98592134,814.21205703)(493.71056271,813.82134937)
\curveto(493.36449931,813.32271651)(493.19146862,812.40174672)(493.19147013,811.05843725)
\closepath
}
}
{
\newrgbcolor{curcolor}{0 0 0}
\pscustom[linestyle=none,fillstyle=solid,fillcolor=curcolor]
{
\newpath
\moveto(507.46930812,815.20559625)
\lineto(508.5521464,815.20559625)
\lineto(508.5521464,810.47794663)
\curveto(508.55213907,809.65558227)(508.45911182,809.00253097)(508.27306437,808.51879076)
\curveto(508.08700282,808.03504756)(507.75117444,807.64154229)(507.26557824,807.33827377)
\curveto(506.77996994,807.03500461)(506.14273327,806.8833702)(505.35386632,806.88337006)
\curveto(504.58731764,806.8833702)(503.96031396,807.01546889)(503.47285342,807.27966654)
\curveto(502.98538837,807.54386368)(502.63746646,807.92620568)(502.42908662,808.42669369)
\curveto(502.22070437,808.9271789)(502.11651385,809.61092919)(502.11651475,810.47794663)
\lineto(502.11651475,815.20559625)
\lineto(503.19935303,815.20559625)
\lineto(503.19935303,810.48352827)
\curveto(503.19935105,809.77279661)(503.2654004,809.24905319)(503.39750128,808.91229643)
\curveto(503.52959779,808.57553589)(503.75658428,808.31598986)(504.07846143,808.13365756)
\curveto(504.40033286,807.95132304)(504.79383813,807.86015633)(505.25897843,807.86015717)
\curveto(506.05528766,807.86015633)(506.62275389,808.0406292)(506.96137882,808.40157631)
\curveto(507.29999227,808.76252066)(507.46930187,809.45650396)(507.46930812,810.48352827)
\closepath
}
}
{
\newrgbcolor{curcolor}{0 0 0}
\pscustom[linestyle=none,fillstyle=solid,fillcolor=curcolor]
{
\newpath
\moveto(510.2322182,807.02291107)
\lineto(510.2322182,812.95061343)
\lineto(511.13644399,812.95061343)
\lineto(511.13644399,812.10778569)
\curveto(511.57180986,812.75897137)(512.20067408,813.08456674)(513.02303852,813.08457281)
\curveto(513.38025962,813.08456674)(513.70864582,813.02037794)(514.0081981,812.8920062)
\curveto(514.30774131,812.76362273)(514.53193699,812.5952434)(514.68078579,812.38686773)
\curveto(514.82962419,812.17848132)(514.93381471,811.93102883)(514.99335767,811.64450952)
\curveto(515.03056305,811.4584504)(515.0491685,811.13285502)(515.04917407,810.66772241)
\lineto(515.04917407,807.02291107)
\lineto(514.04447876,807.02291107)
\lineto(514.04447876,810.62865092)
\curveto(514.04447419,811.03796722)(514.00540275,811.34402688)(513.92726431,811.54683081)
\curveto(513.84911697,811.74962569)(513.71050636,811.91149311)(513.51143208,812.03243355)
\curveto(513.31234973,812.15336396)(513.07885133,812.21383167)(512.81093618,812.21383687)
\curveto(512.38300749,812.21383167)(512.01368931,812.07801189)(511.70298051,811.8063771)
\curveto(511.39226727,811.53473274)(511.23691176,811.01936177)(511.23691352,810.26026264)
\lineto(511.23691352,807.02291107)
\closepath
}
}
{
\newrgbcolor{curcolor}{0 0 0}
\pscustom[linestyle=none,fillstyle=solid,fillcolor=curcolor]
{
\newpath
\moveto(516.60086747,814.05019664)
\lineto(516.60086747,815.20559625)
\lineto(517.60556279,815.20559625)
\lineto(517.60556279,814.05019664)
\closepath
\moveto(516.60086747,807.02291107)
\lineto(516.60086747,812.95061343)
\lineto(517.60556279,812.95061343)
\lineto(517.60556279,807.02291107)
\closepath
}
}
{
\newrgbcolor{curcolor}{0 0 0}
\pscustom[linestyle=none,fillstyle=solid,fillcolor=curcolor]
{
\newpath
\moveto(518.47072363,807.02291107)
\lineto(520.6364002,810.1039767)
\lineto(518.63259121,812.95061343)
\lineto(519.88846036,812.95061343)
\lineto(520.79826778,811.56078491)
\curveto(520.96943551,811.29658298)(521.10711584,811.07517812)(521.21130919,810.89656968)
\curveto(521.37503432,811.14215774)(521.52573847,811.35984151)(521.66342208,811.54962163)
\lineto(522.66253575,812.95061343)
\lineto(523.86258849,812.95061343)
\lineto(521.81412638,810.15979311)
\lineto(524.01887443,807.02291107)
\lineto(522.78533185,807.02291107)
\lineto(521.56853419,808.86485248)
\lineto(521.24479903,809.3616185)
\lineto(519.68752129,807.02291107)
\closepath
}
}
{
\newrgbcolor{curcolor}{0 0 0}
\pscustom[linestyle=none,fillstyle=solid,fillcolor=curcolor]
{
\newpath
\moveto(602.03904024,936.32491303)
\lineto(602.03904024,938.28406889)
\lineto(598.4891168,938.28406889)
\lineto(598.4891168,939.20503959)
\lineto(602.22323438,944.5075982)
\lineto(603.04373556,944.5075982)
\lineto(603.04373556,939.20503959)
\lineto(604.1489004,939.20503959)
\lineto(604.1489004,938.28406889)
\lineto(603.04373556,938.28406889)
\lineto(603.04373556,936.32491303)
\closepath
\moveto(602.03904024,939.20503959)
\lineto(602.03904024,942.89450406)
\lineto(599.47706719,939.20503959)
\closepath
}
}
{
\newrgbcolor{curcolor}{0 0 0}
\pscustom[linestyle=none,fillstyle=solid,fillcolor=curcolor]
{
\newpath
\moveto(605.74524979,936.32491303)
\lineto(605.74524979,937.46914936)
\lineto(606.88948612,937.46914936)
\lineto(606.88948612,936.32491303)
\closepath
}
}
{
\newrgbcolor{curcolor}{0 0 0}
\pscustom[linestyle=none,fillstyle=solid,fillcolor=curcolor]
{
\newpath
\moveto(608.35745773,938.48500795)
\lineto(609.36215305,938.61896733)
\curveto(609.47750535,938.04963826)(609.67379285,937.63938808)(609.95101614,937.38821557)
\curveto(610.22823527,937.13704093)(610.56592419,937.01145414)(610.96408391,937.01145482)
\curveto(611.43665926,937.01145414)(611.83574617,937.1751821)(612.16134583,937.5026392)
\curveto(612.48693692,937.83009395)(612.64973461,938.23569276)(612.64973938,938.71943686)
\curveto(612.64973461,939.18084963)(612.49903046,939.56133109)(612.19762649,939.86088237)
\curveto(611.89621388,940.16042658)(611.5129416,940.31020046)(611.04780852,940.31020444)
\curveto(610.85802976,940.31020046)(610.62174054,940.27298956)(610.33894016,940.19857163)
\lineto(610.45057297,941.08047085)
\curveto(610.51755002,941.07302391)(610.57150583,941.06930282)(610.61244055,941.06930757)
\curveto(611.04036317,941.06930282)(611.42549599,941.18093552)(611.76784016,941.40420601)
\curveto(612.11017656,941.62746633)(612.2813467,941.97166716)(612.2813511,942.43680952)
\curveto(612.2813467,942.80519133)(612.15669018,943.11032071)(611.90738118,943.35219859)
\curveto(611.65806412,943.59406241)(611.33618983,943.71499784)(610.94175735,943.71500523)
\curveto(610.55103983,943.71499784)(610.22544445,943.59220187)(609.96497024,943.34661695)
\curveto(609.70449185,943.10101798)(609.53704279,942.73263007)(609.46262258,942.2414521)
\lineto(608.45792726,942.4200646)
\curveto(608.58072266,943.0935758)(608.85980441,943.61545868)(609.29517336,943.9857148)
\curveto(609.73053948,944.3559556)(610.27195808,944.54107983)(610.91943079,944.54108804)
\curveto(611.36595855,944.54107983)(611.777139,944.44526176)(612.15297337,944.25363355)
\curveto(612.52879919,944.06198949)(612.81625339,943.80058291)(613.01533685,943.46941304)
\curveto(613.21441002,943.13822888)(613.31394918,942.78658588)(613.31395462,942.41448296)
\curveto(613.31394918,942.06097332)(613.21906139,941.73909903)(613.02929095,941.44885913)
\curveto(612.8395102,941.15860898)(612.55856791,940.9279014)(612.18646321,940.75673569)
\curveto(612.67020061,940.64509856)(613.0460307,940.4134607)(613.31395462,940.06182143)
\curveto(613.58186767,939.71017469)(613.71582691,939.27015579)(613.71583275,938.74176342)
\curveto(613.71582691,938.02731172)(613.4553506,937.42170431)(612.93440306,936.92493939)
\curveto(612.41344539,936.42817327)(611.75481246,936.17979051)(610.95850227,936.17979037)
\curveto(610.24032881,936.17979051)(609.64402413,936.39375319)(609.16958645,936.82167904)
\curveto(608.69514617,937.2496039)(608.42443687,937.80404632)(608.35745773,938.48500795)
\closepath
}
}
{
\newrgbcolor{curcolor}{0 0 0}
\pscustom[linestyle=none,fillstyle=solid,fillcolor=curcolor]
{
\newpath
\moveto(615.07775322,936.32491303)
\lineto(615.07775322,944.5075982)
\lineto(618.14765557,944.5075982)
\curveto(618.77279479,944.50759002)(619.27421167,944.42479577)(619.65190772,944.25921519)
\curveto(620.02959295,944.09361875)(620.32541961,943.83872408)(620.53938859,943.49453043)
\curveto(620.75334496,943.15032243)(620.8603263,942.79030697)(620.86033292,942.41448296)
\curveto(620.8603263,942.06469441)(620.76543851,941.73537794)(620.57566925,941.42653257)
\curveto(620.38588732,941.11767699)(620.09936339,940.86836396)(619.71609659,940.67859272)
\curveto(620.21099609,940.53346586)(620.59147755,940.28601337)(620.8575421,939.93623452)
\curveto(621.12359342,939.58644844)(621.25662239,939.17340745)(621.25662941,938.6971103)
\curveto(621.25662239,938.31383565)(621.17568868,937.95754128)(621.01382804,937.62822611)
\curveto(620.85195385,937.29890834)(620.65194526,937.04494395)(620.41380167,936.86633217)
\curveto(620.17564574,936.6877193)(619.87702826,936.55282979)(619.51794835,936.46166322)
\curveto(619.15885788,936.37049638)(618.71883899,936.32491303)(618.19789034,936.32491303)
\closepath
\moveto(616.16059151,941.06930757)
\lineto(617.92997159,941.06930757)
\curveto(618.40998851,941.06930282)(618.75418934,941.10093209)(618.96257511,941.16419546)
\curveto(619.23793105,941.2460546)(619.44538182,941.38187439)(619.58492804,941.57165523)
\curveto(619.72446357,941.76142557)(619.79423401,941.99957533)(619.79423956,942.28610523)
\curveto(619.79423401,942.55773884)(619.72911493,942.79681887)(619.59888214,943.00334605)
\curveto(619.46863863,943.20985987)(619.28258413,943.35126129)(619.04071807,943.42755074)
\curveto(618.79884242,943.50382598)(618.38394088,943.54196715)(617.79601221,943.54197437)
\lineto(616.16059151,943.54197437)
\closepath
\moveto(616.16059151,937.29053686)
\lineto(618.19789034,937.29053686)
\curveto(618.54766884,937.29053589)(618.79326079,937.30355971)(618.9346669,937.32960834)
\curveto(619.18397524,937.37426042)(619.39235628,937.44868222)(619.55981065,937.55287397)
\curveto(619.72725439,937.65706326)(619.86493472,937.80869768)(619.97285206,938.00777768)
\curveto(620.08075794,938.20685431)(620.13471375,938.43663162)(620.13471964,938.6971103)
\curveto(620.13471375,939.00223731)(620.05657086,939.26736497)(619.90029073,939.49249409)
\curveto(619.74399929,939.71761687)(619.5272458,939.87576319)(619.2500296,939.96693354)
\curveto(618.97280338,940.05809661)(618.57371647,940.10367996)(618.05276768,940.10368374)
\lineto(616.16059151,940.10368374)
\closepath
}
}
{
\newrgbcolor{curcolor}{0 0 0}
\pscustom[linestyle=none,fillstyle=solid,fillcolor=curcolor]
{
\newpath
\moveto(622.38970303,938.95386577)
\lineto(623.41114327,939.04317202)
\curveto(623.4595159,938.63384939)(623.57207888,938.29802102)(623.74883253,938.03568588)
\curveto(623.92558243,937.77334732)(624.20001282,937.56124519)(624.57212452,937.39937885)
\curveto(624.94423083,937.23751036)(625.36285346,937.15657665)(625.82799366,937.15657748)
\curveto(626.24103071,937.15657665)(626.60569753,937.21797463)(626.92199523,937.34077162)
\curveto(627.23828284,937.46356658)(627.47364178,937.6319459)(627.62807277,937.8459101)
\curveto(627.78249226,938.05987126)(627.85970487,938.29336966)(627.85971086,938.546406)
\curveto(627.85970487,938.80315899)(627.78528307,939.02735467)(627.63644523,939.2189937)
\curveto(627.48759587,939.41062694)(627.24200393,939.57156408)(626.89966867,939.70180561)
\curveto(626.68011933,939.78738731)(626.19451708,939.92041627)(625.44286046,940.10089292)
\curveto(624.69119671,940.28136201)(624.16466247,940.45160188)(623.86325616,940.61161304)
\curveto(623.47253972,940.8162687)(623.18136442,941.0702331)(622.9897294,941.37350698)
\curveto(622.79809215,941.67677077)(622.70227408,942.01632024)(622.70227491,942.3921564)
\curveto(622.70227408,942.80519133)(622.81948842,943.19125442)(623.05391827,943.55034683)
\curveto(623.28834576,943.90942479)(623.63068605,944.18199464)(624.08094015,944.36805719)
\curveto(624.53118983,944.55410364)(625.03167644,944.64713089)(625.58240148,944.64713922)
\curveto(626.18893545,944.64713089)(626.72384214,944.54945228)(627.18712316,944.35410308)
\curveto(627.65039356,944.15873783)(628.00668793,943.87128362)(628.25600734,943.4917396)
\curveto(628.505314,943.11218125)(628.63927324,942.68239535)(628.65788547,942.20238062)
\lineto(627.61970031,942.12423765)
\curveto(627.56387822,942.64146336)(627.3750329,943.03217782)(627.05316379,943.29638218)
\curveto(626.73128432,943.5605726)(626.25591507,943.6926713)(625.6270546,943.69267867)
\curveto(624.972139,943.6926713)(624.49490921,943.57266615)(624.19536378,943.33266285)
\curveto(623.89581371,943.09264553)(623.74603984,942.80333078)(623.74604171,942.46471773)
\curveto(623.74603984,942.17074547)(623.8520909,941.92887462)(624.06419522,941.73910444)
\curveto(624.27257408,941.54932344)(624.8167835,941.35489648)(625.69682511,941.155823)
\curveto(626.57685908,940.95673985)(627.18060594,940.78277889)(627.5080675,940.6339396)
\curveto(627.98436139,940.41439098)(628.3360044,940.1362395)(628.56299758,939.79948432)
\curveto(628.78997738,939.4627222)(628.90347063,939.07479856)(628.90347766,938.63571225)
\curveto(628.90347063,938.2003424)(628.77881411,937.79009223)(628.52950773,937.40496049)
\curveto(628.28018805,937.01982659)(627.92203313,936.72027884)(627.45504191,936.50631635)
\curveto(626.98803953,936.29235349)(626.46243556,936.18537215)(625.87822843,936.18537201)
\curveto(625.13772751,936.18537215)(624.51723575,936.29328376)(624.01675128,936.50910717)
\curveto(623.51626253,936.7249302)(623.12368753,937.04959531)(622.8390251,937.48310346)
\curveto(622.55436075,937.91660929)(622.40458688,938.4068629)(622.38970303,938.95386577)
\closepath
}
}
{
\newrgbcolor{curcolor}{0 0 0}
\pscustom[linestyle=none,fillstyle=solid,fillcolor=curcolor]
{
\newpath
\moveto(630.39377437,936.32491303)
\lineto(630.39377437,944.5075982)
\lineto(633.2125029,944.5075982)
\curveto(633.84880559,944.50759002)(634.33440784,944.46851857)(634.6693111,944.39038375)
\curveto(635.13816329,944.28246407)(635.53818047,944.08710684)(635.86936384,943.80431148)
\curveto(636.30100393,943.43963718)(636.62380849,942.97357065)(636.83777849,942.4061105)
\curveto(637.05173384,941.83863819)(637.15871518,941.19023825)(637.15872283,940.46090874)
\curveto(637.15871518,939.83948257)(637.08615392,939.28876124)(636.94103884,938.80874311)
\curveto(636.7959089,938.32872001)(636.6098544,937.93149365)(636.38287478,937.61706283)
\curveto(636.15588142,937.30262943)(635.90749866,937.05517695)(635.63772575,936.87470463)
\curveto(635.3679406,936.69423121)(635.04234522,936.55748115)(634.66093864,936.46445404)
\curveto(634.27952176,936.37142665)(633.84136341,936.32491303)(633.34646227,936.32491303)
\closepath
\moveto(631.47661266,937.29053686)
\lineto(633.22366618,937.29053686)
\curveto(633.76322052,937.29053589)(634.18649451,937.34077061)(634.49348942,937.44124115)
\curveto(634.80047437,937.54170947)(635.04513604,937.68311089)(635.22747517,937.86544584)
\curveto(635.48422466,938.12219951)(635.68423325,938.46733061)(635.82750153,938.90084018)
\curveto(635.97075719,939.33434459)(636.04238817,939.85994856)(636.0423947,940.47765366)
\curveto(636.04238817,941.33350022)(635.90191702,941.99120288)(635.62098083,942.45076363)
\curveto(635.34003242,942.91031212)(634.99862241,943.21823232)(634.59674977,943.37452515)
\curveto(634.30649967,943.4861508)(633.83950287,943.54196715)(633.19575797,943.54197437)
\lineto(631.47661266,943.54197437)
\closepath
}
}
{
\newrgbcolor{curcolor}{0 0 0}
\pscustom[linestyle=none,fillstyle=solid,fillcolor=curcolor]
{
\newpath
\moveto(682.33748555,897.94287944)
\lineto(683.35892578,898.03218569)
\curveto(683.40729842,897.62286307)(683.51986139,897.28703469)(683.69661504,897.02469955)
\curveto(683.87336495,896.76236099)(684.14779534,896.55025886)(684.51990704,896.38839252)
\curveto(684.89201335,896.22652403)(685.31063597,896.14559032)(685.77577618,896.14559115)
\curveto(686.18881322,896.14559032)(686.55348005,896.20698831)(686.86977775,896.32978529)
\curveto(687.18606535,896.45258025)(687.4214243,896.62095957)(687.57585529,896.83492377)
\curveto(687.73027477,897.04888493)(687.80748739,897.28238333)(687.80749337,897.53541967)
\curveto(687.80748739,897.79217266)(687.73306559,898.01636834)(687.58422775,898.20800737)
\curveto(687.43537839,898.39964061)(687.18978644,898.56057776)(686.84745118,898.69081928)
\curveto(686.62790185,898.77640098)(686.1422996,898.90942995)(685.39064298,899.08990659)
\curveto(684.63897922,899.27037568)(684.11244498,899.44061555)(683.81103868,899.60062671)
\curveto(683.42032224,899.80528237)(683.12914694,900.05924677)(682.93751192,900.36252065)
\curveto(682.74587467,900.66578444)(682.6500566,901.00533391)(682.65005742,901.38117007)
\curveto(682.6500566,901.794205)(682.76727093,902.18026809)(683.00170078,902.5393605)
\curveto(683.23612828,902.89843847)(683.57846856,903.17100831)(684.02872266,903.35707086)
\curveto(684.47897235,903.54311732)(684.97945896,903.63614457)(685.53018399,903.63615289)
\curveto(686.13671796,903.63614457)(686.67162466,903.53846595)(687.13490568,903.34311676)
\curveto(687.59817608,903.1477515)(687.95447045,902.86029729)(688.20378986,902.48075328)
\curveto(688.45309651,902.10119493)(688.58705575,901.67140903)(688.60566799,901.19139429)
\lineto(687.56748283,901.11325132)
\curveto(687.51166073,901.63047704)(687.32281541,902.02119149)(687.0009463,902.28539585)
\curveto(686.67906684,902.54958628)(686.20369758,902.68168497)(685.57483712,902.68169234)
\curveto(684.91992152,902.68168497)(684.44269172,902.56167982)(684.14314629,902.32167652)
\curveto(683.84359623,902.0816592)(683.69382235,901.79234445)(683.69382422,901.4537314)
\curveto(683.69382235,901.15975915)(683.79987342,900.91788829)(684.01197774,900.72811812)
\curveto(684.22035659,900.53833711)(684.76456601,900.34391016)(685.64460763,900.14483667)
\curveto(686.5246416,899.94575352)(687.12838846,899.77179256)(687.45585001,899.62295327)
\curveto(687.93214391,899.40340465)(688.28378692,899.12525317)(688.51078009,898.78849799)
\curveto(688.7377599,898.45173587)(688.85125315,898.06381224)(688.85126017,897.62472592)
\curveto(688.85125315,897.18935608)(688.72659663,896.7791059)(688.47729025,896.39397416)
\curveto(688.22797056,896.00884026)(687.86981565,895.70929251)(687.40282443,895.49533002)
\curveto(686.93582205,895.28136716)(686.41021808,895.17438582)(685.82601095,895.17438568)
\curveto(685.08551003,895.17438582)(684.46501826,895.28229743)(683.96453379,895.49812084)
\curveto(683.46404504,895.71394388)(683.07147004,896.03860898)(682.78680762,896.47211713)
\curveto(682.50214327,896.90562296)(682.35236939,897.39587657)(682.33748555,897.94287944)
\closepath
}
}
{
\newrgbcolor{curcolor}{0 0 0}
\pscustom[linestyle=none,fillstyle=solid,fillcolor=curcolor]
{
\newpath
\moveto(694.09800247,895.3139267)
\lineto(694.09800247,896.18466264)
\curveto(693.63658267,895.51486556)(693.009579,895.17996746)(692.21698957,895.17996732)
\curveto(691.86720435,895.17996746)(691.5406787,895.24694708)(691.23741164,895.38090638)
\curveto(690.93414103,895.51486556)(690.70901508,895.68324488)(690.56203313,895.88604486)
\curveto(690.41504897,896.0888437)(690.31178872,896.33722646)(690.25225207,896.63119389)
\curveto(690.21131929,896.82841034)(690.19085329,897.14098191)(690.19085402,897.56890951)
\lineto(690.19085402,901.24162906)
\lineto(691.19554934,901.24162906)
\lineto(691.19554934,897.95404272)
\curveto(691.1955476,897.42936638)(691.2160136,897.07586283)(691.25694739,896.893531)
\curveto(691.32020412,896.62933203)(691.45416336,896.42188126)(691.65882551,896.27117807)
\curveto(691.86348326,896.12047296)(692.11651739,896.04512089)(692.41792864,896.04512162)
\curveto(692.71933397,896.04512089)(693.00213682,896.12233351)(693.26633801,896.27675971)
\curveto(693.5305316,896.43118398)(693.71751638,896.64142557)(693.8272929,896.9074851)
\curveto(693.93706069,897.17354144)(693.99194677,897.55960453)(693.9919513,898.06567553)
\lineto(693.9919513,901.24162906)
\lineto(694.99664661,901.24162906)
\lineto(694.99664661,895.3139267)
\closepath
}
}
{
\newrgbcolor{curcolor}{0 0 0}
\pscustom[linestyle=none,fillstyle=solid,fillcolor=curcolor]
{
\newpath
\moveto(696.5762506,895.3139267)
\lineto(696.5762506,901.24162906)
\lineto(697.48047638,901.24162906)
\lineto(697.48047638,900.39880132)
\curveto(697.91584226,901.04998699)(698.54470647,901.37558237)(699.36707092,901.37558843)
\curveto(699.72429202,901.37558237)(700.05267821,901.31139357)(700.35223049,901.18302183)
\curveto(700.65177371,901.05463835)(700.87596938,900.88625903)(701.02481819,900.67788335)
\curveto(701.17365659,900.46949694)(701.27784711,900.22204446)(701.33739006,899.93552515)
\curveto(701.37459545,899.74946602)(701.3932009,899.42387064)(701.39320647,898.95873803)
\lineto(701.39320647,895.3139267)
\lineto(700.38851115,895.3139267)
\lineto(700.38851115,898.91966655)
\curveto(700.38850659,899.32898285)(700.34943514,899.6350425)(700.2712967,899.83784643)
\curveto(700.19314936,900.04064132)(700.05453876,900.20250873)(699.85546447,900.32344917)
\curveto(699.65638212,900.44437959)(699.42288372,900.5048473)(699.15496857,900.50485249)
\curveto(698.72703989,900.5048473)(698.3577217,900.36902751)(698.04701291,900.09739272)
\curveto(697.73629966,899.82574837)(697.58094415,899.3103774)(697.58094591,898.55127827)
\lineto(697.58094591,895.3139267)
\closepath
}
}
{
\newrgbcolor{curcolor}{0 0 0}
\pscustom[linestyle=none,fillstyle=solid,fillcolor=curcolor]
{
\newpath
\moveto(702.73838298,899.29921811)
\curveto(702.73838242,900.65741199)(703.10304925,901.72071347)(703.83238454,902.48912574)
\curveto(704.56171654,903.25752366)(705.50315232,903.6417262)(706.6566947,903.64173453)
\curveto(707.41207151,903.6417262)(708.09303099,903.46125333)(708.69957518,903.10031539)
\curveto(709.30610634,902.73936187)(709.76845178,902.23608444)(710.08661288,901.59048159)
\curveto(710.40475817,900.9448662)(710.56383477,900.21274173)(710.56384315,899.394106)
\curveto(710.56383477,898.56429885)(710.39638572,897.82194138)(710.06149549,897.16703139)
\curveto(709.72658951,896.51211769)(709.25215053,896.01628244)(708.63817713,895.67952416)
\curveto(708.02419082,895.34276514)(707.3618368,895.17438582)(706.65111306,895.17438568)
\curveto(705.88084296,895.17438582)(705.1924413,895.36044032)(704.58590603,895.73254974)
\curveto(703.97936595,896.10465833)(703.51981133,896.61258712)(703.20724079,897.25633764)
\curveto(702.8946682,897.90008427)(702.73838242,898.58104375)(702.73838298,899.29921811)
\closepath
\moveto(703.8547111,899.28247319)
\curveto(703.85470943,898.29638036)(704.1198371,897.51960282)(704.65009489,896.95213822)
\curveto(705.18034776,896.38467036)(705.8454926,896.10093724)(706.64553142,896.10093803)
\curveto(707.46044568,896.10093724)(708.13117216,896.38746117)(708.65771287,896.96051069)
\curveto(709.18424064,897.5335569)(709.44750776,898.34661508)(709.44751502,899.39968764)
\curveto(709.44750776,900.06575868)(709.33494479,900.64717899)(709.10982576,901.14395034)
\curveto(708.88469289,901.64071003)(708.55537642,902.02584285)(708.12187537,902.29934996)
\curveto(707.68836245,902.57284309)(707.20182992,902.70959315)(706.66227634,902.70960054)
\curveto(705.89572732,902.70959315)(705.23616411,902.44632603)(704.68358474,901.91979839)
\curveto(704.13100037,901.39325755)(703.85470943,900.51415002)(703.8547111,899.28247319)
\closepath
}
}
{
\newrgbcolor{curcolor}{0 0 0}
\pscustom[linestyle=none,fillstyle=solid,fillcolor=curcolor]
{
\newpath
\moveto(711.58528256,897.94287944)
\lineto(712.6067228,898.03218569)
\curveto(712.65509543,897.62286307)(712.76765841,897.28703469)(712.94441206,897.02469955)
\curveto(713.12116196,896.76236099)(713.39559235,896.55025886)(713.76770405,896.38839252)
\curveto(714.13981036,896.22652403)(714.55843299,896.14559032)(715.02357319,896.14559115)
\curveto(715.43661024,896.14559032)(715.80127706,896.20698831)(716.11757476,896.32978529)
\curveto(716.43386237,896.45258025)(716.66922131,896.62095957)(716.8236523,896.83492377)
\curveto(716.97807178,897.04888493)(717.0552844,897.28238333)(717.05529039,897.53541967)
\curveto(717.0552844,897.79217266)(716.9808626,898.01636834)(716.83202476,898.20800737)
\curveto(716.6831754,898.39964061)(716.43758346,898.56057776)(716.0952482,898.69081928)
\curveto(715.87569886,898.77640098)(715.39009661,898.90942995)(714.63843999,899.08990659)
\curveto(713.88677624,899.27037568)(713.36024199,899.44061555)(713.05883569,899.60062671)
\curveto(712.66811925,899.80528237)(712.37694395,900.05924677)(712.18530893,900.36252065)
\curveto(711.99367168,900.66578444)(711.89785361,901.00533391)(711.89785444,901.38117007)
\curveto(711.89785361,901.794205)(712.01506795,902.18026809)(712.2494978,902.5393605)
\curveto(712.48392529,902.89843847)(712.82626557,903.17100831)(713.27651967,903.35707086)
\curveto(713.72676936,903.54311732)(714.22725597,903.63614457)(714.77798101,903.63615289)
\curveto(715.38451498,903.63614457)(715.91942167,903.53846595)(716.38270269,903.34311676)
\curveto(716.84597309,903.1477515)(717.20226746,902.86029729)(717.45158687,902.48075328)
\curveto(717.70089352,902.10119493)(717.83485277,901.67140903)(717.853465,901.19139429)
\lineto(716.81527984,901.11325132)
\curveto(716.75945774,901.63047704)(716.57061243,902.02119149)(716.24874331,902.28539585)
\curveto(715.92686385,902.54958628)(715.4514946,902.68168497)(714.82263413,902.68169234)
\curveto(714.16771853,902.68168497)(713.69048874,902.56167982)(713.39094331,902.32167652)
\curveto(713.09139324,902.0816592)(712.94161937,901.79234445)(712.94162124,901.4537314)
\curveto(712.94161937,901.15975915)(713.04767043,900.91788829)(713.25977475,900.72811812)
\curveto(713.46815361,900.53833711)(714.01236302,900.34391016)(714.89240464,900.14483667)
\curveto(715.77243861,899.94575352)(716.37618547,899.77179256)(716.70364703,899.62295327)
\curveto(717.17994092,899.40340465)(717.53158393,899.12525317)(717.75857711,898.78849799)
\curveto(717.98555691,898.45173587)(718.09905016,898.06381224)(718.09905719,897.62472592)
\curveto(718.09905016,897.18935608)(717.97439364,896.7791059)(717.72508726,896.39397416)
\curveto(717.47576758,896.00884026)(717.11761266,895.70929251)(716.65062144,895.49533002)
\curveto(716.18361906,895.28136716)(715.65801509,895.17438582)(715.07380796,895.17438568)
\curveto(714.33330704,895.17438582)(713.71281528,895.28229743)(713.21233081,895.49812084)
\curveto(712.71184206,895.71394388)(712.31926706,896.03860898)(712.03460463,896.47211713)
\curveto(711.74994028,896.90562296)(711.60016641,897.39587657)(711.58528256,897.94287944)
\closepath
}
}
{
\newrgbcolor{curcolor}{0 0 0}
\pscustom[linestyle=none,fillstyle=solid,fillcolor=curcolor]
{
\newpath
\moveto(722.35784779,897.47402162)
\lineto(723.3625431,897.607981)
\curveto(723.47789541,897.03865193)(723.67418291,896.62840175)(723.95140619,896.37722924)
\curveto(724.22862532,896.1260546)(724.56631424,896.00046781)(724.96447396,896.0004685)
\curveto(725.43704931,896.00046781)(725.83613622,896.16419577)(726.16173588,896.49165287)
\curveto(726.48732698,896.81910762)(726.65012466,897.22470643)(726.65012944,897.70845053)
\curveto(726.65012466,898.1698633)(726.49942052,898.55034476)(726.19801654,898.84989604)
\curveto(725.89660393,899.14944025)(725.51333166,899.29921413)(725.04819857,899.29921811)
\curveto(724.85841981,899.29921413)(724.62213059,899.26200323)(724.33933021,899.1875853)
\lineto(724.45096302,900.06948452)
\curveto(724.51794007,900.06203759)(724.57189588,900.0583165)(724.6128306,900.05832124)
\curveto(725.04075322,900.0583165)(725.42588604,900.1699492)(725.76823022,900.39321968)
\curveto(726.11056661,900.61648)(726.28173675,900.96068083)(726.28174115,901.4258232)
\curveto(726.28173675,901.794205)(726.15708023,902.09933438)(725.90777123,902.34121226)
\curveto(725.65845417,902.58307609)(725.33657988,902.70401151)(724.9421474,902.7040189)
\curveto(724.55142988,902.70401151)(724.2258345,902.58121554)(723.96536029,902.33563062)
\curveto(723.7048819,902.09003166)(723.53743285,901.72164374)(723.46301263,901.23046577)
\lineto(722.45831732,901.40907827)
\curveto(722.58111271,902.08258948)(722.86019446,902.60447235)(723.29556341,902.97472847)
\curveto(723.73092953,903.34496927)(724.27234813,903.5300935)(724.91982084,903.53010172)
\curveto(725.3663486,903.5300935)(725.77752905,903.43427543)(726.15336342,903.24264722)
\curveto(726.52918924,903.05100316)(726.81664344,902.78959658)(727.0157269,902.45842671)
\curveto(727.21480008,902.12724256)(727.31433924,901.77559955)(727.31434467,901.40349663)
\curveto(727.31433924,901.04998699)(727.21945144,900.7281127)(727.029681,900.4378728)
\curveto(726.83990026,900.14762266)(726.55895796,899.91691507)(726.18685326,899.74574936)
\curveto(726.67059066,899.63411223)(727.04642075,899.40247438)(727.31434467,899.0508351)
\curveto(727.58225772,898.69918836)(727.71621696,898.25916946)(727.7162228,897.73077709)
\curveto(727.71621696,897.01632539)(727.45574066,896.41071799)(726.93479311,895.91395307)
\curveto(726.41383545,895.41718695)(725.75520251,895.16880419)(724.95889232,895.16880404)
\curveto(724.24071886,895.16880419)(723.64441419,895.38276686)(723.1699765,895.81069271)
\curveto(722.69553623,896.23861757)(722.42482693,896.79305999)(722.35784779,897.47402162)
\closepath
}
}
{
\newrgbcolor{curcolor}{0 0 0}
\pscustom[linestyle=none,fillstyle=solid,fillcolor=curcolor]
{
\newpath
\moveto(729.27908329,895.3139267)
\lineto(729.27908329,896.45816303)
\lineto(730.42331962,896.45816303)
\lineto(730.42331962,895.3139267)
\closepath
}
}
{
\newrgbcolor{curcolor}{0 0 0}
\pscustom[linestyle=none,fillstyle=solid,fillcolor=curcolor]
{
\newpath
\moveto(737.16594164,896.27955053)
\lineto(737.16594164,895.3139267)
\lineto(731.75733186,895.3139267)
\curveto(731.74988934,895.55579755)(731.78896078,895.78836568)(731.87454631,896.01163178)
\curveto(732.01222618,896.38001899)(732.23270077,896.74282527)(732.53597073,897.1000517)
\curveto(732.83923844,897.45727456)(733.2773968,897.87031555)(733.8504471,898.33917592)
\curveto(734.73978518,899.06850655)(735.34074122,899.64620577)(735.65331703,900.07227534)
\curveto(735.96588435,900.49833539)(736.12217013,900.90114339)(736.12217484,901.28070054)
\curveto(736.12217013,901.67885121)(735.97983844,902.01467958)(735.69517933,902.28818667)
\curveto(735.41051166,902.56167982)(735.03933293,902.69842988)(734.58164202,902.69843726)
\curveto(734.09789715,902.69842988)(733.71090378,902.55330737)(733.42066077,902.26306929)
\curveto(733.13041374,901.97281732)(732.98343068,901.5709396)(732.97971116,901.05743491)
\lineto(731.94710764,901.16348609)
\curveto(732.01780782,901.93374587)(732.28386576,902.52074783)(732.74528225,902.92449371)
\curveto(733.20669608,903.32822437)(733.82625758,903.5300935)(734.60396859,903.53010172)
\curveto(735.38911539,903.5300935)(736.01053743,903.31240973)(736.46823656,902.87704976)
\curveto(736.92592558,902.44167466)(737.15477262,901.90211661)(737.15477836,901.25837398)
\curveto(737.15477262,900.93091211)(737.08779299,900.60903782)(736.9538393,900.29275015)
\curveto(736.81987451,899.97645251)(736.59753938,899.64341496)(736.28683324,899.29363647)
\curveto(735.97611735,898.94385003)(735.4598161,898.46382941)(734.73792796,897.85357319)
\curveto(734.13510805,897.3475024)(733.74811469,897.00423185)(733.57694671,896.82376049)
\curveto(733.4057744,896.64328611)(733.26437298,896.46188297)(733.15274202,896.27955053)
\closepath
}
}
{
\newrgbcolor{curcolor}{0 0 0}
\pscustom[linestyle=none,fillstyle=solid,fillcolor=curcolor]
{
\newpath
\moveto(821.61140987,895.30892181)
\lineto(821.61140987,903.49160699)
\lineto(823.24124893,903.49160699)
\lineto(825.17807823,897.69786401)
\curveto(825.35668614,897.15830356)(825.48692429,896.75456529)(825.56879308,896.48664799)
\curveto(825.66181552,896.78433401)(825.80693803,897.22156209)(826.00416105,897.79833354)
\lineto(827.96331691,903.49160699)
\lineto(829.42012512,903.49160699)
\lineto(829.42012512,895.30892181)
\lineto(828.37635832,895.30892181)
\lineto(828.37635832,902.15759488)
\lineto(825.99857941,895.30892181)
\lineto(825.0217923,895.30892181)
\lineto(822.65517666,902.27480933)
\lineto(822.65517666,895.30892181)
\closepath
}
}
{
\newrgbcolor{curcolor}{0 0 0}
\pscustom[linestyle=none,fillstyle=solid,fillcolor=curcolor]
{
\newpath
\moveto(831.36253635,895.30892181)
\lineto(831.36253635,903.49160699)
\lineto(832.44537463,903.49160699)
\lineto(832.44537463,895.30892181)
\closepath
}
}
{
\newrgbcolor{curcolor}{0 0 0}
\pscustom[linestyle=none,fillstyle=solid,fillcolor=curcolor]
{
\newpath
\moveto(834.33755093,895.30892181)
\lineto(834.33755093,903.49160699)
\lineto(835.44829742,903.49160699)
\lineto(839.74616071,897.06713862)
\lineto(839.74616071,903.49160699)
\lineto(840.78434587,903.49160699)
\lineto(840.78434587,895.30892181)
\lineto(839.67359938,895.30892181)
\lineto(835.37573609,901.73897183)
\lineto(835.37573609,895.30892181)
\closepath
}
}
{
\newrgbcolor{curcolor}{0 0 0}
\pscustom[linestyle=none,fillstyle=solid,fillcolor=curcolor]
{
\newpath
\moveto(842.79373542,895.30892181)
\lineto(842.79373542,903.49160699)
\lineto(843.87657371,903.49160699)
\lineto(843.87657371,895.30892181)
\closepath
}
}
{
\newrgbcolor{curcolor}{0 0 0}
\pscustom[linestyle=none,fillstyle=solid,fillcolor=curcolor]
{
\newpath
\moveto(844.94824883,895.30892181)
\lineto(848.11303908,899.57329526)
\lineto(845.32221876,903.49160699)
\lineto(846.61157774,903.49160699)
\lineto(848.09629415,901.39291011)
\curveto(848.40514143,900.95753649)(848.62468574,900.62263839)(848.75492775,900.3882148)
\curveto(848.9372573,900.68589692)(849.15308053,900.99660794)(849.40239806,901.32034878)
\lineto(851.04898205,903.49160699)
\lineto(852.22670823,903.49160699)
\lineto(849.3521633,899.63469331)
\lineto(852.44997385,895.30892181)
\lineto(851.1103801,895.30892181)
\lineto(849.0507547,898.22811987)
\curveto(848.93539676,898.395566)(848.81632188,898.57789941)(848.6935297,898.77512065)
\curveto(848.5111925,898.47742998)(848.38095434,898.27277003)(848.30281486,898.16114018)
\lineto(846.2487711,895.30892181)
\closepath
}
}
{
\newrgbcolor{curcolor}{0 0 0}
\pscustom[linestyle=none,fillstyle=solid,fillcolor=curcolor]
{
\newpath
\moveto(859.96286285,895.30892181)
\lineto(858.95816754,895.30892181)
\lineto(858.95816754,901.71106363)
\curveto(858.71629343,901.48034964)(858.3990705,901.24964206)(858.00649781,901.01894019)
\curveto(857.61392051,900.78822689)(857.26134723,900.61519621)(856.94877691,900.49984761)
\lineto(856.94877691,901.47105308)
\curveto(857.51066026,901.73524431)(858.00184414,902.05525805)(858.42233004,902.43109527)
\curveto(858.84281049,902.80691824)(859.14049769,903.17158506)(859.31539254,903.52509683)
\lineto(859.96286285,903.52509683)
\closepath
}
}
{
\newrgbcolor{curcolor}{0 0 0}
\pscustom[linestyle=none,fillstyle=solid,fillcolor=curcolor]
{
\newpath
\moveto(863.10532765,895.30892181)
\lineto(863.10532765,896.45315815)
\lineto(864.24956398,896.45315815)
\lineto(864.24956398,895.30892181)
\closepath
}
}
{
\newrgbcolor{curcolor}{0 0 0}
\pscustom[linestyle=none,fillstyle=solid,fillcolor=curcolor]
{
\newpath
\moveto(869.4963063,895.30892181)
\lineto(868.49161099,895.30892181)
\lineto(868.49161099,901.71106363)
\curveto(868.24973688,901.48034964)(867.93251396,901.24964206)(867.53994126,901.01894019)
\curveto(867.14736396,900.78822689)(866.79479068,900.61519621)(866.48222036,900.49984761)
\lineto(866.48222036,901.47105308)
\curveto(867.04410371,901.73524431)(867.53528759,902.05525805)(867.95577349,902.43109527)
\curveto(868.37625394,902.80691824)(868.67394115,903.17158506)(868.84883599,903.52509683)
\lineto(869.4963063,903.52509683)
\closepath
}
}
{
\newrgbcolor{curcolor}{0 0 0}
\pscustom[linestyle=none,fillstyle=solid,fillcolor=curcolor]
{
\newpath
\moveto(68.54243788,862.26058892)
\lineto(69.62527617,862.26058892)
\lineto(69.62527617,857.5329393)
\curveto(69.62526883,856.71057495)(69.53224158,856.05752365)(69.34619413,855.57378344)
\curveto(69.16013258,855.09004024)(68.8243042,854.69653497)(68.338708,854.39326644)
\curveto(67.8530997,854.08999729)(67.21586303,853.93836287)(66.42699608,853.93836273)
\curveto(65.6604474,853.93836287)(65.03344372,854.07046157)(64.54598318,854.33465922)
\curveto(64.05851813,854.59885635)(63.71059622,854.98119835)(63.50221638,855.48168637)
\curveto(63.29383413,855.98217157)(63.18964361,856.66592187)(63.18964451,857.5329393)
\lineto(63.18964451,862.26058892)
\lineto(64.27248279,862.26058892)
\lineto(64.27248279,857.53852094)
\curveto(64.27248081,856.82778929)(64.33853016,856.30404586)(64.47063104,855.9672891)
\curveto(64.60272755,855.63052857)(64.82971404,855.37098254)(65.15159119,855.18865023)
\curveto(65.47346262,855.00631571)(65.86696789,854.91514901)(66.33210819,854.91514984)
\curveto(67.12841742,854.91514901)(67.69588365,855.09562187)(68.03450858,855.45656899)
\curveto(68.37312203,855.81751334)(68.54243163,856.51149663)(68.54243788,857.53852094)
\closepath
}
}
{
\newrgbcolor{curcolor}{0 0 0}
\pscustom[linestyle=none,fillstyle=solid,fillcolor=curcolor]
{
\newpath
\moveto(71.42256432,854.07790375)
\lineto(71.42256432,862.26058892)
\lineto(72.53331081,862.26058892)
\lineto(76.8311741,855.83612055)
\lineto(76.8311741,862.26058892)
\lineto(77.86935926,862.26058892)
\lineto(77.86935926,854.07790375)
\lineto(76.75861277,854.07790375)
\lineto(72.46074948,860.50795376)
\lineto(72.46074948,854.07790375)
\closepath
}
}
{
\newrgbcolor{curcolor}{0 0 0}
\pscustom[linestyle=none,fillstyle=solid,fillcolor=curcolor]
{
\newpath
\moveto(79.87874976,854.07790375)
\lineto(79.87874976,862.26058892)
\lineto(80.96158805,862.26058892)
\lineto(80.96158805,854.07790375)
\closepath
}
}
{
\newrgbcolor{curcolor}{0 0 0}
\pscustom[linestyle=none,fillstyle=solid,fillcolor=curcolor]
{
\newpath
\moveto(82.03326318,854.07790375)
\lineto(85.19805342,858.3422772)
\lineto(82.4072331,862.26058892)
\lineto(83.69659209,862.26058892)
\lineto(85.1813085,860.16189204)
\curveto(85.49015577,859.72651843)(85.70970008,859.39162032)(85.83994209,859.15719673)
\curveto(86.02227165,859.45487885)(86.23809487,859.76558987)(86.48741241,860.08933072)
\lineto(88.1339964,862.26058892)
\lineto(89.31172257,862.26058892)
\lineto(86.43717764,858.40367524)
\lineto(89.5349882,854.07790375)
\lineto(88.19539444,854.07790375)
\lineto(86.13576905,856.9971018)
\curveto(86.0204111,857.16454793)(85.90133622,857.34688135)(85.77854405,857.54410258)
\curveto(85.59620684,857.24641192)(85.46596869,857.04175196)(85.3878292,856.93012211)
\lineto(83.33378545,854.07790375)
\closepath
}
}
{
\newrgbcolor{curcolor}{0 0 0}
\pscustom[linestyle=none,fillstyle=solid,fillcolor=curcolor]
{
\newpath
\moveto(93.30259633,856.70685649)
\lineto(94.32403656,856.79616274)
\curveto(94.3724092,856.38684012)(94.48497217,856.05101174)(94.66172582,855.7886766)
\curveto(94.83847573,855.52633805)(95.11290612,855.31423591)(95.48501782,855.15236957)
\curveto(95.85712412,854.99050108)(96.27574675,854.90956737)(96.74088696,854.9095682)
\curveto(97.153924,854.90956737)(97.51859083,854.97096536)(97.83488853,855.09376234)
\curveto(98.15117613,855.2165573)(98.38653508,855.38493662)(98.54096607,855.59890082)
\curveto(98.69538555,855.81286198)(98.77259817,856.04636038)(98.77260415,856.29939672)
\curveto(98.77259817,856.55614971)(98.69817637,856.78034539)(98.54933853,856.97198442)
\curveto(98.40048917,857.16361766)(98.15489722,857.32455481)(97.81256196,857.45479633)
\curveto(97.59301263,857.54037803)(97.10741038,857.673407)(96.35575376,857.85388364)
\curveto(95.60409,858.03435273)(95.07755576,858.2045926)(94.77614946,858.36460376)
\curveto(94.38543301,858.56925942)(94.09425772,858.82322382)(93.9026227,859.12649771)
\curveto(93.71098545,859.42976149)(93.61516738,859.76931096)(93.6151682,860.14514712)
\curveto(93.61516738,860.55818205)(93.73238171,860.94424514)(93.96681156,861.30333755)
\curveto(94.20123906,861.66241552)(94.54357934,861.93498536)(94.99383344,862.12104791)
\curveto(95.44408313,862.30709437)(95.94456974,862.40012162)(96.49529477,862.40012994)
\curveto(97.10182874,862.40012162)(97.63673543,862.302443)(98.10001646,862.10709381)
\curveto(98.56328685,861.91172855)(98.91958123,861.62427434)(99.16890064,861.24473033)
\curveto(99.41820729,860.86517198)(99.55216653,860.43538608)(99.57077876,859.95537134)
\lineto(98.53259361,859.87722837)
\curveto(98.47677151,860.39445409)(98.28792619,860.78516854)(97.96605708,861.04937291)
\curveto(97.64417762,861.31356333)(97.16880836,861.44566202)(96.5399479,861.44566939)
\curveto(95.8850323,861.44566202)(95.4078025,861.32565687)(95.10825707,861.08565357)
\curveto(94.80870701,860.84563625)(94.65893313,860.5563215)(94.658935,860.21770845)
\curveto(94.65893313,859.9237362)(94.7649842,859.68186534)(94.97708852,859.49209517)
\curveto(95.18546737,859.30231416)(95.72967679,859.10788721)(96.60971841,858.90881372)
\curveto(97.48975238,858.70973057)(98.09349924,858.53576961)(98.42096079,858.38693032)
\curveto(98.89725469,858.1673817)(99.24889769,857.88923022)(99.47589087,857.55247505)
\curveto(99.70287068,857.21571292)(99.81636392,856.82778929)(99.81637095,856.38870297)
\curveto(99.81636392,855.95333313)(99.69170741,855.54308295)(99.44240103,855.15795121)
\curveto(99.19308134,854.77281731)(98.83492643,854.47326956)(98.36793521,854.25930707)
\curveto(97.90093283,854.04534421)(97.37532886,853.93836287)(96.79112173,853.93836273)
\curveto(96.05062081,853.93836287)(95.43012904,854.04627448)(94.92964457,854.26209789)
\curveto(94.42915582,854.47792093)(94.03658082,854.80258603)(93.7519184,855.23609418)
\curveto(93.46725405,855.66960001)(93.31748017,856.15985362)(93.30259633,856.70685649)
\closepath
}
}
{
\newrgbcolor{curcolor}{0 0 0}
\pscustom[linestyle=none,fillstyle=solid,fillcolor=curcolor]
{
\newpath
\moveto(101.13363872,851.79501273)
\lineto(101.0220059,852.73830999)
\curveto(101.24154962,852.67877389)(101.43318576,852.64900517)(101.59691489,852.64900374)
\curveto(101.82017912,852.64900517)(101.99879144,852.68621607)(102.13275239,852.76063656)
\curveto(102.26670992,852.83505968)(102.37648208,852.9392502)(102.46206919,853.07320843)
\curveto(102.52532568,853.17367887)(102.62765566,853.4229919)(102.76905942,853.82114828)
\curveto(102.78766253,853.87696489)(102.81743125,853.95882887)(102.85836567,854.06674047)
\lineto(100.6089645,860.00560611)
\lineto(101.69180278,860.00560611)
\lineto(102.92534536,856.57289711)
\curveto(103.08534973,856.13752708)(103.2286117,855.67983301)(103.35513169,855.19981352)
\curveto(103.47048255,855.66122756)(103.60816288,856.11147945)(103.7681731,856.55057055)
\lineto(105.03520552,860.00560611)
\lineto(106.03990084,860.00560611)
\lineto(103.78491802,853.97743422)
\curveto(103.54304381,853.32624356)(103.35512876,852.87785221)(103.22117232,852.63225882)
\curveto(103.0425572,852.30108325)(102.83789724,852.05828213)(102.60719184,851.90385472)
\curveto(102.37648208,851.74943166)(102.10112142,851.67221904)(101.78110903,851.67221663)
\curveto(101.58761099,851.67221904)(101.37178777,851.71315103)(101.13363872,851.79501273)
\closepath
}
}
{
\newrgbcolor{curcolor}{0 0 0}
\pscustom[linestyle=none,fillstyle=solid,fillcolor=curcolor]
{
\newpath
\moveto(106.49201278,855.84728383)
\lineto(107.48554481,856.00356977)
\curveto(107.54135981,855.60541121)(107.69671532,855.30028183)(107.9516118,855.0881807)
\curveto(108.20650466,854.87607756)(108.56279903,854.77002649)(109.02049598,854.77002719)
\curveto(109.48190827,854.77002649)(109.82424855,854.86398402)(110.04751786,855.05190004)
\curveto(110.27077936,855.23981411)(110.38241206,855.4602887)(110.3824163,855.71332445)
\curveto(110.38241206,855.94030931)(110.28380317,856.11892163)(110.08658935,856.24916196)
\curveto(109.94890507,856.33846595)(109.60656479,856.45195919)(109.05956747,856.58964204)
\curveto(108.32278872,856.77569403)(107.81206911,856.93663117)(107.52740711,857.07245395)
\curveto(107.24274234,857.20827074)(107.02691912,857.39618579)(106.8799368,857.63619965)
\curveto(106.732953,857.8762064)(106.65946148,858.14133407)(106.65946199,858.43158345)
\curveto(106.65946148,858.69577648)(106.71992919,858.94043815)(106.84086532,859.16556919)
\curveto(106.96180004,859.39069005)(107.12645828,859.57767482)(107.33484051,859.72652407)
\curveto(107.4911251,859.84187222)(107.7041575,859.93955083)(107.97393837,860.01956021)
\curveto(108.24371556,860.0995577)(108.53303031,860.13955942)(108.84188348,860.13956548)
\curveto(109.30701704,860.13955942)(109.71540667,860.0725798)(110.0670536,859.93862642)
\curveto(110.41869269,859.80466132)(110.67823872,859.62325818)(110.84569247,859.39441646)
\curveto(111.01313682,859.1655641)(111.12849061,858.85950445)(111.19175419,858.47623657)
\lineto(110.20938544,858.3422772)
\curveto(110.16472829,858.64740231)(110.03542041,858.88555208)(109.82146142,859.0567272)
\curveto(109.60749506,859.22789236)(109.30515649,859.31347743)(108.91444481,859.31348267)
\curveto(108.45302687,859.31347743)(108.12371041,859.23719509)(107.92649442,859.0846354)
\curveto(107.72927486,858.9320657)(107.63066598,858.75345338)(107.63066747,858.5487979)
\curveto(107.63066598,858.41855528)(107.67159797,858.30134094)(107.75346356,858.19715454)
\curveto(107.83532593,858.08923881)(107.96370353,857.99993265)(108.13859676,857.92923579)
\curveto(108.2390642,857.89202104)(108.53489085,857.80643597)(109.02607763,857.67248032)
\curveto(109.73680294,857.48270113)(110.23263818,857.32734562)(110.51358486,857.20641333)
\curveto(110.79452278,857.08547477)(111.01499736,856.90965327)(111.17500927,856.67894829)
\curveto(111.33501111,856.4482381)(111.41501454,856.16171417)(111.41501982,855.81937563)
\curveto(111.41501454,855.48447578)(111.31733593,855.1691134)(111.12198369,854.87328754)
\curveto(110.92662148,854.57746009)(110.64474891,854.34861305)(110.27636513,854.18674574)
\curveto(109.90797308,854.02487821)(109.49121099,853.94394451)(109.02607763,853.94394437)
\curveto(108.2558091,853.94394451)(107.66880715,854.10395138)(107.26507,854.42396547)
\curveto(106.86133061,854.74397886)(106.60364512,855.21841784)(106.49201278,855.84728383)
\closepath
}
}
{
\newrgbcolor{curcolor}{0 0 0}
\pscustom[linestyle=none,fillstyle=solid,fillcolor=curcolor]
{
\newpath
\moveto(114.80307473,854.97654789)
\lineto(114.94819739,854.08906703)
\curveto(114.66539145,854.02952958)(114.41235733,853.99976086)(114.18909426,853.99976078)
\curveto(113.82442511,853.99976086)(113.54162226,854.05743775)(113.34068489,854.17279164)
\curveto(113.13974454,854.28814533)(112.99834312,854.43977975)(112.9164802,854.62769535)
\curveto(112.83461516,854.81560985)(112.79368316,855.21097566)(112.7936841,855.81379399)
\lineto(112.7936841,859.22417642)
\lineto(112.05690754,859.22417642)
\lineto(112.05690754,860.00560611)
\lineto(112.7936841,860.00560611)
\lineto(112.7936841,861.47357759)
\lineto(113.79279778,862.07639478)
\lineto(113.79279778,860.00560611)
\lineto(114.80307473,860.00560611)
\lineto(114.80307473,859.22417642)
\lineto(113.79279778,859.22417642)
\lineto(113.79279778,855.75797758)
\curveto(113.79279584,855.47145197)(113.81047102,855.28725801)(113.84582336,855.20539516)
\curveto(113.88117173,855.12353005)(113.93884862,855.05841097)(114.01885422,855.01003773)
\curveto(114.0988555,854.96166263)(114.21327901,854.93747555)(114.36212512,854.93747641)
\curveto(114.47375532,854.93747555)(114.62073837,854.95049936)(114.80307473,854.97654789)
\closepath
}
}
{
\newrgbcolor{curcolor}{0 0 0}
\pscustom[linestyle=none,fillstyle=solid,fillcolor=curcolor]
{
\newpath
\moveto(119.83771471,855.98682485)
\lineto(120.87589987,855.85844711)
\curveto(120.71216606,855.25190765)(120.40889722,854.78118976)(119.96609245,854.44629203)
\curveto(119.5232778,854.11139356)(118.95767211,853.94394451)(118.2692737,853.94394437)
\curveto(117.40225647,853.94394451)(116.71478509,854.21093272)(116.20685748,854.7449098)
\curveto(115.69892751,855.27888556)(115.44496311,856.02775493)(115.44496353,856.99152016)
\curveto(115.44496311,857.98876938)(115.70171833,858.76275611)(116.21522994,859.31348267)
\curveto(116.72873918,859.86419876)(117.39481429,860.13955942)(118.21345729,860.13956548)
\curveto(119.00604628,860.13955942)(119.65351595,859.86978039)(120.15586823,859.33022759)
\curveto(120.65821026,858.79066428)(120.90938383,858.03156191)(120.90938972,857.05291821)
\curveto(120.90938383,856.99337779)(120.90752329,856.90407163)(120.90380808,856.78499946)
\lineto(116.48314869,856.78499946)
\curveto(116.52035813,856.13380599)(116.70455209,855.63517993)(117.03573111,855.28911977)
\curveto(117.36690612,854.94305718)(117.77994711,854.77002649)(118.27485534,854.77002719)
\curveto(118.64324,854.77002649)(118.95767211,854.86677484)(119.2181526,855.0602725)
\curveto(119.47862471,855.2537682)(119.68514521,855.56261867)(119.83771471,855.98682485)
\closepath
\moveto(116.5389651,857.61108227)
\lineto(119.848878,857.61108227)
\curveto(119.80422009,858.1097048)(119.67770303,858.48367435)(119.46932643,858.73299204)
\curveto(119.14930825,859.11998075)(118.73440671,859.31347743)(118.22462057,859.31348267)
\curveto(117.76320221,859.31347743)(117.37527857,859.15905219)(117.0608485,858.85020649)
\curveto(116.74641435,858.54135125)(116.5724534,858.12831025)(116.5389651,857.61108227)
\closepath
}
}
{
\newrgbcolor{curcolor}{0 0 0}
\pscustom[linestyle=none,fillstyle=solid,fillcolor=curcolor]
{
\newpath
\moveto(122.14293341,854.07790375)
\lineto(122.14293341,860.00560611)
\lineto(123.04157756,860.00560611)
\lineto(123.04157756,859.17394165)
\curveto(123.2276304,859.46418158)(123.47508289,859.69767998)(123.78393576,859.87443755)
\curveto(124.09278384,860.05118353)(124.44442685,860.13955942)(124.83886584,860.13956548)
\curveto(125.27795102,860.13955942)(125.63796648,860.04839271)(125.9189133,859.86606509)
\curveto(126.19985107,859.68372589)(126.39799912,859.42883122)(126.51335803,859.10138032)
\curveto(126.98221025,859.79349805)(127.59246902,860.13955942)(128.34413616,860.13956548)
\curveto(128.93206143,860.13955942)(129.38417387,859.97676173)(129.70047484,859.65117193)
\curveto(130.01675918,859.32557097)(130.17490551,858.82415409)(130.17491429,858.14691977)
\lineto(130.17491429,854.07790375)
\lineto(129.17580062,854.07790375)
\lineto(129.17580062,857.81202133)
\curveto(129.17579283,858.21389532)(129.14323329,858.50321008)(129.07812191,858.67996645)
\curveto(129.01299514,858.85671363)(128.89485053,858.99904532)(128.72368773,859.10696196)
\curveto(128.55251025,859.21486855)(128.35157139,859.26882435)(128.12087054,859.26882954)
\curveto(127.70410172,859.26882435)(127.35804035,859.13021375)(127.08268538,858.85299731)
\curveto(126.80731902,858.57577133)(126.66963869,858.13203134)(126.66964397,857.52177602)
\lineto(126.66964397,854.07790375)
\lineto(125.66494866,854.07790375)
\lineto(125.66494866,857.92923579)
\curveto(125.66494438,858.37576274)(125.5830804,858.71066084)(125.41935647,858.9339311)
\curveto(125.25562448,859.15719165)(124.98770599,859.26882435)(124.61560022,859.26882954)
\curveto(124.33279415,859.26882435)(124.07138757,859.19440255)(123.83137971,859.04556392)
\curveto(123.59136696,858.89671535)(123.417406,858.67903158)(123.30949631,858.39251196)
\curveto(123.20158277,858.10598371)(123.14762697,857.69294272)(123.14762873,857.15338774)
\lineto(123.14762873,854.07790375)
\closepath
}
}
{
\newrgbcolor{curcolor}{0 0 0}
\pscustom[linestyle=none,fillstyle=solid,fillcolor=curcolor]
{
\newpath
\moveto(137.31383022,854.07790375)
\lineto(134.14345834,862.26058892)
\lineto(135.31560287,862.26058892)
\lineto(137.44220795,856.31614164)
\curveto(137.61337475,855.83983988)(137.75663671,855.39330908)(137.87199428,854.97654789)
\curveto(137.99850757,855.4230778)(138.14549062,855.8696086)(138.31294389,856.31614164)
\lineto(140.52327359,862.26058892)
\lineto(141.62843843,862.26058892)
\lineto(138.42457671,854.07790375)
\closepath
}
}
{
\newrgbcolor{curcolor}{0 0 0}
\pscustom[linestyle=none,fillstyle=solid,fillcolor=curcolor]
{
\newpath
\moveto(142.76709751,854.07790375)
\lineto(142.76709751,855.22214008)
\lineto(143.91133384,855.22214008)
\lineto(143.91133384,854.07790375)
\closepath
}
}
{
\newrgbcolor{curcolor}{0 0 0}
\pscustom[linestyle=none,fillstyle=solid,fillcolor=curcolor]
{
\newpath
\moveto(145.37930546,856.23799867)
\lineto(146.38400077,856.37195805)
\curveto(146.49935308,855.80262898)(146.69564058,855.3923788)(146.97286386,855.14120629)
\curveto(147.25008299,854.89003165)(147.58777192,854.76444486)(147.98593164,854.76444555)
\curveto(148.45850698,854.76444486)(148.85759389,854.92817282)(149.18319355,855.25562992)
\curveto(149.50878465,855.58308467)(149.67158234,855.98868348)(149.67158711,856.47242758)
\curveto(149.67158234,856.93384035)(149.52087819,857.31432181)(149.21947422,857.61387309)
\curveto(148.9180616,857.9134173)(148.53478933,858.06319118)(148.06965625,858.06319516)
\curveto(147.87987748,858.06319118)(147.64358827,858.02598028)(147.36078788,857.95156235)
\lineto(147.4724207,858.83346157)
\curveto(147.53939774,858.82601464)(147.59335355,858.82229355)(147.63428828,858.82229829)
\curveto(148.06221089,858.82229355)(148.44734371,858.93392625)(148.78968789,859.15719673)
\curveto(149.13202428,859.38045705)(149.30319442,859.72465788)(149.30319883,860.18980025)
\curveto(149.30319442,860.55818205)(149.17853791,860.86331143)(148.9292289,861.10518931)
\curveto(148.67991184,861.34705314)(148.35803755,861.46798856)(147.96360507,861.46799595)
\curveto(147.57288755,861.46798856)(147.24729218,861.34519259)(146.98681796,861.09960767)
\curveto(146.72633957,860.85400871)(146.55889052,860.48562079)(146.4844703,859.99444282)
\lineto(145.47977499,860.17305532)
\curveto(145.60257038,860.84656653)(145.88165213,861.3684494)(146.31702108,861.73870552)
\curveto(146.7523872,862.10894632)(147.2938058,862.29407055)(147.94127851,862.29407877)
\curveto(148.38780627,862.29407055)(148.79898672,862.19825248)(149.17482109,862.00662427)
\curveto(149.55064691,861.81498021)(149.83810112,861.55357363)(150.03718457,861.22240377)
\curveto(150.23625775,860.89121961)(150.33579691,860.5395766)(150.33580234,860.16747368)
\curveto(150.33579691,859.81396404)(150.24090911,859.49208975)(150.05113867,859.20184985)
\curveto(149.86135793,858.91159971)(149.58041563,858.68089212)(149.20831094,858.50972641)
\curveto(149.69204833,858.39808928)(150.06787843,858.16645143)(150.33580234,857.81481216)
\curveto(150.60371539,857.46316541)(150.73767463,857.02314651)(150.73768047,856.49475414)
\curveto(150.73767463,855.78030244)(150.47719833,855.17469504)(149.95625078,854.67793012)
\curveto(149.43529312,854.181164)(148.77666018,853.93278124)(147.98035,853.93278109)
\curveto(147.26217654,853.93278124)(146.66587186,854.14674391)(146.19143417,854.57466976)
\curveto(145.7169939,855.00259462)(145.4462846,855.55703704)(145.37930546,856.23799867)
\closepath
}
}
{
\newrgbcolor{curcolor}{0 0 0}
\pscustom[linestyle=none,fillstyle=solid,fillcolor=curcolor]
{
\newpath
\moveto(583.48807589,850.11190033)
\lineto(583.48807589,852.07105619)
\lineto(579.93815244,852.07105619)
\lineto(579.93815244,852.9920269)
\lineto(583.67227003,858.29458551)
\lineto(584.4927712,858.29458551)
\lineto(584.4927712,852.9920269)
\lineto(585.59793605,852.9920269)
\lineto(585.59793605,852.07105619)
\lineto(584.4927712,852.07105619)
\lineto(584.4927712,850.11190033)
\closepath
\moveto(583.48807589,852.9920269)
\lineto(583.48807589,856.68149136)
\lineto(580.92610283,852.9920269)
\closepath
}
}
{
\newrgbcolor{curcolor}{0 0 0}
\pscustom[linestyle=none,fillstyle=solid,fillcolor=curcolor]
{
\newpath
\moveto(587.19428543,850.11190033)
\lineto(587.19428543,851.25613666)
\lineto(588.33852176,851.25613666)
\lineto(588.33852176,850.11190033)
\closepath
}
}
{
\newrgbcolor{curcolor}{0 0 0}
\pscustom[linestyle=none,fillstyle=solid,fillcolor=curcolor]
{
\newpath
\moveto(589.80649338,852.27199526)
\lineto(590.81118869,852.40595463)
\curveto(590.926541,851.83662556)(591.1228285,851.42637539)(591.40005178,851.17520287)
\curveto(591.67727091,850.92402823)(592.01495983,850.79844144)(592.41311956,850.79844213)
\curveto(592.8856949,850.79844144)(593.28478181,850.9621694)(593.61038147,851.2896265)
\curveto(593.93597257,851.61708125)(594.09877026,852.02268006)(594.09877503,852.50642416)
\curveto(594.09877026,852.96783693)(593.94806611,853.34831839)(593.64666214,853.64786967)
\curveto(593.34524952,853.94741389)(592.96197725,854.09718776)(592.49684417,854.09719175)
\curveto(592.3070654,854.09718776)(592.07077619,854.05997686)(591.7879758,853.98555893)
\lineto(591.89960862,854.86745815)
\curveto(591.96658566,854.86001122)(592.02054147,854.85629013)(592.0614762,854.85629487)
\curveto(592.48939881,854.85629013)(592.87453163,854.96792283)(593.21687581,855.19119331)
\curveto(593.5592122,855.41445363)(593.73038234,855.75865446)(593.73038675,856.22379683)
\curveto(593.73038234,856.59217863)(593.60572583,856.89730801)(593.35641682,857.13918589)
\curveto(593.10709976,857.38104972)(592.78522547,857.50198515)(592.39079299,857.50199254)
\curveto(592.00007547,857.50198515)(591.6744801,857.37918917)(591.41400588,857.13360425)
\curveto(591.15352749,856.88800529)(590.98607844,856.51961738)(590.91165822,856.02843941)
\lineto(589.90696291,856.20705191)
\curveto(590.0297583,856.88056311)(590.30884005,857.40244599)(590.744209,857.77270211)
\curveto(591.17957512,858.1429429)(591.72099372,858.32806713)(592.36846643,858.32807535)
\curveto(592.81499419,858.32806713)(593.22617464,858.23224906)(593.60200901,858.04062086)
\curveto(593.97783483,857.84897679)(594.26528904,857.58757022)(594.46437249,857.25640035)
\curveto(594.66344567,856.92521619)(594.76298483,856.57357318)(594.76299026,856.20147027)
\curveto(594.76298483,855.84796062)(594.66809703,855.52608633)(594.47832659,855.23584644)
\curveto(594.28854585,854.94559629)(594.00760355,854.71488871)(593.63549886,854.543723)
\curveto(594.11923625,854.43208586)(594.49506634,854.20044801)(594.76299026,853.84880874)
\curveto(595.03090331,853.49716199)(595.16486255,853.05714309)(595.16486839,852.52875073)
\curveto(595.16486255,851.81429902)(594.90438625,851.20869162)(594.3834387,850.7119267)
\curveto(593.86248104,850.21516058)(593.2038481,849.96677782)(592.40753791,849.96677767)
\curveto(591.68936446,849.96677782)(591.09305978,850.1807405)(590.61862209,850.60866635)
\curveto(590.14418182,851.0365912)(589.87347252,851.59103362)(589.80649338,852.27199526)
\closepath
}
}
{
\newrgbcolor{curcolor}{0 0 0}
\pscustom[linestyle=none,fillstyle=solid,fillcolor=curcolor]
{
\newpath
\moveto(596.52678887,850.11190033)
\lineto(596.52678887,858.29458551)
\lineto(599.59669122,858.29458551)
\curveto(600.22183044,858.29457732)(600.72324732,858.21178307)(601.10094337,858.0462025)
\curveto(601.4786286,857.88060606)(601.77445525,857.62571139)(601.98842423,857.28151773)
\curveto(602.20238061,856.93730973)(602.30936195,856.57729427)(602.30936857,856.20147027)
\curveto(602.30936195,855.85168171)(602.21447415,855.52236524)(602.02470489,855.21351987)
\curveto(601.83492297,854.9046643)(601.54839903,854.65535127)(601.16513224,854.46558003)
\curveto(601.66003174,854.32045316)(602.04051319,854.07300068)(602.30657775,853.72322182)
\curveto(602.57262907,853.37343575)(602.70565804,852.96039475)(602.70566505,852.4840976)
\curveto(602.70565804,852.10082296)(602.62472433,851.74452858)(602.46286369,851.41521342)
\curveto(602.3009895,851.08589565)(602.10098091,850.83193125)(601.86283732,850.65331947)
\curveto(601.62468138,850.47470661)(601.32606391,850.33981709)(600.96698399,850.24865053)
\curveto(600.60789353,850.15748368)(600.16787463,850.11190033)(599.64692598,850.11190033)
\closepath
\moveto(597.60962715,854.85629487)
\lineto(599.37900723,854.85629487)
\curveto(599.85902416,854.85629013)(600.20322499,854.88791939)(600.41161075,854.95118276)
\curveto(600.68696669,855.0330419)(600.89441746,855.16886169)(601.03396368,855.35864253)
\curveto(601.17349921,855.54841288)(601.24326965,855.78656264)(601.24327521,856.07309253)
\curveto(601.24326965,856.34472614)(601.17815058,856.58380618)(601.04791778,856.79033335)
\curveto(600.91767427,856.99684717)(600.73161977,857.13824859)(600.48975372,857.21453804)
\curveto(600.24787807,857.29081329)(599.83297653,857.32895446)(599.24504786,857.32896168)
\lineto(597.60962715,857.32896168)
\closepath
\moveto(597.60962715,851.07752416)
\lineto(599.64692598,851.07752416)
\curveto(599.99670449,851.07752319)(600.24229643,851.09054701)(600.38370255,851.11659564)
\curveto(600.63301089,851.16124772)(600.84139193,851.23566952)(601.0088463,851.33986127)
\curveto(601.17629003,851.44405056)(601.31397036,851.59568498)(601.42188771,851.79476498)
\curveto(601.52979358,851.99384162)(601.58374939,852.22361893)(601.58375528,852.4840976)
\curveto(601.58374939,852.78922461)(601.5056065,853.05435228)(601.34932638,853.27948139)
\curveto(601.19303494,853.50460417)(600.97628144,853.6627505)(600.69906524,853.75392085)
\curveto(600.42183903,853.84508391)(600.02275212,853.89066726)(599.50180333,853.89067104)
\lineto(597.60962715,853.89067104)
\closepath
}
}
{
\newrgbcolor{curcolor}{0 0 0}
\pscustom[linestyle=none,fillstyle=solid,fillcolor=curcolor]
{
\newpath
\moveto(603.83873868,852.74085307)
\lineto(604.86017891,852.83015932)
\curveto(604.90855155,852.4208367)(605.02111452,852.08500832)(605.19786817,851.82267319)
\curveto(605.37461808,851.56033463)(605.64904847,851.34823249)(606.02116017,851.18636615)
\curveto(606.39326647,851.02449766)(606.8118891,850.94356395)(607.27702931,850.94356478)
\curveto(607.69006635,850.94356395)(608.05473318,851.00496194)(608.37103087,851.12775893)
\curveto(608.68731848,851.25055388)(608.92267743,851.41893321)(609.07710842,851.6328974)
\curveto(609.2315279,851.84685856)(609.30874052,852.08035696)(609.3087465,852.3333933)
\curveto(609.30874052,852.5901463)(609.23431872,852.81434197)(609.08548088,853.005981)
\curveto(608.93663151,853.19761424)(608.69103957,853.35855139)(608.34870431,853.48879292)
\curveto(608.12915498,853.57437461)(607.64355273,853.70740358)(606.89189611,853.88788022)
\curveto(606.14023235,854.06834931)(605.61369811,854.23858918)(605.3122918,854.39860034)
\curveto(604.92157536,854.60325601)(604.63040007,854.8572204)(604.43876504,855.16049429)
\curveto(604.24712779,855.46375808)(604.15130973,855.80330754)(604.15131055,856.1791437)
\curveto(604.15130973,856.59217863)(604.26852406,856.97824172)(604.50295391,857.33733414)
\curveto(604.73738141,857.6964121)(605.07972169,857.96898194)(605.52997579,858.15504449)
\curveto(605.98022548,858.34109095)(606.48071209,858.4341182)(607.03143712,858.43412652)
\curveto(607.63797109,858.4341182)(608.17287778,858.33643959)(608.6361588,858.14109039)
\curveto(609.0994292,857.94572513)(609.45572357,857.65827093)(609.70504299,857.27872691)
\curveto(609.95434964,856.89916856)(610.08830888,856.46938266)(610.10692111,855.98936792)
\lineto(609.06873595,855.91122495)
\curveto(609.01291386,856.42845067)(608.82406854,856.81916512)(608.50219943,857.08336949)
\curveto(608.18031996,857.34755991)(607.70495071,857.4796586)(607.07609025,857.47966597)
\curveto(606.42117465,857.4796586)(605.94394485,857.35965345)(605.64439942,857.11965015)
\curveto(605.34484936,856.87963284)(605.19507548,856.59031809)(605.19507735,856.25170503)
\curveto(605.19507548,855.95773278)(605.30112655,855.71586193)(605.51323087,855.52609175)
\curveto(605.72160972,855.33631074)(606.26581914,855.14188379)(607.14586075,854.9428103)
\curveto(608.02589473,854.74372715)(608.62964159,854.5697662)(608.95710314,854.4209269)
\curveto(609.43339703,854.20137828)(609.78504004,853.9232268)(610.01203322,853.58647163)
\curveto(610.23901303,853.2497095)(610.35250627,852.86178587)(610.3525133,852.42269955)
\curveto(610.35250627,851.98732971)(610.22784976,851.57707953)(609.97854338,851.19194779)
\curveto(609.72922369,850.80681389)(609.37106878,850.50726615)(608.90407756,850.29330365)
\curveto(608.43707518,850.07934079)(607.91147121,849.97235945)(607.32726408,849.97235931)
\curveto(606.58676316,849.97235945)(605.96627139,850.08027106)(605.46578692,850.29609447)
\curveto(604.96529817,850.51191751)(604.57272317,850.83658261)(604.28806075,851.27009076)
\curveto(604.0033964,851.70359659)(603.85362252,852.19385021)(603.83873868,852.74085307)
\closepath
}
}
{
\newrgbcolor{curcolor}{0 0 0}
\pscustom[linestyle=none,fillstyle=solid,fillcolor=curcolor]
{
\newpath
\moveto(611.84281002,850.11190033)
\lineto(611.84281002,858.29458551)
\lineto(614.66153854,858.29458551)
\curveto(615.29784124,858.29457732)(615.78344349,858.25550588)(616.11834675,858.17737105)
\curveto(616.58719893,858.06945138)(616.98721611,857.87409415)(617.31839948,857.59129879)
\curveto(617.75003957,857.22662448)(618.07284413,856.76055795)(618.28681414,856.19309781)
\curveto(618.50076949,855.62562549)(618.60775082,854.97722555)(618.60775847,854.24789604)
\curveto(618.60775082,853.62646987)(618.53518957,853.07574855)(618.39007449,852.59573041)
\curveto(618.24494455,852.11570732)(618.05889004,851.71848095)(617.83191042,851.40405014)
\curveto(617.60491706,851.08961674)(617.3565343,850.84216425)(617.0867614,850.66169193)
\curveto(616.81697624,850.48121852)(616.49138087,850.34446846)(616.10997429,850.25144135)
\curveto(615.72855741,850.15841396)(615.29039906,850.11190033)(614.79549792,850.11190033)
\closepath
\moveto(612.9256483,851.07752416)
\lineto(614.67270182,851.07752416)
\curveto(615.21225617,851.07752319)(615.63553016,851.12775791)(615.94252507,851.22822846)
\curveto(616.24951001,851.32869677)(616.49417168,851.47009819)(616.67651081,851.65243315)
\curveto(616.93326031,851.90918682)(617.1332689,852.25431792)(617.27653718,852.68782748)
\curveto(617.41979283,853.1213319)(617.49142381,853.64693587)(617.49143034,854.26464096)
\curveto(617.49142381,855.12048752)(617.35095266,855.77819019)(617.07001648,856.23775093)
\curveto(616.78906807,856.69729942)(616.44765806,857.00521962)(616.04578542,857.16151246)
\curveto(615.75553531,857.27313811)(615.28853851,857.32895446)(614.64479362,857.32896168)
\lineto(612.9256483,857.32896168)
\closepath
}
}
{
\newrgbcolor{curcolor}{0 0 0}
\pscustom[linestyle=none,fillstyle=solid,fillcolor=curcolor]
{
\newpath
\moveto(625.35596391,850.11190033)
\lineto(625.35596391,857.32896168)
\lineto(622.66003148,857.32896168)
\lineto(622.66003148,858.29458551)
\lineto(629.1458979,858.29458551)
\lineto(629.1458979,857.32896168)
\lineto(626.43880219,857.32896168)
\lineto(626.43880219,850.11190033)
\closepath
}
}
{
\newrgbcolor{curcolor}{0 0 0}
\pscustom[linestyle=none,fillstyle=solid,fillcolor=curcolor]
{
\newpath
\moveto(634.00192377,850.84309525)
\curveto(633.62981014,850.52680187)(633.27165522,850.30353647)(632.92745794,850.17329838)
\curveto(632.58325357,850.04306016)(632.21393538,849.97794109)(631.81950228,849.97794095)
\curveto(631.16830908,849.97794109)(630.66782247,850.13701769)(630.31804094,850.45517123)
\curveto(629.96825754,850.77332408)(629.79336631,851.17985317)(629.79336672,851.67475971)
\curveto(629.79336631,851.96500317)(629.85941566,852.23013083)(629.99151497,852.4701435)
\curveto(630.12361305,852.71015145)(630.29664374,852.90271786)(630.51060755,853.04784331)
\curveto(630.72456909,853.19296288)(630.96550967,853.30273504)(631.23343001,853.3771601)
\curveto(631.43064593,853.4292521)(631.72833313,853.47948681)(632.12649251,853.5278644)
\curveto(632.93768739,853.62460933)(633.53492234,853.73996312)(633.91819916,853.87392612)
\curveto(633.92191571,854.01160269)(633.92377625,854.09904831)(633.9237808,854.13626323)
\curveto(633.92377625,854.54557911)(633.82888846,854.83396359)(633.63911712,855.00141753)
\curveto(633.38235765,855.22839913)(633.00094592,855.34189238)(632.49488079,855.34189761)
\curveto(632.02229924,855.34189238)(631.67344705,855.25909812)(631.44832317,855.0935146)
\curveto(631.22319516,854.92792111)(631.05667638,854.63488527)(630.94876634,854.2144062)
\lineto(629.96639758,854.34836557)
\curveto(630.05570316,854.76884451)(630.20268622,855.10839398)(630.40734719,855.36701499)
\curveto(630.61200612,855.62562549)(630.90783278,855.82470381)(631.29482806,855.96425054)
\curveto(631.68181951,856.10378556)(632.13021086,856.173556)(632.64000345,856.17356206)
\curveto(633.14606844,856.173556)(633.55724888,856.11401856)(633.87354603,855.99494956)
\curveto(634.18983419,855.8758688)(634.42240232,855.72609492)(634.57125111,855.54562749)
\curveto(634.72008952,855.36514919)(634.82428004,855.13723243)(634.88382299,854.86187651)
\curveto(634.91730729,854.69070162)(634.9340522,854.38185115)(634.93405775,853.93532417)
\lineto(634.93405775,852.59573041)
\curveto(634.9340522,851.66173433)(634.95544847,851.07101129)(634.99824662,850.82355951)
\curveto(635.04103354,850.57610631)(635.12568834,850.33888682)(635.25221127,850.11190033)
\lineto(634.20286283,850.11190033)
\curveto(634.09866749,850.32028137)(634.03168786,850.56401277)(634.00192377,850.84309525)
\closepath
\moveto(633.91819916,853.08691479)
\curveto(633.55352779,852.93806821)(633.00652756,852.81155115)(632.27719681,852.70736323)
\curveto(631.86415292,852.64782319)(631.57204735,852.58084357)(631.40087923,852.50642416)
\curveto(631.22970707,852.43199997)(631.09760837,852.32315808)(631.00458274,852.17989819)
\curveto(630.91155387,852.03663415)(630.86504024,851.87755755)(630.86504173,851.70266791)
\curveto(630.86504024,851.43474784)(630.96643995,851.21148244)(631.16924114,851.03287104)
\curveto(631.37203876,850.85425779)(631.66879569,850.76495163)(632.05951282,850.76495228)
\curveto(632.44650351,850.76495163)(632.79070434,850.84960643)(633.09211634,851.01891693)
\curveto(633.39352092,851.18822562)(633.61492578,851.41986348)(633.75633158,851.71383119)
\curveto(633.86423881,851.94081608)(633.91819462,852.27571419)(633.91819916,852.71852651)
\closepath
}
}
{
\newrgbcolor{curcolor}{0 0 0}
\pscustom[linestyle=none,fillstyle=solid,fillcolor=curcolor]
{
\newpath
\moveto(636.49691825,850.11190033)
\lineto(636.49691825,858.29458551)
\lineto(637.50161356,858.29458551)
\lineto(637.50161356,855.35864253)
\curveto(637.97046915,855.90191643)(638.56212246,856.173556)(639.27657528,856.17356206)
\curveto(639.71566037,856.173556)(640.0970721,856.08704066)(640.42081161,855.91401577)
\curveto(640.74454177,855.74097928)(640.97617962,855.50189925)(641.11572587,855.19677495)
\curveto(641.25526138,854.89164048)(641.32503182,854.44883077)(641.3250374,853.86834448)
\lineto(641.3250374,850.11190033)
\lineto(640.32034208,850.11190033)
\lineto(640.32034208,853.86834448)
\curveto(640.32033751,854.37068788)(640.21149562,854.73628497)(639.99381611,854.96513686)
\curveto(639.77612809,855.19397905)(639.46820789,855.30840257)(639.07005458,855.30840776)
\curveto(638.77236405,855.30840257)(638.49235202,855.23118995)(638.23001766,855.07676968)
\curveto(637.96767833,854.92233948)(637.78069355,854.71302816)(637.66906278,854.44883511)
\curveto(637.55742815,854.18463338)(637.5016118,853.81996655)(637.50161356,853.35483354)
\lineto(637.50161356,850.11190033)
\closepath
}
}
{
\newrgbcolor{curcolor}{0 0 0}
\pscustom[linestyle=none,fillstyle=solid,fillcolor=curcolor]
{
\newpath
\moveto(642.48601976,853.07575151)
\curveto(642.48601938,854.17347011)(642.79114877,854.98652828)(643.40140883,855.51492847)
\curveto(643.91119687,855.95401169)(644.53261891,856.173556)(645.2656768,856.17356206)
\curveto(646.08059236,856.173556)(646.74666748,855.90656779)(647.26390415,855.37259663)
\curveto(647.78113051,854.83861495)(648.03974627,854.10090885)(648.0397522,853.15947612)
\curveto(648.03974627,852.39664961)(647.92532275,851.79662384)(647.6964813,851.35939701)
\curveto(647.46762867,850.92216769)(647.13459111,850.58261822)(646.69736763,850.3407476)
\curveto(646.26013496,850.09887651)(645.78290516,849.97794109)(645.2656768,849.97794095)
\curveto(644.43587056,849.97794109)(643.76514409,850.24399903)(643.25349535,850.77611557)
\curveto(642.74184432,851.30823078)(642.48601938,852.07477532)(642.48601976,853.07575151)
\closepath
\moveto(643.51862328,853.07575151)
\curveto(643.51862187,852.31664618)(643.68421038,851.74824967)(644.0153893,851.37056029)
\curveto(644.3465644,850.9928684)(644.76332649,850.80402308)(645.2656768,850.80402377)
\curveto(645.76429971,850.80402308)(646.17920125,850.99379867)(646.51038266,851.37335111)
\curveto(646.84155527,851.75290104)(647.00714378,852.33153054)(647.00714868,853.10924135)
\curveto(647.00714378,853.84229309)(646.840625,854.39766578)(646.50759184,854.77536108)
\curveto(646.17454988,855.15304706)(645.76057862,855.34189238)(645.2656768,855.34189761)
\curveto(644.76332649,855.34189238)(644.3465644,855.15397733)(644.0153893,854.7781519)
\curveto(643.68421038,854.40231714)(643.51862187,853.83485091)(643.51862328,853.07575151)
\closepath
}
}
{
\newrgbcolor{curcolor}{0 0 0}
\pscustom[linestyle=none,fillstyle=solid,fillcolor=curcolor]
{
\newpath
\moveto(653.28091006,852.02082143)
\lineto(654.31909522,851.89244369)
\curveto(654.15536141,851.28590424)(653.85209257,850.81518635)(653.40928779,850.48028861)
\curveto(652.96647314,850.14539014)(652.40086745,849.97794109)(651.71246904,849.97794095)
\curveto(650.84545182,849.97794109)(650.15798043,850.2449293)(649.65005282,850.77890639)
\curveto(649.14212285,851.31288214)(648.88815846,852.06175151)(648.88815888,853.02551674)
\curveto(648.88815846,854.02276596)(649.14491367,854.79675269)(649.65842528,855.34747925)
\curveto(650.17193452,855.89819534)(650.83800964,856.173556)(651.65665263,856.17356206)
\curveto(652.44924162,856.173556)(653.09671129,855.90377697)(653.59906357,855.36422417)
\curveto(654.1014056,854.82466086)(654.35257918,854.0655585)(654.35258506,853.08691479)
\curveto(654.35257918,853.02737437)(654.35071863,852.93806821)(654.34700342,852.81899604)
\lineto(649.92634403,852.81899604)
\curveto(649.96355348,852.16780258)(650.14774743,851.66917651)(650.47892646,851.32311635)
\curveto(650.81010146,850.97705376)(651.22314246,850.80402308)(651.71805068,850.80402377)
\curveto(652.08643534,850.80402308)(652.40086745,850.90077142)(652.66134795,851.09426908)
\curveto(652.92182006,851.28776478)(653.12834056,851.59661525)(653.28091006,852.02082143)
\closepath
\moveto(649.98216044,853.64507885)
\lineto(653.29207334,853.64507885)
\curveto(653.24741544,854.14370139)(653.12089837,854.51767093)(652.91252178,854.76698862)
\curveto(652.59250359,855.15397733)(652.17760205,855.34747401)(651.66781591,855.34747925)
\curveto(651.20639755,855.34747401)(650.81847391,855.19304878)(650.50404384,854.88420308)
\curveto(650.1896097,854.57534783)(650.01564874,854.16230684)(649.98216044,853.64507885)
\closepath
}
}
{
\newrgbcolor{curcolor}{0 0 0}
\pscustom[linestyle=none,fillstyle=solid,fillcolor=curcolor]
{
\newpath
\moveto(607.70080841,1290.01689911)
\lineto(606.69611309,1290.01689911)
\lineto(606.69611309,1296.41904092)
\curveto(606.45423899,1296.18832694)(606.13701606,1295.95761935)(605.74444336,1295.72691748)
\curveto(605.35186606,1295.49620419)(604.99929278,1295.3231735)(604.68672246,1295.2078249)
\lineto(604.68672246,1296.17903037)
\curveto(605.24860581,1296.4432216)(605.7397897,1296.76323535)(606.16027559,1297.13907256)
\curveto(606.58075605,1297.51489554)(606.87844325,1297.87956236)(607.05333809,1298.23307413)
\lineto(607.70080841,1298.23307413)
\closepath
}
}
{
\newrgbcolor{curcolor}{0 0 0}
\pscustom[linestyle=none,fillstyle=solid,fillcolor=curcolor]
{
\newpath
\moveto(610.64233319,1290.01689911)
\lineto(610.64233319,1298.19958429)
\lineto(613.71223554,1298.19958429)
\curveto(614.33737476,1298.1995761)(614.83879164,1298.11678185)(615.21648769,1297.95120128)
\curveto(615.59417292,1297.78560484)(615.88999957,1297.53071017)(616.10396855,1297.18651651)
\curveto(616.31792493,1296.84230851)(616.42490627,1296.48229305)(616.42491289,1296.10646905)
\curveto(616.42490627,1295.75668049)(616.33001847,1295.42736402)(616.14024921,1295.11851865)
\curveto(615.95046729,1294.80966308)(615.66394335,1294.56035005)(615.28067656,1294.37057881)
\curveto(615.77557606,1294.22545194)(616.15605751,1293.97799945)(616.42212207,1293.6282206)
\curveto(616.68817339,1293.27843453)(616.82120236,1292.86539353)(616.82120937,1292.38909638)
\curveto(616.82120236,1292.00582173)(616.74026865,1291.64952736)(616.578408,1291.3202122)
\curveto(616.41653381,1290.99089443)(616.21652522,1290.73693003)(615.97838164,1290.55831825)
\curveto(615.7402257,1290.37970539)(615.44160822,1290.24481587)(615.08252831,1290.1536493)
\curveto(614.72343785,1290.06248246)(614.28341895,1290.01689911)(613.7624703,1290.01689911)
\closepath
\moveto(611.72517147,1294.76129365)
\lineto(613.49455155,1294.76129365)
\curveto(613.97456848,1294.76128891)(614.31876931,1294.79291817)(614.52715507,1294.85618154)
\curveto(614.80251101,1294.93804068)(615.00996178,1295.07386047)(615.149508,1295.26364131)
\curveto(615.28904353,1295.45341165)(615.35881397,1295.69156142)(615.35881952,1295.97809131)
\curveto(615.35881397,1296.24972492)(615.2936949,1296.48880496)(615.1634621,1296.69533213)
\curveto(615.03321859,1296.90184595)(614.84716409,1297.04324737)(614.60529804,1297.11953682)
\curveto(614.36342239,1297.19581206)(613.94852085,1297.23395324)(613.36059218,1297.23396045)
\lineto(611.72517147,1297.23396045)
\closepath
\moveto(611.72517147,1290.98252294)
\lineto(613.7624703,1290.98252294)
\curveto(614.11224881,1290.98252197)(614.35784075,1290.99554579)(614.49924687,1291.02159442)
\curveto(614.7485552,1291.0662465)(614.95693625,1291.1406683)(615.12439062,1291.24486005)
\curveto(615.29183435,1291.34904934)(615.42951468,1291.50068376)(615.53743203,1291.69976376)
\curveto(615.6453379,1291.8988404)(615.69929371,1292.12861771)(615.6992996,1292.38909638)
\curveto(615.69929371,1292.69422339)(615.62115082,1292.95935106)(615.4648707,1293.18448017)
\curveto(615.30857926,1293.40960295)(615.09182576,1293.56774928)(614.81460956,1293.65891963)
\curveto(614.53738335,1293.75008269)(614.13829644,1293.79566604)(613.61734765,1293.79566982)
\lineto(611.72517147,1293.79566982)
\closepath
}
}
{
\newrgbcolor{curcolor}{0 0 0}
\pscustom[linestyle=none,fillstyle=solid,fillcolor=curcolor]
{
\newpath
\moveto(617.95428204,1292.64585185)
\lineto(618.97572228,1292.7351581)
\curveto(619.02409491,1292.32583548)(619.13665789,1291.9900071)(619.31341154,1291.72767196)
\curveto(619.49016144,1291.46533341)(619.76459183,1291.25323127)(620.13670353,1291.09136493)
\curveto(620.50880984,1290.92949644)(620.92743247,1290.84856273)(621.39257267,1290.84856356)
\curveto(621.80560972,1290.84856273)(622.17027654,1290.90996072)(622.48657424,1291.03275771)
\curveto(622.80286185,1291.15555266)(623.03822079,1291.32393198)(623.19265178,1291.53789618)
\curveto(623.34707127,1291.75185734)(623.42428388,1291.98535574)(623.42428987,1292.23839208)
\curveto(623.42428388,1292.49514507)(623.34986208,1292.71934075)(623.20102424,1292.91097978)
\curveto(623.05217488,1293.10261302)(622.80658294,1293.26355017)(622.46424768,1293.3937917)
\curveto(622.24469834,1293.47937339)(621.75909609,1293.61240236)(621.00743947,1293.792879)
\curveto(620.25577572,1293.97334809)(619.72924148,1294.14358796)(619.42783517,1294.30359912)
\curveto(619.03711873,1294.50825478)(618.74594343,1294.76221918)(618.55430841,1295.06549307)
\curveto(618.36267116,1295.36875686)(618.26685309,1295.70830632)(618.26685392,1296.08414248)
\curveto(618.26685309,1296.49717741)(618.38406743,1296.8832405)(618.61849728,1297.24233292)
\curveto(618.85292477,1297.60141088)(619.19526506,1297.87398072)(619.64551916,1298.06004327)
\curveto(620.09576884,1298.24608973)(620.59625545,1298.33911698)(621.14698049,1298.3391253)
\curveto(621.75351446,1298.33911698)(622.28842115,1298.24143837)(622.75170217,1298.04608917)
\curveto(623.21497257,1297.85072391)(623.57126694,1297.56326971)(623.82058635,1297.18372569)
\curveto(624.06989301,1296.80416734)(624.20385225,1296.37438144)(624.22246448,1295.8943667)
\lineto(623.18427932,1295.81622373)
\curveto(623.12845723,1296.33344945)(622.93961191,1296.7241639)(622.6177428,1296.98836827)
\curveto(622.29586333,1297.25255869)(621.82049408,1297.38465738)(621.19163361,1297.38466475)
\curveto(620.53671801,1297.38465738)(620.05948822,1297.26465223)(619.75994279,1297.02464893)
\curveto(619.46039272,1296.78463162)(619.31061885,1296.49531687)(619.31062072,1296.15670381)
\curveto(619.31061885,1295.86273156)(619.41666991,1295.62086071)(619.62877423,1295.43109053)
\curveto(619.83715309,1295.24130952)(620.38136251,1295.04688257)(621.26140412,1294.84780908)
\curveto(622.14143809,1294.64872593)(622.74518495,1294.47476497)(623.07264651,1294.32592568)
\curveto(623.5489404,1294.10637706)(623.90058341,1293.82822558)(624.12757659,1293.49147041)
\curveto(624.35455639,1293.15470828)(624.46804964,1292.76678465)(624.46805667,1292.32769833)
\curveto(624.46804964,1291.89232849)(624.34339312,1291.48207831)(624.09408674,1291.09694657)
\curveto(623.84476706,1290.71181267)(623.48661214,1290.41226493)(623.01962092,1290.19830243)
\curveto(622.55261854,1289.98433957)(622.02701457,1289.87735823)(621.44280744,1289.87735809)
\curveto(620.70230652,1289.87735823)(620.08181476,1289.98526984)(619.58133029,1290.20109325)
\curveto(619.08084154,1290.41691629)(618.68826654,1290.74158139)(618.40360411,1291.17508954)
\curveto(618.11893976,1291.60859537)(617.96916589,1292.09884899)(617.95428204,1292.64585185)
\closepath
}
}
{
\newrgbcolor{curcolor}{0 0 0}
\pscustom[linestyle=none,fillstyle=solid,fillcolor=curcolor]
{
\newpath
\moveto(625.95835529,1290.01689911)
\lineto(625.95835529,1298.19958429)
\lineto(628.77708381,1298.19958429)
\curveto(629.41338651,1298.1995761)(629.89898876,1298.16050466)(630.23389202,1298.08236983)
\curveto(630.70274421,1297.97445016)(631.10276139,1297.77909293)(631.43394476,1297.49629756)
\curveto(631.86558484,1297.13162326)(632.1883894,1296.66555673)(632.40235941,1296.09809658)
\curveto(632.61631476,1295.53062427)(632.7232961,1294.88222433)(632.72330374,1294.15289482)
\curveto(632.7232961,1293.53146865)(632.65073484,1292.98074732)(632.50561976,1292.50072919)
\curveto(632.36048982,1292.02070609)(632.17443532,1291.62347973)(631.9474557,1291.30904892)
\curveto(631.72046233,1290.99461552)(631.47207957,1290.74716303)(631.20230667,1290.56669071)
\curveto(630.93252152,1290.3862173)(630.60692614,1290.24946724)(630.22551956,1290.15644013)
\curveto(629.84410268,1290.06341273)(629.40594433,1290.01689911)(628.91104319,1290.01689911)
\closepath
\moveto(627.04119357,1290.98252294)
\lineto(628.78824709,1290.98252294)
\curveto(629.32780144,1290.98252197)(629.75107543,1291.03275669)(630.05807034,1291.13322724)
\curveto(630.36505529,1291.23369555)(630.60971696,1291.37509697)(630.79205608,1291.55743193)
\curveto(631.04880558,1291.8141856)(631.24881417,1292.1593167)(631.39208245,1292.59282626)
\curveto(631.5353381,1293.02633068)(631.60696909,1293.55193465)(631.60697562,1294.16963974)
\curveto(631.60696909,1295.0254863)(631.46649794,1295.68318896)(631.18556175,1296.14274971)
\curveto(630.90461334,1296.6022982)(630.56320333,1296.9102184)(630.16133069,1297.06651124)
\curveto(629.87108058,1297.17813689)(629.40408378,1297.23395324)(628.76033889,1297.23396045)
\lineto(627.04119357,1297.23396045)
\closepath
}
}
{
\newrgbcolor{curcolor}{0 0 0}
\pscustom[linestyle=none,fillstyle=solid,fillcolor=curcolor]
{
\newpath
\moveto(609.40664904,1247.19251439)
\lineto(609.40664904,1246.22689056)
\lineto(603.99803926,1246.22689056)
\curveto(603.99059673,1246.46876142)(604.02966818,1246.70132954)(604.11525371,1246.92459564)
\curveto(604.25293358,1247.29298286)(604.47340816,1247.65578914)(604.77667813,1248.01301557)
\curveto(605.07994584,1248.37023843)(605.51810419,1248.78327942)(606.0911545,1249.25213979)
\curveto(606.98049258,1249.98147041)(607.58144862,1250.55916964)(607.89402442,1250.98523921)
\curveto(608.20659175,1251.41129926)(608.36287753,1251.81410726)(608.36288224,1252.19366441)
\curveto(608.36287753,1252.59181507)(608.22054583,1252.92764345)(607.93588673,1253.20115054)
\curveto(607.65121906,1253.47464369)(607.28004033,1253.61139374)(606.82234942,1253.61140113)
\curveto(606.33860455,1253.61139374)(605.95161118,1253.46627123)(605.66136817,1253.17603316)
\curveto(605.37112114,1252.88578119)(605.22413808,1252.48390346)(605.22041856,1251.97039878)
\lineto(604.18781504,1252.07644995)
\curveto(604.25851522,1252.84670974)(604.52457315,1253.4337117)(604.98598965,1253.83745757)
\curveto(605.44740348,1254.24118823)(606.06696497,1254.44305737)(606.84467598,1254.44306558)
\curveto(607.62982279,1254.44305737)(608.25124483,1254.2253736)(608.70894396,1253.79001363)
\curveto(609.16663298,1253.35463853)(609.39548001,1252.81508048)(609.39548576,1252.17133784)
\curveto(609.39548001,1251.84387598)(609.32850039,1251.52200169)(609.19454669,1251.20571401)
\curveto(609.06058191,1250.88941638)(608.83824678,1250.55637882)(608.52754064,1250.20660034)
\curveto(608.21682474,1249.8568139)(607.7005235,1249.37679328)(606.97863536,1248.76653705)
\curveto(606.37581545,1248.26046627)(605.98882208,1247.91719571)(605.81765411,1247.73672436)
\curveto(605.6464818,1247.55624998)(605.50508038,1247.37484684)(605.39344942,1247.19251439)
\closepath
}
}
{
\newrgbcolor{curcolor}{0 0 0}
\pscustom[linestyle=none,fillstyle=solid,fillcolor=curcolor]
{
\newpath
\moveto(610.85229412,1246.22689056)
\lineto(610.85229412,1254.40957574)
\lineto(613.92219647,1254.40957574)
\curveto(614.54733569,1254.40956756)(615.04875258,1254.3267733)(615.42644863,1254.16119273)
\curveto(615.80413385,1253.99559629)(616.09996051,1253.74070162)(616.31392949,1253.39650796)
\curveto(616.52788587,1253.05229997)(616.6348672,1252.69228451)(616.63487382,1252.3164605)
\curveto(616.6348672,1251.96667195)(616.53997941,1251.63735548)(616.35021015,1251.32851011)
\curveto(616.16042822,1251.01965453)(615.87390429,1250.7703415)(615.49063749,1250.58057026)
\curveto(615.98553699,1250.4354434)(616.36601845,1250.18799091)(616.632083,1249.83821206)
\curveto(616.89813432,1249.48842598)(617.03116329,1249.07538499)(617.03117031,1248.59908784)
\curveto(617.03116329,1248.21581319)(616.95022959,1247.85951882)(616.78836894,1247.53020365)
\curveto(616.62649475,1247.20088588)(616.42648616,1246.94692149)(616.18834257,1246.76830971)
\curveto(615.95018664,1246.58969684)(615.65156916,1246.45480733)(615.29248925,1246.36364076)
\curveto(614.93339878,1246.27247392)(614.49337989,1246.22689056)(613.97243124,1246.22689056)
\closepath
\moveto(611.93513241,1250.97128511)
\lineto(613.70451249,1250.97128511)
\curveto(614.18452941,1250.97128036)(614.52873024,1251.00290963)(614.73711601,1251.066173)
\curveto(615.01247195,1251.14803214)(615.21992272,1251.28385193)(615.35946894,1251.47363276)
\curveto(615.49900447,1251.66340311)(615.56877491,1251.90155287)(615.56878046,1252.18808277)
\curveto(615.56877491,1252.45971638)(615.50365583,1252.69879641)(615.37342304,1252.90532359)
\curveto(615.24317953,1253.11183741)(615.05712503,1253.25323883)(614.81525898,1253.32952828)
\curveto(614.57338332,1253.40580352)(614.15848178,1253.44394469)(613.57055311,1253.44395191)
\lineto(611.93513241,1253.44395191)
\closepath
\moveto(611.93513241,1247.19251439)
\lineto(613.97243124,1247.19251439)
\curveto(614.32220975,1247.19251343)(614.56780169,1247.20553724)(614.7092078,1247.23158588)
\curveto(614.95851614,1247.27623795)(615.16689718,1247.35065976)(615.33435156,1247.4548515)
\curveto(615.50179529,1247.5590408)(615.63947562,1247.71067522)(615.74739296,1247.90975522)
\curveto(615.85529884,1248.10883185)(615.90925465,1248.33860916)(615.90926054,1248.59908784)
\curveto(615.90925465,1248.90421485)(615.83111176,1249.16934251)(615.67483163,1249.39447163)
\curveto(615.51854019,1249.61959441)(615.3017867,1249.77774073)(615.0245705,1249.86891108)
\curveto(614.74734428,1249.96007414)(614.34825738,1250.0056575)(613.82730858,1250.00566128)
\lineto(611.93513241,1250.00566128)
\closepath
}
}
{
\newrgbcolor{curcolor}{0 0 0}
\pscustom[linestyle=none,fillstyle=solid,fillcolor=curcolor]
{
\newpath
\moveto(618.16424298,1248.8558433)
\lineto(619.18568322,1248.94514956)
\curveto(619.23405585,1248.53582693)(619.34661883,1248.19999856)(619.52337247,1247.93766342)
\curveto(619.70012238,1247.67532486)(619.97455277,1247.46322273)(620.34666447,1247.30135639)
\curveto(620.71877078,1247.1394879)(621.13739341,1247.05855419)(621.60253361,1247.05855502)
\curveto(622.01557065,1247.05855419)(622.38023748,1247.11995217)(622.69653518,1247.24274916)
\curveto(623.01282278,1247.36554412)(623.24818173,1247.53392344)(623.40261272,1247.74788764)
\curveto(623.5570322,1247.96184879)(623.63424482,1248.19534719)(623.6342508,1248.44838354)
\curveto(623.63424482,1248.70513653)(623.55982302,1248.9293322)(623.41098518,1249.12097124)
\curveto(623.26213582,1249.31260448)(623.01654387,1249.47354162)(622.67420861,1249.60378315)
\curveto(622.45465928,1249.68936484)(621.96905703,1249.82239381)(621.21740041,1250.00287046)
\curveto(620.46573665,1250.18333955)(619.93920241,1250.35357942)(619.63779611,1250.51359057)
\curveto(619.24707967,1250.71824624)(618.95590437,1250.97221064)(618.76426935,1251.27548452)
\curveto(618.5726321,1251.57874831)(618.47681403,1251.91829778)(618.47681485,1252.29413394)
\curveto(618.47681403,1252.70716887)(618.59402836,1253.09323196)(618.82845822,1253.45232437)
\curveto(619.06288571,1253.81140233)(619.40522599,1254.08397218)(619.85548009,1254.27003472)
\curveto(620.30572978,1254.45608118)(620.80621639,1254.54910843)(621.35694142,1254.54911676)
\curveto(621.96347539,1254.54910843)(622.49838209,1254.45142982)(622.96166311,1254.25608062)
\curveto(623.42493351,1254.06071537)(623.78122788,1253.77326116)(624.03054729,1253.39371714)
\curveto(624.27985394,1253.01415879)(624.41381318,1252.58437289)(624.43242542,1252.10435816)
\lineto(623.39424026,1252.02621519)
\curveto(623.33841816,1252.5434409)(623.14957284,1252.93415536)(622.82770373,1253.19835972)
\curveto(622.50582427,1253.46255014)(622.03045501,1253.59464884)(621.40159455,1253.59465621)
\curveto(620.74667895,1253.59464884)(620.26944915,1253.47464369)(619.96990373,1253.23464039)
\curveto(619.67035366,1252.99462307)(619.52057978,1252.70530832)(619.52058165,1252.36669527)
\curveto(619.52057978,1252.07272301)(619.62663085,1251.83085216)(619.83873517,1251.64108198)
\curveto(620.04711402,1251.45130098)(620.59132344,1251.25687402)(621.47136506,1251.05780054)
\curveto(622.35139903,1250.85871739)(622.95514589,1250.68475643)(623.28260744,1250.53591714)
\curveto(623.75890134,1250.31636852)(624.11054435,1250.03821704)(624.33753753,1249.70146186)
\curveto(624.56451733,1249.36469974)(624.67801058,1248.9767761)(624.6780176,1248.53768979)
\curveto(624.67801058,1248.10231994)(624.55335406,1247.69206977)(624.30404768,1247.30693803)
\curveto(624.054728,1246.92180413)(623.69657308,1246.62225638)(623.22958186,1246.40829388)
\curveto(622.76257948,1246.19433103)(622.23697551,1246.08734969)(621.65276838,1246.08734955)
\curveto(620.91226746,1246.08734969)(620.29177569,1246.1952613)(619.79129123,1246.41108471)
\curveto(619.29080247,1246.62690774)(618.89822748,1246.95157285)(618.61356505,1247.385081)
\curveto(618.3289007,1247.81858683)(618.17912683,1248.30884044)(618.16424298,1248.8558433)
\closepath
}
}
{
\newrgbcolor{curcolor}{0 0 0}
\pscustom[linestyle=none,fillstyle=solid,fillcolor=curcolor]
{
\newpath
\moveto(626.16831623,1246.22689056)
\lineto(626.16831623,1254.40957574)
\lineto(628.98704475,1254.40957574)
\curveto(629.62334745,1254.40956756)(630.1089497,1254.37049611)(630.44385296,1254.29236129)
\curveto(630.91270514,1254.18444161)(631.31272232,1253.98908438)(631.64390569,1253.70628902)
\curveto(632.07554578,1253.34161472)(632.39835034,1252.87554819)(632.61232035,1252.30808804)
\curveto(632.8262757,1251.74061573)(632.93325704,1251.09221579)(632.93326468,1250.36288628)
\curveto(632.93325704,1249.7414601)(632.86069578,1249.19073878)(632.7155807,1248.71072065)
\curveto(632.57045076,1248.23069755)(632.38439625,1247.83347119)(632.15741663,1247.51904037)
\curveto(631.93042327,1247.20460697)(631.68204051,1246.95715448)(631.41226761,1246.77668217)
\curveto(631.14248245,1246.59620875)(630.81688708,1246.45945869)(630.4354805,1246.36643158)
\curveto(630.05406362,1246.27340419)(629.61590527,1246.22689056)(629.12100413,1246.22689056)
\closepath
\moveto(627.25115451,1247.19251439)
\lineto(628.99820803,1247.19251439)
\curveto(629.53776238,1247.19251343)(629.96103637,1247.24274814)(630.26803128,1247.34321869)
\curveto(630.57501622,1247.44368701)(630.81967789,1247.58508843)(631.00201702,1247.76742338)
\curveto(631.25876652,1248.02417705)(631.45877511,1248.36930815)(631.60204339,1248.80281772)
\curveto(631.74529904,1249.23632213)(631.81693002,1249.7619261)(631.81693655,1250.3796312)
\curveto(631.81693002,1251.23547776)(631.67645887,1251.89318042)(631.39552269,1252.35274116)
\curveto(631.11457428,1252.81228966)(630.77316427,1253.12020986)(630.37129163,1253.27650269)
\curveto(630.08104152,1253.38812834)(629.61404472,1253.44394469)(628.97029983,1253.44395191)
\lineto(627.25115451,1253.44395191)
\closepath
}
}
{
\newrgbcolor{curcolor}{0 0 0}
\pscustom[linestyle=none,fillstyle=solid,fillcolor=curcolor]
{
\newpath
\moveto(604.11503086,1203.86800356)
\lineto(605.11972618,1204.00196293)
\curveto(605.23507848,1203.43263386)(605.43136598,1203.02238369)(605.70858926,1202.77121117)
\curveto(605.9858084,1202.52003653)(606.32349732,1202.39444974)(606.72165704,1202.39445043)
\curveto(607.19423239,1202.39444974)(607.59331929,1202.5581777)(607.91891896,1202.88563481)
\curveto(608.24451005,1203.21308955)(608.40730774,1203.61868837)(608.40731251,1204.10243246)
\curveto(608.40730774,1204.56384523)(608.25660359,1204.94432669)(607.95519962,1205.24387798)
\curveto(607.65378701,1205.54342219)(607.27051473,1205.69319606)(606.80538165,1205.69320005)
\curveto(606.61560289,1205.69319606)(606.37931367,1205.65598516)(606.09651329,1205.58156723)
\lineto(606.2081461,1206.46346645)
\curveto(606.27512315,1206.45601952)(606.32907895,1206.45229843)(606.37001368,1206.45230317)
\curveto(606.7979363,1206.45229843)(607.18306912,1206.56393113)(607.52541329,1206.78720161)
\curveto(607.86774968,1207.01046193)(608.03891983,1207.35466276)(608.03892423,1207.81980513)
\curveto(608.03891983,1208.18818693)(607.91426331,1208.49331631)(607.66495431,1208.73519419)
\curveto(607.41563724,1208.97705802)(607.09376296,1209.09799345)(606.69933048,1209.09800084)
\curveto(606.30861296,1209.09799345)(605.98301758,1208.97519747)(605.72254336,1208.72961255)
\curveto(605.46206497,1208.48401359)(605.29461592,1208.11562568)(605.22019571,1207.62444771)
\lineto(604.21550039,1207.80306021)
\curveto(604.33829578,1208.47657141)(604.61737754,1208.99845429)(605.05274649,1209.36871041)
\curveto(605.4881126,1209.7389512)(606.0295312,1209.92407543)(606.67700391,1209.92408365)
\curveto(607.12353168,1209.92407543)(607.53471213,1209.82825737)(607.91054649,1209.63662916)
\curveto(608.28637231,1209.44498509)(608.57382652,1209.18357852)(608.77290997,1208.85240865)
\curveto(608.97198315,1208.52122449)(609.07152231,1208.16958148)(609.07152775,1207.79747857)
\curveto(609.07152231,1207.44396892)(608.97663451,1207.12209464)(608.78686407,1206.83185474)
\curveto(608.59708333,1206.54160459)(608.31614103,1206.31089701)(607.94403634,1206.1397313)
\curveto(608.42777373,1206.02809416)(608.80360383,1205.79645631)(609.07152775,1205.44481704)
\curveto(609.33944079,1205.09317029)(609.47340004,1204.6531514)(609.47340587,1204.12475903)
\curveto(609.47340004,1203.41030732)(609.21292373,1202.80469992)(608.69197618,1202.307935)
\curveto(608.17101852,1201.81116888)(607.51238559,1201.56278612)(606.7160754,1201.56278597)
\curveto(605.99790194,1201.56278612)(605.40159726,1201.7767488)(604.92715957,1202.20467465)
\curveto(604.4527193,1202.63259951)(604.18201,1203.18704192)(604.11503086,1203.86800356)
\closepath
}
}
{
\newrgbcolor{curcolor}{0 0 0}
\pscustom[linestyle=none,fillstyle=solid,fillcolor=curcolor]
{
\newpath
\moveto(610.83532635,1201.70790863)
\lineto(610.83532635,1209.89059381)
\lineto(613.9052287,1209.89059381)
\curveto(614.53036792,1209.89058562)(615.0317848,1209.80779137)(615.40948085,1209.6422108)
\curveto(615.78716608,1209.47661436)(616.08299274,1209.22171969)(616.29696171,1208.87752603)
\curveto(616.51091809,1208.53331803)(616.61789943,1208.17330257)(616.61790605,1207.79747857)
\curveto(616.61789943,1207.44769001)(616.52301163,1207.11837355)(616.33324238,1206.80952817)
\curveto(616.14346045,1206.5006726)(615.85693652,1206.25135957)(615.47366972,1206.06158833)
\curveto(615.96856922,1205.91646146)(616.34905068,1205.66900898)(616.61511523,1205.31923012)
\curveto(616.88116655,1204.96944405)(617.01419552,1204.55640305)(617.01420254,1204.0801059)
\curveto(617.01419552,1203.69683126)(616.93326181,1203.34053688)(616.77140117,1203.01122172)
\curveto(616.60952698,1202.68190395)(616.40951839,1202.42793955)(616.1713748,1202.24932777)
\curveto(615.93321886,1202.07071491)(615.63460139,1201.9358254)(615.27552148,1201.84465883)
\curveto(614.91643101,1201.75349198)(614.47641211,1201.70790863)(613.95546347,1201.70790863)
\closepath
\moveto(611.91816463,1206.45230317)
\lineto(613.68754472,1206.45230317)
\curveto(614.16756164,1206.45229843)(614.51176247,1206.48392769)(614.72014823,1206.54719106)
\curveto(614.99550417,1206.62905021)(615.20295494,1206.76486999)(615.34250116,1206.95465083)
\curveto(615.4820367,1207.14442118)(615.55180714,1207.38257094)(615.55181269,1207.66910083)
\curveto(615.55180714,1207.94073444)(615.48668806,1208.17981448)(615.35645527,1208.38634165)
\curveto(615.22621176,1208.59285547)(615.04015725,1208.73425689)(614.7982912,1208.81054634)
\curveto(614.55641555,1208.88682159)(614.14151401,1208.92496276)(613.55358534,1208.92496998)
\lineto(611.91816463,1208.92496998)
\closepath
\moveto(611.91816463,1202.67353246)
\lineto(613.95546347,1202.67353246)
\curveto(614.30524197,1202.6735315)(614.55083391,1202.68655531)(614.69224003,1202.71260395)
\curveto(614.94154837,1202.75725602)(615.14992941,1202.83167782)(615.31738378,1202.93586957)
\curveto(615.48482751,1203.04005886)(615.62250785,1203.19169328)(615.73042519,1203.39077328)
\curveto(615.83833107,1203.58984992)(615.89228687,1203.81962723)(615.89229277,1204.0801059)
\curveto(615.89228687,1204.38523291)(615.81414398,1204.65036058)(615.65786386,1204.87548969)
\curveto(615.50157242,1205.10061247)(615.28481892,1205.2587588)(615.00760273,1205.34992915)
\curveto(614.73037651,1205.44109221)(614.3312896,1205.48667556)(613.81034081,1205.48667934)
\lineto(611.91816463,1205.48667934)
\closepath
}
}
{
\newrgbcolor{curcolor}{0 0 0}
\pscustom[linestyle=none,fillstyle=solid,fillcolor=curcolor]
{
\newpath
\moveto(618.14727521,1204.33686137)
\lineto(619.16871544,1204.42616762)
\curveto(619.21708808,1204.016845)(619.32965105,1203.68101662)(619.5064047,1203.41868149)
\curveto(619.68315461,1203.15634293)(619.957585,1202.9442408)(620.3296967,1202.78237445)
\curveto(620.701803,1202.62050596)(621.12042563,1202.53957225)(621.58556584,1202.53957309)
\curveto(621.99860288,1202.53957225)(622.36326971,1202.60097024)(622.6795674,1202.72376723)
\curveto(622.99585501,1202.84656218)(623.23121396,1203.01494151)(623.38564494,1203.2289057)
\curveto(623.54006443,1203.44286686)(623.61727705,1203.67636526)(623.61728303,1203.9294016)
\curveto(623.61727705,1204.1861546)(623.54285525,1204.41035027)(623.39401741,1204.6019893)
\curveto(623.24516804,1204.79362254)(622.9995761,1204.95455969)(622.65724084,1205.08480122)
\curveto(622.43769151,1205.17038291)(621.95208926,1205.30341188)(621.20043263,1205.48388852)
\curveto(620.44876888,1205.66435761)(619.92223464,1205.83459748)(619.62082833,1205.99460864)
\curveto(619.23011189,1206.19926431)(618.9389366,1206.4532287)(618.74730157,1206.75650259)
\curveto(618.55566432,1207.05976638)(618.45984626,1207.39931584)(618.45984708,1207.775152)
\curveto(618.45984626,1208.18818693)(618.57706059,1208.57425002)(618.81149044,1208.93334244)
\curveto(619.04591794,1209.2924204)(619.38825822,1209.56499025)(619.83851232,1209.75105279)
\curveto(620.28876201,1209.93709925)(620.78924862,1210.0301265)(621.33997365,1210.03013482)
\curveto(621.94650762,1210.0301265)(622.48141431,1209.93244789)(622.94469533,1209.73709869)
\curveto(623.40796573,1209.54173343)(623.7642601,1209.25427923)(624.01357952,1208.87473521)
\curveto(624.26288617,1208.49517686)(624.39684541,1208.06539096)(624.41545764,1207.58537622)
\lineto(623.37727248,1207.50723325)
\curveto(623.32145039,1208.02445897)(623.13260507,1208.41517342)(622.81073596,1208.67937779)
\curveto(622.48885649,1208.94356821)(622.01348724,1209.07566691)(621.38462678,1209.07567427)
\curveto(620.72971118,1209.07566691)(620.25248138,1208.95566175)(619.95293595,1208.71565845)
\curveto(619.65338589,1208.47564114)(619.50361201,1208.18632639)(619.50361388,1207.84771333)
\curveto(619.50361201,1207.55374108)(619.60966308,1207.31187023)(619.8217674,1207.12210005)
\curveto(620.03014625,1206.93231904)(620.57435567,1206.73789209)(621.45439728,1206.5388186)
\curveto(622.33443126,1206.33973546)(622.93817812,1206.1657745)(623.26563967,1206.0169352)
\curveto(623.74193356,1205.79738658)(624.09357657,1205.5192351)(624.32056975,1205.18247993)
\curveto(624.54754956,1204.84571781)(624.6610428,1204.45779417)(624.66104983,1204.01870785)
\curveto(624.6610428,1203.58333801)(624.53638629,1203.17308783)(624.28707991,1202.78795609)
\curveto(624.03776022,1202.4028222)(623.67960531,1202.10327445)(623.21261409,1201.88931195)
\curveto(622.74561171,1201.67534909)(622.22000774,1201.56836775)(621.6358006,1201.56836761)
\curveto(620.89529969,1201.56836775)(620.27480792,1201.67627937)(619.77432345,1201.89210277)
\curveto(619.2738347,1202.10792581)(618.8812597,1202.43259092)(618.59659728,1202.86609906)
\curveto(618.31193293,1203.29960489)(618.16215905,1203.78985851)(618.14727521,1204.33686137)
\closepath
}
}
{
\newrgbcolor{curcolor}{0 0 0}
\pscustom[linestyle=none,fillstyle=solid,fillcolor=curcolor]
{
\newpath
\moveto(626.15134845,1201.70790863)
\lineto(626.15134845,1209.89059381)
\lineto(628.97007698,1209.89059381)
\curveto(629.60637967,1209.89058562)(630.09198192,1209.85151418)(630.42688518,1209.77337935)
\curveto(630.89573737,1209.66545968)(631.29575455,1209.47010245)(631.62693792,1209.18730709)
\curveto(632.05857801,1208.82263278)(632.38138257,1208.35656626)(632.59535257,1207.78910611)
\curveto(632.80930792,1207.22163379)(632.91628926,1206.57323386)(632.91629691,1205.84390434)
\curveto(632.91628926,1205.22247817)(632.84372801,1204.67175685)(632.69861292,1204.19173871)
\curveto(632.55348298,1203.71171562)(632.36742848,1203.31448925)(632.14044886,1203.00005844)
\curveto(631.9134555,1202.68562504)(631.66507274,1202.43817255)(631.39529983,1202.25770023)
\curveto(631.12551468,1202.07722682)(630.7999193,1201.94047676)(630.41851272,1201.84744965)
\curveto(630.03709585,1201.75442226)(629.59893749,1201.70790863)(629.10403635,1201.70790863)
\closepath
\moveto(627.23418674,1202.67353246)
\lineto(628.98124026,1202.67353246)
\curveto(629.5207946,1202.6735315)(629.94406859,1202.72376621)(630.2510635,1202.82423676)
\curveto(630.55804845,1202.92470507)(630.80271012,1203.06610649)(630.98504925,1203.24844145)
\curveto(631.24179874,1203.50519512)(631.44180733,1203.85032622)(631.58507562,1204.28383579)
\curveto(631.72833127,1204.7173402)(631.79996225,1205.24294417)(631.79996878,1205.86064927)
\curveto(631.79996225,1206.71649582)(631.6594911,1207.37419849)(631.37855491,1207.83375923)
\curveto(631.09760651,1208.29330773)(630.75619649,1208.60122793)(630.35432386,1208.75752076)
\curveto(630.06407375,1208.86914641)(629.59707695,1208.92496276)(628.95333206,1208.92496998)
\lineto(627.23418674,1208.92496998)
\closepath
}
}
{
\newrgbcolor{curcolor}{0 0 0}
\pscustom[linestyle=none,fillstyle=solid,fillcolor=curcolor]
{
\newpath
\moveto(462.29043898,1254.40957574)
\lineto(463.37327726,1254.40957574)
\lineto(463.37327726,1249.68192612)
\curveto(463.37326993,1248.85956177)(463.28024268,1248.20651046)(463.09419523,1247.72277026)
\curveto(462.90813367,1247.23902705)(462.5723053,1246.84552178)(462.0867091,1246.54225326)
\curveto(461.6011008,1246.23898411)(460.96386413,1246.08734969)(460.17499718,1246.08734955)
\curveto(459.40844849,1246.08734969)(458.78144482,1246.21944838)(458.29398428,1246.48364603)
\curveto(457.80651923,1246.74784317)(457.45859731,1247.13018517)(457.25021748,1247.63067318)
\curveto(457.04183523,1248.13115839)(456.93764471,1248.81490869)(456.93764561,1249.68192612)
\lineto(456.93764561,1254.40957574)
\lineto(458.02048389,1254.40957574)
\lineto(458.02048389,1249.68750776)
\curveto(458.02048191,1248.9767761)(458.08653126,1248.45303268)(458.21863213,1248.11627592)
\curveto(458.35072865,1247.77951538)(458.57771514,1247.51996935)(458.89959229,1247.33763705)
\curveto(459.22146372,1247.15530253)(459.61496899,1247.06413582)(460.08010929,1247.06413666)
\curveto(460.87641851,1247.06413582)(461.44388474,1247.24460869)(461.78250968,1247.6055558)
\curveto(462.12112313,1247.96650016)(462.29043273,1248.66048345)(462.29043898,1249.68750776)
\closepath
}
}
{
\newrgbcolor{curcolor}{0 0 0}
\pscustom[linestyle=none,fillstyle=solid,fillcolor=curcolor]
{
\newpath
\moveto(465.17056542,1246.22689056)
\lineto(465.17056542,1254.40957574)
\lineto(466.28131191,1254.40957574)
\lineto(470.5791752,1247.98510737)
\lineto(470.5791752,1254.40957574)
\lineto(471.61736036,1254.40957574)
\lineto(471.61736036,1246.22689056)
\lineto(470.50661387,1246.22689056)
\lineto(466.20875058,1252.65694058)
\lineto(466.20875058,1246.22689056)
\closepath
}
}
{
\newrgbcolor{curcolor}{0 0 0}
\pscustom[linestyle=none,fillstyle=solid,fillcolor=curcolor]
{
\newpath
\moveto(473.62675086,1246.22689056)
\lineto(473.62675086,1254.40957574)
\lineto(474.70958915,1254.40957574)
\lineto(474.70958915,1246.22689056)
\closepath
}
}
{
\newrgbcolor{curcolor}{0 0 0}
\pscustom[linestyle=none,fillstyle=solid,fillcolor=curcolor]
{
\newpath
\moveto(475.78126428,1246.22689056)
\lineto(478.94605452,1250.49126401)
\lineto(476.1552342,1254.40957574)
\lineto(477.44459319,1254.40957574)
\lineto(478.9293096,1252.31087886)
\curveto(479.23815687,1251.87550524)(479.45770118,1251.54060714)(479.58794319,1251.30618355)
\curveto(479.77027275,1251.60386567)(479.98609597,1251.91457669)(480.23541351,1252.23831753)
\lineto(481.88199749,1254.40957574)
\lineto(483.05972367,1254.40957574)
\lineto(480.18517874,1250.55266206)
\lineto(483.28298929,1246.22689056)
\lineto(481.94339554,1246.22689056)
\lineto(479.88377015,1249.14608862)
\curveto(479.7684122,1249.31353475)(479.64933732,1249.49586816)(479.52654514,1249.6930894)
\curveto(479.34420794,1249.39539873)(479.21396979,1249.19073878)(479.1358303,1249.07910893)
\lineto(477.08178654,1246.22689056)
\closepath
}
}
{
\newrgbcolor{curcolor}{0 0 0}
\pscustom[linestyle=none,fillstyle=solid,fillcolor=curcolor]
{
\newpath
\moveto(483.36671448,1246.08734955)
\lineto(485.73891175,1254.54911676)
\lineto(486.542668,1254.54911676)
\lineto(484.17605237,1246.08734955)
\closepath
}
}
{
\newrgbcolor{curcolor}{0 0 0}
\pscustom[linestyle=none,fillstyle=solid,fillcolor=curcolor]
{
\newpath
\moveto(487.01710758,1248.38698549)
\lineto(488.0218029,1248.52094487)
\curveto(488.1371552,1247.9516158)(488.3334427,1247.54136562)(488.61066598,1247.29019311)
\curveto(488.88788512,1247.03901847)(489.22557404,1246.91343168)(489.62373376,1246.91343236)
\curveto(490.09630911,1246.91343168)(490.49539601,1247.07715964)(490.82099568,1247.40461674)
\curveto(491.14658677,1247.73207148)(491.30938446,1248.1376703)(491.30938923,1248.6214144)
\curveto(491.30938446,1249.08282717)(491.15868031,1249.46330862)(490.85727634,1249.76285991)
\curveto(490.55586373,1250.06240412)(490.17259145,1250.21217799)(489.70745837,1250.21218198)
\curveto(489.51767961,1250.21217799)(489.28139039,1250.17496709)(488.99859001,1250.10054917)
\lineto(489.11022282,1250.98244839)
\curveto(489.17719987,1250.97500145)(489.23115567,1250.97128036)(489.2720904,1250.97128511)
\curveto(489.70001302,1250.97128036)(490.08514584,1251.08291306)(490.42749001,1251.30618355)
\curveto(490.7698264,1251.52944387)(490.94099655,1251.8736447)(490.94100095,1252.33878706)
\curveto(490.94099655,1252.70716887)(490.81634003,1253.01229825)(490.56703103,1253.25417613)
\curveto(490.31771397,1253.49603995)(489.99583968,1253.61697538)(489.6014072,1253.61698277)
\curveto(489.21068968,1253.61697538)(488.8850943,1253.49417941)(488.62462009,1253.24859449)
\curveto(488.3641417,1253.00299552)(488.19669264,1252.63460761)(488.12227243,1252.14342964)
\lineto(487.11757711,1252.32204214)
\curveto(487.2403725,1252.99555334)(487.51945426,1253.51743622)(487.95482321,1253.88769234)
\curveto(488.39018933,1254.25793314)(488.93160793,1254.44305737)(489.57908063,1254.44306558)
\curveto(490.0256084,1254.44305737)(490.43678885,1254.3472393)(490.81262322,1254.15561109)
\curveto(491.18844903,1253.96396703)(491.47590324,1253.70256045)(491.67498669,1253.37139058)
\curveto(491.87405987,1253.04020642)(491.97359903,1252.68856341)(491.97360447,1252.3164605)
\curveto(491.97359903,1251.96295086)(491.87871124,1251.64107657)(491.6889408,1251.35083667)
\curveto(491.49916005,1251.06058652)(491.21821775,1250.82987894)(490.84611306,1250.65871323)
\curveto(491.32985046,1250.5470761)(491.70568055,1250.31543824)(491.97360447,1249.96379897)
\curveto(492.24151751,1249.61215223)(492.37547676,1249.17213333)(492.37548259,1248.64374096)
\curveto(492.37547676,1247.92928926)(492.11500045,1247.32368185)(491.59405291,1246.82691693)
\curveto(491.07309524,1246.33015081)(490.41446231,1246.08176805)(489.61815212,1246.08176791)
\curveto(488.89997866,1246.08176805)(488.30367398,1246.29573073)(487.82923629,1246.72365658)
\curveto(487.35479602,1247.15158144)(487.08408672,1247.70602385)(487.01710758,1248.38698549)
\closepath
}
}
{
\newrgbcolor{curcolor}{0 0 0}
\pscustom[linestyle=none,fillstyle=solid,fillcolor=curcolor]
{
\newpath
\moveto(498.65482561,1247.19251439)
\lineto(498.65482561,1246.22689056)
\lineto(493.24621583,1246.22689056)
\curveto(493.23877331,1246.46876142)(493.27784475,1246.70132954)(493.36343029,1246.92459564)
\curveto(493.50111015,1247.29298286)(493.72158474,1247.65578914)(494.0248547,1248.01301557)
\curveto(494.32812242,1248.37023843)(494.76628077,1248.78327942)(495.33933107,1249.25213979)
\curveto(496.22866915,1249.98147041)(496.82962519,1250.55916964)(497.142201,1250.98523921)
\curveto(497.45476832,1251.41129926)(497.6110541,1251.81410726)(497.61105881,1252.19366441)
\curveto(497.6110541,1252.59181507)(497.46872241,1252.92764345)(497.1840633,1253.20115054)
\curveto(496.89939563,1253.47464369)(496.5282169,1253.61139374)(496.070526,1253.61140113)
\curveto(495.58678112,1253.61139374)(495.19978776,1253.46627123)(494.90954474,1253.17603316)
\curveto(494.61929771,1252.88578119)(494.47231465,1252.48390346)(494.46859513,1251.97039878)
\lineto(493.43599162,1252.07644995)
\curveto(493.50669179,1252.84670974)(493.77274973,1253.4337117)(494.23416623,1253.83745757)
\curveto(494.69558006,1254.24118823)(495.31514155,1254.44305737)(496.09285256,1254.44306558)
\curveto(496.87799936,1254.44305737)(497.4994214,1254.2253736)(497.95712053,1253.79001363)
\curveto(498.41480955,1253.35463853)(498.64365659,1252.81508048)(498.64366233,1252.17133784)
\curveto(498.64365659,1251.84387598)(498.57667697,1251.52200169)(498.44272327,1251.20571401)
\curveto(498.30875848,1250.88941638)(498.08642335,1250.55637882)(497.77571721,1250.20660034)
\curveto(497.46500132,1249.8568139)(496.94870007,1249.37679328)(496.22681193,1248.76653705)
\curveto(495.62399202,1248.26046627)(495.23699866,1247.91719571)(495.06583068,1247.73672436)
\curveto(494.89465837,1247.55624998)(494.75325695,1247.37484684)(494.64162599,1247.19251439)
\closepath
}
}
{
\newrgbcolor{curcolor}{0 0 0}
\pscustom[linestyle=none,fillstyle=solid,fillcolor=curcolor]
{
\newpath
\moveto(502.4838322,1246.22689056)
\lineto(499.31346032,1254.40957574)
\lineto(500.48560486,1254.40957574)
\lineto(502.61220994,1248.46512846)
\curveto(502.78337673,1247.9888267)(502.9266387,1247.54229589)(503.04199627,1247.12553471)
\curveto(503.16850955,1247.57206461)(503.31549261,1248.01859542)(503.48294588,1248.46512846)
\lineto(505.69327557,1254.40957574)
\lineto(506.79844042,1254.40957574)
\lineto(503.59457869,1246.22689056)
\closepath
}
}
{
\newrgbcolor{curcolor}{0 0 0}
\pscustom[linestyle=none,fillstyle=solid,fillcolor=curcolor]
{
\newpath
\moveto(755.17666857,1202.67353246)
\lineto(755.17666857,1201.70790863)
\lineto(749.76805879,1201.70790863)
\curveto(749.76061626,1201.94977948)(749.79968771,1202.18234761)(749.88527324,1202.40561371)
\curveto(750.02295311,1202.77400093)(750.2434277,1203.13680721)(750.54669766,1203.49403363)
\curveto(750.84996537,1203.85125649)(751.28812372,1204.26429749)(751.86117403,1204.73315786)
\curveto(752.75051211,1205.46248848)(753.35146815,1206.04018771)(753.66404396,1206.46625727)
\curveto(753.97661128,1206.89231733)(754.13289706,1207.29512532)(754.13290177,1207.67468247)
\curveto(754.13289706,1208.07283314)(753.99056536,1208.40866152)(753.70590626,1208.68216861)
\curveto(753.42123859,1208.95566175)(753.05005986,1209.09241181)(752.59236895,1209.0924192)
\curveto(752.10862408,1209.09241181)(751.72163071,1208.9472893)(751.4313877,1208.65705123)
\curveto(751.14114067,1208.36679925)(750.99415761,1207.96492153)(750.99043809,1207.45141685)
\lineto(749.95783457,1207.55746802)
\curveto(750.02853475,1208.32772781)(750.29459268,1208.91472976)(750.75600918,1209.31847564)
\curveto(751.21742301,1209.7222063)(751.8369845,1209.92407543)(752.61469552,1209.92408365)
\curveto(753.39984232,1209.92407543)(754.02126436,1209.70639167)(754.47896349,1209.2710317)
\curveto(754.93665251,1208.8356566)(755.16549954,1208.29609854)(755.16550529,1207.65235591)
\curveto(755.16549954,1207.32489404)(755.09851992,1207.00301975)(754.96456622,1206.68673208)
\curveto(754.83060144,1206.37043445)(754.60826631,1206.03739689)(754.29756017,1205.68761841)
\curveto(753.98684427,1205.33783196)(753.47054303,1204.85781135)(752.74865489,1204.24755512)
\curveto(752.14583498,1203.74148434)(751.75884161,1203.39821378)(751.58767364,1203.21774242)
\curveto(751.41650133,1203.03726805)(751.27509991,1202.85586491)(751.16346895,1202.67353246)
\closepath
}
}
{
\newrgbcolor{curcolor}{0 0 0}
\pscustom[linestyle=none,fillstyle=solid,fillcolor=curcolor]
{
\newpath
\moveto(756.82325272,1201.70790863)
\lineto(756.82325272,1202.85214496)
\lineto(757.96748905,1202.85214496)
\lineto(757.96748905,1201.70790863)
\closepath
}
}
{
\newrgbcolor{curcolor}{0 0 0}
\pscustom[linestyle=none,fillstyle=solid,fillcolor=curcolor]
{
\newpath
\moveto(759.49685871,1208.8189188)
\lineto(759.49685871,1209.78454263)
\lineto(764.79383568,1209.78454263)
\lineto(764.79383568,1209.00311295)
\curveto(764.27287723,1208.44866323)(763.75657599,1207.71188741)(763.2449304,1206.79278325)
\curveto(762.73327623,1205.87366893)(762.33791041,1204.92851206)(762.05883176,1203.95730981)
\curveto(761.8578898,1203.27262699)(761.72951219,1202.52282735)(761.67369856,1201.70790863)
\lineto(760.64109504,1201.70790863)
\curveto(760.65225662,1202.35165721)(760.77877369,1203.12936503)(761.0206466,1204.04103442)
\curveto(761.26251539,1204.95269914)(761.60950704,1205.83180667)(762.06162258,1206.67835962)
\curveto(762.51373192,1207.52490263)(762.9946828,1208.23842165)(763.50447669,1208.8189188)
\closepath
}
}
{
\newrgbcolor{curcolor}{0 0 0}
\pscustom[linestyle=none,fillstyle=solid,fillcolor=curcolor]
{
\newpath
\moveto(765.94365381,1203.60008481)
\lineto(766.90927764,1203.68939106)
\curveto(766.99114003,1203.23541609)(767.14742581,1202.90609962)(767.37813545,1202.70144066)
\curveto(767.60884097,1202.49677972)(767.90466763,1202.39444974)(768.26561631,1202.39445043)
\curveto(768.57446384,1202.39444974)(768.84517314,1202.46515045)(769.07774503,1202.60655277)
\curveto(769.31030939,1202.7479533)(769.50101526,1202.93679862)(769.64986319,1203.1730893)
\curveto(769.79870246,1203.40937705)(769.92335898,1203.72846052)(770.02383311,1204.13034067)
\curveto(770.12429784,1204.53221597)(770.17453255,1204.94153587)(770.17453741,1205.35830161)
\curveto(770.17453255,1205.40295104)(770.17267201,1205.46993066)(770.16895577,1205.55924067)
\curveto(769.96801206,1205.23922308)(769.69358167,1204.97967705)(769.34566378,1204.7806018)
\curveto(768.99773783,1204.58152041)(768.62097746,1204.48198125)(768.21538155,1204.48198403)
\curveto(767.53814026,1204.48198125)(766.9650924,1204.7275732)(766.49623623,1205.21876059)
\curveto(766.02737771,1205.70994097)(765.79294904,1206.35741063)(765.79294951,1207.16117153)
\curveto(765.79294904,1207.99096916)(766.03761071,1208.65890482)(766.52693525,1209.16498052)
\curveto(767.01625739,1209.67104131)(767.62930697,1209.92407543)(768.36608584,1209.92408365)
\curveto(768.89819867,1209.92407543)(769.38473119,1209.78081347)(769.82568487,1209.49429732)
\curveto(770.26662953,1209.2077656)(770.60152764,1208.79937597)(770.83038019,1208.2691272)
\curveto(771.05922171,1207.73886531)(771.17364523,1206.97139049)(771.17365109,1205.96670044)
\curveto(771.17364523,1204.92106988)(771.06015198,1204.08847598)(770.83317101,1203.46891625)
\curveto(770.606179,1202.849353)(770.26849008,1202.37770484)(769.82010323,1202.05397035)
\curveto(769.37170738,1201.73023517)(768.84610341,1201.56836775)(768.24328975,1201.56836761)
\curveto(767.60325934,1201.56836775)(767.08044619,1201.7460498)(766.67484873,1202.1014143)
\curveto(766.26924856,1202.456778)(766.02551716,1202.95633434)(765.94365381,1203.60008481)
\closepath
\moveto(770.05732296,1207.2114063)
\curveto(770.05731822,1207.78816975)(769.90382325,1208.24586383)(769.5968376,1208.5844899)
\curveto(769.2898434,1208.92310221)(768.92052521,1209.09241181)(768.48888194,1209.0924192)
\curveto(768.04234796,1209.09241181)(767.65349405,1208.9100784)(767.32231904,1208.54541841)
\curveto(766.99114003,1208.18074475)(766.82555152,1207.70816632)(766.82555303,1207.12768169)
\curveto(766.82555152,1206.60672367)(766.98276757,1206.18344967)(767.29720166,1205.85785845)
\curveto(767.61163179,1205.53225892)(767.99955543,1205.36946123)(768.46097373,1205.36946489)
\curveto(768.92610685,1205.36946123)(769.30844885,1205.53225892)(769.60800089,1205.85785845)
\curveto(769.90754434,1206.18344967)(770.05731822,1206.63463184)(770.05732296,1207.2114063)
\closepath
}
}
{
\newrgbcolor{curcolor}{0 0 0}
\pscustom[linestyle=none,fillstyle=solid,fillcolor=curcolor]
{
\newpath
\moveto(772.51882569,1201.70790863)
\lineto(772.51882569,1209.89059381)
\lineto(775.58872804,1209.89059381)
\curveto(776.21386726,1209.89058562)(776.71528414,1209.80779137)(777.09298019,1209.6422108)
\curveto(777.47066542,1209.47661436)(777.76649207,1209.22171969)(777.98046105,1208.87752603)
\curveto(778.19441743,1208.53331803)(778.30139877,1208.17330257)(778.30140539,1207.79747857)
\curveto(778.30139877,1207.44769001)(778.20651097,1207.11837355)(778.01674171,1206.80952817)
\curveto(777.82695979,1206.5006726)(777.54043585,1206.25135957)(777.15716906,1206.06158833)
\curveto(777.65206856,1205.91646146)(778.03255001,1205.66900898)(778.29861457,1205.31923012)
\curveto(778.56466589,1204.96944405)(778.69769486,1204.55640305)(778.69770187,1204.0801059)
\curveto(778.69769486,1203.69683126)(778.61676115,1203.34053688)(778.4549005,1203.01122172)
\curveto(778.29302631,1202.68190395)(778.09301773,1202.42793955)(777.85487414,1202.24932777)
\curveto(777.6167182,1202.07071491)(777.31810072,1201.9358254)(776.95902081,1201.84465883)
\curveto(776.59993035,1201.75349198)(776.15991145,1201.70790863)(775.6389628,1201.70790863)
\closepath
\moveto(773.60166397,1206.45230317)
\lineto(775.37104405,1206.45230317)
\curveto(775.85106098,1206.45229843)(776.19526181,1206.48392769)(776.40364757,1206.54719106)
\curveto(776.67900351,1206.62905021)(776.88645428,1206.76486999)(777.0260005,1206.95465083)
\curveto(777.16553603,1207.14442118)(777.23530647,1207.38257094)(777.23531203,1207.66910083)
\curveto(777.23530647,1207.94073444)(777.1701874,1208.17981448)(777.0399546,1208.38634165)
\curveto(776.90971109,1208.59285547)(776.72365659,1208.73425689)(776.48179054,1208.81054634)
\curveto(776.23991489,1208.88682159)(775.82501335,1208.92496276)(775.23708468,1208.92496998)
\lineto(773.60166397,1208.92496998)
\closepath
\moveto(773.60166397,1202.67353246)
\lineto(775.6389628,1202.67353246)
\curveto(775.98874131,1202.6735315)(776.23433325,1202.68655531)(776.37573937,1202.71260395)
\curveto(776.62504771,1202.75725602)(776.83342875,1202.83167782)(777.00088312,1202.93586957)
\curveto(777.16832685,1203.04005886)(777.30600718,1203.19169328)(777.41392453,1203.39077328)
\curveto(777.5218304,1203.58984992)(777.57578621,1203.81962723)(777.5757921,1204.0801059)
\curveto(777.57578621,1204.38523291)(777.49764332,1204.65036058)(777.3413632,1204.87548969)
\curveto(777.18507176,1205.10061247)(776.96831826,1205.2587588)(776.69110206,1205.34992915)
\curveto(776.41387585,1205.44109221)(776.01478894,1205.48667556)(775.49384015,1205.48667934)
\lineto(773.60166397,1205.48667934)
\closepath
}
}
{
\newrgbcolor{curcolor}{0 0 0}
\pscustom[linestyle=none,fillstyle=solid,fillcolor=curcolor]
{
\newpath
\moveto(779.8307755,1204.33686137)
\lineto(780.85221573,1204.42616762)
\curveto(780.90058837,1204.016845)(781.01315134,1203.68101662)(781.18990499,1203.41868149)
\curveto(781.3666549,1203.15634293)(781.64108529,1202.9442408)(782.01319699,1202.78237445)
\curveto(782.38530329,1202.62050596)(782.80392592,1202.53957225)(783.26906613,1202.53957309)
\curveto(783.68210317,1202.53957225)(784.04676999,1202.60097024)(784.36306769,1202.72376723)
\curveto(784.6793553,1202.84656218)(784.91471425,1203.01494151)(785.06914523,1203.2289057)
\curveto(785.22356472,1203.44286686)(785.30077734,1203.67636526)(785.30078332,1203.9294016)
\curveto(785.30077734,1204.1861546)(785.22635554,1204.41035027)(785.0775177,1204.6019893)
\curveto(784.92866833,1204.79362254)(784.68307639,1204.95455969)(784.34074113,1205.08480122)
\curveto(784.1211918,1205.17038291)(783.63558955,1205.30341188)(782.88393292,1205.48388852)
\curveto(782.13226917,1205.66435761)(781.60573493,1205.83459748)(781.30432862,1205.99460864)
\curveto(780.91361218,1206.19926431)(780.62243689,1206.4532287)(780.43080186,1206.75650259)
\curveto(780.23916461,1207.05976638)(780.14334655,1207.39931584)(780.14334737,1207.775152)
\curveto(780.14334655,1208.18818693)(780.26056088,1208.57425002)(780.49499073,1208.93334244)
\curveto(780.72941823,1209.2924204)(781.07175851,1209.56499025)(781.52201261,1209.75105279)
\curveto(781.9722623,1209.93709925)(782.47274891,1210.0301265)(783.02347394,1210.03013482)
\curveto(783.63000791,1210.0301265)(784.1649146,1209.93244789)(784.62819562,1209.73709869)
\curveto(785.09146602,1209.54173343)(785.44776039,1209.25427923)(785.69707981,1208.87473521)
\curveto(785.94638646,1208.49517686)(786.0803457,1208.06539096)(786.09895793,1207.58537622)
\lineto(785.06077277,1207.50723325)
\curveto(785.00495068,1208.02445897)(784.81610536,1208.41517342)(784.49423625,1208.67937779)
\curveto(784.17235678,1208.94356821)(783.69698753,1209.07566691)(783.06812707,1209.07567427)
\curveto(782.41321147,1209.07566691)(781.93598167,1208.95566175)(781.63643624,1208.71565845)
\curveto(781.33688617,1208.47564114)(781.1871123,1208.18632639)(781.18711417,1207.84771333)
\curveto(781.1871123,1207.55374108)(781.29316337,1207.31187023)(781.50526769,1207.12210005)
\curveto(781.71364654,1206.93231904)(782.25785596,1206.73789209)(783.13789757,1206.5388186)
\curveto(784.01793155,1206.33973546)(784.62167841,1206.1657745)(784.94913996,1206.0169352)
\curveto(785.42543385,1205.79738658)(785.77707686,1205.5192351)(786.00407004,1205.18247993)
\curveto(786.23104985,1204.84571781)(786.34454309,1204.45779417)(786.34455012,1204.01870785)
\curveto(786.34454309,1203.58333801)(786.21988658,1203.17308783)(785.9705802,1202.78795609)
\curveto(785.72126051,1202.4028222)(785.3631056,1202.10327445)(784.89611437,1201.88931195)
\curveto(784.429112,1201.67534909)(783.90350803,1201.56836775)(783.31930089,1201.56836761)
\curveto(782.57879998,1201.56836775)(781.95830821,1201.67627937)(781.45782374,1201.89210277)
\curveto(780.95733499,1202.10792581)(780.56475999,1202.43259092)(780.28009757,1202.86609906)
\curveto(779.99543322,1203.29960489)(779.84565934,1203.78985851)(779.8307755,1204.33686137)
\closepath
}
}
{
\newrgbcolor{curcolor}{0 0 0}
\pscustom[linestyle=none,fillstyle=solid,fillcolor=curcolor]
{
\newpath
\moveto(787.83484874,1201.70790863)
\lineto(787.83484874,1209.89059381)
\lineto(790.65357727,1209.89059381)
\curveto(791.28987996,1209.89058562)(791.77548221,1209.85151418)(792.11038547,1209.77337935)
\curveto(792.57923766,1209.66545968)(792.97925484,1209.47010245)(793.31043821,1209.18730709)
\curveto(793.7420783,1208.82263278)(794.06488286,1208.35656626)(794.27885286,1207.78910611)
\curveto(794.49280821,1207.22163379)(794.59978955,1206.57323386)(794.5997972,1205.84390434)
\curveto(794.59978955,1205.22247817)(794.5272283,1204.67175685)(794.38211321,1204.19173871)
\curveto(794.23698327,1203.71171562)(794.05092877,1203.31448925)(793.82394915,1203.00005844)
\curveto(793.59695579,1202.68562504)(793.34857303,1202.43817255)(793.07880012,1202.25770023)
\curveto(792.80901497,1202.07722682)(792.48341959,1201.94047676)(792.10201301,1201.84744965)
\curveto(791.72059613,1201.75442226)(791.28243778,1201.70790863)(790.78753664,1201.70790863)
\closepath
\moveto(788.91768703,1202.67353246)
\lineto(790.66474055,1202.67353246)
\curveto(791.20429489,1202.6735315)(791.62756888,1202.72376621)(791.93456379,1202.82423676)
\curveto(792.24154874,1202.92470507)(792.48621041,1203.06610649)(792.66854954,1203.24844145)
\curveto(792.92529903,1203.50519512)(793.12530762,1203.85032622)(793.26857591,1204.28383579)
\curveto(793.41183156,1204.7173402)(793.48346254,1205.24294417)(793.48346907,1205.86064927)
\curveto(793.48346254,1206.71649582)(793.34299139,1207.37419849)(793.0620552,1207.83375923)
\curveto(792.7811068,1208.29330773)(792.43969678,1208.60122793)(792.03782415,1208.75752076)
\curveto(791.74757404,1208.86914641)(791.28057724,1208.92496276)(790.63683235,1208.92496998)
\lineto(788.91768703,1208.92496998)
\closepath
}
}
{
\newrgbcolor{curcolor}{0 0 0}
\pscustom[linestyle=none,fillstyle=solid,fillcolor=curcolor]
{
\newpath
\moveto(602.03904024,1158.36592865)
\lineto(602.03904024,1160.32508451)
\lineto(598.4891168,1160.32508451)
\lineto(598.4891168,1161.24605522)
\lineto(602.22323438,1166.54861383)
\lineto(603.04373556,1166.54861383)
\lineto(603.04373556,1161.24605522)
\lineto(604.1489004,1161.24605522)
\lineto(604.1489004,1160.32508451)
\lineto(603.04373556,1160.32508451)
\lineto(603.04373556,1158.36592865)
\closepath
\moveto(602.03904024,1161.24605522)
\lineto(602.03904024,1164.93551968)
\lineto(599.47706719,1161.24605522)
\closepath
}
}
{
\newrgbcolor{curcolor}{0 0 0}
\pscustom[linestyle=none,fillstyle=solid,fillcolor=curcolor]
{
\newpath
\moveto(605.74524979,1158.36592865)
\lineto(605.74524979,1159.51016498)
\lineto(606.88948612,1159.51016498)
\lineto(606.88948612,1158.36592865)
\closepath
}
}
{
\newrgbcolor{curcolor}{0 0 0}
\pscustom[linestyle=none,fillstyle=solid,fillcolor=curcolor]
{
\newpath
\moveto(612.13622844,1158.36592865)
\lineto(611.13153313,1158.36592865)
\lineto(611.13153313,1164.76807046)
\curveto(610.88965902,1164.53735648)(610.5724361,1164.3066489)(610.1798634,1164.07594702)
\curveto(609.7872861,1163.84523373)(609.43471282,1163.67220304)(609.1221425,1163.55685444)
\lineto(609.1221425,1164.52805991)
\curveto(609.68402585,1164.79225115)(610.17520974,1165.11226489)(610.59569563,1165.4881021)
\curveto(611.01617608,1165.86392508)(611.31386329,1166.2285919)(611.48875813,1166.58210367)
\lineto(612.13622844,1166.58210367)
\closepath
}
}
{
\newrgbcolor{curcolor}{0 0 0}
\pscustom[linestyle=none,fillstyle=solid,fillcolor=curcolor]
{
\newpath
\moveto(615.07775322,1158.36592865)
\lineto(615.07775322,1166.54861383)
\lineto(618.14765557,1166.54861383)
\curveto(618.77279479,1166.54860564)(619.27421167,1166.46581139)(619.65190772,1166.30023082)
\curveto(620.02959295,1166.13463438)(620.32541961,1165.87973971)(620.53938859,1165.53554605)
\curveto(620.75334496,1165.19133805)(620.8603263,1164.83132259)(620.86033292,1164.45549859)
\curveto(620.8603263,1164.10571003)(620.76543851,1163.77639357)(620.57566925,1163.46754819)
\curveto(620.38588732,1163.15869262)(620.09936339,1162.90937959)(619.71609659,1162.71960835)
\curveto(620.21099609,1162.57448148)(620.59147755,1162.327029)(620.8575421,1161.97725014)
\curveto(621.12359342,1161.62746407)(621.25662239,1161.21442307)(621.25662941,1160.73812592)
\curveto(621.25662239,1160.35485128)(621.17568868,1159.9985569)(621.01382804,1159.66924174)
\curveto(620.85195385,1159.33992397)(620.65194526,1159.08595957)(620.41380167,1158.90734779)
\curveto(620.17564574,1158.72873493)(619.87702826,1158.59384542)(619.51794835,1158.50267885)
\curveto(619.15885788,1158.411512)(618.71883899,1158.36592865)(618.19789034,1158.36592865)
\closepath
\moveto(616.16059151,1163.11032319)
\lineto(617.92997159,1163.11032319)
\curveto(618.40998851,1163.11031845)(618.75418934,1163.14194771)(618.96257511,1163.20521108)
\curveto(619.23793105,1163.28707023)(619.44538182,1163.42289001)(619.58492804,1163.61267085)
\curveto(619.72446357,1163.8024412)(619.79423401,1164.04059096)(619.79423956,1164.32712085)
\curveto(619.79423401,1164.59875446)(619.72911493,1164.8378345)(619.59888214,1165.04436167)
\curveto(619.46863863,1165.25087549)(619.28258413,1165.39227691)(619.04071807,1165.46856636)
\curveto(618.79884242,1165.54484161)(618.38394088,1165.58298278)(617.79601221,1165.58299)
\lineto(616.16059151,1165.58299)
\closepath
\moveto(616.16059151,1159.33155248)
\lineto(618.19789034,1159.33155248)
\curveto(618.54766884,1159.33155152)(618.79326079,1159.34457533)(618.9346669,1159.37062397)
\curveto(619.18397524,1159.41527604)(619.39235628,1159.48969784)(619.55981065,1159.59388959)
\curveto(619.72725439,1159.69807888)(619.86493472,1159.8497133)(619.97285206,1160.0487933)
\curveto(620.08075794,1160.24786994)(620.13471375,1160.47764725)(620.13471964,1160.73812592)
\curveto(620.13471375,1161.04325293)(620.05657086,1161.3083806)(619.90029073,1161.53350971)
\curveto(619.74399929,1161.75863249)(619.5272458,1161.91677882)(619.2500296,1162.00794917)
\curveto(618.97280338,1162.09911223)(618.57371647,1162.14469558)(618.05276768,1162.14469936)
\lineto(616.16059151,1162.14469936)
\closepath
}
}
{
\newrgbcolor{curcolor}{0 0 0}
\pscustom[linestyle=none,fillstyle=solid,fillcolor=curcolor]
{
\newpath
\moveto(622.38970303,1160.99488139)
\lineto(623.41114327,1161.08418764)
\curveto(623.4595159,1160.67486502)(623.57207888,1160.33903664)(623.74883253,1160.07670151)
\curveto(623.92558243,1159.81436295)(624.20001282,1159.60226082)(624.57212452,1159.44039447)
\curveto(624.94423083,1159.27852598)(625.36285346,1159.19759227)(625.82799366,1159.19759311)
\curveto(626.24103071,1159.19759227)(626.60569753,1159.25899026)(626.92199523,1159.38178725)
\curveto(627.23828284,1159.5045822)(627.47364178,1159.67296153)(627.62807277,1159.88692572)
\curveto(627.78249226,1160.10088688)(627.85970487,1160.33438528)(627.85971086,1160.58742162)
\curveto(627.85970487,1160.84417462)(627.78528307,1161.06837029)(627.63644523,1161.26000932)
\curveto(627.48759587,1161.45164256)(627.24200393,1161.61257971)(626.89966867,1161.74282124)
\curveto(626.68011933,1161.82840293)(626.19451708,1161.9614319)(625.44286046,1162.14190854)
\curveto(624.69119671,1162.32237763)(624.16466247,1162.4926175)(623.86325616,1162.65262866)
\curveto(623.47253972,1162.85728433)(623.18136442,1163.11124872)(622.9897294,1163.41452261)
\curveto(622.79809215,1163.7177864)(622.70227408,1164.05733586)(622.70227491,1164.43317202)
\curveto(622.70227408,1164.84620695)(622.81948842,1165.23227004)(623.05391827,1165.59136246)
\curveto(623.28834576,1165.95044042)(623.63068605,1166.22301027)(624.08094015,1166.40907281)
\curveto(624.53118983,1166.59511927)(625.03167644,1166.68814652)(625.58240148,1166.68815484)
\curveto(626.18893545,1166.68814652)(626.72384214,1166.59046791)(627.18712316,1166.39511871)
\curveto(627.65039356,1166.19975345)(628.00668793,1165.91229925)(628.25600734,1165.53275523)
\curveto(628.505314,1165.15319688)(628.63927324,1164.72341098)(628.65788547,1164.24339624)
\lineto(627.61970031,1164.16525327)
\curveto(627.56387822,1164.68247899)(627.3750329,1165.07319344)(627.05316379,1165.33739781)
\curveto(626.73128432,1165.60158823)(626.25591507,1165.73368693)(625.6270546,1165.73369429)
\curveto(624.972139,1165.73368693)(624.49490921,1165.61368177)(624.19536378,1165.37367847)
\curveto(623.89581371,1165.13366116)(623.74603984,1164.84434641)(623.74604171,1164.50573335)
\curveto(623.74603984,1164.2117611)(623.8520909,1163.96989025)(624.06419522,1163.78012007)
\curveto(624.27257408,1163.59033906)(624.8167835,1163.39591211)(625.69682511,1163.19683862)
\curveto(626.57685908,1162.99775547)(627.18060594,1162.82379452)(627.5080675,1162.67495522)
\curveto(627.98436139,1162.4554066)(628.3360044,1162.17725512)(628.56299758,1161.84049995)
\curveto(628.78997738,1161.50373782)(628.90347063,1161.11581419)(628.90347766,1160.67672787)
\curveto(628.90347063,1160.24135803)(628.77881411,1159.83110785)(628.52950773,1159.44597611)
\curveto(628.28018805,1159.06084221)(627.92203313,1158.76129447)(627.45504191,1158.54733197)
\curveto(626.98803953,1158.33336911)(626.46243556,1158.22638777)(625.87822843,1158.22638763)
\curveto(625.13772751,1158.22638777)(624.51723575,1158.33429938)(624.01675128,1158.55012279)
\curveto(623.51626253,1158.76594583)(623.12368753,1159.09061094)(622.8390251,1159.52411908)
\curveto(622.55436075,1159.95762491)(622.40458688,1160.44787853)(622.38970303,1160.99488139)
\closepath
}
}
{
\newrgbcolor{curcolor}{0 0 0}
\pscustom[linestyle=none,fillstyle=solid,fillcolor=curcolor]
{
\newpath
\moveto(630.39377437,1158.36592865)
\lineto(630.39377437,1166.54861383)
\lineto(633.2125029,1166.54861383)
\curveto(633.84880559,1166.54860564)(634.33440784,1166.5095342)(634.6693111,1166.43139937)
\curveto(635.13816329,1166.3234797)(635.53818047,1166.12812247)(635.86936384,1165.84532711)
\curveto(636.30100393,1165.4806528)(636.62380849,1165.01458628)(636.83777849,1164.44712613)
\curveto(637.05173384,1163.87965381)(637.15871518,1163.23125387)(637.15872283,1162.50192436)
\curveto(637.15871518,1161.88049819)(637.08615392,1161.32977687)(636.94103884,1160.84975873)
\curveto(636.7959089,1160.36973564)(636.6098544,1159.97250927)(636.38287478,1159.65807846)
\curveto(636.15588142,1159.34364506)(635.90749866,1159.09619257)(635.63772575,1158.91572025)
\curveto(635.3679406,1158.73524684)(635.04234522,1158.59849678)(634.66093864,1158.50546967)
\curveto(634.27952176,1158.41244228)(633.84136341,1158.36592865)(633.34646227,1158.36592865)
\closepath
\moveto(631.47661266,1159.33155248)
\lineto(633.22366618,1159.33155248)
\curveto(633.76322052,1159.33155152)(634.18649451,1159.38178623)(634.49348942,1159.48225678)
\curveto(634.80047437,1159.58272509)(635.04513604,1159.72412651)(635.22747517,1159.90646147)
\curveto(635.48422466,1160.16321514)(635.68423325,1160.50834624)(635.82750153,1160.9418558)
\curveto(635.97075719,1161.37536022)(636.04238817,1161.90096419)(636.0423947,1162.51866929)
\curveto(636.04238817,1163.37451584)(635.90191702,1164.03221851)(635.62098083,1164.49177925)
\curveto(635.34003242,1164.95132774)(634.99862241,1165.25924795)(634.59674977,1165.41554078)
\curveto(634.30649967,1165.52716643)(633.83950287,1165.58298278)(633.19575797,1165.58299)
\lineto(631.47661266,1165.58299)
\closepath
}
}
{
\newrgbcolor{curcolor}{0 0 0}
\pscustom[linestyle=none,fillstyle=solid,fillcolor=curcolor]
{
\newpath
\moveto(95.92843947,1123.94155816)
\lineto(97.01127775,1123.94155816)
\lineto(97.01127775,1119.21390854)
\curveto(97.01127042,1118.39154419)(96.91824317,1117.73849289)(96.73219572,1117.25475268)
\curveto(96.54613416,1116.77100948)(96.21030579,1116.3775042)(95.72470958,1116.07423568)
\curveto(95.23910129,1115.77096653)(94.60186462,1115.61933211)(93.81299767,1115.61933197)
\curveto(93.04644898,1115.61933211)(92.41944531,1115.75143081)(91.93198477,1116.01562846)
\curveto(91.44451972,1116.27982559)(91.0965978,1116.66216759)(90.88821797,1117.16265561)
\curveto(90.67983572,1117.66314081)(90.5756452,1118.34689111)(90.5756461,1119.21390854)
\lineto(90.5756461,1123.94155816)
\lineto(91.65848438,1123.94155816)
\lineto(91.65848438,1119.21949018)
\curveto(91.6584824,1118.50875852)(91.72453175,1117.9850151)(91.85663262,1117.64825834)
\curveto(91.98872914,1117.3114978)(92.21571563,1117.05195177)(92.53759278,1116.86961947)
\curveto(92.85946421,1116.68728495)(93.25296948,1116.59611824)(93.71810978,1116.59611908)
\curveto(94.514419,1116.59611824)(95.08188523,1116.77659111)(95.42051017,1117.13753822)
\curveto(95.75912362,1117.49848258)(95.92843322,1118.19246587)(95.92843947,1119.21949018)
\closepath
}
}
{
\newrgbcolor{curcolor}{0 0 0}
\pscustom[linestyle=none,fillstyle=solid,fillcolor=curcolor]
{
\newpath
\moveto(98.80856591,1115.75887299)
\lineto(98.80856591,1123.94155816)
\lineto(99.9193124,1123.94155816)
\lineto(104.21717569,1117.51708979)
\lineto(104.21717569,1123.94155816)
\lineto(105.25536085,1123.94155816)
\lineto(105.25536085,1115.75887299)
\lineto(104.14461436,1115.75887299)
\lineto(99.84675107,1122.188923)
\lineto(99.84675107,1115.75887299)
\closepath
}
}
{
\newrgbcolor{curcolor}{0 0 0}
\pscustom[linestyle=none,fillstyle=solid,fillcolor=curcolor]
{
\newpath
\moveto(107.26475135,1115.75887299)
\lineto(107.26475135,1123.94155816)
\lineto(108.34758964,1123.94155816)
\lineto(108.34758964,1115.75887299)
\closepath
}
}
{
\newrgbcolor{curcolor}{0 0 0}
\pscustom[linestyle=none,fillstyle=solid,fillcolor=curcolor]
{
\newpath
\moveto(109.41926476,1115.75887299)
\lineto(112.58405501,1120.02324643)
\lineto(109.79323469,1123.94155816)
\lineto(111.08259367,1123.94155816)
\lineto(112.56731008,1121.84286128)
\curveto(112.87615736,1121.40748766)(113.09570167,1121.07258956)(113.22594368,1120.83816597)
\curveto(113.40827323,1121.13584809)(113.62409646,1121.44655911)(113.87341399,1121.77029995)
\lineto(115.51999798,1123.94155816)
\lineto(116.69772416,1123.94155816)
\lineto(113.82317923,1120.08464448)
\lineto(116.92098978,1115.75887299)
\lineto(115.58139603,1115.75887299)
\lineto(113.52177063,1118.67807104)
\curveto(113.40641269,1118.84551717)(113.28733781,1119.02785058)(113.16454563,1119.22507182)
\curveto(112.98220843,1118.92738115)(112.85197027,1118.7227212)(112.77383079,1118.61109135)
\lineto(110.71978703,1115.75887299)
\closepath
}
}
{
\newrgbcolor{curcolor}{0 0 0}
\pscustom[linestyle=none,fillstyle=solid,fillcolor=curcolor]
{
\newpath
\moveto(120.68859791,1118.38782573)
\lineto(121.71003815,1118.47713198)
\curveto(121.75841079,1118.06780935)(121.87097376,1117.73198098)(122.04772741,1117.46964584)
\curveto(122.22447731,1117.20730728)(122.4989077,1116.99520515)(122.8710194,1116.83333881)
\curveto(123.24312571,1116.67147032)(123.66174834,1116.59053661)(124.12688855,1116.59053744)
\curveto(124.53992559,1116.59053661)(124.90459241,1116.6519346)(125.22089011,1116.77473158)
\curveto(125.53717772,1116.89752654)(125.77253666,1117.06590586)(125.92696765,1117.27987006)
\curveto(126.08138714,1117.49383122)(126.15859976,1117.72732962)(126.15860574,1117.98036596)
\curveto(126.15859976,1118.23711895)(126.08417796,1118.46131463)(125.93534011,1118.65295366)
\curveto(125.78649075,1118.8445869)(125.54089881,1119.00552404)(125.19856355,1119.13576557)
\curveto(124.97901421,1119.22134727)(124.49341196,1119.35437624)(123.74175534,1119.53485288)
\curveto(122.99009159,1119.71532197)(122.46355735,1119.88556184)(122.16215104,1120.045573)
\curveto(121.7714346,1120.25022866)(121.48025931,1120.50419306)(121.28862428,1120.80746694)
\curveto(121.09698703,1121.11073073)(121.00116896,1121.4502802)(121.00116979,1121.82611636)
\curveto(121.00116896,1122.23915129)(121.1183833,1122.62521438)(121.35281315,1122.98430679)
\curveto(121.58724064,1123.34338476)(121.92958093,1123.6159546)(122.37983503,1123.80201715)
\curveto(122.83008472,1123.9880636)(123.33057133,1124.08109086)(123.88129636,1124.08109918)
\curveto(124.48783033,1124.08109086)(125.02273702,1123.98341224)(125.48601804,1123.78806304)
\curveto(125.94928844,1123.59269779)(126.30558281,1123.30524358)(126.55490222,1122.92569957)
\curveto(126.80420888,1122.54614122)(126.93816812,1122.11635532)(126.95678035,1121.63634058)
\lineto(125.91859519,1121.55819761)
\curveto(125.8627731,1122.07542333)(125.67392778,1122.46613778)(125.35205867,1122.73034214)
\curveto(125.0301792,1122.99453256)(124.55480995,1123.12663126)(123.92594948,1123.12663863)
\curveto(123.27103389,1123.12663126)(122.79380409,1123.00662611)(122.49425866,1122.76662281)
\curveto(122.19470859,1122.52660549)(122.04493472,1122.23729074)(122.04493659,1121.89867769)
\curveto(122.04493472,1121.60470544)(122.15098579,1121.36283458)(122.36309011,1121.17306441)
\curveto(122.57146896,1120.9832834)(123.11567838,1120.78885644)(123.99571999,1120.58978296)
\curveto(124.87575397,1120.39069981)(125.47950082,1120.21673885)(125.80696238,1120.06789956)
\curveto(126.28325627,1119.84835094)(126.63489928,1119.57019946)(126.86189246,1119.23344428)
\curveto(127.08887227,1118.89668216)(127.20236551,1118.50875852)(127.20237254,1118.06967221)
\curveto(127.20236551,1117.63430237)(127.077709,1117.22405219)(126.82840262,1116.83892045)
\curveto(126.57908293,1116.45378655)(126.22092801,1116.1542388)(125.75393679,1115.94027631)
\curveto(125.28693441,1115.72631345)(124.76133045,1115.61933211)(124.17712331,1115.61933197)
\curveto(123.43662239,1115.61933211)(122.81613063,1115.72724372)(122.31564616,1115.94306713)
\curveto(121.81515741,1116.15889017)(121.42258241,1116.48355527)(121.13791999,1116.91706342)
\curveto(120.85325563,1117.35056925)(120.70348176,1117.84082286)(120.68859791,1118.38782573)
\closepath
}
}
{
\newrgbcolor{curcolor}{0 0 0}
\pscustom[linestyle=none,fillstyle=solid,fillcolor=curcolor]
{
\newpath
\moveto(128.5196403,1113.47598197)
\lineto(128.40800749,1114.41927923)
\curveto(128.62755121,1114.35974313)(128.81918734,1114.32997441)(128.98291648,1114.32997298)
\curveto(129.20618071,1114.32997441)(129.38479303,1114.36718531)(129.51875398,1114.4416058)
\curveto(129.65271151,1114.51602891)(129.76248367,1114.62021943)(129.84807078,1114.75417767)
\curveto(129.91132727,1114.85464811)(130.01365724,1115.10396114)(130.15506101,1115.50211752)
\curveto(130.17366412,1115.55793412)(130.20343284,1115.6397981)(130.24436726,1115.7477097)
\lineto(127.99496608,1121.68657534)
\lineto(129.07780437,1121.68657534)
\lineto(130.31134695,1118.25386635)
\curveto(130.47135132,1117.81849632)(130.61461329,1117.36080225)(130.74113328,1116.88078275)
\curveto(130.85648414,1117.3421968)(130.99416447,1117.79244869)(131.15417468,1118.23153979)
\lineto(132.42120711,1121.68657534)
\lineto(133.42590242,1121.68657534)
\lineto(131.17091961,1115.65840345)
\curveto(130.92904539,1115.0072128)(130.74113035,1114.55882145)(130.6071739,1114.31322806)
\curveto(130.42855878,1113.98205249)(130.22389883,1113.73925137)(129.99319343,1113.58482396)
\curveto(129.76248367,1113.43040089)(129.487123,1113.35318828)(129.16711062,1113.35318587)
\curveto(128.97361258,1113.35318828)(128.75778936,1113.39412027)(128.5196403,1113.47598197)
\closepath
}
}
{
\newrgbcolor{curcolor}{0 0 0}
\pscustom[linestyle=none,fillstyle=solid,fillcolor=curcolor]
{
\newpath
\moveto(133.87801436,1117.52825307)
\lineto(134.8715464,1117.68453901)
\curveto(134.9273614,1117.28638045)(135.08271691,1116.98125106)(135.33761339,1116.76914994)
\curveto(135.59250625,1116.5570468)(135.94880062,1116.45099573)(136.40649757,1116.45099643)
\curveto(136.86790986,1116.45099573)(137.21025014,1116.54495326)(137.43351945,1116.73286928)
\curveto(137.65678094,1116.92078335)(137.76841365,1117.14125794)(137.76841789,1117.39429369)
\curveto(137.76841365,1117.62127855)(137.66980476,1117.79989087)(137.47259093,1117.93013119)
\curveto(137.33490666,1118.01943518)(136.99256637,1118.13292843)(136.44556906,1118.27061127)
\curveto(135.70879031,1118.45666326)(135.1980707,1118.61760041)(134.9134087,1118.75342319)
\curveto(134.62874393,1118.88923998)(134.4129207,1119.07715503)(134.26593839,1119.31716889)
\curveto(134.11895459,1119.55717564)(134.04546306,1119.82230331)(134.04546358,1120.11255268)
\curveto(134.04546306,1120.37674572)(134.10593078,1120.62140739)(134.2268669,1120.84653843)
\curveto(134.34780163,1121.07165929)(134.51245986,1121.25864406)(134.7208421,1121.40749331)
\curveto(134.87712669,1121.52284145)(135.09015909,1121.62052007)(135.35993995,1121.70052945)
\curveto(135.62971715,1121.78052694)(135.9190319,1121.82052866)(136.22788507,1121.82053472)
\curveto(136.69301862,1121.82052866)(137.10140826,1121.75354904)(137.45305519,1121.61959566)
\curveto(137.80469427,1121.48563055)(138.0642403,1121.30422742)(138.23169406,1121.07538569)
\curveto(138.39913841,1120.84653334)(138.5144922,1120.54047368)(138.57775578,1120.15720581)
\lineto(137.59538703,1120.02324643)
\curveto(137.55072988,1120.32837155)(137.421422,1120.56652132)(137.207463,1120.73769644)
\curveto(136.99349664,1120.9088616)(136.69115808,1120.99444667)(136.3004464,1120.9944519)
\curveto(135.83902846,1120.99444667)(135.50971199,1120.91816432)(135.31249601,1120.76560464)
\curveto(135.11527645,1120.61303494)(135.01666756,1120.43442262)(135.01666905,1120.22976714)
\curveto(135.01666756,1120.09952452)(135.05759955,1119.98231018)(135.13946515,1119.87812378)
\curveto(135.22132751,1119.77020805)(135.34970512,1119.68090189)(135.52459835,1119.61020503)
\curveto(135.62506578,1119.57299027)(135.92089244,1119.4874052)(136.41207921,1119.35344956)
\curveto(137.12280452,1119.16367037)(137.61863977,1119.00831486)(137.89958644,1118.88738256)
\curveto(138.18052437,1118.76644401)(138.40099895,1118.5906225)(138.56101086,1118.35991752)
\curveto(138.7210127,1118.12920734)(138.80101613,1117.84268341)(138.80102141,1117.50034487)
\curveto(138.80101613,1117.16544502)(138.70333752,1116.85008264)(138.50798527,1116.55425678)
\curveto(138.31262306,1116.25842932)(138.03075049,1116.02958229)(137.66236672,1115.86771498)
\curveto(137.29397467,1115.70584745)(136.87721258,1115.62491374)(136.41207921,1115.62491361)
\curveto(135.64181069,1115.62491374)(135.05480874,1115.78492062)(134.65107159,1116.10493471)
\curveto(134.2473322,1116.4249481)(133.98964671,1116.89938708)(133.87801436,1117.52825307)
\closepath
}
}
{
\newrgbcolor{curcolor}{0 0 0}
\pscustom[linestyle=none,fillstyle=solid,fillcolor=curcolor]
{
\newpath
\moveto(142.18907632,1116.65751713)
\lineto(142.33419898,1115.77003627)
\curveto(142.05139304,1115.71049882)(141.79835892,1115.6807301)(141.57509585,1115.68073002)
\curveto(141.21042669,1115.6807301)(140.92762385,1115.73840699)(140.72668647,1115.85376088)
\curveto(140.52574613,1115.96911457)(140.3843447,1116.12074899)(140.30248178,1116.30866459)
\curveto(140.22061674,1116.49657909)(140.17968475,1116.8919449)(140.17968569,1117.49476322)
\lineto(140.17968569,1120.90514565)
\lineto(139.44290913,1120.90514565)
\lineto(139.44290913,1121.68657534)
\lineto(140.17968569,1121.68657534)
\lineto(140.17968569,1123.15454683)
\lineto(141.17879936,1123.75736402)
\lineto(141.17879936,1121.68657534)
\lineto(142.18907632,1121.68657534)
\lineto(142.18907632,1120.90514565)
\lineto(141.17879936,1120.90514565)
\lineto(141.17879936,1117.43894682)
\curveto(141.17879743,1117.15242121)(141.1964726,1116.96822725)(141.23182495,1116.8863644)
\curveto(141.26717332,1116.80449929)(141.32485021,1116.73938021)(141.40485581,1116.69100697)
\curveto(141.48485708,1116.64263187)(141.5992806,1116.61844478)(141.74812671,1116.61844564)
\curveto(141.8597569,1116.61844478)(142.00673996,1116.6314686)(142.18907632,1116.65751713)
\closepath
}
}
{
\newrgbcolor{curcolor}{0 0 0}
\pscustom[linestyle=none,fillstyle=solid,fillcolor=curcolor]
{
\newpath
\moveto(147.2237163,1117.66779408)
\lineto(148.26190146,1117.53941635)
\curveto(148.09816765,1116.93287689)(147.79489881,1116.462159)(147.35209404,1116.12726127)
\curveto(146.90927938,1115.7923628)(146.3436737,1115.62491374)(145.65527528,1115.62491361)
\curveto(144.78825806,1115.62491374)(144.10078668,1115.89190195)(143.59285907,1116.42587904)
\curveto(143.0849291,1116.9598548)(142.8309647,1117.70872417)(142.83096512,1118.6724894)
\curveto(142.8309647,1119.66973862)(143.08771991,1120.44372534)(143.60123153,1120.9944519)
\curveto(144.11474076,1121.54516799)(144.78081588,1121.82052866)(145.59945888,1121.82053472)
\curveto(146.39204787,1121.82052866)(147.03951753,1121.55074963)(147.54186982,1121.01119683)
\curveto(148.04421184,1120.47163352)(148.29538542,1119.71253115)(148.2953913,1118.73388745)
\curveto(148.29538542,1118.67434703)(148.29352488,1118.58504087)(148.28980966,1118.4659687)
\lineto(143.86915028,1118.4659687)
\curveto(143.90635972,1117.81477523)(144.09055368,1117.31614917)(144.4217327,1116.970089)
\curveto(144.75290771,1116.62402642)(145.1659487,1116.45099573)(145.66085692,1116.45099643)
\curveto(146.02924159,1116.45099573)(146.3436737,1116.54774407)(146.60415419,1116.74124174)
\curveto(146.8646263,1116.93473744)(147.0711468,1117.24358791)(147.2237163,1117.66779408)
\closepath
\moveto(143.92496668,1119.29205151)
\lineto(147.23487958,1119.29205151)
\curveto(147.19022168,1119.79067404)(147.06370462,1120.16464359)(146.85532802,1120.41396128)
\curveto(146.53530983,1120.80094999)(146.12040829,1120.99444667)(145.61062216,1120.9944519)
\curveto(145.14920379,1120.99444667)(144.76128016,1120.84002143)(144.44685008,1120.53117573)
\curveto(144.13241594,1120.22232049)(143.95845498,1119.80927949)(143.92496668,1119.29205151)
\closepath
}
}
{
\newrgbcolor{curcolor}{0 0 0}
\pscustom[linestyle=none,fillstyle=solid,fillcolor=curcolor]
{
\newpath
\moveto(149.528935,1115.75887299)
\lineto(149.528935,1121.68657534)
\lineto(150.42757914,1121.68657534)
\lineto(150.42757914,1120.85491089)
\curveto(150.61363199,1121.14515082)(150.86108448,1121.37864922)(151.16993735,1121.55540679)
\curveto(151.47878543,1121.73215277)(151.83042843,1121.82052866)(152.22486743,1121.82053472)
\curveto(152.6639526,1121.82052866)(153.02396806,1121.72936195)(153.30491489,1121.54703433)
\curveto(153.58585266,1121.36469513)(153.7840007,1121.10980046)(153.89935962,1120.78234956)
\curveto(154.36821184,1121.47446728)(154.97847061,1121.82052866)(155.73013775,1121.82053472)
\curveto(156.31806302,1121.82052866)(156.77017546,1121.65773097)(157.08647642,1121.33214116)
\curveto(157.40276077,1121.00654021)(157.56090709,1120.50512333)(157.56091588,1119.82788901)
\lineto(157.56091588,1115.75887299)
\lineto(156.5618022,1115.75887299)
\lineto(156.5618022,1119.49299057)
\curveto(156.56179442,1119.89486456)(156.52923488,1120.18417931)(156.46412349,1120.36093569)
\curveto(156.39899673,1120.53768287)(156.28085212,1120.68001456)(156.10968931,1120.7879312)
\curveto(155.93851184,1120.89583778)(155.73757297,1120.94979359)(155.50687212,1120.94979878)
\curveto(155.09010331,1120.94979359)(154.74404193,1120.81118299)(154.46868696,1120.53396655)
\curveto(154.19332061,1120.25674057)(154.05564028,1119.81300058)(154.05564556,1119.20274526)
\lineto(154.05564556,1115.75887299)
\lineto(153.05095024,1115.75887299)
\lineto(153.05095024,1119.61020503)
\curveto(153.05094597,1120.05673198)(152.96908199,1120.39163008)(152.80535805,1120.61490034)
\curveto(152.64162606,1120.83816089)(152.37370758,1120.94979359)(152.0016018,1120.94979878)
\curveto(151.71879573,1120.94979359)(151.45738916,1120.87537179)(151.21738129,1120.72653315)
\curveto(150.97736854,1120.57768459)(150.80340758,1120.36000082)(150.69549789,1120.0734812)
\curveto(150.58758436,1119.78695295)(150.53362856,1119.37391196)(150.53363031,1118.83435698)
\lineto(150.53363031,1115.75887299)
\closepath
}
}
{
\newrgbcolor{curcolor}{0 0 0}
\pscustom[linestyle=none,fillstyle=solid,fillcolor=curcolor]
{
\newpath
\moveto(162.54531852,1115.75887299)
\lineto(162.54531852,1123.94155816)
\lineto(163.6281568,1123.94155816)
\lineto(163.6281568,1115.75887299)
\closepath
}
}
{
\newrgbcolor{curcolor}{0 0 0}
\pscustom[linestyle=none,fillstyle=solid,fillcolor=curcolor]
{
\newpath
\moveto(165.71569053,1115.75887299)
\lineto(165.71569053,1123.94155816)
\lineto(166.79852881,1123.94155816)
\lineto(166.79852881,1115.75887299)
\closepath
}
}
{
\newrgbcolor{curcolor}{0 0 0}
\pscustom[linestyle=none,fillstyle=solid,fillcolor=curcolor]
{
\newpath
\moveto(168.88606254,1115.75887299)
\lineto(168.88606254,1123.94155816)
\lineto(169.96890082,1123.94155816)
\lineto(169.96890082,1115.75887299)
\closepath
}
}
{
\newrgbcolor{curcolor}{0 0 0}
\pscustom[linestyle=none,fillstyle=solid,fillcolor=curcolor]
{
\newpath
\moveto(598.89102999,1114.65090179)
\lineto(598.89102999,1116.61005766)
\lineto(595.34110654,1116.61005766)
\lineto(595.34110654,1117.53102836)
\lineto(599.07522413,1122.83358697)
\lineto(599.8957253,1122.83358697)
\lineto(599.8957253,1117.53102836)
\lineto(601.00089015,1117.53102836)
\lineto(601.00089015,1116.61005766)
\lineto(599.8957253,1116.61005766)
\lineto(599.8957253,1114.65090179)
\closepath
\moveto(598.89102999,1117.53102836)
\lineto(598.89102999,1121.22049283)
\lineto(596.32905694,1117.53102836)
\closepath
}
}
{
\newrgbcolor{curcolor}{0 0 0}
\pscustom[linestyle=none,fillstyle=solid,fillcolor=curcolor]
{
\newpath
\moveto(602.59723953,1114.65090179)
\lineto(602.59723953,1115.79513813)
\lineto(603.74147586,1115.79513813)
\lineto(603.74147586,1114.65090179)
\closepath
}
}
{
\newrgbcolor{curcolor}{0 0 0}
\pscustom[linestyle=none,fillstyle=solid,fillcolor=curcolor]
{
\newpath
\moveto(608.98821819,1114.65090179)
\lineto(607.98352288,1114.65090179)
\lineto(607.98352288,1121.05304361)
\curveto(607.74164877,1120.82232962)(607.42442584,1120.59162204)(607.03185315,1120.36092017)
\curveto(606.63927585,1120.13020688)(606.28670256,1119.95717619)(605.97413225,1119.84182759)
\lineto(605.97413225,1120.81303306)
\curveto(606.5360156,1121.07722429)(607.02719948,1121.39723803)(607.44768537,1121.77307525)
\curveto(607.86816583,1122.14889822)(608.16585303,1122.51356504)(608.34074788,1122.86707681)
\lineto(608.98821819,1122.86707681)
\closepath
}
}
{
\newrgbcolor{curcolor}{0 0 0}
\pscustom[linestyle=none,fillstyle=solid,fillcolor=curcolor]
{
\newpath
\moveto(615.71409532,1116.82216)
\lineto(616.70204571,1116.69378227)
\curveto(616.59412849,1116.01282075)(616.31783756,1115.4797746)(615.87317208,1115.09464223)
\curveto(615.42849704,1114.70950896)(614.88242708,1114.51694255)(614.23496055,1114.51694242)
\curveto(613.42375978,1114.51694255)(612.77163875,1114.78207022)(612.27859551,1115.31232621)
\curveto(611.78554989,1115.84258088)(611.53902768,1116.60261352)(611.53902812,1117.59242641)
\curveto(611.53902768,1118.23245096)(611.64507874,1118.79247501)(611.85718164,1119.27250024)
\curveto(612.06928301,1119.75251624)(612.39208757,1120.1125317)(612.82559629,1120.35254771)
\curveto(613.25910155,1120.59255231)(613.73074971,1120.71255747)(614.24054219,1120.71256353)
\curveto(614.88428762,1120.71255747)(615.41082186,1120.54975978)(615.82014649,1120.22416997)
\curveto(616.22946167,1119.89856902)(616.49179852,1119.43622358)(616.60715782,1118.83713227)
\lineto(615.63037071,1118.68642798)
\curveto(615.53733892,1119.08458057)(615.37268069,1119.38412832)(615.13639551,1119.58507212)
\curveto(614.90010225,1119.78600605)(614.61450859,1119.88647548)(614.27961368,1119.88648071)
\curveto(613.77354224,1119.88647548)(613.3623618,1119.70507234)(613.0460711,1119.34227075)
\curveto(612.72977649,1118.97945978)(612.57163016,1118.40548164)(612.57163164,1117.62033461)
\curveto(612.57163016,1116.82401838)(612.72419485,1116.24538888)(613.02932617,1115.88444438)
\curveto(613.33445362,1115.52349741)(613.73261025,1115.34302454)(614.22379727,1115.34302523)
\curveto(614.61822968,1115.34302454)(614.94754615,1115.46395997)(615.21174766,1115.70583188)
\curveto(615.47594094,1115.94770167)(615.64338999,1116.31981068)(615.71409532,1116.82216)
\closepath
}
}
{
\newrgbcolor{curcolor}{0 0 0}
\pscustom[linestyle=none,fillstyle=solid,fillcolor=curcolor]
{
\newpath
\moveto(617.64534203,1114.65090179)
\lineto(617.64534203,1122.83358697)
\lineto(620.71524438,1122.83358697)
\curveto(621.3403836,1122.83357879)(621.84180048,1122.75078453)(622.21949653,1122.58520396)
\curveto(622.59718176,1122.41960752)(622.89300842,1122.16471285)(623.10697739,1121.82051919)
\curveto(623.32093377,1121.4763112)(623.42791511,1121.11629574)(623.42792173,1120.74047173)
\curveto(623.42791511,1120.39068318)(623.33302731,1120.06136671)(623.14325806,1119.75252134)
\curveto(622.95347613,1119.44366576)(622.6669522,1119.19435273)(622.2836854,1119.00458149)
\curveto(622.7785849,1118.85945463)(623.15906635,1118.61200214)(623.42513091,1118.26222329)
\curveto(623.69118223,1117.91243721)(623.8242112,1117.49939622)(623.82421821,1117.02309907)
\curveto(623.8242112,1116.63982442)(623.74327749,1116.28353005)(623.58141685,1115.95421488)
\curveto(623.41954266,1115.62489711)(623.21953407,1115.37093272)(622.98139048,1115.19232094)
\curveto(622.74323454,1115.01370807)(622.44461707,1114.87881856)(622.08553716,1114.78765199)
\curveto(621.72644669,1114.69648515)(621.28642779,1114.65090179)(620.76547914,1114.65090179)
\closepath
\moveto(618.72818031,1119.39529634)
\lineto(620.49756039,1119.39529634)
\curveto(620.97757732,1119.39529159)(621.32177815,1119.42692086)(621.53016391,1119.49018423)
\curveto(621.80551985,1119.57204337)(622.01297062,1119.70786316)(622.15251684,1119.89764399)
\curveto(622.29205238,1120.08741434)(622.36182281,1120.3255641)(622.36182837,1120.612094)
\curveto(622.36182281,1120.88372761)(622.29670374,1121.12280764)(622.16647094,1121.32933482)
\curveto(622.03622744,1121.53584864)(621.85017293,1121.67725006)(621.60830688,1121.75353951)
\curveto(621.36643123,1121.82981475)(620.95152969,1121.86795592)(620.36360102,1121.86796314)
\lineto(618.72818031,1121.86796314)
\closepath
\moveto(618.72818031,1115.61652563)
\lineto(620.76547914,1115.61652563)
\curveto(621.11525765,1115.61652466)(621.36084959,1115.62954847)(621.50225571,1115.65559711)
\curveto(621.75156405,1115.70024919)(621.95994509,1115.77467099)(622.12739946,1115.87886274)
\curveto(622.29484319,1115.98305203)(622.43252352,1116.13468645)(622.54044087,1116.33376645)
\curveto(622.64834675,1116.53284308)(622.70230255,1116.76262039)(622.70230845,1117.02309907)
\curveto(622.70230255,1117.32822608)(622.62415966,1117.59335374)(622.46787954,1117.81848286)
\curveto(622.3115881,1118.04360564)(622.0948346,1118.20175196)(621.8176184,1118.29292231)
\curveto(621.54039219,1118.38408538)(621.14130528,1118.42966873)(620.62035649,1118.42967251)
\lineto(618.72818031,1118.42967251)
\closepath
}
}
{
\newrgbcolor{curcolor}{0 0 0}
\pscustom[linestyle=none,fillstyle=solid,fillcolor=curcolor]
{
\newpath
\moveto(624.95729184,1117.27985454)
\lineto(625.97873207,1117.36916079)
\curveto(626.02710471,1116.95983816)(626.13966768,1116.62400979)(626.31642133,1116.36167465)
\curveto(626.49317124,1116.09933609)(626.76760163,1115.88723396)(627.13971333,1115.72536762)
\curveto(627.51181964,1115.56349913)(627.93044226,1115.48256542)(628.39558247,1115.48256625)
\curveto(628.80861951,1115.48256542)(629.17328634,1115.5439634)(629.48958404,1115.66676039)
\curveto(629.80587164,1115.78955535)(630.04123059,1115.95793467)(630.19566158,1116.17189887)
\curveto(630.35008106,1116.38586002)(630.42729368,1116.61935842)(630.42729966,1116.87239477)
\curveto(630.42729368,1117.12914776)(630.35287188,1117.35334343)(630.20403404,1117.54498247)
\curveto(630.05518468,1117.73661571)(629.80959273,1117.89755285)(629.46725747,1118.02779438)
\curveto(629.24770814,1118.11337607)(628.76210589,1118.24640504)(628.01044927,1118.42688169)
\curveto(627.25878551,1118.60735078)(626.73225127,1118.77759065)(626.43084497,1118.93760181)
\curveto(626.04012853,1119.14225747)(625.74895323,1119.39622187)(625.55731821,1119.69949575)
\curveto(625.36568096,1120.00275954)(625.26986289,1120.34230901)(625.26986371,1120.71814517)
\curveto(625.26986289,1121.1311801)(625.38707722,1121.51724319)(625.62150707,1121.8763356)
\curveto(625.85593457,1122.23541356)(626.19827485,1122.50798341)(626.64852895,1122.69404595)
\curveto(627.09877864,1122.88009241)(627.59926525,1122.97311966)(628.14999028,1122.97312799)
\curveto(628.75652425,1122.97311966)(629.29143095,1122.87544105)(629.75471197,1122.68009185)
\curveto(630.21798237,1122.4847266)(630.57427674,1122.19727239)(630.82359615,1121.81772837)
\curveto(631.0729028,1121.43817002)(631.20686204,1121.00838412)(631.22547427,1120.52836939)
\lineto(630.18728912,1120.45022642)
\curveto(630.13146702,1120.96745213)(629.9426217,1121.35816659)(629.62075259,1121.62237095)
\curveto(629.29887313,1121.88656137)(628.82350387,1122.01866007)(628.19464341,1122.01866744)
\curveto(627.53972781,1122.01866007)(627.06249801,1121.89865492)(626.76295258,1121.65865162)
\curveto(626.46340252,1121.4186343)(626.31362864,1121.12931955)(626.31363051,1120.7907065)
\curveto(626.31362864,1120.49673424)(626.41967971,1120.25486339)(626.63178403,1120.06509321)
\curveto(626.84016288,1119.87531221)(627.3843723,1119.68088525)(628.26441392,1119.48181177)
\curveto(629.14444789,1119.28272862)(629.74819475,1119.10876766)(630.0756563,1118.95992837)
\curveto(630.5519502,1118.74037975)(630.9035932,1118.46222827)(631.13058638,1118.12547309)
\curveto(631.35756619,1117.78871097)(631.47105944,1117.40078733)(631.47106646,1116.96170102)
\curveto(631.47105944,1116.52633117)(631.34640292,1116.116081)(631.09709654,1115.73094926)
\curveto(630.84777685,1115.34581536)(630.48962194,1115.04626761)(630.02263072,1114.83230512)
\curveto(629.55562834,1114.61834226)(629.03002437,1114.51136092)(628.44581724,1114.51136078)
\curveto(627.70531632,1114.51136092)(627.08482455,1114.61927253)(626.58434008,1114.83509594)
\curveto(626.08385133,1115.05091897)(625.69127633,1115.37558408)(625.40661391,1115.80909223)
\curveto(625.12194956,1116.24259806)(624.97217568,1116.73285167)(624.95729184,1117.27985454)
\closepath
}
}
{
\newrgbcolor{curcolor}{0 0 0}
\pscustom[linestyle=none,fillstyle=solid,fillcolor=curcolor]
{
\newpath
\moveto(632.96136318,1114.65090179)
\lineto(632.96136318,1122.83358697)
\lineto(635.7800917,1122.83358697)
\curveto(636.4163944,1122.83357879)(636.90199665,1122.79450734)(637.23689991,1122.71637252)
\curveto(637.7057521,1122.60845284)(638.10576928,1122.41309561)(638.43695265,1122.13030025)
\curveto(638.86859273,1121.76562595)(639.19139729,1121.29955942)(639.4053673,1120.73209927)
\curveto(639.61932265,1120.16462696)(639.72630399,1119.51622702)(639.72631163,1118.78689751)
\curveto(639.72630399,1118.16547134)(639.65374273,1117.61475001)(639.50862765,1117.13473188)
\curveto(639.36349771,1116.65470878)(639.17744321,1116.25748242)(638.95046358,1115.9430516)
\curveto(638.72347022,1115.6286182)(638.47508746,1115.38116571)(638.20531456,1115.2006934)
\curveto(637.93552941,1115.02021998)(637.60993403,1114.88346992)(637.22852745,1114.79044281)
\curveto(636.84711057,1114.69741542)(636.40895222,1114.65090179)(635.91405108,1114.65090179)
\closepath
\moveto(634.04420146,1115.61652563)
\lineto(635.79125498,1115.61652563)
\curveto(636.33080933,1115.61652466)(636.75408332,1115.66675938)(637.06107823,1115.76722992)
\curveto(637.36806318,1115.86769824)(637.61272485,1116.00909966)(637.79506397,1116.19143461)
\curveto(638.05181347,1116.44818828)(638.25182206,1116.79331938)(638.39509034,1117.22682895)
\curveto(638.53834599,1117.66033336)(638.60997698,1118.18593733)(638.60998351,1118.80364243)
\curveto(638.60997698,1119.65948899)(638.46950583,1120.31719165)(638.18856964,1120.7767524)
\curveto(637.90762123,1121.23630089)(637.56621122,1121.54422109)(637.16433858,1121.70051392)
\curveto(636.87408847,1121.81213957)(636.40709167,1121.86795592)(635.76334678,1121.86796314)
\lineto(634.04420146,1121.86796314)
\closepath
}
}
{
\newrgbcolor{curcolor}{0 0 0}
\pscustom[linestyle=none,fillstyle=solid,fillcolor=curcolor]
{
\newpath
\moveto(96.56213698,1079.00857476)
\lineto(97.64497526,1079.00857476)
\lineto(97.64497526,1074.28092514)
\curveto(97.64496793,1073.45856079)(97.55194068,1072.80550949)(97.36589323,1072.32176928)
\curveto(97.17983167,1071.83802608)(96.8440033,1071.44452081)(96.35840709,1071.14125228)
\curveto(95.8727988,1070.83798313)(95.23556213,1070.68634871)(94.44669518,1070.68634857)
\curveto(93.68014649,1070.68634871)(93.05314282,1070.81844741)(92.56568228,1071.08264506)
\curveto(92.07821723,1071.34684219)(91.73029531,1071.72918419)(91.52191548,1072.22967221)
\curveto(91.31353323,1072.73015741)(91.20934271,1073.41390771)(91.20934361,1074.28092514)
\lineto(91.20934361,1079.00857476)
\lineto(92.29218189,1079.00857476)
\lineto(92.29218189,1074.28650678)
\curveto(92.29217991,1073.57577513)(92.35822926,1073.0520317)(92.49033013,1072.71527494)
\curveto(92.62242665,1072.37851441)(92.84941314,1072.11896838)(93.17129029,1071.93663607)
\curveto(93.49316172,1071.75430155)(93.88666699,1071.66313485)(94.35180729,1071.66313568)
\curveto(95.14811651,1071.66313485)(95.71558274,1071.84360771)(96.05420768,1072.20455483)
\curveto(96.39282113,1072.56549918)(96.56213073,1073.25948247)(96.56213698,1074.28650678)
\closepath
}
}
{
\newrgbcolor{curcolor}{0 0 0}
\pscustom[linestyle=none,fillstyle=solid,fillcolor=curcolor]
{
\newpath
\moveto(99.44226342,1070.82588959)
\lineto(99.44226342,1079.00857476)
\lineto(100.55300991,1079.00857476)
\lineto(104.8508732,1072.58410639)
\lineto(104.8508732,1079.00857476)
\lineto(105.88905836,1079.00857476)
\lineto(105.88905836,1070.82588959)
\lineto(104.77831187,1070.82588959)
\lineto(100.48044858,1077.2559396)
\lineto(100.48044858,1070.82588959)
\closepath
}
}
{
\newrgbcolor{curcolor}{0 0 0}
\pscustom[linestyle=none,fillstyle=solid,fillcolor=curcolor]
{
\newpath
\moveto(107.89844886,1070.82588959)
\lineto(107.89844886,1079.00857476)
\lineto(108.98128715,1079.00857476)
\lineto(108.98128715,1070.82588959)
\closepath
}
}
{
\newrgbcolor{curcolor}{0 0 0}
\pscustom[linestyle=none,fillstyle=solid,fillcolor=curcolor]
{
\newpath
\moveto(110.05296227,1070.82588959)
\lineto(113.21775252,1075.09026304)
\lineto(110.4269322,1079.00857476)
\lineto(111.71629118,1079.00857476)
\lineto(113.20100759,1076.90987788)
\curveto(113.50985487,1076.47450427)(113.72939918,1076.13960616)(113.85964119,1075.90518257)
\curveto(114.04197074,1076.20286469)(114.25779397,1076.51357571)(114.5071115,1076.83731656)
\lineto(116.15369549,1079.00857476)
\lineto(117.33142167,1079.00857476)
\lineto(114.45687674,1075.15166108)
\lineto(117.55468729,1070.82588959)
\lineto(116.21509354,1070.82588959)
\lineto(114.15546814,1073.74508764)
\curveto(114.0401102,1073.91253377)(113.92103532,1074.09486719)(113.79824314,1074.29208842)
\curveto(113.61590594,1073.99439775)(113.48566778,1073.7897378)(113.4075283,1073.67810795)
\lineto(111.35348454,1070.82588959)
\closepath
}
}
{
\newrgbcolor{curcolor}{0 0 0}
\pscustom[linestyle=none,fillstyle=solid,fillcolor=curcolor]
{
\newpath
\moveto(121.32229542,1073.45484233)
\lineto(122.34373566,1073.54414858)
\curveto(122.3921083,1073.13482596)(122.50467127,1072.79899758)(122.68142492,1072.53666244)
\curveto(122.85817482,1072.27432388)(123.13260521,1072.06222175)(123.50471691,1071.90035541)
\curveto(123.87682322,1071.73848692)(124.29544585,1071.65755321)(124.76058606,1071.65755404)
\curveto(125.1736231,1071.65755321)(125.53828992,1071.7189512)(125.85458762,1071.84174818)
\curveto(126.17087523,1071.96454314)(126.40623417,1072.13292246)(126.56066516,1072.34688666)
\curveto(126.71508465,1072.56084782)(126.79229727,1072.79434622)(126.79230325,1073.04738256)
\curveto(126.79229727,1073.30413555)(126.71787546,1073.52833123)(126.56903762,1073.71997026)
\curveto(126.42018826,1073.9116035)(126.17459632,1074.07254065)(125.83226106,1074.20278217)
\curveto(125.61271172,1074.28836387)(125.12710947,1074.42139284)(124.37545285,1074.60186948)
\curveto(123.6237891,1074.78233857)(123.09725486,1074.95257844)(122.79584855,1075.1125896)
\curveto(122.40513211,1075.31724526)(122.11395682,1075.57120966)(121.92232179,1075.87448355)
\curveto(121.73068454,1076.17774733)(121.63486647,1076.5172968)(121.6348673,1076.89313296)
\curveto(121.63486647,1077.30616789)(121.75208081,1077.69223098)(121.98651066,1078.05132339)
\curveto(122.22093815,1078.41040136)(122.56327844,1078.6829712)(123.01353254,1078.86903375)
\curveto(123.46378223,1079.05508021)(123.96426884,1079.14810746)(124.51499387,1079.14811578)
\curveto(125.12152784,1079.14810746)(125.65643453,1079.05042884)(126.11971555,1078.85507965)
\curveto(126.58298595,1078.65971439)(126.93928032,1078.37226018)(127.18859973,1077.99271617)
\curveto(127.43790639,1077.61315782)(127.57186563,1077.18337192)(127.59047786,1076.70335718)
\lineto(126.5522927,1076.62521421)
\curveto(126.49647061,1077.14243993)(126.30762529,1077.53315438)(125.98575618,1077.79735875)
\curveto(125.66387671,1078.06154917)(125.18850746,1078.19364786)(124.55964699,1078.19365523)
\curveto(123.9047314,1078.19364786)(123.4275016,1078.07364271)(123.12795617,1077.83363941)
\curveto(122.8284061,1077.59362209)(122.67863223,1077.30430734)(122.6786341,1076.96569429)
\curveto(122.67863223,1076.67172204)(122.78468329,1076.42985118)(122.99678762,1076.24008101)
\curveto(123.20516647,1076.0503)(123.74937589,1075.85587305)(124.6294175,1075.65679956)
\curveto(125.50945148,1075.45771641)(126.11319833,1075.28375545)(126.44065989,1075.13491616)
\curveto(126.91695378,1074.91536754)(127.26859679,1074.63721606)(127.49558997,1074.30046089)
\curveto(127.72256978,1073.96369876)(127.83606302,1073.57577513)(127.83607005,1073.13668881)
\curveto(127.83606302,1072.70131897)(127.71140651,1072.29106879)(127.46210013,1071.90593705)
\curveto(127.21278044,1071.52080315)(126.85462552,1071.2212554)(126.3876343,1071.00729291)
\curveto(125.92063192,1070.79333005)(125.39502796,1070.68634871)(124.81082082,1070.68634857)
\curveto(124.0703199,1070.68634871)(123.44982814,1070.79426032)(122.94934367,1071.01008373)
\curveto(122.44885492,1071.22590677)(122.05627992,1071.55057187)(121.77161749,1071.98408002)
\curveto(121.48695314,1072.41758585)(121.33717927,1072.90783946)(121.32229542,1073.45484233)
\closepath
}
}
{
\newrgbcolor{curcolor}{0 0 0}
\pscustom[linestyle=none,fillstyle=solid,fillcolor=curcolor]
{
\newpath
\moveto(129.15333781,1068.54299857)
\lineto(129.041705,1069.48629583)
\curveto(129.26124872,1069.42675973)(129.45288485,1069.39699101)(129.61661399,1069.39698958)
\curveto(129.83987822,1069.39699101)(130.01849054,1069.43420191)(130.15245149,1069.5086224)
\curveto(130.28640902,1069.58304552)(130.39618118,1069.68723604)(130.48176828,1069.82119427)
\curveto(130.54502478,1069.92166471)(130.64735475,1070.17097774)(130.78875852,1070.56913412)
\curveto(130.80736163,1070.62495073)(130.83713035,1070.70681471)(130.87806477,1070.81472631)
\lineto(128.62866359,1076.75359195)
\lineto(129.71150188,1076.75359195)
\lineto(130.94504446,1073.32088295)
\curveto(131.10504883,1072.88551292)(131.2483108,1072.42781885)(131.37483079,1071.94779936)
\curveto(131.49018165,1072.4092134)(131.62786198,1072.85946529)(131.78787219,1073.29855639)
\lineto(133.05490462,1076.75359195)
\lineto(134.05959993,1076.75359195)
\lineto(131.80461712,1070.72542006)
\curveto(131.5627429,1070.0742294)(131.37482786,1069.62583805)(131.24087141,1069.38024466)
\curveto(131.06225629,1069.04906909)(130.85759634,1068.80626797)(130.62689094,1068.65184056)
\curveto(130.39618118,1068.4974175)(130.12082051,1068.42020488)(129.80080813,1068.42020247)
\curveto(129.60731009,1068.42020488)(129.39148687,1068.46113687)(129.15333781,1068.54299857)
\closepath
}
}
{
\newrgbcolor{curcolor}{0 0 0}
\pscustom[linestyle=none,fillstyle=solid,fillcolor=curcolor]
{
\newpath
\moveto(134.51171187,1072.59526967)
\lineto(135.50524391,1072.75155561)
\curveto(135.56105891,1072.35339705)(135.71641442,1072.04826767)(135.9713109,1071.83616654)
\curveto(136.22620376,1071.6240634)(136.58249813,1071.51801233)(137.04019508,1071.51801303)
\curveto(137.50160737,1071.51801233)(137.84394765,1071.61196986)(138.06721696,1071.79988588)
\curveto(138.29047845,1071.98779995)(138.40211116,1072.20827454)(138.4021154,1072.46131029)
\curveto(138.40211116,1072.68829515)(138.30350227,1072.86690747)(138.10628844,1072.9971478)
\curveto(137.96860417,1073.08645179)(137.62626388,1073.19994503)(137.07926657,1073.33762787)
\curveto(136.34248782,1073.52367987)(135.83176821,1073.68461701)(135.54710621,1073.82043979)
\curveto(135.26244144,1073.95625658)(135.04661821,1074.14417163)(134.8996359,1074.38418549)
\curveto(134.7526521,1074.62419224)(134.67916057,1074.88931991)(134.67916109,1075.17956929)
\curveto(134.67916057,1075.44376232)(134.73962829,1075.68842399)(134.86056441,1075.91355503)
\curveto(134.98149914,1076.13867589)(135.14615737,1076.32566066)(135.35453961,1076.47450991)
\curveto(135.5108242,1076.58985806)(135.7238566,1076.68753667)(135.99363746,1076.76754605)
\curveto(136.26341466,1076.84754354)(136.55272941,1076.88754526)(136.86158258,1076.88755132)
\curveto(137.32671613,1076.88754526)(137.73510577,1076.82056564)(138.0867527,1076.68661226)
\curveto(138.43839178,1076.55264716)(138.69793781,1076.37124402)(138.86539157,1076.1424023)
\curveto(139.03283592,1075.91354994)(139.14818971,1075.60749029)(139.21145329,1075.22422241)
\lineto(138.22908454,1075.09026304)
\curveto(138.18442739,1075.39538815)(138.05511951,1075.63353792)(137.84116051,1075.80471304)
\curveto(137.62719415,1075.9758782)(137.32485559,1076.06146327)(136.93414391,1076.06146851)
\curveto(136.47272597,1076.06146327)(136.1434095,1075.98518093)(135.94619352,1075.83262124)
\curveto(135.74897396,1075.68005154)(135.65036507,1075.50143922)(135.65036656,1075.29678374)
\curveto(135.65036507,1075.16654112)(135.69129706,1075.04932678)(135.77316266,1074.94514038)
\curveto(135.85502502,1074.83722465)(135.98340263,1074.74791849)(136.15829586,1074.67722163)
\curveto(136.25876329,1074.64000688)(136.55458995,1074.55442181)(137.04577672,1074.42046616)
\curveto(137.75650203,1074.23068697)(138.25233728,1074.07533146)(138.53328395,1073.95439917)
\curveto(138.81422188,1073.83346061)(139.03469646,1073.65763911)(139.19470837,1073.42693413)
\curveto(139.3547102,1073.19622394)(139.43471364,1072.90970001)(139.43471892,1072.56736147)
\curveto(139.43471364,1072.23246162)(139.33703503,1071.91709924)(139.14168278,1071.62127338)
\curveto(138.94632057,1071.32544593)(138.664448,1071.09659889)(138.29606423,1070.93473158)
\curveto(137.92767218,1070.77286405)(137.51091009,1070.69193035)(137.04577672,1070.69193021)
\curveto(136.2755082,1070.69193035)(135.68850625,1070.85193722)(135.2847691,1071.17195131)
\curveto(134.88102971,1071.4919647)(134.62334422,1071.96640368)(134.51171187,1072.59526967)
\closepath
}
}
{
\newrgbcolor{curcolor}{0 0 0}
\pscustom[linestyle=none,fillstyle=solid,fillcolor=curcolor]
{
\newpath
\moveto(142.82277383,1071.72453373)
\lineto(142.96789649,1070.83705287)
\curveto(142.68509055,1070.77751542)(142.43205643,1070.7477467)(142.20879336,1070.74774662)
\curveto(141.8441242,1070.7477467)(141.56132136,1070.80542359)(141.36038398,1070.92077748)
\curveto(141.15944364,1071.03613117)(141.01804221,1071.18776559)(140.93617929,1071.37568119)
\curveto(140.85431425,1071.56359569)(140.81338226,1071.9589615)(140.8133832,1072.56177983)
\lineto(140.8133832,1075.97216226)
\lineto(140.07660664,1075.97216226)
\lineto(140.07660664,1076.75359195)
\lineto(140.8133832,1076.75359195)
\lineto(140.8133832,1078.22156343)
\lineto(141.81249687,1078.82438062)
\lineto(141.81249687,1076.75359195)
\lineto(142.82277383,1076.75359195)
\lineto(142.82277383,1075.97216226)
\lineto(141.81249687,1075.97216226)
\lineto(141.81249687,1072.50596342)
\curveto(141.81249494,1072.21943781)(141.83017011,1072.03524385)(141.86552246,1071.953381)
\curveto(141.90087083,1071.87151589)(141.95854772,1071.80639681)(142.03855332,1071.75802357)
\curveto(142.11855459,1071.70964847)(142.23297811,1071.68546139)(142.38182422,1071.68546225)
\curveto(142.49345441,1071.68546139)(142.64043747,1071.6984852)(142.82277383,1071.72453373)
\closepath
}
}
{
\newrgbcolor{curcolor}{0 0 0}
\pscustom[linestyle=none,fillstyle=solid,fillcolor=curcolor]
{
\newpath
\moveto(147.85741381,1072.73481069)
\lineto(148.89559897,1072.60643295)
\curveto(148.73186516,1071.99989349)(148.42859632,1071.5291756)(147.98579155,1071.19427787)
\curveto(147.54297689,1070.8593794)(146.97737121,1070.69193035)(146.28897279,1070.69193021)
\curveto(145.42195557,1070.69193035)(144.73448419,1070.95891856)(144.22655658,1071.49289564)
\curveto(143.71862661,1072.0268714)(143.46466221,1072.77574077)(143.46466263,1073.739506)
\curveto(143.46466221,1074.73675522)(143.72141742,1075.51074195)(144.23492904,1076.06146851)
\curveto(144.74843827,1076.6121846)(145.41451339,1076.88754526)(146.23315639,1076.88755132)
\curveto(147.02574538,1076.88754526)(147.67321504,1076.61776623)(148.17556733,1076.07821343)
\curveto(148.67790935,1075.53865012)(148.92908293,1074.77954775)(148.92908881,1073.80090405)
\curveto(148.92908293,1073.74136363)(148.92722239,1073.65205747)(148.92350717,1073.5329853)
\lineto(144.50284779,1073.5329853)
\curveto(144.54005723,1072.88179183)(144.72425119,1072.38316577)(145.05543021,1072.03710561)
\curveto(145.38660522,1071.69104302)(145.79964621,1071.51801233)(146.29455443,1071.51801303)
\curveto(146.6629391,1071.51801233)(146.97737121,1071.61476068)(147.2378517,1071.80825834)
\curveto(147.49832381,1072.00175404)(147.70484431,1072.31060451)(147.85741381,1072.73481069)
\closepath
\moveto(144.55866419,1074.35906811)
\lineto(147.86857709,1074.35906811)
\curveto(147.82391919,1074.85769064)(147.69740213,1075.23166019)(147.48902553,1075.48097788)
\curveto(147.16900734,1075.86796659)(146.7541058,1076.06146327)(146.24431967,1076.06146851)
\curveto(145.7829013,1076.06146327)(145.39497767,1075.90703803)(145.08054759,1075.59819233)
\curveto(144.76611345,1075.28933709)(144.59215249,1074.87629609)(144.55866419,1074.35906811)
\closepath
}
}
{
\newrgbcolor{curcolor}{0 0 0}
\pscustom[linestyle=none,fillstyle=solid,fillcolor=curcolor]
{
\newpath
\moveto(150.16263251,1070.82588959)
\lineto(150.16263251,1076.75359195)
\lineto(151.06127665,1076.75359195)
\lineto(151.06127665,1075.92192749)
\curveto(151.2473295,1076.21216742)(151.49478199,1076.44566582)(151.80363486,1076.62242339)
\curveto(152.11248294,1076.79916937)(152.46412594,1076.88754526)(152.85856494,1076.88755132)
\curveto(153.29765011,1076.88754526)(153.65766557,1076.79637855)(153.9386124,1076.61405093)
\curveto(154.21955017,1076.43171173)(154.41769821,1076.17681706)(154.53305713,1075.84936616)
\curveto(155.00190935,1076.54148389)(155.61216812,1076.88754526)(156.36383526,1076.88755132)
\curveto(156.95176053,1076.88754526)(157.40387297,1076.72474757)(157.72017393,1076.39915777)
\curveto(158.03645828,1076.07355681)(158.1946046,1075.57213993)(158.19461339,1074.89490561)
\lineto(158.19461339,1070.82588959)
\lineto(157.19549971,1070.82588959)
\lineto(157.19549971,1074.56000717)
\curveto(157.19549193,1074.96188116)(157.16293239,1075.25119592)(157.097821,1075.42795229)
\curveto(157.03269424,1075.60469947)(156.91454963,1075.74703116)(156.74338682,1075.8549478)
\curveto(156.57220935,1075.96285438)(156.37127048,1076.01681019)(156.14056963,1076.01681538)
\curveto(155.72380082,1076.01681019)(155.37773944,1075.87819959)(155.10238447,1075.60098315)
\curveto(154.82701812,1075.32375717)(154.68933779,1074.88001718)(154.68934307,1074.26976186)
\lineto(154.68934307,1070.82588959)
\lineto(153.68464775,1070.82588959)
\lineto(153.68464775,1074.67722163)
\curveto(153.68464348,1075.12374858)(153.6027795,1075.45864668)(153.43905556,1075.68191694)
\curveto(153.27532357,1075.90517749)(153.00740509,1076.01681019)(152.63529931,1076.01681538)
\curveto(152.35249324,1076.01681019)(152.09108667,1075.94238839)(151.8510788,1075.79354976)
\curveto(151.61106605,1075.64470119)(151.43710509,1075.42701742)(151.3291954,1075.1404978)
\curveto(151.22128187,1074.85396955)(151.16732607,1074.44092856)(151.16732782,1073.90137358)
\lineto(151.16732782,1070.82588959)
\closepath
}
}
{
\newrgbcolor{curcolor}{0 0 0}
\pscustom[linestyle=none,fillstyle=solid,fillcolor=curcolor]
{
\newpath
\moveto(165.33352932,1070.82588959)
\lineto(162.16315743,1079.00857476)
\lineto(163.33530197,1079.00857476)
\lineto(165.46190705,1073.06412748)
\curveto(165.63307384,1072.58782572)(165.77633581,1072.14129492)(165.89169338,1071.72453373)
\curveto(166.01820666,1072.17106364)(166.16518972,1072.61759444)(166.33264299,1073.06412748)
\lineto(168.54297268,1079.00857476)
\lineto(169.64813753,1079.00857476)
\lineto(166.4442758,1070.82588959)
\closepath
}
}
{
\newrgbcolor{curcolor}{0 0 0}
\pscustom[linestyle=none,fillstyle=solid,fillcolor=curcolor]
{
\newpath
\moveto(170.78679661,1070.82588959)
\lineto(170.78679661,1071.97012592)
\lineto(171.93103294,1071.97012592)
\lineto(171.93103294,1070.82588959)
\closepath
}
}
{
\newrgbcolor{curcolor}{0 0 0}
\pscustom[linestyle=none,fillstyle=solid,fillcolor=curcolor]
{
\newpath
\moveto(173.39342291,1074.86141577)
\curveto(173.39342244,1075.82889514)(173.4929616,1076.60753323)(173.69204069,1077.19733238)
\curveto(173.89111823,1077.78711878)(174.18694489,1078.24202203)(174.57952155,1078.56204351)
\curveto(174.97209489,1078.88204952)(175.46606959,1079.04205639)(176.06144714,1079.04206461)
\curveto(176.50053262,1079.04205639)(176.88566544,1078.9536805)(177.21684675,1078.77693668)
\curveto(177.54801947,1078.60017695)(177.82151958,1078.34528228)(178.03734792,1078.01225191)
\curveto(178.25316603,1077.67920716)(178.42247563,1077.27360835)(178.54527722,1076.79545425)
\curveto(178.66806757,1076.31728821)(178.72946555,1075.67260936)(178.72947136,1074.86141577)
\curveto(178.72946555,1073.9013705)(178.63085667,1073.1264535)(178.43364441,1072.53666244)
\curveto(178.23642112,1071.94686796)(177.94152474,1071.49103443)(177.54895437,1071.16916049)
\curveto(177.15637474,1070.84728586)(176.66053949,1070.68634871)(176.06144714,1070.68634857)
\curveto(175.27257291,1070.68634871)(174.65301142,1070.96915155)(174.20276081,1071.53475795)
\curveto(173.66320147,1072.21571672)(173.39342244,1073.32460155)(173.39342291,1074.86141577)
\closepath
\moveto(174.42602643,1074.86141577)
\curveto(174.42602492,1073.51809823)(174.58324098,1072.62410635)(174.89767507,1072.17943744)
\curveto(175.2121052,1071.73476583)(175.60002883,1071.5124307)(176.06144714,1071.51243139)
\curveto(176.52285916,1071.5124307)(176.9107828,1071.7356961)(177.22521921,1072.18222826)
\curveto(177.53964701,1072.62875771)(177.69686307,1073.52181932)(177.69686785,1074.86141577)
\curveto(177.69686307,1076.20844633)(177.53964701,1077.10336848)(177.22521921,1077.54618492)
\curveto(176.9107828,1077.98898791)(176.51913807,1078.21039277)(176.05028386,1078.21040015)
\curveto(175.58886556,1078.21039277)(175.22047765,1078.01503554)(174.94511901,1077.62432789)
\curveto(174.59905561,1077.12569502)(174.42602492,1076.20472524)(174.42602643,1074.86141577)
\closepath
}
}
{
\newrgbcolor{curcolor}{0 0 0}
\pscustom[linestyle=none,fillstyle=solid,fillcolor=curcolor]
{
\newpath
\moveto(949.51693497,1072.92736176)
\lineto(949.51693497,1073.88740395)
\lineto(952.98313381,1073.89298559)
\lineto(952.98313381,1070.85657309)
\curveto(952.45100976,1070.43236768)(951.90214898,1070.11328421)(951.33654982,1069.89932172)
\curveto(950.7709376,1069.68535886)(950.19044756,1069.57837752)(949.59507794,1069.57837738)
\curveto(948.7913177,1069.57837752)(948.06105378,1069.75047793)(947.40428399,1070.09467914)
\curveto(946.747509,1070.43887959)(946.25167375,1070.93657538)(945.91677676,1071.58776801)
\curveto(945.58187755,1072.2389569)(945.41442849,1072.96643)(945.4144291,1073.7701895)
\curveto(945.41442849,1074.56649872)(945.58094727,1075.30978645)(945.91398594,1076.00005493)
\curveto(946.24702239,1076.69031085)(946.72611273,1077.20289101)(947.3512584,1077.53779693)
\curveto(947.97639899,1077.87268721)(948.69642991,1078.04013627)(949.51135333,1078.04014459)
\curveto(950.10300194,1078.04013627)(950.63790864,1077.9443182)(951.11607502,1077.7526901)
\curveto(951.59422877,1077.56104592)(951.9691286,1077.29405771)(952.2407756,1076.95172466)
\curveto(952.51240774,1076.60937715)(952.71892824,1076.16284634)(952.86033771,1075.61213091)
\lineto(951.8835506,1075.34421216)
\curveto(951.76074755,1075.76096862)(951.60818286,1076.08842454)(951.42585607,1076.32658091)
\curveto(951.24351604,1076.56472407)(950.98303974,1076.75542993)(950.64442638,1076.89869908)
\curveto(950.30580135,1077.04195386)(949.92997126,1077.11358485)(949.51693497,1077.11359224)
\curveto(949.02202529,1077.11358485)(948.59409993,1077.03823277)(948.23315763,1076.8875358)
\curveto(947.87220846,1076.73682448)(947.58103317,1076.53867644)(947.35963087,1076.29309107)
\curveto(947.13822345,1076.04749255)(946.96612304,1075.77771352)(946.84332911,1075.48375318)
\curveto(946.63494603,1074.97767916)(946.53075551,1074.42881838)(946.53075723,1073.83716919)
\curveto(946.53075551,1073.10783142)(946.65634229,1072.49757265)(946.90751797,1072.00639106)
\curveto(947.15868945,1071.51520488)(947.52428655,1071.15053806)(948.00431036,1070.91238949)
\curveto(948.48432778,1070.67423854)(948.99411711,1070.55516365)(949.53367989,1070.55516449)
\curveto(950.00253251,1070.55516365)(950.46022659,1070.64540009)(950.90676349,1070.82587406)
\curveto(951.3532882,1071.00634582)(951.69190739,1071.19891223)(951.92262209,1071.40357387)
\lineto(951.92262209,1072.92736176)
\closepath
}
}
{
\newrgbcolor{curcolor}{0 0 0}
\pscustom[linestyle=none,fillstyle=solid,fillcolor=curcolor]
{
\newpath
\moveto(954.56273824,1069.7179184)
\lineto(954.56273824,1077.90060357)
\lineto(955.67348472,1077.90060357)
\lineto(959.97134801,1071.4761352)
\lineto(959.97134801,1077.90060357)
\lineto(961.00953317,1077.90060357)
\lineto(961.00953317,1069.7179184)
\lineto(959.89878669,1069.7179184)
\lineto(955.60092339,1076.14796841)
\lineto(955.60092339,1069.7179184)
\closepath
}
}
{
\newrgbcolor{curcolor}{0 0 0}
\pscustom[linestyle=none,fillstyle=solid,fillcolor=curcolor]
{
\newpath
\moveto(968.20426878,1077.90060357)
\lineto(969.28710707,1077.90060357)
\lineto(969.28710707,1073.17295395)
\curveto(969.28709973,1072.3505896)(969.19407248,1071.6975383)(969.00802504,1071.21379809)
\curveto(968.82196348,1070.73005489)(968.4861351,1070.33654962)(968.0005389,1070.03328109)
\curveto(967.5149306,1069.73001194)(966.87769393,1069.57837752)(966.08882698,1069.57837738)
\curveto(965.3222783,1069.57837752)(964.69527463,1069.71047622)(964.20781409,1069.97467387)
\curveto(963.72034904,1070.238871)(963.37242712,1070.621213)(963.16404729,1071.12170102)
\curveto(962.95566503,1071.62218622)(962.85147451,1072.30593652)(962.85147541,1073.17295395)
\lineto(962.85147541,1077.90060357)
\lineto(963.93431369,1077.90060357)
\lineto(963.93431369,1073.17853559)
\curveto(963.93431171,1072.46780393)(964.00036106,1071.94406051)(964.13246194,1071.60730375)
\curveto(964.26455845,1071.27054321)(964.49154495,1071.01099718)(964.8134221,1070.82866488)
\curveto(965.13529352,1070.64633036)(965.52879879,1070.55516365)(965.99393909,1070.55516449)
\curveto(966.79024832,1070.55516365)(967.35771455,1070.73563652)(967.69633949,1071.09658363)
\curveto(968.03495294,1071.45752799)(968.20426253,1072.15151128)(968.20426878,1073.17853559)
\closepath
}
}
{
\newrgbcolor{curcolor}{0 0 0}
\pscustom[linestyle=none,fillstyle=solid,fillcolor=curcolor]
{
\newpath
\moveto(105.18590008,1464.46190643)
\lineto(105.18590008,1472.64459161)
\lineto(108.27254735,1472.64459161)
\curveto(108.81582253,1472.64458343)(109.23072406,1472.6185358)(109.51725321,1472.56644864)
\curveto(109.91912572,1472.49946092)(110.25588437,1472.37201358)(110.52753017,1472.18410626)
\curveto(110.79916352,1471.99618349)(111.01777756,1471.73291637)(111.18337294,1471.39430411)
\curveto(111.34895457,1471.05567798)(111.43174882,1470.68356898)(111.43175595,1470.27797598)
\curveto(111.43174882,1469.58212633)(111.21034396,1468.99326383)(110.76754071,1468.51138672)
\curveto(110.32472454,1468.02950151)(109.52469018,1467.78856093)(108.36743524,1467.78856425)
\lineto(106.26873836,1467.78856425)
\lineto(106.26873836,1464.46190643)
\closepath
\moveto(106.26873836,1468.75418808)
\lineto(108.38418016,1468.75418808)
\curveto(109.08374101,1468.75418379)(109.58050653,1468.88442194)(109.87447821,1469.14490293)
\curveto(110.16843875,1469.40537455)(110.31542181,1469.77190192)(110.31542782,1470.24448613)
\curveto(110.31542181,1470.58682064)(110.22890647,1470.87985648)(110.05588153,1471.12359454)
\curveto(109.88284509,1471.36731927)(109.65492833,1471.52825642)(109.37213055,1471.60640645)
\curveto(109.18979207,1471.65477348)(108.85303343,1471.67896056)(108.3618536,1471.67896778)
\lineto(106.26873836,1471.67896778)
\closepath
}
}
{
\newrgbcolor{curcolor}{0 0 0}
\pscustom[linestyle=none,fillstyle=solid,fillcolor=curcolor]
{
\newpath
\moveto(114.25048457,1464.46190643)
\lineto(112.07922636,1472.64459161)
\lineto(113.18997285,1472.64459161)
\lineto(114.43467871,1467.28063496)
\curveto(114.56863546,1466.71874754)(114.68398925,1466.16058404)(114.78074043,1465.60614276)
\curveto(114.98911863,1466.48059778)(115.1119146,1466.98480548)(115.14912871,1467.11876738)
\lineto(116.70640645,1472.64459161)
\lineto(118.01251036,1472.64459161)
\lineto(119.18465489,1468.50301426)
\curveto(119.47861376,1467.47598936)(119.69071589,1466.5103665)(119.82096193,1465.60614276)
\curveto(119.92514457,1466.12337314)(120.06096435,1466.716887)(120.22842169,1467.38668613)
\lineto(121.51219904,1472.64459161)
\lineto(122.60061896,1472.64459161)
\lineto(120.35679943,1464.46190643)
\lineto(119.31303263,1464.46190643)
\lineto(117.58830567,1470.69659903)
\curveto(117.44317751,1471.2175454)(117.35759244,1471.53755914)(117.3315502,1471.65664122)
\curveto(117.24595974,1471.28080393)(117.1659563,1470.96079018)(117.09153965,1470.69659903)
\lineto(115.35564942,1464.46190643)
\closepath
}
}
{
\newrgbcolor{curcolor}{0 0 0}
\pscustom[linestyle=none,fillstyle=solid,fillcolor=curcolor]
{
\newpath
\moveto(123.56066147,1464.46190643)
\lineto(123.56066147,1472.64459161)
\lineto(126.63056382,1472.64459161)
\curveto(127.25570304,1472.64458343)(127.75711992,1472.56178917)(128.13481597,1472.3962086)
\curveto(128.5125012,1472.23061216)(128.80832786,1471.97571749)(129.02229684,1471.63152383)
\curveto(129.23625321,1471.28731584)(129.34323455,1470.92730037)(129.34324117,1470.55147637)
\curveto(129.34323455,1470.20168782)(129.24834676,1469.87237135)(129.0585775,1469.56352598)
\curveto(128.86879557,1469.2546704)(128.58227164,1469.00535737)(128.19900484,1468.81558613)
\curveto(128.69390434,1468.67045927)(129.0743858,1468.42300678)(129.34045035,1468.07322793)
\curveto(129.60650167,1467.72344185)(129.73953064,1467.31040086)(129.73953766,1466.8341037)
\curveto(129.73953064,1466.45082906)(129.65859693,1466.09453469)(129.49673629,1465.76521952)
\curveto(129.3348621,1465.43590175)(129.13485351,1465.18193736)(128.89670992,1465.00332558)
\curveto(128.65855399,1464.82471271)(128.35993651,1464.6898232)(128.0008566,1464.59865663)
\curveto(127.64176613,1464.50748979)(127.20174724,1464.46190643)(126.68079859,1464.46190643)
\closepath
\moveto(124.64349975,1469.20630098)
\lineto(126.41287984,1469.20630098)
\curveto(126.89289676,1469.20629623)(127.23709759,1469.2379255)(127.44548335,1469.30118887)
\curveto(127.7208393,1469.38304801)(127.92829007,1469.51886779)(128.06783629,1469.70864863)
\curveto(128.20737182,1469.89841898)(128.27714226,1470.13656874)(128.27714781,1470.42309864)
\curveto(128.27714226,1470.69473225)(128.21202318,1470.93381228)(128.08179039,1471.14033946)
\curveto(127.95154688,1471.34685328)(127.76549238,1471.4882547)(127.52362632,1471.56454415)
\curveto(127.28175067,1471.64081939)(126.86684913,1471.67896056)(126.27892046,1471.67896778)
\lineto(124.64349975,1471.67896778)
\closepath
\moveto(124.64349975,1465.42753026)
\lineto(126.68079859,1465.42753026)
\curveto(127.03057709,1465.4275293)(127.27616904,1465.44055311)(127.41757515,1465.46660175)
\curveto(127.66688349,1465.51125382)(127.87526453,1465.58567562)(128.0427189,1465.68986737)
\curveto(128.21016264,1465.79405667)(128.34784297,1465.94569109)(128.45576031,1466.14477109)
\curveto(128.56366619,1466.34384772)(128.61762199,1466.57362503)(128.61762789,1466.8341037)
\curveto(128.61762199,1467.13923072)(128.5394791,1467.40435838)(128.38319898,1467.6294875)
\curveto(128.22690754,1467.85461028)(128.01015405,1468.0127566)(127.73293785,1468.10392695)
\curveto(127.45571163,1468.19509001)(127.05662472,1468.24067337)(126.53567593,1468.24067715)
\lineto(124.64349975,1468.24067715)
\closepath
}
}
{
\newrgbcolor{curcolor}{0 0 0}
\pscustom[linestyle=none,fillstyle=solid,fillcolor=curcolor]
{
\newpath
\moveto(130.35910034,1464.32236542)
\lineto(132.73129761,1472.78413263)
\lineto(133.53505387,1472.78413263)
\lineto(131.16843823,1464.32236542)
\closepath
}
}
{
\newrgbcolor{curcolor}{0 0 0}
\pscustom[linestyle=none,fillstyle=solid,fillcolor=curcolor]
{
\newpath
\moveto(139.78090796,1472.64459161)
\lineto(140.86374624,1472.64459161)
\lineto(140.86374624,1467.91694199)
\curveto(140.86373891,1467.09457764)(140.77071166,1466.44152633)(140.58466421,1465.95778612)
\curveto(140.39860265,1465.47404292)(140.06277428,1465.08053765)(139.57717808,1464.77726913)
\curveto(139.09156978,1464.47399998)(138.45433311,1464.32236556)(137.66546616,1464.32236542)
\curveto(136.89891747,1464.32236556)(136.2719138,1464.45446425)(135.78445326,1464.7186619)
\curveto(135.29698821,1464.98285904)(134.94906629,1465.36520104)(134.74068646,1465.86568905)
\curveto(134.53230421,1466.36617426)(134.42811369,1467.04992455)(134.42811459,1467.91694199)
\lineto(134.42811459,1472.64459161)
\lineto(135.51095287,1472.64459161)
\lineto(135.51095287,1467.92252363)
\curveto(135.51095089,1467.21179197)(135.57700024,1466.68804855)(135.70910111,1466.35129179)
\curveto(135.84119763,1466.01453125)(136.06818412,1465.75498522)(136.39006127,1465.57265292)
\curveto(136.7119327,1465.3903184)(137.10543797,1465.29915169)(137.57057827,1465.29915253)
\curveto(138.36688749,1465.29915169)(138.93435372,1465.47962456)(139.27297866,1465.84057167)
\curveto(139.61159211,1466.20151603)(139.78090171,1466.89549932)(139.78090796,1467.92252363)
\closepath
}
}
{
\newrgbcolor{curcolor}{0 0 0}
\pscustom[linestyle=none,fillstyle=solid,fillcolor=curcolor]
{
\newpath
\moveto(142.6610344,1464.46190643)
\lineto(142.6610344,1472.64459161)
\lineto(143.77178089,1472.64459161)
\lineto(148.06964418,1466.22012323)
\lineto(148.06964418,1472.64459161)
\lineto(149.10782934,1472.64459161)
\lineto(149.10782934,1464.46190643)
\lineto(147.99708285,1464.46190643)
\lineto(143.69921956,1470.89195645)
\lineto(143.69921956,1464.46190643)
\closepath
}
}
{
\newrgbcolor{curcolor}{0 0 0}
\pscustom[linestyle=none,fillstyle=solid,fillcolor=curcolor]
{
\newpath
\moveto(151.11722175,1464.46190643)
\lineto(151.11722175,1472.64459161)
\lineto(152.20006003,1472.64459161)
\lineto(152.20006003,1464.46190643)
\closepath
}
}
{
\newrgbcolor{curcolor}{0 0 0}
\pscustom[linestyle=none,fillstyle=solid,fillcolor=curcolor]
{
\newpath
\moveto(153.27173516,1464.46190643)
\lineto(156.4365254,1468.72627988)
\lineto(153.64570509,1472.64459161)
\lineto(154.93506407,1472.64459161)
\lineto(156.41978048,1470.54589473)
\curveto(156.72862776,1470.11052111)(156.94817207,1469.77562301)(157.07841408,1469.54119941)
\curveto(157.26074363,1469.83888154)(157.47656686,1470.14959256)(157.72588439,1470.4733334)
\lineto(159.37246838,1472.64459161)
\lineto(160.55019455,1472.64459161)
\lineto(157.67564963,1468.78767793)
\lineto(160.77346018,1464.46190643)
\lineto(159.43386643,1464.46190643)
\lineto(157.37424103,1467.38110449)
\curveto(157.25888309,1467.54855062)(157.13980821,1467.73088403)(157.01701603,1467.92810527)
\curveto(156.83467882,1467.6304146)(156.70444067,1467.42575465)(156.62630119,1467.3141248)
\lineto(154.57225743,1464.46190643)
\closepath
}
}
{
\newrgbcolor{curcolor}{0 0 0}
\pscustom[linestyle=none,fillstyle=solid,fillcolor=curcolor]
{
\newpath
\moveto(682.14552998,1117.27985454)
\lineto(683.16697022,1117.36916079)
\curveto(683.21534285,1116.95983816)(683.32790583,1116.62400979)(683.50465948,1116.36167465)
\curveto(683.68140938,1116.09933609)(683.95583977,1115.88723396)(684.32795147,1115.72536762)
\curveto(684.70005778,1115.56349913)(685.11868041,1115.48256542)(685.58382061,1115.48256625)
\curveto(685.99685766,1115.48256542)(686.36152448,1115.5439634)(686.67782218,1115.66676039)
\curveto(686.99410979,1115.78955535)(687.22946873,1115.95793467)(687.38389972,1116.17189887)
\curveto(687.53831921,1116.38586002)(687.61553182,1116.61935842)(687.61553781,1116.87239477)
\curveto(687.61553182,1117.12914776)(687.54111002,1117.35334343)(687.39227218,1117.54498247)
\curveto(687.24342282,1117.73661571)(686.99783088,1117.89755285)(686.65549562,1118.02779438)
\curveto(686.43594628,1118.11337607)(685.95034403,1118.24640504)(685.19868741,1118.42688169)
\curveto(684.44702366,1118.60735078)(683.92048942,1118.77759065)(683.61908311,1118.93760181)
\curveto(683.22836667,1119.14225747)(682.93719137,1119.39622187)(682.74555635,1119.69949575)
\curveto(682.5539191,1120.00275954)(682.45810103,1120.34230901)(682.45810186,1120.71814517)
\curveto(682.45810103,1121.1311801)(682.57531537,1121.51724319)(682.80974522,1121.8763356)
\curveto(683.04417271,1122.23541356)(683.386513,1122.50798341)(683.8367671,1122.69404595)
\curveto(684.28701678,1122.88009241)(684.7875034,1122.97311966)(685.33822843,1122.97312799)
\curveto(685.9447624,1122.97311966)(686.47966909,1122.87544105)(686.94295011,1122.68009185)
\curveto(687.40622051,1122.4847266)(687.76251488,1122.19727239)(688.01183429,1121.81772837)
\curveto(688.26114095,1121.43817002)(688.39510019,1121.00838412)(688.41371242,1120.52836939)
\lineto(687.37552726,1120.45022642)
\curveto(687.31970517,1120.96745213)(687.13085985,1121.35816659)(686.80899074,1121.62237095)
\curveto(686.48711127,1121.88656137)(686.01174202,1122.01866007)(685.38288155,1122.01866744)
\curveto(684.72796595,1122.01866007)(684.25073616,1121.89865492)(683.95119073,1121.65865162)
\curveto(683.65164066,1121.4186343)(683.50186679,1121.12931955)(683.50186866,1120.7907065)
\curveto(683.50186679,1120.49673424)(683.60791785,1120.25486339)(683.82002217,1120.06509321)
\curveto(684.02840103,1119.87531221)(684.57261045,1119.68088525)(685.45265206,1119.48181177)
\curveto(686.33268603,1119.28272862)(686.93643289,1119.10876766)(687.26389445,1118.95992837)
\curveto(687.74018834,1118.74037975)(688.09183135,1118.46222827)(688.31882453,1118.12547309)
\curveto(688.54580433,1117.78871097)(688.65929758,1117.40078733)(688.65930461,1116.96170102)
\curveto(688.65929758,1116.52633117)(688.53464106,1116.116081)(688.28533468,1115.73094926)
\curveto(688.036015,1115.34581536)(687.67786008,1115.04626761)(687.21086886,1114.83230512)
\curveto(686.74386648,1114.61834226)(686.21826251,1114.51136092)(685.63405538,1114.51136078)
\curveto(684.89355446,1114.51136092)(684.2730627,1114.61927253)(683.77257823,1114.83509594)
\curveto(683.27208948,1115.05091897)(682.87951448,1115.37558408)(682.59485205,1115.80909223)
\curveto(682.3101877,1116.24259806)(682.16041383,1116.73285167)(682.14552998,1117.27985454)
\closepath
}
}
{
\newrgbcolor{curcolor}{0 0 0}
\pscustom[linestyle=none,fillstyle=solid,fillcolor=curcolor]
{
\newpath
\moveto(693.9060469,1114.65090179)
\lineto(693.9060469,1115.52163773)
\curveto(693.4446271,1114.85184066)(692.81762343,1114.51694255)(692.02503401,1114.51694242)
\curveto(691.67524879,1114.51694255)(691.34872314,1114.58392217)(691.04545608,1114.71788148)
\curveto(690.74218546,1114.85184066)(690.51705951,1115.02021998)(690.37007756,1115.22301996)
\curveto(690.2230934,1115.4258188)(690.11983315,1115.67420156)(690.0602965,1115.96816899)
\curveto(690.01936372,1116.16538544)(689.99889773,1116.477957)(689.99889846,1116.90588461)
\lineto(689.99889846,1120.57860415)
\lineto(691.00359377,1120.57860415)
\lineto(691.00359377,1117.29101782)
\curveto(691.00359204,1116.76634148)(691.02405803,1116.41283793)(691.06499182,1116.2305061)
\curveto(691.12824855,1115.96630712)(691.26220779,1115.75885635)(691.46686994,1115.60815316)
\curveto(691.6715277,1115.45744806)(691.92456182,1115.38209599)(692.22597307,1115.38209672)
\curveto(692.52737841,1115.38209599)(692.81018125,1115.45930861)(693.07438245,1115.6137348)
\curveto(693.33857603,1115.76815908)(693.52556081,1115.97840067)(693.63533733,1116.2444602)
\curveto(693.74510512,1116.51051654)(693.7999912,1116.89657963)(693.79999573,1117.40265063)
\lineto(693.79999573,1120.57860415)
\lineto(694.80469105,1120.57860415)
\lineto(694.80469105,1114.65090179)
\closepath
}
}
{
\newrgbcolor{curcolor}{0 0 0}
\pscustom[linestyle=none,fillstyle=solid,fillcolor=curcolor]
{
\newpath
\moveto(696.38429503,1114.65090179)
\lineto(696.38429503,1120.57860415)
\lineto(697.28852081,1120.57860415)
\lineto(697.28852081,1119.73577642)
\curveto(697.72388669,1120.38696209)(698.35275091,1120.71255747)(699.17511535,1120.71256353)
\curveto(699.53233645,1120.71255747)(699.86072264,1120.64836866)(700.16027492,1120.51999693)
\curveto(700.45981814,1120.39161345)(700.68401382,1120.22323413)(700.83286262,1120.01485845)
\curveto(700.98170102,1119.80647204)(701.08589154,1119.55901955)(701.1454345,1119.27250024)
\curveto(701.18263988,1119.08644112)(701.20124533,1118.76084574)(701.2012509,1118.29571313)
\lineto(701.2012509,1114.65090179)
\lineto(700.19655559,1114.65090179)
\lineto(700.19655559,1118.25664165)
\curveto(700.19655102,1118.66595795)(700.15747958,1118.9720176)(700.07934113,1119.17482153)
\curveto(700.00119379,1119.37761642)(699.86258319,1119.53948383)(699.66350891,1119.66042427)
\curveto(699.46442656,1119.78135468)(699.23092816,1119.8418224)(698.96301301,1119.84182759)
\curveto(698.53508432,1119.8418224)(698.16576613,1119.70600261)(697.85505734,1119.43436782)
\curveto(697.5443441,1119.16272347)(697.38898859,1118.6473525)(697.38899035,1117.88825336)
\lineto(697.38899035,1114.65090179)
\closepath
}
}
{
\newrgbcolor{curcolor}{0 0 0}
\pscustom[linestyle=none,fillstyle=solid,fillcolor=curcolor]
{
\newpath
\moveto(702.54642741,1118.63619321)
\curveto(702.54642686,1119.99438709)(702.91109368,1121.05768857)(703.64042897,1121.82610084)
\curveto(704.36976098,1122.59449875)(705.31119675,1122.9787013)(706.46473914,1122.97870963)
\curveto(707.22011594,1122.9787013)(707.90107542,1122.79822843)(708.50761961,1122.43729049)
\curveto(709.11415077,1122.07633697)(709.57649621,1121.57305954)(709.89465731,1120.92745669)
\curveto(710.21280261,1120.28184129)(710.37187921,1119.54971683)(710.37188758,1118.7310811)
\curveto(710.37187921,1117.90127394)(710.20443015,1117.15891648)(709.86953993,1116.50400649)
\curveto(709.53463395,1115.84909279)(709.06019497,1115.35325754)(708.44622156,1115.01649926)
\curveto(707.83223526,1114.67974024)(707.16988123,1114.51136092)(706.4591575,1114.51136078)
\curveto(705.68888739,1114.51136092)(705.00048574,1114.69741542)(704.39395046,1115.06952484)
\curveto(703.78741038,1115.44163343)(703.32785576,1115.94956222)(703.01528522,1116.59331274)
\curveto(702.70271264,1117.23705937)(702.54642686,1117.91801885)(702.54642741,1118.63619321)
\closepath
\moveto(703.66275554,1118.61944829)
\curveto(703.66275387,1117.63335546)(703.92788153,1116.85657791)(704.45813933,1116.28911332)
\curveto(704.98839219,1115.72164545)(705.65353704,1115.43791234)(706.45357586,1115.43791312)
\curveto(707.26849011,1115.43791234)(707.93921659,1115.72443627)(708.46575731,1116.29748578)
\curveto(708.99228507,1116.870532)(709.25555219,1117.68359018)(709.25555946,1118.73666274)
\curveto(709.25555219,1119.40273377)(709.14298922,1119.98415409)(708.9178702,1120.48092544)
\curveto(708.69273733,1120.97768513)(708.36342086,1121.36281795)(707.9299198,1121.63632505)
\curveto(707.49640688,1121.90981819)(707.00987436,1122.04656824)(706.47032078,1122.04657564)
\curveto(705.70377175,1122.04656824)(705.04420854,1121.78330112)(704.49162917,1121.25677349)
\curveto(703.9390448,1120.73023264)(703.66275387,1119.85112512)(703.66275554,1118.61944829)
\closepath
}
}
{
\newrgbcolor{curcolor}{0 0 0}
\pscustom[linestyle=none,fillstyle=solid,fillcolor=curcolor]
{
\newpath
\moveto(711.39332699,1117.27985454)
\lineto(712.41476723,1117.36916079)
\curveto(712.46313987,1116.95983816)(712.57570284,1116.62400979)(712.75245649,1116.36167465)
\curveto(712.92920639,1116.09933609)(713.20363678,1115.88723396)(713.57574848,1115.72536762)
\curveto(713.94785479,1115.56349913)(714.36647742,1115.48256542)(714.83161763,1115.48256625)
\curveto(715.24465467,1115.48256542)(715.60932149,1115.5439634)(715.92561919,1115.66676039)
\curveto(716.2419068,1115.78955535)(716.47726574,1115.95793467)(716.63169673,1116.17189887)
\curveto(716.78611622,1116.38586002)(716.86332884,1116.61935842)(716.86333482,1116.87239477)
\curveto(716.86332884,1117.12914776)(716.78890704,1117.35334343)(716.64006919,1117.54498247)
\curveto(716.49121983,1117.73661571)(716.24562789,1117.89755285)(715.90329263,1118.02779438)
\curveto(715.68374329,1118.11337607)(715.19814104,1118.24640504)(714.44648442,1118.42688169)
\curveto(713.69482067,1118.60735078)(713.16828643,1118.77759065)(712.86688012,1118.93760181)
\curveto(712.47616368,1119.14225747)(712.18498839,1119.39622187)(711.99335336,1119.69949575)
\curveto(711.80171611,1120.00275954)(711.70589804,1120.34230901)(711.70589887,1120.71814517)
\curveto(711.70589804,1121.1311801)(711.82311238,1121.51724319)(712.05754223,1121.8763356)
\curveto(712.29196972,1122.23541356)(712.63431001,1122.50798341)(713.08456411,1122.69404595)
\curveto(713.5348138,1122.88009241)(714.03530041,1122.97311966)(714.58602544,1122.97312799)
\curveto(715.19255941,1122.97311966)(715.7274661,1122.87544105)(716.19074712,1122.68009185)
\curveto(716.65401752,1122.4847266)(717.01031189,1122.19727239)(717.25963131,1121.81772837)
\curveto(717.50893796,1121.43817002)(717.6428972,1121.00838412)(717.66150943,1120.52836939)
\lineto(716.62332427,1120.45022642)
\curveto(716.56750218,1120.96745213)(716.37865686,1121.35816659)(716.05678775,1121.62237095)
\curveto(715.73490828,1121.88656137)(715.25953903,1122.01866007)(714.63067856,1122.01866744)
\curveto(713.97576297,1122.01866007)(713.49853317,1121.89865492)(713.19898774,1121.65865162)
\curveto(712.89943767,1121.4186343)(712.7496638,1121.12931955)(712.74966567,1120.7907065)
\curveto(712.7496638,1120.49673424)(712.85571487,1120.25486339)(713.06781919,1120.06509321)
\curveto(713.27619804,1119.87531221)(713.82040746,1119.68088525)(714.70044907,1119.48181177)
\curveto(715.58048305,1119.28272862)(716.1842299,1119.10876766)(716.51169146,1118.95992837)
\curveto(716.98798535,1118.74037975)(717.33962836,1118.46222827)(717.56662154,1118.12547309)
\curveto(717.79360135,1117.78871097)(717.90709459,1117.40078733)(717.90710162,1116.96170102)
\curveto(717.90709459,1116.52633117)(717.78243808,1116.116081)(717.5331317,1115.73094926)
\curveto(717.28381201,1115.34581536)(716.92565709,1115.04626761)(716.45866587,1114.83230512)
\curveto(715.99166349,1114.61834226)(715.46605953,1114.51136092)(714.88185239,1114.51136078)
\curveto(714.14135147,1114.51136092)(713.52085971,1114.61927253)(713.02037524,1114.83509594)
\curveto(712.51988649,1115.05091897)(712.12731149,1115.37558408)(711.84264907,1115.80909223)
\curveto(711.55798471,1116.24259806)(711.40821084,1116.73285167)(711.39332699,1117.27985454)
\closepath
}
}
{
\newrgbcolor{curcolor}{0 0 0}
\pscustom[linestyle=none,fillstyle=solid,fillcolor=curcolor]
{
\newpath
\moveto(725.94466293,1114.65090179)
\lineto(724.93996762,1114.65090179)
\lineto(724.93996762,1121.05304361)
\curveto(724.69809351,1120.82232962)(724.38087058,1120.59162204)(723.98829789,1120.36092017)
\curveto(723.59572059,1120.13020688)(723.24314731,1119.95717619)(722.93057699,1119.84182759)
\lineto(722.93057699,1120.81303306)
\curveto(723.49246034,1121.07722429)(723.98364422,1121.39723803)(724.40413012,1121.77307525)
\curveto(724.82461057,1122.14889822)(725.12229777,1122.51356504)(725.29719262,1122.86707681)
\lineto(725.94466293,1122.86707681)
\closepath
}
}
{
\newrgbcolor{curcolor}{0 0 0}
\pscustom[linestyle=none,fillstyle=solid,fillcolor=curcolor]
{
\newpath
\moveto(729.08712772,1114.65090179)
\lineto(729.08712772,1115.79513813)
\lineto(730.23136406,1115.79513813)
\lineto(730.23136406,1114.65090179)
\closepath
}
}
{
\newrgbcolor{curcolor}{0 0 0}
\pscustom[linestyle=none,fillstyle=solid,fillcolor=curcolor]
{
\newpath
\moveto(731.69375403,1118.68642798)
\curveto(731.69375356,1119.65390735)(731.79329271,1120.43254544)(731.9923718,1121.02234458)
\curveto(732.19144935,1121.61213098)(732.48727601,1122.06703424)(732.87985267,1122.38705572)
\curveto(733.272426,1122.70706173)(733.76640071,1122.8670686)(734.36177825,1122.86707681)
\curveto(734.80086374,1122.8670686)(735.18599656,1122.77869271)(735.51717787,1122.60194888)
\curveto(735.84835058,1122.42518916)(736.1218507,1122.17029449)(736.33767904,1121.83726412)
\curveto(736.55349714,1121.50421937)(736.72280674,1121.09862056)(736.84560834,1120.62046646)
\curveto(736.96839868,1120.14230042)(737.02979667,1119.49762157)(737.02980248,1118.68642798)
\curveto(737.02979667,1117.72638271)(736.93118778,1116.95146571)(736.73397553,1116.36167465)
\curveto(736.53675224,1115.77188017)(736.24185585,1115.31604664)(735.84928548,1114.99417269)
\curveto(735.45670586,1114.67229806)(734.96087061,1114.51136092)(734.36177825,1114.51136078)
\curveto(733.57290402,1114.51136092)(732.95334253,1114.79416376)(732.50309192,1115.35977016)
\curveto(731.96353258,1116.04072892)(731.69375356,1117.14961376)(731.69375403,1118.68642798)
\closepath
\moveto(732.72635755,1118.68642798)
\curveto(732.72635604,1117.34311044)(732.88357209,1116.44911856)(733.19800618,1116.00444965)
\curveto(733.51243631,1115.55977804)(733.90035995,1115.33744291)(734.36177825,1115.33744359)
\curveto(734.82319028,1115.33744291)(735.21111391,1115.56070831)(735.52555033,1116.00724047)
\curveto(735.83997813,1116.45376992)(735.99719418,1117.34683153)(735.99719896,1118.68642798)
\curveto(735.99719418,1120.03345853)(735.83997813,1120.92838069)(735.52555033,1121.37119712)
\curveto(735.21111391,1121.81400012)(734.81946919,1122.03540497)(734.35061497,1122.03541236)
\curveto(733.88919668,1122.03540497)(733.52080876,1121.84004775)(733.24545013,1121.44934009)
\curveto(732.89938673,1120.95070723)(732.72635604,1120.02973744)(732.72635755,1118.68642798)
\closepath
}
}
{
\newrgbcolor{curcolor}{0 0 0}
\pscustom[linestyle=none,fillstyle=solid,fillcolor=curcolor]
{
\newpath
\moveto(585.8490378,762.34590912)
\lineto(585.8490378,764.30506498)
\lineto(582.29911436,764.30506498)
\lineto(582.29911436,765.22603569)
\lineto(586.03323194,770.52859429)
\lineto(586.85373312,770.52859429)
\lineto(586.85373312,765.22603569)
\lineto(587.95889796,765.22603569)
\lineto(587.95889796,764.30506498)
\lineto(586.85373312,764.30506498)
\lineto(586.85373312,762.34590912)
\closepath
\moveto(585.8490378,765.22603569)
\lineto(585.8490378,768.91550015)
\lineto(583.28706475,765.22603569)
\closepath
}
}
{
\newrgbcolor{curcolor}{0 0 0}
\pscustom[linestyle=none,fillstyle=solid,fillcolor=curcolor]
{
\newpath
\moveto(589.55524735,762.34590912)
\lineto(589.55524735,763.49014545)
\lineto(590.69948368,763.49014545)
\lineto(590.69948368,762.34590912)
\closepath
}
}
{
\newrgbcolor{curcolor}{0 0 0}
\pscustom[linestyle=none,fillstyle=solid,fillcolor=curcolor]
{
\newpath
\moveto(592.16745529,764.50600405)
\lineto(593.17215061,764.63996342)
\curveto(593.28750291,764.07063435)(593.48379041,763.66038417)(593.76101369,763.40921166)
\curveto(594.03823283,763.15803702)(594.37592175,763.03245023)(594.77408147,763.03245092)
\curveto(595.24665682,763.03245023)(595.64574372,763.19617819)(595.97134339,763.52363529)
\curveto(596.29693448,763.85109004)(596.45973217,764.25668885)(596.45973694,764.74043295)
\curveto(596.45973217,765.20184572)(596.30902802,765.58232718)(596.00762405,765.88187846)
\curveto(595.70621144,766.18142268)(595.32293916,766.33119655)(594.85780608,766.33120053)
\curveto(594.66802732,766.33119655)(594.4317381,766.29398565)(594.14893772,766.21956772)
\lineto(594.26057053,767.10146694)
\curveto(594.32754758,767.09402001)(594.38150338,767.09029892)(594.42243811,767.09030366)
\curveto(594.85036073,767.09029892)(595.23549355,767.20193162)(595.57783772,767.4252021)
\curveto(595.92017411,767.64846242)(596.09134426,767.99266325)(596.09134866,768.45780562)
\curveto(596.09134426,768.82618742)(595.96668774,769.1313168)(595.71737874,769.37319468)
\curveto(595.46806167,769.61505851)(595.14618739,769.73599393)(594.75175491,769.73600132)
\curveto(594.36103739,769.73599393)(594.03544201,769.61319796)(593.7749678,769.36761304)
\curveto(593.51448941,769.12201408)(593.34704035,768.75362616)(593.27262014,768.2624482)
\lineto(592.26792482,768.4410607)
\curveto(592.39072021,769.1145719)(592.66980197,769.63645478)(593.10517092,770.0067109)
\curveto(593.54053704,770.37695169)(594.08195564,770.56207592)(594.72942834,770.56208414)
\curveto(595.17595611,770.56207592)(595.58713656,770.46625785)(595.96297093,770.27462965)
\curveto(596.33879674,770.08298558)(596.62625095,769.821579)(596.8253344,769.49040914)
\curveto(597.02440758,769.15922498)(597.12394674,768.80758197)(597.12395218,768.43547906)
\curveto(597.12394674,768.08196941)(597.02905895,767.76009512)(596.83928851,767.46985522)
\curveto(596.64950776,767.17960508)(596.36856546,766.9488975)(595.99646077,766.77773179)
\curveto(596.48019817,766.66609465)(596.85602826,766.4344568)(597.12395218,766.08281753)
\curveto(597.39186522,765.73117078)(597.52582447,765.29115188)(597.5258303,764.76275952)
\curveto(597.52582447,764.04830781)(597.26534816,763.44270041)(596.74440061,762.94593549)
\curveto(596.22344295,762.44916937)(595.56481002,762.20078661)(594.76849983,762.20078646)
\curveto(594.05032637,762.20078661)(593.45402169,762.41474928)(592.979584,762.84267514)
\curveto(592.50514373,763.27059999)(592.23443443,763.82504241)(592.16745529,764.50600405)
\closepath
}
}
{
\newrgbcolor{curcolor}{0 0 0}
\pscustom[linestyle=none,fillstyle=solid,fillcolor=curcolor]
{
\newpath
\moveto(598.88775078,762.34590912)
\lineto(598.88775078,770.52859429)
\lineto(601.95765313,770.52859429)
\curveto(602.58279235,770.52858611)(603.08420923,770.44579186)(603.46190528,770.28021129)
\curveto(603.83959051,770.11461485)(604.13541717,769.85972018)(604.34938614,769.51552652)
\curveto(604.56334252,769.17131852)(604.67032386,768.81130306)(604.67033048,768.43547906)
\curveto(604.67032386,768.0856905)(604.57543607,767.75637403)(604.38566681,767.44752866)
\curveto(604.19588488,767.13867309)(603.90936095,766.88936006)(603.52609415,766.69958882)
\curveto(604.02099365,766.55446195)(604.40147511,766.30700946)(604.66753966,765.95723061)
\curveto(604.93359098,765.60744454)(605.06661995,765.19440354)(605.06662697,764.71810639)
\curveto(605.06661995,764.33483174)(604.98568624,763.97853737)(604.8238256,763.64922221)
\curveto(604.66195141,763.31990444)(604.46194282,763.06594004)(604.22379923,762.88732826)
\curveto(603.98564329,762.7087154)(603.68702582,762.57382588)(603.32794591,762.48265931)
\curveto(602.96885544,762.39149247)(602.52883654,762.34590912)(602.0078879,762.34590912)
\closepath
\moveto(599.97058906,767.09030366)
\lineto(601.73996915,767.09030366)
\curveto(602.21998607,767.09029892)(602.5641869,767.12192818)(602.77257266,767.18519155)
\curveto(603.04792861,767.26705069)(603.25537937,767.40287048)(603.3949256,767.59265132)
\curveto(603.53446113,767.78242166)(603.60423157,768.02057143)(603.60423712,768.30710132)
\curveto(603.60423157,768.57873493)(603.53911249,768.81781497)(603.4088797,769.02434214)
\curveto(603.27863619,769.23085596)(603.09258169,769.37225738)(602.85071563,769.44854683)
\curveto(602.60883998,769.52482207)(602.19393844,769.56296325)(601.60600977,769.56297046)
\lineto(599.97058906,769.56297046)
\closepath
\moveto(599.97058906,763.31153295)
\lineto(602.0078879,763.31153295)
\curveto(602.3576664,763.31153198)(602.60325835,763.3245558)(602.74466446,763.35060443)
\curveto(602.9939728,763.39525651)(603.20235384,763.46967831)(603.36980821,763.57387006)
\curveto(603.53725195,763.67805935)(603.67493228,763.82969377)(603.78284962,764.02877377)
\curveto(603.8907555,764.22785041)(603.9447113,764.45762772)(603.9447172,764.71810639)
\curveto(603.9447113,765.0232334)(603.86656841,765.28836107)(603.71028829,765.51349018)
\curveto(603.55399685,765.73861296)(603.33724336,765.89675929)(603.06002716,765.98792964)
\curveto(602.78280094,766.0790927)(602.38371403,766.12467605)(601.86276524,766.12467983)
\lineto(599.97058906,766.12467983)
\closepath
}
}
{
\newrgbcolor{curcolor}{0 0 0}
\pscustom[linestyle=none,fillstyle=solid,fillcolor=curcolor]
{
\newpath
\moveto(606.19970059,764.97486186)
\lineto(607.22114083,765.06416811)
\curveto(607.26951346,764.65484549)(607.38207644,764.31901711)(607.55883009,764.05668197)
\curveto(607.73557999,763.79434342)(608.01001038,763.58224128)(608.38212208,763.42037494)
\curveto(608.75422839,763.25850645)(609.17285102,763.17757274)(609.63799122,763.17757357)
\curveto(610.05102827,763.17757274)(610.41569509,763.23897073)(610.73199279,763.36176772)
\curveto(611.0482804,763.48456267)(611.28363934,763.65294199)(611.43807033,763.86690619)
\curveto(611.59248981,764.08086735)(611.66970243,764.31436575)(611.66970842,764.56740209)
\curveto(611.66970243,764.82415508)(611.59528063,765.04835076)(611.44644279,765.23998979)
\curveto(611.29759343,765.43162303)(611.05200149,765.59256018)(610.70966623,765.72280171)
\curveto(610.49011689,765.8083834)(610.00451464,765.94141237)(609.25285802,766.12188901)
\curveto(608.50119427,766.3023581)(607.97466002,766.47259797)(607.67325372,766.63260913)
\curveto(607.28253728,766.83726479)(606.99136198,767.09122919)(606.79972696,767.39450308)
\curveto(606.60808971,767.69776687)(606.51227164,768.03731633)(606.51227247,768.41315249)
\curveto(606.51227164,768.82618742)(606.62948598,769.21225051)(606.86391583,769.57134293)
\curveto(607.09834332,769.93042089)(607.4406836,770.20299073)(607.8909377,770.38905328)
\curveto(608.34118739,770.57509974)(608.841674,770.66812699)(609.39239904,770.66813531)
\curveto(609.99893301,770.66812699)(610.5338397,770.57044838)(610.99712072,770.37509918)
\curveto(611.46039112,770.17973392)(611.81668549,769.89227972)(612.0660049,769.5127357)
\curveto(612.31531155,769.13317735)(612.4492708,768.70339145)(612.46788303,768.22337671)
\lineto(611.42969787,768.14523374)
\curveto(611.37387577,768.66245946)(611.18503045,769.05317391)(610.86316134,769.31737828)
\curveto(610.54128188,769.5815687)(610.06591263,769.71366739)(609.43705216,769.71367476)
\curveto(608.78213656,769.71366739)(608.30490677,769.59366224)(608.00536134,769.35365894)
\curveto(607.70581127,769.11364163)(607.5560374,768.82432687)(607.55603927,768.48571382)
\curveto(607.5560374,768.19174157)(607.66208846,767.94987072)(607.87419278,767.76010054)
\curveto(608.08257164,767.57031953)(608.62678105,767.37589258)(609.50682267,767.17681909)
\curveto(610.38685664,766.97773594)(610.9906035,766.80377498)(611.31806506,766.65493569)
\curveto(611.79435895,766.43538707)(612.14600196,766.15723559)(612.37299514,765.82048042)
\curveto(612.59997494,765.48371829)(612.71346819,765.09579466)(612.71347522,764.65670834)
\curveto(612.71346819,764.2213385)(612.58881167,763.81108832)(612.33950529,763.42595658)
\curveto(612.09018561,763.04082268)(611.73203069,762.74127494)(611.26503947,762.52731244)
\curveto(610.79803709,762.31334958)(610.27243312,762.20636824)(609.68822599,762.2063681)
\curveto(608.94772507,762.20636824)(608.32723331,762.31427985)(607.82674884,762.53010326)
\curveto(607.32626009,762.7459263)(606.93368509,763.0705914)(606.64902266,763.50409955)
\curveto(606.36435831,763.93760538)(606.21458444,764.427859)(606.19970059,764.97486186)
\closepath
}
}
{
\newrgbcolor{curcolor}{0 0 0}
\pscustom[linestyle=none,fillstyle=solid,fillcolor=curcolor]
{
\newpath
\moveto(614.20377193,762.34590912)
\lineto(614.20377193,770.52859429)
\lineto(617.02250045,770.52859429)
\curveto(617.65880315,770.52858611)(618.1444054,770.48951467)(618.47930866,770.41137984)
\curveto(618.94816085,770.30346016)(619.34817803,770.10810294)(619.6793614,769.82530757)
\curveto(620.11100149,769.46063327)(620.43380605,768.99456674)(620.64777605,768.42710659)
\curveto(620.8617314,767.85963428)(620.96871274,767.21123434)(620.96872039,766.48190483)
\curveto(620.96871274,765.86047866)(620.89615148,765.30975733)(620.7510364,764.8297392)
\curveto(620.60590646,764.3497161)(620.41985196,763.95248974)(620.19287234,763.63805893)
\curveto(619.96587897,763.32362553)(619.71749621,763.07617304)(619.44772331,762.89570072)
\curveto(619.17793816,762.71522731)(618.85234278,762.57847725)(618.4709362,762.48545013)
\curveto(618.08951932,762.39242274)(617.65136097,762.34590912)(617.15645983,762.34590912)
\closepath
\moveto(615.28661022,763.31153295)
\lineto(617.03366374,763.31153295)
\curveto(617.57321808,763.31153198)(617.99649207,763.3617667)(618.30348698,763.46223725)
\curveto(618.61047193,763.56270556)(618.8551336,763.70410698)(619.03747272,763.88644194)
\curveto(619.29422222,764.14319561)(619.49423081,764.48832671)(619.63749909,764.92183627)
\curveto(619.78075474,765.35534069)(619.85238573,765.88094465)(619.85239226,766.49864975)
\curveto(619.85238573,767.35449631)(619.71191458,768.01219897)(619.43097839,768.47175972)
\curveto(619.15002998,768.93130821)(618.80861997,769.23922841)(618.40674733,769.39552125)
\curveto(618.11649722,769.5071469)(617.64950043,769.56296325)(617.00575553,769.56297046)
\lineto(615.28661022,769.56297046)
\closepath
}
}
{
\newrgbcolor{curcolor}{0 0 0}
\pscustom[linestyle=none,fillstyle=solid,fillcolor=curcolor]
{
\newpath
\moveto(625.65171879,762.34590912)
\lineto(625.65171879,770.52859429)
\lineto(629.2797852,770.52859429)
\curveto(630.00911432,770.52858611)(630.56355674,770.45509458)(630.94311411,770.30811949)
\curveto(631.32265911,770.16112847)(631.62592795,769.90158244)(631.85292154,769.52948062)
\curveto(632.07990093,769.15736443)(632.19339418,768.74618398)(632.19340162,768.29593804)
\curveto(632.19339418,767.71544204)(632.00547913,767.2261187)(631.62965591,766.82796655)
\curveto(631.25381894,766.42980544)(630.6733289,766.17677131)(629.88818403,766.06886342)
\curveto(630.17470283,765.93117937)(630.3923866,765.79535958)(630.54123599,765.66140366)
\curveto(630.85752285,765.37115532)(631.1570706,765.00834904)(631.43988013,764.57298373)
\lineto(632.86319849,762.34590912)
\lineto(631.50127818,762.34590912)
\lineto(630.41843989,764.04830951)
\curveto(630.10214158,764.5394917)(629.84166527,764.91532179)(629.6370102,765.17580092)
\curveto(629.43234537,765.4362744)(629.24908168,765.61860781)(629.0872186,765.72280171)
\curveto(628.92534685,765.82698885)(628.76068862,765.89955011)(628.59324341,765.94048569)
\curveto(628.47044359,765.96652973)(628.26950473,765.97955354)(627.99042622,765.97955717)
\lineto(626.73455707,765.97955717)
\lineto(626.73455707,762.34590912)
\closepath
\moveto(626.73455707,766.9172728)
\lineto(629.06210122,766.9172728)
\curveto(629.55700188,766.91726823)(629.94399525,766.96843322)(630.22308247,767.07076792)
\curveto(630.50215875,767.17309317)(630.71426089,767.33682113)(630.8593895,767.5619523)
\curveto(631.00450591,767.78707303)(631.07706717,768.0317347)(631.07707349,768.29593804)
\curveto(631.07706717,768.68292545)(630.93659602,769.00107865)(630.65565962,769.25039859)
\curveto(630.37471142,769.49970472)(629.93097143,769.62436123)(629.32443833,769.62436851)
\lineto(626.73455707,769.62436851)
\closepath
}
}
{
\newrgbcolor{curcolor}{0 0 0}
\pscustom[linestyle=none,fillstyle=solid,fillcolor=curcolor]
{
\newpath
\moveto(637.82527499,764.25483022)
\lineto(638.86346015,764.12645248)
\curveto(638.69972633,763.51991303)(638.3964575,763.04919514)(637.95365272,762.7142974)
\curveto(637.51083807,762.37939893)(636.94523238,762.21194988)(636.25683397,762.21194974)
\curveto(635.38981675,762.21194988)(634.70234536,762.47893809)(634.19441775,763.01291518)
\curveto(633.68648778,763.54689093)(633.43252339,764.2957603)(633.4325238,765.25952553)
\curveto(633.43252339,766.25677475)(633.6892786,767.03076148)(634.20279021,767.58148804)
\curveto(634.71629945,768.13220413)(635.38237457,768.40756479)(636.20101756,768.40757085)
\curveto(636.99360655,768.40756479)(637.64107622,768.13778576)(638.1434285,767.59823296)
\curveto(638.64577053,767.05866965)(638.89694411,766.29956728)(638.89694999,765.32092358)
\curveto(638.89694411,765.26138316)(638.89508356,765.172077)(638.89136835,765.05300483)
\lineto(634.47070896,765.05300483)
\curveto(634.50791841,764.40181136)(634.69211236,763.9031853)(635.02329139,763.55712514)
\curveto(635.35446639,763.21106255)(635.76750738,763.03803187)(636.26241561,763.03803256)
\curveto(636.63080027,763.03803187)(636.94523238,763.13478021)(637.20571288,763.32827787)
\curveto(637.46618499,763.52177357)(637.67270548,763.83062404)(637.82527499,764.25483022)
\closepath
\moveto(634.52652537,765.87908764)
\lineto(637.83643827,765.87908764)
\curveto(637.79178037,766.37771017)(637.6652633,766.75167972)(637.45688671,767.00099741)
\curveto(637.13686852,767.38798612)(636.72196698,767.5814828)(636.21218084,767.58148804)
\curveto(635.75076248,767.5814828)(635.36283884,767.42705757)(635.04840877,767.11821186)
\curveto(634.73397463,766.80935662)(634.56001367,766.39631563)(634.52652537,765.87908764)
\closepath
}
}
{
\newrgbcolor{curcolor}{0 0 0}
\pscustom[linestyle=none,fillstyle=solid,fillcolor=curcolor]
{
\newpath
\moveto(640.13049368,762.34590912)
\lineto(640.13049368,768.27361148)
\lineto(641.03471947,768.27361148)
\lineto(641.03471947,767.43078374)
\curveto(641.47008534,768.08196941)(642.09894956,768.40756479)(642.921314,768.40757085)
\curveto(643.2785351,768.40756479)(643.6069213,768.34337599)(643.90647358,768.21500425)
\curveto(644.2060168,768.08662077)(644.43021247,767.91824145)(644.57906127,767.70986577)
\curveto(644.72789967,767.50147937)(644.83209019,767.25402688)(644.89163315,766.96750757)
\curveto(644.92883853,766.78144844)(644.94744399,766.45585307)(644.94744956,765.99072046)
\lineto(644.94744956,762.34590912)
\lineto(643.94275424,762.34590912)
\lineto(643.94275424,765.95164897)
\curveto(643.94274967,766.36096527)(643.90367823,766.66702493)(643.82553979,766.86982886)
\curveto(643.74739245,767.07262374)(643.60878184,767.23449116)(643.40970756,767.35543159)
\curveto(643.21062521,767.47636201)(642.97712681,767.53682972)(642.70921166,767.53683491)
\curveto(642.28128297,767.53682972)(641.91196479,767.40100994)(641.60125599,767.12937515)
\curveto(641.29054275,766.85773079)(641.13518724,766.34235982)(641.135189,765.58326069)
\lineto(641.135189,762.34590912)
\closepath
}
}
{
\newrgbcolor{curcolor}{0 0 0}
\pscustom[linestyle=none,fillstyle=solid,fillcolor=curcolor]
{
\newpath
\moveto(646.1195952,765.3097603)
\curveto(646.11959482,766.4074789)(646.42472421,767.22053707)(647.03498427,767.74893726)
\curveto(647.54477231,768.18802048)(648.16619434,768.40756479)(648.89925224,768.40757085)
\curveto(649.7141678,768.40756479)(650.38024292,768.14057658)(650.89747959,767.60660542)
\curveto(651.41470595,767.07262374)(651.67332171,766.33491764)(651.67332764,765.39348491)
\curveto(651.67332171,764.6306584)(651.55889819,764.03063263)(651.33005674,763.5934058)
\curveto(651.10120411,763.15617647)(650.76816655,762.81662701)(650.33094307,762.57475639)
\curveto(649.89371039,762.3328853)(649.4164806,762.21194988)(648.89925224,762.21194974)
\curveto(648.069446,762.21194988)(647.39871952,762.47800782)(646.88707079,763.01012435)
\curveto(646.37541976,763.54223957)(646.11959482,764.30878411)(646.1195952,765.3097603)
\closepath
\moveto(647.15219872,765.3097603)
\curveto(647.15219731,764.55065497)(647.31778582,763.98225846)(647.64896474,763.60456908)
\curveto(647.98013984,763.22687719)(648.39690193,763.03803187)(648.89925224,763.03803256)
\curveto(649.39787515,763.03803187)(649.81277669,763.22780746)(650.1439581,763.6073599)
\curveto(650.47513071,763.98690983)(650.64071922,764.56553933)(650.64072412,765.34325014)
\curveto(650.64071922,766.07630188)(650.47420044,766.63167457)(650.14116728,767.00936987)
\curveto(649.80812532,767.38705585)(649.39415406,767.57590117)(648.89925224,767.5759064)
\curveto(648.39690193,767.57590117)(647.98013984,767.38798612)(647.64896474,767.01216069)
\curveto(647.31778582,766.63632593)(647.15219731,766.0688597)(647.15219872,765.3097603)
\closepath
}
}
{
\newrgbcolor{curcolor}{0 0 0}
\pscustom[linestyle=none,fillstyle=solid,fillcolor=curcolor]
{
\newpath
\moveto(144.86850667,722.25586772)
\lineto(145.8899469,722.34517397)
\curveto(145.93831954,721.93585135)(146.05088251,721.60002297)(146.22763616,721.33768783)
\curveto(146.40438607,721.07534928)(146.67881646,720.86324714)(147.05092816,720.7013808)
\curveto(147.42303446,720.53951231)(147.84165709,720.4585786)(148.3067973,720.45857943)
\curveto(148.71983434,720.4585786)(149.08450117,720.51997659)(149.40079886,720.64277357)
\curveto(149.71708647,720.76556853)(149.95244542,720.93394785)(150.10687641,721.14791205)
\curveto(150.26129589,721.36187321)(150.33850851,721.59537161)(150.33851449,721.84840795)
\curveto(150.33850851,722.10516094)(150.26408671,722.32935662)(150.11524887,722.52099565)
\curveto(149.9663995,722.71262889)(149.72080756,722.87356604)(149.3784723,723.00380756)
\curveto(149.15892297,723.08938926)(148.67332072,723.22241823)(147.9216641,723.40289487)
\curveto(147.17000034,723.58336396)(146.6434661,723.75360383)(146.34205979,723.91361499)
\curveto(145.95134335,724.11827065)(145.66016806,724.37223505)(145.46853303,724.67550894)
\curveto(145.27689578,724.97877273)(145.18107772,725.31832219)(145.18107854,725.69415835)
\curveto(145.18107772,726.10719328)(145.29829205,726.49325637)(145.5327219,726.85234878)
\curveto(145.7671494,727.21142675)(146.10948968,727.48399659)(146.55974378,727.67005914)
\curveto(147.00999347,727.8561056)(147.51048008,727.94913285)(148.06120511,727.94914117)
\curveto(148.66773908,727.94913285)(149.20264577,727.85145423)(149.66592679,727.65610504)
\curveto(150.12919719,727.46073978)(150.48549156,727.17328558)(150.73481098,726.79374156)
\curveto(150.98411763,726.41418321)(151.11807687,725.98439731)(151.1366891,725.50438257)
\lineto(150.09850394,725.4262396)
\curveto(150.04268185,725.94346532)(149.85383653,726.33417977)(149.53196742,726.59838414)
\curveto(149.21008795,726.86257456)(148.7347187,726.99467325)(148.10585824,726.99468062)
\curveto(147.45094264,726.99467325)(146.97371284,726.8746681)(146.67416741,726.6346648)
\curveto(146.37461735,726.39464748)(146.22484347,726.10533273)(146.22484534,725.76671968)
\curveto(146.22484347,725.47274743)(146.33089454,725.23087658)(146.54299886,725.0411064)
\curveto(146.75137771,724.85132539)(147.29558713,724.65689844)(148.17562874,724.45782495)
\curveto(149.05566272,724.2587418)(149.65940958,724.08478084)(149.98687113,723.93594155)
\curveto(150.46316502,723.71639293)(150.81480803,723.43824145)(151.04180121,723.10148628)
\curveto(151.26878102,722.76472415)(151.38227426,722.37680052)(151.38228129,721.9377142)
\curveto(151.38227426,721.50234436)(151.25761775,721.09209418)(151.00831137,720.70696244)
\curveto(150.75899168,720.32182854)(150.40083677,720.02228079)(149.93384555,719.8083183)
\curveto(149.46684317,719.59435544)(148.9412392,719.4873741)(148.35703207,719.48737396)
\curveto(147.61653115,719.4873741)(146.99603938,719.59528571)(146.49555491,719.81110912)
\curveto(145.99506616,720.02693216)(145.60249116,720.35159726)(145.31782874,720.78510541)
\curveto(145.03316439,721.21861124)(144.88339051,721.70886485)(144.86850667,722.25586772)
\closepath
}
}
{
\newrgbcolor{curcolor}{0 0 0}
\pscustom[linestyle=none,fillstyle=solid,fillcolor=curcolor]
{
\newpath
\moveto(152.37023178,722.59076616)
\curveto(152.3702314,723.68848475)(152.67536078,724.50154293)(153.28562085,725.02994312)
\curveto(153.79540889,725.46902634)(154.41683092,725.68857065)(155.14988882,725.68857671)
\curveto(155.96480438,725.68857065)(156.63087949,725.42158244)(157.14811617,724.88761128)
\curveto(157.66534253,724.3536296)(157.92395828,723.6159235)(157.92396422,722.67449077)
\curveto(157.92395828,721.91166426)(157.80953476,721.31163849)(157.58069332,720.87441166)
\curveto(157.35184069,720.43718233)(157.01880313,720.09763287)(156.58157964,719.85576224)
\curveto(156.14434697,719.61389116)(155.66711718,719.49295574)(155.14988882,719.4929556)
\curveto(154.32008258,719.49295574)(153.6493561,719.75901367)(153.13770737,720.29113021)
\curveto(152.62605634,720.82324543)(152.3702314,721.58978997)(152.37023178,722.59076616)
\closepath
\moveto(153.4028353,722.59076616)
\curveto(153.40283389,721.83166083)(153.56842239,721.26326432)(153.89960132,720.88557494)
\curveto(154.23077642,720.50788304)(154.6475385,720.31903773)(155.14988882,720.31903842)
\curveto(155.64851172,720.31903773)(156.06341326,720.50881332)(156.39459468,720.88836576)
\curveto(156.72576729,721.26791569)(156.8913558,721.84654519)(156.8913607,722.624256)
\curveto(156.8913558,723.35730774)(156.72483702,723.91268043)(156.39180386,724.29037573)
\curveto(156.0587619,724.66806171)(155.64479063,724.85690703)(155.14988882,724.85691226)
\curveto(154.6475385,724.85690703)(154.23077642,724.66899198)(153.89960132,724.29316655)
\curveto(153.56842239,723.91733179)(153.40283389,723.34986556)(153.4028353,722.59076616)
\closepath
}
}
{
\newrgbcolor{curcolor}{0 0 0}
\pscustom[linestyle=none,fillstyle=solid,fillcolor=curcolor]
{
\newpath
\moveto(159.08494515,719.62691498)
\lineto(159.08494515,727.80960015)
\lineto(160.08964047,727.80960015)
\lineto(160.08964047,719.62691498)
\closepath
}
}
{
\newrgbcolor{curcolor}{0 0 0}
\pscustom[linestyle=none,fillstyle=solid,fillcolor=curcolor]
{
\newpath
\moveto(165.52057764,720.3581099)
\curveto(165.14846401,720.04181652)(164.79030909,719.81855112)(164.44611181,719.68831303)
\curveto(164.10190744,719.55807481)(163.73258925,719.49295574)(163.33815615,719.4929556)
\curveto(162.68696295,719.49295574)(162.18647634,719.65203234)(161.83669481,719.97018588)
\curveto(161.48691141,720.28833873)(161.31202018,720.69486782)(161.31202059,721.18977436)
\curveto(161.31202018,721.48001782)(161.37806953,721.74514548)(161.51016884,721.98515815)
\curveto(161.64226692,722.2251661)(161.81529761,722.41773251)(162.02926142,722.56285795)
\curveto(162.24322296,722.70797753)(162.48416354,722.81774969)(162.75208388,722.89217475)
\curveto(162.9492998,722.94426675)(163.246987,722.99450146)(163.64514638,723.04287905)
\curveto(164.45634126,723.13962397)(165.05357621,723.25497777)(165.43685303,723.38894077)
\curveto(165.44056958,723.52661734)(165.44243012,723.61406295)(165.44243467,723.65127788)
\curveto(165.44243012,724.06059376)(165.34754233,724.34897824)(165.15777099,724.51643218)
\curveto(164.90101152,724.74341378)(164.51959979,724.85690703)(164.01353466,724.85691226)
\curveto(163.54095311,724.85690703)(163.19210092,724.77411277)(162.96697704,724.60852925)
\curveto(162.74184903,724.44293576)(162.57533025,724.14989992)(162.46742021,723.72942085)
\lineto(161.48505145,723.86338022)
\curveto(161.57435703,724.28385916)(161.72134009,724.62340863)(161.92600106,724.88202964)
\curveto(162.13065999,725.14064014)(162.42648665,725.33971846)(162.81348193,725.47926519)
\curveto(163.20047338,725.61880021)(163.64886472,725.68857065)(164.15865732,725.68857671)
\curveto(164.66472231,725.68857065)(165.07590275,725.62903321)(165.3921999,725.50996421)
\curveto(165.70848806,725.39088345)(165.94105619,725.24110957)(166.08990498,725.06064214)
\curveto(166.23874339,724.88016384)(166.34293391,724.65224707)(166.40247686,724.37689116)
\curveto(166.43596116,724.20571627)(166.45270607,723.8968658)(166.45271162,723.45033882)
\lineto(166.45271162,722.11074506)
\curveto(166.45270607,721.17674898)(166.47410234,720.58602594)(166.51690049,720.33857416)
\curveto(166.55968741,720.09112096)(166.64434221,719.85390147)(166.77086514,719.62691498)
\lineto(165.7215167,719.62691498)
\curveto(165.61732136,719.83529602)(165.55034173,720.07902742)(165.52057764,720.3581099)
\closepath
\moveto(165.43685303,722.60192944)
\curveto(165.07218166,722.45308286)(164.52518143,722.3265658)(163.79585068,722.22237788)
\curveto(163.38280679,722.16283784)(163.09070122,722.09585822)(162.9195331,722.02143881)
\curveto(162.74836094,721.94701462)(162.61626224,721.83817273)(162.52323661,721.69491283)
\curveto(162.43020774,721.5516488)(162.38369411,721.3925722)(162.3836956,721.21768256)
\curveto(162.38369411,720.94976249)(162.48509382,720.72649708)(162.68789501,720.54788568)
\curveto(162.89069263,720.36927244)(163.18744956,720.27996628)(163.57816669,720.27996693)
\curveto(163.96515738,720.27996628)(164.30935821,720.36462108)(164.61077021,720.53393158)
\curveto(164.91217479,720.70324027)(165.13357965,720.93487813)(165.27498545,721.22884584)
\curveto(165.38289268,721.45583073)(165.43684849,721.79072884)(165.43685303,722.23354116)
\closepath
}
}
{
\newrgbcolor{curcolor}{0 0 0}
\pscustom[linestyle=none,fillstyle=solid,fillcolor=curcolor]
{
\newpath
\moveto(168.00440693,719.62691498)
\lineto(168.00440693,725.55461734)
\lineto(168.90863271,725.55461734)
\lineto(168.90863271,724.65597319)
\curveto(169.13933865,725.07645134)(169.35237105,725.35367255)(169.54773056,725.48763765)
\curveto(169.7430855,725.62159103)(169.95797845,725.68857065)(170.19241006,725.68857671)
\curveto(170.53102632,725.68857065)(170.87522715,725.58065904)(171.22501357,725.36484155)
\lineto(170.87895186,724.43270757)
\curveto(170.6333563,724.57782527)(170.38776435,724.65038653)(170.14217529,724.65039155)
\curveto(169.9226281,724.65038653)(169.72541033,724.58433718)(169.55052138,724.45224331)
\curveto(169.37562786,724.32013979)(169.25097135,724.1368761)(169.17655146,723.90245171)
\curveto(169.06491684,723.54522279)(169.00910049,723.15450833)(169.00910224,722.73030717)
\lineto(169.00910224,719.62691498)
\closepath
}
}
{
\newrgbcolor{curcolor}{0 0 0}
\pscustom[linestyle=none,fillstyle=solid,fillcolor=curcolor]
{
\newpath
\moveto(171.83899433,726.65420054)
\lineto(171.83899433,727.80960015)
\lineto(172.84368965,727.80960015)
\lineto(172.84368965,726.65420054)
\closepath
\moveto(171.83899433,719.62691498)
\lineto(171.83899433,725.55461734)
\lineto(172.84368965,725.55461734)
\lineto(172.84368965,719.62691498)
\closepath
}
}
{
\newrgbcolor{curcolor}{0 0 0}
\pscustom[linestyle=none,fillstyle=solid,fillcolor=curcolor]
{
\newpath
\moveto(173.97676352,721.39629506)
\lineto(174.97029556,721.552581)
\curveto(175.02611056,721.15442244)(175.18146607,720.84929306)(175.43636255,720.63719193)
\curveto(175.69125541,720.42508879)(176.04754978,720.31903773)(176.50524673,720.31903842)
\curveto(176.96665902,720.31903773)(177.3089993,720.41299525)(177.53226861,720.60091127)
\curveto(177.7555301,720.78882534)(177.86716281,721.00929993)(177.86716705,721.26233569)
\curveto(177.86716281,721.48932054)(177.76855392,721.66793286)(177.57134009,721.79817319)
\curveto(177.43365582,721.88747718)(177.09131553,722.00097042)(176.54431822,722.13865327)
\curveto(175.80753947,722.32470526)(175.29681986,722.4856424)(175.01215786,722.62146518)
\curveto(174.72749309,722.75728197)(174.51166986,722.94519702)(174.36468755,723.18521089)
\curveto(174.21770375,723.42521763)(174.14421222,723.6903453)(174.14421274,723.98059468)
\curveto(174.14421222,724.24478772)(174.20467994,724.48944939)(174.32561606,724.71458042)
\curveto(174.44655079,724.93970128)(174.61120902,725.12668605)(174.81959126,725.2755353)
\curveto(174.97587585,725.39088345)(175.18890825,725.48856206)(175.45868911,725.56857144)
\curveto(175.72846631,725.64856893)(176.01778106,725.68857065)(176.32663423,725.68857671)
\curveto(176.79176779,725.68857065)(177.20015742,725.62159103)(177.55180435,725.48763765)
\curveto(177.90344343,725.35367255)(178.16298946,725.17226941)(178.33044322,724.94342769)
\curveto(178.49788757,724.71457533)(178.61324136,724.40851568)(178.67650494,724.0252478)
\lineto(177.69413619,723.89128843)
\curveto(177.64947904,724.19641354)(177.52017116,724.43456331)(177.30621216,724.60573843)
\curveto(177.09224581,724.77690359)(176.78990724,724.86248866)(176.39919556,724.8624939)
\curveto(175.93777762,724.86248866)(175.60846115,724.78620632)(175.41124517,724.63364663)
\curveto(175.21402561,724.48107693)(175.11541672,724.30246461)(175.11541821,724.09780913)
\curveto(175.11541672,723.96756651)(175.15634871,723.85035217)(175.23821431,723.74616577)
\curveto(175.32007668,723.63825004)(175.44845428,723.54894388)(175.62334751,723.47824702)
\curveto(175.72381494,723.44103227)(176.0196416,723.3554472)(176.51082837,723.22149155)
\curveto(177.22155368,723.03171236)(177.71738893,722.87635685)(177.9983356,722.75542456)
\curveto(178.27927353,722.634486)(178.49974811,722.4586645)(178.65976002,722.22795952)
\curveto(178.81976186,721.99724933)(178.89976529,721.7107254)(178.89977057,721.36838686)
\curveto(178.89976529,721.03348701)(178.80208668,720.71812463)(178.60673443,720.42229877)
\curveto(178.41137222,720.12647132)(178.12949965,719.89762428)(177.76111588,719.73575697)
\curveto(177.39272383,719.57388945)(176.97596174,719.49295574)(176.51082837,719.4929556)
\curveto(175.74055985,719.49295574)(175.1535579,719.65296261)(174.74982075,719.9729767)
\curveto(174.34608136,720.2929901)(174.08839587,720.76742907)(173.97676352,721.39629506)
\closepath
}
}
{
\newrgbcolor{curcolor}{0 0 0}
\pscustom[linestyle=none,fillstyle=solid,fillcolor=curcolor]
{
\newpath
\moveto(188.26576464,720.59253881)
\lineto(188.26576464,719.62691498)
\lineto(182.85715486,719.62691498)
\curveto(182.84971233,719.86878583)(182.88878378,720.10135396)(182.97436931,720.32462006)
\curveto(183.11204918,720.69300727)(183.33252377,721.05581355)(183.63579373,721.41303998)
\curveto(183.93906144,721.77026284)(184.37721979,722.18330383)(184.9502701,722.6521642)
\curveto(185.83960818,723.38149483)(186.44056422,723.95919405)(186.75314002,724.38526362)
\curveto(187.06570735,724.81132367)(187.22199313,725.21413167)(187.22199784,725.59368882)
\curveto(187.22199313,725.99183949)(187.07966143,726.32766786)(186.79500233,726.60117496)
\curveto(186.51033466,726.8746681)(186.13915593,727.01141816)(185.68146502,727.01142554)
\curveto(185.19772015,727.01141816)(184.81072678,726.86629565)(184.52048377,726.57605757)
\curveto(184.23023674,726.2858056)(184.08325368,725.88392788)(184.07953416,725.3704232)
\lineto(183.04693064,725.47647437)
\curveto(183.11763082,726.24673416)(183.38368875,726.83373611)(183.84510525,727.23748199)
\curveto(184.30651908,727.64121265)(184.92608057,727.84308178)(185.70379158,727.84309)
\curveto(186.48893839,727.84308178)(187.11036043,727.62539801)(187.56805956,727.19003804)
\curveto(188.02574858,726.75466295)(188.25459561,726.21510489)(188.25460136,725.57136226)
\curveto(188.25459561,725.24390039)(188.18761599,724.9220261)(188.05366229,724.60573843)
\curveto(187.91969751,724.2894408)(187.69736238,723.95640324)(187.38665624,723.60662475)
\curveto(187.07594034,723.25683831)(186.5596391,722.7768177)(185.83775096,722.16656147)
\curveto(185.23493105,721.66049068)(184.84793768,721.31722013)(184.67676971,721.13674877)
\curveto(184.5055974,720.95627439)(184.36419598,720.77487125)(184.25256502,720.59253881)
\closepath
}
}
{
\newrgbcolor{curcolor}{0 0 0}
\pscustom[linestyle=none,fillstyle=solid,fillcolor=curcolor]
{
\newpath
\moveto(271.8122522,720.17690277)
\lineto(271.8122522,728.35958795)
\lineto(274.88215455,728.35958795)
\curveto(275.50729377,728.35957976)(276.00871065,728.27678551)(276.3864067,728.11120494)
\curveto(276.76409193,727.9456085)(277.05991859,727.69071383)(277.27388756,727.34652017)
\curveto(277.48784394,727.00231217)(277.59482528,726.64229671)(277.5948319,726.26647271)
\curveto(277.59482528,725.91668415)(277.49993748,725.58736769)(277.31016823,725.27852231)
\curveto(277.1203863,724.96966674)(276.83386237,724.72035371)(276.45059557,724.53058247)
\curveto(276.94549507,724.3854556)(277.32597653,724.13800312)(277.59204108,723.78822426)
\curveto(277.8580924,723.43843819)(277.99112137,723.0253972)(277.99112839,722.54910004)
\curveto(277.99112137,722.1658254)(277.91018766,721.80953103)(277.74832702,721.48021586)
\curveto(277.58645283,721.15089809)(277.38644424,720.89693369)(277.14830065,720.71832191)
\curveto(276.91014471,720.53970905)(276.61152724,720.40481954)(276.25244733,720.31365297)
\curveto(275.89335686,720.22248612)(275.45333796,720.17690277)(274.93238932,720.17690277)
\closepath
\moveto(272.89509048,724.92129731)
\lineto(274.66447057,724.92129731)
\curveto(275.14448749,724.92129257)(275.48868832,724.95292183)(275.69707408,725.0161852)
\curveto(275.97243002,725.09804435)(276.17988079,725.23386413)(276.31942701,725.42364497)
\curveto(276.45896255,725.61341532)(276.52873299,725.85156508)(276.52873854,726.13809497)
\curveto(276.52873299,726.40972858)(276.46361391,726.64880862)(276.33338112,726.8553358)
\curveto(276.20313761,727.06184961)(276.0170831,727.20325104)(275.77521705,727.27954048)
\curveto(275.5333414,727.35581573)(275.11843986,727.3939569)(274.53051119,727.39396412)
\lineto(272.89509048,727.39396412)
\closepath
\moveto(272.89509048,721.1425266)
\lineto(274.93238932,721.1425266)
\curveto(275.28216782,721.14252564)(275.52775976,721.15554945)(275.66916588,721.18159809)
\curveto(275.91847422,721.22625016)(276.12685526,721.30067196)(276.29430963,721.40486371)
\curveto(276.46175336,721.509053)(276.5994337,721.66068742)(276.70735104,721.85976742)
\curveto(276.81525692,722.05884406)(276.86921272,722.28862137)(276.86921862,722.54910004)
\curveto(276.86921272,722.85422705)(276.79106983,723.11935472)(276.63478971,723.34448383)
\curveto(276.47849827,723.56960661)(276.26174477,723.72775294)(275.98452858,723.81892329)
\curveto(275.70730236,723.91008635)(275.30821545,723.9556697)(274.78726666,723.95567348)
\lineto(272.89509048,723.95567348)
\closepath
}
}
{
\newrgbcolor{curcolor}{0 0 0}
\pscustom[linestyle=none,fillstyle=solid,fillcolor=curcolor]
{
\newpath
\moveto(279.12420153,722.80585551)
\lineto(280.14564177,722.89516176)
\curveto(280.1940144,722.48583914)(280.30657738,722.15001076)(280.48333103,721.88767563)
\curveto(280.66008093,721.62533707)(280.93451132,721.41323494)(281.30662302,721.25136859)
\curveto(281.67872933,721.0895001)(282.09735196,721.00856639)(282.56249217,721.00856723)
\curveto(282.97552921,721.00856639)(283.34019603,721.06996438)(283.65649373,721.19276137)
\curveto(283.97278134,721.31555632)(284.20814028,721.48393565)(284.36257127,721.69789985)
\curveto(284.51699076,721.911861)(284.59420337,722.1453594)(284.59420936,722.39839575)
\curveto(284.59420337,722.65514874)(284.51978157,722.87934441)(284.37094373,723.07098344)
\curveto(284.22209437,723.26261669)(283.97650243,723.42355383)(283.63416717,723.55379536)
\curveto(283.41461783,723.63937705)(282.92901558,723.77240602)(282.17735896,723.95288266)
\curveto(281.42569521,724.13335175)(280.89916097,724.30359162)(280.59775466,724.46360278)
\curveto(280.20703822,724.66825845)(279.91586292,724.92222284)(279.7242279,725.22549673)
\curveto(279.53259065,725.52876052)(279.43677258,725.86830998)(279.43677341,726.24414615)
\curveto(279.43677258,726.65718107)(279.55398692,727.04324416)(279.78841677,727.40233658)
\curveto(280.02284426,727.76141454)(280.36518455,728.03398439)(280.81543865,728.22004693)
\curveto(281.26568834,728.40609339)(281.76617495,728.49912064)(282.31689998,728.49912896)
\curveto(282.92343395,728.49912064)(283.45834064,728.40144203)(283.92162166,728.20609283)
\curveto(284.38489206,728.01072757)(284.74118643,727.72327337)(284.99050584,727.34372935)
\curveto(285.2398125,726.964171)(285.37377174,726.5343851)(285.39238397,726.05437036)
\lineto(284.35419881,725.97622739)
\curveto(284.29837672,726.49345311)(284.1095314,726.88416756)(283.78766229,727.14837193)
\curveto(283.46578282,727.41256235)(282.99041357,727.54466105)(282.3615531,727.54466841)
\curveto(281.70663751,727.54466105)(281.22940771,727.42465589)(280.92986228,727.18465259)
\curveto(280.63031221,726.94463528)(280.48053834,726.65532053)(280.48054021,726.31670747)
\curveto(280.48053834,726.02273522)(280.5865894,725.78086437)(280.79869372,725.59109419)
\curveto(281.00707258,725.40131318)(281.551282,725.20688623)(282.43132361,725.00781274)
\curveto(283.31135758,724.8087296)(283.91510444,724.63476864)(284.242566,724.48592934)
\curveto(284.71885989,724.26638072)(285.0705029,723.98822924)(285.29749608,723.65147407)
\curveto(285.52447588,723.31471195)(285.63796913,722.92678831)(285.63797616,722.487702)
\curveto(285.63796913,722.05233215)(285.51331261,721.64208197)(285.26400623,721.25695023)
\curveto(285.01468655,720.87181634)(284.65653163,720.57226859)(284.18954041,720.35830609)
\curveto(283.72253803,720.14434323)(283.19693407,720.03736189)(282.61272693,720.03736176)
\curveto(281.87222601,720.03736189)(281.25173425,720.14527351)(280.75124978,720.36109691)
\curveto(280.25076103,720.57691995)(279.85818603,720.90158506)(279.5735236,721.3350932)
\curveto(279.28885925,721.76859903)(279.13908538,722.25885265)(279.12420153,722.80585551)
\closepath
}
}
{
\newrgbcolor{curcolor}{0 0 0}
\pscustom[linestyle=none,fillstyle=solid,fillcolor=curcolor]
{
\newpath
\moveto(287.1282743,720.17690277)
\lineto(287.1282743,728.35958795)
\lineto(289.94700283,728.35958795)
\curveto(290.58330552,728.35957976)(291.06890777,728.32050832)(291.40381103,728.24237349)
\curveto(291.87266322,728.13445382)(292.2726804,727.93909659)(292.60386377,727.65630123)
\curveto(293.03550386,727.29162692)(293.35830842,726.8255604)(293.57227842,726.25810025)
\curveto(293.78623377,725.69062793)(293.89321511,725.042228)(293.89322276,724.31289848)
\curveto(293.89321511,723.69147231)(293.82065386,723.14075099)(293.67553877,722.66073286)
\curveto(293.53040883,722.18070976)(293.34435433,721.7834834)(293.11737471,721.46905258)
\curveto(292.89038135,721.15461918)(292.64199859,720.90716669)(292.37222568,720.72669437)
\curveto(292.10244053,720.54622096)(291.77684515,720.4094709)(291.39543857,720.31644379)
\curveto(291.01402169,720.2234164)(290.57586334,720.17690277)(290.0809622,720.17690277)
\closepath
\moveto(288.21111259,721.1425266)
\lineto(289.95816611,721.1425266)
\curveto(290.49772045,721.14252564)(290.92099444,721.19276035)(291.22798935,721.2932309)
\curveto(291.5349743,721.39369921)(291.77963597,721.53510064)(291.9619751,721.71743559)
\curveto(292.21872459,721.97418926)(292.41873318,722.31932036)(292.56200147,722.75282993)
\curveto(292.70525712,723.18633434)(292.7768881,723.71193831)(292.77689463,724.32964341)
\curveto(292.7768881,725.18548996)(292.63641695,725.84319263)(292.35548076,726.30275337)
\curveto(292.07453236,726.76230187)(291.73312234,727.07022207)(291.33124971,727.2265149)
\curveto(291.0409996,727.33814055)(290.5740028,727.3939569)(289.93025791,727.39396412)
\lineto(288.21111259,727.39396412)
\closepath
}
}
{
\newrgbcolor{curcolor}{0 0 0}
\pscustom[linestyle=none,fillstyle=solid,fillcolor=curcolor]
{
\newpath
\moveto(298.5483101,720.17690277)
\lineto(298.5483101,728.35958795)
\lineto(299.65905658,728.35958795)
\lineto(303.95691988,721.93511957)
\lineto(303.95691988,728.35958795)
\lineto(304.99510503,728.35958795)
\lineto(304.99510503,720.17690277)
\lineto(303.88435855,720.17690277)
\lineto(299.58649526,726.60695279)
\lineto(299.58649526,720.17690277)
\closepath
}
}
{
\newrgbcolor{curcolor}{0 0 0}
\pscustom[linestyle=none,fillstyle=solid,fillcolor=curcolor]
{
\newpath
\moveto(310.74977831,722.08582387)
\lineto(311.78796347,721.95744613)
\curveto(311.62422966,721.35090668)(311.32096082,720.88018879)(310.87815605,720.54529105)
\curveto(310.43534139,720.21039258)(309.86973571,720.04294353)(309.18133729,720.0429434)
\curveto(308.31432007,720.04294353)(307.62684869,720.30993174)(307.11892108,720.84390883)
\curveto(306.61099111,721.37788458)(306.35702671,722.12675395)(306.35702713,723.09051918)
\curveto(306.35702671,724.0877684)(306.61378193,724.86175513)(307.12729354,725.41248169)
\curveto(307.64080278,725.96319778)(308.30687789,726.23855844)(309.12552089,726.2385645)
\curveto(309.91810988,726.23855844)(310.56557955,725.96877942)(311.06793183,725.42922661)
\curveto(311.57027386,724.8896633)(311.82144743,724.13056094)(311.82145332,723.15191723)
\curveto(311.82144743,723.09237682)(311.81958689,723.00307066)(311.81587168,722.88399848)
\lineto(307.39521229,722.88399848)
\curveto(307.43242173,722.23280502)(307.61661569,721.73417895)(307.94779471,721.38811879)
\curveto(308.27896972,721.04205621)(308.69201071,720.86902552)(309.18691893,720.86902621)
\curveto(309.5553036,720.86902552)(309.86973571,720.96577386)(310.1302162,721.15927152)
\curveto(310.39068831,721.35276722)(310.59720881,721.6616177)(310.74977831,722.08582387)
\closepath
\moveto(307.4510287,723.7100813)
\lineto(310.76094159,723.7100813)
\curveto(310.71628369,724.20870383)(310.58976663,724.58267338)(310.38139003,724.83199106)
\curveto(310.06137185,725.21897977)(309.64647031,725.41247645)(309.13668417,725.41248169)
\curveto(308.67526581,725.41247645)(308.28734217,725.25805122)(307.9729121,724.94920552)
\curveto(307.65847795,724.64035027)(307.48451699,724.22730928)(307.4510287,723.7100813)
\closepath
}
}
{
\newrgbcolor{curcolor}{0 0 0}
\pscustom[linestyle=none,fillstyle=solid,fillcolor=curcolor]
{
\newpath
\moveto(315.24857797,721.07554691)
\lineto(315.39370062,720.18806605)
\curveto(315.11089469,720.1285286)(314.85786057,720.09875988)(314.6345975,720.0987598)
\curveto(314.26992834,720.09875988)(313.9871255,720.15643678)(313.78618812,720.27179066)
\curveto(313.58524777,720.38714436)(313.44384635,720.53877878)(313.36198343,720.72669437)
\curveto(313.28011839,720.91460887)(313.2391864,721.30997469)(313.23918734,721.91279301)
\lineto(313.23918734,725.32317544)
\lineto(312.50241077,725.32317544)
\lineto(312.50241077,726.10460513)
\lineto(313.23918734,726.10460513)
\lineto(313.23918734,727.57257662)
\lineto(314.23830101,728.17539381)
\lineto(314.23830101,726.10460513)
\lineto(315.24857797,726.10460513)
\lineto(315.24857797,725.32317544)
\lineto(314.23830101,725.32317544)
\lineto(314.23830101,721.8569766)
\curveto(314.23829907,721.57045099)(314.25597425,721.38625703)(314.2913266,721.30439418)
\curveto(314.32667496,721.22252907)(314.38435186,721.15741)(314.46435746,721.10903676)
\curveto(314.54435873,721.06066166)(314.65878225,721.03647457)(314.80762836,721.03647543)
\curveto(314.91925855,721.03647457)(315.06624161,721.04949839)(315.24857797,721.07554691)
\closepath
}
}
{
\newrgbcolor{curcolor}{0 0 0}
\pscustom[linestyle=none,fillstyle=solid,fillcolor=curcolor]
{
\newpath
\moveto(315.47184372,720.03736176)
\lineto(317.84404099,728.49912896)
\lineto(318.64779724,728.49912896)
\lineto(316.28118161,720.03736176)
\closepath
}
}
{
\newrgbcolor{curcolor}{0 0 0}
\pscustom[linestyle=none,fillstyle=solid,fillcolor=curcolor]
{
\newpath
\moveto(324.39688723,721.1425266)
\lineto(324.39688723,720.17690277)
\lineto(318.98827745,720.17690277)
\curveto(318.98083492,720.41877362)(319.01990637,720.65134175)(319.1054919,720.87460785)
\curveto(319.24317177,721.24299507)(319.46364635,721.60580135)(319.76691632,721.96302778)
\curveto(320.07018403,722.32025063)(320.50834238,722.73329163)(321.08139269,723.202152)
\curveto(321.97073077,723.93148262)(322.57168681,724.50918185)(322.88426261,724.93525142)
\curveto(323.19682994,725.36131147)(323.35311572,725.76411946)(323.35312043,726.14367661)
\curveto(323.35311572,726.54182728)(323.21078402,726.87765566)(322.92612492,727.15116275)
\curveto(322.64145725,727.42465589)(322.27027852,727.56140595)(321.81258761,727.56141334)
\curveto(321.32884274,727.56140595)(320.94184937,727.41628344)(320.65160636,727.12604537)
\curveto(320.36135933,726.83579339)(320.21437627,726.43391567)(320.21065675,725.92041099)
\lineto(319.17805323,726.02646216)
\curveto(319.2487534,726.79672195)(319.51481134,727.3837239)(319.97622784,727.78746978)
\curveto(320.43764167,728.19120044)(321.05720316,728.39306957)(321.83491417,728.39307779)
\curveto(322.62006098,728.39306957)(323.24148302,728.17538581)(323.69918215,727.74002584)
\curveto(324.15687116,727.30465074)(324.3857182,726.76509268)(324.38572395,726.12135005)
\curveto(324.3857182,725.79388818)(324.31873858,725.4720139)(324.18478488,725.15572622)
\curveto(324.0508201,724.83942859)(323.82848497,724.50639103)(323.51777883,724.15661255)
\curveto(323.20706293,723.8068261)(322.69076169,723.32680549)(321.96887355,722.71654926)
\curveto(321.36605364,722.21047848)(320.97906027,721.86720792)(320.8078923,721.68673656)
\curveto(320.63671999,721.50626219)(320.49531857,721.32485905)(320.38368761,721.1425266)
\closepath
}
}
{
\newrgbcolor{curcolor}{0 0 0}
\pscustom[linestyle=none,fillstyle=solid,fillcolor=curcolor]
{
\newpath
\moveto(868.82021729,720.26192474)
\lineto(868.82021729,728.44460992)
\lineto(869.90305557,728.44460992)
\lineto(869.90305557,721.22754857)
\lineto(873.93300011,721.22754857)
\lineto(873.93300011,720.26192474)
\closepath
}
}
{
\newrgbcolor{curcolor}{0 0 0}
\pscustom[linestyle=none,fillstyle=solid,fillcolor=curcolor]
{
\newpath
\moveto(875.10514481,727.28921031)
\lineto(875.10514481,728.44460992)
\lineto(876.10984012,728.44460992)
\lineto(876.10984012,727.28921031)
\closepath
\moveto(875.10514481,720.26192474)
\lineto(875.10514481,726.1896271)
\lineto(876.10984012,726.1896271)
\lineto(876.10984012,720.26192474)
\closepath
}
}
{
\newrgbcolor{curcolor}{0 0 0}
\pscustom[linestyle=none,fillstyle=solid,fillcolor=curcolor]
{
\newpath
\moveto(877.64479117,720.26192474)
\lineto(877.64479117,726.1896271)
\lineto(878.54901695,726.1896271)
\lineto(878.54901695,725.34679937)
\curveto(878.98438283,725.99798504)(879.61324705,726.32358042)(880.43561149,726.32358648)
\curveto(880.79283259,726.32358042)(881.12121878,726.25939161)(881.42077106,726.13101988)
\curveto(881.72031428,726.0026364)(881.94450996,725.83425708)(882.09335876,725.6258814)
\curveto(882.24219716,725.41749499)(882.34638768,725.1700425)(882.40593063,724.88352319)
\curveto(882.44313602,724.69746407)(882.46174147,724.37186869)(882.46174704,723.90673608)
\lineto(882.46174704,720.26192474)
\lineto(881.45705173,720.26192474)
\lineto(881.45705173,723.8676646)
\curveto(881.45704716,724.27698089)(881.41797571,724.58304055)(881.33983727,724.78584448)
\curveto(881.26168993,724.98863936)(881.12307933,725.15050678)(880.92400505,725.27144722)
\curveto(880.7249227,725.39237763)(880.4914243,725.45284535)(880.22350914,725.45285054)
\curveto(879.79558046,725.45284535)(879.42626227,725.31702556)(879.11555348,725.04539077)
\curveto(878.80484024,724.77374641)(878.64948473,724.25837544)(878.64948648,723.49927631)
\lineto(878.64948648,720.26192474)
\closepath
}
}
{
\newrgbcolor{curcolor}{0 0 0}
\pscustom[linestyle=none,fillstyle=solid,fillcolor=curcolor]
{
\newpath
\moveto(887.89268354,720.26192474)
\lineto(887.89268354,721.13266068)
\curveto(887.43126374,720.46286361)(886.80426007,720.1279655)(886.01167065,720.12796537)
\curveto(885.66188543,720.1279655)(885.33535978,720.19494512)(885.03209272,720.32890443)
\curveto(884.7288221,720.46286361)(884.50369615,720.63124293)(884.3567142,720.83404291)
\curveto(884.20973004,721.03684174)(884.10646979,721.2852245)(884.04693314,721.57919193)
\curveto(884.00600036,721.77640839)(883.98553436,722.08897995)(883.9855351,722.51690756)
\lineto(883.9855351,726.1896271)
\lineto(884.99023041,726.1896271)
\lineto(884.99023041,722.90204077)
\curveto(884.99022867,722.37736443)(885.01069467,722.02386088)(885.05162846,721.84152904)
\curveto(885.11488519,721.57733007)(885.24884443,721.3698793)(885.45350658,721.21917611)
\curveto(885.65816434,721.06847101)(885.91119846,720.99311894)(886.21260971,720.99311967)
\curveto(886.51401504,720.99311894)(886.79681789,721.07033155)(887.06101909,721.22475775)
\curveto(887.32521267,721.37918203)(887.51219745,721.58942361)(887.62197397,721.85548315)
\curveto(887.73174176,722.12153949)(887.78662784,722.50760258)(887.78663237,723.01367358)
\lineto(887.78663237,726.1896271)
\lineto(888.79132769,726.1896271)
\lineto(888.79132769,720.26192474)
\closepath
}
}
{
\newrgbcolor{curcolor}{0 0 0}
\pscustom[linestyle=none,fillstyle=solid,fillcolor=curcolor]
{
\newpath
\moveto(889.70113527,720.26192474)
\lineto(891.86681184,723.34299038)
\lineto(889.86300285,726.1896271)
\lineto(891.11887199,726.1896271)
\lineto(892.02867942,724.79979858)
\curveto(892.19984715,724.53559665)(892.33752748,724.3141918)(892.44172082,724.13558335)
\curveto(892.60544596,724.38117142)(892.75615011,724.59885518)(892.89383371,724.7886353)
\lineto(893.89294739,726.1896271)
\lineto(895.09300013,726.1896271)
\lineto(893.04453801,723.39880678)
\lineto(895.24928606,720.26192474)
\lineto(894.01574348,720.26192474)
\lineto(892.79894582,722.10386615)
\lineto(892.47521067,722.60063217)
\lineto(890.91793293,720.26192474)
\closepath
}
}
{
\newrgbcolor{curcolor}{0 0 0}
\pscustom[linestyle=none,fillstyle=solid,fillcolor=curcolor]
{
\newpath
\moveto(898.97782309,724.29745093)
\curveto(898.97782262,725.2649303)(899.07736177,726.04356839)(899.27644086,726.63336753)
\curveto(899.47551841,727.22315393)(899.77134507,727.67805719)(900.16392173,727.99807867)
\curveto(900.55649506,728.31808468)(901.05046977,728.47809155)(901.64584732,728.47809976)
\curveto(902.0849328,728.47809155)(902.47006562,728.38971566)(902.80124693,728.21297183)
\curveto(903.13241964,728.03621211)(903.40591976,727.78131744)(903.6217481,727.44828707)
\curveto(903.83756621,727.11524232)(904.0068758,726.70964351)(904.1296774,726.23148941)
\curveto(904.25246774,725.75332337)(904.31386573,725.10864452)(904.31387154,724.29745093)
\curveto(904.31386573,723.33740566)(904.21525684,722.56248866)(904.01804459,721.9726976)
\curveto(903.8208213,721.38290312)(903.52592492,720.92706959)(903.13335455,720.60519564)
\curveto(902.74077492,720.28332101)(902.24493967,720.12238387)(901.64584732,720.12238373)
\curveto(900.85697309,720.12238387)(900.23741159,720.40518671)(899.78716098,720.97079311)
\curveto(899.24760164,721.65175187)(898.97782262,722.7606367)(898.97782309,724.29745093)
\closepath
\moveto(900.01042661,724.29745093)
\curveto(900.0104251,722.95413339)(900.16764116,722.0601415)(900.48207524,721.6154726)
\curveto(900.79650537,721.17080099)(901.18442901,720.94846586)(901.64584732,720.94846654)
\curveto(902.10725934,720.94846586)(902.49518297,721.17173126)(902.80961939,721.61826342)
\curveto(903.12404719,722.06479287)(903.28126325,722.95785448)(903.28126802,724.29745093)
\curveto(903.28126325,725.64448148)(903.12404719,726.53940364)(902.80961939,726.98222007)
\curveto(902.49518297,727.42502307)(902.10353825,727.64642792)(901.63468403,727.64643531)
\curveto(901.17326574,727.64642792)(900.80487782,727.4510707)(900.52951919,727.06036304)
\curveto(900.18345579,726.56173018)(900.0104251,725.64076039)(900.01042661,724.29745093)
\closepath
}
}
{
\newrgbcolor{curcolor}{0 0 0}
\pscustom[linestyle=none,fillstyle=solid,fillcolor=curcolor]
{
\newpath
\moveto(905.90464024,720.26192474)
\lineto(905.90464024,721.40616107)
\lineto(907.04887657,721.40616107)
\lineto(907.04887657,720.26192474)
\closepath
}
}
{
\newrgbcolor{curcolor}{0 0 0}
\pscustom[linestyle=none,fillstyle=solid,fillcolor=curcolor]
{
\newpath
\moveto(908.51126273,724.29745093)
\curveto(908.51126225,725.2649303)(908.61080141,726.04356839)(908.8098805,726.63336753)
\curveto(909.00895805,727.22315393)(909.3047847,727.67805719)(909.69736136,727.99807867)
\curveto(910.0899347,728.31808468)(910.5839094,728.47809155)(911.17928695,728.47809976)
\curveto(911.61837243,728.47809155)(912.00350525,728.38971566)(912.33468656,728.21297183)
\curveto(912.66585928,728.03621211)(912.9393594,727.78131744)(913.15518774,727.44828707)
\curveto(913.37100584,727.11524232)(913.54031544,726.70964351)(913.66311704,726.23148941)
\curveto(913.78590738,725.75332337)(913.84730537,725.10864452)(913.84731118,724.29745093)
\curveto(913.84730537,723.33740566)(913.74869648,722.56248866)(913.55148422,721.9726976)
\curveto(913.35426094,721.38290312)(913.05936455,720.92706959)(912.66679418,720.60519564)
\curveto(912.27421455,720.28332101)(911.77837931,720.12238387)(911.17928695,720.12238373)
\curveto(910.39041272,720.12238387)(909.77085123,720.40518671)(909.32060062,720.97079311)
\curveto(908.78104128,721.65175187)(908.51126225,722.7606367)(908.51126273,724.29745093)
\closepath
\moveto(909.54386625,724.29745093)
\curveto(909.54386474,722.95413339)(909.70108079,722.0601415)(910.01551488,721.6154726)
\curveto(910.32994501,721.17080099)(910.71786864,720.94846586)(911.17928695,720.94846654)
\curveto(911.64069897,720.94846586)(912.02862261,721.17173126)(912.34305903,721.61826342)
\curveto(912.65748683,722.06479287)(912.81470288,722.95785448)(912.81470766,724.29745093)
\curveto(912.81470288,725.64448148)(912.65748683,726.53940364)(912.34305903,726.98222007)
\curveto(912.02862261,727.42502307)(911.63697788,727.64642792)(911.16812367,727.64643531)
\curveto(910.70670537,727.64642792)(910.33831746,727.4510707)(910.06295882,727.06036304)
\curveto(909.71689542,726.56173018)(909.54386474,725.64076039)(909.54386625,724.29745093)
\closepath
}
}
{
\newrgbcolor{curcolor}{0 0 0}
\pscustom[linestyle=none,fillstyle=solid,fillcolor=curcolor]
{
\newpath
\moveto(915.43807987,720.26192474)
\lineto(915.43807987,721.40616107)
\lineto(916.5823162,721.40616107)
\lineto(916.5823162,720.26192474)
\closepath
}
}
{
\newrgbcolor{curcolor}{0 0 0}
\pscustom[linestyle=none,fillstyle=solid,fillcolor=curcolor]
{
\newpath
\moveto(921.82905853,720.26192474)
\lineto(920.82436322,720.26192474)
\lineto(920.82436322,726.66406656)
\curveto(920.58248911,726.43335257)(920.26526618,726.20264499)(919.87269349,725.97194312)
\curveto(919.48011619,725.74122982)(919.1275429,725.56819914)(918.81497259,725.45285054)
\lineto(918.81497259,726.42405601)
\curveto(919.37685594,726.68824724)(919.86803982,727.00826098)(920.28852571,727.3840982)
\curveto(920.70900617,727.75992117)(921.00669337,728.12458799)(921.18158822,728.47809976)
\lineto(921.82905853,728.47809976)
\closepath
}
}
{
\newrgbcolor{curcolor}{0 0 0}
\pscustom[linestyle=none,fillstyle=solid,fillcolor=curcolor]
{
\newpath
\moveto(69.30773863,684.68960504)
\lineto(70.39057692,684.68960504)
\lineto(70.39057692,679.96195542)
\curveto(70.39056958,679.13959106)(70.29754233,678.48653976)(70.11149488,678.00279955)
\curveto(69.92543333,677.51905635)(69.58960495,677.12555108)(69.10400875,676.82228256)
\curveto(68.61840045,676.5190134)(67.98116378,676.36737898)(67.19229683,676.36737885)
\curveto(66.42574815,676.36737898)(65.79874447,676.49947768)(65.31128393,676.76367533)
\curveto(64.82381888,677.02787247)(64.47589697,677.41021447)(64.26751714,677.91070248)
\curveto(64.05913488,678.41118769)(63.95494436,679.09493798)(63.95494526,679.96195542)
\lineto(63.95494526,684.68960504)
\lineto(65.03778354,684.68960504)
\lineto(65.03778354,679.96753706)
\curveto(65.03778156,679.2568054)(65.10383091,678.73306198)(65.23593179,678.39630522)
\curveto(65.3680283,678.05954468)(65.5950148,677.79999865)(65.91689194,677.61766635)
\curveto(66.23876337,677.43533183)(66.63226864,677.34416512)(67.09740894,677.34416596)
\curveto(67.89371817,677.34416512)(68.4611844,677.52463799)(68.79980933,677.8855851)
\curveto(69.13842278,678.24652945)(69.30773238,678.94051275)(69.30773863,679.96753706)
\closepath
}
}
{
\newrgbcolor{curcolor}{0 0 0}
\pscustom[linestyle=none,fillstyle=solid,fillcolor=curcolor]
{
\newpath
\moveto(72.07065062,676.50691986)
\lineto(72.07065062,682.43462222)
\lineto(72.9748764,682.43462222)
\lineto(72.9748764,681.59179448)
\curveto(73.41024228,682.24298015)(74.0391065,682.56857553)(74.86147094,682.56858159)
\curveto(75.21869204,682.56857553)(75.54707824,682.50438673)(75.84663051,682.37601499)
\curveto(76.14617373,682.24763152)(76.37036941,682.07925219)(76.51921821,681.87087651)
\curveto(76.66805661,681.66249011)(76.77224713,681.41503762)(76.83179009,681.12851831)
\curveto(76.86899547,680.94245919)(76.88760092,680.61686381)(76.88760649,680.1517312)
\lineto(76.88760649,676.50691986)
\lineto(75.88291118,676.50691986)
\lineto(75.88291118,680.11265971)
\curveto(75.88290661,680.52197601)(75.84383517,680.82803567)(75.76569672,681.0308396)
\curveto(75.68754938,681.23363448)(75.54893878,681.3955019)(75.3498645,681.51644233)
\curveto(75.15078215,681.63737275)(74.91728375,681.69784046)(74.6493686,681.69784565)
\curveto(74.22143991,681.69784046)(73.85212172,681.56202068)(73.54141293,681.29038589)
\curveto(73.23069969,681.01874153)(73.07534418,680.50337056)(73.07534594,679.74427143)
\lineto(73.07534594,676.50691986)
\closepath
}
}
{
\newrgbcolor{curcolor}{0 0 0}
\pscustom[linestyle=none,fillstyle=solid,fillcolor=curcolor]
{
\newpath
\moveto(78.43930275,683.53420542)
\lineto(78.43930275,684.68960504)
\lineto(79.44399806,684.68960504)
\lineto(79.44399806,683.53420542)
\closepath
\moveto(78.43930275,676.50691986)
\lineto(78.43930275,682.43462222)
\lineto(79.44399806,682.43462222)
\lineto(79.44399806,676.50691986)
\closepath
}
}
{
\newrgbcolor{curcolor}{0 0 0}
\pscustom[linestyle=none,fillstyle=solid,fillcolor=curcolor]
{
\newpath
\moveto(80.30915224,676.50691986)
\lineto(82.4748288,679.58798549)
\lineto(80.47101982,682.43462222)
\lineto(81.72688896,682.43462222)
\lineto(82.63669638,681.0447937)
\curveto(82.80786411,680.78059177)(82.94554445,680.55918691)(83.04973779,680.38057846)
\curveto(83.21346293,680.62616653)(83.36416707,680.8438503)(83.50185068,681.03363042)
\lineto(84.50096436,682.43462222)
\lineto(85.70101709,682.43462222)
\lineto(83.65255498,679.6438019)
\lineto(85.85730303,676.50691986)
\lineto(84.62376045,676.50691986)
\lineto(83.40696279,678.34886127)
\lineto(83.08322763,678.84562729)
\lineto(81.5259499,676.50691986)
\closepath
}
}
{
\newrgbcolor{curcolor}{0 0 0}
\pscustom[linestyle=none,fillstyle=solid,fillcolor=curcolor]
{
\newpath
\moveto(88.25182782,676.50691986)
\lineto(86.08056961,684.68960504)
\lineto(87.1913161,684.68960504)
\lineto(88.43602196,679.32564838)
\curveto(88.56997871,678.76376097)(88.6853325,678.20559746)(88.78208368,677.65115619)
\curveto(88.99046188,678.52561121)(89.11325785,679.02981891)(89.15047196,679.16378081)
\lineto(90.7077497,684.68960504)
\lineto(92.01385361,684.68960504)
\lineto(93.18599814,680.54802768)
\curveto(93.47995701,679.52100279)(93.69205914,678.55537993)(93.82230518,677.65115619)
\curveto(93.92648782,678.16838656)(94.0623076,678.76190042)(94.22976494,679.43169956)
\lineto(95.51354229,684.68960504)
\lineto(96.60196221,684.68960504)
\lineto(94.35814268,676.50691986)
\lineto(93.31437588,676.50691986)
\lineto(91.58964892,682.74161245)
\curveto(91.44452076,683.26255882)(91.35893569,683.58257257)(91.33289345,683.70165464)
\curveto(91.24730299,683.32581736)(91.16729955,683.00580361)(91.0928829,682.74161245)
\lineto(89.35699267,676.50691986)
\closepath
}
}
{
\newrgbcolor{curcolor}{0 0 0}
\pscustom[linestyle=none,fillstyle=solid,fillcolor=curcolor]
{
\newpath
\moveto(101.3463566,677.23811478)
\curveto(100.97424297,676.9218214)(100.61608806,676.698556)(100.27189077,676.56831791)
\curveto(99.9276864,676.4380797)(99.55836821,676.37296062)(99.16393511,676.37296049)
\curveto(98.51274191,676.37296062)(98.0122553,676.53203722)(97.66247378,676.85019076)
\curveto(97.31269037,677.16834362)(97.13779914,677.5748727)(97.13779956,678.06977924)
\curveto(97.13779914,678.3600227)(97.20384849,678.62515037)(97.3359478,678.86516303)
\curveto(97.46804588,679.10517098)(97.64107657,679.29773739)(97.85504038,679.44286284)
\curveto(98.06900192,679.58798241)(98.3099425,679.69775457)(98.57786284,679.77217963)
\curveto(98.77507876,679.82427163)(99.07276596,679.87450635)(99.47092534,679.92288393)
\curveto(100.28212022,680.01962886)(100.87935518,680.13498265)(101.26263199,680.26894565)
\curveto(101.26634854,680.40662222)(101.26820908,680.49406784)(101.26821363,680.53128276)
\curveto(101.26820908,680.94059864)(101.17332129,681.22898312)(100.98354996,681.39643706)
\curveto(100.72679048,681.62341866)(100.34537876,681.73691191)(99.83931362,681.73691714)
\curveto(99.36673208,681.73691191)(99.01787988,681.65411766)(98.792756,681.48853413)
\curveto(98.56762799,681.32294064)(98.40110921,681.0299048)(98.29319917,680.60942573)
\lineto(97.31083042,680.74338511)
\curveto(97.40013599,681.16386404)(97.54711905,681.50341351)(97.75178003,681.76203452)
\curveto(97.95643895,682.02064502)(98.25226561,682.21972334)(98.63926089,682.35927007)
\curveto(99.02625234,682.49880509)(99.47464369,682.56857553)(99.98443628,682.56858159)
\curveto(100.49050127,682.56857553)(100.90168172,682.50903809)(101.21797886,682.38996909)
\curveto(101.53426702,682.27088833)(101.76683515,682.12111446)(101.91568394,681.94064702)
\curveto(102.06452235,681.76016872)(102.16871287,681.53225196)(102.22825582,681.25689604)
\curveto(102.26174012,681.08572115)(102.27848503,680.77687068)(102.27849058,680.3303437)
\lineto(102.27849058,678.99074995)
\curveto(102.27848503,678.05675386)(102.2998813,677.46603082)(102.34267945,677.21857904)
\curveto(102.38546637,676.97112584)(102.47012117,676.73390635)(102.5966441,676.50691986)
\lineto(101.54729566,676.50691986)
\curveto(101.44310032,676.7153009)(101.3761207,676.9590323)(101.3463566,677.23811478)
\closepath
\moveto(101.26263199,679.48193432)
\curveto(100.89796063,679.33308774)(100.35096039,679.20657068)(99.62162964,679.10238276)
\curveto(99.20858575,679.04284272)(98.91648018,678.9758631)(98.74531206,678.9014437)
\curveto(98.5741399,678.8270195)(98.4420412,678.71817762)(98.34901557,678.57491772)
\curveto(98.2559867,678.43165368)(98.20947307,678.27257708)(98.20947456,678.09768744)
\curveto(98.20947307,677.82976737)(98.31087278,677.60650197)(98.51367397,677.42789057)
\curveto(98.71647159,677.24927732)(99.01322852,677.15997116)(99.40394565,677.15997182)
\curveto(99.79093634,677.15997116)(100.13513717,677.24462596)(100.43654917,677.41393646)
\curveto(100.73795375,677.58324515)(100.95935861,677.81488301)(101.10076441,678.10885072)
\curveto(101.20867164,678.33583561)(101.26262745,678.67073372)(101.26263199,679.11354604)
\closepath
}
}
{
\newrgbcolor{curcolor}{0 0 0}
\pscustom[linestyle=none,fillstyle=solid,fillcolor=curcolor]
{
\newpath
\moveto(103.83018779,676.50691986)
\lineto(103.83018779,682.43462222)
\lineto(104.73441358,682.43462222)
\lineto(104.73441358,681.53597808)
\curveto(104.96511951,681.95645622)(105.17815192,682.23367743)(105.37351143,682.36764253)
\curveto(105.56886637,682.50159591)(105.78375932,682.56857553)(106.01819093,682.56858159)
\curveto(106.35680719,682.56857553)(106.70100802,682.46066392)(107.05079444,682.24484644)
\lineto(106.70473272,681.31271245)
\curveto(106.45913716,681.45783016)(106.21354522,681.53039141)(105.96795616,681.53039644)
\curveto(105.74840897,681.53039141)(105.55119119,681.46434206)(105.37630225,681.33224819)
\curveto(105.20140873,681.20014467)(105.07675222,681.01688099)(105.00233233,680.78245659)
\curveto(104.89069771,680.42522767)(104.83488136,680.03451322)(104.83488311,679.61031206)
\lineto(104.83488311,676.50691986)
\closepath
}
}
{
\newrgbcolor{curcolor}{0 0 0}
\pscustom[linestyle=none,fillstyle=solid,fillcolor=curcolor]
{
\newpath
\moveto(111.7170444,678.41584096)
\lineto(112.75522956,678.28746322)
\curveto(112.59149574,677.68092377)(112.28822691,677.21020588)(111.84542213,676.87530814)
\curveto(111.40260748,676.54040967)(110.83700179,676.37296062)(110.14860338,676.37296049)
\curveto(109.28158616,676.37296062)(108.59411477,676.63994883)(108.08618716,677.17392592)
\curveto(107.57825719,677.70790167)(107.3242928,678.45677104)(107.32429321,679.42053627)
\curveto(107.3242928,680.41778549)(107.58104801,681.19177222)(108.09455962,681.74249878)
\curveto(108.60806886,682.29321487)(109.27414398,682.56857553)(110.09278697,682.56858159)
\curveto(110.88537596,682.56857553)(111.53284563,682.29879651)(112.03519791,681.7592437)
\curveto(112.53753994,681.21968039)(112.78871352,680.46057803)(112.7887194,679.48193432)
\curveto(112.78871352,679.42239391)(112.78685297,679.33308774)(112.78313776,679.21401557)
\lineto(108.36247837,679.21401557)
\curveto(108.39968782,678.56282211)(108.58388177,678.06419604)(108.9150608,677.71813588)
\curveto(109.2462358,677.37207329)(109.65927679,677.19904261)(110.15418502,677.1990433)
\curveto(110.52256968,677.19904261)(110.83700179,677.29579095)(111.09748229,677.48928861)
\curveto(111.3579544,677.68278431)(111.56447489,677.99163479)(111.7170444,678.41584096)
\closepath
\moveto(108.41829478,680.04009839)
\lineto(111.72820768,680.04009839)
\curveto(111.68354977,680.53872092)(111.55703271,680.91269047)(111.34865611,681.16200815)
\curveto(111.02863793,681.54899686)(110.61373639,681.74249354)(110.10395025,681.74249878)
\curveto(109.64253189,681.74249354)(109.25460825,681.58806831)(108.94017818,681.27922261)
\curveto(108.62574404,680.97036736)(108.45178308,680.55732637)(108.41829478,680.04009839)
\closepath
}
}
{
\newrgbcolor{curcolor}{0 0 0}
\pscustom[linestyle=none,fillstyle=solid,fillcolor=curcolor]
{
\newpath
\moveto(120.69790542,676.50691986)
\lineto(119.69321011,676.50691986)
\lineto(119.69321011,682.90906167)
\curveto(119.451336,682.67834769)(119.13411308,682.44764011)(118.74154038,682.21693823)
\curveto(118.34896308,681.98622494)(117.9963898,681.81319426)(117.68381948,681.69784565)
\lineto(117.68381948,682.66905113)
\curveto(118.24570283,682.93324236)(118.73688672,683.2532561)(119.15737261,683.62909332)
\curveto(119.57785306,684.00491629)(119.87554027,684.36958311)(120.05043511,684.72309488)
\lineto(120.69790542,684.72309488)
\closepath
}
}
{
\newrgbcolor{curcolor}{0 0 0}
\pscustom[linestyle=none,fillstyle=solid,fillcolor=curcolor]
{
\newpath
\moveto(123.8403664,676.50691986)
\lineto(123.8403664,677.65115619)
\lineto(124.98460273,677.65115619)
\lineto(124.98460273,676.50691986)
\closepath
}
}
{
\newrgbcolor{curcolor}{0 0 0}
\pscustom[linestyle=none,fillstyle=solid,fillcolor=curcolor]
{
\newpath
\moveto(126.44699271,680.54244604)
\curveto(126.44699223,681.50992542)(126.54653139,682.28856351)(126.74561048,682.87836265)
\curveto(126.94468803,683.46814905)(127.24051468,683.92305231)(127.63309134,684.24307379)
\curveto(128.02566468,684.56307979)(128.51963938,684.72308666)(129.11501693,684.72309488)
\curveto(129.55410241,684.72308666)(129.93923523,684.63471078)(130.27041654,684.45796695)
\curveto(130.60158926,684.28120722)(130.87508938,684.02631256)(131.09091772,683.69328218)
\curveto(131.30673582,683.36023744)(131.47604542,682.95463862)(131.59884702,682.47648452)
\curveto(131.72163736,681.99831848)(131.78303535,681.35363964)(131.78304116,680.54244604)
\curveto(131.78303535,679.58240078)(131.68442646,678.80748378)(131.4872142,678.21769272)
\curveto(131.28999092,677.62789824)(130.99509453,677.17206471)(130.60252416,676.85019076)
\curveto(130.20994453,676.52831613)(129.71410929,676.36737898)(129.11501693,676.36737885)
\curveto(128.3261427,676.36737898)(127.70658121,676.65018183)(127.2563306,677.21578822)
\curveto(126.71677126,677.89674699)(126.44699223,679.00563182)(126.44699271,680.54244604)
\closepath
\moveto(127.47959623,680.54244604)
\curveto(127.47959472,679.1991285)(127.63681077,678.30513662)(127.95124486,677.86046772)
\curveto(128.26567499,677.4157961)(128.65359863,677.19346097)(129.11501693,677.19346166)
\curveto(129.57642895,677.19346097)(129.96435259,677.41672638)(130.27878901,677.86325854)
\curveto(130.59321681,678.30978798)(130.75043286,679.20284959)(130.75043764,680.54244604)
\curveto(130.75043286,681.8894766)(130.59321681,682.78439875)(130.27878901,683.22721519)
\curveto(129.96435259,683.67001818)(129.57270786,683.89142304)(129.10385365,683.89143043)
\curveto(128.64243536,683.89142304)(128.27404744,683.69606581)(127.9986888,683.30535816)
\curveto(127.65262541,682.8067253)(127.47959472,681.88575551)(127.47959623,680.54244604)
\closepath
}
}
{
\newrgbcolor{curcolor}{0 0 0}
\pscustom[linestyle=none,fillstyle=solid,fillcolor=curcolor]
{
\newpath
\moveto(208.79925171,677.05684662)
\lineto(208.79925171,685.23953179)
\lineto(211.86915406,685.23953179)
\curveto(212.49429328,685.23952361)(212.99571016,685.15672936)(213.37340621,684.99114879)
\curveto(213.75109144,684.82555235)(214.0469181,684.57065768)(214.26088708,684.22646402)
\curveto(214.47484345,683.88225602)(214.58182479,683.52224056)(214.58183141,683.14641656)
\curveto(214.58182479,682.796628)(214.486937,682.46731153)(214.29716774,682.15846616)
\curveto(214.10738581,681.84961059)(213.82086188,681.60029756)(213.43759508,681.41052632)
\curveto(213.93249458,681.26539945)(214.31297604,681.01794696)(214.57904059,680.66816811)
\curveto(214.84509191,680.31838204)(214.97812088,679.90534104)(214.9781279,679.42904389)
\curveto(214.97812088,679.04576924)(214.89718717,678.68947487)(214.73532653,678.36015971)
\curveto(214.57345234,678.03084194)(214.37344375,677.77687754)(214.13530016,677.59826576)
\curveto(213.89714423,677.4196529)(213.59852675,677.28476338)(213.23944684,677.19359681)
\curveto(212.88035637,677.10242997)(212.44033748,677.05684662)(211.91938883,677.05684662)
\closepath
\moveto(209.88208999,681.80124116)
\lineto(211.65147008,681.80124116)
\curveto(212.131487,681.80123642)(212.47568783,681.83286568)(212.6840736,681.89612905)
\curveto(212.95942954,681.97798819)(213.16688031,682.11380798)(213.30642653,682.30358882)
\curveto(213.44596206,682.49335916)(213.5157325,682.73150893)(213.51573805,683.01803882)
\curveto(213.5157325,683.28967243)(213.45061342,683.52875247)(213.32038063,683.73527964)
\curveto(213.19013712,683.94179346)(213.00408262,684.08319488)(212.76221656,684.15948433)
\curveto(212.52034091,684.23575957)(212.10543937,684.27390075)(211.5175107,684.27390796)
\lineto(209.88208999,684.27390796)
\closepath
\moveto(209.88208999,678.02247045)
\lineto(211.91938883,678.02247045)
\curveto(212.26916733,678.02246948)(212.51475928,678.0354933)(212.65616539,678.06154193)
\curveto(212.90547373,678.10619401)(213.11385477,678.18061581)(213.28130914,678.28480756)
\curveto(213.44875288,678.38899685)(213.58643321,678.54063127)(213.69435055,678.73971127)
\curveto(213.80225643,678.93878791)(213.85621224,679.16856522)(213.85621813,679.42904389)
\curveto(213.85621224,679.7341709)(213.77806934,679.99929857)(213.62178922,680.22442768)
\curveto(213.46549778,680.44955046)(213.24874429,680.60769679)(212.97152809,680.69886714)
\curveto(212.69430187,680.7900302)(212.29521496,680.83561355)(211.77426617,680.83561733)
\lineto(209.88208999,680.83561733)
\closepath
}
}
{
\newrgbcolor{curcolor}{0 0 0}
\pscustom[linestyle=none,fillstyle=solid,fillcolor=curcolor]
{
\newpath
\moveto(216.11120104,679.68579936)
\lineto(217.13264128,679.77510561)
\curveto(217.18101392,679.36578299)(217.29357689,679.02995461)(217.47033054,678.76761947)
\curveto(217.64708044,678.50528092)(217.92151083,678.29317878)(218.29362253,678.13131244)
\curveto(218.66572884,677.96944395)(219.08435147,677.88851024)(219.54949168,677.88851107)
\curveto(219.96252872,677.88851024)(220.32719554,677.94990823)(220.64349324,678.07270522)
\curveto(220.95978085,678.19550017)(221.19513979,678.36387949)(221.34957078,678.57784369)
\curveto(221.50399027,678.79180485)(221.58120289,679.02530325)(221.58120887,679.27833959)
\curveto(221.58120289,679.53509258)(221.50678109,679.75928826)(221.35794324,679.95092729)
\curveto(221.20909388,680.14256053)(220.96350194,680.30349768)(220.62116668,680.43373921)
\curveto(220.40161734,680.5193209)(219.91601509,680.65234987)(219.16435847,680.83282651)
\curveto(218.41269472,681.0132956)(217.88616048,681.18353547)(217.58475417,681.34354663)
\curveto(217.19403773,681.54820229)(216.90286244,681.80216669)(216.71122741,682.10544058)
\curveto(216.51959016,682.40870437)(216.42377209,682.74825383)(216.42377292,683.12408999)
\curveto(216.42377209,683.53712492)(216.54098643,683.92318801)(216.77541628,684.28228043)
\curveto(217.00984377,684.64135839)(217.35218406,684.91392823)(217.80243816,685.09999078)
\curveto(218.25268785,685.28603724)(218.75317446,685.37906449)(219.30389949,685.37907281)
\curveto(219.91043346,685.37906449)(220.44534015,685.28138588)(220.90862117,685.08603668)
\curveto(221.37189157,684.89067142)(221.72818594,684.60321722)(221.97750536,684.2236732)
\curveto(222.22681201,683.84411485)(222.36077125,683.41432895)(222.37938348,682.93431421)
\lineto(221.34119832,682.85617124)
\curveto(221.28537623,683.37339696)(221.09653091,683.76411141)(220.7746618,684.02831578)
\curveto(220.45278233,684.2925062)(219.97741308,684.42460489)(219.34855261,684.42461226)
\curveto(218.69363702,684.42460489)(218.21640722,684.30459974)(217.91686179,684.06459644)
\curveto(217.61731172,683.82457913)(217.46753785,683.53526437)(217.46753972,683.19665132)
\curveto(217.46753785,682.90267907)(217.57358892,682.66080822)(217.78569324,682.47103804)
\curveto(217.99407209,682.28125703)(218.53828151,682.08683008)(219.41832312,681.88775659)
\curveto(220.2983571,681.68867344)(220.90210395,681.51471248)(221.22956551,681.36587319)
\curveto(221.7058594,681.14632457)(222.05750241,680.86817309)(222.28449559,680.53141792)
\curveto(222.5114754,680.19465579)(222.62496864,679.80673216)(222.62497567,679.36764584)
\curveto(222.62496864,678.932276)(222.50031213,678.52202582)(222.25100575,678.13689408)
\curveto(222.00168606,677.75176018)(221.64353114,677.45221244)(221.17653992,677.23824994)
\curveto(220.70953754,677.02428708)(220.18393358,676.91730574)(219.59972644,676.9173056)
\curveto(218.85922552,676.91730574)(218.23873376,677.02521735)(217.73824929,677.24104076)
\curveto(217.23776054,677.4568638)(216.84518554,677.7815289)(216.56052312,678.21503705)
\curveto(216.27585876,678.64854288)(216.12608489,679.1387965)(216.11120104,679.68579936)
\closepath
}
}
{
\newrgbcolor{curcolor}{0 0 0}
\pscustom[linestyle=none,fillstyle=solid,fillcolor=curcolor]
{
\newpath
\moveto(224.11527382,677.05684662)
\lineto(224.11527382,685.23953179)
\lineto(226.93400234,685.23953179)
\curveto(227.57030503,685.23952361)(228.05590728,685.20045217)(228.39081055,685.12231734)
\curveto(228.85966273,685.01439766)(229.25967991,684.81904044)(229.59086328,684.53624507)
\curveto(230.02250337,684.17157077)(230.34530793,683.70550424)(230.55927793,683.13804409)
\curveto(230.77323328,682.57057178)(230.88021462,681.92217184)(230.88022227,681.19284233)
\curveto(230.88021462,680.57141616)(230.80765337,680.02069483)(230.66253829,679.5406767)
\curveto(230.51740834,679.0606536)(230.33135384,678.66342724)(230.10437422,678.34899643)
\curveto(229.87738086,678.03456303)(229.6289981,677.78711054)(229.3592252,677.60663822)
\curveto(229.08944004,677.42616481)(228.76384466,677.28941475)(228.38243808,677.19638763)
\curveto(228.00102121,677.10336024)(227.56286285,677.05684662)(227.06796171,677.05684662)
\closepath
\moveto(225.1981121,678.02247045)
\lineto(226.94516562,678.02247045)
\curveto(227.48471996,678.02246948)(227.90799396,678.0727042)(228.21498887,678.17317475)
\curveto(228.52197381,678.27364306)(228.76663548,678.41504448)(228.94897461,678.59737944)
\curveto(229.20572411,678.85413311)(229.4057327,679.19926421)(229.54900098,679.63277377)
\curveto(229.69225663,680.06627819)(229.76388761,680.59188215)(229.76389414,681.20958725)
\curveto(229.76388761,682.06543381)(229.62341646,682.72313647)(229.34248027,683.18269722)
\curveto(229.06153187,683.64224571)(228.72012186,683.95016591)(228.31824922,684.10645875)
\curveto(228.02799911,684.2180844)(227.56100231,684.27390075)(226.91725742,684.27390796)
\lineto(225.1981121,684.27390796)
\closepath
}
}
{
\newrgbcolor{curcolor}{0 0 0}
\pscustom[linestyle=none,fillstyle=solid,fillcolor=curcolor]
{
\newpath
\moveto(231.49420166,676.9173056)
\lineto(233.86639893,685.37907281)
\lineto(234.67015518,685.37907281)
\lineto(232.30353955,676.9173056)
\closepath
}
}
{
\newrgbcolor{curcolor}{0 0 0}
\pscustom[linestyle=none,fillstyle=solid,fillcolor=curcolor]
{
\newpath
\moveto(235.14459476,679.21694155)
\lineto(236.14929008,679.35090092)
\curveto(236.26464239,678.78157185)(236.46092989,678.37132167)(236.73815317,678.12014916)
\curveto(237.0153723,677.86897452)(237.35306122,677.74338773)(237.75122094,677.74338842)
\curveto(238.22379629,677.74338773)(238.6228832,677.90711569)(238.94848286,678.23457279)
\curveto(239.27407395,678.56202754)(239.43687164,678.96762635)(239.43687642,679.45137045)
\curveto(239.43687164,679.91278322)(239.2861675,680.29326468)(238.98476352,680.59281596)
\curveto(238.68335091,680.89236018)(238.30007864,681.04213405)(237.83494555,681.04213803)
\curveto(237.64516679,681.04213405)(237.40887757,681.00492315)(237.12607719,680.93050522)
\lineto(237.23771,681.81240444)
\curveto(237.30468705,681.80495751)(237.35864286,681.80123642)(237.39957758,681.80124116)
\curveto(237.8275002,681.80123642)(238.21263302,681.91286912)(238.55497719,682.1361396)
\curveto(238.89731359,682.35939992)(239.06848373,682.70360075)(239.06848813,683.16874312)
\curveto(239.06848373,683.53712492)(238.94382721,683.8422543)(238.69451821,684.08413218)
\curveto(238.44520115,684.32599601)(238.12332686,684.44693143)(237.72889438,684.44693882)
\curveto(237.33817686,684.44693143)(237.01258148,684.32413546)(236.75210727,684.07855054)
\curveto(236.49162888,683.83295158)(236.32417983,683.46456366)(236.24975961,682.9733857)
\lineto(235.2450643,683.1519982)
\curveto(235.36785969,683.8255094)(235.64694144,684.34739228)(236.08231039,684.7176484)
\curveto(236.51767651,685.08788919)(237.05909511,685.27301342)(237.70656782,685.27302164)
\curveto(238.15309558,685.27301342)(238.56427603,685.17719535)(238.9401104,684.98556715)
\curveto(239.31593622,684.79392308)(239.60339042,684.5325165)(239.80247388,684.20134664)
\curveto(240.00154706,683.87016248)(240.10108621,683.51851947)(240.10109165,683.14641656)
\curveto(240.10108621,682.79290691)(240.00619842,682.47103262)(239.81642798,682.18079272)
\curveto(239.62664723,681.89054258)(239.34570494,681.659835)(238.97360024,681.48866929)
\curveto(239.45733764,681.37703215)(239.83316773,681.1453943)(240.10109165,680.79375503)
\curveto(240.3690047,680.44210828)(240.50296394,680.00208938)(240.50296978,679.47369702)
\curveto(240.50296394,678.75924531)(240.24248764,678.15363791)(239.72154009,677.65687299)
\curveto(239.20058243,677.16010687)(238.54194949,676.91172411)(237.7456393,676.91172396)
\curveto(237.02746584,676.91172411)(236.43116116,677.12568678)(235.95672348,677.55361264)
\curveto(235.48228321,677.98153749)(235.2115739,678.53597991)(235.14459476,679.21694155)
\closepath
}
}
{
\newrgbcolor{curcolor}{0 0 0}
\pscustom[linestyle=none,fillstyle=solid,fillcolor=curcolor]
{
\newpath
\moveto(243.04819902,681.49425093)
\curveto(242.63143492,681.64681118)(242.32258444,681.86449495)(242.12164668,682.14730288)
\curveto(241.92070672,682.43010063)(241.82023729,682.76871983)(241.82023808,683.16316148)
\curveto(241.82023729,683.75852978)(242.03419997,684.25901639)(242.46212676,684.66462281)
\curveto(242.89005068,685.07021402)(243.45937745,685.27301342)(244.17010879,685.27302164)
\curveto(244.88455494,685.27301342)(245.45946335,685.06556265)(245.89483575,684.65066871)
\curveto(246.33019842,684.23575957)(246.54788218,683.7306216)(246.5478877,683.13525327)
\curveto(246.54788218,682.75569601)(246.44834302,682.42544927)(246.24926993,682.14451206)
\curveto(246.05018639,681.86356468)(245.74784782,681.64681118)(245.34225332,681.49425093)
\curveto(245.84459617,681.33051853)(246.22693817,681.06632113)(246.48928048,680.70165796)
\curveto(246.75161186,680.33698749)(246.88278029,679.90161995)(246.88278614,679.39555405)
\curveto(246.88278029,678.69598678)(246.6353278,678.10805455)(246.14042794,677.6317556)
\curveto(245.64551785,677.1554555)(244.99432709,676.91730574)(244.18685371,676.9173056)
\curveto(243.37937402,676.91730574)(242.72818326,677.15638578)(242.23327949,677.63454643)
\curveto(241.73837331,678.11270592)(241.49092082,678.7090106)(241.49092128,679.42346225)
\curveto(241.49092082,679.95557576)(241.62581033,680.40117629)(241.89559023,680.76026518)
\curveto(242.16536839,681.11934667)(242.54957094,681.36400834)(243.04819902,681.49425093)
\closepath
\moveto(242.84725996,683.19665132)
\curveto(242.84725814,682.80965182)(242.97191466,682.49335916)(243.22122988,682.24777241)
\curveto(243.47054072,682.00217528)(243.79427555,681.87937931)(244.19243535,681.87938413)
\curveto(244.57942555,681.87937931)(244.89664848,682.00124501)(245.14410508,682.24498159)
\curveto(245.39155345,682.4887078)(245.5152797,682.78732528)(245.51528418,683.14083491)
\curveto(245.5152797,683.50921674)(245.38783236,683.81899749)(245.1329418,684.07017808)
\curveto(244.87804303,684.32134465)(244.5608201,684.44693143)(244.18127207,684.44693882)
\curveto(243.79799664,684.44693143)(243.47984345,684.32413546)(243.22681152,684.07855054)
\curveto(242.9737752,683.83295158)(242.84725814,683.53898546)(242.84725996,683.19665132)
\closepath
\moveto(242.5235248,679.41788061)
\curveto(242.52352331,679.13135432)(242.5914332,678.85413311)(242.72725469,678.58621615)
\curveto(242.86307277,678.31829614)(243.06494191,678.11084537)(243.33286269,677.96386322)
\curveto(243.60077887,677.81687926)(243.88916335,677.74338773)(244.19801699,677.74338842)
\curveto(244.67803444,677.74338773)(245.07433053,677.89781297)(245.38690645,678.20666459)
\curveto(245.69947365,678.51551391)(245.85575944,678.90808891)(245.85576426,679.38439077)
\curveto(245.85575944,679.86813014)(245.69482229,680.26814732)(245.37295235,680.5844435)
\curveto(245.05107371,680.90073263)(244.64826572,681.05887895)(244.16452715,681.05888296)
\curveto(243.69194558,681.05887895)(243.30030085,680.90259317)(242.9895918,680.59002514)
\curveto(242.67887882,680.27745005)(242.52352331,679.88673559)(242.5235248,679.41788061)
\closepath
}
}
{
\newrgbcolor{curcolor}{0 0 0}
\pscustom[linestyle=none,fillstyle=solid,fillcolor=curcolor]
{
\newpath
\moveto(253.07840836,683.23572281)
\lineto(252.07929469,683.15757984)
\curveto(251.98998384,683.55200928)(251.86346678,683.83853321)(251.69974313,684.0171525)
\curveto(251.42809924,684.30366947)(251.09320114,684.44693143)(250.69504781,684.44693882)
\curveto(250.37503076,684.44693143)(250.09408847,684.35762527)(249.85222007,684.17902007)
\curveto(249.53592496,683.94830537)(249.28661193,683.61154672)(249.10428023,683.16874312)
\curveto(248.9219451,682.72592729)(248.82705731,682.09520253)(248.81961656,681.27656694)
\curveto(249.06148598,681.64495064)(249.35731264,681.91845075)(249.70709742,682.09706812)
\curveto(250.05687756,682.2756754)(250.42340493,682.36498156)(250.80668062,682.36498687)
\curveto(251.47647341,682.36498156)(252.04673046,682.11845934)(252.51745348,681.62541948)
\curveto(252.98816624,681.13237048)(253.22352519,680.49513381)(253.22353102,679.71370756)
\curveto(253.22352519,679.20019448)(253.11282276,678.72296468)(252.8914234,678.28201674)
\curveto(252.67001304,677.84106634)(252.36581393,677.50337742)(251.97882516,677.26894896)
\curveto(251.59182721,677.03452008)(251.15273858,676.91730574)(250.66155797,676.9173056)
\curveto(249.82430944,676.91730574)(249.14148942,677.22522594)(248.61309585,677.84106713)
\curveto(248.08469984,678.45690675)(247.82050245,679.47183405)(247.82050288,680.8858521)
\curveto(247.82050245,682.46731153)(248.11260802,683.61712836)(248.69682046,684.33530601)
\curveto(249.20660849,684.96044186)(249.8931496,685.27301342)(250.75644586,685.27302164)
\curveto(251.40019107,685.27301342)(251.92765558,685.09254056)(252.33884098,684.7316025)
\curveto(252.75001648,684.37064909)(252.99653869,683.87202302)(253.07840836,683.23572281)
\closepath
\moveto(248.97590249,679.70812592)
\curveto(248.97590091,679.3620619)(249.04939244,679.03088488)(249.1963773,678.71459389)
\curveto(249.34335855,678.39829958)(249.54894877,678.157359)(249.81314859,677.99177143)
\curveto(250.07734356,677.82618198)(250.35456477,677.74338773)(250.64481304,677.74338842)
\curveto(251.06901406,677.74338773)(251.43368088,677.91455787)(251.73881461,678.25689936)
\curveto(252.04393965,678.59923844)(252.19650434,679.06437469)(252.19650914,679.65230952)
\curveto(252.19650434,680.21791261)(252.04580019,680.66351314)(251.74439625,680.98911245)
\curveto(251.4429836,681.31470389)(251.06343242,681.47750158)(250.60574156,681.477506)
\curveto(250.15176536,681.47750158)(249.76663254,681.31470389)(249.45034195,680.98911245)
\curveto(249.13404724,680.66351314)(248.97590091,680.23651806)(248.97590249,679.70812592)
\closepath
}
}
{
\newrgbcolor{curcolor}{0 0 0}
\pscustom[linestyle=none,fillstyle=solid,fillcolor=curcolor]
{
\newpath
\moveto(257.39859564,681.0923728)
\curveto(257.39859517,682.05985217)(257.49813433,682.83849027)(257.69721342,683.42828941)
\curveto(257.89629096,684.01807581)(258.19211762,684.47297906)(258.58469428,684.79300054)
\curveto(258.97726762,685.11300655)(259.47124232,685.27301342)(260.06661987,685.27302164)
\curveto(260.50570535,685.27301342)(260.89083817,685.18463753)(261.22201948,685.00789371)
\curveto(261.5531922,684.83113398)(261.82669231,684.57623931)(262.04252065,684.24320894)
\curveto(262.25833876,683.9101642)(262.42764835,683.50456538)(262.55044995,683.02641128)
\curveto(262.6732403,682.54824524)(262.73463828,681.90356639)(262.73464409,681.0923728)
\curveto(262.73463828,680.13232754)(262.6360294,679.35741053)(262.43881714,678.76761947)
\curveto(262.24159385,678.17782499)(261.94669747,677.72199146)(261.5541271,677.40011752)
\curveto(261.16154747,677.07824289)(260.66571222,676.91730574)(260.06661987,676.9173056)
\curveto(259.27774564,676.91730574)(258.65818415,677.20010859)(258.20793354,677.76571498)
\curveto(257.6683742,678.44667375)(257.39859517,679.55555858)(257.39859564,681.0923728)
\closepath
\moveto(258.43119916,681.0923728)
\curveto(258.43119765,679.74905526)(258.58841371,678.85506338)(258.9028478,678.41039447)
\curveto(259.21727792,677.96572286)(259.60520156,677.74338773)(260.06661987,677.74338842)
\curveto(260.52803189,677.74338773)(260.91595553,677.96665313)(261.23039194,678.41318529)
\curveto(261.54481974,678.85971474)(261.7020358,679.75277635)(261.70204058,681.0923728)
\curveto(261.7020358,682.43940336)(261.54481974,683.33432551)(261.23039194,683.77714195)
\curveto(260.91595553,684.21994494)(260.5243108,684.4413498)(260.05545659,684.44135718)
\curveto(259.59403829,684.4413498)(259.22565038,684.24599257)(258.95029174,683.85528492)
\curveto(258.60422834,683.35665205)(258.43119765,682.43568227)(258.43119916,681.0923728)
\closepath
}
}
{
\newrgbcolor{curcolor}{0 0 0}
\pscustom[linestyle=none,fillstyle=solid,fillcolor=curcolor]
{
\newpath
\moveto(264.32541279,677.05684662)
\lineto(264.32541279,678.20108295)
\lineto(265.46964912,678.20108295)
\lineto(265.46964912,677.05684662)
\closepath
}
}
{
\newrgbcolor{curcolor}{0 0 0}
\pscustom[linestyle=none,fillstyle=solid,fillcolor=curcolor]
{
\newpath
\moveto(266.93762073,679.21694155)
\lineto(267.94231605,679.35090092)
\curveto(268.05766836,678.78157185)(268.25395586,678.37132167)(268.53117914,678.12014916)
\curveto(268.80839827,677.86897452)(269.14608719,677.74338773)(269.54424691,677.74338842)
\curveto(270.01682226,677.74338773)(270.41590917,677.90711569)(270.74150883,678.23457279)
\curveto(271.06709992,678.56202754)(271.22989761,678.96762635)(271.22990239,679.45137045)
\curveto(271.22989761,679.91278322)(271.07919347,680.29326468)(270.77778949,680.59281596)
\curveto(270.47637688,680.89236018)(270.09310461,681.04213405)(269.62797152,681.04213803)
\curveto(269.43819276,681.04213405)(269.20190354,681.00492315)(268.91910316,680.93050522)
\lineto(269.03073597,681.81240444)
\curveto(269.09771302,681.80495751)(269.15166883,681.80123642)(269.19260355,681.80124116)
\curveto(269.62052617,681.80123642)(270.00565899,681.91286912)(270.34800316,682.1361396)
\curveto(270.69033956,682.35939992)(270.8615097,682.70360075)(270.8615141,683.16874312)
\curveto(270.8615097,683.53712492)(270.73685318,683.8422543)(270.48754418,684.08413218)
\curveto(270.23822712,684.32599601)(269.91635283,684.44693143)(269.52192035,684.44693882)
\curveto(269.13120283,684.44693143)(268.80560745,684.32413546)(268.54513324,684.07855054)
\curveto(268.28465485,683.83295158)(268.1172058,683.46456366)(268.04278558,682.9733857)
\lineto(267.03809027,683.1519982)
\curveto(267.16088566,683.8255094)(267.43996741,684.34739228)(267.87533636,684.7176484)
\curveto(268.31070248,685.08788919)(268.85212108,685.27301342)(269.49959379,685.27302164)
\curveto(269.94612155,685.27301342)(270.357302,685.17719535)(270.73313637,684.98556715)
\curveto(271.10896219,684.79392308)(271.39641639,684.5325165)(271.59549985,684.20134664)
\curveto(271.79457303,683.87016248)(271.89411219,683.51851947)(271.89411762,683.14641656)
\curveto(271.89411219,682.79290691)(271.79922439,682.47103262)(271.60945395,682.18079272)
\curveto(271.41967321,681.89054258)(271.13873091,681.659835)(270.76662621,681.48866929)
\curveto(271.25036361,681.37703215)(271.6261937,681.1453943)(271.89411762,680.79375503)
\curveto(272.16203067,680.44210828)(272.29598991,680.00208938)(272.29599575,679.47369702)
\curveto(272.29598991,678.75924531)(272.03551361,678.15363791)(271.51456606,677.65687299)
\curveto(270.9936084,677.16010687)(270.33497546,676.91172411)(269.53866527,676.91172396)
\curveto(268.82049181,676.91172411)(268.22418714,677.12568678)(267.74974945,677.55361264)
\curveto(267.27530918,677.98153749)(267.00459988,678.53597991)(266.93762073,679.21694155)
\closepath
}
}
{
\newrgbcolor{curcolor}{0 0 0}
\pscustom[linestyle=none,fillstyle=solid,fillcolor=curcolor]
{
\newpath
\moveto(273.85885624,677.05684662)
\lineto(273.85885624,678.20108295)
\lineto(275.00309257,678.20108295)
\lineto(275.00309257,677.05684662)
\closepath
}
}
{
\newrgbcolor{curcolor}{0 0 0}
\pscustom[linestyle=none,fillstyle=solid,fillcolor=curcolor]
{
\newpath
\moveto(280.2498349,677.05684662)
\lineto(279.24513958,677.05684662)
\lineto(279.24513958,683.45898843)
\curveto(279.00326548,683.22827445)(278.68604255,682.99756686)(278.29346985,682.76686499)
\curveto(277.90089255,682.5361517)(277.54831927,682.36312101)(277.23574895,682.24777241)
\lineto(277.23574895,683.21897788)
\curveto(277.7976323,683.48316911)(278.28881619,683.80318286)(278.70930208,684.17902007)
\curveto(279.12978254,684.55484305)(279.42746974,684.91950987)(279.60236458,685.27302164)
\lineto(280.2498349,685.27302164)
\closepath
}
}
{
\newrgbcolor{curcolor}{0 0 0}
\pscustom[linestyle=none,fillstyle=solid,fillcolor=curcolor]
{
\newpath
\moveto(307.29700718,678.76699037)
\lineto(308.30170249,678.90094975)
\curveto(308.4170548,678.33162068)(308.6133423,677.9213705)(308.89056558,677.67019799)
\curveto(309.16778472,677.41902335)(309.50547364,677.29343656)(309.90363336,677.29343725)
\curveto(310.37620871,677.29343656)(310.77529561,677.45716452)(311.10089527,677.78462162)
\curveto(311.42648637,678.11207637)(311.58928406,678.51767518)(311.58928883,679.00141928)
\curveto(311.58928406,679.46283205)(311.43857991,679.84331351)(311.13717594,680.14286479)
\curveto(310.83576332,680.442409)(310.45249105,680.59218288)(309.98735797,680.59218686)
\curveto(309.7975792,680.59218288)(309.56128999,680.55497198)(309.27848961,680.48055405)
\lineto(309.39012242,681.36245327)
\curveto(309.45709947,681.35500634)(309.51105527,681.35128525)(309.55199,681.35128999)
\curveto(309.97991262,681.35128525)(310.36504544,681.46291795)(310.70738961,681.68618843)
\curveto(311.049726,681.90944875)(311.22089614,682.25364958)(311.22090055,682.71879195)
\curveto(311.22089614,683.08717375)(311.09623963,683.39230313)(310.84693062,683.63418101)
\curveto(310.59761356,683.87604484)(310.27573927,683.99698026)(309.88130679,683.99698765)
\curveto(309.49058928,683.99698026)(309.1649939,683.87418429)(308.90451968,683.62859937)
\curveto(308.64404129,683.38300041)(308.47659224,683.01461249)(308.40217203,682.52343452)
\lineto(307.39747671,682.70204702)
\curveto(307.5202721,683.37555823)(307.79935385,683.8974411)(308.23472281,684.26769722)
\curveto(308.67008892,684.63793802)(309.21150752,684.82306225)(309.85898023,684.82307047)
\curveto(310.30550799,684.82306225)(310.71668844,684.72724418)(311.09252281,684.53561597)
\curveto(311.46834863,684.34397191)(311.75580284,684.08256533)(311.95488629,683.75139546)
\curveto(312.15395947,683.42021131)(312.25349863,683.0685683)(312.25350407,682.69646538)
\curveto(312.25349863,682.34295574)(312.15861083,682.02108145)(311.96884039,681.73084155)
\curveto(311.77905965,681.44059141)(311.49811735,681.20988382)(311.12601266,681.03871811)
\curveto(311.60975005,680.92708098)(311.98558015,680.69544313)(312.25350407,680.34380385)
\curveto(312.52141711,679.99215711)(312.65537635,679.55213821)(312.65538219,679.02374584)
\curveto(312.65537635,678.30929414)(312.39490005,677.70368674)(311.8739525,677.20692182)
\curveto(311.35299484,676.7101557)(310.6943619,676.46177294)(309.89805172,676.46177279)
\curveto(309.17987826,676.46177294)(308.58357358,676.67573561)(308.10913589,677.10366146)
\curveto(307.63469562,677.53158632)(307.36398632,678.08602874)(307.29700718,678.76699037)
\closepath
}
}
{
\newrgbcolor{curcolor}{0 0 0}
\pscustom[linestyle=none,fillstyle=solid,fillcolor=curcolor]
{
\newpath
\moveto(315.20061048,681.04429975)
\curveto(314.78384638,681.19686001)(314.47499591,681.41454378)(314.27405814,681.69735171)
\curveto(314.07311818,681.98014946)(313.97264875,682.31876866)(313.97264954,682.71321031)
\curveto(313.97264875,683.30857861)(314.18661143,683.80906522)(314.61453822,684.21467164)
\curveto(315.04246214,684.62026284)(315.61178891,684.82306225)(316.32252025,684.82307047)
\curveto(317.0369664,684.82306225)(317.61187481,684.61561148)(318.04724721,684.20071754)
\curveto(318.48260988,683.7858084)(318.70029364,683.28067043)(318.70029916,682.6853021)
\curveto(318.70029364,682.30574484)(318.60075448,681.9754981)(318.40168139,681.69456089)
\curveto(318.20259785,681.4136135)(317.90025928,681.19686001)(317.49466479,681.04429975)
\curveto(317.99700763,680.88056736)(318.37934963,680.61636996)(318.64169194,680.25170678)
\curveto(318.90402332,679.88703632)(319.03519175,679.45166878)(319.0351976,678.94560287)
\curveto(319.03519175,678.24603561)(318.78773926,677.65810338)(318.2928394,677.18180443)
\curveto(317.79792931,676.70550433)(317.14673855,676.46735457)(316.33926517,676.46735443)
\curveto(315.53178548,676.46735457)(314.88059472,676.70643461)(314.38569095,677.18459525)
\curveto(313.89078477,677.66275475)(313.64333228,678.25905942)(313.64333274,678.97351108)
\curveto(313.64333228,679.50562459)(313.7782218,679.95122512)(314.04800169,680.31031401)
\curveto(314.31777985,680.6693955)(314.7019824,680.91405717)(315.20061048,681.04429975)
\closepath
\moveto(314.99967142,682.74670015)
\curveto(314.9996696,682.35970065)(315.12432612,682.04340799)(315.37364134,681.79782124)
\curveto(315.62295218,681.55222411)(315.94668702,681.42942814)(316.34484681,681.42943296)
\curveto(316.73183701,681.42942814)(317.04905994,681.55129383)(317.29651654,681.79503042)
\curveto(317.54396491,682.03875663)(317.66769116,682.33737411)(317.66769565,682.69088374)
\curveto(317.66769116,683.05926557)(317.54024382,683.36904632)(317.28535326,683.62022691)
\curveto(317.03045449,683.87139347)(316.71323156,683.99698026)(316.33368353,683.99698765)
\curveto(315.95040811,683.99698026)(315.63225491,683.87418429)(315.37922298,683.62859937)
\curveto(315.12618666,683.38300041)(314.9996696,683.08903429)(314.99967142,682.74670015)
\closepath
\moveto(314.67593626,678.96792944)
\curveto(314.67593477,678.68140314)(314.74384466,678.40418194)(314.87966615,678.13626498)
\curveto(315.01548423,677.86834497)(315.21735337,677.6608942)(315.48527416,677.51391205)
\curveto(315.75319033,677.36692809)(316.04157481,677.29343656)(316.35042845,677.29343725)
\curveto(316.8304459,677.29343656)(317.22674199,677.4478618)(317.53931791,677.75671342)
\curveto(317.85188511,678.06556274)(318.0081709,678.45813774)(318.00817572,678.93443959)
\curveto(318.0081709,679.41817897)(317.84723375,679.81819615)(317.52536381,680.13449233)
\curveto(317.20348518,680.45078146)(316.80067718,680.60892778)(316.31693861,680.60893178)
\curveto(315.84435704,680.60892778)(315.45271231,680.452642)(315.14200326,680.14007397)
\curveto(314.83129028,679.82749887)(314.67593477,679.43678442)(314.67593626,678.96792944)
\closepath
}
}
{
\newrgbcolor{curcolor}{0 0 0}
\pscustom[linestyle=none,fillstyle=solid,fillcolor=curcolor]
{
\newpath
\moveto(325.23081887,682.78577163)
\lineto(324.2317052,682.70762866)
\curveto(324.14239435,683.10205811)(324.01587729,683.38858204)(323.85215363,683.56720132)
\curveto(323.58050975,683.8537183)(323.24561165,683.99698026)(322.84745832,683.99698765)
\curveto(322.52744127,683.99698026)(322.24649897,683.9076741)(322.00463058,683.7290689)
\curveto(321.68833547,683.4983542)(321.43902243,683.16159555)(321.25669074,682.71879195)
\curveto(321.07435561,682.27597612)(320.97946781,681.64525136)(320.97202706,680.82661577)
\curveto(321.21389649,681.19499946)(321.50972314,681.46849958)(321.85950792,681.64711694)
\curveto(322.20928807,681.82572423)(322.57581544,681.91503039)(322.95909113,681.91503569)
\curveto(323.62888392,681.91503039)(324.19914097,681.66850817)(324.66986399,681.17546831)
\curveto(325.14057675,680.68241931)(325.37593569,680.04518264)(325.37594153,679.26375639)
\curveto(325.37593569,678.75024331)(325.26523327,678.27301351)(325.04383391,677.83206557)
\curveto(324.82242355,677.39111517)(324.51822444,677.05342625)(324.13123566,676.81899779)
\curveto(323.74423771,676.58456891)(323.30514909,676.46735457)(322.81396847,676.46735443)
\curveto(321.97671994,676.46735457)(321.29389992,676.77527477)(320.76550636,677.39111596)
\curveto(320.23711035,678.00695557)(319.97291296,679.02188288)(319.97291339,680.43590093)
\curveto(319.97291296,682.01736036)(320.26501853,683.16717718)(320.84923097,683.88535484)
\curveto(321.359019,684.51049069)(322.04556011,684.82306225)(322.90885636,684.82307047)
\curveto(323.55260158,684.82306225)(324.08006609,684.64258938)(324.49125149,684.28165132)
\curveto(324.90242699,683.92069792)(325.1489492,683.42207185)(325.23081887,682.78577163)
\closepath
\moveto(321.128313,679.25817475)
\curveto(321.12831142,678.91211073)(321.20180294,678.58093371)(321.34878781,678.26464272)
\curveto(321.49576906,677.94834841)(321.70135928,677.70740783)(321.9655591,677.54182025)
\curveto(322.22975407,677.37623081)(322.50697528,677.29343656)(322.79722355,677.29343725)
\curveto(323.22142456,677.29343656)(323.58609139,677.4646067)(323.89122512,677.80694818)
\curveto(324.19635015,678.14928727)(324.34891484,678.61442352)(324.34891965,679.20235834)
\curveto(324.34891484,679.76796143)(324.1982107,680.21356197)(323.89680676,680.53916128)
\curveto(323.59539411,680.86475272)(323.21584293,681.02755041)(322.75815207,681.02755483)
\curveto(322.30417587,681.02755041)(321.91904305,680.86475272)(321.60275246,680.53916128)
\curveto(321.28645774,680.21356197)(321.12831142,679.78656688)(321.128313,679.25817475)
\closepath
}
}
{
\newrgbcolor{curcolor}{0 0 0}
\pscustom[linestyle=none,fillstyle=solid,fillcolor=curcolor]
{
\newpath
\moveto(326.74344269,676.60689545)
\lineto(326.74344269,684.78958062)
\lineto(329.81334504,684.78958062)
\curveto(330.43848426,684.78957244)(330.93990114,684.70677819)(331.31759719,684.54119761)
\curveto(331.69528242,684.37560117)(331.99110908,684.12070651)(332.20507805,683.77651285)
\curveto(332.41903443,683.43230485)(332.52601577,683.07228939)(332.52602239,682.69646538)
\curveto(332.52601577,682.34667683)(332.43112798,682.01736036)(332.24135872,681.70851499)
\curveto(332.05157679,681.39965942)(331.76505286,681.15034638)(331.38178606,680.96057515)
\curveto(331.87668556,680.81544828)(332.25716702,680.56799579)(332.52323157,680.21821694)
\curveto(332.78928289,679.86843087)(332.92231186,679.45538987)(332.92231888,678.97909272)
\curveto(332.92231186,678.59581807)(332.84137815,678.2395237)(332.67951751,677.91020854)
\curveto(332.51764332,677.58089076)(332.31763473,677.32692637)(332.07949114,677.14831459)
\curveto(331.8413352,676.96970173)(331.54271773,676.83481221)(331.18363782,676.74364564)
\curveto(330.82454735,676.6524788)(330.38452845,676.60689545)(329.86357981,676.60689545)
\closepath
\moveto(327.82628097,681.35128999)
\lineto(329.59566106,681.35128999)
\curveto(330.07567798,681.35128525)(330.41987881,681.38291451)(330.62826457,681.44617788)
\curveto(330.90362052,681.52803702)(331.11107128,681.66385681)(331.25061751,681.85363765)
\curveto(331.39015304,682.04340799)(331.45992348,682.28155775)(331.45992903,682.56808765)
\curveto(331.45992348,682.83972126)(331.3948044,683.0788013)(331.26457161,683.28532847)
\curveto(331.1343281,683.49184229)(330.9482736,683.63324371)(330.70640754,683.70953316)
\curveto(330.46453189,683.7858084)(330.04963035,683.82394958)(329.46170168,683.82395679)
\lineto(327.82628097,683.82395679)
\closepath
\moveto(327.82628097,677.57251928)
\lineto(329.86357981,677.57251928)
\curveto(330.21335831,677.57251831)(330.45895026,677.58554213)(330.60035637,677.61159076)
\curveto(330.84966471,677.65624284)(331.05804575,677.73066464)(331.22550012,677.83485639)
\curveto(331.39294386,677.93904568)(331.53062419,678.0906801)(331.63854153,678.2897601)
\curveto(331.74644741,678.48883673)(331.80040321,678.71861404)(331.80040911,678.97909272)
\curveto(331.80040321,679.28421973)(331.72226032,679.54934739)(331.5659802,679.77447651)
\curveto(331.40968876,679.99959929)(331.19293527,680.15774562)(330.91571907,680.24891596)
\curveto(330.63849285,680.34007903)(330.23940594,680.38566238)(329.71845715,680.38566616)
\lineto(327.82628097,680.38566616)
\closepath
}
}
{
\newrgbcolor{curcolor}{0 0 0}
\pscustom[linestyle=none,fillstyle=solid,fillcolor=curcolor]
{
\newpath
\moveto(334.0553925,679.23584819)
\lineto(335.07683274,679.32515444)
\curveto(335.12520537,678.91583182)(335.23776835,678.58000344)(335.414522,678.3176683)
\curveto(335.5912719,678.05532974)(335.86570229,677.84322761)(336.23781399,677.68136127)
\curveto(336.6099203,677.51949278)(337.02854293,677.43855907)(337.49368313,677.4385599)
\curveto(337.90672018,677.43855907)(338.271387,677.49995706)(338.5876847,677.62275404)
\curveto(338.90397231,677.745549)(339.13933125,677.91392832)(339.29376224,678.12789252)
\curveto(339.44818172,678.34185368)(339.52539434,678.57535208)(339.52540033,678.82838842)
\curveto(339.52539434,679.08514141)(339.45097254,679.30933709)(339.3021347,679.50097612)
\curveto(339.15328534,679.69260936)(338.9076934,679.85354651)(338.56535814,679.98378803)
\curveto(338.3458088,680.06936973)(337.86020655,680.2023987)(337.10854993,680.38287534)
\curveto(336.35688617,680.56334443)(335.83035193,680.7335843)(335.52894563,680.89359546)
\curveto(335.13822919,681.09825112)(334.84705389,681.35221552)(334.65541887,681.6554894)
\curveto(334.46378162,681.95875319)(334.36796355,682.29830266)(334.36796438,682.67413882)
\curveto(334.36796355,683.08717375)(334.48517789,683.47323684)(334.71960774,683.83232925)
\curveto(334.95403523,684.19140722)(335.29637551,684.46397706)(335.74662961,684.65003961)
\curveto(336.1968793,684.83608607)(336.69736591,684.92911332)(337.24809095,684.92912164)
\curveto(337.85462492,684.92911332)(338.38953161,684.8314347)(338.85281263,684.63608551)
\curveto(339.31608303,684.44072025)(339.6723774,684.15326604)(339.92169681,683.77372203)
\curveto(340.17100346,683.39416368)(340.30496271,682.96437778)(340.32357494,682.48436304)
\lineto(339.28538978,682.40622007)
\curveto(339.22956768,682.92344579)(339.04072236,683.31416024)(338.71885325,683.5783646)
\curveto(338.39697379,683.84255503)(337.92160454,683.97465372)(337.29274407,683.97466109)
\curveto(336.63782847,683.97465372)(336.16059868,683.85464857)(335.86105325,683.61464527)
\curveto(335.56150318,683.37462795)(335.41172931,683.0853132)(335.41173118,682.74670015)
\curveto(335.41172931,682.4527279)(335.51778037,682.21085704)(335.72988469,682.02108687)
\curveto(335.93826355,681.83130586)(336.48247296,681.63687891)(337.36251458,681.43780542)
\curveto(338.24254855,681.23872227)(338.84629541,681.06476131)(339.17375697,680.91592202)
\curveto(339.65005086,680.6963734)(340.00169387,680.41822192)(340.22868705,680.08146674)
\curveto(340.45566685,679.74470462)(340.5691601,679.35678099)(340.56916713,678.91769467)
\curveto(340.5691601,678.48232483)(340.44450358,678.07207465)(340.1951972,677.68694291)
\curveto(339.94587752,677.30180901)(339.5877226,677.00226126)(339.12073138,676.78829877)
\curveto(338.653729,676.57433591)(338.12812503,676.46735457)(337.5439179,676.46735443)
\curveto(336.80341698,676.46735457)(336.18292522,676.57526618)(335.68244075,676.79108959)
\curveto(335.181952,677.00691263)(334.789377,677.33157773)(334.50471457,677.76508588)
\curveto(334.22005022,678.19859171)(334.07027635,678.68884532)(334.0553925,679.23584819)
\closepath
}
}
{
\newrgbcolor{curcolor}{0 0 0}
\pscustom[linestyle=none,fillstyle=solid,fillcolor=curcolor]
{
\newpath
\moveto(342.05946575,676.60689545)
\lineto(342.05946575,684.78958062)
\lineto(344.87819427,684.78958062)
\curveto(345.51449697,684.78957244)(346.00009922,684.75050099)(346.33500248,684.67236617)
\curveto(346.80385467,684.56444649)(347.20387184,684.36908927)(347.53505522,684.0862939)
\curveto(347.9666953,683.7216196)(348.28949986,683.25555307)(348.50346987,682.68809292)
\curveto(348.71742522,682.12062061)(348.82440656,681.47222067)(348.8244142,680.74289116)
\curveto(348.82440656,680.12146499)(348.7518453,679.57074366)(348.60673022,679.09072553)
\curveto(348.46160028,678.61070243)(348.27554578,678.21347607)(348.04856615,677.89904525)
\curveto(347.82157279,677.58461185)(347.57319003,677.33715937)(347.30341713,677.15668705)
\curveto(347.03363198,676.97621363)(346.7080366,676.83946357)(346.32663002,676.74643646)
\curveto(345.94521314,676.65340907)(345.50705479,676.60689545)(345.01215365,676.60689545)
\closepath
\moveto(343.14230403,677.57251928)
\lineto(344.88935755,677.57251928)
\curveto(345.4289119,677.57251831)(345.85218589,677.62275303)(346.1591808,677.72322357)
\curveto(346.46616574,677.82369189)(346.71082741,677.96509331)(346.89316654,678.14742826)
\curveto(347.14991604,678.40418194)(347.34992463,678.74931304)(347.49319291,679.1828226)
\curveto(347.63644856,679.61632702)(347.70807954,680.14193098)(347.70808608,680.75963608)
\curveto(347.70807954,681.61548264)(347.5676084,682.2731853)(347.28667221,682.73274605)
\curveto(347.0057238,683.19229454)(346.66431379,683.50021474)(346.26244115,683.65650757)
\curveto(345.97219104,683.76813322)(345.50519424,683.82394958)(344.86144935,683.82395679)
\lineto(343.14230403,683.82395679)
\closepath
}
}
{
\newrgbcolor{curcolor}{0 0 0}
\pscustom[linestyle=none,fillstyle=solid,fillcolor=curcolor]
{
\newpath
\moveto(353.0832041,680.64242163)
\curveto(353.08320363,681.609901)(353.18274279,682.38853909)(353.38182188,682.97833824)
\curveto(353.58089942,683.56812464)(353.87672608,684.02302789)(354.26930274,684.34304937)
\curveto(354.66187608,684.66305538)(355.15585078,684.82306225)(355.75122833,684.82307047)
\curveto(356.19031381,684.82306225)(356.57544663,684.73468636)(356.90662794,684.55794254)
\curveto(357.23780066,684.38118281)(357.51130077,684.12628814)(357.72712911,683.79325777)
\curveto(357.94294722,683.46021302)(358.11225681,683.05461421)(358.23505841,682.57646011)
\curveto(358.35784876,682.09829407)(358.41924674,681.45361522)(358.41925255,680.64242163)
\curveto(358.41924674,679.68237636)(358.32063786,678.90745936)(358.1234256,678.3176683)
\curveto(357.92620231,677.72787382)(357.63130593,677.27204029)(357.23873556,676.95016635)
\curveto(356.84615593,676.62829171)(356.35032068,676.46735457)(355.75122833,676.46735443)
\curveto(354.9623541,676.46735457)(354.34279261,676.75015741)(353.892542,677.31576381)
\curveto(353.35298266,677.99672258)(353.08320363,679.10560741)(353.0832041,680.64242163)
\closepath
\moveto(354.11580762,680.64242163)
\curveto(354.11580611,679.29910409)(354.27302217,678.40511221)(354.58745625,677.9604433)
\curveto(354.90188638,677.51577169)(355.28981002,677.29343656)(355.75122833,677.29343725)
\curveto(356.21264035,677.29343656)(356.60056399,677.51670196)(356.9150004,677.96323412)
\curveto(357.2294282,678.40976357)(357.38664426,679.30282518)(357.38664903,680.64242163)
\curveto(357.38664426,681.98945219)(357.2294282,682.88437434)(356.9150004,683.32719078)
\curveto(356.60056399,683.76999377)(356.20891926,683.99139863)(355.74006505,683.99140601)
\curveto(355.27864675,683.99139863)(354.91025884,683.7960414)(354.6349002,683.40533374)
\curveto(354.2888368,682.90670088)(354.11580611,681.9857311)(354.11580762,680.64242163)
\closepath
}
}
{
\newrgbcolor{curcolor}{0 0 0}
\pscustom[linestyle=none,fillstyle=solid,fillcolor=curcolor]
{
\newpath
\moveto(360.01002125,676.60689545)
\lineto(360.01002125,677.75113178)
\lineto(361.15425758,677.75113178)
\lineto(361.15425758,676.60689545)
\closepath
}
}
{
\newrgbcolor{curcolor}{0 0 0}
\pscustom[linestyle=none,fillstyle=solid,fillcolor=curcolor]
{
\newpath
\moveto(362.61664755,680.64242163)
\curveto(362.61664708,681.609901)(362.71618624,682.38853909)(362.91526533,682.97833824)
\curveto(363.11434287,683.56812464)(363.41016953,684.02302789)(363.80274619,684.34304937)
\curveto(364.19531953,684.66305538)(364.68929423,684.82306225)(365.28467178,684.82307047)
\curveto(365.72375726,684.82306225)(366.10889008,684.73468636)(366.44007139,684.55794254)
\curveto(366.77124411,684.38118281)(367.04474422,684.12628814)(367.26057256,683.79325777)
\curveto(367.47639067,683.46021302)(367.64570027,683.05461421)(367.76850186,682.57646011)
\curveto(367.89129221,682.09829407)(367.95269019,681.45361522)(367.952696,680.64242163)
\curveto(367.95269019,679.68237636)(367.85408131,678.90745936)(367.65686905,678.3176683)
\curveto(367.45964576,677.72787382)(367.16474938,677.27204029)(366.77217901,676.95016635)
\curveto(366.37959938,676.62829171)(365.88376413,676.46735457)(365.28467178,676.46735443)
\curveto(364.49579755,676.46735457)(363.87623606,676.75015741)(363.42598545,677.31576381)
\curveto(362.88642611,677.99672258)(362.61664708,679.10560741)(362.61664755,680.64242163)
\closepath
\moveto(363.64925107,680.64242163)
\curveto(363.64924956,679.29910409)(363.80646562,678.40511221)(364.12089971,677.9604433)
\curveto(364.43532984,677.51577169)(364.82325347,677.29343656)(365.28467178,677.29343725)
\curveto(365.7460838,677.29343656)(366.13400744,677.51670196)(366.44844385,677.96323412)
\curveto(366.76287165,678.40976357)(366.92008771,679.30282518)(366.92009249,680.64242163)
\curveto(366.92008771,681.98945219)(366.76287165,682.88437434)(366.44844385,683.32719078)
\curveto(366.13400744,683.76999377)(365.74236271,683.99139863)(365.2735085,683.99140601)
\curveto(364.8120902,683.99139863)(364.44370229,683.7960414)(364.16834365,683.40533374)
\curveto(363.82228025,682.90670088)(363.64924956,681.9857311)(363.64925107,680.64242163)
\closepath
}
}
{
\newrgbcolor{curcolor}{0 0 0}
\pscustom[linestyle=none,fillstyle=solid,fillcolor=curcolor]
{
\newpath
\moveto(399.33503208,678.76699037)
\lineto(400.3397274,678.90094975)
\curveto(400.4550797,678.33162068)(400.6513672,677.9213705)(400.92859048,677.67019799)
\curveto(401.20580962,677.41902335)(401.54349854,677.29343656)(401.94165826,677.29343725)
\curveto(402.41423361,677.29343656)(402.81332051,677.45716452)(403.13892018,677.78462162)
\curveto(403.46451127,678.11207637)(403.62730896,678.51767518)(403.62731373,679.00141928)
\curveto(403.62730896,679.46283205)(403.47660481,679.84331351)(403.17520084,680.14286479)
\curveto(402.87378823,680.442409)(402.49051595,680.59218288)(402.02538287,680.59218686)
\curveto(401.83560411,680.59218288)(401.59931489,680.55497198)(401.31651451,680.48055405)
\lineto(401.42814732,681.36245327)
\curveto(401.49512437,681.35500634)(401.54908017,681.35128525)(401.5900149,681.35128999)
\curveto(402.01793752,681.35128525)(402.40307034,681.46291795)(402.74541451,681.68618843)
\curveto(403.0877509,681.90944875)(403.25892105,682.25364958)(403.25892545,682.71879195)
\curveto(403.25892105,683.08717375)(403.13426453,683.39230313)(402.88495553,683.63418101)
\curveto(402.63563846,683.87604484)(402.31376418,683.99698026)(401.9193317,683.99698765)
\curveto(401.52861418,683.99698026)(401.2030188,683.87418429)(400.94254459,683.62859937)
\curveto(400.68206619,683.38300041)(400.51461714,683.01461249)(400.44019693,682.52343452)
\lineto(399.43550161,682.70204702)
\curveto(399.558297,683.37555823)(399.83737876,683.8974411)(400.27274771,684.26769722)
\curveto(400.70811383,684.63793802)(401.24953243,684.82306225)(401.89700513,684.82307047)
\curveto(402.3435329,684.82306225)(402.75471335,684.72724418)(403.13054772,684.53561597)
\curveto(403.50637353,684.34397191)(403.79382774,684.08256533)(403.99291119,683.75139546)
\curveto(404.19198437,683.42021131)(404.29152353,683.0685683)(404.29152897,682.69646538)
\curveto(404.29152353,682.34295574)(404.19663574,682.02108145)(404.0068653,681.73084155)
\curveto(403.81708455,681.44059141)(403.53614225,681.20988382)(403.16403756,681.03871811)
\curveto(403.64777496,680.92708098)(404.02360505,680.69544313)(404.29152897,680.34380385)
\curveto(404.55944201,679.99215711)(404.69340126,679.55213821)(404.69340709,679.02374584)
\curveto(404.69340126,678.30929414)(404.43292495,677.70368674)(403.9119774,677.20692182)
\curveto(403.39101974,676.7101557)(402.73238681,676.46177294)(401.93607662,676.46177279)
\curveto(401.21790316,676.46177294)(400.62159848,676.67573561)(400.14716079,677.10366146)
\curveto(399.67272052,677.53158632)(399.40201122,678.08602874)(399.33503208,678.76699037)
\closepath
}
}
{
\newrgbcolor{curcolor}{0 0 0}
\pscustom[linestyle=none,fillstyle=solid,fillcolor=curcolor]
{
\newpath
\moveto(407.23863539,681.04429975)
\curveto(406.82187128,681.19686001)(406.51302081,681.41454378)(406.31208304,681.69735171)
\curveto(406.11114308,681.98014946)(406.01067365,682.31876866)(406.01067444,682.71321031)
\curveto(406.01067365,683.30857861)(406.22463633,683.80906522)(406.65256312,684.21467164)
\curveto(407.08048704,684.62026284)(407.64981381,684.82306225)(408.36054515,684.82307047)
\curveto(409.0749913,684.82306225)(409.64989971,684.61561148)(410.08527211,684.20071754)
\curveto(410.52063478,683.7858084)(410.73831855,683.28067043)(410.73832407,682.6853021)
\curveto(410.73831855,682.30574484)(410.63877939,681.9754981)(410.43970629,681.69456089)
\curveto(410.24062275,681.4136135)(409.93828419,681.19686001)(409.53268969,681.04429975)
\curveto(410.03503253,680.88056736)(410.41737453,680.61636996)(410.67971684,680.25170678)
\curveto(410.94204823,679.88703632)(411.07321665,679.45166878)(411.0732225,678.94560287)
\curveto(411.07321665,678.24603561)(410.82576416,677.65810338)(410.3308643,677.18180443)
\curveto(409.83595421,676.70550433)(409.18476345,676.46735457)(408.37729008,676.46735443)
\curveto(407.56981038,676.46735457)(406.91861962,676.70643461)(406.42371585,677.18459525)
\curveto(405.92880967,677.66275475)(405.68135718,678.25905942)(405.68135765,678.97351108)
\curveto(405.68135718,679.50562459)(405.8162467,679.95122512)(406.08602659,680.31031401)
\curveto(406.35580475,680.6693955)(406.7400073,680.91405717)(407.23863539,681.04429975)
\closepath
\moveto(407.03769632,682.74670015)
\curveto(407.0376945,682.35970065)(407.16235102,682.04340799)(407.41166625,681.79782124)
\curveto(407.66097708,681.55222411)(407.98471192,681.42942814)(408.38287172,681.42943296)
\curveto(408.76986192,681.42942814)(409.08708484,681.55129383)(409.33454145,681.79503042)
\curveto(409.58198982,682.03875663)(409.70571606,682.33737411)(409.70572055,682.69088374)
\curveto(409.70571606,683.05926557)(409.57826873,683.36904632)(409.32337816,683.62022691)
\curveto(409.06847939,683.87139347)(408.75125647,683.99698026)(408.37170844,683.99698765)
\curveto(407.98843301,683.99698026)(407.67027981,683.87418429)(407.41724789,683.62859937)
\curveto(407.16421156,683.38300041)(407.0376945,683.08903429)(407.03769632,682.74670015)
\closepath
\moveto(406.71396117,678.96792944)
\curveto(406.71395967,678.68140314)(406.78186956,678.40418194)(406.91769105,678.13626498)
\curveto(407.05350914,677.86834497)(407.25537827,677.6608942)(407.52329906,677.51391205)
\curveto(407.79121524,677.36692809)(408.07959971,677.29343656)(408.38845336,677.29343725)
\curveto(408.8684708,677.29343656)(409.26476689,677.4478618)(409.57734281,677.75671342)
\curveto(409.88991002,678.06556274)(410.0461958,678.45813774)(410.04620063,678.93443959)
\curveto(410.0461958,679.41817897)(409.88525865,679.81819615)(409.56338871,680.13449233)
\curveto(409.24151008,680.45078146)(408.83870208,680.60892778)(408.35496351,680.60893178)
\curveto(407.88238194,680.60892778)(407.49073721,680.452642)(407.18002816,680.14007397)
\curveto(406.86931518,679.82749887)(406.71395967,679.43678442)(406.71396117,678.96792944)
\closepath
}
}
{
\newrgbcolor{curcolor}{0 0 0}
\pscustom[linestyle=none,fillstyle=solid,fillcolor=curcolor]
{
\newpath
\moveto(417.26884377,682.78577163)
\lineto(416.2697301,682.70762866)
\curveto(416.18041925,683.10205811)(416.05390219,683.38858204)(415.89017854,683.56720132)
\curveto(415.61853465,683.8537183)(415.28363655,683.99698026)(414.88548322,683.99698765)
\curveto(414.56546617,683.99698026)(414.28452387,683.9076741)(414.04265548,683.7290689)
\curveto(413.72636037,683.4983542)(413.47704734,683.16159555)(413.29471564,682.71879195)
\curveto(413.11238051,682.27597612)(413.01749272,681.64525136)(413.01005197,680.82661577)
\curveto(413.25192139,681.19499946)(413.54774805,681.46849958)(413.89753283,681.64711694)
\curveto(414.24731297,681.82572423)(414.61384034,681.91503039)(414.99711603,681.91503569)
\curveto(415.66690882,681.91503039)(416.23716587,681.66850817)(416.70788889,681.17546831)
\curveto(417.17860165,680.68241931)(417.4139606,680.04518264)(417.41396643,679.26375639)
\curveto(417.4139606,678.75024331)(417.30325817,678.27301351)(417.08185881,677.83206557)
\curveto(416.86044845,677.39111517)(416.55624934,677.05342625)(416.16926057,676.81899779)
\curveto(415.78226262,676.58456891)(415.34317399,676.46735457)(414.85199338,676.46735443)
\curveto(414.01474485,676.46735457)(413.33192483,676.77527477)(412.80353126,677.39111596)
\curveto(412.27513525,678.00695557)(412.01093786,679.02188288)(412.01093829,680.43590093)
\curveto(412.01093786,682.01736036)(412.30304343,683.16717718)(412.88725587,683.88535484)
\curveto(413.3970439,684.51049069)(414.08358501,684.82306225)(414.94688127,684.82307047)
\curveto(415.59062648,684.82306225)(416.11809099,684.64258938)(416.52927639,684.28165132)
\curveto(416.94045189,683.92069792)(417.1869741,683.42207185)(417.26884377,682.78577163)
\closepath
\moveto(413.1663379,679.25817475)
\curveto(413.16633632,678.91211073)(413.23982785,678.58093371)(413.38681271,678.26464272)
\curveto(413.53379396,677.94834841)(413.73938418,677.70740783)(414.003584,677.54182025)
\curveto(414.26777897,677.37623081)(414.54500018,677.29343656)(414.83524845,677.29343725)
\curveto(415.25944946,677.29343656)(415.62411629,677.4646067)(415.92925002,677.80694818)
\curveto(416.23437505,678.14928727)(416.38693975,678.61442352)(416.38694455,679.20235834)
\curveto(416.38693975,679.76796143)(416.2362356,680.21356197)(415.93483166,680.53916128)
\curveto(415.63341901,680.86475272)(415.25386783,681.02755041)(414.79617697,681.02755483)
\curveto(414.34220077,681.02755041)(413.95706795,680.86475272)(413.64077736,680.53916128)
\curveto(413.32448265,680.21356197)(413.16633632,679.78656688)(413.1663379,679.25817475)
\closepath
}
}
{
\newrgbcolor{curcolor}{0 0 0}
\pscustom[linestyle=none,fillstyle=solid,fillcolor=curcolor]
{
\newpath
\moveto(418.78146759,676.60689545)
\lineto(418.78146759,684.78958062)
\lineto(421.85136994,684.78958062)
\curveto(422.47650916,684.78957244)(422.97792605,684.70677819)(423.3556221,684.54119761)
\curveto(423.73330732,684.37560117)(424.02913398,684.12070651)(424.24310296,683.77651285)
\curveto(424.45705934,683.43230485)(424.56404067,683.07228939)(424.56404729,682.69646538)
\curveto(424.56404067,682.34667683)(424.46915288,682.01736036)(424.27938362,681.70851499)
\curveto(424.08960169,681.39965942)(423.80307776,681.15034638)(423.41981096,680.96057515)
\curveto(423.91471046,680.81544828)(424.29519192,680.56799579)(424.56125647,680.21821694)
\curveto(424.82730779,679.86843087)(424.96033676,679.45538987)(424.96034378,678.97909272)
\curveto(424.96033676,678.59581807)(424.87940305,678.2395237)(424.71754241,677.91020854)
\curveto(424.55566822,677.58089076)(424.35565963,677.32692637)(424.11751604,677.14831459)
\curveto(423.87936011,676.96970173)(423.58074263,676.83481221)(423.22166272,676.74364564)
\curveto(422.86257225,676.6524788)(422.42255336,676.60689545)(421.90160471,676.60689545)
\closepath
\moveto(419.86430588,681.35128999)
\lineto(421.63368596,681.35128999)
\curveto(422.11370288,681.35128525)(422.45790371,681.38291451)(422.66628948,681.44617788)
\curveto(422.94164542,681.52803702)(423.14909619,681.66385681)(423.28864241,681.85363765)
\curveto(423.42817794,682.04340799)(423.49794838,682.28155775)(423.49795393,682.56808765)
\curveto(423.49794838,682.83972126)(423.4328293,683.0788013)(423.30259651,683.28532847)
\curveto(423.172353,683.49184229)(422.9862985,683.63324371)(422.74443245,683.70953316)
\curveto(422.50255679,683.7858084)(422.08765525,683.82394958)(421.49972658,683.82395679)
\lineto(419.86430588,683.82395679)
\closepath
\moveto(419.86430588,677.57251928)
\lineto(421.90160471,677.57251928)
\curveto(422.25138322,677.57251831)(422.49697516,677.58554213)(422.63838127,677.61159076)
\curveto(422.88768961,677.65624284)(423.09607065,677.73066464)(423.26352502,677.83485639)
\curveto(423.43096876,677.93904568)(423.56864909,678.0906801)(423.67656643,678.2897601)
\curveto(423.78447231,678.48883673)(423.83842812,678.71861404)(423.83843401,678.97909272)
\curveto(423.83842812,679.28421973)(423.76028523,679.54934739)(423.6040051,679.77447651)
\curveto(423.44771366,679.99959929)(423.23096017,680.15774562)(422.95374397,680.24891596)
\curveto(422.67651775,680.34007903)(422.27743085,680.38566238)(421.75648205,680.38566616)
\lineto(419.86430588,680.38566616)
\closepath
}
}
{
\newrgbcolor{curcolor}{0 0 0}
\pscustom[linestyle=none,fillstyle=solid,fillcolor=curcolor]
{
\newpath
\moveto(426.0934174,679.23584819)
\lineto(427.11485764,679.32515444)
\curveto(427.16323027,678.91583182)(427.27579325,678.58000344)(427.4525469,678.3176683)
\curveto(427.6292968,678.05532974)(427.90372719,677.84322761)(428.27583889,677.68136127)
\curveto(428.6479452,677.51949278)(429.06656783,677.43855907)(429.53170804,677.4385599)
\curveto(429.94474508,677.43855907)(430.3094119,677.49995706)(430.6257096,677.62275404)
\curveto(430.94199721,677.745549)(431.17735615,677.91392832)(431.33178714,678.12789252)
\curveto(431.48620663,678.34185368)(431.56341924,678.57535208)(431.56342523,678.82838842)
\curveto(431.56341924,679.08514141)(431.48899744,679.30933709)(431.3401596,679.50097612)
\curveto(431.19131024,679.69260936)(430.9457183,679.85354651)(430.60338304,679.98378803)
\curveto(430.3838337,680.06936973)(429.89823145,680.2023987)(429.14657483,680.38287534)
\curveto(428.39491108,680.56334443)(427.86837684,680.7335843)(427.56697053,680.89359546)
\curveto(427.17625409,681.09825112)(426.88507879,681.35221552)(426.69344377,681.6554894)
\curveto(426.50180652,681.95875319)(426.40598845,682.29830266)(426.40598928,682.67413882)
\curveto(426.40598845,683.08717375)(426.52320279,683.47323684)(426.75763264,683.83232925)
\curveto(426.99206013,684.19140722)(427.33440042,684.46397706)(427.78465452,684.65003961)
\curveto(428.23490421,684.83608607)(428.73539082,684.92911332)(429.28611585,684.92912164)
\curveto(429.89264982,684.92911332)(430.42755651,684.8314347)(430.89083753,684.63608551)
\curveto(431.35410793,684.44072025)(431.7104023,684.15326604)(431.95972171,683.77372203)
\curveto(432.20902837,683.39416368)(432.34298761,682.96437778)(432.36159984,682.48436304)
\lineto(431.32341468,682.40622007)
\curveto(431.26759259,682.92344579)(431.07874727,683.31416024)(430.75687816,683.5783646)
\curveto(430.43499869,683.84255503)(429.95962944,683.97465372)(429.33076897,683.97466109)
\curveto(428.67585338,683.97465372)(428.19862358,683.85464857)(427.89907815,683.61464527)
\curveto(427.59952808,683.37462795)(427.44975421,683.0853132)(427.44975608,682.74670015)
\curveto(427.44975421,682.4527279)(427.55580527,682.21085704)(427.76790959,682.02108687)
\curveto(427.97628845,681.83130586)(428.52049787,681.63687891)(429.40053948,681.43780542)
\curveto(430.28057345,681.23872227)(430.88432031,681.06476131)(431.21178187,680.91592202)
\curveto(431.68807576,680.6963734)(432.03971877,680.41822192)(432.26671195,680.08146674)
\curveto(432.49369175,679.74470462)(432.607185,679.35678099)(432.60719203,678.91769467)
\curveto(432.607185,678.48232483)(432.48252848,678.07207465)(432.2332221,677.68694291)
\curveto(431.98390242,677.30180901)(431.6257475,677.00226126)(431.15875628,676.78829877)
\curveto(430.6917539,676.57433591)(430.16614994,676.46735457)(429.5819428,676.46735443)
\curveto(428.84144188,676.46735457)(428.22095012,676.57526618)(427.72046565,676.79108959)
\curveto(427.2199769,677.00691263)(426.8274019,677.33157773)(426.54273947,677.76508588)
\curveto(426.25807512,678.19859171)(426.10830125,678.68884532)(426.0934174,679.23584819)
\closepath
}
}
{
\newrgbcolor{curcolor}{0 0 0}
\pscustom[linestyle=none,fillstyle=solid,fillcolor=curcolor]
{
\newpath
\moveto(434.09749065,676.60689545)
\lineto(434.09749065,684.78958062)
\lineto(436.91621917,684.78958062)
\curveto(437.55252187,684.78957244)(438.03812412,684.75050099)(438.37302738,684.67236617)
\curveto(438.84187957,684.56444649)(439.24189675,684.36908927)(439.57308012,684.0862939)
\curveto(440.0047202,683.7216196)(440.32752477,683.25555307)(440.54149477,682.68809292)
\curveto(440.75545012,682.12062061)(440.86243146,681.47222067)(440.86243911,680.74289116)
\curveto(440.86243146,680.12146499)(440.7898702,679.57074366)(440.64475512,679.09072553)
\curveto(440.49962518,678.61070243)(440.31357068,678.21347607)(440.08659106,677.89904525)
\curveto(439.85959769,677.58461185)(439.61121493,677.33715937)(439.34144203,677.15668705)
\curveto(439.07165688,676.97621363)(438.7460615,676.83946357)(438.36465492,676.74643646)
\curveto(437.98323804,676.65340907)(437.54507969,676.60689545)(437.05017855,676.60689545)
\closepath
\moveto(435.18032894,677.57251928)
\lineto(436.92738246,677.57251928)
\curveto(437.4669368,677.57251831)(437.89021079,677.62275303)(438.1972057,677.72322357)
\curveto(438.50419065,677.82369189)(438.74885232,677.96509331)(438.93119144,678.14742826)
\curveto(439.18794094,678.40418194)(439.38794953,678.74931304)(439.53121781,679.1828226)
\curveto(439.67447346,679.61632702)(439.74610445,680.14193098)(439.74611098,680.75963608)
\curveto(439.74610445,681.61548264)(439.6056333,682.2731853)(439.32469711,682.73274605)
\curveto(439.0437487,683.19229454)(438.70233869,683.50021474)(438.30046605,683.65650757)
\curveto(438.01021594,683.76813322)(437.54321914,683.82394958)(436.89947425,683.82395679)
\lineto(435.18032894,683.82395679)
\closepath
}
}
{
\newrgbcolor{curcolor}{0 0 0}
\pscustom[linestyle=none,fillstyle=solid,fillcolor=curcolor]
{
\newpath
\moveto(445.12122901,680.64242163)
\curveto(445.12122853,681.609901)(445.22076769,682.38853909)(445.41984678,682.97833824)
\curveto(445.61892432,683.56812464)(445.91475098,684.02302789)(446.30732764,684.34304937)
\curveto(446.69990098,684.66305538)(447.19387568,684.82306225)(447.78925323,684.82307047)
\curveto(448.22833871,684.82306225)(448.61347153,684.73468636)(448.94465284,684.55794254)
\curveto(449.27582556,684.38118281)(449.54932568,684.12628814)(449.76515402,683.79325777)
\curveto(449.98097212,683.46021302)(450.15028172,683.05461421)(450.27308331,682.57646011)
\curveto(450.39587366,682.09829407)(450.45727164,681.45361522)(450.45727746,680.64242163)
\curveto(450.45727164,679.68237636)(450.35866276,678.90745936)(450.1614505,678.3176683)
\curveto(449.96422721,677.72787382)(449.66933083,677.27204029)(449.27676046,676.95016635)
\curveto(448.88418083,676.62829171)(448.38834558,676.46735457)(447.78925323,676.46735443)
\curveto(447.000379,676.46735457)(446.38081751,676.75015741)(445.9305669,677.31576381)
\curveto(445.39100756,677.99672258)(445.12122853,679.10560741)(445.12122901,680.64242163)
\closepath
\moveto(446.15383252,680.64242163)
\curveto(446.15383102,679.29910409)(446.31104707,678.40511221)(446.62548116,677.9604433)
\curveto(446.93991129,677.51577169)(447.32783492,677.29343656)(447.78925323,677.29343725)
\curveto(448.25066525,677.29343656)(448.63858889,677.51670196)(448.9530253,677.96323412)
\curveto(449.26745311,678.40976357)(449.42466916,679.30282518)(449.42467394,680.64242163)
\curveto(449.42466916,681.98945219)(449.26745311,682.88437434)(448.9530253,683.32719078)
\curveto(448.63858889,683.76999377)(448.24694416,683.99139863)(447.77808995,683.99140601)
\curveto(447.31667165,683.99139863)(446.94828374,683.7960414)(446.6729251,683.40533374)
\curveto(446.3268617,682.90670088)(446.15383102,681.9857311)(446.15383252,680.64242163)
\closepath
}
}
{
\newrgbcolor{curcolor}{0 0 0}
\pscustom[linestyle=none,fillstyle=solid,fillcolor=curcolor]
{
\newpath
\moveto(452.04804615,676.60689545)
\lineto(452.04804615,677.75113178)
\lineto(453.19228248,677.75113178)
\lineto(453.19228248,676.60689545)
\closepath
}
}
{
\newrgbcolor{curcolor}{0 0 0}
\pscustom[linestyle=none,fillstyle=solid,fillcolor=curcolor]
{
\newpath
\moveto(458.43902481,676.60689545)
\lineto(457.43432949,676.60689545)
\lineto(457.43432949,683.00903726)
\curveto(457.19245539,682.77832327)(456.87523246,682.54761569)(456.48265977,682.31691382)
\curveto(456.09008246,682.08620053)(455.73750918,681.91316984)(455.42493886,681.79782124)
\lineto(455.42493886,682.76902671)
\curveto(455.98682222,683.03321794)(456.4780061,683.35323169)(456.89849199,683.7290689)
\curveto(457.31897245,684.10489187)(457.61665965,684.4695587)(457.79155449,684.82307047)
\lineto(458.43902481,684.82307047)
\closepath
}
}
{
\newrgbcolor{curcolor}{0 0 0}
\pscustom[linestyle=none,fillstyle=solid,fillcolor=curcolor]
{
\newpath
\moveto(931.86788712,675.40630463)
\lineto(931.86788712,676.36634682)
\lineto(935.33408596,676.37192846)
\lineto(935.33408596,673.33551596)
\curveto(934.8019619,672.91131055)(934.25310112,672.59222708)(933.68750197,672.37826459)
\curveto(933.12188975,672.16430173)(932.54139971,672.05732039)(931.94603009,672.05732025)
\curveto(931.14226985,672.05732039)(930.41200593,672.22942081)(929.75523614,672.57362201)
\curveto(929.09846115,672.91782246)(928.6026259,673.41551825)(928.26772891,674.06671088)
\curveto(927.93282969,674.71789977)(927.76538064,675.44537287)(927.76538125,676.24913237)
\curveto(927.76538064,677.04544159)(927.93189942,677.78872932)(928.26493809,678.47899781)
\curveto(928.59797454,679.16925373)(929.07706488,679.68183388)(929.70221055,680.0167398)
\curveto(930.32735113,680.35163009)(931.04738206,680.51907914)(931.86230548,680.51908746)
\curveto(932.45395409,680.51907914)(932.98886078,680.42326107)(933.46702716,680.23163297)
\curveto(933.94518092,680.03998879)(934.32008074,679.77300058)(934.59172775,679.43066753)
\curveto(934.86335989,679.08832002)(935.06988039,678.64178921)(935.21128986,678.09107378)
\lineto(934.23450275,677.82315503)
\curveto(934.1116997,678.23991149)(933.95913501,678.56736741)(933.77680822,678.80552378)
\curveto(933.59446819,679.04366694)(933.33399188,679.2343728)(932.99537853,679.37764195)
\curveto(932.6567535,679.52089673)(932.2809234,679.59252772)(931.86788712,679.59253511)
\curveto(931.37297743,679.59252772)(930.94505208,679.51717564)(930.58410977,679.36647867)
\curveto(930.22316061,679.21576735)(929.93198532,679.01761931)(929.71058301,678.77203394)
\curveto(929.4891756,678.52643542)(929.31707519,678.25665639)(929.19428125,677.96269605)
\curveto(928.98589818,677.45662204)(928.88170765,676.90776126)(928.88170938,676.31611206)
\curveto(928.88170765,675.58677429)(929.00729444,674.97651553)(929.25847012,674.48533393)
\curveto(929.5096416,673.99414776)(929.87523869,673.62948093)(930.35526251,673.39133236)
\curveto(930.83527992,673.15318141)(931.34506926,673.03410653)(931.88463204,673.03410736)
\curveto(932.35348466,673.03410653)(932.81117873,673.12434296)(933.25771564,673.30481693)
\curveto(933.70424034,673.48528869)(934.04285954,673.6778551)(934.27357424,673.88251674)
\lineto(934.27357424,675.40630463)
\closepath
}
}
{
\newrgbcolor{curcolor}{0 0 0}
\pscustom[linestyle=none,fillstyle=solid,fillcolor=curcolor]
{
\newpath
\moveto(936.91369038,672.19686127)
\lineto(936.91369038,680.37954644)
\lineto(938.02443687,680.37954644)
\lineto(942.32230016,673.95507807)
\lineto(942.32230016,680.37954644)
\lineto(943.36048532,680.37954644)
\lineto(943.36048532,672.19686127)
\lineto(942.24973883,672.19686127)
\lineto(937.95187554,678.62691128)
\lineto(937.95187554,672.19686127)
\closepath
}
}
{
\newrgbcolor{curcolor}{0 0 0}
\pscustom[linestyle=none,fillstyle=solid,fillcolor=curcolor]
{
\newpath
\moveto(950.55522093,680.37954644)
\lineto(951.63805922,680.37954644)
\lineto(951.63805922,675.65189682)
\curveto(951.63805188,674.82953247)(951.54502463,674.17648117)(951.35897718,673.69274096)
\curveto(951.17291563,673.20899776)(950.83708725,672.81549249)(950.35149105,672.51222396)
\curveto(949.86588275,672.20895481)(949.22864608,672.05732039)(948.43977913,672.05732025)
\curveto(947.67323045,672.05732039)(947.04622677,672.18941909)(946.55876623,672.45361674)
\curveto(946.07130118,672.71781387)(945.72337927,673.10015587)(945.51499944,673.60064389)
\curveto(945.30661718,674.10112909)(945.20242666,674.78487939)(945.20242756,675.65189682)
\lineto(945.20242756,680.37954644)
\lineto(946.28526584,680.37954644)
\lineto(946.28526584,675.65747846)
\curveto(946.28526386,674.94674681)(946.35131321,674.42300338)(946.48341409,674.08624662)
\curveto(946.6155106,673.74948609)(946.8424971,673.48994006)(947.16437424,673.30760775)
\curveto(947.48624567,673.12527323)(947.87975094,673.03410653)(948.34489124,673.03410736)
\curveto(949.14120047,673.03410653)(949.7086667,673.21457939)(950.04729163,673.57552651)
\curveto(950.38590508,673.93647086)(950.55521468,674.63045415)(950.55522093,675.65747846)
\closepath
}
}
{
\newrgbcolor{curcolor}{0 0 0}
\pscustom[linestyle=none,fillstyle=solid,fillcolor=curcolor]
{
\newpath
\moveto(952.56460953,672.05732025)
\lineto(954.9368068,680.51908746)
\lineto(955.74056305,680.51908746)
\lineto(953.37394742,672.05732025)
\closepath
}
}
{
\newrgbcolor{curcolor}{0 0 0}
\pscustom[linestyle=none,fillstyle=solid,fillcolor=curcolor]
{
\newpath
\moveto(956.57222763,672.19686127)
\lineto(956.57222763,680.37954644)
\lineto(957.65506592,680.37954644)
\lineto(957.65506592,673.1624851)
\lineto(961.68501046,673.1624851)
\lineto(961.68501046,672.19686127)
\closepath
}
}
{
\newrgbcolor{curcolor}{0 0 0}
\pscustom[linestyle=none,fillstyle=solid,fillcolor=curcolor]
{
\newpath
\moveto(962.85715611,679.22414683)
\lineto(962.85715611,680.37954644)
\lineto(963.86185142,680.37954644)
\lineto(963.86185142,679.22414683)
\closepath
\moveto(962.85715611,672.19686127)
\lineto(962.85715611,678.12456363)
\lineto(963.86185142,678.12456363)
\lineto(963.86185142,672.19686127)
\closepath
}
}
{
\newrgbcolor{curcolor}{0 0 0}
\pscustom[linestyle=none,fillstyle=solid,fillcolor=curcolor]
{
\newpath
\moveto(965.39680152,672.19686127)
\lineto(965.39680152,678.12456363)
\lineto(966.3010273,678.12456363)
\lineto(966.3010273,677.28173589)
\curveto(966.73639318,677.93292156)(967.36525739,678.25851694)(968.18762183,678.258523)
\curveto(968.54484293,678.25851694)(968.87322913,678.19432814)(969.17278141,678.0659564)
\curveto(969.47232463,677.93757292)(969.6965203,677.7691936)(969.8453691,677.56081792)
\curveto(969.9942075,677.35243151)(970.09839802,677.10497903)(970.15794098,676.81845972)
\curveto(970.19514637,676.63240059)(970.21375182,676.30680521)(970.21375739,675.8416726)
\lineto(970.21375739,672.19686127)
\lineto(969.20906207,672.19686127)
\lineto(969.20906207,675.80260112)
\curveto(969.20905751,676.21191742)(969.16998606,676.51797707)(969.09184762,676.720781)
\curveto(969.01370028,676.92357589)(968.87508967,677.0854433)(968.67601539,677.20638374)
\curveto(968.47693304,677.32731416)(968.24343464,677.38778187)(967.97551949,677.38778706)
\curveto(967.5475908,677.38778187)(967.17827262,677.25196208)(966.86756382,676.98032729)
\curveto(966.55685058,676.70868294)(966.40149507,676.19331197)(966.40149683,675.43421284)
\lineto(966.40149683,672.19686127)
\closepath
}
}
{
\newrgbcolor{curcolor}{0 0 0}
\pscustom[linestyle=none,fillstyle=solid,fillcolor=curcolor]
{
\newpath
\moveto(975.64469484,672.19686127)
\lineto(975.64469484,673.06759721)
\curveto(975.18327504,672.39780013)(974.55627137,672.06290203)(973.76368195,672.06290189)
\curveto(973.41389673,672.06290203)(973.08737107,672.12988165)(972.78410401,672.26384096)
\curveto(972.4808334,672.39780013)(972.25570745,672.56617945)(972.1087255,672.76897943)
\curveto(971.96174134,672.97177827)(971.85848109,673.22016103)(971.79894444,673.51412846)
\curveto(971.75801166,673.71134491)(971.73754566,674.02391648)(971.73754639,674.45184409)
\lineto(971.73754639,678.12456363)
\lineto(972.74224171,678.12456363)
\lineto(972.74224171,674.83697729)
\curveto(972.74223997,674.31230095)(972.76270597,673.9587974)(972.80363976,673.77646557)
\curveto(972.86689649,673.5122666)(973.00085573,673.30481583)(973.20551788,673.15411264)
\curveto(973.41017564,673.00340753)(973.66320976,672.92805546)(973.96462101,672.92805619)
\curveto(974.26602634,672.92805546)(974.54882919,673.00526808)(974.81303039,673.15969428)
\curveto(975.07722397,673.31411855)(975.26420875,673.52436014)(975.37398527,673.79041967)
\curveto(975.48375306,674.05647601)(975.53863914,674.44253911)(975.53864367,674.9486101)
\lineto(975.53864367,678.12456363)
\lineto(976.54333898,678.12456363)
\lineto(976.54333898,672.19686127)
\closepath
}
}
{
\newrgbcolor{curcolor}{0 0 0}
\pscustom[linestyle=none,fillstyle=solid,fillcolor=curcolor]
{
\newpath
\moveto(977.45314752,672.19686127)
\lineto(979.61882409,675.2779269)
\lineto(977.6150151,678.12456363)
\lineto(978.87088424,678.12456363)
\lineto(979.78069167,676.73473511)
\curveto(979.9518594,676.47053318)(980.08953973,676.24912832)(980.19373308,676.07051987)
\curveto(980.35745821,676.31610794)(980.50816236,676.53379171)(980.64584597,676.72357183)
\lineto(981.64495964,678.12456363)
\lineto(982.84501238,678.12456363)
\lineto(980.79655026,675.33374331)
\lineto(983.00129832,672.19686127)
\lineto(981.76775574,672.19686127)
\lineto(980.55095808,674.03880268)
\lineto(980.22722292,674.53556869)
\lineto(978.66994518,672.19686127)
\closepath
}
}
{
\newrgbcolor{curcolor}{0 0 0}
\pscustom[linestyle=none,fillstyle=solid,fillcolor=curcolor]
{
\newpath
\moveto(464.62073594,631.65694428)
\lineto(464.62073594,639.83962945)
\lineto(465.73148243,639.83962945)
\lineto(470.02934572,633.41516108)
\lineto(470.02934572,639.83962945)
\lineto(471.06753088,639.83962945)
\lineto(471.06753088,631.65694428)
\lineto(469.95678439,631.65694428)
\lineto(465.6589211,638.08699429)
\lineto(465.6589211,631.65694428)
\closepath
}
}
{
\newrgbcolor{curcolor}{0 0 0}
\pscustom[linestyle=none,fillstyle=solid,fillcolor=curcolor]
{
\newpath
\moveto(476.82220225,633.56586537)
\lineto(477.86038741,633.43748764)
\curveto(477.6966536,632.83094818)(477.39338476,632.36023029)(476.95057998,632.02533256)
\curveto(476.50776533,631.69043409)(475.94215964,631.52298503)(475.25376123,631.5229849)
\curveto(474.38674401,631.52298503)(473.69927262,631.78997324)(473.19134501,632.32395033)
\curveto(472.68341504,632.85792608)(472.42945065,633.60679546)(472.42945107,634.57056069)
\curveto(472.42945065,635.5678099)(472.68620586,636.34179663)(473.19971747,636.89252319)
\curveto(473.71322671,637.44323928)(474.37930183,637.71859995)(475.19794482,637.71860601)
\curveto(475.99053381,637.71859995)(476.63800348,637.44882092)(477.14035576,636.90926812)
\curveto(477.64269779,636.36970481)(477.89387137,635.61060244)(477.89387725,634.63195874)
\curveto(477.89387137,634.57241832)(477.89201082,634.48311216)(477.88829561,634.36403998)
\lineto(473.46763622,634.36403998)
\curveto(473.50484567,633.71284652)(473.68903963,633.21422046)(474.02021865,632.86816029)
\curveto(474.35139365,632.52209771)(474.76443465,632.34906702)(475.25934287,632.34906771)
\curveto(475.62772753,632.34906702)(475.94215964,632.44581536)(476.20264014,632.63931303)
\curveto(476.46311225,632.83280873)(476.66963275,633.1416592)(476.82220225,633.56586537)
\closepath
\moveto(473.52345263,635.1901228)
\lineto(476.83336553,635.1901228)
\curveto(476.78870763,635.68874533)(476.66219057,636.06271488)(476.45381397,636.31203257)
\curveto(476.13379578,636.69902128)(475.71889424,636.89251796)(475.2091081,636.89252319)
\curveto(474.74768974,636.89251796)(474.3597661,636.73809272)(474.04533603,636.42924702)
\curveto(473.73090189,636.12039178)(473.55694093,635.70735078)(473.52345263,635.1901228)
\closepath
}
}
{
\newrgbcolor{curcolor}{0 0 0}
\pscustom[linestyle=none,fillstyle=solid,fillcolor=curcolor]
{
\newpath
\moveto(481.32100476,632.55558842)
\lineto(481.46612742,631.66810756)
\curveto(481.18332148,631.6085701)(480.93028736,631.57880138)(480.70702429,631.57880131)
\curveto(480.34235514,631.57880138)(480.05955229,631.63647828)(479.85861492,631.75183217)
\curveto(479.65767457,631.86718586)(479.51627315,632.01882028)(479.43441023,632.20673588)
\curveto(479.35254519,632.39465038)(479.3116132,632.79001619)(479.31161413,633.39283451)
\lineto(479.31161413,636.80321694)
\lineto(478.57483757,636.80321694)
\lineto(478.57483757,637.58464663)
\lineto(479.31161413,637.58464663)
\lineto(479.31161413,639.05261812)
\lineto(480.31072781,639.65543531)
\lineto(480.31072781,637.58464663)
\lineto(481.32100476,637.58464663)
\lineto(481.32100476,636.80321694)
\lineto(480.31072781,636.80321694)
\lineto(480.31072781,633.33701811)
\curveto(480.31072587,633.05049249)(480.32840105,632.86629854)(480.36375339,632.78443568)
\curveto(480.39910176,632.70257058)(480.45677865,632.6374515)(480.53678425,632.58907826)
\curveto(480.61678553,632.54070316)(480.73120905,632.51651607)(480.88005515,632.51651693)
\curveto(480.99168535,632.51651607)(481.1386684,632.52953989)(481.32100476,632.55558842)
\closepath
}
}
{
\newrgbcolor{curcolor}{0 0 0}
\pscustom[linestyle=none,fillstyle=solid,fillcolor=curcolor]
{
\newpath
\moveto(482.38151566,631.65694428)
\lineto(482.38151566,639.83962945)
\lineto(485.45141801,639.83962945)
\curveto(486.07655723,639.83962127)(486.57797411,639.75682702)(486.95567016,639.59124644)
\curveto(487.33335539,639.42565)(487.62918205,639.17075533)(487.84315102,638.82656168)
\curveto(488.0571074,638.48235368)(488.16408874,638.12233822)(488.16409536,637.74651421)
\curveto(488.16408874,637.39672566)(488.06920094,637.06740919)(487.87943169,636.75856382)
\curveto(487.68964976,636.44970824)(487.40312583,636.20039521)(487.01985903,636.01062397)
\curveto(487.51475853,635.86549711)(487.89523998,635.61804462)(488.16130454,635.26826577)
\curveto(488.42735586,634.91847969)(488.56038483,634.5054387)(488.56039184,634.02914155)
\curveto(488.56038483,633.6458669)(488.47945112,633.28957253)(488.31759048,632.96025736)
\curveto(488.15571629,632.63093959)(487.9557077,632.3769752)(487.71756411,632.19836342)
\curveto(487.47940817,632.01975055)(487.1807907,631.88486104)(486.82171079,631.79369447)
\curveto(486.46262032,631.70252763)(486.02260142,631.65694428)(485.50165277,631.65694428)
\closepath
\moveto(483.46435394,636.40133882)
\lineto(485.23373402,636.40133882)
\curveto(485.71375095,636.40133407)(486.05795178,636.43296334)(486.26633754,636.49622671)
\curveto(486.54169348,636.57808585)(486.74914425,636.71390564)(486.88869047,636.90368648)
\curveto(487.02822601,637.09345682)(487.09799644,637.33160658)(487.098002,637.61813648)
\curveto(487.09799644,637.88977009)(487.03287737,638.12885012)(486.90264457,638.3353773)
\curveto(486.77240106,638.54189112)(486.58634656,638.68329254)(486.34448051,638.75958199)
\curveto(486.10260486,638.83585723)(485.68770332,638.8739984)(485.09977465,638.87400562)
\lineto(483.46435394,638.87400562)
\closepath
\moveto(483.46435394,632.62256811)
\lineto(485.50165277,632.62256811)
\curveto(485.85143128,632.62256714)(486.09702322,632.63559096)(486.23842934,632.66163959)
\curveto(486.48773768,632.70629167)(486.69611872,632.78071347)(486.86357309,632.88490522)
\curveto(487.03101682,632.98909451)(487.16869715,633.14072893)(487.2766145,633.33980893)
\curveto(487.38452038,633.53888556)(487.43847618,633.76866287)(487.43848208,634.02914155)
\curveto(487.43847618,634.33426856)(487.36033329,634.59939622)(487.20405317,634.82452534)
\curveto(487.04776173,635.04964812)(486.83100823,635.20779444)(486.55379203,635.29896479)
\curveto(486.27656582,635.39012786)(485.87747891,635.43571121)(485.35653012,635.43571499)
\lineto(483.46435394,635.43571499)
\closepath
}
}
{
\newrgbcolor{curcolor}{0 0 0}
\pscustom[linestyle=none,fillstyle=solid,fillcolor=curcolor]
{
\newpath
\moveto(489.69346547,634.28589702)
\lineto(490.7149057,634.37520327)
\curveto(490.76327834,633.96588064)(490.87584131,633.63005227)(491.05259496,633.36771713)
\curveto(491.22934487,633.10537857)(491.50377526,632.89327644)(491.87588696,632.7314101)
\curveto(492.24799327,632.56954161)(492.66661589,632.4886079)(493.1317561,632.48860873)
\curveto(493.54479314,632.4886079)(493.90945997,632.55000588)(494.22575767,632.67280287)
\curveto(494.54204527,632.79559783)(494.77740422,632.96397715)(494.93183521,633.17794135)
\curveto(495.08625469,633.39190251)(495.16346731,633.62540091)(495.16347329,633.87843725)
\curveto(495.16346731,634.13519024)(495.08904551,634.35938592)(494.94020767,634.55102495)
\curveto(494.79135831,634.74265819)(494.54576636,634.90359533)(494.2034311,635.03383686)
\curveto(493.98388177,635.11941856)(493.49827952,635.25244752)(492.7466229,635.43292417)
\curveto(491.99495914,635.61339326)(491.4684249,635.78363313)(491.1670186,635.94364429)
\curveto(490.77630216,636.14829995)(490.48512686,636.40226435)(490.29349184,636.70553823)
\curveto(490.10185459,637.00880202)(490.00603652,637.34835149)(490.00603734,637.72418765)
\curveto(490.00603652,638.13722258)(490.12325085,638.52328567)(490.3576807,638.88237808)
\curveto(490.5921082,639.24145604)(490.93444848,639.51402589)(491.38470258,639.70008844)
\curveto(491.83495227,639.88613489)(492.33543888,639.97916214)(492.88616391,639.97917047)
\curveto(493.49269788,639.97916214)(494.02760458,639.88148353)(494.4908856,639.68613433)
\curveto(494.954156,639.49076908)(495.31045037,639.20331487)(495.55976978,638.82377085)
\curveto(495.80907643,638.4442125)(495.94303567,638.0144266)(495.9616479,637.53441187)
\lineto(494.92346275,637.4562689)
\curveto(494.86764065,637.97349461)(494.67879533,638.36420907)(494.35692622,638.62841343)
\curveto(494.03504676,638.89260385)(493.5596775,639.02470255)(492.93081704,639.02470992)
\curveto(492.27590144,639.02470255)(491.79867164,638.9046974)(491.49912621,638.6646941)
\curveto(491.19957615,638.42467678)(491.04980227,638.13536203)(491.04980414,637.79674898)
\curveto(491.04980227,637.50277672)(491.15585334,637.26090587)(491.36795766,637.07113569)
\curveto(491.57633651,636.88135469)(492.12054593,636.68692773)(493.00058755,636.48785425)
\curveto(493.88062152,636.2887711)(494.48436838,636.11481014)(494.81182993,635.96597085)
\curveto(495.28812383,635.74642223)(495.63976683,635.46827075)(495.86676001,635.13151557)
\curveto(496.09373982,634.79475345)(496.20723307,634.40682981)(496.20724009,633.9677435)
\curveto(496.20723307,633.53237365)(496.08257655,633.12212348)(495.83327017,632.73699174)
\curveto(495.58395048,632.35185784)(495.22579557,632.05231009)(494.75880435,631.8383476)
\curveto(494.29180197,631.62438474)(493.766198,631.5174034)(493.18199087,631.51740326)
\curveto(492.44148995,631.5174034)(491.82099818,631.62531501)(491.32051371,631.84113842)
\curveto(490.82002496,632.05696145)(490.42744996,632.38162656)(490.14278754,632.81513471)
\curveto(489.85812319,633.24864054)(489.70834931,633.73889415)(489.69346547,634.28589702)
\closepath
}
}
{
\newrgbcolor{curcolor}{0 0 0}
\pscustom[linestyle=none,fillstyle=solid,fillcolor=curcolor]
{
\newpath
\moveto(497.69753872,631.65694428)
\lineto(497.69753872,639.83962945)
\lineto(500.51626724,639.83962945)
\curveto(501.15256993,639.83962127)(501.63817218,639.80054982)(501.97307545,639.722415)
\curveto(502.44192763,639.61449532)(502.84194481,639.41913809)(503.17312818,639.13634273)
\curveto(503.60476827,638.77166843)(503.92757283,638.3056019)(504.14154283,637.73814175)
\curveto(504.35549819,637.17066944)(504.46247952,636.5222695)(504.46248717,635.79293999)
\curveto(504.46247952,635.17151382)(504.38991827,634.62079249)(504.24480319,634.14077436)
\curveto(504.09967324,633.66075126)(503.91361874,633.2635249)(503.68663912,632.94909408)
\curveto(503.45964576,632.63466068)(503.211263,632.3872082)(502.9414901,632.20673588)
\curveto(502.67170494,632.02626246)(502.34610956,631.8895124)(501.96470298,631.79648529)
\curveto(501.58328611,631.7034579)(501.14512775,631.65694428)(500.65022661,631.65694428)
\closepath
\moveto(498.780377,632.62256811)
\lineto(500.52743052,632.62256811)
\curveto(501.06698486,632.62256714)(501.49025886,632.67280186)(501.79725377,632.7732724)
\curveto(502.10423871,632.87374072)(502.34890038,633.01514214)(502.53123951,633.19747709)
\curveto(502.78798901,633.45423076)(502.9879976,633.79936186)(503.13126588,634.23287143)
\curveto(503.27452153,634.66637584)(503.34615251,635.19197981)(503.34615904,635.80968491)
\curveto(503.34615251,636.66553147)(503.20568136,637.32323413)(502.92474517,637.78279488)
\curveto(502.64379677,638.24234337)(502.30238676,638.55026357)(501.90051412,638.7065564)
\curveto(501.61026401,638.81818205)(501.14326721,638.8739984)(500.49952232,638.87400562)
\lineto(498.780377,638.87400562)
\closepath
}
}
{
\newrgbcolor{curcolor}{0 0 0}
\pscustom[linestyle=none,fillstyle=solid,fillcolor=curcolor]
{
\newpath
\moveto(508.72127707,635.69247046)
\curveto(508.7212766,636.65994983)(508.82081575,637.43858792)(509.01989484,638.02838706)
\curveto(509.21897239,638.61817346)(509.51479905,639.07307672)(509.90737571,639.3930982)
\curveto(510.29994904,639.71310421)(510.79392375,639.87311108)(511.3893013,639.87311929)
\curveto(511.82838678,639.87311108)(512.2135196,639.78473519)(512.54470091,639.60799136)
\curveto(512.87587362,639.43123164)(513.14937374,639.17633697)(513.36520208,638.8433066)
\curveto(513.58102019,638.51026185)(513.75032978,638.10466304)(513.87313138,637.62650894)
\curveto(513.99592172,637.1483429)(514.05731971,636.50366405)(514.05732552,635.69247046)
\curveto(514.05731971,634.73242519)(513.95871082,633.95750819)(513.76149857,633.36771713)
\curveto(513.56427528,632.77792265)(513.26937889,632.32208912)(512.87680853,632.00021517)
\curveto(512.4842289,631.67834054)(511.98839365,631.5174034)(511.3893013,631.51740326)
\curveto(510.60042706,631.5174034)(509.98086557,631.80020624)(509.53061496,632.36581264)
\curveto(508.99105562,633.0467714)(508.7212766,634.15565624)(508.72127707,635.69247046)
\closepath
\moveto(509.75388059,635.69247046)
\curveto(509.75387908,634.34915292)(509.91109514,633.45516104)(510.22552922,633.01049213)
\curveto(510.53995935,632.56582052)(510.92788299,632.34348539)(511.3893013,632.34348607)
\curveto(511.85071332,632.34348539)(512.23863695,632.56675079)(512.55307337,633.01328295)
\curveto(512.86750117,633.4598124)(513.02471722,634.35287401)(513.024722,635.69247046)
\curveto(513.02471722,637.03950101)(512.86750117,637.93442317)(512.55307337,638.3772396)
\curveto(512.23863695,638.8200426)(511.84699223,639.04144746)(511.37813801,639.04145484)
\curveto(510.91671972,639.04144746)(510.5483318,638.84609023)(510.27297317,638.45538257)
\curveto(509.92690977,637.95674971)(509.75387908,637.03577992)(509.75388059,635.69247046)
\closepath
}
}
{
\newrgbcolor{curcolor}{0 0 0}
\pscustom[linestyle=none,fillstyle=solid,fillcolor=curcolor]
{
\newpath
\moveto(515.64809422,631.65694428)
\lineto(515.64809422,632.80118061)
\lineto(516.79233055,632.80118061)
\lineto(516.79233055,631.65694428)
\closepath
}
}
{
\newrgbcolor{curcolor}{0 0 0}
\pscustom[linestyle=none,fillstyle=solid,fillcolor=curcolor]
{
\newpath
\moveto(519.80083498,636.09434858)
\curveto(519.38407087,636.24690884)(519.0752204,636.4645926)(518.87428263,636.74740054)
\curveto(518.67334268,637.03019829)(518.57287324,637.36881748)(518.57287404,637.76325913)
\curveto(518.57287324,638.35862743)(518.78683592,638.85911404)(519.21476271,639.26472047)
\curveto(519.64268663,639.67031167)(520.21201341,639.87311108)(520.92274475,639.87311929)
\curveto(521.63719089,639.87311108)(522.2120993,639.66566031)(522.6474717,639.25076636)
\curveto(523.08283437,638.83585723)(523.30051814,638.33071926)(523.30052366,637.73535093)
\curveto(523.30051814,637.35579367)(523.20097898,637.02554693)(523.00190588,636.74460972)
\curveto(522.80282235,636.46366233)(522.50048378,636.24690884)(522.09488928,636.09434858)
\curveto(522.59723212,635.93061618)(522.97957412,635.66641879)(523.24191643,635.30175561)
\curveto(523.50424782,634.93708514)(523.63541624,634.50171761)(523.6354221,633.9956517)
\curveto(523.63541624,633.29608444)(523.38796375,632.70815221)(522.89306389,632.23185326)
\curveto(522.3981538,631.75555316)(521.74696305,631.5174034)(520.93948967,631.51740326)
\curveto(520.13200997,631.5174034)(519.48081921,631.75648343)(518.98591544,632.23464408)
\curveto(518.49100926,632.71280357)(518.24355678,633.30910825)(518.24355724,634.02355991)
\curveto(518.24355678,634.55567341)(518.37844629,635.00127395)(518.64822619,635.36036284)
\curveto(518.91800435,635.71944432)(519.30220689,635.96410599)(519.80083498,636.09434858)
\closepath
\moveto(519.59989592,637.79674898)
\curveto(519.5998941,637.40974947)(519.72455061,637.09345682)(519.97386584,636.84787007)
\curveto(520.22317668,636.60227294)(520.54691151,636.47947696)(520.94507131,636.47948179)
\curveto(521.33206151,636.47947696)(521.64928443,636.60134266)(521.89674104,636.84507925)
\curveto(522.14418941,637.08880546)(522.26791565,637.38742293)(522.26792014,637.74093257)
\curveto(522.26791565,638.1093144)(522.14046832,638.41909515)(521.88557776,638.67027574)
\curveto(521.63067898,638.9214423)(521.31345606,639.04702909)(520.93390803,639.04703648)
\curveto(520.5506326,639.04702909)(520.2324794,638.92423312)(519.97944748,638.6786482)
\curveto(519.72641116,638.43304923)(519.5998941,638.13908312)(519.59989592,637.79674898)
\closepath
\moveto(519.27616076,634.01797827)
\curveto(519.27615926,633.73145197)(519.34406916,633.45423076)(519.47989064,633.18631381)
\curveto(519.61570873,632.9183938)(519.81757786,632.71094303)(520.08549865,632.56396088)
\curveto(520.35341483,632.41697692)(520.64179931,632.34348539)(520.95065295,632.34348607)
\curveto(521.43067039,632.34348539)(521.82696648,632.49791062)(522.13954241,632.80676225)
\curveto(522.45210961,633.11561157)(522.60839539,633.50818657)(522.60840022,633.98448842)
\curveto(522.60839539,634.4682278)(522.44745825,634.86824498)(522.1255883,635.18454116)
\curveto(521.80370967,635.50083028)(521.40090167,635.65897661)(520.91716311,635.65898061)
\curveto(520.44458153,635.65897661)(520.05293681,635.50269083)(519.74222775,635.1901228)
\curveto(519.43151477,634.8775477)(519.27615926,634.48683325)(519.27616076,634.01797827)
\closepath
}
}
{
\newrgbcolor{curcolor}{0 0 0}
\pscustom[linestyle=none,fillstyle=solid,fillcolor=curcolor]
{
\newpath
\moveto(212.86972246,631.65694428)
\lineto(212.86972246,639.83962945)
\lineto(218.38996505,639.83962945)
\lineto(218.38996505,638.87400562)
\lineto(213.95256075,638.87400562)
\lineto(213.95256075,636.33994077)
\lineto(217.79272951,636.33994077)
\lineto(217.79272951,635.37431694)
\lineto(213.95256075,635.37431694)
\lineto(213.95256075,631.65694428)
\closepath
}
}
{
\newrgbcolor{curcolor}{0 0 0}
\pscustom[linestyle=none,fillstyle=solid,fillcolor=curcolor]
{
\newpath
\moveto(219.66257906,631.65694428)
\lineto(219.66257906,637.58464663)
\lineto(220.56680484,637.58464663)
\lineto(220.56680484,636.68600249)
\curveto(220.79751078,637.10648064)(221.01054318,637.38370184)(221.20590269,637.51766695)
\curveto(221.40125763,637.65162033)(221.61615058,637.71859995)(221.85058219,637.71860601)
\curveto(222.18919845,637.71859995)(222.53339928,637.61068834)(222.88318571,637.39487085)
\lineto(222.53712399,636.46273686)
\curveto(222.29152843,636.60785457)(222.04593648,636.68041583)(221.80034742,636.68042085)
\curveto(221.58080023,636.68041583)(221.38358246,636.61436648)(221.20869351,636.48227261)
\curveto(221.03379999,636.35016909)(220.90914348,636.1669054)(220.83472359,635.932481)
\curveto(220.72308898,635.57525209)(220.66727262,635.18453763)(220.66727437,634.76033647)
\lineto(220.66727437,631.65694428)
\closepath
}
}
{
\newrgbcolor{curcolor}{0 0 0}
\pscustom[linestyle=none,fillstyle=solid,fillcolor=curcolor]
{
\newpath
\moveto(227.54943709,633.56586537)
\lineto(228.58762225,633.43748764)
\curveto(228.42388844,632.83094818)(228.1206196,632.36023029)(227.67781482,632.02533256)
\curveto(227.23500017,631.69043409)(226.66939448,631.52298503)(225.98099607,631.5229849)
\curveto(225.11397885,631.52298503)(224.42650746,631.78997324)(223.91857985,632.32395033)
\curveto(223.41064988,632.85792608)(223.15668549,633.60679546)(223.15668591,634.57056069)
\curveto(223.15668549,635.5678099)(223.4134407,636.34179663)(223.92695231,636.89252319)
\curveto(224.44046155,637.44323928)(225.10653667,637.71859995)(225.92517966,637.71860601)
\curveto(226.71776865,637.71859995)(227.36523832,637.44882092)(227.86759061,636.90926812)
\curveto(228.36993263,636.36970481)(228.62110621,635.61060244)(228.62111209,634.63195874)
\curveto(228.62110621,634.57241832)(228.61924566,634.48311216)(228.61553045,634.36403998)
\lineto(224.19487107,634.36403998)
\curveto(224.23208051,633.71284652)(224.41627447,633.21422046)(224.74745349,632.86816029)
\curveto(225.07862849,632.52209771)(225.49166949,632.34906702)(225.98657771,632.34906771)
\curveto(226.35496238,632.34906702)(226.66939448,632.44581536)(226.92987498,632.63931303)
\curveto(227.19034709,632.83280873)(227.39686759,633.1416592)(227.54943709,633.56586537)
\closepath
\moveto(224.25068747,635.1901228)
\lineto(227.56060037,635.1901228)
\curveto(227.51594247,635.68874533)(227.38942541,636.06271488)(227.18104881,636.31203257)
\curveto(226.86103062,636.69902128)(226.44612908,636.89251796)(225.93634294,636.89252319)
\curveto(225.47492458,636.89251796)(225.08700094,636.73809272)(224.77257087,636.42924702)
\curveto(224.45813673,636.12039178)(224.28417577,635.70735078)(224.25068747,635.1901228)
\closepath
}
}
{
\newrgbcolor{curcolor}{0 0 0}
\pscustom[linestyle=none,fillstyle=solid,fillcolor=curcolor]
{
\newpath
\moveto(233.91250758,633.56586537)
\lineto(234.95069274,633.43748764)
\curveto(234.78695892,632.83094818)(234.48369009,632.36023029)(234.04088531,632.02533256)
\curveto(233.59807066,631.69043409)(233.03246497,631.52298503)(232.34406656,631.5229849)
\curveto(231.47704934,631.52298503)(230.78957795,631.78997324)(230.28165034,632.32395033)
\curveto(229.77372037,632.85792608)(229.51975598,633.60679546)(229.51975639,634.57056069)
\curveto(229.51975598,635.5678099)(229.77651119,636.34179663)(230.2900228,636.89252319)
\curveto(230.80353204,637.44323928)(231.46960716,637.71859995)(232.28825015,637.71860601)
\curveto(233.08083914,637.71859995)(233.72830881,637.44882092)(234.23066109,636.90926812)
\curveto(234.73300312,636.36970481)(234.9841767,635.61060244)(234.98418258,634.63195874)
\curveto(234.9841767,634.57241832)(234.98231615,634.48311216)(234.97860094,634.36403998)
\lineto(230.55794155,634.36403998)
\curveto(230.595151,633.71284652)(230.77934495,633.21422046)(231.11052398,632.86816029)
\curveto(231.44169898,632.52209771)(231.85473997,632.34906702)(232.3496482,632.34906771)
\curveto(232.71803286,632.34906702)(233.03246497,632.44581536)(233.29294547,632.63931303)
\curveto(233.55341758,632.83280873)(233.75993807,633.1416592)(233.91250758,633.56586537)
\closepath
\moveto(230.61375796,635.1901228)
\lineto(233.92367086,635.1901228)
\curveto(233.87901296,635.68874533)(233.75249589,636.06271488)(233.54411929,636.31203257)
\curveto(233.22410111,636.69902128)(232.80919957,636.89251796)(232.29941343,636.89252319)
\curveto(231.83799507,636.89251796)(231.45007143,636.73809272)(231.13564136,636.42924702)
\curveto(230.82120722,636.12039178)(230.64724626,635.70735078)(230.61375796,635.1901228)
\closepath
}
}
{
\newrgbcolor{curcolor}{0 0 0}
\pscustom[linestyle=none,fillstyle=solid,fillcolor=curcolor]
{
\newpath
\moveto(239.47182099,631.65694428)
\lineto(239.47182099,639.83962945)
\lineto(242.54172334,639.83962945)
\curveto(243.16686256,639.83962127)(243.66827944,639.75682702)(244.04597549,639.59124644)
\curveto(244.42366072,639.42565)(244.71948737,639.17075533)(244.93345635,638.82656168)
\curveto(245.14741273,638.48235368)(245.25439407,638.12233822)(245.25440069,637.74651421)
\curveto(245.25439407,637.39672566)(245.15950627,637.06740919)(244.96973701,636.75856382)
\curveto(244.77995509,636.44970824)(244.49343115,636.20039521)(244.11016436,636.01062397)
\curveto(244.60506386,635.86549711)(244.98554531,635.61804462)(245.25160987,635.26826577)
\curveto(245.51766119,634.91847969)(245.65069016,634.5054387)(245.65069717,634.02914155)
\curveto(245.65069016,633.6458669)(245.56975645,633.28957253)(245.4078958,632.96025736)
\curveto(245.24602161,632.63093959)(245.04601303,632.3769752)(244.80786944,632.19836342)
\curveto(244.5697135,632.01975055)(244.27109602,631.88486104)(243.91201611,631.79369447)
\curveto(243.55292565,631.70252763)(243.11290675,631.65694428)(242.5919581,631.65694428)
\closepath
\moveto(240.55465927,636.40133882)
\lineto(242.32403935,636.40133882)
\curveto(242.80405628,636.40133407)(243.14825711,636.43296334)(243.35664287,636.49622671)
\curveto(243.63199881,636.57808585)(243.83944958,636.71390564)(243.9789958,636.90368648)
\curveto(244.11853133,637.09345682)(244.18830177,637.33160658)(244.18830733,637.61813648)
\curveto(244.18830177,637.88977009)(244.1231827,638.12885012)(243.9929499,638.3353773)
\curveto(243.86270639,638.54189112)(243.67665189,638.68329254)(243.43478584,638.75958199)
\curveto(243.19291019,638.83585723)(242.77800865,638.8739984)(242.19007998,638.87400562)
\lineto(240.55465927,638.87400562)
\closepath
\moveto(240.55465927,632.62256811)
\lineto(242.5919581,632.62256811)
\curveto(242.94173661,632.62256714)(243.18732855,632.63559096)(243.32873467,632.66163959)
\curveto(243.57804301,632.70629167)(243.78642405,632.78071347)(243.95387842,632.88490522)
\curveto(244.12132215,632.98909451)(244.25900248,633.14072893)(244.36691983,633.33980893)
\curveto(244.4748257,633.53888556)(244.52878151,633.76866287)(244.5287874,634.02914155)
\curveto(244.52878151,634.33426856)(244.45063862,634.59939622)(244.2943585,634.82452534)
\curveto(244.13806706,635.04964812)(243.92131356,635.20779444)(243.64409736,635.29896479)
\curveto(243.36687115,635.39012786)(242.96778424,635.43571121)(242.44683545,635.43571499)
\lineto(240.55465927,635.43571499)
\closepath
}
}
{
\newrgbcolor{curcolor}{0 0 0}
\pscustom[linestyle=none,fillstyle=solid,fillcolor=curcolor]
{
\newpath
\moveto(246.7837708,634.28589702)
\lineto(247.80521103,634.37520327)
\curveto(247.85358367,633.96588064)(247.96614664,633.63005227)(248.14290029,633.36771713)
\curveto(248.3196502,633.10537857)(248.59408059,632.89327644)(248.96619229,632.7314101)
\curveto(249.33829859,632.56954161)(249.75692122,632.4886079)(250.22206143,632.48860873)
\curveto(250.63509847,632.4886079)(250.9997653,632.55000588)(251.31606299,632.67280287)
\curveto(251.6323506,632.79559783)(251.86770955,632.96397715)(252.02214054,633.17794135)
\curveto(252.17656002,633.39190251)(252.25377264,633.62540091)(252.25377862,633.87843725)
\curveto(252.25377264,634.13519024)(252.17935084,634.35938592)(252.030513,634.55102495)
\curveto(251.88166363,634.74265819)(251.63607169,634.90359533)(251.29373643,635.03383686)
\curveto(251.0741871,635.11941856)(250.58858485,635.25244752)(249.83692823,635.43292417)
\curveto(249.08526447,635.61339326)(248.55873023,635.78363313)(248.25732392,635.94364429)
\curveto(247.86660748,636.14829995)(247.57543219,636.40226435)(247.38379716,636.70553823)
\curveto(247.19215991,637.00880202)(247.09634185,637.34835149)(247.09634267,637.72418765)
\curveto(247.09634185,638.13722258)(247.21355618,638.52328567)(247.44798603,638.88237808)
\curveto(247.68241353,639.24145604)(248.02475381,639.51402589)(248.47500791,639.70008844)
\curveto(248.9252576,639.88613489)(249.42574421,639.97916214)(249.97646924,639.97917047)
\curveto(250.58300321,639.97916214)(251.1179099,639.88148353)(251.58119092,639.68613433)
\curveto(252.04446132,639.49076908)(252.40075569,639.20331487)(252.65007511,638.82377085)
\curveto(252.89938176,638.4442125)(253.033341,638.0144266)(253.05195323,637.53441187)
\lineto(252.01376807,637.4562689)
\curveto(251.95794598,637.97349461)(251.76910066,638.36420907)(251.44723155,638.62841343)
\curveto(251.12535208,638.89260385)(250.64998283,639.02470255)(250.02112237,639.02470992)
\curveto(249.36620677,639.02470255)(248.88897697,638.9046974)(248.58943154,638.6646941)
\curveto(248.28988148,638.42467678)(248.1401076,638.13536203)(248.14010947,637.79674898)
\curveto(248.1401076,637.50277672)(248.24615867,637.26090587)(248.45826299,637.07113569)
\curveto(248.66664184,636.88135469)(249.21085126,636.68692773)(250.09089287,636.48785425)
\curveto(250.97092685,636.2887711)(251.57467371,636.11481014)(251.90213526,635.96597085)
\curveto(252.37842915,635.74642223)(252.73007216,635.46827075)(252.95706534,635.13151557)
\curveto(253.18404515,634.79475345)(253.29753839,634.40682981)(253.29754542,633.9677435)
\curveto(253.29753839,633.53237365)(253.17288188,633.12212348)(252.9235755,632.73699174)
\curveto(252.67425581,632.35185784)(252.3161009,632.05231009)(251.84910968,631.8383476)
\curveto(251.3821073,631.62438474)(250.85650333,631.5174034)(250.27229619,631.51740326)
\curveto(249.53179528,631.5174034)(248.91130351,631.62531501)(248.41081904,631.84113842)
\curveto(247.91033029,632.05696145)(247.51775529,632.38162656)(247.23309287,632.81513471)
\curveto(246.94842852,633.24864054)(246.79865464,633.73889415)(246.7837708,634.28589702)
\closepath
}
}
{
\newrgbcolor{curcolor}{0 0 0}
\pscustom[linestyle=none,fillstyle=solid,fillcolor=curcolor]
{
\newpath
\moveto(254.78784405,631.65694428)
\lineto(254.78784405,639.83962945)
\lineto(257.60657257,639.83962945)
\curveto(258.24287526,639.83962127)(258.72847751,639.80054982)(259.06338077,639.722415)
\curveto(259.53223296,639.61449532)(259.93225014,639.41913809)(260.26343351,639.13634273)
\curveto(260.6950736,638.77166843)(261.01787816,638.3056019)(261.23184816,637.73814175)
\curveto(261.44580351,637.17066944)(261.55278485,636.5222695)(261.5527925,635.79293999)
\curveto(261.55278485,635.17151382)(261.4802236,634.62079249)(261.33510851,634.14077436)
\curveto(261.18997857,633.66075126)(261.00392407,633.2635249)(260.77694445,632.94909408)
\curveto(260.54995109,632.63466068)(260.30156833,632.3872082)(260.03179542,632.20673588)
\curveto(259.76201027,632.02626246)(259.43641489,631.8895124)(259.05500831,631.79648529)
\curveto(258.67359144,631.7034579)(258.23543308,631.65694428)(257.74053194,631.65694428)
\closepath
\moveto(255.87068233,632.62256811)
\lineto(257.61773585,632.62256811)
\curveto(258.15729019,632.62256714)(258.58056418,632.67280186)(258.88755909,632.7732724)
\curveto(259.19454404,632.87374072)(259.43920571,633.01514214)(259.62154484,633.19747709)
\curveto(259.87829433,633.45423076)(260.07830292,633.79936186)(260.22157121,634.23287143)
\curveto(260.36482686,634.66637584)(260.43645784,635.19197981)(260.43646437,635.80968491)
\curveto(260.43645784,636.66553147)(260.29598669,637.32323413)(260.0150505,637.78279488)
\curveto(259.7341021,638.24234337)(259.39269209,638.55026357)(258.99081945,638.7065564)
\curveto(258.70056934,638.81818205)(258.23357254,638.8739984)(257.58982765,638.87400562)
\lineto(255.87068233,638.87400562)
\closepath
}
}
{
\newrgbcolor{curcolor}{0 0 0}
\pscustom[linestyle=none,fillstyle=solid,fillcolor=curcolor]
{
\newpath
\moveto(269.59593475,631.65694428)
\lineto(268.59123944,631.65694428)
\lineto(268.59123944,638.05908609)
\curveto(268.34936533,637.8283721)(268.0321424,637.59766452)(267.63956971,637.36696265)
\curveto(267.24699241,637.13624936)(266.89441913,636.96321867)(266.58184881,636.84787007)
\lineto(266.58184881,637.81907554)
\curveto(267.14373216,638.08326677)(267.63491604,638.40328051)(268.05540194,638.77911773)
\curveto(268.47588239,639.1549407)(268.77356959,639.51960753)(268.94846444,639.87311929)
\lineto(269.59593475,639.87311929)
\closepath
}
}
{
\newrgbcolor{curcolor}{0 0 0}
\pscustom[linestyle=none,fillstyle=solid,fillcolor=curcolor]
{
\newpath
\moveto(272.73839954,631.65694428)
\lineto(272.73839954,632.80118061)
\lineto(273.88263588,632.80118061)
\lineto(273.88263588,631.65694428)
\closepath
}
}
{
\newrgbcolor{curcolor}{0 0 0}
\pscustom[linestyle=none,fillstyle=solid,fillcolor=curcolor]
{
\newpath
\moveto(275.34502585,635.69247046)
\curveto(275.34502538,636.65994983)(275.44456453,637.43858792)(275.64364362,638.02838706)
\curveto(275.84272117,638.61817346)(276.13854783,639.07307672)(276.53112449,639.3930982)
\curveto(276.92369782,639.71310421)(277.41767253,639.87311108)(278.01305007,639.87311929)
\curveto(278.45213556,639.87311108)(278.83726838,639.78473519)(279.16844969,639.60799136)
\curveto(279.4996224,639.43123164)(279.77312252,639.17633697)(279.98895086,638.8433066)
\curveto(280.20476896,638.51026185)(280.37407856,638.10466304)(280.49688016,637.62650894)
\curveto(280.6196705,637.1483429)(280.68106849,636.50366405)(280.6810743,635.69247046)
\curveto(280.68106849,634.73242519)(280.5824596,633.95750819)(280.38524735,633.36771713)
\curveto(280.18802406,632.77792265)(279.89312767,632.32208912)(279.5005573,632.00021517)
\curveto(279.10797768,631.67834054)(278.61214243,631.5174034)(278.01305007,631.51740326)
\curveto(277.22417584,631.5174034)(276.60461435,631.80020624)(276.15436374,632.36581264)
\curveto(275.6148044,633.0467714)(275.34502538,634.15565624)(275.34502585,635.69247046)
\closepath
\moveto(276.37762937,635.69247046)
\curveto(276.37762786,634.34915292)(276.53484391,633.45516104)(276.849278,633.01049213)
\curveto(277.16370813,632.56582052)(277.55163177,632.34348539)(278.01305007,632.34348607)
\curveto(278.4744621,632.34348539)(278.86238573,632.56675079)(279.17682215,633.01328295)
\curveto(279.49124995,633.4598124)(279.648466,634.35287401)(279.64847078,635.69247046)
\curveto(279.648466,637.03950101)(279.49124995,637.93442317)(279.17682215,638.3772396)
\curveto(278.86238573,638.8200426)(278.47074101,639.04144746)(278.00188679,639.04145484)
\curveto(277.5404685,639.04144746)(277.17208058,638.84609023)(276.89672195,638.45538257)
\curveto(276.55065855,637.95674971)(276.37762786,637.03577992)(276.37762937,635.69247046)
\closepath
}
}
{
\newrgbcolor{curcolor}{0 0 0}
\pscustom[linestyle=none,fillstyle=solid,fillcolor=curcolor]
{
\newpath
\moveto(464.62073594,587.85689545)
\lineto(464.62073594,596.03958062)
\lineto(465.73148243,596.03958062)
\lineto(470.02934572,589.61511225)
\lineto(470.02934572,596.03958062)
\lineto(471.06753088,596.03958062)
\lineto(471.06753088,587.85689545)
\lineto(469.95678439,587.85689545)
\lineto(465.6589211,594.28694546)
\lineto(465.6589211,587.85689545)
\closepath
}
}
{
\newrgbcolor{curcolor}{0 0 0}
\pscustom[linestyle=none,fillstyle=solid,fillcolor=curcolor]
{
\newpath
\moveto(476.82220225,589.76581655)
\lineto(477.86038741,589.63743881)
\curveto(477.6966536,589.03089935)(477.39338476,588.56018146)(476.95057998,588.22528373)
\curveto(476.50776533,587.89038526)(475.94215964,587.72293621)(475.25376123,587.72293607)
\curveto(474.38674401,587.72293621)(473.69927262,587.98992442)(473.19134501,588.5239015)
\curveto(472.68341504,589.05787726)(472.42945065,589.80674663)(472.42945107,590.77051186)
\curveto(472.42945065,591.76776108)(472.68620586,592.5417478)(473.19971747,593.09247437)
\curveto(473.71322671,593.64319046)(474.37930183,593.91855112)(475.19794482,593.91855718)
\curveto(475.99053381,593.91855112)(476.63800348,593.64877209)(477.14035576,593.10921929)
\curveto(477.64269779,592.56965598)(477.89387137,591.81055361)(477.89387725,590.83190991)
\curveto(477.89387137,590.77236949)(477.89201082,590.68306333)(477.88829561,590.56399116)
\lineto(473.46763622,590.56399116)
\curveto(473.50484567,589.91279769)(473.68903963,589.41417163)(474.02021865,589.06811147)
\curveto(474.35139365,588.72204888)(474.76443465,588.54901819)(475.25934287,588.54901889)
\curveto(475.62772753,588.54901819)(475.94215964,588.64576654)(476.20264014,588.8392642)
\curveto(476.46311225,589.0327599)(476.66963275,589.34161037)(476.82220225,589.76581655)
\closepath
\moveto(473.52345263,591.39007397)
\lineto(476.83336553,591.39007397)
\curveto(476.78870763,591.8886965)(476.66219057,592.26266605)(476.45381397,592.51198374)
\curveto(476.13379578,592.89897245)(475.71889424,593.09246913)(475.2091081,593.09247437)
\curveto(474.74768974,593.09246913)(474.3597661,592.93804389)(474.04533603,592.62919819)
\curveto(473.73090189,592.32034295)(473.55694093,591.90730195)(473.52345263,591.39007397)
\closepath
}
}
{
\newrgbcolor{curcolor}{0 0 0}
\pscustom[linestyle=none,fillstyle=solid,fillcolor=curcolor]
{
\newpath
\moveto(481.32100476,588.75553959)
\lineto(481.46612742,587.86805873)
\curveto(481.18332148,587.80852128)(480.93028736,587.77875256)(480.70702429,587.77875248)
\curveto(480.34235514,587.77875256)(480.05955229,587.83642945)(479.85861492,587.95178334)
\curveto(479.65767457,588.06713703)(479.51627315,588.21877145)(479.43441023,588.40668705)
\curveto(479.35254519,588.59460155)(479.3116132,588.98996736)(479.31161413,589.59278569)
\lineto(479.31161413,593.00316812)
\lineto(478.57483757,593.00316812)
\lineto(478.57483757,593.78459781)
\lineto(479.31161413,593.78459781)
\lineto(479.31161413,595.25256929)
\lineto(480.31072781,595.85538648)
\lineto(480.31072781,593.78459781)
\lineto(481.32100476,593.78459781)
\lineto(481.32100476,593.00316812)
\lineto(480.31072781,593.00316812)
\lineto(480.31072781,589.53696928)
\curveto(480.31072587,589.25044367)(480.32840105,589.06624971)(480.36375339,588.98438686)
\curveto(480.39910176,588.90252175)(480.45677865,588.83740267)(480.53678425,588.78902943)
\curveto(480.61678553,588.74065433)(480.73120905,588.71646725)(480.88005515,588.71646811)
\curveto(480.99168535,588.71646725)(481.1386684,588.72949106)(481.32100476,588.75553959)
\closepath
}
}
{
\newrgbcolor{curcolor}{0 0 0}
\pscustom[linestyle=none,fillstyle=solid,fillcolor=curcolor]
{
\newpath
\moveto(482.38151566,587.85689545)
\lineto(482.38151566,596.03958062)
\lineto(485.45141801,596.03958062)
\curveto(486.07655723,596.03957244)(486.57797411,595.95677819)(486.95567016,595.79119761)
\curveto(487.33335539,595.62560117)(487.62918205,595.37070651)(487.84315102,595.02651285)
\curveto(488.0571074,594.68230485)(488.16408874,594.32228939)(488.16409536,593.94646538)
\curveto(488.16408874,593.59667683)(488.06920094,593.26736036)(487.87943169,592.95851499)
\curveto(487.68964976,592.64965942)(487.40312583,592.40034638)(487.01985903,592.21057515)
\curveto(487.51475853,592.06544828)(487.89523998,591.81799579)(488.16130454,591.46821694)
\curveto(488.42735586,591.11843087)(488.56038483,590.70538987)(488.56039184,590.22909272)
\curveto(488.56038483,589.84581807)(488.47945112,589.4895237)(488.31759048,589.16020854)
\curveto(488.15571629,588.83089076)(487.9557077,588.57692637)(487.71756411,588.39831459)
\curveto(487.47940817,588.21970173)(487.1807907,588.08481221)(486.82171079,587.99364564)
\curveto(486.46262032,587.9024788)(486.02260142,587.85689545)(485.50165277,587.85689545)
\closepath
\moveto(483.46435394,592.60128999)
\lineto(485.23373402,592.60128999)
\curveto(485.71375095,592.60128525)(486.05795178,592.63291451)(486.26633754,592.69617788)
\curveto(486.54169348,592.77803702)(486.74914425,592.91385681)(486.88869047,593.10363765)
\curveto(487.02822601,593.29340799)(487.09799644,593.53155775)(487.098002,593.81808765)
\curveto(487.09799644,594.08972126)(487.03287737,594.3288013)(486.90264457,594.53532847)
\curveto(486.77240106,594.74184229)(486.58634656,594.88324371)(486.34448051,594.95953316)
\curveto(486.10260486,595.0358084)(485.68770332,595.07394958)(485.09977465,595.07395679)
\lineto(483.46435394,595.07395679)
\closepath
\moveto(483.46435394,588.82251928)
\lineto(485.50165277,588.82251928)
\curveto(485.85143128,588.82251831)(486.09702322,588.83554213)(486.23842934,588.86159076)
\curveto(486.48773768,588.90624284)(486.69611872,588.98066464)(486.86357309,589.08485639)
\curveto(487.03101682,589.18904568)(487.16869715,589.3406801)(487.2766145,589.5397601)
\curveto(487.38452038,589.73883673)(487.43847618,589.96861404)(487.43848208,590.22909272)
\curveto(487.43847618,590.53421973)(487.36033329,590.79934739)(487.20405317,591.02447651)
\curveto(487.04776173,591.24959929)(486.83100823,591.40774562)(486.55379203,591.49891596)
\curveto(486.27656582,591.59007903)(485.87747891,591.63566238)(485.35653012,591.63566616)
\lineto(483.46435394,591.63566616)
\closepath
}
}
{
\newrgbcolor{curcolor}{0 0 0}
\pscustom[linestyle=none,fillstyle=solid,fillcolor=curcolor]
{
\newpath
\moveto(489.69346547,590.48584819)
\lineto(490.7149057,590.57515444)
\curveto(490.76327834,590.16583182)(490.87584131,589.83000344)(491.05259496,589.5676683)
\curveto(491.22934487,589.30532974)(491.50377526,589.09322761)(491.87588696,588.93136127)
\curveto(492.24799327,588.76949278)(492.66661589,588.68855907)(493.1317561,588.6885599)
\curveto(493.54479314,588.68855907)(493.90945997,588.74995706)(494.22575767,588.87275404)
\curveto(494.54204527,588.995549)(494.77740422,589.16392832)(494.93183521,589.37789252)
\curveto(495.08625469,589.59185368)(495.16346731,589.82535208)(495.16347329,590.07838842)
\curveto(495.16346731,590.33514141)(495.08904551,590.55933709)(494.94020767,590.75097612)
\curveto(494.79135831,590.94260936)(494.54576636,591.10354651)(494.2034311,591.23378803)
\curveto(493.98388177,591.31936973)(493.49827952,591.4523987)(492.7466229,591.63287534)
\curveto(491.99495914,591.81334443)(491.4684249,591.9835843)(491.1670186,592.14359546)
\curveto(490.77630216,592.34825112)(490.48512686,592.60221552)(490.29349184,592.9054894)
\curveto(490.10185459,593.20875319)(490.00603652,593.54830266)(490.00603734,593.92413882)
\curveto(490.00603652,594.33717375)(490.12325085,594.72323684)(490.3576807,595.08232925)
\curveto(490.5921082,595.44140722)(490.93444848,595.71397706)(491.38470258,595.90003961)
\curveto(491.83495227,596.08608607)(492.33543888,596.17911332)(492.88616391,596.17912164)
\curveto(493.49269788,596.17911332)(494.02760458,596.0814347)(494.4908856,595.88608551)
\curveto(494.954156,595.69072025)(495.31045037,595.40326604)(495.55976978,595.02372203)
\curveto(495.80907643,594.64416368)(495.94303567,594.21437778)(495.9616479,593.73436304)
\lineto(494.92346275,593.65622007)
\curveto(494.86764065,594.17344579)(494.67879533,594.56416024)(494.35692622,594.8283646)
\curveto(494.03504676,595.09255503)(493.5596775,595.22465372)(492.93081704,595.22466109)
\curveto(492.27590144,595.22465372)(491.79867164,595.10464857)(491.49912621,594.86464527)
\curveto(491.19957615,594.62462795)(491.04980227,594.3353132)(491.04980414,593.99670015)
\curveto(491.04980227,593.7027279)(491.15585334,593.46085704)(491.36795766,593.27108687)
\curveto(491.57633651,593.08130586)(492.12054593,592.88687891)(493.00058755,592.68780542)
\curveto(493.88062152,592.48872227)(494.48436838,592.31476131)(494.81182993,592.16592202)
\curveto(495.28812383,591.9463734)(495.63976683,591.66822192)(495.86676001,591.33146674)
\curveto(496.09373982,590.99470462)(496.20723307,590.60678099)(496.20724009,590.16769467)
\curveto(496.20723307,589.73232483)(496.08257655,589.32207465)(495.83327017,588.93694291)
\curveto(495.58395048,588.55180901)(495.22579557,588.25226126)(494.75880435,588.03829877)
\curveto(494.29180197,587.82433591)(493.766198,587.71735457)(493.18199087,587.71735443)
\curveto(492.44148995,587.71735457)(491.82099818,587.82526618)(491.32051371,588.04108959)
\curveto(490.82002496,588.25691263)(490.42744996,588.58157773)(490.14278754,589.01508588)
\curveto(489.85812319,589.44859171)(489.70834931,589.93884532)(489.69346547,590.48584819)
\closepath
}
}
{
\newrgbcolor{curcolor}{0 0 0}
\pscustom[linestyle=none,fillstyle=solid,fillcolor=curcolor]
{
\newpath
\moveto(497.69753872,587.85689545)
\lineto(497.69753872,596.03958062)
\lineto(500.51626724,596.03958062)
\curveto(501.15256993,596.03957244)(501.63817218,596.00050099)(501.97307545,595.92236617)
\curveto(502.44192763,595.81444649)(502.84194481,595.61908927)(503.17312818,595.3362939)
\curveto(503.60476827,594.9716196)(503.92757283,594.50555307)(504.14154283,593.93809292)
\curveto(504.35549819,593.37062061)(504.46247952,592.72222067)(504.46248717,591.99289116)
\curveto(504.46247952,591.37146499)(504.38991827,590.82074366)(504.24480319,590.34072553)
\curveto(504.09967324,589.86070243)(503.91361874,589.46347607)(503.68663912,589.14904525)
\curveto(503.45964576,588.83461185)(503.211263,588.58715937)(502.9414901,588.40668705)
\curveto(502.67170494,588.22621363)(502.34610956,588.08946357)(501.96470298,587.99643646)
\curveto(501.58328611,587.90340907)(501.14512775,587.85689545)(500.65022661,587.85689545)
\closepath
\moveto(498.780377,588.82251928)
\lineto(500.52743052,588.82251928)
\curveto(501.06698486,588.82251831)(501.49025886,588.87275303)(501.79725377,588.97322357)
\curveto(502.10423871,589.07369189)(502.34890038,589.21509331)(502.53123951,589.39742826)
\curveto(502.78798901,589.65418194)(502.9879976,589.99931304)(503.13126588,590.4328226)
\curveto(503.27452153,590.86632702)(503.34615251,591.39193098)(503.34615904,592.00963608)
\curveto(503.34615251,592.86548264)(503.20568136,593.5231853)(502.92474517,593.98274605)
\curveto(502.64379677,594.44229454)(502.30238676,594.75021474)(501.90051412,594.90650757)
\curveto(501.61026401,595.01813322)(501.14326721,595.07394958)(500.49952232,595.07395679)
\lineto(498.780377,595.07395679)
\closepath
}
}
{
\newrgbcolor{curcolor}{0 0 0}
\pscustom[linestyle=none,fillstyle=solid,fillcolor=curcolor]
{
\newpath
\moveto(512.50562942,587.85689545)
\lineto(511.50093411,587.85689545)
\lineto(511.50093411,594.25903726)
\curveto(511.25906,594.02832327)(510.94183708,593.79761569)(510.54926438,593.56691382)
\curveto(510.15668708,593.33620053)(509.8041138,593.16316984)(509.49154348,593.04782124)
\lineto(509.49154348,594.01902671)
\curveto(510.05342683,594.28321794)(510.54461071,594.60323169)(510.96509661,594.9790689)
\curveto(511.38557706,595.35489187)(511.68326427,595.7195587)(511.85815911,596.07307047)
\lineto(512.50562942,596.07307047)
\closepath
}
}
{
\newrgbcolor{curcolor}{0 0 0}
\pscustom[linestyle=none,fillstyle=solid,fillcolor=curcolor]
{
\newpath
\moveto(515.64809422,587.85689545)
\lineto(515.64809422,589.00113178)
\lineto(516.79233055,589.00113178)
\lineto(516.79233055,587.85689545)
\closepath
}
}
{
\newrgbcolor{curcolor}{0 0 0}
\pscustom[linestyle=none,fillstyle=solid,fillcolor=curcolor]
{
\newpath
\moveto(518.25472052,591.89242163)
\curveto(518.25472005,592.859901)(518.35425921,593.63853909)(518.5533383,594.22833824)
\curveto(518.75241584,594.81812464)(519.0482425,595.27302789)(519.44081916,595.59304937)
\curveto(519.8333925,595.91305538)(520.3273672,596.07306225)(520.92274475,596.07307047)
\curveto(521.36183023,596.07306225)(521.74696305,595.98468636)(522.07814436,595.80794254)
\curveto(522.40931707,595.63118281)(522.68281719,595.37628814)(522.89864553,595.04325777)
\curveto(523.11446364,594.71021302)(523.28377323,594.30461421)(523.40657483,593.82646011)
\curveto(523.52936518,593.34829407)(523.59076316,592.70361522)(523.59076897,591.89242163)
\curveto(523.59076316,590.93237636)(523.49215427,590.15745936)(523.29494202,589.5676683)
\curveto(523.09771873,588.97787382)(522.80282235,588.52204029)(522.41025198,588.20016635)
\curveto(522.01767235,587.87829171)(521.5218371,587.71735457)(520.92274475,587.71735443)
\curveto(520.13387052,587.71735457)(519.51430902,588.00015741)(519.06405841,588.56576381)
\curveto(518.52449907,589.24672258)(518.25472005,590.35560741)(518.25472052,591.89242163)
\closepath
\moveto(519.28732404,591.89242163)
\curveto(519.28732253,590.54910409)(519.44453859,589.65511221)(519.75897267,589.2104433)
\curveto(520.0734028,588.76577169)(520.46132644,588.54343656)(520.92274475,588.54343725)
\curveto(521.38415677,588.54343656)(521.7720804,588.76670196)(522.08651682,589.21323412)
\curveto(522.40094462,589.65976357)(522.55816068,590.55282518)(522.55816545,591.89242163)
\curveto(522.55816068,593.23945219)(522.40094462,594.13437434)(522.08651682,594.57719078)
\curveto(521.7720804,595.01999377)(521.38043568,595.24139863)(520.91158147,595.24140601)
\curveto(520.45016317,595.24139863)(520.08177526,595.0460414)(519.80641662,594.65533374)
\curveto(519.46035322,594.15670088)(519.28732253,593.2357311)(519.28732404,591.89242163)
\closepath
}
}
{
\newrgbcolor{curcolor}{0 0 0}
\pscustom[linestyle=none,fillstyle=solid,fillcolor=curcolor]
{
\newpath
\moveto(331.56403599,590.01699037)
\lineto(332.5687313,590.15094975)
\curveto(332.68408361,589.58162068)(332.88037111,589.1713705)(333.15759439,588.92019799)
\curveto(333.43481352,588.66902335)(333.77250244,588.54343656)(334.17066217,588.54343725)
\curveto(334.64323751,588.54343656)(335.04232442,588.70716452)(335.36792408,589.03462162)
\curveto(335.69351518,589.36207637)(335.85631287,589.76767518)(335.85631764,590.25141928)
\curveto(335.85631287,590.71283205)(335.70560872,591.09331351)(335.40420475,591.39286479)
\curveto(335.10279213,591.692409)(334.71951986,591.84218288)(334.25438678,591.84218686)
\curveto(334.06460801,591.84218288)(333.8283188,591.80497198)(333.54551841,591.73055405)
\lineto(333.65715123,592.61245327)
\curveto(333.72412827,592.60500634)(333.77808408,592.60128525)(333.81901881,592.60128999)
\curveto(334.24694142,592.60128525)(334.63207424,592.71291795)(334.97441842,592.93618843)
\curveto(335.31675481,593.15944875)(335.48792495,593.50364958)(335.48792936,593.96879195)
\curveto(335.48792495,594.33717375)(335.36326844,594.64230313)(335.11395943,594.88418101)
\curveto(334.86464237,595.12604484)(334.54276808,595.24698026)(334.1483356,595.24698765)
\curveto(333.75761808,595.24698026)(333.43202271,595.12418429)(333.17154849,594.87859937)
\curveto(332.9110701,594.63300041)(332.74362105,594.26461249)(332.66920083,593.77343452)
\lineto(331.66450552,593.95204702)
\curveto(331.78730091,594.62555823)(332.06638266,595.1474411)(332.50175161,595.51769722)
\curveto(332.93711773,595.88793802)(333.47853633,596.07306225)(334.12600904,596.07307047)
\curveto(334.5725368,596.07306225)(334.98371725,595.97724418)(335.35955162,595.78561597)
\curveto(335.73537744,595.59397191)(336.02283165,595.33256533)(336.2219151,595.00139546)
\curveto(336.42098828,594.67021131)(336.52052744,594.3185683)(336.52053287,593.94646538)
\curveto(336.52052744,593.59295574)(336.42563964,593.27108145)(336.2358692,592.98084155)
\curveto(336.04608846,592.69059141)(335.76514616,592.45988382)(335.39304147,592.28871811)
\curveto(335.87677886,592.17708098)(336.25260896,591.94544313)(336.52053287,591.59380385)
\curveto(336.78844592,591.24215711)(336.92240516,590.80213821)(336.922411,590.27374584)
\curveto(336.92240516,589.55929414)(336.66192886,588.95368674)(336.14098131,588.45692182)
\curveto(335.62002365,587.9601557)(334.96139071,587.71177294)(334.16508052,587.71177279)
\curveto(333.44690707,587.71177294)(332.85060239,587.92573561)(332.3761647,588.35366146)
\curveto(331.90172443,588.78158632)(331.63101513,589.33602874)(331.56403599,590.01699037)
\closepath
}
}
{
\newrgbcolor{curcolor}{0 0 0}
\pscustom[linestyle=none,fillstyle=solid,fillcolor=curcolor]
{
\newpath
\moveto(339.46763929,592.29429975)
\curveto(339.05087519,592.44686001)(338.74202471,592.66454378)(338.54108695,592.94735171)
\curveto(338.34014699,593.23014946)(338.23967756,593.56876866)(338.23967835,593.96321031)
\curveto(338.23967756,594.55857861)(338.45364024,595.05906522)(338.88156702,595.46467164)
\curveto(339.30949094,595.87026284)(339.87881772,596.07306225)(340.58954906,596.07307047)
\curveto(341.3039952,596.07306225)(341.87890362,595.86561148)(342.31427602,595.45071754)
\curveto(342.74963868,595.0358084)(342.96732245,594.53067043)(342.96732797,593.9353021)
\curveto(342.96732245,593.55574484)(342.86778329,593.2254981)(342.6687102,592.94456089)
\curveto(342.46962666,592.6636135)(342.16728809,592.44686001)(341.76169359,592.29429975)
\curveto(342.26403643,592.13056736)(342.64637844,591.86636996)(342.90872075,591.50170678)
\curveto(343.17105213,591.13703632)(343.30222056,590.70166878)(343.30222641,590.19560287)
\curveto(343.30222056,589.49603561)(343.05476807,588.90810338)(342.55986821,588.43180443)
\curveto(342.06495812,587.95550433)(341.41376736,587.71735457)(340.60629398,587.71735443)
\curveto(339.79881428,587.71735457)(339.14762353,587.95643461)(338.65271976,588.43459525)
\curveto(338.15781358,588.91275475)(337.91036109,589.50905942)(337.91036155,590.22351108)
\curveto(337.91036109,590.75562459)(338.0452506,591.20122512)(338.3150305,591.56031401)
\curveto(338.58480866,591.9193955)(338.96901121,592.16405717)(339.46763929,592.29429975)
\closepath
\moveto(339.26670023,593.99670015)
\curveto(339.26669841,593.60970065)(339.39135493,593.29340799)(339.64067015,593.04782124)
\curveto(339.88998099,592.80222411)(340.21371582,592.67942814)(340.61187562,592.67943296)
\curveto(340.99886582,592.67942814)(341.31608875,592.80129383)(341.56354535,593.04503042)
\curveto(341.81099372,593.28875663)(341.93471997,593.58737411)(341.93472445,593.94088374)
\curveto(341.93471997,594.30926557)(341.80727263,594.61904632)(341.55238207,594.87022691)
\curveto(341.2974833,595.12139347)(340.98026037,595.24698026)(340.60071234,595.24698765)
\curveto(340.21743691,595.24698026)(339.89928372,595.12418429)(339.64625179,594.87859937)
\curveto(339.39321547,594.63300041)(339.26669841,594.33903429)(339.26670023,593.99670015)
\closepath
\moveto(338.94296507,590.21792944)
\curveto(338.94296358,589.93140314)(339.01087347,589.65418194)(339.14669495,589.38626498)
\curveto(339.28251304,589.11834497)(339.48438218,588.9108942)(339.75230296,588.76391205)
\curveto(340.02021914,588.61692809)(340.30860362,588.54343656)(340.61745726,588.54343725)
\curveto(341.09747471,588.54343656)(341.4937708,588.6978618)(341.80634672,589.00671342)
\curveto(342.11891392,589.31556274)(342.2751997,589.70813774)(342.27520453,590.18443959)
\curveto(342.2751997,590.66817897)(342.11426256,591.06819615)(341.79239262,591.38449233)
\curveto(341.47051398,591.70078146)(341.06770599,591.85892778)(340.58396742,591.85893178)
\curveto(340.11138585,591.85892778)(339.71974112,591.702642)(339.40903206,591.39007397)
\curveto(339.09831908,591.07749887)(338.94296358,590.68678442)(338.94296507,590.21792944)
\closepath
}
}
{
\newrgbcolor{curcolor}{0 0 0}
\pscustom[linestyle=none,fillstyle=solid,fillcolor=curcolor]
{
\newpath
\moveto(349.49784768,594.03577163)
\lineto(348.498734,593.95762866)
\curveto(348.40942316,594.35205811)(348.28290609,594.63858204)(348.11918244,594.81720132)
\curveto(347.84753856,595.1037183)(347.51264046,595.24698026)(347.11448713,595.24698765)
\curveto(346.79447008,595.24698026)(346.51352778,595.1576741)(346.27165939,594.9790689)
\curveto(345.95536428,594.7483542)(345.70605124,594.41159555)(345.52371954,593.96879195)
\curveto(345.34138442,593.52597612)(345.24649662,592.89525136)(345.23905587,592.07661577)
\curveto(345.4809253,592.44499946)(345.77675195,592.71849958)(346.12653673,592.89711694)
\curveto(346.47631688,593.07572423)(346.84284425,593.16503039)(347.22611994,593.16503569)
\curveto(347.89591273,593.16503039)(348.46616978,592.91850817)(348.93689279,592.42546831)
\curveto(349.40760556,591.93241931)(349.6429645,591.29518264)(349.64297034,590.51375639)
\curveto(349.6429645,590.00024331)(349.53226207,589.52301351)(349.31086272,589.08206557)
\curveto(349.08945236,588.64111517)(348.78525325,588.30342625)(348.39826447,588.06899779)
\curveto(348.01126652,587.83456891)(347.5721779,587.71735457)(347.08099728,587.71735443)
\curveto(346.24374875,587.71735457)(345.56092873,588.02527477)(345.03253517,588.64111596)
\curveto(344.50413916,589.25695557)(344.23994177,590.27188288)(344.2399422,591.68590093)
\curveto(344.23994177,593.26736036)(344.53204734,594.41717718)(345.11625978,595.13535484)
\curveto(345.62604781,595.76049069)(346.31258892,596.07306225)(347.17588517,596.07307047)
\curveto(347.81963038,596.07306225)(348.3470949,595.89258938)(348.75828029,595.53165132)
\curveto(349.1694558,595.17069792)(349.41597801,594.67207185)(349.49784768,594.03577163)
\closepath
\moveto(345.39534181,590.50817475)
\curveto(345.39534022,590.16211073)(345.46883175,589.83093371)(345.61581662,589.51464272)
\curveto(345.76279787,589.19834841)(345.96838809,588.95740783)(346.23258791,588.79182025)
\curveto(346.49678288,588.62623081)(346.77400408,588.54343656)(347.06425236,588.54343725)
\curveto(347.48845337,588.54343656)(347.85312019,588.7146067)(348.15825393,589.05694818)
\curveto(348.46337896,589.39928727)(348.61594365,589.86442352)(348.61594846,590.45235834)
\curveto(348.61594365,591.01796143)(348.46523951,591.46356197)(348.16383557,591.78916128)
\curveto(347.86242292,592.11475272)(347.48287174,592.27755041)(347.02518088,592.27755483)
\curveto(346.57120468,592.27755041)(346.18607186,592.11475272)(345.86978126,591.78916128)
\curveto(345.55348655,591.46356197)(345.39534022,591.03656688)(345.39534181,590.50817475)
\closepath
}
}
{
\newrgbcolor{curcolor}{0 0 0}
\pscustom[linestyle=none,fillstyle=solid,fillcolor=curcolor]
{
\newpath
\moveto(351.0104715,587.85689545)
\lineto(351.0104715,596.03958062)
\lineto(354.08037385,596.03958062)
\curveto(354.70551307,596.03957244)(355.20692995,595.95677819)(355.584626,595.79119761)
\curveto(355.96231123,595.62560117)(356.25813789,595.37070651)(356.47210686,595.02651285)
\curveto(356.68606324,594.68230485)(356.79304458,594.32228939)(356.7930512,593.94646538)
\curveto(356.79304458,593.59667683)(356.69815678,593.26736036)(356.50838753,592.95851499)
\curveto(356.3186056,592.64965942)(356.03208167,592.40034638)(355.64881487,592.21057515)
\curveto(356.14371437,592.06544828)(356.52419582,591.81799579)(356.79026038,591.46821694)
\curveto(357.0563117,591.11843087)(357.18934067,590.70538987)(357.18934769,590.22909272)
\curveto(357.18934067,589.84581807)(357.10840696,589.4895237)(356.94654632,589.16020854)
\curveto(356.78467213,588.83089076)(356.58466354,588.57692637)(356.34651995,588.39831459)
\curveto(356.10836401,588.21970173)(355.80974654,588.08481221)(355.45066663,587.99364564)
\curveto(355.09157616,587.9024788)(354.65155726,587.85689545)(354.13060862,587.85689545)
\closepath
\moveto(352.09330978,592.60128999)
\lineto(353.86268986,592.60128999)
\curveto(354.34270679,592.60128525)(354.68690762,592.63291451)(354.89529338,592.69617788)
\curveto(355.17064932,592.77803702)(355.37810009,592.91385681)(355.51764631,593.10363765)
\curveto(355.65718185,593.29340799)(355.72695228,593.53155775)(355.72695784,593.81808765)
\curveto(355.72695228,594.08972126)(355.66183321,594.3288013)(355.53160042,594.53532847)
\curveto(355.40135691,594.74184229)(355.2153024,594.88324371)(354.97343635,594.95953316)
\curveto(354.7315607,595.0358084)(354.31665916,595.07394958)(353.72873049,595.07395679)
\lineto(352.09330978,595.07395679)
\closepath
\moveto(352.09330978,588.82251928)
\lineto(354.13060862,588.82251928)
\curveto(354.48038712,588.82251831)(354.72597906,588.83554213)(354.86738518,588.86159076)
\curveto(355.11669352,588.90624284)(355.32507456,588.98066464)(355.49252893,589.08485639)
\curveto(355.65997266,589.18904568)(355.797653,589.3406801)(355.90557034,589.5397601)
\curveto(356.01347622,589.73883673)(356.06743202,589.96861404)(356.06743792,590.22909272)
\curveto(356.06743202,590.53421973)(355.98928913,590.79934739)(355.83300901,591.02447651)
\curveto(355.67671757,591.24959929)(355.45996407,591.40774562)(355.18274788,591.49891596)
\curveto(354.90552166,591.59007903)(354.50643475,591.63566238)(353.98548596,591.63566616)
\lineto(352.09330978,591.63566616)
\closepath
}
}
{
\newrgbcolor{curcolor}{0 0 0}
\pscustom[linestyle=none,fillstyle=solid,fillcolor=curcolor]
{
\newpath
\moveto(358.32242131,590.48584819)
\lineto(359.34386155,590.57515444)
\curveto(359.39223418,590.16583182)(359.50479715,589.83000344)(359.6815508,589.5676683)
\curveto(359.85830071,589.30532974)(360.1327311,589.09322761)(360.5048428,588.93136127)
\curveto(360.87694911,588.76949278)(361.29557174,588.68855907)(361.76071194,588.6885599)
\curveto(362.17374898,588.68855907)(362.53841581,588.74995706)(362.85471351,588.87275404)
\curveto(363.17100111,588.995549)(363.40636006,589.16392832)(363.56079105,589.37789252)
\curveto(363.71521053,589.59185368)(363.79242315,589.82535208)(363.79242913,590.07838842)
\curveto(363.79242315,590.33514141)(363.71800135,590.55933709)(363.56916351,590.75097612)
\curveto(363.42031415,590.94260936)(363.1747222,591.10354651)(362.83238694,591.23378803)
\curveto(362.61283761,591.31936973)(362.12723536,591.4523987)(361.37557874,591.63287534)
\curveto(360.62391498,591.81334443)(360.09738074,591.9835843)(359.79597444,592.14359546)
\curveto(359.405258,592.34825112)(359.1140827,592.60221552)(358.92244768,592.9054894)
\curveto(358.73081043,593.20875319)(358.63499236,593.54830266)(358.63499318,593.92413882)
\curveto(358.63499236,594.33717375)(358.75220669,594.72323684)(358.98663654,595.08232925)
\curveto(359.22106404,595.44140722)(359.56340432,595.71397706)(360.01365842,595.90003961)
\curveto(360.46390811,596.08608607)(360.96439472,596.17911332)(361.51511975,596.17912164)
\curveto(362.12165372,596.17911332)(362.65656042,596.0814347)(363.11984144,595.88608551)
\curveto(363.58311184,595.69072025)(363.93940621,595.40326604)(364.18872562,595.02372203)
\curveto(364.43803227,594.64416368)(364.57199151,594.21437778)(364.59060375,593.73436304)
\lineto(363.55241859,593.65622007)
\curveto(363.49659649,594.17344579)(363.30775117,594.56416024)(362.98588206,594.8283646)
\curveto(362.6640026,595.09255503)(362.18863334,595.22465372)(361.55977288,595.22466109)
\curveto(360.90485728,595.22465372)(360.42762748,595.10464857)(360.12808206,594.86464527)
\curveto(359.82853199,594.62462795)(359.67875811,594.3353132)(359.67875998,593.99670015)
\curveto(359.67875811,593.7027279)(359.78480918,593.46085704)(359.9969135,593.27108687)
\curveto(360.20529235,593.08130586)(360.74950177,592.88687891)(361.62954339,592.68780542)
\curveto(362.50957736,592.48872227)(363.11332422,592.31476131)(363.44078577,592.16592202)
\curveto(363.91707967,591.9463734)(364.26872268,591.66822192)(364.49571585,591.33146674)
\curveto(364.72269566,590.99470462)(364.83618891,590.60678099)(364.83619593,590.16769467)
\curveto(364.83618891,589.73232483)(364.71153239,589.32207465)(364.46222601,588.93694291)
\curveto(364.21290633,588.55180901)(363.85475141,588.25226126)(363.38776019,588.03829877)
\curveto(362.92075781,587.82433591)(362.39515384,587.71735457)(361.81094671,587.71735443)
\curveto(361.07044579,587.71735457)(360.44995402,587.82526618)(359.94946955,588.04108959)
\curveto(359.4489808,588.25691263)(359.05640581,588.58157773)(358.77174338,589.01508588)
\curveto(358.48707903,589.44859171)(358.33730516,589.93884532)(358.32242131,590.48584819)
\closepath
}
}
{
\newrgbcolor{curcolor}{0 0 0}
\pscustom[linestyle=none,fillstyle=solid,fillcolor=curcolor]
{
\newpath
\moveto(366.32649456,587.85689545)
\lineto(366.32649456,596.03958062)
\lineto(369.14522308,596.03958062)
\curveto(369.78152578,596.03957244)(370.26712803,596.00050099)(370.60203129,595.92236617)
\curveto(371.07088347,595.81444649)(371.47090065,595.61908927)(371.80208402,595.3362939)
\curveto(372.23372411,594.9716196)(372.55652867,594.50555307)(372.77049867,593.93809292)
\curveto(372.98445403,593.37062061)(373.09143536,592.72222067)(373.09144301,591.99289116)
\curveto(373.09143536,591.37146499)(373.01887411,590.82074366)(372.87375903,590.34072553)
\curveto(372.72862909,589.86070243)(372.54257458,589.46347607)(372.31559496,589.14904525)
\curveto(372.0886016,588.83461185)(371.84021884,588.58715937)(371.57044594,588.40668705)
\curveto(371.30066078,588.22621363)(370.97506541,588.08946357)(370.59365883,587.99643646)
\curveto(370.21224195,587.90340907)(369.7740836,587.85689545)(369.27918246,587.85689545)
\closepath
\moveto(367.40933284,588.82251928)
\lineto(369.15638636,588.82251928)
\curveto(369.69594071,588.82251831)(370.1192147,588.87275303)(370.42620961,588.97322357)
\curveto(370.73319455,589.07369189)(370.97785622,589.21509331)(371.16019535,589.39742826)
\curveto(371.41694485,589.65418194)(371.61695344,589.99931304)(371.76022172,590.4328226)
\curveto(371.90347737,590.86632702)(371.97510835,591.39193098)(371.97511488,592.00963608)
\curveto(371.97510835,592.86548264)(371.8346372,593.5231853)(371.55370102,593.98274605)
\curveto(371.27275261,594.44229454)(370.9313426,594.75021474)(370.52946996,594.90650757)
\curveto(370.23921985,595.01813322)(369.77222305,595.07394958)(369.12847816,595.07395679)
\lineto(367.40933284,595.07395679)
\closepath
}
}
{
\newrgbcolor{curcolor}{0 0 0}
\pscustom[linestyle=none,fillstyle=solid,fillcolor=curcolor]
{
\newpath
\moveto(381.13458526,587.85689545)
\lineto(380.12988995,587.85689545)
\lineto(380.12988995,594.25903726)
\curveto(379.88801584,594.02832327)(379.57079292,593.79761569)(379.17822022,593.56691382)
\curveto(378.78564292,593.33620053)(378.43306964,593.16316984)(378.12049932,593.04782124)
\lineto(378.12049932,594.01902671)
\curveto(378.68238267,594.28321794)(379.17356656,594.60323169)(379.59405245,594.9790689)
\curveto(380.0145329,595.35489187)(380.31222011,595.7195587)(380.48711495,596.07307047)
\lineto(381.13458526,596.07307047)
\closepath
}
}
{
\newrgbcolor{curcolor}{0 0 0}
\pscustom[linestyle=none,fillstyle=solid,fillcolor=curcolor]
{
\newpath
\moveto(384.27705006,587.85689545)
\lineto(384.27705006,589.00113178)
\lineto(385.42128639,589.00113178)
\lineto(385.42128639,587.85689545)
\closepath
}
}
{
\newrgbcolor{curcolor}{0 0 0}
\pscustom[linestyle=none,fillstyle=solid,fillcolor=curcolor]
{
\newpath
\moveto(386.88367636,591.89242163)
\curveto(386.88367589,592.859901)(386.98321505,593.63853909)(387.18229414,594.22833824)
\curveto(387.38137168,594.81812464)(387.67719834,595.27302789)(388.069775,595.59304937)
\curveto(388.46234834,595.91305538)(388.95632304,596.07306225)(389.55170059,596.07307047)
\curveto(389.99078607,596.07306225)(390.37591889,595.98468636)(390.7071002,595.80794254)
\curveto(391.03827292,595.63118281)(391.31177303,595.37628814)(391.52760137,595.04325777)
\curveto(391.74341948,594.71021302)(391.91272907,594.30461421)(392.03553067,593.82646011)
\curveto(392.15832102,593.34829407)(392.219719,592.70361522)(392.21972481,591.89242163)
\curveto(392.219719,590.93237636)(392.12111012,590.15745936)(391.92389786,589.5676683)
\curveto(391.72667457,588.97787382)(391.43177819,588.52204029)(391.03920782,588.20016635)
\curveto(390.64662819,587.87829171)(390.15079294,587.71735457)(389.55170059,587.71735443)
\curveto(388.76282636,587.71735457)(388.14326487,588.00015741)(387.69301425,588.56576381)
\curveto(387.15345492,589.24672258)(386.88367589,590.35560741)(386.88367636,591.89242163)
\closepath
\moveto(387.91627988,591.89242163)
\curveto(387.91627837,590.54910409)(388.07349443,589.65511221)(388.38792851,589.2104433)
\curveto(388.70235864,588.76577169)(389.09028228,588.54343656)(389.55170059,588.54343725)
\curveto(390.01311261,588.54343656)(390.40103625,588.76670196)(390.71547266,589.21323412)
\curveto(391.02990046,589.65976357)(391.18711652,590.55282518)(391.18712129,591.89242163)
\curveto(391.18711652,593.23945219)(391.02990046,594.13437434)(390.71547266,594.57719078)
\curveto(390.40103625,595.01999377)(390.00939152,595.24139863)(389.54053731,595.24140601)
\curveto(389.07911901,595.24139863)(388.7107311,595.0460414)(388.43537246,594.65533374)
\curveto(388.08930906,594.15670088)(387.91627837,593.2357311)(387.91627988,591.89242163)
\closepath
}
}
{
\newrgbcolor{curcolor}{0 0 0}
\pscustom[linestyle=none,fillstyle=solid,fillcolor=curcolor]
{
\newpath
\moveto(589.40506808,587.75691986)
\lineto(589.40506808,589.71607573)
\lineto(585.85514463,589.71607573)
\lineto(585.85514463,590.63704643)
\lineto(589.58926222,595.93960504)
\lineto(590.40976339,595.93960504)
\lineto(590.40976339,590.63704643)
\lineto(591.51492824,590.63704643)
\lineto(591.51492824,589.71607573)
\lineto(590.40976339,589.71607573)
\lineto(590.40976339,587.75691986)
\closepath
\moveto(589.40506808,590.63704643)
\lineto(589.40506808,594.32651089)
\lineto(586.84309502,590.63704643)
\closepath
}
}
{
\newrgbcolor{curcolor}{0 0 0}
\pscustom[linestyle=none,fillstyle=solid,fillcolor=curcolor]
{
\newpath
\moveto(593.11127762,587.75691986)
\lineto(593.11127762,588.90115619)
\lineto(594.25551395,588.90115619)
\lineto(594.25551395,587.75691986)
\closepath
}
}
{
\newrgbcolor{curcolor}{0 0 0}
\pscustom[linestyle=none,fillstyle=solid,fillcolor=curcolor]
{
\newpath
\moveto(598.93851057,587.75691986)
\lineto(598.93851057,589.71607573)
\lineto(595.38858713,589.71607573)
\lineto(595.38858713,590.63704643)
\lineto(599.12270471,595.93960504)
\lineto(599.94320589,595.93960504)
\lineto(599.94320589,590.63704643)
\lineto(601.04837073,590.63704643)
\lineto(601.04837073,589.71607573)
\lineto(599.94320589,589.71607573)
\lineto(599.94320589,587.75691986)
\closepath
\moveto(598.93851057,590.63704643)
\lineto(598.93851057,594.32651089)
\lineto(596.37653752,590.63704643)
\closepath
}
}
{
\newrgbcolor{curcolor}{0 0 0}
\pscustom[linestyle=none,fillstyle=solid,fillcolor=curcolor]
{
\newpath
\moveto(602.44378105,587.75691986)
\lineto(602.44378105,595.93960504)
\lineto(605.5136834,595.93960504)
\curveto(606.13882262,595.93959685)(606.64023951,595.8568026)(607.01793556,595.69122203)
\curveto(607.39562078,595.52562559)(607.69144744,595.27073092)(607.90541642,594.92653726)
\curveto(608.1193728,594.58232926)(608.22635414,594.2223138)(608.22636076,593.8464898)
\curveto(608.22635414,593.49670124)(608.13146634,593.16738478)(607.94169708,592.8585394)
\curveto(607.75191516,592.54968383)(607.46539122,592.3003708)(607.08212442,592.11059956)
\curveto(607.57702392,591.96547269)(607.95750538,591.71802021)(608.22356993,591.36824135)
\curveto(608.48962126,591.01845528)(608.62265022,590.60541429)(608.62265724,590.12911713)
\curveto(608.62265022,589.74584249)(608.54171652,589.38954812)(608.37985587,589.06023295)
\curveto(608.21798168,588.73091518)(608.01797309,588.47695078)(607.7798295,588.298339)
\curveto(607.54167357,588.11972614)(607.24305609,587.98483663)(606.88397618,587.89367006)
\curveto(606.52488572,587.80250321)(606.08486682,587.75691986)(605.56391817,587.75691986)
\closepath
\moveto(603.52661934,592.5013144)
\lineto(605.29599942,592.5013144)
\curveto(605.77601635,592.50130966)(606.12021717,592.53293892)(606.32860294,592.59620229)
\curveto(606.60395888,592.67806144)(606.81140965,592.81388122)(606.95095587,593.00366206)
\curveto(607.0904914,593.19343241)(607.16026184,593.43158217)(607.16026739,593.71811206)
\curveto(607.16026184,593.98974567)(607.09514276,594.22882571)(606.96490997,594.43535288)
\curveto(606.83466646,594.6418667)(606.64861196,594.78326813)(606.40674591,594.85955757)
\curveto(606.16487025,594.93583282)(605.74996872,594.97397399)(605.16204004,594.97398121)
\lineto(603.52661934,594.97398121)
\closepath
\moveto(603.52661934,588.72254369)
\lineto(605.56391817,588.72254369)
\curveto(605.91369668,588.72254273)(606.15928862,588.73556654)(606.30069473,588.76161518)
\curveto(606.55000307,588.80626725)(606.75838412,588.88068905)(606.92583849,588.9848808)
\curveto(607.09328222,589.08907009)(607.23096255,589.24070451)(607.33887989,589.43978451)
\curveto(607.44678577,589.63886115)(607.50074158,589.86863846)(607.50074747,590.12911713)
\curveto(607.50074158,590.43424414)(607.42259869,590.69937181)(607.26631857,590.92450092)
\curveto(607.11002712,591.1496237)(606.89327363,591.30777003)(606.61605743,591.39894038)
\curveto(606.33883121,591.49010344)(605.93974431,591.53568679)(605.41879551,591.53569057)
\lineto(603.52661934,591.53569057)
\closepath
}
}
{
\newrgbcolor{curcolor}{0 0 0}
\pscustom[linestyle=none,fillstyle=solid,fillcolor=curcolor]
{
\newpath
\moveto(609.75573086,590.3858726)
\lineto(610.7771711,590.47517885)
\curveto(610.82554374,590.06585623)(610.93810671,589.73002785)(611.11486036,589.46769272)
\curveto(611.29161026,589.20535416)(611.56604065,588.99325203)(611.93815235,588.83138568)
\curveto(612.31025866,588.66951719)(612.72888129,588.58858348)(613.1940215,588.58858432)
\curveto(613.60705854,588.58858348)(613.97172536,588.64998147)(614.28802306,588.77277846)
\curveto(614.60431067,588.89557341)(614.83966961,589.06395274)(614.9941006,589.27791694)
\curveto(615.14852009,589.49187809)(615.22573271,589.72537649)(615.22573869,589.97841284)
\curveto(615.22573271,590.23516583)(615.15131091,590.4593615)(615.00247306,590.65100053)
\curveto(614.8536237,590.84263378)(614.60803176,591.00357092)(614.2656965,591.13381245)
\curveto(614.04614716,591.21939414)(613.56054491,591.35242311)(612.80888829,591.53289975)
\curveto(612.05722454,591.71336884)(611.5306903,591.88360871)(611.22928399,592.04361987)
\curveto(610.83856755,592.24827554)(610.54739226,592.50223993)(610.35575723,592.80551382)
\curveto(610.16411998,593.10877761)(610.06830191,593.44832707)(610.06830274,593.82416324)
\curveto(610.06830191,594.23719816)(610.18551625,594.62326125)(610.4199461,594.98235367)
\curveto(610.65437359,595.34143163)(610.99671388,595.61400148)(611.44696798,595.80006402)
\curveto(611.89721767,595.98611048)(612.39770428,596.07913773)(612.94842931,596.07914605)
\curveto(613.55496328,596.07913773)(614.08986997,595.98145912)(614.55315099,595.78610992)
\curveto(615.01642139,595.59074466)(615.37271576,595.30329046)(615.62203517,594.92374644)
\curveto(615.87134183,594.54418809)(616.00530107,594.11440219)(616.0239133,593.63438745)
\lineto(614.98572814,593.55624448)
\curveto(614.92990605,594.0734702)(614.74106073,594.46418465)(614.41919162,594.72838902)
\curveto(614.09731215,594.99257944)(613.6219429,595.12467814)(612.99308243,595.1246855)
\curveto(612.33816684,595.12467814)(611.86093704,595.00467298)(611.56139161,594.76466968)
\curveto(611.26184154,594.52465237)(611.11206767,594.23533762)(611.11206954,593.89672456)
\curveto(611.11206767,593.60275231)(611.21811874,593.36088146)(611.43022306,593.17111128)
\curveto(611.63860191,592.98133027)(612.18281133,592.78690332)(613.06285294,592.58782983)
\curveto(613.94288692,592.38874669)(614.54663377,592.21478573)(614.87409533,592.06594643)
\curveto(615.35038922,591.84639781)(615.70203223,591.56824633)(615.92902541,591.23149116)
\curveto(616.15600522,590.89472904)(616.26949846,590.5068054)(616.26950549,590.06771909)
\curveto(616.26949846,589.63234924)(616.14484195,589.22209906)(615.89553557,588.83696732)
\curveto(615.64621588,588.45183343)(615.28806096,588.15228568)(614.82106974,587.93832318)
\curveto(614.35406736,587.72436032)(613.8284634,587.61737898)(613.24425626,587.61737885)
\curveto(612.50375534,587.61737898)(611.88326358,587.7252906)(611.38277911,587.941114)
\curveto(610.88229036,588.15693704)(610.48971536,588.48160215)(610.20505294,588.91511029)
\curveto(609.92038858,589.34861612)(609.77061471,589.83886974)(609.75573086,590.3858726)
\closepath
}
}
{
\newrgbcolor{curcolor}{0 0 0}
\pscustom[linestyle=none,fillstyle=solid,fillcolor=curcolor]
{
\newpath
\moveto(617.75980221,587.75691986)
\lineto(617.75980221,595.93960504)
\lineto(620.57853073,595.93960504)
\curveto(621.21483342,595.93959685)(621.70043567,595.90052541)(622.03533893,595.82239058)
\curveto(622.50419112,595.71447091)(622.9042083,595.51911368)(623.23539167,595.23631832)
\curveto(623.66703176,594.87164401)(623.98983632,594.40557749)(624.20380632,593.83811734)
\curveto(624.41776167,593.27064502)(624.52474301,592.62224509)(624.52475066,591.89291557)
\curveto(624.52474301,591.2714894)(624.45218176,590.72076808)(624.30706667,590.24074995)
\curveto(624.16193673,589.76072685)(623.97588223,589.36350048)(623.74890261,589.04906967)
\curveto(623.52190925,588.73463627)(623.27352649,588.48718378)(623.00375359,588.30671146)
\curveto(622.73396843,588.12623805)(622.40837305,587.98948799)(622.02696647,587.89646088)
\curveto(621.6455496,587.80343349)(621.20739124,587.75691986)(620.7124901,587.75691986)
\closepath
\moveto(618.84264049,588.72254369)
\lineto(620.58969401,588.72254369)
\curveto(621.12924835,588.72254273)(621.55252234,588.77277744)(621.85951725,588.87324799)
\curveto(622.1665022,588.9737163)(622.41116387,589.11511772)(622.593503,589.29745268)
\curveto(622.8502525,589.55420635)(623.05026108,589.89933745)(623.19352937,590.33284702)
\curveto(623.33678502,590.76635143)(623.408416,591.2919554)(623.40842253,591.9096605)
\curveto(623.408416,592.76550705)(623.26794485,593.42320972)(622.98700866,593.88277046)
\curveto(622.70606026,594.34231896)(622.36465025,594.65023916)(621.96277761,594.80653199)
\curveto(621.6725275,594.91815764)(621.2055307,594.97397399)(620.56178581,594.97398121)
\lineto(618.84264049,594.97398121)
\closepath
}
}
{
\newrgbcolor{curcolor}{0 0 0}
\pscustom[linestyle=none,fillstyle=solid,fillcolor=curcolor]
{
\newpath
\moveto(625.50153955,590.21284174)
\lineto(625.50153955,591.2231187)
\lineto(628.58818682,591.2231187)
\lineto(628.58818682,590.21284174)
\closepath
}
}
{
\newrgbcolor{curcolor}{0 0 0}
\pscustom[linestyle=none,fillstyle=solid,fillcolor=curcolor]
{
\newpath
\moveto(629.79381958,587.75691986)
\lineto(629.79381958,595.93960504)
\lineto(630.87665787,595.93960504)
\lineto(630.87665787,588.72254369)
\lineto(634.90660241,588.72254369)
\lineto(634.90660241,587.75691986)
\closepath
}
}
{
\newrgbcolor{curcolor}{0 0 0}
\pscustom[linestyle=none,fillstyle=solid,fillcolor=curcolor]
{
\newpath
\moveto(636.07874805,594.78420542)
\lineto(636.07874805,595.93960504)
\lineto(637.08344337,595.93960504)
\lineto(637.08344337,594.78420542)
\closepath
\moveto(636.07874805,587.75691986)
\lineto(636.07874805,593.68462222)
\lineto(637.08344337,593.68462222)
\lineto(637.08344337,587.75691986)
\closepath
}
}
{
\newrgbcolor{curcolor}{0 0 0}
\pscustom[linestyle=none,fillstyle=solid,fillcolor=curcolor]
{
\newpath
\moveto(640.81197824,588.655564)
\lineto(640.95710089,587.76808314)
\curveto(640.67429496,587.70854569)(640.42126083,587.67877697)(640.19799777,587.67877689)
\curveto(639.83332861,587.67877697)(639.55052577,587.73645387)(639.34958839,587.85180775)
\curveto(639.14864804,587.96716145)(639.00724662,588.11879587)(638.9253837,588.30671146)
\curveto(638.84351866,588.49462596)(638.80258667,588.88999178)(638.80258761,589.4928101)
\lineto(638.80258761,592.90319253)
\lineto(638.06581104,592.90319253)
\lineto(638.06581104,593.68462222)
\lineto(638.80258761,593.68462222)
\lineto(638.80258761,595.15259371)
\lineto(639.80170128,595.7554109)
\lineto(639.80170128,593.68462222)
\lineto(640.81197824,593.68462222)
\lineto(640.81197824,592.90319253)
\lineto(639.80170128,592.90319253)
\lineto(639.80170128,589.43699369)
\curveto(639.80169934,589.15046808)(639.81937452,588.96627412)(639.85472687,588.88441127)
\curveto(639.89007523,588.80254616)(639.94775213,588.73742709)(640.02775773,588.68905385)
\curveto(640.107759,588.64067875)(640.22218252,588.61649166)(640.37102863,588.61649252)
\curveto(640.48265882,588.61649166)(640.62964188,588.62951548)(640.81197824,588.655564)
\closepath
}
}
{
\newrgbcolor{curcolor}{0 0 0}
\pscustom[linestyle=none,fillstyle=solid,fillcolor=curcolor]
{
\newpath
\moveto(645.84661822,589.66584096)
\lineto(646.88480338,589.53746322)
\curveto(646.72106957,588.93092377)(646.41780073,588.46020588)(645.97499595,588.12530814)
\curveto(645.5321813,587.79040967)(644.96657561,587.62296062)(644.2781772,587.62296049)
\curveto(643.41115998,587.62296062)(642.72368859,587.88994883)(642.21576098,588.42392592)
\curveto(641.70783101,588.95790167)(641.45386662,589.70677104)(641.45386704,590.67053627)
\curveto(641.45386662,591.66778549)(641.71062183,592.44177222)(642.22413344,592.99249878)
\curveto(642.73764268,593.54321487)(643.4037178,593.81857553)(644.22236079,593.81858159)
\curveto(645.01494978,593.81857553)(645.66241945,593.54879651)(646.16477173,593.0092437)
\curveto(646.66711376,592.46968039)(646.91828734,591.71057803)(646.91829322,590.73193432)
\curveto(646.91828734,590.67239391)(646.91642679,590.58308774)(646.91271158,590.46401557)
\lineto(642.49205219,590.46401557)
\curveto(642.52926164,589.81282211)(642.7134556,589.31419604)(643.04463462,588.96813588)
\curveto(643.37580962,588.62207329)(643.78885062,588.44904261)(644.28375884,588.4490433)
\curveto(644.6521435,588.44904261)(644.96657561,588.54579095)(645.22705611,588.73928861)
\curveto(645.48752822,588.93278431)(645.69404872,589.24163479)(645.84661822,589.66584096)
\closepath
\moveto(642.5478686,591.29009839)
\lineto(645.8577815,591.29009839)
\curveto(645.8131236,591.78872092)(645.68660654,592.16269047)(645.47822994,592.41200815)
\curveto(645.15821175,592.79899686)(644.74331021,592.99249354)(644.23352407,592.99249878)
\curveto(643.77210571,592.99249354)(643.38418207,592.83806831)(643.069752,592.52922261)
\curveto(642.75531786,592.22036736)(642.5813569,591.80732637)(642.5478686,591.29009839)
\closepath
}
}
{
\newrgbcolor{curcolor}{0 0 0}
\pscustom[linestyle=none,fillstyle=solid,fillcolor=curcolor]
{
\newpath
\moveto(676.63851094,590.48584819)
\lineto(677.65995118,590.57515444)
\curveto(677.70832381,590.16583182)(677.82088678,589.83000344)(677.99764043,589.5676683)
\curveto(678.17439034,589.30532974)(678.44882073,589.09322761)(678.82093243,588.93136127)
\curveto(679.19303874,588.76949278)(679.61166137,588.68855907)(680.07680157,588.6885599)
\curveto(680.48983861,588.68855907)(680.85450544,588.74995706)(681.17080314,588.87275404)
\curveto(681.48709074,588.995549)(681.72244969,589.16392832)(681.87688068,589.37789252)
\curveto(682.03130016,589.59185368)(682.10851278,589.82535208)(682.10851876,590.07838842)
\curveto(682.10851278,590.33514141)(682.03409098,590.55933709)(681.88525314,590.75097612)
\curveto(681.73640378,590.94260936)(681.49081183,591.10354651)(681.14847657,591.23378803)
\curveto(680.92892724,591.31936973)(680.44332499,591.4523987)(679.69166837,591.63287534)
\curveto(678.94000461,591.81334443)(678.41347037,591.9835843)(678.11206407,592.14359546)
\curveto(677.72134763,592.34825112)(677.43017233,592.60221552)(677.23853731,592.9054894)
\curveto(677.04690006,593.20875319)(676.95108199,593.54830266)(676.95108281,593.92413882)
\curveto(676.95108199,594.33717375)(677.06829632,594.72323684)(677.30272617,595.08232925)
\curveto(677.53715367,595.44140722)(677.87949395,595.71397706)(678.32974805,595.90003961)
\curveto(678.77999774,596.08608607)(679.28048435,596.17911332)(679.83120938,596.17912164)
\curveto(680.43774335,596.17911332)(680.97265005,596.0814347)(681.43593107,595.88608551)
\curveto(681.89920147,595.69072025)(682.25549584,595.40326604)(682.50481525,595.02372203)
\curveto(682.7541219,594.64416368)(682.88808114,594.21437778)(682.90669338,593.73436304)
\lineto(681.86850822,593.65622007)
\curveto(681.81268612,594.17344579)(681.6238408,594.56416024)(681.30197169,594.8283646)
\curveto(680.98009223,595.09255503)(680.50472297,595.22465372)(679.87586251,595.22466109)
\curveto(679.22094691,595.22465372)(678.74371711,595.10464857)(678.44417169,594.86464527)
\curveto(678.14462162,594.62462795)(677.99484774,594.3353132)(677.99484961,593.99670015)
\curveto(677.99484774,593.7027279)(678.10089881,593.46085704)(678.31300313,593.27108687)
\curveto(678.52138198,593.08130586)(679.0655914,592.88687891)(679.94563302,592.68780542)
\curveto(680.82566699,592.48872227)(681.42941385,592.31476131)(681.7568754,592.16592202)
\curveto(682.2331693,591.9463734)(682.58481231,591.66822192)(682.81180548,591.33146674)
\curveto(683.03878529,590.99470462)(683.15227854,590.60678099)(683.15228556,590.16769467)
\curveto(683.15227854,589.73232483)(683.02762202,589.32207465)(682.77831564,588.93694291)
\curveto(682.52899596,588.55180901)(682.17084104,588.25226126)(681.70384982,588.03829877)
\curveto(681.23684744,587.82433591)(680.71124347,587.71735457)(680.12703634,587.71735443)
\curveto(679.38653542,587.71735457)(678.76604365,587.82526618)(678.26555918,588.04108959)
\curveto(677.76507043,588.25691263)(677.37249544,588.58157773)(677.08783301,589.01508588)
\curveto(676.80316866,589.44859171)(676.65339479,589.93884532)(676.63851094,590.48584819)
\closepath
}
}
{
\newrgbcolor{curcolor}{0 0 0}
\pscustom[linestyle=none,fillstyle=solid,fillcolor=curcolor]
{
\newpath
\moveto(688.39902786,587.85689545)
\lineto(688.39902786,588.72763139)
\curveto(687.93760806,588.05783431)(687.31060439,587.72293621)(686.51801497,587.72293607)
\curveto(686.16822974,587.72293621)(685.84170409,587.78991583)(685.53843703,587.92387513)
\curveto(685.23516642,588.05783431)(685.01004047,588.22621363)(684.86305852,588.42901361)
\curveto(684.71607436,588.63181245)(684.61281411,588.88019521)(684.55327746,589.17416264)
\curveto(684.51234468,589.37137909)(684.49187868,589.68395066)(684.49187941,590.11187826)
\lineto(684.49187941,593.78459781)
\lineto(685.49657473,593.78459781)
\lineto(685.49657473,590.49701147)
\curveto(685.49657299,589.97233513)(685.51703899,589.61883158)(685.55797278,589.43649975)
\curveto(685.62122951,589.17230078)(685.75518875,588.96485001)(685.9598509,588.81414682)
\curveto(686.16450865,588.66344171)(686.41754278,588.58808964)(686.71895403,588.58809037)
\curveto(687.02035936,588.58808964)(687.30316221,588.66530226)(687.56736341,588.81972846)
\curveto(687.83155699,588.97415273)(688.01854177,589.18439432)(688.12831829,589.45045385)
\curveto(688.23808608,589.71651019)(688.29297216,590.10257328)(688.29297669,590.60864428)
\lineto(688.29297669,593.78459781)
\lineto(689.297672,593.78459781)
\lineto(689.297672,587.85689545)
\closepath
}
}
{
\newrgbcolor{curcolor}{0 0 0}
\pscustom[linestyle=none,fillstyle=solid,fillcolor=curcolor]
{
\newpath
\moveto(690.87727599,587.85689545)
\lineto(690.87727599,593.78459781)
\lineto(691.78150177,593.78459781)
\lineto(691.78150177,592.94177007)
\curveto(692.21686765,593.59295574)(692.84573186,593.91855112)(693.66809631,593.91855718)
\curveto(694.02531741,593.91855112)(694.3537036,593.85436232)(694.65325588,593.72599058)
\curveto(694.9527991,593.5976071)(695.17699477,593.42922778)(695.32584358,593.2208521)
\curveto(695.47468198,593.01246569)(695.5788725,592.76501321)(695.63841545,592.4784939)
\curveto(695.67562084,592.29243477)(695.69422629,591.96683939)(695.69423186,591.50170678)
\lineto(695.69423186,587.85689545)
\lineto(694.68953654,587.85689545)
\lineto(694.68953654,591.4626353)
\curveto(694.68953198,591.8719516)(694.65046053,592.17801125)(694.57232209,592.38081518)
\curveto(694.49417475,592.58361007)(694.35556415,592.74547748)(694.15648986,592.86641792)
\curveto(693.95740751,592.98734834)(693.72390911,593.04781605)(693.45599396,593.04782124)
\curveto(693.02806528,593.04781605)(692.65874709,592.91199626)(692.3480383,592.64036147)
\curveto(692.03732505,592.36871712)(691.88196954,591.85334615)(691.8819713,591.09424702)
\lineto(691.8819713,587.85689545)
\closepath
}
}
{
\newrgbcolor{curcolor}{0 0 0}
\pscustom[linestyle=none,fillstyle=solid,fillcolor=curcolor]
{
\newpath
\moveto(697.03940837,591.84218686)
\curveto(697.03940781,593.20038074)(697.40407464,594.26368222)(698.13340993,595.03209449)
\curveto(698.86274193,595.80049241)(699.80417771,596.18469495)(700.95772009,596.18470328)
\curveto(701.7130969,596.18469495)(702.39405638,596.00422208)(703.00060057,595.64328414)
\curveto(703.60713173,595.28233062)(704.06947717,594.77905319)(704.38763827,594.13345034)
\curveto(704.70578356,593.48783495)(704.86486016,592.75571048)(704.86486854,591.93707475)
\curveto(704.86486016,591.1072676)(704.69741111,590.36491013)(704.36252088,589.71000014)
\curveto(704.0276149,589.05508644)(703.55317592,588.55925119)(702.93920252,588.22249291)
\curveto(702.32521621,587.88573389)(701.66286219,587.71735457)(700.95213845,587.71735443)
\curveto(700.18186835,587.71735457)(699.49346669,587.90340907)(698.88693142,588.27551849)
\curveto(698.28039134,588.64762708)(697.82083672,589.15555587)(697.50826618,589.79930639)
\curveto(697.1956936,590.44305302)(697.03940781,591.1240125)(697.03940837,591.84218686)
\closepath
\moveto(698.15573649,591.82544194)
\curveto(698.15573482,590.83934911)(698.42086249,590.06257157)(698.95112028,589.49510697)
\curveto(699.48137315,588.92763911)(700.146518,588.64390599)(700.94655681,588.64390678)
\curveto(701.76147107,588.64390599)(702.43219755,588.93042992)(702.95873826,589.50347944)
\curveto(703.48526603,590.07652565)(703.74853315,590.88958383)(703.74854041,591.94265639)
\curveto(703.74853315,592.60872743)(703.63597018,593.19014774)(703.41085115,593.68691909)
\curveto(703.18571828,594.18367878)(702.85640182,594.5688116)(702.42290076,594.84231871)
\curveto(701.98938784,595.11581184)(701.50285531,595.2525619)(700.96330173,595.25256929)
\curveto(700.19675271,595.2525619)(699.5371895,594.98929478)(698.98461013,594.46276714)
\curveto(698.43202576,593.9362263)(698.15573482,593.05711877)(698.15573649,591.82544194)
\closepath
}
}
{
\newrgbcolor{curcolor}{0 0 0}
\pscustom[linestyle=none,fillstyle=solid,fillcolor=curcolor]
{
\newpath
\moveto(705.88630795,590.48584819)
\lineto(706.90774819,590.57515444)
\curveto(706.95612082,590.16583182)(707.0686838,589.83000344)(707.24543745,589.5676683)
\curveto(707.42218735,589.30532974)(707.69661774,589.09322761)(708.06872944,588.93136127)
\curveto(708.44083575,588.76949278)(708.85945838,588.68855907)(709.32459858,588.6885599)
\curveto(709.73763563,588.68855907)(710.10230245,588.74995706)(710.41860015,588.87275404)
\curveto(710.73488776,588.995549)(710.9702467,589.16392832)(711.12467769,589.37789252)
\curveto(711.27909717,589.59185368)(711.35630979,589.82535208)(711.35631578,590.07838842)
\curveto(711.35630979,590.33514141)(711.28188799,590.55933709)(711.13305015,590.75097612)
\curveto(710.98420079,590.94260936)(710.73860885,591.10354651)(710.39627359,591.23378803)
\curveto(710.17672425,591.31936973)(709.691122,591.4523987)(708.93946538,591.63287534)
\curveto(708.18780163,591.81334443)(707.66126739,591.9835843)(707.35986108,592.14359546)
\curveto(706.96914464,592.34825112)(706.67796934,592.60221552)(706.48633432,592.9054894)
\curveto(706.29469707,593.20875319)(706.198879,593.54830266)(706.19887983,593.92413882)
\curveto(706.198879,594.33717375)(706.31609334,594.72323684)(706.55052319,595.08232925)
\curveto(706.78495068,595.44140722)(707.12729097,595.71397706)(707.57754506,595.90003961)
\curveto(708.02779475,596.08608607)(708.52828136,596.17911332)(709.0790064,596.17912164)
\curveto(709.68554037,596.17911332)(710.22044706,596.0814347)(710.68372808,595.88608551)
\curveto(711.14699848,595.69072025)(711.50329285,595.40326604)(711.75261226,595.02372203)
\curveto(712.00191891,594.64416368)(712.13587816,594.21437778)(712.15449039,593.73436304)
\lineto(711.11630523,593.65622007)
\curveto(711.06048314,594.17344579)(710.87163782,594.56416024)(710.5497687,594.8283646)
\curveto(710.22788924,595.09255503)(709.75251999,595.22465372)(709.12365952,595.22466109)
\curveto(708.46874392,595.22465372)(707.99151413,595.10464857)(707.6919687,594.86464527)
\curveto(707.39241863,594.62462795)(707.24264476,594.3353132)(707.24264663,593.99670015)
\curveto(707.24264476,593.7027279)(707.34869582,593.46085704)(707.56080014,593.27108687)
\curveto(707.769179,593.08130586)(708.31338841,592.88687891)(709.19343003,592.68780542)
\curveto(710.073464,592.48872227)(710.67721086,592.31476131)(711.00467242,592.16592202)
\curveto(711.48096631,591.9463734)(711.83260932,591.66822192)(712.0596025,591.33146674)
\curveto(712.2865823,590.99470462)(712.40007555,590.60678099)(712.40008258,590.16769467)
\curveto(712.40007555,589.73232483)(712.27541903,589.32207465)(712.02611265,588.93694291)
\curveto(711.77679297,588.55180901)(711.41863805,588.25226126)(710.95164683,588.03829877)
\curveto(710.48464445,587.82433591)(709.95904048,587.71735457)(709.37483335,587.71735443)
\curveto(708.63433243,587.71735457)(708.01384067,587.82526618)(707.5133562,588.04108959)
\curveto(707.01286745,588.25691263)(706.62029245,588.58157773)(706.33563002,589.01508588)
\curveto(706.05096567,589.44859171)(705.9011918,589.93884532)(705.88630795,590.48584819)
\closepath
}
}
{
\newrgbcolor{curcolor}{0 0 0}
\pscustom[linestyle=none,fillstyle=solid,fillcolor=curcolor]
{
\newpath
\moveto(719.87389818,587.85689545)
\lineto(719.87389818,589.81605131)
\lineto(716.32397474,589.81605131)
\lineto(716.32397474,590.73702202)
\lineto(720.05809232,596.03958062)
\lineto(720.8785935,596.03958062)
\lineto(720.8785935,590.73702202)
\lineto(721.98375835,590.73702202)
\lineto(721.98375835,589.81605131)
\lineto(720.8785935,589.81605131)
\lineto(720.8785935,587.85689545)
\closepath
\moveto(719.87389818,590.73702202)
\lineto(719.87389818,594.42648648)
\lineto(717.31192513,590.73702202)
\closepath
}
}
{
\newrgbcolor{curcolor}{0 0 0}
\pscustom[linestyle=none,fillstyle=solid,fillcolor=curcolor]
{
\newpath
\moveto(723.58010868,587.85689545)
\lineto(723.58010868,589.00113178)
\lineto(724.72434501,589.00113178)
\lineto(724.72434501,587.85689545)
\closepath
}
}
{
\newrgbcolor{curcolor}{0 0 0}
\pscustom[linestyle=none,fillstyle=solid,fillcolor=curcolor]
{
\newpath
\moveto(729.97108734,587.85689545)
\lineto(728.96639202,587.85689545)
\lineto(728.96639202,594.25903726)
\curveto(728.72451792,594.02832327)(728.40729499,593.79761569)(728.0147223,593.56691382)
\curveto(727.62214499,593.33620053)(727.26957171,593.16316984)(726.95700139,593.04782124)
\lineto(726.95700139,594.01902671)
\curveto(727.51888475,594.28321794)(728.01006863,594.60323169)(728.43055452,594.9790689)
\curveto(728.85103498,595.35489187)(729.14872218,595.7195587)(729.32361703,596.07307047)
\lineto(729.97108734,596.07307047)
\closepath
}
}
{
\newrgbcolor{curcolor}{0 0 0}
\pscustom[linestyle=none,fillstyle=solid,fillcolor=curcolor]
{
\newpath
\moveto(733.11355213,587.85689545)
\lineto(733.11355213,589.00113178)
\lineto(734.25778846,589.00113178)
\lineto(734.25778846,587.85689545)
\closepath
}
}
{
\newrgbcolor{curcolor}{0 0 0}
\pscustom[linestyle=none,fillstyle=solid,fillcolor=curcolor]
{
\newpath
\moveto(738.94078509,587.85689545)
\lineto(738.94078509,589.81605131)
\lineto(735.39086164,589.81605131)
\lineto(735.39086164,590.73702202)
\lineto(739.12497923,596.03958062)
\lineto(739.9454804,596.03958062)
\lineto(739.9454804,590.73702202)
\lineto(741.05064525,590.73702202)
\lineto(741.05064525,589.81605131)
\lineto(739.9454804,589.81605131)
\lineto(739.9454804,587.85689545)
\closepath
\moveto(738.94078509,590.73702202)
\lineto(738.94078509,594.42648648)
\lineto(736.37881203,590.73702202)
\closepath
}
}
{
\newrgbcolor{curcolor}{0 0 0}
\pscustom[linestyle=none,fillstyle=solid,fillcolor=curcolor]
{
\newpath
\moveto(464.76874619,543.30684662)
\lineto(464.76874619,551.48953179)
\lineto(465.87949268,551.48953179)
\lineto(470.17735597,545.06506342)
\lineto(470.17735597,551.48953179)
\lineto(471.21554113,551.48953179)
\lineto(471.21554113,543.30684662)
\lineto(470.10479464,543.30684662)
\lineto(465.80693135,549.73689663)
\lineto(465.80693135,543.30684662)
\closepath
}
}
{
\newrgbcolor{curcolor}{0 0 0}
\pscustom[linestyle=none,fillstyle=solid,fillcolor=curcolor]
{
\newpath
\moveto(476.9702125,545.21576772)
\lineto(478.00839766,545.08738998)
\curveto(477.84466385,544.48085053)(477.54139501,544.01013264)(477.09859024,543.6752349)
\curveto(476.65577558,543.34033643)(476.0901699,543.17288738)(475.40177148,543.17288724)
\curveto(474.53475426,543.17288738)(473.84728288,543.43987559)(473.33935527,543.97385268)
\curveto(472.8314253,544.50782843)(472.5774609,545.2566978)(472.57746132,546.22046303)
\curveto(472.5774609,547.21771225)(472.83421611,547.99169898)(473.34772773,548.54242554)
\curveto(473.86123696,549.09314163)(474.52731208,549.36850229)(475.34595508,549.36850835)
\curveto(476.13854407,549.36850229)(476.78601373,549.09872326)(477.28836602,548.55917046)
\curveto(477.79070804,548.01960715)(478.04188162,547.26050478)(478.0418875,546.28186108)
\curveto(478.04188162,546.22232066)(478.04002108,546.1330145)(478.03630586,546.01394233)
\lineto(473.61564648,546.01394233)
\curveto(473.65285592,545.36274886)(473.83704988,544.8641228)(474.1682289,544.51806264)
\curveto(474.49940391,544.17200005)(474.9124449,543.99896937)(475.40735312,543.99897006)
\curveto(475.77573779,543.99896937)(476.0901699,544.09571771)(476.35065039,544.28921537)
\curveto(476.6111225,544.48271107)(476.817643,544.79156154)(476.9702125,545.21576772)
\closepath
\moveto(473.67146289,546.84002514)
\lineto(476.98137578,546.84002514)
\curveto(476.93671788,547.33864767)(476.81020082,547.71261722)(476.60182422,547.96193491)
\curveto(476.28180603,548.34892362)(475.86690449,548.5424203)(475.35711836,548.54242554)
\curveto(474.89569999,548.5424203)(474.50777636,548.38799507)(474.19334628,548.07914936)
\curveto(473.87891214,547.77029412)(473.70495118,547.35725313)(473.67146289,546.84002514)
\closepath
}
}
{
\newrgbcolor{curcolor}{0 0 0}
\pscustom[linestyle=none,fillstyle=solid,fillcolor=curcolor]
{
\newpath
\moveto(481.46901502,544.20549076)
\lineto(481.61413767,543.3180099)
\curveto(481.33133174,543.25847245)(481.07829762,543.22870373)(480.85503455,543.22870365)
\curveto(480.49036539,543.22870373)(480.20756255,543.28638062)(480.00662517,543.40173451)
\curveto(479.80568482,543.51708821)(479.6642834,543.66872263)(479.58242048,543.85663822)
\curveto(479.50055544,544.04455272)(479.45962345,544.43991854)(479.45962439,545.04273686)
\lineto(479.45962439,548.45311929)
\lineto(478.72284782,548.45311929)
\lineto(478.72284782,549.23454898)
\lineto(479.45962439,549.23454898)
\lineto(479.45962439,550.70252046)
\lineto(480.45873806,551.30533765)
\lineto(480.45873806,549.23454898)
\lineto(481.46901502,549.23454898)
\lineto(481.46901502,548.45311929)
\lineto(480.45873806,548.45311929)
\lineto(480.45873806,544.98692045)
\curveto(480.45873612,544.70039484)(480.4764113,544.51620088)(480.51176365,544.43433803)
\curveto(480.54711201,544.35247292)(480.60478891,544.28735384)(480.68479451,544.23898061)
\curveto(480.76479578,544.1906055)(480.8792193,544.16641842)(481.02806541,544.16641928)
\curveto(481.1396956,544.16641842)(481.28667866,544.17944223)(481.46901502,544.20549076)
\closepath
}
}
{
\newrgbcolor{curcolor}{0 0 0}
\pscustom[linestyle=none,fillstyle=solid,fillcolor=curcolor]
{
\newpath
\moveto(482.52952591,543.30684662)
\lineto(482.52952591,551.48953179)
\lineto(485.59942826,551.48953179)
\curveto(486.22456748,551.48952361)(486.72598436,551.40672936)(487.10368041,551.24114879)
\curveto(487.48136564,551.07555235)(487.7771923,550.82065768)(487.99116128,550.47646402)
\curveto(488.20511765,550.13225602)(488.31209899,549.77224056)(488.31210561,549.39641656)
\curveto(488.31209899,549.046628)(488.2172112,548.71731153)(488.02744194,548.40846616)
\curveto(487.83766001,548.09961059)(487.55113608,547.85029756)(487.16786928,547.66052632)
\curveto(487.66276878,547.51539945)(488.04325024,547.26794696)(488.30931479,546.91816811)
\curveto(488.57536611,546.56838204)(488.70839508,546.15534104)(488.7084021,545.67904389)
\curveto(488.70839508,545.29576924)(488.62746137,544.93947487)(488.46560073,544.61015971)
\curveto(488.30372654,544.28084194)(488.10371795,544.02687754)(487.86557436,543.84826576)
\curveto(487.62741843,543.6696529)(487.32880095,543.53476338)(486.96972104,543.44359681)
\curveto(486.61063057,543.35242997)(486.17061168,543.30684662)(485.64966303,543.30684662)
\closepath
\moveto(483.6123642,548.05124116)
\lineto(485.38174428,548.05124116)
\curveto(485.8617612,548.05123642)(486.20596203,548.08286568)(486.4143478,548.14612905)
\curveto(486.68970374,548.22798819)(486.89715451,548.36380798)(487.03670073,548.55358882)
\curveto(487.17623626,548.74335916)(487.2460067,548.98150893)(487.24601225,549.26803882)
\curveto(487.2460067,549.53967243)(487.18088762,549.77875247)(487.05065483,549.98527964)
\curveto(486.92041132,550.19179346)(486.73435682,550.33319488)(486.49249076,550.40948433)
\curveto(486.25061511,550.48575957)(485.83571357,550.52390075)(485.2477849,550.52390796)
\lineto(483.6123642,550.52390796)
\closepath
\moveto(483.6123642,544.27247045)
\lineto(485.64966303,544.27247045)
\curveto(485.99944153,544.27246948)(486.24503348,544.2854933)(486.38643959,544.31154193)
\curveto(486.63574793,544.35619401)(486.84412897,544.43061581)(487.01158334,544.53480756)
\curveto(487.17902708,544.63899685)(487.31670741,544.79063127)(487.42462475,544.98971127)
\curveto(487.53253063,545.18878791)(487.58648644,545.41856522)(487.58649233,545.67904389)
\curveto(487.58648644,545.9841709)(487.50834354,546.24929857)(487.35206342,546.47442768)
\curveto(487.19577198,546.69955046)(486.97901849,546.85769679)(486.70180229,546.94886714)
\curveto(486.42457607,547.0400302)(486.02548916,547.08561355)(485.50454037,547.08561733)
\lineto(483.6123642,547.08561733)
\closepath
}
}
{
\newrgbcolor{curcolor}{0 0 0}
\pscustom[linestyle=none,fillstyle=solid,fillcolor=curcolor]
{
\newpath
\moveto(489.84147572,545.93579936)
\lineto(490.86291596,546.02510561)
\curveto(490.91128859,545.61578299)(491.02385157,545.27995461)(491.20060522,545.01761947)
\curveto(491.37735512,544.75528092)(491.65178551,544.54317878)(492.02389721,544.38131244)
\curveto(492.39600352,544.21944395)(492.81462615,544.13851024)(493.27976635,544.13851107)
\curveto(493.6928034,544.13851024)(494.05747022,544.19990823)(494.37376792,544.32270522)
\curveto(494.69005553,544.44550017)(494.92541447,544.61387949)(495.07984546,544.82784369)
\curveto(495.23426495,545.04180485)(495.31147756,545.27530325)(495.31148355,545.52833959)
\curveto(495.31147756,545.78509258)(495.23705576,546.00928826)(495.08821792,546.20092729)
\curveto(494.93936856,546.39256053)(494.69377662,546.55349768)(494.35144136,546.68373921)
\curveto(494.13189202,546.7693209)(493.64628977,546.90234987)(492.89463315,547.08282651)
\curveto(492.1429694,547.2632956)(491.61643516,547.43353547)(491.31502885,547.59354663)
\curveto(490.92431241,547.79820229)(490.63313711,548.05216669)(490.44150209,548.35544058)
\curveto(490.24986484,548.65870437)(490.15404677,548.99825383)(490.1540476,549.37408999)
\curveto(490.15404677,549.78712492)(490.27126111,550.17318801)(490.50569096,550.53228043)
\curveto(490.74011845,550.89135839)(491.08245874,551.16392823)(491.53271283,551.34999078)
\curveto(491.98296252,551.53603724)(492.48344913,551.62906449)(493.03417417,551.62907281)
\curveto(493.64070814,551.62906449)(494.17561483,551.53138588)(494.63889585,551.33603668)
\curveto(495.10216625,551.14067142)(495.45846062,550.85321722)(495.70778003,550.4736732)
\curveto(495.95708669,550.09411485)(496.09104593,549.66432895)(496.10965816,549.18431421)
\lineto(495.071473,549.10617124)
\curveto(495.01565091,549.62339696)(494.82680559,550.01411141)(494.50493647,550.27831578)
\curveto(494.18305701,550.5425062)(493.70768776,550.67460489)(493.07882729,550.67461226)
\curveto(492.42391169,550.67460489)(491.9466819,550.55459974)(491.64713647,550.31459644)
\curveto(491.3475864,550.07457913)(491.19781253,549.78526437)(491.1978144,549.44665132)
\curveto(491.19781253,549.15267907)(491.30386359,548.91080822)(491.51596791,548.72103804)
\curveto(491.72434677,548.53125703)(492.26855619,548.33683008)(493.1485978,548.13775659)
\curveto(494.02863177,547.93867344)(494.63237863,547.76471248)(494.95984019,547.61587319)
\curveto(495.43613408,547.39632457)(495.78777709,547.11817309)(496.01477027,546.78141792)
\curveto(496.24175007,546.44465579)(496.35524332,546.05673216)(496.35525035,545.61764584)
\curveto(496.35524332,545.182276)(496.2305868,544.77202582)(495.98128042,544.38689408)
\curveto(495.73196074,544.00176018)(495.37380582,543.70221244)(494.9068146,543.48824994)
\curveto(494.43981222,543.27428708)(493.91420825,543.16730574)(493.33000112,543.1673056)
\curveto(492.5895002,543.16730574)(491.96900844,543.27521735)(491.46852397,543.49104076)
\curveto(490.96803522,543.7068638)(490.57546022,544.0315289)(490.29079779,544.46503705)
\curveto(490.00613344,544.89854288)(489.85635957,545.3887965)(489.84147572,545.93579936)
\closepath
}
}
{
\newrgbcolor{curcolor}{0 0 0}
\pscustom[linestyle=none,fillstyle=solid,fillcolor=curcolor]
{
\newpath
\moveto(497.84554897,543.30684662)
\lineto(497.84554897,551.48953179)
\lineto(500.66427749,551.48953179)
\curveto(501.30058019,551.48952361)(501.78618244,551.45045217)(502.1210857,551.37231734)
\curveto(502.58993789,551.26439766)(502.98995507,551.06904044)(503.32113844,550.78624507)
\curveto(503.75277852,550.42157077)(504.07558308,549.95550424)(504.28955309,549.38804409)
\curveto(504.50350844,548.82057178)(504.61048978,548.17217184)(504.61049742,547.44284233)
\curveto(504.61048978,546.82141616)(504.53792852,546.27069483)(504.39281344,545.7906767)
\curveto(504.2476835,545.3106536)(504.061629,544.91342724)(503.83464938,544.59899643)
\curveto(503.60765601,544.28456303)(503.35927325,544.03711054)(503.08950035,543.85663822)
\curveto(502.8197152,543.67616481)(502.49411982,543.53941475)(502.11271324,543.44638763)
\curveto(501.73129636,543.35336024)(501.29313801,543.30684662)(500.79823687,543.30684662)
\closepath
\moveto(498.92838725,544.27247045)
\lineto(500.67544077,544.27247045)
\curveto(501.21499512,544.27246948)(501.63826911,544.3227042)(501.94526402,544.42317475)
\curveto(502.25224897,544.52364306)(502.49691064,544.66504448)(502.67924976,544.84737944)
\curveto(502.93599926,545.10413311)(503.13600785,545.44926421)(503.27927613,545.88277377)
\curveto(503.42253178,546.31627819)(503.49416277,546.84188215)(503.4941693,547.45958725)
\curveto(503.49416277,548.31543381)(503.35369162,548.97313647)(503.07275543,549.43269722)
\curveto(502.79180702,549.89224571)(502.45039701,550.20016591)(502.04852437,550.35645875)
\curveto(501.75827426,550.4680844)(501.29127746,550.52390075)(500.64753257,550.52390796)
\lineto(498.92838725,550.52390796)
\closepath
}
}
{
\newrgbcolor{curcolor}{0 0 0}
\pscustom[linestyle=none,fillstyle=solid,fillcolor=curcolor]
{
\newpath
\moveto(512.65363968,543.30684662)
\lineto(511.64894436,543.30684662)
\lineto(511.64894436,549.70898843)
\curveto(511.40707026,549.47827445)(511.08984733,549.24756686)(510.69727463,549.01686499)
\curveto(510.30469733,548.7861517)(509.95212405,548.61312101)(509.63955373,548.49777241)
\lineto(509.63955373,549.46897788)
\curveto(510.20143708,549.73316911)(510.69262097,550.05318286)(511.11310686,550.42902007)
\curveto(511.53358732,550.80484305)(511.83127452,551.16950987)(512.00616936,551.52302164)
\lineto(512.65363968,551.52302164)
\closepath
}
}
{
\newrgbcolor{curcolor}{0 0 0}
\pscustom[linestyle=none,fillstyle=solid,fillcolor=curcolor]
{
\newpath
\moveto(515.79610447,543.30684662)
\lineto(515.79610447,544.45108295)
\lineto(516.9403408,544.45108295)
\lineto(516.9403408,543.30684662)
\closepath
}
}
{
\newrgbcolor{curcolor}{0 0 0}
\pscustom[linestyle=none,fillstyle=solid,fillcolor=curcolor]
{
\newpath
\moveto(522.18708313,543.30684662)
\lineto(521.18238781,543.30684662)
\lineto(521.18238781,549.70898843)
\curveto(520.94051371,549.47827445)(520.62329078,549.24756686)(520.23071808,549.01686499)
\curveto(519.83814078,548.7861517)(519.4855675,548.61312101)(519.17299718,548.49777241)
\lineto(519.17299718,549.46897788)
\curveto(519.73488053,549.73316911)(520.22606442,550.05318286)(520.64655031,550.42902007)
\curveto(521.06703077,550.80484305)(521.36471797,551.16950987)(521.53961281,551.52302164)
\lineto(522.18708313,551.52302164)
\closepath
}
}
{
\newrgbcolor{curcolor}{0 0 0}
\pscustom[linestyle=none,fillstyle=solid,fillcolor=curcolor]
{
\newpath
\moveto(214.68971453,543.30684662)
\lineto(214.68971453,551.48953179)
\lineto(220.20995712,551.48953179)
\lineto(220.20995712,550.52390796)
\lineto(215.77255281,550.52390796)
\lineto(215.77255281,547.98984311)
\lineto(219.61272157,547.98984311)
\lineto(219.61272157,547.02421928)
\lineto(215.77255281,547.02421928)
\lineto(215.77255281,543.30684662)
\closepath
}
}
{
\newrgbcolor{curcolor}{0 0 0}
\pscustom[linestyle=none,fillstyle=solid,fillcolor=curcolor]
{
\newpath
\moveto(221.48257112,543.30684662)
\lineto(221.48257112,549.23454898)
\lineto(222.38679691,549.23454898)
\lineto(222.38679691,548.33590483)
\curveto(222.61750284,548.75638298)(222.83053525,549.03360419)(223.02589476,549.16756929)
\curveto(223.2212497,549.30152267)(223.43614265,549.36850229)(223.67057425,549.36850835)
\curveto(224.00919052,549.36850229)(224.35339134,549.26059068)(224.70317777,549.0447732)
\lineto(224.35711605,548.11263921)
\curveto(224.11152049,548.25775691)(223.86592855,548.33031817)(223.62033949,548.33032319)
\curveto(223.40079229,548.33031817)(223.20357452,548.26426882)(223.02868558,548.13217495)
\curveto(222.85379206,548.00007143)(222.72913554,547.81680774)(222.65471566,547.58238335)
\curveto(222.54308104,547.22515443)(222.48726469,546.83443997)(222.48726644,546.41023881)
\lineto(222.48726644,543.30684662)
\closepath
}
}
{
\newrgbcolor{curcolor}{0 0 0}
\pscustom[linestyle=none,fillstyle=solid,fillcolor=curcolor]
{
\newpath
\moveto(229.36942915,545.21576772)
\lineto(230.40761431,545.08738998)
\curveto(230.2438805,544.48085053)(229.94061166,544.01013264)(229.49780689,543.6752349)
\curveto(229.05499223,543.34033643)(228.48938655,543.17288738)(227.80098813,543.17288724)
\curveto(226.93397091,543.17288738)(226.24649953,543.43987559)(225.73857192,543.97385268)
\curveto(225.23064195,544.50782843)(224.97667755,545.2566978)(224.97667797,546.22046303)
\curveto(224.97667755,547.21771225)(225.23343277,547.99169898)(225.74694438,548.54242554)
\curveto(226.26045362,549.09314163)(226.92652873,549.36850229)(227.74517173,549.36850835)
\curveto(228.53776072,549.36850229)(229.18523039,549.09872326)(229.68758267,548.55917046)
\curveto(230.1899247,548.01960715)(230.44109827,547.26050478)(230.44110416,546.28186108)
\curveto(230.44109827,546.22232066)(230.43923773,546.1330145)(230.43552252,546.01394233)
\lineto(226.01486313,546.01394233)
\curveto(226.05207257,545.36274886)(226.23626653,544.8641228)(226.56744555,544.51806264)
\curveto(226.89862056,544.17200005)(227.31166155,543.99896937)(227.80656978,543.99897006)
\curveto(228.17495444,543.99896937)(228.48938655,544.09571771)(228.74986704,544.28921537)
\curveto(229.01033915,544.48271107)(229.21685965,544.79156154)(229.36942915,545.21576772)
\closepath
\moveto(226.07067954,546.84002514)
\lineto(229.38059244,546.84002514)
\curveto(229.33593453,547.33864767)(229.20941747,547.71261722)(229.00104087,547.96193491)
\curveto(228.68102269,548.34892362)(228.26612115,548.5424203)(227.75633501,548.54242554)
\curveto(227.29491665,548.5424203)(226.90699301,548.38799507)(226.59256294,548.07914936)
\curveto(226.27812879,547.77029412)(226.10416783,547.35725313)(226.07067954,546.84002514)
\closepath
}
}
{
\newrgbcolor{curcolor}{0 0 0}
\pscustom[linestyle=none,fillstyle=solid,fillcolor=curcolor]
{
\newpath
\moveto(235.73249964,545.21576772)
\lineto(236.7706848,545.08738998)
\curveto(236.60695099,544.48085053)(236.30368215,544.01013264)(235.86087738,543.6752349)
\curveto(235.41806272,543.34033643)(234.85245704,543.17288738)(234.16405862,543.17288724)
\curveto(233.2970414,543.17288738)(232.60957002,543.43987559)(232.10164241,543.97385268)
\curveto(231.59371244,544.50782843)(231.33974804,545.2566978)(231.33974846,546.22046303)
\curveto(231.33974804,547.21771225)(231.59650325,547.99169898)(232.11001487,548.54242554)
\curveto(232.6235241,549.09314163)(233.28959922,549.36850229)(234.10824222,549.36850835)
\curveto(234.90083121,549.36850229)(235.54830087,549.09872326)(236.05065316,548.55917046)
\curveto(236.55299518,548.01960715)(236.80416876,547.26050478)(236.80417464,546.28186108)
\curveto(236.80416876,546.22232066)(236.80230822,546.1330145)(236.798593,546.01394233)
\lineto(232.37793362,546.01394233)
\curveto(232.41514306,545.36274886)(232.59933702,544.8641228)(232.93051604,544.51806264)
\curveto(233.26169105,544.17200005)(233.67473204,543.99896937)(234.16964026,543.99897006)
\curveto(234.53802493,543.99896937)(234.85245704,544.09571771)(235.11293753,544.28921537)
\curveto(235.37340964,544.48271107)(235.57993014,544.79156154)(235.73249964,545.21576772)
\closepath
\moveto(232.43375002,546.84002514)
\lineto(235.74366292,546.84002514)
\curveto(235.69900502,547.33864767)(235.57248796,547.71261722)(235.36411136,547.96193491)
\curveto(235.04409317,548.34892362)(234.62919163,548.5424203)(234.1194055,548.54242554)
\curveto(233.65798713,548.5424203)(233.2700635,548.38799507)(232.95563342,548.07914936)
\curveto(232.64119928,547.77029412)(232.46723832,547.35725313)(232.43375002,546.84002514)
\closepath
}
}
{
\newrgbcolor{curcolor}{0 0 0}
\pscustom[linestyle=none,fillstyle=solid,fillcolor=curcolor]
{
\newpath
\moveto(238.12144295,543.30684662)
\lineto(238.12144295,551.48953179)
\lineto(241.1913453,551.48953179)
\curveto(241.81648452,551.48952361)(242.3179014,551.40672936)(242.69559745,551.24114879)
\curveto(243.07328268,551.07555235)(243.36910934,550.82065768)(243.58307831,550.47646402)
\curveto(243.79703469,550.13225602)(243.90401603,549.77224056)(243.90402265,549.39641656)
\curveto(243.90401603,549.046628)(243.80912824,548.71731153)(243.61935898,548.40846616)
\curveto(243.42957705,548.09961059)(243.14305312,547.85029756)(242.75978632,547.66052632)
\curveto(243.25468582,547.51539945)(243.63516728,547.26794696)(243.90123183,546.91816811)
\curveto(244.16728315,546.56838204)(244.30031212,546.15534104)(244.30031914,545.67904389)
\curveto(244.30031212,545.29576924)(244.21937841,544.93947487)(244.05751777,544.61015971)
\curveto(243.89564358,544.28084194)(243.69563499,544.02687754)(243.4574914,543.84826576)
\curveto(243.21933546,543.6696529)(242.92071799,543.53476338)(242.56163808,543.44359681)
\curveto(242.20254761,543.35242997)(241.76252871,543.30684662)(241.24158007,543.30684662)
\closepath
\moveto(239.20428123,548.05124116)
\lineto(240.97366132,548.05124116)
\curveto(241.45367824,548.05123642)(241.79787907,548.08286568)(242.00626483,548.14612905)
\curveto(242.28162077,548.22798819)(242.48907154,548.36380798)(242.62861776,548.55358882)
\curveto(242.7681533,548.74335916)(242.83792374,548.98150893)(242.83792929,549.26803882)
\curveto(242.83792374,549.53967243)(242.77280466,549.77875247)(242.64257187,549.98527964)
\curveto(242.51232836,550.19179346)(242.32627385,550.33319488)(242.0844078,550.40948433)
\curveto(241.84253215,550.48575957)(241.42763061,550.52390075)(240.83970194,550.52390796)
\lineto(239.20428123,550.52390796)
\closepath
\moveto(239.20428123,544.27247045)
\lineto(241.24158007,544.27247045)
\curveto(241.59135857,544.27246948)(241.83695051,544.2854933)(241.97835663,544.31154193)
\curveto(242.22766497,544.35619401)(242.43604601,544.43061581)(242.60350038,544.53480756)
\curveto(242.77094411,544.63899685)(242.90862445,544.79063127)(243.01654179,544.98971127)
\curveto(243.12444767,545.18878791)(243.17840347,545.41856522)(243.17840937,545.67904389)
\curveto(243.17840347,545.9841709)(243.10026058,546.24929857)(242.94398046,546.47442768)
\curveto(242.78768902,546.69955046)(242.57093552,546.85769679)(242.29371933,546.94886714)
\curveto(242.01649311,547.0400302)(241.6174062,547.08561355)(241.09645741,547.08561733)
\lineto(239.20428123,547.08561733)
\closepath
}
}
{
\newrgbcolor{curcolor}{0 0 0}
\pscustom[linestyle=none,fillstyle=solid,fillcolor=curcolor]
{
\newpath
\moveto(245.43339085,545.93579936)
\lineto(246.45483109,546.02510561)
\curveto(246.50320372,545.61578299)(246.6157667,545.27995461)(246.79252035,545.01761947)
\curveto(246.96927025,544.75528092)(247.24370064,544.54317878)(247.61581234,544.38131244)
\curveto(247.98791865,544.21944395)(248.40654128,544.13851024)(248.87168149,544.13851107)
\curveto(249.28471853,544.13851024)(249.64938535,544.19990823)(249.96568305,544.32270522)
\curveto(250.28197066,544.44550017)(250.5173296,544.61387949)(250.67176059,544.82784369)
\curveto(250.82618008,545.04180485)(250.90339269,545.27530325)(250.90339868,545.52833959)
\curveto(250.90339269,545.78509258)(250.82897089,546.00928826)(250.68013305,546.20092729)
\curveto(250.53128369,546.39256053)(250.28569175,546.55349768)(249.94335649,546.68373921)
\curveto(249.72380715,546.7693209)(249.2382049,546.90234987)(248.48654828,547.08282651)
\curveto(247.73488453,547.2632956)(247.20835029,547.43353547)(246.90694398,547.59354663)
\curveto(246.51622754,547.79820229)(246.22505224,548.05216669)(246.03341722,548.35544058)
\curveto(245.84177997,548.65870437)(245.7459619,548.99825383)(245.74596273,549.37408999)
\curveto(245.7459619,549.78712492)(245.86317624,550.17318801)(246.09760609,550.53228043)
\curveto(246.33203358,550.89135839)(246.67437387,551.16392823)(247.12462797,551.34999078)
\curveto(247.57487766,551.53603724)(248.07536427,551.62906449)(248.6260893,551.62907281)
\curveto(249.23262327,551.62906449)(249.76752996,551.53138588)(250.23081098,551.33603668)
\curveto(250.69408138,551.14067142)(251.05037575,550.85321722)(251.29969516,550.4736732)
\curveto(251.54900182,550.09411485)(251.68296106,549.66432895)(251.70157329,549.18431421)
\lineto(250.66338813,549.10617124)
\curveto(250.60756604,549.62339696)(250.41872072,550.01411141)(250.09685161,550.27831578)
\curveto(249.77497214,550.5425062)(249.29960289,550.67460489)(248.67074242,550.67461226)
\curveto(248.01582682,550.67460489)(247.53859703,550.55459974)(247.2390516,550.31459644)
\curveto(246.93950153,550.07457913)(246.78972766,549.78526437)(246.78972953,549.44665132)
\curveto(246.78972766,549.15267907)(246.89577872,548.91080822)(247.10788304,548.72103804)
\curveto(247.3162619,548.53125703)(247.86047132,548.33683008)(248.74051293,548.13775659)
\curveto(249.6205469,547.93867344)(250.22429376,547.76471248)(250.55175532,547.61587319)
\curveto(251.02804921,547.39632457)(251.37969222,547.11817309)(251.6066854,546.78141792)
\curveto(251.8336652,546.44465579)(251.94715845,546.05673216)(251.94716548,545.61764584)
\curveto(251.94715845,545.182276)(251.82250193,544.77202582)(251.57319555,544.38689408)
\curveto(251.32387587,544.00176018)(250.96572095,543.70221244)(250.49872973,543.48824994)
\curveto(250.03172735,543.27428708)(249.50612338,543.16730574)(248.92191625,543.1673056)
\curveto(248.18141533,543.16730574)(247.56092357,543.27521735)(247.0604391,543.49104076)
\curveto(246.55995035,543.7068638)(246.16737535,544.0315289)(245.88271292,544.46503705)
\curveto(245.59804857,544.89854288)(245.4482747,545.3887965)(245.43339085,545.93579936)
\closepath
}
}
{
\newrgbcolor{curcolor}{0 0 0}
\pscustom[linestyle=none,fillstyle=solid,fillcolor=curcolor]
{
\newpath
\moveto(253.4374641,543.30684662)
\lineto(253.4374641,551.48953179)
\lineto(256.25619262,551.48953179)
\curveto(256.89249532,551.48952361)(257.37809757,551.45045217)(257.71300083,551.37231734)
\curveto(258.18185302,551.26439766)(258.5818702,551.06904044)(258.91305357,550.78624507)
\curveto(259.34469365,550.42157077)(259.66749822,549.95550424)(259.88146822,549.38804409)
\curveto(260.09542357,548.82057178)(260.20240491,548.17217184)(260.20241255,547.44284233)
\curveto(260.20240491,546.82141616)(260.12984365,546.27069483)(259.98472857,545.7906767)
\curveto(259.83959863,545.3106536)(259.65354413,544.91342724)(259.42656451,544.59899643)
\curveto(259.19957114,544.28456303)(258.95118838,544.03711054)(258.68141548,543.85663822)
\curveto(258.41163033,543.67616481)(258.08603495,543.53941475)(257.70462837,543.44638763)
\curveto(257.32321149,543.35336024)(256.88505314,543.30684662)(256.390152,543.30684662)
\closepath
\moveto(254.52030239,544.27247045)
\lineto(256.2673559,544.27247045)
\curveto(256.80691025,544.27246948)(257.23018424,544.3227042)(257.53717915,544.42317475)
\curveto(257.8441641,544.52364306)(258.08882577,544.66504448)(258.27116489,544.84737944)
\curveto(258.52791439,545.10413311)(258.72792298,545.44926421)(258.87119126,545.88277377)
\curveto(259.01444691,546.31627819)(259.0860779,546.84188215)(259.08608443,547.45958725)
\curveto(259.0860779,548.31543381)(258.94560675,548.97313647)(258.66467056,549.43269722)
\curveto(258.38372215,549.89224571)(258.04231214,550.20016591)(257.6404395,550.35645875)
\curveto(257.35018939,550.4680844)(256.88319259,550.52390075)(256.2394477,550.52390796)
\lineto(254.52030239,550.52390796)
\closepath
}
}
{
\newrgbcolor{curcolor}{0 0 0}
\pscustom[linestyle=none,fillstyle=solid,fillcolor=curcolor]
{
\newpath
\moveto(269.7414345,544.27247045)
\lineto(269.7414345,543.30684662)
\lineto(264.33282472,543.30684662)
\curveto(264.32538219,543.54871747)(264.36445364,543.7812856)(264.45003917,544.0045517)
\curveto(264.58771904,544.37293891)(264.80819363,544.73574519)(265.11146359,545.09297162)
\curveto(265.4147313,545.45019448)(265.85288965,545.86323547)(266.42593996,546.33209584)
\curveto(267.31527804,547.06142647)(267.91623408,547.6391257)(268.22880989,548.06519526)
\curveto(268.54137721,548.49125531)(268.69766299,548.89406331)(268.6976677,549.27362046)
\curveto(268.69766299,549.67177113)(268.55533129,550.0075995)(268.27067219,550.2811066)
\curveto(267.98600452,550.55459974)(267.61482579,550.6913498)(267.15713488,550.69135718)
\curveto(266.67339001,550.6913498)(266.28639664,550.54622729)(265.99615363,550.25598921)
\curveto(265.7059066,549.96573724)(265.55892354,549.56385952)(265.55520402,549.05035484)
\lineto(264.5226005,549.15640601)
\curveto(264.59330068,549.9266658)(264.85935861,550.51366775)(265.32077511,550.91741363)
\curveto(265.78218894,551.32114429)(266.40175044,551.52301342)(267.17946145,551.52302164)
\curveto(267.96460825,551.52301342)(268.58603029,551.30532966)(269.04372942,550.86996968)
\curveto(269.50141844,550.43459459)(269.73026547,549.89503653)(269.73027122,549.2512939)
\curveto(269.73026547,548.92383203)(269.66328585,548.60195774)(269.52933215,548.28567007)
\curveto(269.39536737,547.96937244)(269.17303224,547.63633488)(268.8623261,547.28655639)
\curveto(268.5516102,546.93676995)(268.03530896,546.45674934)(267.31342082,545.84649311)
\curveto(266.71060091,545.34042232)(266.32360754,544.99715177)(266.15243957,544.81668041)
\curveto(265.98126726,544.63620603)(265.83986584,544.4548029)(265.72823488,544.27247045)
\closepath
}
}
{
\newrgbcolor{curcolor}{0 0 0}
\pscustom[linestyle=none,fillstyle=solid,fillcolor=curcolor]
{
\newpath
\moveto(271.3880196,543.30684662)
\lineto(271.3880196,544.45108295)
\lineto(272.53225593,544.45108295)
\lineto(272.53225593,543.30684662)
\closepath
}
}
{
\newrgbcolor{curcolor}{0 0 0}
\pscustom[linestyle=none,fillstyle=solid,fillcolor=curcolor]
{
\newpath
\moveto(273.99464591,547.3423728)
\curveto(273.99464543,548.30985217)(274.09418459,549.08849027)(274.29326368,549.67828941)
\curveto(274.49234122,550.26807581)(274.78816788,550.72297906)(275.18074454,551.04300054)
\curveto(275.57331788,551.36300655)(276.06729258,551.52301342)(276.66267013,551.52302164)
\curveto(277.10175561,551.52301342)(277.48688843,551.43463753)(277.81806974,551.25789371)
\curveto(278.14924246,551.08113398)(278.42274258,550.82623931)(278.63857092,550.49320894)
\curveto(278.85438902,550.1601642)(279.02369862,549.75456538)(279.14650021,549.27641128)
\curveto(279.26929056,548.79824524)(279.33068855,548.15356639)(279.33069436,547.3423728)
\curveto(279.33068855,546.38232754)(279.23207966,545.60741053)(279.0348674,545.01761947)
\curveto(278.83764412,544.42782499)(278.54274773,543.97199146)(278.15017736,543.65011752)
\curveto(277.75759773,543.32824289)(277.26176248,543.16730574)(276.66267013,543.1673056)
\curveto(275.8737959,543.16730574)(275.25423441,543.45010859)(274.8039838,544.01571498)
\curveto(274.26442446,544.69667375)(273.99464543,545.80555858)(273.99464591,547.3423728)
\closepath
\moveto(275.02724942,547.3423728)
\curveto(275.02724792,545.99905526)(275.18446397,545.10506338)(275.49889806,544.66039447)
\curveto(275.81332819,544.21572286)(276.20125182,543.99338773)(276.66267013,543.99338842)
\curveto(277.12408215,543.99338773)(277.51200579,544.21665313)(277.8264422,544.66318529)
\curveto(278.14087001,545.10971474)(278.29808606,546.00277635)(278.29809084,547.3423728)
\curveto(278.29808606,548.68940336)(278.14087001,549.58432551)(277.8264422,550.02714195)
\curveto(277.51200579,550.46994494)(277.12036106,550.6913498)(276.65150685,550.69135718)
\curveto(276.19008855,550.6913498)(275.82170064,550.49599257)(275.546342,550.10528492)
\curveto(275.2002786,549.60665205)(275.02724792,548.68568227)(275.02724942,547.3423728)
\closepath
}
}
{
\newrgbcolor{curcolor}{0 0 0}
\pscustom[linestyle=none,fillstyle=solid,fillcolor=curcolor]
{
\newpath
\moveto(560.10404268,543.20687103)
\lineto(560.10404268,545.1660269)
\lineto(556.55411924,545.1660269)
\lineto(556.55411924,546.0869976)
\lineto(560.28823683,551.38955621)
\lineto(561.108738,551.38955621)
\lineto(561.108738,546.0869976)
\lineto(562.21390285,546.0869976)
\lineto(562.21390285,545.1660269)
\lineto(561.108738,545.1660269)
\lineto(561.108738,543.20687103)
\closepath
\moveto(560.10404268,546.0869976)
\lineto(560.10404268,549.77646206)
\lineto(557.54206963,546.0869976)
\closepath
}
}
{
\newrgbcolor{curcolor}{0 0 0}
\pscustom[linestyle=none,fillstyle=solid,fillcolor=curcolor]
{
\newpath
\moveto(563.81025223,543.20687103)
\lineto(563.81025223,544.35110736)
\lineto(564.95448856,544.35110736)
\lineto(564.95448856,543.20687103)
\closepath
}
}
{
\newrgbcolor{curcolor}{0 0 0}
\pscustom[linestyle=none,fillstyle=solid,fillcolor=curcolor]
{
\newpath
\moveto(569.63748518,543.20687103)
\lineto(569.63748518,545.1660269)
\lineto(566.08756174,545.1660269)
\lineto(566.08756174,546.0869976)
\lineto(569.82167932,551.38955621)
\lineto(570.6421805,551.38955621)
\lineto(570.6421805,546.0869976)
\lineto(571.74734534,546.0869976)
\lineto(571.74734534,545.1660269)
\lineto(570.6421805,545.1660269)
\lineto(570.6421805,543.20687103)
\closepath
\moveto(569.63748518,546.0869976)
\lineto(569.63748518,549.77646206)
\lineto(567.07551213,546.0869976)
\closepath
}
}
{
\newrgbcolor{curcolor}{0 0 0}
\pscustom[linestyle=none,fillstyle=solid,fillcolor=curcolor]
{
\newpath
\moveto(573.14275566,543.20687103)
\lineto(573.14275566,551.38955621)
\lineto(576.21265801,551.38955621)
\curveto(576.83779723,551.38954803)(577.33921412,551.30675377)(577.71691017,551.1411732)
\curveto(578.09459539,550.97557676)(578.39042205,550.72068209)(578.60439103,550.37648843)
\curveto(578.81834741,550.03228043)(578.92532874,549.67226497)(578.92533536,549.29644097)
\curveto(578.92532874,548.94665242)(578.83044095,548.61733595)(578.64067169,548.30849058)
\curveto(578.45088976,547.999635)(578.16436583,547.75032197)(577.78109903,547.56055073)
\curveto(578.27599853,547.41542387)(578.65647999,547.16797138)(578.92254454,546.81819253)
\curveto(579.18859586,546.46840645)(579.32162483,546.05536546)(579.32163185,545.5790683)
\curveto(579.32162483,545.19579366)(579.24069113,544.83949929)(579.07883048,544.51018412)
\curveto(578.91695629,544.18086635)(578.7169477,543.92690196)(578.47880411,543.74829017)
\curveto(578.24064818,543.56967731)(577.9420307,543.4347878)(577.58295079,543.34362123)
\curveto(577.22386032,543.25245439)(576.78384143,543.20687103)(576.26289278,543.20687103)
\closepath
\moveto(574.22559395,547.95126558)
\lineto(575.99497403,547.95126558)
\curveto(576.47499095,547.95126083)(576.81919178,547.9828901)(577.02757755,548.04615347)
\curveto(577.30293349,548.12801261)(577.51038426,548.26383239)(577.64993048,548.45361323)
\curveto(577.78946601,548.64338358)(577.85923645,548.88153334)(577.859242,549.16806323)
\curveto(577.85923645,549.43969685)(577.79411737,549.67877688)(577.66388458,549.88530406)
\curveto(577.53364107,550.09181788)(577.34758657,550.2332193)(577.10572052,550.30950875)
\curveto(576.86384486,550.38578399)(576.44894332,550.42392516)(575.86101465,550.42393238)
\lineto(574.22559395,550.42393238)
\closepath
\moveto(574.22559395,544.17249486)
\lineto(576.26289278,544.17249486)
\curveto(576.61267129,544.1724939)(576.85826323,544.18551771)(576.99966934,544.21156635)
\curveto(577.24897768,544.25621842)(577.45735872,544.33064022)(577.6248131,544.43483197)
\curveto(577.79225683,544.53902127)(577.92993716,544.69065569)(578.0378545,544.88973569)
\curveto(578.14576038,545.08881232)(578.19971619,545.31858963)(578.19972208,545.5790683)
\curveto(578.19971619,545.88419532)(578.1215733,546.14932298)(577.96529317,546.3744521)
\curveto(577.80900173,546.59957488)(577.59224824,546.7577212)(577.31503204,546.84889155)
\curveto(577.03780582,546.94005461)(576.63871892,546.98563797)(576.11777012,546.98564175)
\lineto(574.22559395,546.98564175)
\closepath
}
}
{
\newrgbcolor{curcolor}{0 0 0}
\pscustom[linestyle=none,fillstyle=solid,fillcolor=curcolor]
{
\newpath
\moveto(580.45470547,545.83582377)
\lineto(581.47614571,545.92513002)
\curveto(581.52451835,545.5158074)(581.63708132,545.17997903)(581.81383497,544.91764389)
\curveto(581.99058487,544.65530533)(582.26501526,544.4432032)(582.63712696,544.28133686)
\curveto(583.00923327,544.11946836)(583.4278559,544.03853466)(583.89299611,544.03853549)
\curveto(584.30603315,544.03853466)(584.67069997,544.09993264)(584.98699767,544.22272963)
\curveto(585.30328528,544.34552458)(585.53864422,544.51390391)(585.69307521,544.72786811)
\curveto(585.8474947,544.94182926)(585.92470732,545.17532766)(585.9247133,545.42836401)
\curveto(585.92470732,545.685117)(585.85028551,545.90931267)(585.70144767,546.1009517)
\curveto(585.55259831,546.29258495)(585.30700637,546.45352209)(584.96467111,546.58376362)
\curveto(584.74512177,546.66934531)(584.25951952,546.80237428)(583.5078629,546.98285092)
\curveto(582.75619915,547.16332002)(582.22966491,547.33355988)(581.9282586,547.49357104)
\curveto(581.53754216,547.69822671)(581.24636687,547.9521911)(581.05473184,548.25546499)
\curveto(580.86309459,548.55872878)(580.76727652,548.89827825)(580.76727735,549.27411441)
\curveto(580.76727652,549.68714933)(580.88449086,550.07321243)(581.11892071,550.43230484)
\curveto(581.3533482,550.7913828)(581.69568849,551.06395265)(582.14594259,551.25001519)
\curveto(582.59619228,551.43606165)(583.09667889,551.5290889)(583.64740392,551.52909722)
\curveto(584.25393789,551.5290889)(584.78884458,551.43141029)(585.2521256,551.23606109)
\curveto(585.715396,551.04069584)(586.07169037,550.75324163)(586.32100978,550.37369761)
\curveto(586.57031644,549.99413926)(586.70427568,549.56435336)(586.72288791,549.08433863)
\lineto(585.68470275,549.00619566)
\curveto(585.62888066,549.52342137)(585.44003534,549.91413583)(585.11816623,550.17834019)
\curveto(584.79628676,550.44253061)(584.32091751,550.57462931)(583.69205704,550.57463668)
\curveto(583.03714145,550.57462931)(582.55991165,550.45462415)(582.26036622,550.21462085)
\curveto(581.96081615,549.97460354)(581.81104228,549.68528879)(581.81104415,549.34667574)
\curveto(581.81104228,549.05270348)(581.91709334,548.81083263)(582.12919766,548.62106245)
\curveto(582.33757652,548.43128145)(582.88178594,548.23685449)(583.76182755,548.03778101)
\curveto(584.64186152,547.83869786)(585.24560838,547.6647369)(585.57306994,547.51589761)
\curveto(586.04936383,547.29634898)(586.40100684,547.0181975)(586.62800002,546.68144233)
\curveto(586.85497982,546.34468021)(586.96847307,545.95675657)(586.9684801,545.51767026)
\curveto(586.96847307,545.08230041)(586.84381655,544.67205024)(586.59451018,544.2869185)
\curveto(586.34519049,543.9017846)(585.98703557,543.60223685)(585.52004435,543.38827435)
\curveto(585.05304197,543.1743115)(584.52743801,543.06733016)(583.94323087,543.06733002)
\curveto(583.20272995,543.06733016)(582.58223819,543.17524177)(582.08175372,543.39106517)
\curveto(581.58126497,543.60688821)(581.18868997,543.93155332)(580.90402754,544.36506147)
\curveto(580.61936319,544.7985673)(580.46958932,545.28882091)(580.45470547,545.83582377)
\closepath
}
}
{
\newrgbcolor{curcolor}{0 0 0}
\pscustom[linestyle=none,fillstyle=solid,fillcolor=curcolor]
{
\newpath
\moveto(588.45877681,543.20687103)
\lineto(588.45877681,551.38955621)
\lineto(591.27750534,551.38955621)
\curveto(591.91380803,551.38954803)(592.39941028,551.35047658)(592.73431354,551.27234176)
\curveto(593.20316573,551.16442208)(593.60318291,550.96906485)(593.93436628,550.68626949)
\curveto(594.36600637,550.32159519)(594.68881093,549.85552866)(594.90278093,549.28806851)
\curveto(595.11673628,548.7205962)(595.22371762,548.07219626)(595.22372527,547.34286675)
\curveto(595.22371762,546.72144057)(595.15115637,546.17071925)(595.00604128,545.69070112)
\curveto(594.86091134,545.21067802)(594.67485684,544.81345166)(594.44787722,544.49902084)
\curveto(594.22088386,544.18458744)(593.9725011,543.93713495)(593.70272819,543.75666264)
\curveto(593.43294304,543.57618922)(593.10734766,543.43943916)(592.72594108,543.34641205)
\curveto(592.3445242,543.25338466)(591.90636585,543.20687103)(591.41146471,543.20687103)
\closepath
\moveto(589.5416151,544.17249486)
\lineto(591.28866862,544.17249486)
\curveto(591.82822296,544.1724939)(592.25149695,544.22272861)(592.55849186,544.32319916)
\curveto(592.86547681,544.42366748)(593.11013848,544.5650689)(593.29247761,544.74740385)
\curveto(593.5492271,545.00415752)(593.74923569,545.34928862)(593.89250398,545.78279819)
\curveto(594.03575963,546.2163026)(594.10739061,546.74190657)(594.10739714,547.35961167)
\curveto(594.10739061,548.21545822)(593.96691946,548.87316089)(593.68598327,549.33272163)
\curveto(593.40503487,549.79227013)(593.06362485,550.10019033)(592.66175222,550.25648316)
\curveto(592.37150211,550.36810881)(591.90450531,550.42392516)(591.26076042,550.42393238)
\lineto(589.5416151,550.42393238)
\closepath
}
}
{
\newrgbcolor{curcolor}{0 0 0}
\pscustom[linestyle=none,fillstyle=solid,fillcolor=curcolor]
{
\newpath
\moveto(596.20051416,545.66279291)
\lineto(596.20051416,546.67306987)
\lineto(599.28716143,546.67306987)
\lineto(599.28716143,545.66279291)
\closepath
}
}
{
\newrgbcolor{curcolor}{0 0 0}
\pscustom[linestyle=none,fillstyle=solid,fillcolor=curcolor]
{
\newpath
\moveto(600.49279419,543.20687103)
\lineto(600.49279419,551.38955621)
\lineto(601.57563248,551.38955621)
\lineto(601.57563248,544.17249486)
\lineto(605.60557702,544.17249486)
\lineto(605.60557702,543.20687103)
\closepath
}
}
{
\newrgbcolor{curcolor}{0 0 0}
\pscustom[linestyle=none,fillstyle=solid,fillcolor=curcolor]
{
\newpath
\moveto(606.77772266,550.2341566)
\lineto(606.77772266,551.38955621)
\lineto(607.78241798,551.38955621)
\lineto(607.78241798,550.2341566)
\closepath
\moveto(606.77772266,543.20687103)
\lineto(606.77772266,549.13457339)
\lineto(607.78241798,549.13457339)
\lineto(607.78241798,543.20687103)
\closepath
}
}
{
\newrgbcolor{curcolor}{0 0 0}
\pscustom[linestyle=none,fillstyle=solid,fillcolor=curcolor]
{
\newpath
\moveto(611.51095285,544.10551518)
\lineto(611.6560755,543.21803431)
\curveto(611.37326957,543.15849686)(611.12023544,543.12872814)(610.89697237,543.12872806)
\curveto(610.53230322,543.12872814)(610.24950038,543.18640504)(610.048563,543.30175892)
\curveto(609.84762265,543.41711262)(609.70622123,543.56874704)(609.62435831,543.75666264)
\curveto(609.54249327,543.94457713)(609.50156128,544.33994295)(609.50156222,544.94276127)
\lineto(609.50156222,548.3531437)
\lineto(608.76478565,548.3531437)
\lineto(608.76478565,549.13457339)
\lineto(609.50156222,549.13457339)
\lineto(609.50156222,550.60254488)
\lineto(610.50067589,551.20536207)
\lineto(610.50067589,549.13457339)
\lineto(611.51095285,549.13457339)
\lineto(611.51095285,548.3531437)
\lineto(610.50067589,548.3531437)
\lineto(610.50067589,544.88694487)
\curveto(610.50067395,544.60041925)(610.51834913,544.4162253)(610.55370148,544.33436244)
\curveto(610.58904984,544.25249733)(610.64672674,544.18737826)(610.72673234,544.13900502)
\curveto(610.80673361,544.09062992)(610.92115713,544.06644283)(611.07000323,544.06644369)
\curveto(611.18163343,544.06644283)(611.32861649,544.07946665)(611.51095285,544.10551518)
\closepath
}
}
{
\newrgbcolor{curcolor}{0 0 0}
\pscustom[linestyle=none,fillstyle=solid,fillcolor=curcolor]
{
\newpath
\moveto(616.54559283,545.11579213)
\lineto(617.58377799,544.9874144)
\curveto(617.42004418,544.38087494)(617.11677534,543.91015705)(616.67397056,543.57525932)
\curveto(616.23115591,543.24036084)(615.66555022,543.07291179)(614.97715181,543.07291166)
\curveto(614.11013459,543.07291179)(613.4226632,543.3399)(612.91473559,543.87387709)
\curveto(612.40680562,544.40785284)(612.15284123,545.15672221)(612.15284165,546.12048745)
\curveto(612.15284123,547.11773666)(612.40959644,547.89172339)(612.92310805,548.44244995)
\curveto(613.43661729,548.99316604)(614.10269241,549.2685267)(614.9213354,549.26853277)
\curveto(615.71392439,549.2685267)(616.36139406,548.99874768)(616.86374634,548.45919487)
\curveto(617.36608837,547.91963157)(617.61726195,547.1605292)(617.61726783,546.18188549)
\curveto(617.61726195,546.12234508)(617.6154014,546.03303892)(617.61168619,545.91396674)
\lineto(613.1910268,545.91396674)
\curveto(613.22823625,545.26277328)(613.4124302,544.76414721)(613.74360923,544.41808705)
\curveto(614.07478423,544.07202447)(614.48782523,543.89899378)(614.98273345,543.89899447)
\curveto(615.35111811,543.89899378)(615.66555022,543.99574212)(615.92603072,544.18923979)
\curveto(616.18650283,544.38273548)(616.39302332,544.69158596)(616.54559283,545.11579213)
\closepath
\moveto(613.24684321,546.74004956)
\lineto(616.55675611,546.74004956)
\curveto(616.51209821,547.23867209)(616.38558114,547.61264164)(616.17720455,547.86195933)
\curveto(615.85718636,548.24894803)(615.44228482,548.44244472)(614.93249868,548.44244995)
\curveto(614.47108032,548.44244472)(614.08315668,548.28801948)(613.76872661,547.97917378)
\curveto(613.45429247,547.67031853)(613.28033151,547.25727754)(613.24684321,546.74004956)
\closepath
}
}
{
\newrgbcolor{curcolor}{0 0 0}
\pscustom[linestyle=none,fillstyle=solid,fillcolor=curcolor]
{
\newpath
\moveto(622.16630619,543.20687103)
\lineto(622.16630619,551.38955621)
\lineto(625.79437261,551.38955621)
\curveto(626.52370173,551.38954803)(627.07814414,551.3160565)(627.45770152,551.1690814)
\curveto(627.83724651,551.02209038)(628.14051535,550.76254435)(628.36750894,550.39044253)
\curveto(628.59448833,550.01832635)(628.70798158,549.6071459)(628.70798902,549.15689995)
\curveto(628.70798158,548.57640396)(628.52006653,548.08708062)(628.14424331,547.68892847)
\curveto(627.76840634,547.29076735)(627.1879163,547.03773323)(626.40277144,546.92982534)
\curveto(626.68929023,546.79214128)(626.906974,546.6563215)(627.05582339,546.52236557)
\curveto(627.37211026,546.23211723)(627.671658,545.86931096)(627.95446753,545.43394565)
\lineto(629.3777859,543.20687103)
\lineto(628.01586558,543.20687103)
\lineto(626.9330273,544.90927143)
\curveto(626.61672898,545.40045361)(626.35625268,545.7762837)(626.15159761,546.03676284)
\curveto(625.94693277,546.29723631)(625.76366909,546.47956972)(625.601806,546.58376362)
\curveto(625.43993425,546.68795076)(625.27527602,546.76051202)(625.10783081,546.8014476)
\curveto(624.985031,546.82749164)(624.78409213,546.84051545)(624.50501362,546.84051909)
\lineto(623.24914447,546.84051909)
\lineto(623.24914447,543.20687103)
\closepath
\moveto(623.24914447,547.77823472)
\lineto(625.57668862,547.77823472)
\curveto(626.07158929,547.77823014)(626.45858265,547.82939513)(626.73766987,547.93172983)
\curveto(627.01674616,548.03405508)(627.22884829,548.19778305)(627.37397691,548.42291421)
\curveto(627.51909331,548.64803494)(627.59165457,548.89269661)(627.59166089,549.15689995)
\curveto(627.59165457,549.54388737)(627.45118342,549.86204057)(627.17024702,550.1113605)
\curveto(626.88929882,550.36066663)(626.44555884,550.48532315)(625.83902573,550.48533043)
\lineto(623.24914447,550.48533043)
\closepath
}
}
{
\newrgbcolor{curcolor}{0 0 0}
\pscustom[linestyle=none,fillstyle=solid,fillcolor=curcolor]
{
\newpath
\moveto(634.33986239,545.11579213)
\lineto(635.37804755,544.9874144)
\curveto(635.21431374,544.38087494)(634.9110449,543.91015705)(634.46824012,543.57525932)
\curveto(634.02542547,543.24036084)(633.45981978,543.07291179)(632.77142137,543.07291166)
\curveto(631.90440415,543.07291179)(631.21693276,543.3399)(630.70900515,543.87387709)
\curveto(630.20107518,544.40785284)(629.94711079,545.15672221)(629.94711121,546.12048745)
\curveto(629.94711079,547.11773666)(630.203866,547.89172339)(630.71737762,548.44244995)
\curveto(631.23088685,548.99316604)(631.89696197,549.2685267)(632.71560496,549.26853277)
\curveto(633.50819395,549.2685267)(634.15566362,548.99874768)(634.65801591,548.45919487)
\curveto(635.16035793,547.91963157)(635.41153151,547.1605292)(635.41153739,546.18188549)
\curveto(635.41153151,546.12234508)(635.40967096,546.03303892)(635.40595575,545.91396674)
\lineto(630.98529637,545.91396674)
\curveto(631.02250581,545.26277328)(631.20669977,544.76414721)(631.53787879,544.41808705)
\curveto(631.86905379,544.07202447)(632.28209479,543.89899378)(632.77700301,543.89899447)
\curveto(633.14538768,543.89899378)(633.45981978,543.99574212)(633.72030028,544.18923979)
\curveto(633.98077239,544.38273548)(634.18729289,544.69158596)(634.33986239,545.11579213)
\closepath
\moveto(631.04111277,546.74004956)
\lineto(634.35102567,546.74004956)
\curveto(634.30636777,547.23867209)(634.17985071,547.61264164)(633.97147411,547.86195933)
\curveto(633.65145592,548.24894803)(633.23655438,548.44244472)(632.72676824,548.44244995)
\curveto(632.26534988,548.44244472)(631.87742625,548.28801948)(631.56299617,547.97917378)
\curveto(631.24856203,547.67031853)(631.07460107,547.25727754)(631.04111277,546.74004956)
\closepath
}
}
{
\newrgbcolor{curcolor}{0 0 0}
\pscustom[linestyle=none,fillstyle=solid,fillcolor=curcolor]
{
\newpath
\moveto(636.62275834,543.20687103)
\lineto(636.62275834,551.38955621)
\lineto(637.62745365,551.38955621)
\lineto(637.62745365,543.20687103)
\closepath
}
}
{
\newrgbcolor{curcolor}{0 0 0}
\pscustom[linestyle=none,fillstyle=solid,fillcolor=curcolor]
{
\newpath
\moveto(643.2481647,545.11579213)
\lineto(644.28634986,544.9874144)
\curveto(644.12261604,544.38087494)(643.81934721,543.91015705)(643.37654243,543.57525932)
\curveto(642.93372778,543.24036084)(642.36812209,543.07291179)(641.67972368,543.07291166)
\curveto(640.81270646,543.07291179)(640.12523507,543.3399)(639.61730746,543.87387709)
\curveto(639.10937749,544.40785284)(638.8554131,545.15672221)(638.85541351,546.12048745)
\curveto(638.8554131,547.11773666)(639.11216831,547.89172339)(639.62567992,548.44244995)
\curveto(640.13918916,548.99316604)(640.80526428,549.2685267)(641.62390727,549.26853277)
\curveto(642.41649626,549.2685267)(643.06396593,548.99874768)(643.56631821,548.45919487)
\curveto(644.06866024,547.91963157)(644.31983382,547.1605292)(644.3198397,546.18188549)
\curveto(644.31983382,546.12234508)(644.31797327,546.03303892)(644.31425806,545.91396674)
\lineto(639.89359867,545.91396674)
\curveto(639.93080812,545.26277328)(640.11500207,544.76414721)(640.4461811,544.41808705)
\curveto(640.7773561,544.07202447)(641.19039709,543.89899378)(641.68530532,543.89899447)
\curveto(642.05368998,543.89899378)(642.36812209,543.99574212)(642.62860259,544.18923979)
\curveto(642.8890747,544.38273548)(643.09559519,544.69158596)(643.2481647,545.11579213)
\closepath
\moveto(639.94941508,546.74004956)
\lineto(643.25932798,546.74004956)
\curveto(643.21467007,547.23867209)(643.08815301,547.61264164)(642.87977641,547.86195933)
\curveto(642.55975823,548.24894803)(642.14485669,548.44244472)(641.63507055,548.44244995)
\curveto(641.17365219,548.44244472)(640.78572855,548.28801948)(640.47129848,547.97917378)
\curveto(640.15686434,547.67031853)(639.98290338,547.25727754)(639.94941508,546.74004956)
\closepath
}
}
{
\newrgbcolor{curcolor}{0 0 0}
\pscustom[linestyle=none,fillstyle=solid,fillcolor=curcolor]
{
\newpath
\moveto(649.42145654,543.93806596)
\curveto(649.04934292,543.62177257)(648.691188,543.39850717)(648.34699072,543.26826908)
\curveto(648.00278634,543.13803087)(647.63346816,543.07291179)(647.23903505,543.07291166)
\curveto(646.58784186,543.07291179)(646.08735525,543.23198839)(645.73757372,543.55014193)
\curveto(645.38779032,543.86829479)(645.21289909,544.27482387)(645.2128995,544.76973041)
\curveto(645.21289909,545.05997387)(645.27894844,545.32510154)(645.41104774,545.5651142)
\curveto(645.54314583,545.80512215)(645.71617652,545.99768856)(645.93014032,546.14281401)
\curveto(646.14410187,546.28793358)(646.38504245,546.39770574)(646.65296279,546.47213081)
\curveto(646.8501787,546.5242228)(647.14786591,546.57445752)(647.54602529,546.6228351)
\curveto(648.35722017,546.71958003)(648.95445512,546.83493382)(649.33773193,546.96889682)
\curveto(649.34144848,547.10657339)(649.34330903,547.19401901)(649.34331357,547.23123393)
\curveto(649.34330903,547.64054981)(649.24842123,547.92893429)(649.0586499,548.09638823)
\curveto(648.80189043,548.32336984)(648.4204787,548.43686308)(647.91441357,548.43686831)
\curveto(647.44183202,548.43686308)(647.09297983,548.35406883)(646.86785595,548.1884853)
\curveto(646.64272793,548.02289181)(646.47620916,547.72985597)(646.36829911,547.3093769)
\lineto(645.38593036,547.44333628)
\curveto(645.47523594,547.86381522)(645.62221899,548.20336468)(645.82687997,548.46198569)
\curveto(646.0315389,548.7205962)(646.32736555,548.91967451)(646.71436083,549.05922124)
\curveto(647.10135228,549.19875627)(647.54974363,549.2685267)(648.05953623,549.26853277)
\curveto(648.56560121,549.2685267)(648.97678166,549.20898926)(649.29307881,549.08992027)
\curveto(649.60936697,548.9708395)(649.84193509,548.82106563)(649.99078389,548.64059819)
\curveto(650.1396223,548.46011989)(650.24381282,548.23220313)(650.30335576,547.95684722)
\curveto(650.33684007,547.78567232)(650.35358497,547.47682185)(650.35359053,547.03029487)
\lineto(650.35359053,545.69070112)
\curveto(650.35358497,544.75670503)(650.37498124,544.16598199)(650.4177794,543.91853021)
\curveto(650.46056631,543.67107702)(650.54522111,543.43385753)(650.67174404,543.20687103)
\lineto(649.6223956,543.20687103)
\curveto(649.51820026,543.41525208)(649.45122064,543.65898347)(649.42145654,543.93806596)
\closepath
\moveto(649.33773193,546.18188549)
\curveto(648.97306057,546.03303892)(648.42606034,545.90652186)(647.69672958,545.80233393)
\curveto(647.28368569,545.74279389)(646.99158013,545.67581427)(646.820412,545.60139487)
\curveto(646.64923984,545.52697067)(646.51714115,545.41812879)(646.42411552,545.27486889)
\curveto(646.33108664,545.13160486)(646.28457302,544.97252826)(646.2845745,544.79763861)
\curveto(646.28457302,544.52971854)(646.38597272,544.30645314)(646.58877392,544.12784174)
\curveto(646.79157154,543.9492285)(647.08832847,543.85992233)(647.4790456,543.85992299)
\curveto(647.86603628,543.85992233)(648.21023711,543.94457713)(648.51164912,544.11388764)
\curveto(648.8130537,544.28319633)(649.03445856,544.51483418)(649.17586435,544.8088019)
\curveto(649.28377159,545.03578679)(649.33772739,545.37068489)(649.33773193,545.81349721)
\closepath
}
}
{
\newrgbcolor{curcolor}{0 0 0}
\pscustom[linestyle=none,fillstyle=solid,fillcolor=curcolor]
{
\newpath
\moveto(651.51456908,544.97625112)
\lineto(652.50810111,545.13253705)
\curveto(652.56391612,544.73437849)(652.71927163,544.42924911)(652.97416811,544.21714799)
\curveto(653.22906096,544.00504485)(653.58535533,543.89899378)(654.04305229,543.89899447)
\curveto(654.50446457,543.89899378)(654.84680486,543.9929513)(655.07007417,544.18086732)
\curveto(655.29333566,544.3687814)(655.40496836,544.58925598)(655.40497261,544.84229174)
\curveto(655.40496836,545.0692766)(655.30635948,545.24788892)(655.10914565,545.37812924)
\curveto(654.97146137,545.46743323)(654.62912109,545.58092648)(654.08212377,545.71860932)
\curveto(653.34534503,545.90466131)(652.83462542,546.06559845)(652.54996342,546.20142124)
\curveto(652.26529864,546.33723803)(652.04947542,546.52515307)(651.9024931,546.76516694)
\curveto(651.75550931,547.00517369)(651.68201778,547.27030135)(651.6820183,547.56055073)
\curveto(651.68201778,547.82474377)(651.74248549,548.06940544)(651.86342162,548.29453647)
\curveto(651.98435635,548.51965733)(652.14901458,548.70664211)(652.35739682,548.85549136)
\curveto(652.5136814,548.9708395)(652.72671381,549.06851812)(652.99649467,549.14852749)
\curveto(653.26627186,549.22852499)(653.55558661,549.2685267)(653.86443979,549.26853277)
\curveto(654.32957334,549.2685267)(654.73796297,549.20154708)(655.08960991,549.0675937)
\curveto(655.44124899,548.9336286)(655.70079502,548.75222546)(655.86824878,548.52338374)
\curveto(656.03569312,548.29453139)(656.15104692,547.98847173)(656.2143105,547.60520386)
\lineto(655.23194175,547.47124448)
\curveto(655.1872846,547.7763696)(655.05797672,548.01451936)(654.84401772,548.18569448)
\curveto(654.63005136,548.35685965)(654.3277128,548.44244472)(653.93700112,548.44244995)
\curveto(653.47558318,548.44244472)(653.14626671,548.36616237)(652.94905072,548.21360269)
\curveto(652.75183117,548.06103299)(652.65322228,547.88242067)(652.65322377,547.67776518)
\curveto(652.65322228,547.54752256)(652.69415427,547.43030823)(652.77601986,547.32612182)
\curveto(652.85788223,547.21820609)(652.98625984,547.12889993)(653.16115307,547.05820307)
\curveto(653.2616205,547.02098832)(653.55744716,546.93540325)(654.04863393,546.8014476)
\curveto(654.75935924,546.61166842)(655.25519449,546.45631291)(655.53614116,546.33538061)
\curveto(655.81707909,546.21444206)(656.03755367,546.03862055)(656.19756558,545.80791557)
\curveto(656.35756741,545.57720539)(656.43757085,545.29068145)(656.43757612,544.94834291)
\curveto(656.43757085,544.61344307)(656.33989224,544.29808069)(656.14453999,544.00225482)
\curveto(655.94917778,543.70642737)(655.66730521,543.47758033)(655.29892143,543.31571303)
\curveto(654.93052938,543.1538455)(654.5137673,543.07291179)(654.04863393,543.07291166)
\curveto(653.27836541,543.07291179)(652.69136345,543.23291866)(652.28762631,543.55293275)
\curveto(651.88388692,543.87294615)(651.62620143,544.34738513)(651.51456908,544.97625112)
\closepath
}
}
{
\newrgbcolor{curcolor}{0 0 0}
\pscustom[linestyle=none,fillstyle=solid,fillcolor=curcolor]
{
\newpath
\moveto(661.68990664,545.11579213)
\lineto(662.7280918,544.9874144)
\curveto(662.56435799,544.38087494)(662.26108915,543.91015705)(661.81828437,543.57525932)
\curveto(661.37546972,543.24036084)(660.80986403,543.07291179)(660.12146562,543.07291166)
\curveto(659.2544484,543.07291179)(658.56697701,543.3399)(658.0590494,543.87387709)
\curveto(657.55111943,544.40785284)(657.29715504,545.15672221)(657.29715546,546.12048745)
\curveto(657.29715504,547.11773666)(657.55391025,547.89172339)(658.06742187,548.44244995)
\curveto(658.5809311,548.99316604)(659.24700622,549.2685267)(660.06564921,549.26853277)
\curveto(660.85823821,549.2685267)(661.50570787,548.99874768)(662.00806016,548.45919487)
\curveto(662.51040218,547.91963157)(662.76157576,547.1605292)(662.76158164,546.18188549)
\curveto(662.76157576,546.12234508)(662.75971521,546.03303892)(662.756,545.91396674)
\lineto(658.33534062,545.91396674)
\curveto(658.37255006,545.26277328)(658.55674402,544.76414721)(658.88792304,544.41808705)
\curveto(659.21909804,544.07202447)(659.63213904,543.89899378)(660.12704726,543.89899447)
\curveto(660.49543193,543.89899378)(660.80986403,543.99574212)(661.07034453,544.18923979)
\curveto(661.33081664,544.38273548)(661.53733714,544.69158596)(661.68990664,545.11579213)
\closepath
\moveto(658.39115702,546.74004956)
\lineto(661.70106992,546.74004956)
\curveto(661.65641202,547.23867209)(661.52989496,547.61264164)(661.32151836,547.86195933)
\curveto(661.00150017,548.24894803)(660.58659863,548.44244472)(660.0768125,548.44244995)
\curveto(659.61539413,548.44244472)(659.2274705,548.28801948)(658.91304042,547.97917378)
\curveto(658.59860628,547.67031853)(658.42464532,547.25727754)(658.39115702,546.74004956)
\closepath
}
}
{
\newrgbcolor{curcolor}{0 0 0}
\pscustom[linestyle=none,fillstyle=solid,fillcolor=curcolor]
{
\newpath
\moveto(672.16664354,544.17249486)
\lineto(672.16664354,543.20687103)
\lineto(666.75803377,543.20687103)
\curveto(666.75059124,543.44874189)(666.78966268,543.68131001)(666.87524822,543.90457611)
\curveto(667.01292809,544.27296333)(667.23340267,544.63576961)(667.53667263,544.99299604)
\curveto(667.83994035,545.35021889)(668.2780987,545.76325989)(668.851149,546.23212026)
\curveto(669.74048708,546.96145088)(670.34144313,547.53915011)(670.65401893,547.96521968)
\curveto(670.96658625,548.39127973)(671.12287203,548.79408772)(671.12287674,549.17364488)
\curveto(671.12287203,549.57179554)(670.98054034,549.90762392)(670.69588124,550.18113101)
\curveto(670.41121356,550.45462415)(670.04003483,550.59137421)(669.58234393,550.5913816)
\curveto(669.09859905,550.59137421)(668.71160569,550.4462517)(668.42136268,550.15601363)
\curveto(668.13111564,549.86576166)(667.98413259,549.46388393)(667.98041307,548.95037925)
\lineto(666.94780955,549.05643042)
\curveto(667.01850972,549.82669021)(667.28456766,550.41369216)(667.74598416,550.81743804)
\curveto(668.20739799,551.2211687)(668.82695948,551.42303784)(669.60467049,551.42304605)
\curveto(670.3898173,551.42303784)(671.01123933,551.20535407)(671.46893846,550.7699941)
\curveto(671.92662748,550.334619)(672.15547452,549.79506095)(672.15548026,549.15131831)
\curveto(672.15547452,548.82385645)(672.0884949,548.50198216)(671.9545412,548.18569448)
\curveto(671.82057642,547.86939685)(671.59824129,547.53635929)(671.28753514,547.18658081)
\curveto(670.97681925,546.83679436)(670.46051801,546.35677375)(669.73862987,545.74651752)
\curveto(669.13580995,545.24044674)(668.74881659,544.89717618)(668.57764861,544.71670483)
\curveto(668.40647631,544.53623045)(668.26507488,544.35482731)(668.15344393,544.17249486)
\closepath
}
}
{
\newrgbcolor{curcolor}{0 0 0}
\pscustom[linestyle=none,fillstyle=solid,fillcolor=curcolor]
{
\newpath
\moveto(331.71957461,505.72219175)
\curveto(331.71957406,507.08038562)(332.08424088,508.1436871)(332.81357618,508.91209937)
\curveto(333.54290818,509.68049729)(334.48434396,510.06469983)(335.63788634,510.06470816)
\curveto(336.39326315,510.06469983)(337.07422262,509.88422697)(337.68076681,509.52328902)
\curveto(338.28729797,509.1623355)(338.74964341,508.65905807)(339.06780451,508.01345523)
\curveto(339.38594981,507.36783983)(339.54502641,506.63571536)(339.54503479,505.81707964)
\curveto(339.54502641,504.98727248)(339.37757736,504.24491502)(339.04268713,503.59000502)
\curveto(338.70778115,502.93509132)(338.23334217,502.43925607)(337.61936877,502.10249779)
\curveto(337.00538246,501.76573878)(336.34302843,501.59735945)(335.6323047,501.59735931)
\curveto(334.8620346,501.59735945)(334.17363294,501.78341396)(333.56709766,502.15552338)
\curveto(332.96055759,502.52763196)(332.50100297,503.03556075)(332.18843242,503.67931127)
\curveto(331.87585984,504.32305791)(331.71957406,505.00401738)(331.71957461,505.72219175)
\closepath
\moveto(332.83590274,505.70544682)
\curveto(332.83590107,504.719354)(333.10102873,503.94257645)(333.63128653,503.37511186)
\curveto(334.1615394,502.80764399)(334.82668424,502.52391087)(335.62672306,502.52391166)
\curveto(336.44163732,502.52391087)(337.1123638,502.81043481)(337.63890451,503.38348432)
\curveto(338.16543228,503.95653054)(338.4286994,504.76958871)(338.42870666,505.82266128)
\curveto(338.4286994,506.48873231)(338.31613642,507.07015263)(338.0910174,507.56692398)
\curveto(337.86588453,508.06368367)(337.53656806,508.44881649)(337.10306701,508.72232359)
\curveto(336.66955408,508.99581672)(336.18302156,509.13256678)(335.64346798,509.13257418)
\curveto(334.87691896,509.13256678)(334.21735575,508.86929966)(333.66477637,508.34277203)
\curveto(333.112192,507.81623118)(332.83590107,506.93712366)(332.83590274,505.70544682)
\closepath
}
}
{
\newrgbcolor{curcolor}{0 0 0}
\pscustom[linestyle=none,fillstyle=solid,fillcolor=curcolor]
{
\newpath
\moveto(340.8064857,499.46517259)
\lineto(340.8064857,507.66460269)
\lineto(341.72187476,507.66460269)
\lineto(341.72187476,506.89433628)
\curveto(341.93769631,507.19573942)(342.18142771,507.42179564)(342.45306969,507.57250562)
\curveto(342.72470686,507.72320393)(343.05402333,507.798556)(343.44102008,507.79856206)
\curveto(343.94708494,507.798556)(344.39361574,507.66831785)(344.78061383,507.40784722)
\curveto(345.16760247,507.14736524)(345.45970804,506.7799076)(345.65693141,506.30547319)
\curveto(345.85414358,505.83102964)(345.95275247,505.31100731)(345.95275837,504.74540463)
\curveto(345.95275247,504.13886395)(345.84391058,503.59279399)(345.62623239,503.10719311)
\curveto(345.40854305,502.62158949)(345.09225039,502.24948048)(344.67735348,501.99086498)
\curveto(344.26244732,501.73224897)(343.82614951,501.60294109)(343.36845875,501.60294095)
\curveto(343.03355733,501.60294109)(342.73307931,501.6736418)(342.46702379,501.8150433)
\curveto(342.20096343,501.95644464)(341.9823494,502.13505696)(341.81118101,502.3508808)
\lineto(341.81118101,499.46517259)
\closepath
\moveto(341.71629312,504.66726167)
\curveto(341.71629146,503.90443528)(341.87071669,503.34069014)(342.17956929,502.97602455)
\curveto(342.48841764,502.61135649)(342.86238719,502.42902308)(343.30147906,502.42902377)
\curveto(343.74800662,502.42902308)(344.13034862,502.6178684)(344.44850621,502.99556029)
\curveto(344.76665502,503.37324967)(344.92573162,503.95839108)(344.92573649,504.75098627)
\curveto(344.92573162,505.50636454)(344.77037611,506.07197022)(344.45966949,506.44780503)
\curveto(344.14895407,506.82363041)(343.77777534,507.01154546)(343.34613219,507.01155073)
\curveto(342.91820354,507.01154546)(342.53958263,506.81153687)(342.21026832,506.41152436)
\curveto(341.88094969,506.01150251)(341.71629146,505.43008219)(341.71629312,504.66726167)
\closepath
}
}
{
\newrgbcolor{curcolor}{0 0 0}
\pscustom[linestyle=none,fillstyle=solid,fillcolor=curcolor]
{
\newpath
\moveto(351.22740893,503.64582143)
\lineto(352.26559409,503.51744369)
\curveto(352.10186028,502.91090424)(351.79859144,502.44018635)(351.35578666,502.10528861)
\curveto(350.91297201,501.77039014)(350.34736632,501.60294109)(349.65896791,501.60294095)
\curveto(348.79195069,501.60294109)(348.1044793,501.8699293)(347.59655169,502.40390639)
\curveto(347.08862172,502.93788214)(346.83465733,503.68675151)(346.83465775,504.65051674)
\curveto(346.83465733,505.64776596)(347.09141254,506.42175269)(347.60492415,506.97247925)
\curveto(348.11843339,507.52319534)(348.78450851,507.798556)(349.6031515,507.79856206)
\curveto(350.39574049,507.798556)(351.04321016,507.52877697)(351.54556245,506.98922417)
\curveto(352.04790447,506.44966086)(352.29907805,505.6905585)(352.29908393,504.71191479)
\curveto(352.29907805,504.65237437)(352.2972175,504.56306821)(352.29350229,504.44399604)
\lineto(347.87284291,504.44399604)
\curveto(347.91005235,503.79280258)(348.09424631,503.29417651)(348.42542533,502.94811635)
\curveto(348.75660033,502.60205376)(349.16964133,502.42902308)(349.66454955,502.42902377)
\curveto(350.03293422,502.42902308)(350.34736632,502.52577142)(350.60784682,502.71926908)
\curveto(350.86831893,502.91276478)(351.07483943,503.22161525)(351.22740893,503.64582143)
\closepath
\moveto(347.92865931,505.27007885)
\lineto(351.23857221,505.27007885)
\curveto(351.19391431,505.76870139)(351.06739725,506.14267093)(350.85902065,506.39198862)
\curveto(350.53900246,506.77897733)(350.12410092,506.97247401)(349.61431478,506.97247925)
\curveto(349.15289642,506.97247401)(348.76497278,506.81804878)(348.45054271,506.50920308)
\curveto(348.13610857,506.20034783)(347.96214761,505.78730684)(347.92865931,505.27007885)
\closepath
}
}
{
\newrgbcolor{curcolor}{0 0 0}
\pscustom[linestyle=none,fillstyle=solid,fillcolor=curcolor]
{
\newpath
\moveto(353.53262763,501.73690033)
\lineto(353.53262763,507.66460269)
\lineto(354.43685341,507.66460269)
\lineto(354.43685341,506.82177495)
\curveto(354.87221929,507.47296062)(355.5010835,507.798556)(356.32344795,507.79856206)
\curveto(356.68066904,507.798556)(357.00905524,507.7343672)(357.30860752,507.60599546)
\curveto(357.60815074,507.47761199)(357.83234641,507.30923266)(357.98119522,507.10085698)
\curveto(358.13003361,506.89247058)(358.23422414,506.64501809)(358.29376709,506.35849878)
\curveto(358.33097248,506.17243965)(358.34957793,505.84684428)(358.3495835,505.38171167)
\lineto(358.3495835,501.73690033)
\lineto(357.34488818,501.73690033)
\lineto(357.34488818,505.34264018)
\curveto(357.34488362,505.75195648)(357.30581217,506.05801614)(357.22767373,506.26082007)
\curveto(357.14952639,506.46361495)(357.01091579,506.62548237)(356.8118415,506.7464228)
\curveto(356.61275915,506.86735322)(356.37926075,506.92782093)(356.1113456,506.92782612)
\curveto(355.68341691,506.92782093)(355.31409873,506.79200115)(355.00338993,506.52036636)
\curveto(354.69267669,506.248722)(354.53732118,505.73335103)(354.53732294,504.9742519)
\lineto(354.53732294,501.73690033)
\closepath
}
}
{
\newrgbcolor{curcolor}{0 0 0}
\pscustom[linestyle=none,fillstyle=solid,fillcolor=curcolor]
{
\newpath
\moveto(359.97942177,501.73690033)
\lineto(359.97942177,509.91958551)
\lineto(363.04932412,509.91958551)
\curveto(363.67446334,509.91957732)(364.17588022,509.83678307)(364.55357627,509.6712025)
\curveto(364.9312615,509.50560606)(365.22708816,509.25071139)(365.44105713,508.90651773)
\curveto(365.65501351,508.56230973)(365.76199485,508.20229427)(365.76200147,507.82647027)
\curveto(365.76199485,507.47668171)(365.66710706,507.14736524)(365.4773378,506.83851987)
\curveto(365.28755587,506.5296643)(365.00103194,506.28035127)(364.61776514,506.09058003)
\curveto(365.11266464,505.94545316)(365.4931461,505.69800068)(365.75921065,505.34822182)
\curveto(366.02526197,504.99843575)(366.15829094,504.58539475)(366.15829796,504.1090976)
\curveto(366.15829094,503.72582296)(366.07735723,503.36952858)(365.91549659,503.04021342)
\curveto(365.7536224,502.71089565)(365.55361381,502.45693125)(365.31547022,502.27831947)
\curveto(365.07731428,502.09970661)(364.77869681,501.96481709)(364.4196169,501.87365053)
\curveto(364.06052643,501.78248368)(363.62050753,501.73690033)(363.09955889,501.73690033)
\closepath
\moveto(361.06226005,506.48129487)
\lineto(362.83164014,506.48129487)
\curveto(363.31165706,506.48129013)(363.65585789,506.51291939)(363.86424365,506.57618276)
\curveto(364.1395996,506.6580419)(364.34705036,506.79386169)(364.48659659,506.98364253)
\curveto(364.62613212,507.17341288)(364.69590256,507.41156264)(364.69590811,507.69809253)
\curveto(364.69590256,507.96972614)(364.63078348,508.20880618)(364.50055069,508.41533335)
\curveto(364.37030718,508.62184717)(364.18425268,508.76324859)(363.94238662,508.83953804)
\curveto(363.70051097,508.91581329)(363.28560943,508.95395446)(362.69768076,508.95396168)
\lineto(361.06226005,508.95396168)
\closepath
\moveto(361.06226005,502.70252416)
\lineto(363.09955889,502.70252416)
\curveto(363.44933739,502.70252319)(363.69492934,502.71554701)(363.83633545,502.74159564)
\curveto(364.08564379,502.78624772)(364.29402483,502.86066952)(364.4614792,502.96486127)
\curveto(364.62892294,503.06905056)(364.76660327,503.22068498)(364.87452061,503.41976498)
\curveto(364.98242649,503.61884162)(365.03638229,503.84861893)(365.03638819,504.1090976)
\curveto(365.03638229,504.41422461)(364.9582394,504.67935228)(364.80195928,504.90448139)
\curveto(364.64566784,505.12960417)(364.42891435,505.2877505)(364.15169815,505.37892085)
\curveto(363.87447193,505.47008391)(363.47538502,505.51566726)(362.95443623,505.51567104)
\lineto(361.06226005,505.51567104)
\closepath
}
}
{
\newrgbcolor{curcolor}{0 0 0}
\pscustom[linestyle=none,fillstyle=solid,fillcolor=curcolor]
{
\newpath
\moveto(367.29136967,504.36585307)
\lineto(368.31280991,504.45515932)
\curveto(368.36118255,504.0458367)(368.47374552,503.71000832)(368.65049917,503.44767319)
\curveto(368.82724907,503.18533463)(369.10167946,502.97323249)(369.47379116,502.81136615)
\curveto(369.84589747,502.64949766)(370.2645201,502.56856395)(370.72966031,502.56856478)
\curveto(371.14269735,502.56856395)(371.50736417,502.62996194)(371.82366187,502.75275893)
\curveto(372.13994948,502.87555388)(372.37530842,503.04393321)(372.52973941,503.2578974)
\curveto(372.6841589,503.47185856)(372.76137151,503.70535696)(372.7613775,503.9583933)
\curveto(372.76137151,504.2151463)(372.68694971,504.43934197)(372.53811187,504.630981)
\curveto(372.38926251,504.82261424)(372.14367057,504.98355139)(371.80133531,505.11379292)
\curveto(371.58178597,505.19937461)(371.09618372,505.33240358)(370.3445271,505.51288022)
\curveto(369.59286335,505.69334931)(369.06632911,505.86358918)(368.7649228,506.02360034)
\curveto(368.37420636,506.22825601)(368.08303106,506.4822204)(367.89139604,506.78549429)
\curveto(367.69975879,507.08875808)(367.60394072,507.42830754)(367.60394155,507.8041437)
\curveto(367.60394072,508.21717863)(367.72115506,508.60324172)(367.95558491,508.96233414)
\curveto(368.1900124,509.3214121)(368.53235269,509.59398194)(368.98260679,509.78004449)
\curveto(369.43285648,509.96609095)(369.93334309,510.0591182)(370.48406812,510.05912652)
\curveto(371.09060209,510.0591182)(371.62550878,509.96143959)(372.0887898,509.76609039)
\curveto(372.5520602,509.57072513)(372.90835457,509.28327093)(373.15767398,508.90372691)
\curveto(373.40698064,508.52416856)(373.54093988,508.09438266)(373.55955211,507.61436792)
\lineto(372.52136695,507.53622495)
\curveto(372.46554486,508.05345067)(372.27669954,508.44416512)(371.95483043,508.70836949)
\curveto(371.63295096,508.97255991)(371.15758171,509.1046586)(370.52872124,509.10466597)
\curveto(369.87380565,509.1046586)(369.39657585,508.98465345)(369.09703042,508.74465015)
\curveto(368.79748035,508.50463284)(368.64770648,508.21531809)(368.64770835,507.87670503)
\curveto(368.64770648,507.58273278)(368.75375754,507.34086193)(368.96586186,507.15109175)
\curveto(369.17424072,506.96131074)(369.71845014,506.76688379)(370.59849175,506.5678103)
\curveto(371.47852572,506.36872715)(372.08227258,506.1947662)(372.40973414,506.0459269)
\curveto(372.88602803,505.82637828)(373.23767104,505.5482268)(373.46466422,505.21147163)
\curveto(373.69164402,504.8747095)(373.80513727,504.48678587)(373.8051443,504.04769955)
\curveto(373.80513727,503.61232971)(373.68048075,503.20207953)(373.43117438,502.81694779)
\curveto(373.18185469,502.43181389)(372.82369977,502.13226615)(372.35670855,501.91830365)
\curveto(371.88970617,501.70434079)(371.36410221,501.59735945)(370.77989507,501.59735931)
\curveto(370.03939415,501.59735945)(369.41890239,501.70527106)(368.91841792,501.92109447)
\curveto(368.41792917,502.13691751)(368.02535417,502.46158261)(367.74069174,502.89509076)
\curveto(367.45602739,503.32859659)(367.30625352,503.81885021)(367.29136967,504.36585307)
\closepath
}
}
{
\newrgbcolor{curcolor}{0 0 0}
\pscustom[linestyle=none,fillstyle=solid,fillcolor=curcolor]
{
\newpath
\moveto(375.29544292,501.73690033)
\lineto(375.29544292,509.91958551)
\lineto(378.11417144,509.91958551)
\curveto(378.75047414,509.91957732)(379.23607639,509.88050588)(379.57097965,509.80237105)
\curveto(380.03983184,509.69445138)(380.43984902,509.49909415)(380.77103239,509.21629879)
\curveto(381.20267248,508.85162448)(381.52547704,508.38555795)(381.73944704,507.81809781)
\curveto(381.95340239,507.25062549)(382.06038373,506.60222555)(382.06039138,505.87289604)
\curveto(382.06038373,505.25146987)(381.98782247,504.70074855)(381.84270739,504.22073041)
\curveto(381.69757745,503.74070732)(381.51152295,503.34348095)(381.28454333,503.02905014)
\curveto(381.05754996,502.71461674)(380.8091672,502.46716425)(380.5393943,502.28669193)
\curveto(380.26960915,502.10621852)(379.94401377,501.96946846)(379.56260719,501.87644135)
\curveto(379.18119031,501.78341396)(378.74303196,501.73690033)(378.24813082,501.73690033)
\closepath
\moveto(376.37828121,502.70252416)
\lineto(378.12533473,502.70252416)
\curveto(378.66488907,502.70252319)(379.08816306,502.75275791)(379.39515797,502.85322846)
\curveto(379.70214292,502.95369677)(379.94680459,503.09509819)(380.12914371,503.27743315)
\curveto(380.38589321,503.53418682)(380.5859018,503.87931792)(380.72917008,504.31282748)
\curveto(380.87242573,504.7463319)(380.94405672,505.27193587)(380.94406325,505.88964096)
\curveto(380.94405672,506.74548752)(380.80358557,507.40319019)(380.52264938,507.86275093)
\curveto(380.24170097,508.32229942)(379.90029096,508.63021962)(379.49841832,508.78651246)
\curveto(379.20816821,508.89813811)(378.74117142,508.95395446)(378.09742652,508.95396168)
\lineto(376.37828121,508.95396168)
\closepath
}
}
{
\newrgbcolor{curcolor}{0 0 0}
\pscustom[linestyle=none,fillstyle=solid,fillcolor=curcolor]
{
\newpath
\moveto(391.59941713,502.70252416)
\lineto(391.59941713,501.73690033)
\lineto(386.19080736,501.73690033)
\curveto(386.18336483,501.97877118)(386.22243627,502.21133931)(386.30802181,502.43460541)
\curveto(386.44570168,502.80299263)(386.66617626,503.1657989)(386.96944622,503.52302533)
\curveto(387.27271394,503.88024819)(387.71087229,504.29328919)(388.28392259,504.76214956)
\curveto(389.17326067,505.49148018)(389.77421672,506.06917941)(390.08679252,506.49524897)
\curveto(390.39935984,506.92130903)(390.55564562,507.32411702)(390.55565033,507.70367417)
\curveto(390.55564562,508.10182484)(390.41331393,508.43765322)(390.12865483,508.71116031)
\curveto(389.84398715,508.98465345)(389.47280842,509.12140351)(389.01511752,509.12141089)
\curveto(388.53137264,509.12140351)(388.14437928,508.976281)(387.85413627,508.68604292)
\curveto(387.56388923,508.39579095)(387.41690618,507.99391323)(387.41318666,507.48040855)
\lineto(386.38058314,507.58645972)
\curveto(386.45128331,508.35671951)(386.71734125,508.94372146)(387.17875775,509.34746734)
\curveto(387.64017158,509.751198)(388.25973307,509.95306713)(389.03744408,509.95307535)
\curveto(389.82259089,509.95306713)(390.44401292,509.73538337)(390.90171205,509.30002339)
\curveto(391.35940107,508.8646483)(391.58824811,508.32509024)(391.58825385,507.68134761)
\curveto(391.58824811,507.35388574)(391.52126849,507.03201145)(391.38731479,506.71572378)
\curveto(391.25335001,506.39942615)(391.03101488,506.06638859)(390.72030873,505.71661011)
\curveto(390.40959284,505.36682366)(389.8932916,504.88680305)(389.17140346,504.27654682)
\curveto(388.56858354,503.77047604)(388.18159018,503.42720548)(388.0104222,503.24673412)
\curveto(387.8392499,503.06625975)(387.69784847,502.88485661)(387.58621752,502.70252416)
\closepath
}
}
{
\newrgbcolor{curcolor}{0 0 0}
\pscustom[linestyle=none,fillstyle=solid,fillcolor=curcolor]
{
\newpath
\moveto(393.24599842,501.73690033)
\lineto(393.24599842,502.88113666)
\lineto(394.39023475,502.88113666)
\lineto(394.39023475,501.73690033)
\closepath
}
}
{
\newrgbcolor{curcolor}{0 0 0}
\pscustom[linestyle=none,fillstyle=solid,fillcolor=curcolor]
{
\newpath
\moveto(395.85262854,505.77242651)
\curveto(395.85262807,506.73990589)(395.95216723,507.51854398)(396.15124632,508.10834312)
\curveto(396.35032386,508.69812952)(396.64615052,509.15303278)(397.03872718,509.47305425)
\curveto(397.43130052,509.79306026)(397.92527522,509.95306713)(398.52065277,509.95307535)
\curveto(398.95973825,509.95306713)(399.34487107,509.86469125)(399.67605238,509.68794742)
\curveto(400.00722509,509.51118769)(400.28072521,509.25629302)(400.49655355,508.92326265)
\curveto(400.71237166,508.59021791)(400.88168125,508.18461909)(401.00448285,507.70646499)
\curveto(401.1272732,507.22829895)(401.18867118,506.5836201)(401.18867699,505.77242651)
\curveto(401.18867118,504.81238125)(401.0900623,504.03746425)(400.89285004,503.44767319)
\curveto(400.69562675,502.8578787)(400.40073037,502.40204517)(400.00816,502.08017123)
\curveto(399.61558037,501.7582966)(399.11974512,501.59735945)(398.52065277,501.59735931)
\curveto(397.73177854,501.59735945)(397.11221704,501.8801623)(396.66196643,502.44576869)
\curveto(396.12240709,503.12672746)(395.85262807,504.23561229)(395.85262854,505.77242651)
\closepath
\moveto(396.88523206,505.77242651)
\curveto(396.88523055,504.42910897)(397.04244661,503.53511709)(397.35688069,503.09044818)
\curveto(397.67131082,502.64577657)(398.05923446,502.42344144)(398.52065277,502.42344213)
\curveto(398.98206479,502.42344144)(399.36998843,502.64670684)(399.68442484,503.09323901)
\curveto(399.99885264,503.53976845)(400.1560687,504.43283006)(400.15607347,505.77242651)
\curveto(400.1560687,507.11945707)(399.99885264,508.01437922)(399.68442484,508.45719566)
\curveto(399.36998843,508.89999865)(398.9783437,509.12140351)(398.50948949,509.12141089)
\curveto(398.04807119,509.12140351)(397.67968328,508.92604628)(397.40432464,508.53533863)
\curveto(397.05826124,508.03670576)(396.88523055,507.11573598)(396.88523206,505.77242651)
\closepath
}
}
{
\newrgbcolor{curcolor}{0 0 0}
\pscustom[linestyle=none,fillstyle=solid,fillcolor=curcolor]
{
\newpath
\moveto(725.0916466,501.50257299)
\lineto(725.0916466,500.53694916)
\lineto(719.68303682,500.53694916)
\curveto(719.67559429,500.77882001)(719.71466574,501.01138814)(719.80025127,501.23465424)
\curveto(719.93793114,501.60304145)(720.15840572,501.96584773)(720.46167569,502.32307416)
\curveto(720.7649434,502.68029702)(721.20310175,503.09333801)(721.77615206,503.56219838)
\curveto(722.66549014,504.29152901)(723.26644618,504.86922823)(723.57902198,505.2952978)
\curveto(723.8915893,505.72135785)(724.04787509,506.12416585)(724.0478798,506.503723)
\curveto(724.04787509,506.90187367)(723.90554339,507.23770204)(723.62088429,507.51120914)
\curveto(723.33621662,507.78470228)(722.96503788,507.92145234)(722.50734698,507.92145972)
\curveto(722.0236021,507.92145234)(721.63660874,507.77632983)(721.34636573,507.48609175)
\curveto(721.0561187,507.19583978)(720.90913564,506.79396206)(720.90541612,506.28045737)
\lineto(719.8728126,506.38650855)
\curveto(719.94351277,507.15676834)(720.20957071,507.74377029)(720.67098721,508.14751617)
\curveto(721.13240104,508.55124683)(721.75196253,508.75311596)(722.52967354,508.75312418)
\curveto(723.31482035,508.75311596)(723.93624238,508.53543219)(724.39394152,508.10007222)
\curveto(724.85163053,507.66469713)(725.08047757,507.12513907)(725.08048331,506.48139644)
\curveto(725.08047757,506.15393457)(725.01349795,505.83206028)(724.87954425,505.51577261)
\curveto(724.74557947,505.19947498)(724.52324434,504.86643742)(724.2125382,504.51665893)
\curveto(723.9018223,504.16687249)(723.38552106,503.68685188)(722.66363292,503.07659565)
\curveto(722.06081301,502.57052486)(721.67381964,502.22725431)(721.50265167,502.04678295)
\curveto(721.33147936,501.86630857)(721.19007794,501.68490543)(721.07844698,501.50257299)
\closepath
}
}
{
\newrgbcolor{curcolor}{0 0 0}
\pscustom[linestyle=none,fillstyle=solid,fillcolor=curcolor]
{
\newpath
\moveto(726.73823074,500.53694916)
\lineto(726.73823074,501.68118549)
\lineto(727.88246708,501.68118549)
\lineto(727.88246708,500.53694916)
\closepath
}
}
{
\newrgbcolor{curcolor}{0 0 0}
\pscustom[linestyle=none,fillstyle=solid,fillcolor=curcolor]
{
\newpath
\moveto(733.1292094,500.53694916)
\lineto(732.12451409,500.53694916)
\lineto(732.12451409,506.93909097)
\curveto(731.88263998,506.70837699)(731.56541705,506.4776694)(731.17284436,506.24696753)
\curveto(730.78026706,506.01625424)(730.42769378,505.84322355)(730.11512346,505.72787495)
\lineto(730.11512346,506.69908042)
\curveto(730.67700681,506.96327165)(731.16819069,507.2832854)(731.58867659,507.65912261)
\curveto(732.00915704,508.03494558)(732.30684424,508.39961241)(732.48173909,508.75312418)
\lineto(733.1292094,508.75312418)
\closepath
}
}
{
\newrgbcolor{curcolor}{0 0 0}
\pscustom[linestyle=none,fillstyle=solid,fillcolor=curcolor]
{
\newpath
\moveto(739.49227989,500.53694916)
\lineto(738.48758458,500.53694916)
\lineto(738.48758458,506.93909097)
\curveto(738.24571047,506.70837699)(737.92848754,506.4776694)(737.53591485,506.24696753)
\curveto(737.14333754,506.01625424)(736.79076426,505.84322355)(736.47819395,505.72787495)
\lineto(736.47819395,506.69908042)
\curveto(737.0400773,506.96327165)(737.53126118,507.2832854)(737.95174707,507.65912261)
\curveto(738.37222753,508.03494558)(738.66991473,508.39961241)(738.84480958,508.75312418)
\lineto(739.49227989,508.75312418)
\closepath
}
}
{
\newrgbcolor{curcolor}{0 0 0}
\pscustom[linestyle=none,fillstyle=solid,fillcolor=curcolor]
{
\newpath
\moveto(742.43380371,500.53694916)
\lineto(742.43380371,508.71963433)
\lineto(745.50370606,508.71963433)
\curveto(746.12884528,508.71962615)(746.63026217,508.6368319)(747.00795822,508.47125133)
\curveto(747.38564344,508.30565488)(747.6814701,508.05076022)(747.89543908,507.70656656)
\curveto(748.10939546,507.36235856)(748.21637679,507.0023431)(748.21638341,506.62651909)
\curveto(748.21637679,506.27673054)(748.121489,505.94741407)(747.93171974,505.6385687)
\curveto(747.74193781,505.32971313)(747.45541388,505.08040009)(747.07214708,504.89062886)
\curveto(747.56704658,504.74550199)(747.94752804,504.4980495)(748.21359259,504.14827065)
\curveto(748.47964391,503.79848458)(748.61267288,503.38544358)(748.6126799,502.90914643)
\curveto(748.61267288,502.52587178)(748.53173918,502.16957741)(748.36987853,501.84026225)
\curveto(748.20800434,501.51094448)(748.00799575,501.25698008)(747.76985216,501.0783683)
\curveto(747.53169623,500.89975544)(747.23307875,500.76486592)(746.87399884,500.67369935)
\curveto(746.51490837,500.58253251)(746.07488948,500.53694916)(745.55394083,500.53694916)
\closepath
\moveto(743.516642,505.2813437)
\lineto(745.28602208,505.2813437)
\curveto(745.766039,505.28133896)(746.11023983,505.31296822)(746.3186256,505.37623159)
\curveto(746.59398154,505.45809073)(746.80143231,505.59391052)(746.94097853,505.78369136)
\curveto(747.08051406,505.9734617)(747.1502845,506.21161147)(747.15029005,506.49814136)
\curveto(747.1502845,506.76977497)(747.08516542,507.00885501)(746.95493263,507.21538218)
\curveto(746.82468912,507.421896)(746.63863462,507.56329742)(746.39676857,507.63958687)
\curveto(746.15489291,507.71586211)(745.73999137,507.75400329)(745.1520627,507.7540105)
\lineto(743.516642,507.7540105)
\closepath
\moveto(743.516642,501.50257299)
\lineto(745.55394083,501.50257299)
\curveto(745.90371934,501.50257202)(746.14931128,501.51559584)(746.29071739,501.54164447)
\curveto(746.54002573,501.58629655)(746.74840677,501.66071835)(746.91586115,501.7649101)
\curveto(747.08330488,501.86909939)(747.22098521,502.02073381)(747.32890255,502.21981381)
\curveto(747.43680843,502.41889044)(747.49076424,502.64866775)(747.49077013,502.90914643)
\curveto(747.49076424,503.21427344)(747.41262135,503.47940111)(747.25634122,503.70453022)
\curveto(747.10004978,503.929653)(746.88329629,504.08779933)(746.60608009,504.17896967)
\curveto(746.32885387,504.27013274)(745.92976697,504.31571609)(745.40881817,504.31571987)
\lineto(743.516642,504.31571987)
\closepath
}
}
{
\newrgbcolor{curcolor}{0 0 0}
\pscustom[linestyle=none,fillstyle=solid,fillcolor=curcolor]
{
\newpath
\moveto(749.74575352,503.1659019)
\lineto(750.76719376,503.25520815)
\curveto(750.8155664,502.84588553)(750.92812937,502.51005715)(751.10488302,502.24772201)
\curveto(751.28163292,501.98538346)(751.55606331,501.77328132)(751.92817501,501.61141498)
\curveto(752.30028132,501.44954649)(752.71890395,501.36861278)(753.18404416,501.36861361)
\curveto(753.5970812,501.36861278)(753.96174802,501.43001077)(754.27804572,501.55280775)
\curveto(754.59433333,501.67560271)(754.82969227,501.84398203)(754.98412326,502.05794623)
\curveto(755.13854275,502.27190739)(755.21575537,502.50540579)(755.21576135,502.75844213)
\curveto(755.21575537,503.01519512)(755.14133356,503.2393908)(754.99249572,503.43102983)
\curveto(754.84364636,503.62266307)(754.59805442,503.78360022)(754.25571916,503.91384174)
\curveto(754.03616982,503.99942344)(753.55056757,504.13245241)(752.79891095,504.31292905)
\curveto(752.0472472,504.49339814)(751.52071296,504.66363801)(751.21930665,504.82364917)
\curveto(750.82859021,505.02830483)(750.53741492,505.28226923)(750.34577989,505.58554312)
\curveto(750.15414264,505.8888069)(750.05832457,506.22835637)(750.0583254,506.60419253)
\curveto(750.05832457,507.01722746)(750.17553891,507.40329055)(750.40996876,507.76238296)
\curveto(750.64439625,508.12146093)(750.98673654,508.39403077)(751.43699064,508.58009332)
\curveto(751.88724033,508.76613978)(752.38772694,508.85916703)(752.93845197,508.85917535)
\curveto(753.54498594,508.85916703)(754.07989263,508.76148841)(754.54317365,508.56613922)
\curveto(755.00644405,508.37077396)(755.36273842,508.08331975)(755.61205783,507.70377574)
\curveto(755.86136449,507.32421739)(755.99532373,506.89443149)(756.01393596,506.41441675)
\lineto(754.9757508,506.33627378)
\curveto(754.91992871,506.8534995)(754.73108339,507.24421395)(754.40921428,507.50841832)
\curveto(754.08733481,507.77260874)(753.61196556,507.90470743)(752.98310509,507.9047148)
\curveto(752.3281895,507.90470743)(751.8509597,507.78470228)(751.55141427,507.54469898)
\curveto(751.2518642,507.30468166)(751.10209033,507.01536691)(751.1020922,506.67675386)
\curveto(751.10209033,506.38278161)(751.20814139,506.14091075)(751.42024571,505.95114058)
\curveto(751.62862457,505.76135957)(752.17283399,505.56693262)(753.0528756,505.36785913)
\curveto(753.93290957,505.16877598)(754.53665643,504.99481502)(754.86411799,504.84597573)
\curveto(755.34041188,504.62642711)(755.69205489,504.34827563)(755.91904807,504.01152046)
\curveto(756.14602787,503.67475833)(756.25952112,503.2868347)(756.25952815,502.84774838)
\curveto(756.25952112,502.41237854)(756.1348646,502.00212836)(755.88555823,501.61699662)
\curveto(755.63623854,501.23186272)(755.27808362,500.93231497)(754.8110924,500.71835248)
\curveto(754.34409002,500.50438962)(753.81848606,500.39740828)(753.23427892,500.39740814)
\curveto(752.493778,500.39740828)(751.87328624,500.50531989)(751.37280177,500.7211433)
\curveto(750.87231302,500.93696634)(750.47973802,501.26163144)(750.19507559,501.69513959)
\curveto(749.91041124,502.12864542)(749.76063737,502.61889903)(749.74575352,503.1659019)
\closepath
}
}
{
\newrgbcolor{curcolor}{0 0 0}
\pscustom[linestyle=none,fillstyle=solid,fillcolor=curcolor]
{
\newpath
\moveto(757.74982677,500.53694916)
\lineto(757.74982677,508.71963433)
\lineto(760.56855529,508.71963433)
\curveto(761.20485799,508.71962615)(761.69046024,508.68055471)(762.0253635,508.60241988)
\curveto(762.49421569,508.4945002)(762.89423287,508.29914298)(763.22541624,508.01634761)
\curveto(763.65705633,507.65167331)(763.97986089,507.18560678)(764.19383089,506.61814663)
\curveto(764.40778624,506.05067432)(764.51476758,505.40227438)(764.51477523,504.67294487)
\curveto(764.51476758,504.0515187)(764.44220632,503.50079737)(764.29709124,503.02077924)
\curveto(764.1519613,502.54075614)(763.9659068,502.14352978)(763.73892718,501.82909897)
\curveto(763.51193381,501.51466557)(763.26355105,501.26721308)(762.99377815,501.08674076)
\curveto(762.723993,500.90626734)(762.39839762,500.76951729)(762.01699104,500.67649017)
\curveto(761.63557416,500.58346278)(761.19741581,500.53694916)(760.70251467,500.53694916)
\closepath
\moveto(758.83266506,501.50257299)
\lineto(760.57971858,501.50257299)
\curveto(761.11927292,501.50257202)(761.54254691,501.55280674)(761.84954182,501.65327729)
\curveto(762.15652677,501.7537456)(762.40118844,501.89514702)(762.58352757,502.07748197)
\curveto(762.84027706,502.33423565)(763.04028565,502.67936675)(763.18355393,503.11287631)
\curveto(763.32680958,503.54638073)(763.39844057,504.07198469)(763.3984471,504.68968979)
\curveto(763.39844057,505.54553635)(763.25796942,506.20323901)(762.97703323,506.66279976)
\curveto(762.69608482,507.12234825)(762.35467481,507.43026845)(761.95280217,507.58656128)
\curveto(761.66255207,507.69818694)(761.19555527,507.75400329)(760.55181037,507.7540105)
\lineto(758.83266506,507.7540105)
\closepath
}
}
{
\newrgbcolor{curcolor}{0 0 0}
\pscustom[linestyle=none,fillstyle=solid,fillcolor=curcolor]
{
\newpath
\moveto(769.18102489,500.53694916)
\lineto(769.18102489,508.71963433)
\lineto(772.26767217,508.71963433)
\curveto(772.81094734,508.71962615)(773.22584888,508.69357852)(773.51237803,508.64149136)
\curveto(773.91425054,508.57450364)(774.25100919,508.44705631)(774.52265498,508.25914898)
\curveto(774.79428833,508.07122621)(775.01290237,507.80795909)(775.17849776,507.46934683)
\curveto(775.34407939,507.13072071)(775.42687364,506.7586117)(775.42688077,506.3530187)
\curveto(775.42687364,505.65716905)(775.20546878,505.06830655)(774.76266553,504.58642944)
\curveto(774.31984935,504.10454423)(773.51981499,503.86360365)(772.36256006,503.86360698)
\lineto(770.26386318,503.86360698)
\lineto(770.26386318,500.53694916)
\closepath
\moveto(770.26386318,504.82923081)
\lineto(772.37930498,504.82923081)
\curveto(773.07886583,504.82922652)(773.57563135,504.95946467)(773.86960303,505.21994565)
\curveto(774.16356357,505.48041727)(774.31054663,505.84694464)(774.31055264,506.31952886)
\curveto(774.31054663,506.66186336)(774.22403128,506.9548992)(774.05100635,507.19863726)
\curveto(773.87796991,507.442362)(773.65005315,507.60329914)(773.36725537,507.68144918)
\curveto(773.18491689,507.7298162)(772.84815824,507.75400329)(772.35697842,507.7540105)
\lineto(770.26386318,507.7540105)
\closepath
}
}
{
\newrgbcolor{curcolor}{0 0 0}
\pscustom[linestyle=none,fillstyle=solid,fillcolor=curcolor]
{
\newpath
\moveto(780.55640909,501.26814408)
\curveto(780.18429546,500.9518507)(779.82614055,500.72858529)(779.48194326,500.59834721)
\curveto(779.13773889,500.46810899)(778.7684207,500.40298992)(778.3739876,500.40298978)
\curveto(777.7227944,500.40298992)(777.22230779,500.56206652)(776.87252627,500.88022006)
\curveto(776.52274286,501.19837291)(776.34785163,501.604902)(776.34785205,502.09980854)
\curveto(776.34785163,502.390052)(776.41390098,502.65517966)(776.54600029,502.89519233)
\curveto(776.67809837,503.13520028)(776.85112906,503.32776669)(777.06509287,503.47289213)
\curveto(777.27905441,503.61801171)(777.51999499,503.72778387)(777.78791533,503.80220893)
\curveto(777.98513125,503.85430093)(778.28281845,503.90453564)(778.68097783,503.95291323)
\curveto(779.49217271,504.04965815)(780.08940767,504.16501194)(780.47268448,504.29897495)
\curveto(780.47640103,504.43665152)(780.47826157,504.52409713)(780.47826612,504.56131206)
\curveto(780.47826157,504.97062794)(780.38337378,505.25901242)(780.19360245,505.42646636)
\curveto(779.93684297,505.65344796)(779.55543125,505.76694121)(779.04936611,505.76694644)
\curveto(778.57678457,505.76694121)(778.22793237,505.68414695)(778.00280849,505.51856343)
\curveto(777.77768048,505.35296994)(777.6111617,505.0599341)(777.50325166,504.63945503)
\lineto(776.52088291,504.7734144)
\curveto(776.61018848,505.19389334)(776.75717154,505.53344281)(776.96183252,505.79206382)
\curveto(777.16649144,506.05067432)(777.4623181,506.24975264)(777.84931338,506.38929937)
\curveto(778.23630483,506.52883439)(778.68469618,506.59860483)(779.19448877,506.59861089)
\curveto(779.70055376,506.59860483)(780.11173421,506.53906739)(780.42803135,506.41999839)
\curveto(780.74431951,506.30091763)(780.97688764,506.15114375)(781.12573643,505.97067632)
\curveto(781.27457484,505.79019802)(781.37876536,505.56228125)(781.43830831,505.28692534)
\curveto(781.47179261,505.11575045)(781.48853752,504.80689998)(781.48854307,504.360373)
\lineto(781.48854307,503.02077924)
\curveto(781.48853752,502.08678316)(781.50993379,501.49606012)(781.55273194,501.24860834)
\curveto(781.59551886,501.00115514)(781.68017366,500.76393565)(781.80669659,500.53694916)
\lineto(780.75734815,500.53694916)
\curveto(780.65315281,500.7453302)(780.58617319,500.9890616)(780.55640909,501.26814408)
\closepath
\moveto(780.47268448,503.51196362)
\curveto(780.10801312,503.36311704)(779.56101288,503.23659998)(778.83168213,503.13241205)
\curveto(778.41863824,503.07287202)(778.12653267,503.0058924)(777.95536455,502.93147299)
\curveto(777.78419239,502.8570488)(777.65209369,502.74820691)(777.55906806,502.60494701)
\curveto(777.46603919,502.46168298)(777.41952556,502.30260638)(777.41952705,502.12771674)
\curveto(777.41952556,501.85979667)(777.52092527,501.63653126)(777.72372646,501.45791986)
\curveto(777.92652408,501.27930662)(778.22328101,501.19000046)(778.61399814,501.19000111)
\curveto(779.00098883,501.19000046)(779.34518966,501.27465526)(779.64660166,501.44396576)
\curveto(779.94800624,501.61327445)(780.1694111,501.84491231)(780.3108169,502.13888002)
\curveto(780.41872413,502.36586491)(780.47267994,502.70076302)(780.47268448,503.14357534)
\closepath
}
}
{
\newrgbcolor{curcolor}{0 0 0}
\pscustom[linestyle=none,fillstyle=solid,fillcolor=curcolor]
{
\newpath
\moveto(785.24498834,501.4355933)
\lineto(785.39011099,500.54811244)
\curveto(785.10730506,500.48857499)(784.85427094,500.45880627)(784.63100787,500.45880619)
\curveto(784.26633871,500.45880627)(783.98353587,500.51648316)(783.78259849,500.63183705)
\curveto(783.58165814,500.74719075)(783.44025672,500.89882516)(783.3583938,501.08674076)
\curveto(783.27652876,501.27465526)(783.23559677,501.67002107)(783.23559771,502.2728394)
\lineto(783.23559771,505.68322183)
\lineto(782.49882114,505.68322183)
\lineto(782.49882114,506.46465152)
\lineto(783.23559771,506.46465152)
\lineto(783.23559771,507.932623)
\lineto(784.23471138,508.53544019)
\lineto(784.23471138,506.46465152)
\lineto(785.24498834,506.46465152)
\lineto(785.24498834,505.68322183)
\lineto(784.23471138,505.68322183)
\lineto(784.23471138,502.21702299)
\curveto(784.23470944,501.93049738)(784.25238462,501.74630342)(784.28773697,501.66444057)
\curveto(784.32308533,501.58257546)(784.38076223,501.51745638)(784.46076783,501.46908314)
\curveto(784.5407691,501.42070804)(784.65519262,501.39652096)(784.80403873,501.39652182)
\curveto(784.91566892,501.39652096)(785.06265198,501.40954477)(785.24498834,501.4355933)
\closepath
}
}
{
\newrgbcolor{curcolor}{0 0 0}
\pscustom[linestyle=none,fillstyle=solid,fillcolor=curcolor]
{
\newpath
\moveto(790.08985254,502.70820737)
\lineto(791.07780293,502.57982963)
\curveto(790.96988571,501.89886811)(790.69359478,501.36582196)(790.2489293,500.98068959)
\curveto(789.80425426,500.59555633)(789.25818429,500.40298992)(788.61071777,500.40298978)
\curveto(787.799517,500.40298992)(787.14739597,500.66811758)(786.65435273,501.19837357)
\curveto(786.16130711,501.72862824)(785.91478489,502.48866088)(785.91478534,503.47847377)
\curveto(785.91478489,504.11849832)(786.02083596,504.67852237)(786.23293886,505.15854761)
\curveto(786.44504022,505.6385636)(786.76784479,505.99857906)(787.20135351,506.23859507)
\curveto(787.63485876,506.47859968)(788.10650693,506.59860483)(788.61629941,506.59861089)
\curveto(789.26004484,506.59860483)(789.78657908,506.43580714)(790.19590371,506.11021734)
\curveto(790.60521889,505.78461638)(790.86755573,505.32227095)(790.98291504,504.72317964)
\lineto(790.00612793,504.57247534)
\curveto(789.91309614,504.97062794)(789.74843791,505.27017569)(789.51215273,505.47111948)
\curveto(789.27585947,505.67205341)(788.99026581,505.77252284)(788.65537089,505.77252808)
\curveto(788.14929946,505.77252284)(787.73811901,505.5911197)(787.42182831,505.22831811)
\curveto(787.10553371,504.86550714)(786.94738738,504.29152901)(786.94738886,503.50638198)
\curveto(786.94738738,502.71006574)(787.09995207,502.13143624)(787.40508339,501.77049174)
\curveto(787.71021084,501.40954477)(788.10836747,501.22907191)(788.59955449,501.2290726)
\curveto(788.9939869,501.22907191)(789.32330337,501.35000733)(789.58750488,501.59187924)
\curveto(789.85169815,501.83374904)(790.01914721,502.20585804)(790.08985254,502.70820737)
\closepath
}
}
{
\newrgbcolor{curcolor}{0 0 0}
\pscustom[linestyle=none,fillstyle=solid,fillcolor=curcolor]
{
\newpath
\moveto(791.93737464,500.53694916)
\lineto(791.93737464,508.71963433)
\lineto(792.94206995,508.71963433)
\lineto(792.94206995,505.78369136)
\curveto(793.41092554,506.32696526)(794.00257885,506.59860483)(794.71703167,506.59861089)
\curveto(795.15611676,506.59860483)(795.53752849,506.51208949)(795.861268,506.3390646)
\curveto(796.18499816,506.16602811)(796.41663602,505.92694808)(796.55618226,505.62182378)
\curveto(796.69571777,505.31668931)(796.76548821,504.8738796)(796.76549379,504.29339331)
\lineto(796.76549379,500.53694916)
\lineto(795.76079847,500.53694916)
\lineto(795.76079847,504.29339331)
\curveto(795.7607939,504.79573671)(795.65195201,505.1613338)(795.4342725,505.39018569)
\curveto(795.21658448,505.61902788)(794.90866428,505.7334514)(794.51051097,505.73345659)
\curveto(794.21282044,505.7334514)(793.93280841,505.65623878)(793.67047405,505.50181851)
\curveto(793.40813472,505.3473883)(793.22114995,505.13807699)(793.10951917,504.87388393)
\curveto(792.99788454,504.6096822)(792.94206819,504.24501538)(792.94206995,503.77988237)
\lineto(792.94206995,500.53694916)
\closepath
}
}
{
\newrgbcolor{curcolor}{0 0 0}
\pscustom[linestyle=none,fillstyle=solid,fillcolor=curcolor]
{
\newpath
\moveto(801.19731388,502.69704409)
\lineto(802.2020092,502.83100346)
\curveto(802.3173615,502.26167439)(802.513649,501.85142421)(802.79087228,501.6002517)
\curveto(803.06809142,501.34907706)(803.40578034,501.22349027)(803.80394006,501.22349096)
\curveto(804.27651541,501.22349027)(804.67560231,501.38721823)(805.00120198,501.71467533)
\curveto(805.32679307,502.04213008)(805.48959076,502.44772889)(805.48959553,502.93147299)
\curveto(805.48959076,503.39288576)(805.33888661,503.77336722)(805.03748264,504.0729185)
\curveto(804.73607003,504.37246271)(804.35279775,504.52223659)(803.88766467,504.52224057)
\curveto(803.69788591,504.52223659)(803.46159669,504.48502569)(803.17879631,504.41060776)
\lineto(803.29042912,505.29250698)
\curveto(803.35740617,505.28506005)(803.41136197,505.28133896)(803.4522967,505.2813437)
\curveto(803.88021932,505.28133896)(804.26535214,505.39297166)(804.60769631,505.61624214)
\curveto(804.9500327,505.83950246)(805.12120285,506.18370329)(805.12120725,506.64884566)
\curveto(805.12120285,507.01722746)(804.99654633,507.32235684)(804.74723733,507.56423472)
\curveto(804.49792026,507.80609855)(804.17604598,507.92703397)(803.7816135,507.92704136)
\curveto(803.39089598,507.92703397)(803.0653006,507.804238)(802.80482638,507.55865308)
\curveto(802.54434799,507.31305412)(802.37689894,506.9446662)(802.30247873,506.45348823)
\lineto(801.29778341,506.63210074)
\curveto(801.4205788,507.30561194)(801.69966056,507.82749481)(802.13502951,508.19775093)
\curveto(802.57039562,508.56799173)(803.11181422,508.75311596)(803.75928693,508.75312418)
\curveto(804.2058147,508.75311596)(804.61699515,508.65729789)(804.99282951,508.46566968)
\curveto(805.36865533,508.27402562)(805.65610954,508.01261904)(805.85519299,507.68144918)
\curveto(806.05426617,507.35026502)(806.15380533,506.99862201)(806.15381077,506.62651909)
\curveto(806.15380533,506.27300945)(806.05891753,505.95113516)(805.86914709,505.66089526)
\curveto(805.67936635,505.37064512)(805.39842405,505.13993753)(805.02631936,504.96877182)
\curveto(805.51005675,504.85713469)(805.88588685,504.62549684)(806.15381077,504.27385757)
\curveto(806.42172381,503.92221082)(806.55568306,503.48219192)(806.55568889,502.95379955)
\curveto(806.55568306,502.23934785)(806.29520675,501.63374045)(805.7742592,501.13697553)
\curveto(805.25330154,500.64020941)(804.59466861,500.39182665)(803.79835842,500.3918265)
\curveto(803.08018496,500.39182665)(802.48388028,500.60578932)(802.00944259,501.03371517)
\curveto(801.53500232,501.46164003)(801.26429302,502.01608245)(801.19731388,502.69704409)
\closepath
}
}
{
\newrgbcolor{curcolor}{0 0 0}
\pscustom[linestyle=none,fillstyle=solid,fillcolor=curcolor]
{
\newpath
\moveto(807.56038914,502.69704409)
\lineto(808.56508445,502.83100346)
\curveto(808.68043676,502.26167439)(808.87672426,501.85142421)(809.15394754,501.6002517)
\curveto(809.43116667,501.34907706)(809.76885559,501.22349027)(810.16701532,501.22349096)
\curveto(810.63959066,501.22349027)(811.03867757,501.38721823)(811.36427723,501.71467533)
\curveto(811.68986833,502.04213008)(811.85266602,502.44772889)(811.85267079,502.93147299)
\curveto(811.85266602,503.39288576)(811.70196187,503.77336722)(811.4005579,504.0729185)
\curveto(811.09914528,504.37246271)(810.71587301,504.52223659)(810.25073992,504.52224057)
\curveto(810.06096116,504.52223659)(809.82467194,504.48502569)(809.54187156,504.41060776)
\lineto(809.65350438,505.29250698)
\curveto(809.72048142,505.28506005)(809.77443723,505.28133896)(809.81537195,505.2813437)
\curveto(810.24329457,505.28133896)(810.62842739,505.39297166)(810.97077157,505.61624214)
\curveto(811.31310796,505.83950246)(811.4842781,506.18370329)(811.48428251,506.64884566)
\curveto(811.4842781,507.01722746)(811.35962159,507.32235684)(811.11031258,507.56423472)
\curveto(810.86099552,507.80609855)(810.53912123,507.92703397)(810.14468875,507.92704136)
\curveto(809.75397123,507.92703397)(809.42837586,507.804238)(809.16790164,507.55865308)
\curveto(808.90742325,507.31305412)(808.7399742,506.9446662)(808.66555398,506.45348823)
\lineto(807.66085867,506.63210074)
\curveto(807.78365406,507.30561194)(808.06273581,507.82749481)(808.49810476,508.19775093)
\curveto(808.93347088,508.56799173)(809.47488948,508.75311596)(810.12236219,508.75312418)
\curveto(810.56888995,508.75311596)(810.9800704,508.65729789)(811.35590477,508.46566968)
\curveto(811.73173059,508.27402562)(812.01918479,508.01261904)(812.21826825,507.68144918)
\curveto(812.41734143,507.35026502)(812.51688059,506.99862201)(812.51688602,506.62651909)
\curveto(812.51688059,506.27300945)(812.42199279,505.95113516)(812.23222235,505.66089526)
\curveto(812.04244161,505.37064512)(811.76149931,505.13993753)(811.38939461,504.96877182)
\curveto(811.87313201,504.85713469)(812.2489621,504.62549684)(812.51688602,504.27385757)
\curveto(812.78479907,503.92221082)(812.91875831,503.48219192)(812.91876415,502.95379955)
\curveto(812.91875831,502.23934785)(812.65828201,501.63374045)(812.13733446,501.13697553)
\curveto(811.6163768,500.64020941)(810.95774386,500.39182665)(810.16143367,500.3918265)
\curveto(809.44326022,500.39182665)(808.84695554,500.60578932)(808.37251785,501.03371517)
\curveto(807.89807758,501.46164003)(807.62736828,502.01608245)(807.56038914,502.69704409)
\closepath
}
}
{
\newrgbcolor{curcolor}{0 0 0}
\pscustom[linestyle=none,fillstyle=solid,fillcolor=curcolor]
{
\newpath
\moveto(813.91787512,502.68029916)
\lineto(814.9728052,502.76960541)
\curveto(815.05094657,502.25609276)(815.2323497,501.87002966)(815.51701517,501.61141498)
\curveto(815.80167648,501.35279815)(816.14494704,501.22349027)(816.54682786,501.22349096)
\curveto(817.03056647,501.22349027)(817.43988637,501.40582368)(817.7747888,501.77049174)
\curveto(818.10968258,502.13515733)(818.27713163,502.61889903)(818.27713646,503.22171831)
\curveto(818.27713163,503.79476349)(818.11619448,504.24687593)(817.79432455,504.57805698)
\curveto(817.47244591,504.90922995)(817.05103246,505.07481846)(816.53008294,505.074823)
\curveto(816.20634502,505.07481846)(815.91423945,505.00132693)(815.65376536,504.85434819)
\curveto(815.39328685,504.70736082)(815.1886269,504.51665495)(815.03978489,504.28223003)
\lineto(814.09648762,504.40502612)
\lineto(814.88908059,508.60800152)
\lineto(818.95809662,508.60800152)
\lineto(818.95809662,507.64795933)
\lineto(815.69283685,507.64795933)
\lineto(815.25188724,505.44879292)
\curveto(815.74306931,505.79112829)(816.25844028,505.96229843)(816.79800169,505.96230386)
\curveto(817.51244763,505.96229843)(818.11526421,505.71484595)(818.60645326,505.21994565)
\curveto(819.09763198,504.725036)(819.34322392,504.0887296)(819.34322982,503.31102456)
\curveto(819.34322392,502.57052486)(819.1274007,501.93049738)(818.69575951,501.39094018)
\curveto(818.17108056,500.72858529)(817.45477073,500.39740828)(816.54682786,500.39740814)
\curveto(815.80260675,500.39740828)(815.1951388,500.60578932)(814.7244222,501.02255189)
\curveto(814.25370302,501.43931349)(813.98485427,501.99189536)(813.91787512,502.68029916)
\closepath
}
}
{
\newrgbcolor{curcolor}{0 0 0}
\pscustom[linestyle=none,fillstyle=solid,fillcolor=curcolor]
{
\newpath
\moveto(466.03772813,413.85689545)
\lineto(466.03772813,422.03958062)
\lineto(467.14847461,422.03958062)
\lineto(471.44633791,415.61511225)
\lineto(471.44633791,422.03958062)
\lineto(472.48452306,422.03958062)
\lineto(472.48452306,413.85689545)
\lineto(471.37377658,413.85689545)
\lineto(467.07591329,420.28694546)
\lineto(467.07591329,413.85689545)
\closepath
}
}
{
\newrgbcolor{curcolor}{0 0 0}
\pscustom[linestyle=none,fillstyle=solid,fillcolor=curcolor]
{
\newpath
\moveto(478.23919444,415.76581655)
\lineto(479.27737959,415.63743881)
\curveto(479.11364578,415.03089935)(478.81037695,414.56018146)(478.36757217,414.22528373)
\curveto(477.92475752,413.89038526)(477.35915183,413.72293621)(476.67075342,413.72293607)
\curveto(475.80373619,413.72293621)(475.11626481,413.98992442)(474.6083372,414.5239015)
\curveto(474.10040723,415.05787726)(473.84644283,415.80674663)(473.84644325,416.77051186)
\curveto(473.84644283,417.76776108)(474.10319805,418.5417478)(474.61670966,419.09247437)
\curveto(475.1302189,419.64319046)(475.79629401,419.91855112)(476.61493701,419.91855718)
\curveto(477.407526,419.91855112)(478.05499567,419.64877209)(478.55734795,419.10921929)
\curveto(479.05968998,418.56965598)(479.31086356,417.81055361)(479.31086944,416.83190991)
\curveto(479.31086356,416.77236949)(479.30900301,416.68306333)(479.3052878,416.56399116)
\lineto(474.88462841,416.56399116)
\curveto(474.92183786,415.91279769)(475.10603181,415.41417163)(475.43721084,415.06811147)
\curveto(475.76838584,414.72204888)(476.18142683,414.54901819)(476.67633506,414.54901889)
\curveto(477.04471972,414.54901819)(477.35915183,414.64576654)(477.61963233,414.8392642)
\curveto(477.88010444,415.0327599)(478.08662493,415.34161037)(478.23919444,415.76581655)
\closepath
\moveto(474.94044482,417.39007397)
\lineto(478.25035772,417.39007397)
\curveto(478.20569981,417.8886965)(478.07918275,418.26266605)(477.87080615,418.51198374)
\curveto(477.55078797,418.89897245)(477.13588643,419.09246913)(476.62610029,419.09247437)
\curveto(476.16468193,419.09246913)(475.77675829,418.93804389)(475.46232822,418.62919819)
\curveto(475.14789408,418.32034295)(474.97393312,417.90730195)(474.94044482,417.39007397)
\closepath
}
}
{
\newrgbcolor{curcolor}{0 0 0}
\pscustom[linestyle=none,fillstyle=solid,fillcolor=curcolor]
{
\newpath
\moveto(482.73799695,414.75553959)
\lineto(482.88311961,413.86805873)
\curveto(482.60031367,413.80852128)(482.34727955,413.77875256)(482.12401648,413.77875248)
\curveto(481.75934732,413.77875256)(481.47654448,413.83642945)(481.2756071,413.95178334)
\curveto(481.07466676,414.06713703)(480.93326534,414.21877145)(480.85140241,414.40668705)
\curveto(480.76953737,414.59460155)(480.72860538,414.98996736)(480.72860632,415.59278569)
\lineto(480.72860632,419.00316812)
\lineto(479.99182976,419.00316812)
\lineto(479.99182976,419.78459781)
\lineto(480.72860632,419.78459781)
\lineto(480.72860632,421.25256929)
\lineto(481.72772,421.85538648)
\lineto(481.72772,419.78459781)
\lineto(482.73799695,419.78459781)
\lineto(482.73799695,419.00316812)
\lineto(481.72772,419.00316812)
\lineto(481.72772,415.53696928)
\curveto(481.72771806,415.25044367)(481.74539324,415.06624971)(481.78074558,414.98438686)
\curveto(481.81609395,414.90252175)(481.87377084,414.83740267)(481.95377644,414.78902943)
\curveto(482.03377771,414.74065433)(482.14820123,414.71646725)(482.29704734,414.71646811)
\curveto(482.40867754,414.71646725)(482.55566059,414.72949106)(482.73799695,414.75553959)
\closepath
}
}
{
\newrgbcolor{curcolor}{0 0 0}
\pscustom[linestyle=none,fillstyle=solid,fillcolor=curcolor]
{
\newpath
\moveto(483.79850785,413.85689545)
\lineto(483.79850785,422.03958062)
\lineto(486.8684102,422.03958062)
\curveto(487.49354942,422.03957244)(487.9949663,421.95677819)(488.37266235,421.79119761)
\curveto(488.75034758,421.62560117)(489.04617423,421.37070651)(489.26014321,421.02651285)
\curveto(489.47409959,420.68230485)(489.58108093,420.32228939)(489.58108755,419.94646538)
\curveto(489.58108093,419.59667683)(489.48619313,419.26736036)(489.29642387,418.95851499)
\curveto(489.10664195,418.64965942)(488.82011801,418.40034638)(488.43685122,418.21057515)
\curveto(488.93175071,418.06544828)(489.31223217,417.81799579)(489.57829673,417.46821694)
\curveto(489.84434805,417.11843087)(489.97737702,416.70538987)(489.97738403,416.22909272)
\curveto(489.97737702,415.84581807)(489.89644331,415.4895237)(489.73458266,415.16020854)
\curveto(489.57270847,414.83089076)(489.37269988,414.57692637)(489.1345563,414.39831459)
\curveto(488.89640036,414.21970173)(488.59778288,414.08481221)(488.23870297,413.99364564)
\curveto(487.87961251,413.9024788)(487.43959361,413.85689545)(486.91864496,413.85689545)
\closepath
\moveto(484.88134613,418.60128999)
\lineto(486.65072621,418.60128999)
\curveto(487.13074314,418.60128525)(487.47494397,418.63291451)(487.68332973,418.69617788)
\curveto(487.95868567,418.77803702)(488.16613644,418.91385681)(488.30568266,419.10363765)
\curveto(488.44521819,419.29340799)(488.51498863,419.53155775)(488.51499418,419.81808765)
\curveto(488.51498863,420.08972126)(488.44986956,420.3288013)(488.31963676,420.53532847)
\curveto(488.18939325,420.74184229)(488.00333875,420.88324371)(487.7614727,420.95953316)
\curveto(487.51959705,421.0358084)(487.10469551,421.07394958)(486.51676684,421.07395679)
\lineto(484.88134613,421.07395679)
\closepath
\moveto(484.88134613,414.82251928)
\lineto(486.91864496,414.82251928)
\curveto(487.26842347,414.82251831)(487.51401541,414.83554213)(487.65542153,414.86159076)
\curveto(487.90472986,414.90624284)(488.11311091,414.98066464)(488.28056528,415.08485639)
\curveto(488.44800901,415.18904568)(488.58568934,415.3406801)(488.69360668,415.5397601)
\curveto(488.80151256,415.73883673)(488.85546837,415.96861404)(488.85547426,416.22909272)
\curveto(488.85546837,416.53421973)(488.77732548,416.79934739)(488.62104536,417.02447651)
\curveto(488.46475392,417.24959929)(488.24800042,417.40774562)(487.97078422,417.49891596)
\curveto(487.693558,417.59007903)(487.2944711,417.63566238)(486.77352231,417.63566616)
\lineto(484.88134613,417.63566616)
\closepath
}
}
{
\newrgbcolor{curcolor}{0 0 0}
\pscustom[linestyle=none,fillstyle=solid,fillcolor=curcolor]
{
\newpath
\moveto(491.11045766,416.48584819)
\lineto(492.13189789,416.57515444)
\curveto(492.18027053,416.16583182)(492.2928335,415.83000344)(492.46958715,415.5676683)
\curveto(492.64633705,415.30532974)(492.92076744,415.09322761)(493.29287914,414.93136127)
\curveto(493.66498545,414.76949278)(494.08360808,414.68855907)(494.54874829,414.6885599)
\curveto(494.96178533,414.68855907)(495.32645215,414.74995706)(495.64274985,414.87275404)
\curveto(495.95903746,414.995549)(496.19439641,415.16392832)(496.34882739,415.37789252)
\curveto(496.50324688,415.59185368)(496.5804595,415.82535208)(496.58046548,416.07838842)
\curveto(496.5804595,416.33514141)(496.5060377,416.55933709)(496.35719986,416.75097612)
\curveto(496.20835049,416.94260936)(495.96275855,417.10354651)(495.62042329,417.23378803)
\curveto(495.40087396,417.31936973)(494.91527171,417.4523987)(494.16361508,417.63287534)
\curveto(493.41195133,417.81334443)(492.88541709,417.9835843)(492.58401078,418.14359546)
\curveto(492.19329434,418.34825112)(491.90211905,418.60221552)(491.71048402,418.9054894)
\curveto(491.51884677,419.20875319)(491.4230287,419.54830266)(491.42302953,419.92413882)
\curveto(491.4230287,420.33717375)(491.54024304,420.72323684)(491.77467289,421.08232925)
\curveto(492.00910039,421.44140722)(492.35144067,421.71397706)(492.80169477,421.90003961)
\curveto(493.25194446,422.08608607)(493.75243107,422.17911332)(494.3031561,422.17912164)
\curveto(494.90969007,422.17911332)(495.44459676,422.0814347)(495.90787778,421.88608551)
\curveto(496.37114818,421.69072025)(496.72744255,421.40326604)(496.97676197,421.02372203)
\curveto(497.22606862,420.64416368)(497.36002786,420.21437778)(497.37864009,419.73436304)
\lineto(496.34045493,419.65622007)
\curveto(496.28463284,420.17344579)(496.09578752,420.56416024)(495.77391841,420.8283646)
\curveto(495.45203894,421.09255503)(494.97666969,421.22465372)(494.34780923,421.22466109)
\curveto(493.69289363,421.22465372)(493.21566383,421.10464857)(492.9161184,420.86464527)
\curveto(492.61656833,420.62462795)(492.46679446,420.3353132)(492.46679633,419.99670015)
\curveto(492.46679446,419.7027279)(492.57284553,419.46085704)(492.78494985,419.27108687)
\curveto(492.9933287,419.08130586)(493.53753812,418.88687891)(494.41757973,418.68780542)
\curveto(495.29761371,418.48872227)(495.90136057,418.31476131)(496.22882212,418.16592202)
\curveto(496.70511601,417.9463734)(497.05675902,417.66822192)(497.2837522,417.33146674)
\curveto(497.51073201,416.99470462)(497.62422525,416.60678099)(497.62423228,416.16769467)
\curveto(497.62422525,415.73232483)(497.49956874,415.32207465)(497.25026236,414.93694291)
\curveto(497.00094267,414.55180901)(496.64278776,414.25226126)(496.17579653,414.03829877)
\curveto(495.70879416,413.82433591)(495.18319019,413.71735457)(494.59898305,413.71735443)
\curveto(493.85848213,413.71735457)(493.23799037,413.82526618)(492.7375059,414.04108959)
\curveto(492.23701715,414.25691263)(491.84444215,414.58157773)(491.55977973,415.01508588)
\curveto(491.27511538,415.44859171)(491.1253415,415.93884532)(491.11045766,416.48584819)
\closepath
}
}
{
\newrgbcolor{curcolor}{0 0 0}
\pscustom[linestyle=none,fillstyle=solid,fillcolor=curcolor]
{
\newpath
\moveto(499.1145309,413.85689545)
\lineto(499.1145309,422.03958062)
\lineto(501.93325943,422.03958062)
\curveto(502.56956212,422.03957244)(503.05516437,422.00050099)(503.39006763,421.92236617)
\curveto(503.85891982,421.81444649)(504.258937,421.61908927)(504.59012037,421.3362939)
\curveto(505.02176046,420.9716196)(505.34456502,420.50555307)(505.55853502,419.93809292)
\curveto(505.77249037,419.37062061)(505.87947171,418.72222067)(505.87947936,417.99289116)
\curveto(505.87947171,417.37146499)(505.80691046,416.82074366)(505.66179537,416.34072553)
\curveto(505.51666543,415.86070243)(505.33061093,415.46347607)(505.10363131,415.14904525)
\curveto(504.87663795,414.83461185)(504.62825519,414.58715937)(504.35848228,414.40668705)
\curveto(504.08869713,414.22621363)(503.76310175,414.08946357)(503.38169517,413.99643646)
\curveto(503.00027829,413.90340907)(502.56211994,413.85689545)(502.0672188,413.85689545)
\closepath
\moveto(500.19736919,414.82251928)
\lineto(501.94442271,414.82251928)
\curveto(502.48397705,414.82251831)(502.90725104,414.87275303)(503.21424595,414.97322357)
\curveto(503.5212309,415.07369189)(503.76589257,415.21509331)(503.9482317,415.39742826)
\curveto(504.20498119,415.65418194)(504.40498978,415.99931304)(504.54825807,416.4328226)
\curveto(504.69151372,416.86632702)(504.7631447,417.39193098)(504.76315123,418.00963608)
\curveto(504.7631447,418.86548264)(504.62267355,419.5231853)(504.34173736,419.98274605)
\curveto(504.06078896,420.44229454)(503.71937894,420.75021474)(503.3175063,420.90650757)
\curveto(503.0272562,421.01813322)(502.5602594,421.07394958)(501.9165145,421.07395679)
\lineto(500.19736919,421.07395679)
\closepath
}
}
{
\newrgbcolor{curcolor}{0 0 0}
\pscustom[linestyle=none,fillstyle=solid,fillcolor=curcolor]
{
\newpath
\moveto(513.92262161,413.85689545)
\lineto(512.9179263,413.85689545)
\lineto(512.9179263,420.25903726)
\curveto(512.67605219,420.02832327)(512.35882926,419.79761569)(511.96625657,419.56691382)
\curveto(511.57367927,419.33620053)(511.22110598,419.16316984)(510.90853567,419.04782124)
\lineto(510.90853567,420.01902671)
\curveto(511.47041902,420.28321794)(511.9616029,420.60323169)(512.38208879,420.9790689)
\curveto(512.80256925,421.35489187)(513.10025645,421.7195587)(513.2751513,422.07307047)
\lineto(513.92262161,422.07307047)
\closepath
}
}
{
\newrgbcolor{curcolor}{0 0 0}
\pscustom[linestyle=none,fillstyle=solid,fillcolor=curcolor]
{
\newpath
\moveto(517.0650864,413.85689545)
\lineto(517.0650864,415.00113178)
\lineto(518.20932273,415.00113178)
\lineto(518.20932273,413.85689545)
\closepath
}
}
{
\newrgbcolor{curcolor}{0 0 0}
\pscustom[linestyle=none,fillstyle=solid,fillcolor=curcolor]
{
\newpath
\moveto(519.67729435,416.01699037)
\lineto(520.68198966,416.15094975)
\curveto(520.79734197,415.58162068)(520.99362947,415.1713705)(521.27085275,414.92019799)
\curveto(521.54807189,414.66902335)(521.88576081,414.54343656)(522.28392053,414.54343725)
\curveto(522.75649588,414.54343656)(523.15558278,414.70716452)(523.48118244,415.03462162)
\curveto(523.80677354,415.36207637)(523.96957123,415.76767518)(523.969576,416.25141928)
\curveto(523.96957123,416.71283205)(523.81886708,417.09331351)(523.51746311,417.39286479)
\curveto(523.2160505,417.692409)(522.83277822,417.84218288)(522.36764514,417.84218686)
\curveto(522.17786637,417.84218288)(521.94157716,417.80497198)(521.65877678,417.73055405)
\lineto(521.77040959,418.61245327)
\curveto(521.83738664,418.60500634)(521.89134244,418.60128525)(521.93227717,418.60128999)
\curveto(522.36019979,418.60128525)(522.74533261,418.71291795)(523.08767678,418.93618843)
\curveto(523.43001317,419.15944875)(523.60118331,419.50364958)(523.60118772,419.96879195)
\curveto(523.60118331,420.33717375)(523.4765268,420.64230313)(523.2272178,420.88418101)
\curveto(522.97790073,421.12604484)(522.65602644,421.24698026)(522.26159396,421.24698765)
\curveto(521.87087645,421.24698026)(521.54528107,421.12418429)(521.28480685,420.87859937)
\curveto(521.02432846,420.63300041)(520.85687941,420.26461249)(520.7824592,419.77343452)
\lineto(519.77776388,419.95204702)
\curveto(519.90055927,420.62555823)(520.17964102,421.1474411)(520.61500998,421.51769722)
\curveto(521.05037609,421.88793802)(521.59179469,422.07306225)(522.2392674,422.07307047)
\curveto(522.68579516,422.07306225)(523.09697561,421.97724418)(523.47280998,421.78561597)
\curveto(523.8486358,421.59397191)(524.13609001,421.33256533)(524.33517346,421.00139546)
\curveto(524.53424664,420.67021131)(524.6337858,420.3185683)(524.63379124,419.94646538)
\curveto(524.6337858,419.59295574)(524.538898,419.27108145)(524.34912756,418.98084155)
\curveto(524.15934682,418.69059141)(523.87840452,418.45988382)(523.50629983,418.28871811)
\curveto(523.99003722,418.17708098)(524.36586732,417.94544313)(524.63379124,417.59380385)
\curveto(524.90170428,417.24215711)(525.03566352,416.80213821)(525.03566936,416.27374584)
\curveto(525.03566352,415.55929414)(524.77518722,414.95368674)(524.25423967,414.45692182)
\curveto(523.73328201,413.9601557)(523.07464907,413.71177294)(522.27833889,413.71177279)
\curveto(521.56016543,413.71177294)(520.96386075,413.92573561)(520.48942306,414.35366146)
\curveto(520.01498279,414.78158632)(519.74427349,415.33602874)(519.67729435,416.01699037)
\closepath
}
}
{
\newrgbcolor{curcolor}{0 0 0}
\pscustom[linestyle=none,fillstyle=solid,fillcolor=curcolor]
{
\newpath
\moveto(331.79757754,419.0521478)
\curveto(331.79757699,420.41034168)(332.16224381,421.47364316)(332.89157911,422.24205543)
\curveto(333.62091111,423.01045334)(334.56234689,423.39465589)(335.71588927,423.39466422)
\curveto(336.47126608,423.39465589)(337.15222555,423.21418302)(337.75876974,422.85324508)
\curveto(338.3653009,422.49229156)(338.82764634,421.98901413)(339.14580744,421.34341128)
\curveto(339.46395274,420.69779588)(339.62302934,419.96567142)(339.62303772,419.14703569)
\curveto(339.62302934,418.31722853)(339.45558029,417.57487107)(339.12069006,416.91996108)
\curveto(338.78578408,416.26504738)(338.3113451,415.76921213)(337.6973717,415.43245385)
\curveto(337.08338539,415.09569483)(336.42103136,414.92731551)(335.71030763,414.92731537)
\curveto(334.94003752,414.92731551)(334.25163587,415.11337001)(333.64510059,415.48547943)
\curveto(333.03856052,415.85758802)(332.5790059,416.36551681)(332.26643535,417.00926733)
\curveto(331.95386277,417.65301396)(331.79757699,418.33397344)(331.79757754,419.0521478)
\closepath
\moveto(332.91390567,419.03540288)
\curveto(332.913904,418.04931005)(333.17903166,417.2725325)(333.70928946,416.70506791)
\curveto(334.23954233,416.13760004)(334.90468717,415.85386693)(335.70472599,415.85386771)
\curveto(336.51964025,415.85386693)(337.19036673,416.14039086)(337.71690744,416.71344037)
\curveto(338.24343521,417.28648659)(338.50670233,418.09954477)(338.50670959,419.15261733)
\curveto(338.50670233,419.81868836)(338.39413935,420.40010868)(338.16902033,420.89688003)
\curveto(337.94388746,421.39363972)(337.61457099,421.77877254)(337.18106994,422.05227964)
\curveto(336.74755701,422.32577278)(336.26102449,422.46252283)(335.72147091,422.46253023)
\curveto(334.95492188,422.46252283)(334.29535868,422.19925571)(333.7427793,421.67272808)
\curveto(333.19019493,421.14618723)(332.913904,420.26707971)(332.91390567,419.03540288)
\closepath
}
}
{
\newrgbcolor{curcolor}{0 0 0}
\pscustom[linestyle=none,fillstyle=solid,fillcolor=curcolor]
{
\newpath
\moveto(340.88448863,412.79512864)
\lineto(340.88448863,420.99455874)
\lineto(341.79987769,420.99455874)
\lineto(341.79987769,420.22429233)
\curveto(342.01569924,420.52569547)(342.25943064,420.75175169)(342.53107261,420.90246167)
\curveto(342.80270979,421.05315998)(343.13202626,421.12851206)(343.51902301,421.12851812)
\curveto(344.02508786,421.12851206)(344.47161867,420.9982739)(344.85861676,420.73780327)
\curveto(345.2456054,420.4773213)(345.53771097,420.10986366)(345.73493434,419.63542925)
\curveto(345.93214651,419.1609857)(346.0307554,418.64096337)(346.03076129,418.07536069)
\curveto(346.0307554,417.46882)(345.92191351,416.92275004)(345.70423532,416.43714916)
\curveto(345.48654598,415.95154554)(345.17025332,415.57943654)(344.75535641,415.32082103)
\curveto(344.34045025,415.06220502)(343.90415244,414.93289714)(343.44646168,414.93289701)
\curveto(343.11156026,414.93289714)(342.81108224,415.00359785)(342.54502672,415.14499935)
\curveto(342.27896636,415.2864007)(342.06035232,415.46501302)(341.88918394,415.68083685)
\lineto(341.88918394,412.79512864)
\closepath
\moveto(341.79429605,417.99721772)
\curveto(341.79429439,417.23439133)(341.94871962,416.67064619)(342.25757222,416.30598061)
\curveto(342.56642057,415.94131254)(342.94039012,415.75897913)(343.37948199,415.75897982)
\curveto(343.82600955,415.75897913)(344.20835155,415.94782445)(344.52650914,416.32551635)
\curveto(344.84465795,416.70320573)(345.00373454,417.28834714)(345.00373942,418.08094233)
\curveto(345.00373454,418.83632059)(344.84837904,419.40192628)(344.53767242,419.77776108)
\curveto(344.226957,420.15358647)(343.85577827,420.34150151)(343.42413512,420.34150679)
\curveto(342.99620647,420.34150151)(342.61758556,420.14149292)(342.28827125,419.74148042)
\curveto(341.95895262,419.34145857)(341.79429439,418.76003825)(341.79429605,417.99721772)
\closepath
}
}
{
\newrgbcolor{curcolor}{0 0 0}
\pscustom[linestyle=none,fillstyle=solid,fillcolor=curcolor]
{
\newpath
\moveto(351.30541186,416.97577748)
\lineto(352.34359702,416.84739975)
\curveto(352.17986321,416.24086029)(351.87659437,415.7701424)(351.43378959,415.43524467)
\curveto(350.99097494,415.10034619)(350.42536925,414.93289714)(349.73697084,414.93289701)
\curveto(348.86995362,414.93289714)(348.18248223,415.19988535)(347.67455462,415.73386244)
\curveto(347.16662465,416.26783819)(346.91266026,417.01670756)(346.91266068,417.9804728)
\curveto(346.91266026,418.97772201)(347.16941547,419.75170874)(347.68292708,420.3024353)
\curveto(348.19643632,420.85315139)(348.86251144,421.12851206)(349.68115443,421.12851812)
\curveto(350.47374342,421.12851206)(351.12121309,420.85873303)(351.62356537,420.31918023)
\curveto(352.1259074,419.77961692)(352.37708098,419.02051455)(352.37708686,418.04187084)
\curveto(352.37708098,417.98233043)(352.37522043,417.89302427)(352.37150522,417.77395209)
\lineto(347.95084583,417.77395209)
\curveto(347.98805528,417.12275863)(348.17224924,416.62413257)(348.50342826,416.2780724)
\curveto(348.83460326,415.93200982)(349.24764426,415.75897913)(349.74255248,415.75897982)
\curveto(350.11093714,415.75897913)(350.42536925,415.85572747)(350.68584975,416.04922514)
\curveto(350.94632186,416.24272084)(351.15284236,416.55157131)(351.30541186,416.97577748)
\closepath
\moveto(348.00666224,418.60003491)
\lineto(351.31657514,418.60003491)
\curveto(351.27191724,419.09865744)(351.14540018,419.47262699)(350.93702358,419.72194468)
\curveto(350.61700539,420.10893339)(350.20210385,420.30243007)(349.69231771,420.3024353)
\curveto(349.23089935,420.30243007)(348.84297571,420.14800483)(348.52854564,419.83915913)
\curveto(348.2141115,419.53030388)(348.04015054,419.11726289)(348.00666224,418.60003491)
\closepath
}
}
{
\newrgbcolor{curcolor}{0 0 0}
\pscustom[linestyle=none,fillstyle=solid,fillcolor=curcolor]
{
\newpath
\moveto(353.61063056,415.06685638)
\lineto(353.61063056,420.99455874)
\lineto(354.51485634,420.99455874)
\lineto(354.51485634,420.15173101)
\curveto(354.95022222,420.80291668)(355.57908643,421.12851206)(356.40145088,421.12851812)
\curveto(356.75867197,421.12851206)(357.08705817,421.06432325)(357.38661045,420.93595152)
\curveto(357.68615367,420.80756804)(357.91034934,420.63918872)(358.05919814,420.43081304)
\curveto(358.20803654,420.22242663)(358.31222707,419.97497414)(358.37177002,419.68845483)
\curveto(358.40897541,419.50239571)(358.42758086,419.17680033)(358.42758643,418.71166772)
\lineto(358.42758643,415.06685638)
\lineto(357.42289111,415.06685638)
\lineto(357.42289111,418.67259624)
\curveto(357.42288655,419.08191254)(357.3838151,419.38797219)(357.30567666,419.59077612)
\curveto(357.22752932,419.79357101)(357.08891872,419.95543842)(356.88984443,420.07637886)
\curveto(356.69076208,420.19730927)(356.45726368,420.25777699)(356.18934853,420.25778218)
\curveto(355.76141984,420.25777699)(355.39210166,420.1219572)(355.08139286,419.85032241)
\curveto(354.77067962,419.57867806)(354.61532411,419.06330709)(354.61532587,418.30420795)
\lineto(354.61532587,415.06685638)
\closepath
}
}
{
\newrgbcolor{curcolor}{0 0 0}
\pscustom[linestyle=none,fillstyle=solid,fillcolor=curcolor]
{
\newpath
\moveto(360.0574247,415.06685638)
\lineto(360.0574247,423.24954156)
\lineto(363.12732705,423.24954156)
\curveto(363.75246627,423.24953338)(364.25388315,423.16673912)(364.6315792,423.00115855)
\curveto(365.00926443,422.83556211)(365.30509109,422.58066744)(365.51906006,422.23647378)
\curveto(365.73301644,421.89226579)(365.83999778,421.53225033)(365.8400044,421.15642632)
\curveto(365.83999778,420.80663777)(365.74510999,420.4773213)(365.55534073,420.16847593)
\curveto(365.3655588,419.85962035)(365.07903487,419.61030732)(364.69576807,419.42053608)
\curveto(365.19066757,419.27540922)(365.57114903,419.02795673)(365.83721358,418.67817788)
\curveto(366.1032649,418.3283918)(366.23629387,417.91535081)(366.23630089,417.43905366)
\curveto(366.23629387,417.05577901)(366.15536016,416.69948464)(365.99349952,416.37016947)
\curveto(365.83162533,416.0408517)(365.63161674,415.78688731)(365.39347315,415.60827553)
\curveto(365.15531721,415.42966266)(364.85669974,415.29477315)(364.49761983,415.20360658)
\curveto(364.13852936,415.11243974)(363.69851046,415.06685638)(363.17756182,415.06685638)
\closepath
\moveto(361.14026298,419.81125093)
\lineto(362.90964307,419.81125093)
\curveto(363.38965999,419.81124618)(363.73386082,419.84287545)(363.94224658,419.90613882)
\curveto(364.21760252,419.98799796)(364.42505329,420.12381775)(364.56459952,420.31359858)
\curveto(364.70413505,420.50336893)(364.77390549,420.74151869)(364.77391104,421.02804859)
\curveto(364.77390549,421.2996822)(364.70878641,421.53876223)(364.57855362,421.74528941)
\curveto(364.44831011,421.95180323)(364.26225561,422.09320465)(364.02038955,422.1694941)
\curveto(363.7785139,422.24576934)(363.36361236,422.28391051)(362.77568369,422.28391773)
\lineto(361.14026298,422.28391773)
\closepath
\moveto(361.14026298,416.03248022)
\lineto(363.17756182,416.03248022)
\curveto(363.52734032,416.03247925)(363.77293227,416.04550306)(363.91433838,416.0715517)
\curveto(364.16364672,416.11620378)(364.37202776,416.19062558)(364.53948213,416.29481733)
\curveto(364.70692586,416.39900662)(364.8446062,416.55064104)(364.95252354,416.74972104)
\curveto(365.06042942,416.94879767)(365.11438522,417.17857498)(365.11439112,417.43905366)
\curveto(365.11438522,417.74418067)(365.03624233,418.00930833)(364.87996221,418.23443745)
\curveto(364.72367077,418.45956023)(364.50691728,418.61770655)(364.22970108,418.7088769)
\curveto(363.95247486,418.80003997)(363.55338795,418.84562332)(363.03243916,418.8456271)
\lineto(361.14026298,418.8456271)
\closepath
}
}
{
\newrgbcolor{curcolor}{0 0 0}
\pscustom[linestyle=none,fillstyle=solid,fillcolor=curcolor]
{
\newpath
\moveto(367.3693726,417.69580913)
\lineto(368.39081284,417.78511538)
\curveto(368.43918547,417.37579275)(368.55174845,417.03996438)(368.7285021,416.77762924)
\curveto(368.905252,416.51529068)(369.17968239,416.30318855)(369.55179409,416.14132221)
\curveto(369.9239004,415.97945372)(370.34252303,415.89852001)(370.80766324,415.89852084)
\curveto(371.22070028,415.89852001)(371.5853671,415.95991799)(371.9016648,416.08271498)
\curveto(372.21795241,416.20550994)(372.45331135,416.37388926)(372.60774234,416.58785346)
\curveto(372.76216183,416.80181461)(372.83937444,417.03531301)(372.83938043,417.28834936)
\curveto(372.83937444,417.54510235)(372.76495264,417.76929802)(372.6161148,417.96093706)
\curveto(372.46726544,418.1525703)(372.2216735,418.31350744)(371.87933824,418.44374897)
\curveto(371.6597889,418.52933066)(371.17418665,418.66235963)(370.42253003,418.84283628)
\curveto(369.67086628,419.02330537)(369.14433204,419.19354524)(368.84292573,419.35355639)
\curveto(368.45220929,419.55821206)(368.16103399,419.81217646)(367.96939897,420.11545034)
\curveto(367.77776172,420.41871413)(367.68194365,420.7582636)(367.68194448,421.13409976)
\curveto(367.68194365,421.54713469)(367.79915799,421.93319778)(368.03358784,422.29229019)
\curveto(368.26801533,422.65136815)(368.61035562,422.923938)(369.06060972,423.11000054)
\curveto(369.51085941,423.296047)(370.01134602,423.38907425)(370.56207105,423.38908258)
\curveto(371.16860502,423.38907425)(371.70351171,423.29139564)(372.16679273,423.09604644)
\curveto(372.63006313,422.90068119)(372.9863575,422.61322698)(373.23567691,422.23368296)
\curveto(373.48498357,421.85412461)(373.61894281,421.42433871)(373.63755504,420.94432398)
\lineto(372.59936988,420.86618101)
\curveto(372.54354779,421.38340672)(372.35470247,421.77412118)(372.03283336,422.03832554)
\curveto(371.71095389,422.30251596)(371.23558464,422.43461466)(370.60672417,422.43462203)
\curveto(369.95180858,422.43461466)(369.47457878,422.31460951)(369.17503335,422.07460621)
\curveto(368.87548328,421.83458889)(368.72570941,421.54527414)(368.72571128,421.20666109)
\curveto(368.72570941,420.91268883)(368.83176047,420.67081798)(369.04386479,420.4810478)
\curveto(369.25224365,420.2912668)(369.79645307,420.09683984)(370.67649468,419.89776636)
\curveto(371.55652865,419.69868321)(372.16027551,419.52472225)(372.48773707,419.37588296)
\curveto(372.96403096,419.15633434)(373.31567397,418.87818286)(373.54266715,418.54142768)
\curveto(373.76964695,418.20466556)(373.8831402,417.81674192)(373.88314723,417.37765561)
\curveto(373.8831402,416.94228576)(373.75848368,416.53203559)(373.5091773,416.14690385)
\curveto(373.25985762,415.76176995)(372.9017027,415.4622222)(372.43471148,415.24825971)
\curveto(371.9677091,415.03429685)(371.44210514,414.92731551)(370.857898,414.92731537)
\curveto(370.11739708,414.92731551)(369.49690532,415.03522712)(368.99642085,415.25105053)
\curveto(368.4959321,415.46687356)(368.1033571,415.79153867)(367.81869467,416.22504682)
\curveto(367.53403032,416.65855265)(367.38425645,417.14880626)(367.3693726,417.69580913)
\closepath
}
}
{
\newrgbcolor{curcolor}{0 0 0}
\pscustom[linestyle=none,fillstyle=solid,fillcolor=curcolor]
{
\newpath
\moveto(375.37344585,415.06685638)
\lineto(375.37344585,423.24954156)
\lineto(378.19217437,423.24954156)
\curveto(378.82847707,423.24953338)(379.31407932,423.21046193)(379.64898258,423.13232711)
\curveto(380.11783477,423.02440743)(380.51785195,422.8290502)(380.84903532,422.54625484)
\curveto(381.2806754,422.18158054)(381.60347997,421.71551401)(381.81744997,421.14805386)
\curveto(382.03140532,420.58058155)(382.13838666,419.93218161)(382.13839431,419.2028521)
\curveto(382.13838666,418.58142593)(382.0658254,418.0307046)(381.92071032,417.55068647)
\curveto(381.77558038,417.07066337)(381.58952588,416.67343701)(381.36254626,416.35900619)
\curveto(381.13555289,416.04457279)(380.88717013,415.7971203)(380.61739723,415.61664799)
\curveto(380.34761208,415.43617457)(380.0220167,415.29942451)(379.64061012,415.2063974)
\curveto(379.25919324,415.11337001)(378.82103489,415.06685638)(378.32613375,415.06685638)
\closepath
\moveto(376.45628414,416.03248022)
\lineto(378.20333766,416.03248022)
\curveto(378.742892,416.03247925)(379.16616599,416.08271396)(379.4731609,416.18318451)
\curveto(379.78014585,416.28365283)(380.02480752,416.42505425)(380.20714664,416.6073892)
\curveto(380.46389614,416.86414287)(380.66390473,417.20927397)(380.80717301,417.64278354)
\curveto(380.95042866,418.07628795)(381.02205965,418.60189192)(381.02206618,419.21959702)
\curveto(381.02205965,420.07544358)(380.8815885,420.73314624)(380.60065231,421.19270699)
\curveto(380.3197039,421.65225548)(379.97829389,421.96017568)(379.57642125,422.11646851)
\curveto(379.28617114,422.22809416)(378.81917434,422.28391051)(378.17542945,422.28391773)
\lineto(376.45628414,422.28391773)
\closepath
}
}
{
\newrgbcolor{curcolor}{0 0 0}
\pscustom[linestyle=none,fillstyle=solid,fillcolor=curcolor]
{
\newpath
\moveto(391.67742006,416.03248022)
\lineto(391.67742006,415.06685638)
\lineto(386.26881029,415.06685638)
\curveto(386.26136776,415.30872724)(386.3004392,415.54129536)(386.38602474,415.76456146)
\curveto(386.52370461,416.13294868)(386.74417919,416.49575496)(387.04744915,416.85298139)
\curveto(387.35071687,417.21020425)(387.78887522,417.62324524)(388.36192552,418.09210561)
\curveto(389.2512636,418.82143623)(389.85221965,419.39913546)(390.16479545,419.82520503)
\curveto(390.47736277,420.25126508)(390.63364855,420.65407308)(390.63365326,421.03363023)
\curveto(390.63364855,421.43178089)(390.49131686,421.76760927)(390.20665776,422.04111636)
\curveto(389.92199008,422.31460951)(389.55081135,422.45135956)(389.09312045,422.45136695)
\curveto(388.60937557,422.45135956)(388.22238221,422.30623705)(387.9321392,422.01599898)
\curveto(387.64189216,421.72574701)(387.49490911,421.32386928)(387.49118959,420.8103646)
\lineto(386.45858607,420.91641577)
\curveto(386.52928624,421.68667556)(386.79534418,422.27367752)(387.25676068,422.6774234)
\curveto(387.71817451,423.08115405)(388.337736,423.28302319)(389.11544701,423.2830314)
\curveto(389.90059382,423.28302319)(390.52201585,423.06533942)(390.97971498,422.62997945)
\curveto(391.437404,422.19460435)(391.66625104,421.6550463)(391.66625678,421.01130366)
\curveto(391.66625104,420.6838418)(391.59927142,420.36196751)(391.46531772,420.04567983)
\curveto(391.33135294,419.7293822)(391.10901781,419.39634464)(390.79831166,419.04656616)
\curveto(390.48759577,418.69677972)(389.97129453,418.2167591)(389.24940639,417.60650288)
\curveto(388.64658647,417.10043209)(388.25959311,416.75716153)(388.08842513,416.57669018)
\curveto(387.91725283,416.3962158)(387.7758514,416.21481266)(387.66422044,416.03248022)
\closepath
}
}
{
\newrgbcolor{curcolor}{0 0 0}
\pscustom[linestyle=none,fillstyle=solid,fillcolor=curcolor]
{
\newpath
\moveto(393.32400135,415.06685638)
\lineto(393.32400135,416.21109272)
\lineto(394.46823768,416.21109272)
\lineto(394.46823768,415.06685638)
\closepath
}
}
{
\newrgbcolor{curcolor}{0 0 0}
\pscustom[linestyle=none,fillstyle=solid,fillcolor=curcolor]
{
\newpath
\moveto(395.93621311,417.22695131)
\lineto(396.94090843,417.36091069)
\curveto(397.05626073,416.79158162)(397.25254823,416.38133144)(397.52977151,416.13015893)
\curveto(397.80699065,415.87898429)(398.14467957,415.7533975)(398.54283929,415.75339818)
\curveto(399.01541464,415.7533975)(399.41450154,415.91712546)(399.74010121,416.24458256)
\curveto(400.0656923,416.5720373)(400.22848999,416.97763612)(400.22849476,417.46138022)
\curveto(400.22848999,417.92279299)(400.07778584,418.30327444)(399.77638187,418.60282573)
\curveto(399.47496926,418.90236994)(399.09169698,419.05214382)(398.6265639,419.0521478)
\curveto(398.43678514,419.05214382)(398.20049592,419.01493291)(397.91769554,418.94051499)
\lineto(398.02932835,419.82241421)
\curveto(398.0963054,419.81496727)(398.1502612,419.81124618)(398.19119593,419.81125093)
\curveto(398.61911855,419.81124618)(399.00425137,419.92287888)(399.34659554,420.14614937)
\curveto(399.68893193,420.36940969)(399.86010208,420.71361052)(399.86010648,421.17875288)
\curveto(399.86010208,421.54713469)(399.73544556,421.85226407)(399.48613656,422.09414195)
\curveto(399.2368195,422.33600577)(398.91494521,422.4569412)(398.52051273,422.45694859)
\curveto(398.12979521,422.4569412)(397.80419983,422.33414523)(397.54372562,422.08856031)
\curveto(397.28324723,421.84296134)(397.11579817,421.47457343)(397.04137796,420.98339546)
\lineto(396.03668264,421.16200796)
\curveto(396.15947803,421.83551916)(396.43855979,422.35740204)(396.87392874,422.72765816)
\curveto(397.30929486,423.09789896)(397.85071346,423.28302319)(398.49818616,423.2830314)
\curveto(398.94471393,423.28302319)(399.35589438,423.18720512)(399.73172875,422.99557691)
\curveto(400.10755456,422.80393285)(400.39500877,422.54252627)(400.59409222,422.2113564)
\curveto(400.7931654,421.88017224)(400.89270456,421.52852924)(400.89271,421.15642632)
\curveto(400.89270456,420.80291668)(400.79781677,420.48104239)(400.60804633,420.19080249)
\curveto(400.41826558,419.90055234)(400.13732328,419.66984476)(399.76521859,419.49867905)
\curveto(400.24895599,419.38704192)(400.62478608,419.15540406)(400.89271,418.80376479)
\curveto(401.16062304,418.45211805)(401.29458229,418.01209915)(401.29458812,417.48370678)
\curveto(401.29458229,416.76925508)(401.03410598,416.16364767)(400.51315844,415.66688275)
\curveto(399.99220077,415.17011663)(399.33356784,414.92173387)(398.53725765,414.92173373)
\curveto(397.81908419,414.92173387)(397.22277951,415.13569655)(396.74834182,415.5636224)
\curveto(396.27390155,415.99154726)(396.00319225,416.54598967)(395.93621311,417.22695131)
\closepath
}
}
{
\newrgbcolor{curcolor}{0 0 0}
\pscustom[linestyle=none,fillstyle=solid,fillcolor=curcolor]
{
\newpath
\moveto(214.68971453,413.95687103)
\lineto(214.68971453,422.13955621)
\lineto(220.20995712,422.13955621)
\lineto(220.20995712,421.17393238)
\lineto(215.77255281,421.17393238)
\lineto(215.77255281,418.63986753)
\lineto(219.61272157,418.63986753)
\lineto(219.61272157,417.6742437)
\lineto(215.77255281,417.6742437)
\lineto(215.77255281,413.95687103)
\closepath
}
}
{
\newrgbcolor{curcolor}{0 0 0}
\pscustom[linestyle=none,fillstyle=solid,fillcolor=curcolor]
{
\newpath
\moveto(221.48257112,413.95687103)
\lineto(221.48257112,419.88457339)
\lineto(222.38679691,419.88457339)
\lineto(222.38679691,418.98592925)
\curveto(222.61750284,419.40640739)(222.83053525,419.6836286)(223.02589476,419.8175937)
\curveto(223.2212497,419.95154708)(223.43614265,420.0185267)(223.67057425,420.01853277)
\curveto(224.00919052,420.0185267)(224.35339134,419.91061509)(224.70317777,419.69479761)
\lineto(224.35711605,418.76266362)
\curveto(224.11152049,418.90778133)(223.86592855,418.98034258)(223.62033949,418.98034761)
\curveto(223.40079229,418.98034258)(223.20357452,418.91429324)(223.02868558,418.78219936)
\curveto(222.85379206,418.65009584)(222.72913554,418.46683216)(222.65471566,418.23240776)
\curveto(222.54308104,417.87517884)(222.48726469,417.48446439)(222.48726644,417.06026323)
\lineto(222.48726644,413.95687103)
\closepath
}
}
{
\newrgbcolor{curcolor}{0 0 0}
\pscustom[linestyle=none,fillstyle=solid,fillcolor=curcolor]
{
\newpath
\moveto(229.36942915,415.86579213)
\lineto(230.40761431,415.7374144)
\curveto(230.2438805,415.13087494)(229.94061166,414.66015705)(229.49780689,414.32525932)
\curveto(229.05499223,413.99036084)(228.48938655,413.82291179)(227.80098813,413.82291166)
\curveto(226.93397091,413.82291179)(226.24649953,414.0899)(225.73857192,414.62387709)
\curveto(225.23064195,415.15785284)(224.97667755,415.90672221)(224.97667797,416.87048745)
\curveto(224.97667755,417.86773666)(225.23343277,418.64172339)(225.74694438,419.19244995)
\curveto(226.26045362,419.74316604)(226.92652873,420.0185267)(227.74517173,420.01853277)
\curveto(228.53776072,420.0185267)(229.18523039,419.74874768)(229.68758267,419.20919487)
\curveto(230.1899247,418.66963157)(230.44109827,417.9105292)(230.44110416,416.93188549)
\curveto(230.44109827,416.87234508)(230.43923773,416.78303892)(230.43552252,416.66396674)
\lineto(226.01486313,416.66396674)
\curveto(226.05207257,416.01277328)(226.23626653,415.51414721)(226.56744555,415.16808705)
\curveto(226.89862056,414.82202447)(227.31166155,414.64899378)(227.80656978,414.64899447)
\curveto(228.17495444,414.64899378)(228.48938655,414.74574212)(228.74986704,414.93923979)
\curveto(229.01033915,415.13273548)(229.21685965,415.44158596)(229.36942915,415.86579213)
\closepath
\moveto(226.07067954,417.49004956)
\lineto(229.38059244,417.49004956)
\curveto(229.33593453,417.98867209)(229.20941747,418.36264164)(229.00104087,418.61195933)
\curveto(228.68102269,418.99894803)(228.26612115,419.19244472)(227.75633501,419.19244995)
\curveto(227.29491665,419.19244472)(226.90699301,419.03801948)(226.59256294,418.72917378)
\curveto(226.27812879,418.42031853)(226.10416783,418.00727754)(226.07067954,417.49004956)
\closepath
}
}
{
\newrgbcolor{curcolor}{0 0 0}
\pscustom[linestyle=none,fillstyle=solid,fillcolor=curcolor]
{
\newpath
\moveto(235.73249964,415.86579213)
\lineto(236.7706848,415.7374144)
\curveto(236.60695099,415.13087494)(236.30368215,414.66015705)(235.86087738,414.32525932)
\curveto(235.41806272,413.99036084)(234.85245704,413.82291179)(234.16405862,413.82291166)
\curveto(233.2970414,413.82291179)(232.60957002,414.0899)(232.10164241,414.62387709)
\curveto(231.59371244,415.15785284)(231.33974804,415.90672221)(231.33974846,416.87048745)
\curveto(231.33974804,417.86773666)(231.59650325,418.64172339)(232.11001487,419.19244995)
\curveto(232.6235241,419.74316604)(233.28959922,420.0185267)(234.10824222,420.01853277)
\curveto(234.90083121,420.0185267)(235.54830087,419.74874768)(236.05065316,419.20919487)
\curveto(236.55299518,418.66963157)(236.80416876,417.9105292)(236.80417464,416.93188549)
\curveto(236.80416876,416.87234508)(236.80230822,416.78303892)(236.798593,416.66396674)
\lineto(232.37793362,416.66396674)
\curveto(232.41514306,416.01277328)(232.59933702,415.51414721)(232.93051604,415.16808705)
\curveto(233.26169105,414.82202447)(233.67473204,414.64899378)(234.16964026,414.64899447)
\curveto(234.53802493,414.64899378)(234.85245704,414.74574212)(235.11293753,414.93923979)
\curveto(235.37340964,415.13273548)(235.57993014,415.44158596)(235.73249964,415.86579213)
\closepath
\moveto(232.43375002,417.49004956)
\lineto(235.74366292,417.49004956)
\curveto(235.69900502,417.98867209)(235.57248796,418.36264164)(235.36411136,418.61195933)
\curveto(235.04409317,418.99894803)(234.62919163,419.19244472)(234.1194055,419.19244995)
\curveto(233.65798713,419.19244472)(233.2700635,419.03801948)(232.95563342,418.72917378)
\curveto(232.64119928,418.42031853)(232.46723832,418.00727754)(232.43375002,417.49004956)
\closepath
}
}
{
\newrgbcolor{curcolor}{0 0 0}
\pscustom[linestyle=none,fillstyle=solid,fillcolor=curcolor]
{
\newpath
\moveto(238.12144295,413.95687103)
\lineto(238.12144295,422.13955621)
\lineto(241.1913453,422.13955621)
\curveto(241.81648452,422.13954803)(242.3179014,422.05675377)(242.69559745,421.8911732)
\curveto(243.07328268,421.72557676)(243.36910934,421.47068209)(243.58307831,421.12648843)
\curveto(243.79703469,420.78228043)(243.90401603,420.42226497)(243.90402265,420.04644097)
\curveto(243.90401603,419.69665242)(243.80912824,419.36733595)(243.61935898,419.05849058)
\curveto(243.42957705,418.749635)(243.14305312,418.50032197)(242.75978632,418.31055073)
\curveto(243.25468582,418.16542387)(243.63516728,417.91797138)(243.90123183,417.56819253)
\curveto(244.16728315,417.21840645)(244.30031212,416.80536546)(244.30031914,416.3290683)
\curveto(244.30031212,415.94579366)(244.21937841,415.58949929)(244.05751777,415.26018412)
\curveto(243.89564358,414.93086635)(243.69563499,414.67690196)(243.4574914,414.49829017)
\curveto(243.21933546,414.31967731)(242.92071799,414.1847878)(242.56163808,414.09362123)
\curveto(242.20254761,414.00245439)(241.76252871,413.95687103)(241.24158007,413.95687103)
\closepath
\moveto(239.20428123,418.70126558)
\lineto(240.97366132,418.70126558)
\curveto(241.45367824,418.70126083)(241.79787907,418.7328901)(242.00626483,418.79615347)
\curveto(242.28162077,418.87801261)(242.48907154,419.01383239)(242.62861776,419.20361323)
\curveto(242.7681533,419.39338358)(242.83792374,419.63153334)(242.83792929,419.91806323)
\curveto(242.83792374,420.18969685)(242.77280466,420.42877688)(242.64257187,420.63530406)
\curveto(242.51232836,420.84181788)(242.32627385,420.9832193)(242.0844078,421.05950875)
\curveto(241.84253215,421.13578399)(241.42763061,421.17392516)(240.83970194,421.17393238)
\lineto(239.20428123,421.17393238)
\closepath
\moveto(239.20428123,414.92249486)
\lineto(241.24158007,414.92249486)
\curveto(241.59135857,414.9224939)(241.83695051,414.93551771)(241.97835663,414.96156635)
\curveto(242.22766497,415.00621842)(242.43604601,415.08064022)(242.60350038,415.18483197)
\curveto(242.77094411,415.28902127)(242.90862445,415.44065569)(243.01654179,415.63973569)
\curveto(243.12444767,415.83881232)(243.17840347,416.06858963)(243.17840937,416.3290683)
\curveto(243.17840347,416.63419532)(243.10026058,416.89932298)(242.94398046,417.1244521)
\curveto(242.78768902,417.34957488)(242.57093552,417.5077212)(242.29371933,417.59889155)
\curveto(242.01649311,417.69005461)(241.6174062,417.73563797)(241.09645741,417.73564175)
\lineto(239.20428123,417.73564175)
\closepath
}
}
{
\newrgbcolor{curcolor}{0 0 0}
\pscustom[linestyle=none,fillstyle=solid,fillcolor=curcolor]
{
\newpath
\moveto(245.43339085,416.58582377)
\lineto(246.45483109,416.67513002)
\curveto(246.50320372,416.2658074)(246.6157667,415.92997903)(246.79252035,415.66764389)
\curveto(246.96927025,415.40530533)(247.24370064,415.1932032)(247.61581234,415.03133686)
\curveto(247.98791865,414.86946836)(248.40654128,414.78853466)(248.87168149,414.78853549)
\curveto(249.28471853,414.78853466)(249.64938535,414.84993264)(249.96568305,414.97272963)
\curveto(250.28197066,415.09552458)(250.5173296,415.26390391)(250.67176059,415.47786811)
\curveto(250.82618008,415.69182926)(250.90339269,415.92532766)(250.90339868,416.17836401)
\curveto(250.90339269,416.435117)(250.82897089,416.65931267)(250.68013305,416.8509517)
\curveto(250.53128369,417.04258495)(250.28569175,417.20352209)(249.94335649,417.33376362)
\curveto(249.72380715,417.41934531)(249.2382049,417.55237428)(248.48654828,417.73285092)
\curveto(247.73488453,417.91332002)(247.20835029,418.08355988)(246.90694398,418.24357104)
\curveto(246.51622754,418.44822671)(246.22505224,418.7021911)(246.03341722,419.00546499)
\curveto(245.84177997,419.30872878)(245.7459619,419.64827825)(245.74596273,420.02411441)
\curveto(245.7459619,420.43714933)(245.86317624,420.82321243)(246.09760609,421.18230484)
\curveto(246.33203358,421.5413828)(246.67437387,421.81395265)(247.12462797,422.00001519)
\curveto(247.57487766,422.18606165)(248.07536427,422.2790889)(248.6260893,422.27909722)
\curveto(249.23262327,422.2790889)(249.76752996,422.18141029)(250.23081098,421.98606109)
\curveto(250.69408138,421.79069584)(251.05037575,421.50324163)(251.29969516,421.12369761)
\curveto(251.54900182,420.74413926)(251.68296106,420.31435336)(251.70157329,419.83433863)
\lineto(250.66338813,419.75619566)
\curveto(250.60756604,420.27342137)(250.41872072,420.66413583)(250.09685161,420.92834019)
\curveto(249.77497214,421.19253061)(249.29960289,421.32462931)(248.67074242,421.32463668)
\curveto(248.01582682,421.32462931)(247.53859703,421.20462415)(247.2390516,420.96462085)
\curveto(246.93950153,420.72460354)(246.78972766,420.43528879)(246.78972953,420.09667574)
\curveto(246.78972766,419.80270348)(246.89577872,419.56083263)(247.10788304,419.37106245)
\curveto(247.3162619,419.18128145)(247.86047132,418.98685449)(248.74051293,418.78778101)
\curveto(249.6205469,418.58869786)(250.22429376,418.4147369)(250.55175532,418.26589761)
\curveto(251.02804921,418.04634898)(251.37969222,417.7681975)(251.6066854,417.43144233)
\curveto(251.8336652,417.09468021)(251.94715845,416.70675657)(251.94716548,416.26767026)
\curveto(251.94715845,415.83230041)(251.82250193,415.42205024)(251.57319555,415.0369185)
\curveto(251.32387587,414.6517846)(250.96572095,414.35223685)(250.49872973,414.13827435)
\curveto(250.03172735,413.9243115)(249.50612338,413.81733016)(248.92191625,413.81733002)
\curveto(248.18141533,413.81733016)(247.56092357,413.92524177)(247.0604391,414.14106517)
\curveto(246.55995035,414.35688821)(246.16737535,414.68155332)(245.88271292,415.11506147)
\curveto(245.59804857,415.5485673)(245.4482747,416.03882091)(245.43339085,416.58582377)
\closepath
}
}
{
\newrgbcolor{curcolor}{0 0 0}
\pscustom[linestyle=none,fillstyle=solid,fillcolor=curcolor]
{
\newpath
\moveto(253.4374641,413.95687103)
\lineto(253.4374641,422.13955621)
\lineto(256.25619262,422.13955621)
\curveto(256.89249532,422.13954803)(257.37809757,422.10047658)(257.71300083,422.02234176)
\curveto(258.18185302,421.91442208)(258.5818702,421.71906485)(258.91305357,421.43626949)
\curveto(259.34469365,421.07159519)(259.66749822,420.60552866)(259.88146822,420.03806851)
\curveto(260.09542357,419.4705962)(260.20240491,418.82219626)(260.20241255,418.09286675)
\curveto(260.20240491,417.47144057)(260.12984365,416.92071925)(259.98472857,416.44070112)
\curveto(259.83959863,415.96067802)(259.65354413,415.56345166)(259.42656451,415.24902084)
\curveto(259.19957114,414.93458744)(258.95118838,414.68713495)(258.68141548,414.50666264)
\curveto(258.41163033,414.32618922)(258.08603495,414.18943916)(257.70462837,414.09641205)
\curveto(257.32321149,414.00338466)(256.88505314,413.95687103)(256.390152,413.95687103)
\closepath
\moveto(254.52030239,414.92249486)
\lineto(256.2673559,414.92249486)
\curveto(256.80691025,414.9224939)(257.23018424,414.97272861)(257.53717915,415.07319916)
\curveto(257.8441641,415.17366748)(258.08882577,415.3150689)(258.27116489,415.49740385)
\curveto(258.52791439,415.75415752)(258.72792298,416.09928862)(258.87119126,416.53279819)
\curveto(259.01444691,416.9663026)(259.0860779,417.49190657)(259.08608443,418.10961167)
\curveto(259.0860779,418.96545822)(258.94560675,419.62316089)(258.66467056,420.08272163)
\curveto(258.38372215,420.54227013)(258.04231214,420.85019033)(257.6404395,421.00648316)
\curveto(257.35018939,421.11810881)(256.88319259,421.17392516)(256.2394477,421.17393238)
\lineto(254.52030239,421.17393238)
\closepath
}
}
{
\newrgbcolor{curcolor}{0 0 0}
\pscustom[linestyle=none,fillstyle=solid,fillcolor=curcolor]
{
\newpath
\moveto(264.4667841,416.11696596)
\lineto(265.47147941,416.25092534)
\curveto(265.58683172,415.68159627)(265.78311922,415.27134609)(266.0603425,415.02017357)
\curveto(266.33756163,414.76899893)(266.67525055,414.64341214)(267.07341027,414.64341283)
\curveto(267.54598562,414.64341214)(267.94507253,414.80714011)(268.27067219,415.13459721)
\curveto(268.59626328,415.46205195)(268.75906097,415.86765077)(268.75906575,416.35139487)
\curveto(268.75906097,416.81280764)(268.60835683,417.19328909)(268.30695285,417.49284038)
\curveto(268.00554024,417.79238459)(267.62226797,417.94215846)(267.15713488,417.94216245)
\curveto(266.96735612,417.94215846)(266.7310669,417.90494756)(266.44826652,417.83052964)
\lineto(266.55989933,418.71242886)
\curveto(266.62687638,418.70498192)(266.68083219,418.70126083)(266.72176691,418.70126558)
\curveto(267.14968953,418.70126083)(267.53482235,418.81289353)(267.87716653,419.03616401)
\curveto(268.21950292,419.25942434)(268.39067306,419.60362517)(268.39067746,420.06876753)
\curveto(268.39067306,420.43714933)(268.26601654,420.74227872)(268.01670754,420.9841566)
\curveto(267.76739048,421.22602042)(267.44551619,421.34695585)(267.05108371,421.34696324)
\curveto(266.66036619,421.34695585)(266.33477081,421.22415988)(266.0742966,420.97857496)
\curveto(265.81381821,420.73297599)(265.64636916,420.36458808)(265.57194894,419.87341011)
\lineto(264.56725363,420.05202261)
\curveto(264.69004902,420.72553381)(264.96913077,421.24741669)(265.40449972,421.61767281)
\curveto(265.83986584,421.98791361)(266.38128444,422.17303784)(267.02875715,422.17304605)
\curveto(267.47528491,422.17303784)(267.88646536,422.07721977)(268.26229973,421.88559156)
\curveto(268.63812555,421.69394749)(268.92557975,421.43254092)(269.12466321,421.10137105)
\curveto(269.32373639,420.77018689)(269.42327555,420.41854388)(269.42328098,420.04644097)
\curveto(269.42327555,419.69293133)(269.32838775,419.37105704)(269.13861731,419.08081714)
\curveto(268.94883657,418.79056699)(268.66789427,418.55985941)(268.29578957,418.3886937)
\curveto(268.77952697,418.27705657)(269.15535706,418.04541871)(269.42328098,417.69377944)
\curveto(269.69119403,417.34213269)(269.82515327,416.9021138)(269.82515911,416.37372143)
\curveto(269.82515327,415.65926973)(269.56467697,415.05366232)(269.04372942,414.5568974)
\curveto(268.52277176,414.06013128)(267.86413882,413.81174852)(267.06782863,413.81174838)
\curveto(266.34965517,413.81174852)(265.7533505,414.0257112)(265.27891281,414.45363705)
\curveto(264.80447254,414.88156191)(264.53376324,415.43600432)(264.4667841,416.11696596)
\closepath
}
}
{
\newrgbcolor{curcolor}{0 0 0}
\pscustom[linestyle=none,fillstyle=solid,fillcolor=curcolor]
{
\newpath
\moveto(271.3880196,413.95687103)
\lineto(271.3880196,415.10110736)
\lineto(272.53225593,415.10110736)
\lineto(272.53225593,413.95687103)
\closepath
}
}
{
\newrgbcolor{curcolor}{0 0 0}
\pscustom[linestyle=none,fillstyle=solid,fillcolor=curcolor]
{
\newpath
\moveto(273.99464591,417.99239721)
\curveto(273.99464543,418.95987659)(274.09418459,419.73851468)(274.29326368,420.32831382)
\curveto(274.49234122,420.91810022)(274.78816788,421.37300348)(275.18074454,421.69302496)
\curveto(275.57331788,422.01303096)(276.06729258,422.17303784)(276.66267013,422.17304605)
\curveto(277.10175561,422.17303784)(277.48688843,422.08466195)(277.81806974,421.90791812)
\curveto(278.14924246,421.73115839)(278.42274258,421.47626373)(278.63857092,421.14323335)
\curveto(278.85438902,420.81018861)(279.02369862,420.4045898)(279.14650021,419.9264357)
\curveto(279.26929056,419.44826966)(279.33068855,418.80359081)(279.33069436,417.99239721)
\curveto(279.33068855,417.03235195)(279.23207966,416.25743495)(279.0348674,415.66764389)
\curveto(278.83764412,415.07784941)(278.54274773,414.62201588)(278.15017736,414.30014193)
\curveto(277.75759773,413.9782673)(277.26176248,413.81733016)(276.66267013,413.81733002)
\curveto(275.8737959,413.81733016)(275.25423441,414.100133)(274.8039838,414.66573939)
\curveto(274.26442446,415.34669816)(273.99464543,416.45558299)(273.99464591,417.99239721)
\closepath
\moveto(275.02724942,417.99239721)
\curveto(275.02724792,416.64907968)(275.18446397,415.75508779)(275.49889806,415.31041889)
\curveto(275.81332819,414.86574727)(276.20125182,414.64341214)(276.66267013,414.64341283)
\curveto(277.12408215,414.64341214)(277.51200579,414.86667755)(277.8264422,415.31320971)
\curveto(278.14087001,415.75973916)(278.29808606,416.65280077)(278.29809084,417.99239721)
\curveto(278.29808606,419.33942777)(278.14087001,420.23434993)(277.8264422,420.67716636)
\curveto(277.51200579,421.11996936)(277.12036106,421.34137421)(276.65150685,421.3413816)
\curveto(276.19008855,421.34137421)(275.82170064,421.14601699)(275.546342,420.75530933)
\curveto(275.2002786,420.25667647)(275.02724792,419.33570668)(275.02724942,417.99239721)
\closepath
}
}
{
\newrgbcolor{curcolor}{0 0 0}
\pscustom[linestyle=none,fillstyle=solid,fillcolor=curcolor]
{
\newpath
\moveto(725.11966173,326.36255834)
\lineto(725.11966173,325.39693451)
\lineto(719.71105195,325.39693451)
\curveto(719.70360943,325.63880536)(719.74268087,325.87137349)(719.82826641,326.09463959)
\curveto(719.96594628,326.46302681)(720.18642086,326.82583308)(720.48969082,327.18305951)
\curveto(720.79295854,327.54028237)(721.23111689,327.95332337)(721.80416719,328.42218374)
\curveto(722.69350527,329.15151436)(723.29446131,329.72921359)(723.60703712,330.15528315)
\curveto(723.91960444,330.5813432)(724.07589022,330.9841512)(724.07589493,331.36370835)
\curveto(724.07589022,331.76185902)(723.93355853,332.0976874)(723.64889942,332.37119449)
\curveto(723.36423175,332.64468763)(722.99305302,332.78143769)(722.53536212,332.78144507)
\curveto(722.05161724,332.78143769)(721.66462388,332.63631518)(721.37438086,332.3460771)
\curveto(721.08413383,332.05582513)(720.93715078,331.65394741)(720.93343125,331.14044273)
\lineto(719.90082774,331.2464939)
\curveto(719.97152791,332.01675369)(720.23758585,332.60375564)(720.69900235,333.00750152)
\curveto(721.16041618,333.41123218)(721.77997767,333.61310131)(722.55768868,333.61310953)
\curveto(723.34283548,333.61310131)(723.96425752,333.39541755)(724.42195665,332.96005757)
\curveto(724.87964567,332.52468248)(725.10849271,331.98512442)(725.10849845,331.34138179)
\curveto(725.10849271,331.01391992)(725.04151309,330.69204563)(724.90755939,330.37575796)
\curveto(724.7735946,330.05946033)(724.55125947,329.72642277)(724.24055333,329.37664428)
\curveto(723.92983744,329.02685784)(723.4135362,328.54683723)(722.69164805,327.936581)
\curveto(722.08882814,327.43051022)(721.70183478,327.08723966)(721.5306668,326.9067683)
\curveto(721.35949449,326.72629393)(721.21809307,326.54489079)(721.10646211,326.36255834)
\closepath
}
}
{
\newrgbcolor{curcolor}{0 0 0}
\pscustom[linestyle=none,fillstyle=solid,fillcolor=curcolor]
{
\newpath
\moveto(726.76624588,325.39693451)
\lineto(726.76624588,326.54117084)
\lineto(727.91048221,326.54117084)
\lineto(727.91048221,325.39693451)
\closepath
}
}
{
\newrgbcolor{curcolor}{0 0 0}
\pscustom[linestyle=none,fillstyle=solid,fillcolor=curcolor]
{
\newpath
\moveto(733.15722454,325.39693451)
\lineto(732.15252922,325.39693451)
\lineto(732.15252922,331.79907632)
\curveto(731.91065512,331.56836234)(731.59343219,331.33765475)(731.20085949,331.10695288)
\curveto(730.80828219,330.87623959)(730.45570891,330.7032089)(730.14313859,330.5878603)
\lineto(730.14313859,331.55906577)
\curveto(730.70502195,331.823257)(731.19620583,332.14327075)(731.61669172,332.51910796)
\curveto(732.03717218,332.89493094)(732.33485938,333.25959776)(732.50975422,333.61310953)
\lineto(733.15722454,333.61310953)
\closepath
}
}
{
\newrgbcolor{curcolor}{0 0 0}
\pscustom[linestyle=none,fillstyle=solid,fillcolor=curcolor]
{
\newpath
\moveto(739.52029503,325.39693451)
\lineto(738.51559971,325.39693451)
\lineto(738.51559971,331.79907632)
\curveto(738.27372561,331.56836234)(737.95650268,331.33765475)(737.56392998,331.10695288)
\curveto(737.17135268,330.87623959)(736.8187794,330.7032089)(736.50620908,330.5878603)
\lineto(736.50620908,331.55906577)
\curveto(737.06809243,331.823257)(737.55927632,332.14327075)(737.97976221,332.51910796)
\curveto(738.40024267,332.89493094)(738.69792987,333.25959776)(738.87282471,333.61310953)
\lineto(739.52029503,333.61310953)
\closepath
}
}
{
\newrgbcolor{curcolor}{0 0 0}
\pscustom[linestyle=none,fillstyle=solid,fillcolor=curcolor]
{
\newpath
\moveto(742.46181885,325.39693451)
\lineto(742.46181885,333.57961969)
\lineto(745.5317212,333.57961969)
\curveto(746.15686042,333.5796115)(746.6582773,333.49681725)(747.03597335,333.33123668)
\curveto(747.41365858,333.16564024)(747.70948524,332.91074557)(747.92345421,332.56655191)
\curveto(748.13741059,332.22234391)(748.24439193,331.86232845)(748.24439855,331.48650445)
\curveto(748.24439193,331.13671589)(748.14950414,330.80739942)(747.95973488,330.49855405)
\curveto(747.76995295,330.18969848)(747.48342902,329.94038545)(747.10016222,329.75061421)
\curveto(747.59506172,329.60548734)(747.97554318,329.35803485)(748.24160773,329.008256)
\curveto(748.50765905,328.65846993)(748.64068802,328.24542893)(748.64069504,327.76913178)
\curveto(748.64068802,327.38585713)(748.55975431,327.02956276)(748.39789367,326.7002476)
\curveto(748.23601948,326.37092983)(748.03601089,326.11696543)(747.7978673,325.93835365)
\curveto(747.55971136,325.75974079)(747.26109389,325.62485127)(746.90201398,325.53368471)
\curveto(746.54292351,325.44251786)(746.10290461,325.39693451)(745.58195597,325.39693451)
\closepath
\moveto(743.54465713,330.14132905)
\lineto(745.31403722,330.14132905)
\curveto(745.79405414,330.14132431)(746.13825497,330.17295357)(746.34664073,330.23621694)
\curveto(746.62199667,330.31807608)(746.82944744,330.45389587)(746.96899367,330.64367671)
\curveto(747.1085292,330.83344705)(747.17829964,331.07159682)(747.17830519,331.35812671)
\curveto(747.17829964,331.62976032)(747.11318056,331.86884036)(746.98294777,332.07536753)
\curveto(746.85270426,332.28188135)(746.66664976,332.42328277)(746.4247837,332.49957222)
\curveto(746.18290805,332.57584747)(745.76800651,332.61398864)(745.18007784,332.61399586)
\lineto(743.54465713,332.61399586)
\closepath
\moveto(743.54465713,326.36255834)
\lineto(745.58195597,326.36255834)
\curveto(745.93173447,326.36255737)(746.17732642,326.37558119)(746.31873253,326.40162982)
\curveto(746.56804087,326.4462819)(746.77642191,326.5207037)(746.94387628,326.62489545)
\curveto(747.11132001,326.72908474)(747.24900035,326.88071916)(747.35691769,327.07979916)
\curveto(747.46482357,327.2788758)(747.51877937,327.50865311)(747.51878527,327.76913178)
\curveto(747.51877937,328.07425879)(747.44063648,328.33938646)(747.28435636,328.56451557)
\curveto(747.12806492,328.78963835)(746.91131143,328.94778468)(746.63409523,329.03895503)
\curveto(746.35686901,329.13011809)(745.9577821,329.17570144)(745.43683331,329.17570522)
\lineto(743.54465713,329.17570522)
\closepath
}
}
{
\newrgbcolor{curcolor}{0 0 0}
\pscustom[linestyle=none,fillstyle=solid,fillcolor=curcolor]
{
\newpath
\moveto(749.77376866,328.02588725)
\lineto(750.7952089,328.1151935)
\curveto(750.84358153,327.70587088)(750.95614451,327.3700425)(751.13289816,327.10770737)
\curveto(751.30964806,326.84536881)(751.58407845,326.63326667)(751.95619015,326.47140033)
\curveto(752.32829646,326.30953184)(752.74691909,326.22859813)(753.21205929,326.22859896)
\curveto(753.62509634,326.22859813)(753.98976316,326.28999612)(754.30606086,326.41279311)
\curveto(754.62234847,326.53558806)(754.85770741,326.70396739)(755.0121384,326.91793158)
\curveto(755.16655788,327.13189274)(755.2437705,327.36539114)(755.24377649,327.61842748)
\curveto(755.2437705,327.87518047)(755.1693487,328.09937615)(755.02051086,328.29101518)
\curveto(754.8716615,328.48264842)(754.62606956,328.64358557)(754.2837343,328.7738271)
\curveto(754.06418496,328.85940879)(753.57858271,328.99243776)(752.82692609,329.1729144)
\curveto(752.07526233,329.35338349)(751.54872809,329.52362336)(751.24732179,329.68363452)
\curveto(750.85660535,329.88829019)(750.56543005,330.14225458)(750.37379503,330.44552847)
\curveto(750.18215778,330.74879226)(750.08633971,331.08834172)(750.08634054,331.46417788)
\curveto(750.08633971,331.87721281)(750.20355405,332.2632759)(750.4379839,332.62236832)
\curveto(750.67241139,332.98144628)(751.01475167,333.25401612)(751.46500577,333.44007867)
\curveto(751.91525546,333.62612513)(752.41574207,333.71915238)(752.9664671,333.7191607)
\curveto(753.57300107,333.71915238)(754.10790777,333.62147377)(754.57118879,333.42612457)
\curveto(755.03445919,333.23075931)(755.39075356,332.94330511)(755.64007297,332.56376109)
\curveto(755.88937962,332.18420274)(756.02333886,331.75441684)(756.0419511,331.2744021)
\lineto(755.00376594,331.19625913)
\curveto(754.94794384,331.71348485)(754.75909852,332.1041993)(754.43722941,332.36840367)
\curveto(754.11534995,332.63259409)(753.6399807,332.76469278)(753.01112023,332.76470015)
\curveto(752.35620463,332.76469278)(751.87897484,332.64468763)(751.57942941,332.40468433)
\curveto(751.27987934,332.16466702)(751.13010546,331.87535227)(751.13010733,331.53673921)
\curveto(751.13010546,331.24276696)(751.23615653,331.00089611)(751.44826085,330.81112593)
\curveto(751.65663971,330.62134492)(752.20084912,330.42691797)(753.08089074,330.22784448)
\curveto(753.96092471,330.02876133)(754.56467157,329.85480037)(754.89213313,329.70596108)
\curveto(755.36842702,329.48641246)(755.72007003,329.20826098)(755.94706321,328.87150581)
\curveto(756.17404301,328.53474368)(756.28753626,328.14682005)(756.28754328,327.70773373)
\curveto(756.28753626,327.27236389)(756.16287974,326.86211371)(755.91357336,326.47698197)
\curveto(755.66425368,326.09184807)(755.30609876,325.79230033)(754.83910754,325.57833783)
\curveto(754.37210516,325.36437497)(753.84650119,325.25739363)(753.26229406,325.25739349)
\curveto(752.52179314,325.25739363)(751.90130138,325.36530524)(751.40081691,325.58112865)
\curveto(750.90032816,325.79695169)(750.50775316,326.12161679)(750.22309073,326.55512494)
\curveto(749.93842638,326.98863077)(749.78865251,327.47888439)(749.77376866,328.02588725)
\closepath
}
}
{
\newrgbcolor{curcolor}{0 0 0}
\pscustom[linestyle=none,fillstyle=solid,fillcolor=curcolor]
{
\newpath
\moveto(757.77784191,325.39693451)
\lineto(757.77784191,333.57961969)
\lineto(760.59657043,333.57961969)
\curveto(761.23287313,333.5796115)(761.71847538,333.54054006)(762.05337864,333.46240523)
\curveto(762.52223083,333.35448556)(762.922248,333.15912833)(763.25343138,332.87633297)
\curveto(763.68507146,332.51165866)(764.00787602,332.04559213)(764.22184603,331.47813199)
\curveto(764.43580138,330.91065967)(764.54278272,330.26225973)(764.54279036,329.53293022)
\curveto(764.54278272,328.91150405)(764.47022146,328.36078272)(764.32510638,327.88076459)
\curveto(764.17997644,327.40074149)(763.99392194,327.00351513)(763.76694231,326.68908432)
\curveto(763.53994895,326.37465092)(763.29156619,326.12719843)(763.02179329,325.94672611)
\curveto(762.75200814,325.7662527)(762.42641276,325.62950264)(762.04500618,325.53647553)
\curveto(761.6635893,325.44344814)(761.22543095,325.39693451)(760.73052981,325.39693451)
\closepath
\moveto(758.86068019,326.36255834)
\lineto(760.60773371,326.36255834)
\curveto(761.14728806,326.36255737)(761.57056205,326.41279209)(761.87755696,326.51326264)
\curveto(762.1845419,326.61373095)(762.42920357,326.75513237)(762.6115427,326.93746733)
\curveto(762.8682922,327.194221)(763.06830079,327.5393521)(763.21156907,327.97286166)
\curveto(763.35482472,328.40636608)(763.4264557,328.93197005)(763.42646223,329.54967514)
\curveto(763.4264557,330.4055217)(763.28598456,331.06322436)(763.00504837,331.52278511)
\curveto(762.72409996,331.9823336)(762.38268995,332.2902538)(761.98081731,332.44654664)
\curveto(761.6905672,332.55817229)(761.2235704,332.61398864)(760.57982551,332.61399586)
\lineto(758.86068019,332.61399586)
\closepath
}
}
{
\newrgbcolor{curcolor}{0 0 0}
\pscustom[linestyle=none,fillstyle=solid,fillcolor=curcolor]
{
\newpath
\moveto(769.20904003,325.39693451)
\lineto(769.20904003,333.57961969)
\lineto(772.2956873,333.57961969)
\curveto(772.83896248,333.5796115)(773.25386402,333.55356387)(773.54039316,333.50147672)
\curveto(773.94226568,333.43448899)(774.27902432,333.30704166)(774.55067012,333.11913433)
\curveto(774.82230347,332.93121156)(775.04091751,332.66794444)(775.20651289,332.32933218)
\curveto(775.37209452,331.99070606)(775.45488878,331.61859705)(775.4548959,331.21300405)
\curveto(775.45488878,330.5171544)(775.23348392,329.9282919)(774.79068067,329.44641479)
\curveto(774.34786449,328.96452958)(773.54783013,328.723589)(772.39057519,328.72359233)
\lineto(770.29187831,328.72359233)
\lineto(770.29187831,325.39693451)
\closepath
\moveto(770.29187831,329.68921616)
\lineto(772.40732011,329.68921616)
\curveto(773.10688096,329.68921187)(773.60364648,329.81945002)(773.89761817,330.07993101)
\curveto(774.19157871,330.34040262)(774.33856176,330.70692999)(774.33856778,331.17951421)
\curveto(774.33856176,331.52184871)(774.25204642,331.81488455)(774.07902149,332.05862261)
\curveto(773.90598505,332.30234735)(773.67806828,332.46328449)(773.39527051,332.54143453)
\curveto(773.21293203,332.58980155)(772.87617338,332.61398864)(772.38499355,332.61399586)
\lineto(770.29187831,332.61399586)
\closepath
}
}
{
\newrgbcolor{curcolor}{0 0 0}
\pscustom[linestyle=none,fillstyle=solid,fillcolor=curcolor]
{
\newpath
\moveto(780.58442422,326.12812943)
\curveto(780.2123106,325.81183605)(779.85415568,325.58857065)(779.5099584,325.45833256)
\curveto(779.16575403,325.32809434)(778.79643584,325.26297527)(778.40200273,325.26297513)
\curveto(777.75080954,325.26297527)(777.25032293,325.42205187)(776.9005414,325.74020541)
\curveto(776.550758,326.05835826)(776.37586677,326.46488735)(776.37586718,326.95979389)
\curveto(776.37586677,327.25003735)(776.44191612,327.51516501)(776.57401543,327.75517768)
\curveto(776.70611351,327.99518563)(776.8791442,328.18775204)(777.093108,328.33287749)
\curveto(777.30706955,328.47799706)(777.54801013,328.58776922)(777.81593047,328.66219428)
\curveto(778.01314639,328.71428628)(778.31083359,328.76452099)(778.70899297,328.81289858)
\curveto(779.52018785,328.90964351)(780.1174228,329.0249973)(780.50069961,329.1589603)
\curveto(780.50441617,329.29663687)(780.50627671,329.38408249)(780.50628125,329.42129741)
\curveto(780.50627671,329.83061329)(780.41138892,330.11899777)(780.22161758,330.28645171)
\curveto(779.96485811,330.51343331)(779.58344638,330.62692656)(779.07738125,330.62693179)
\curveto(778.6047997,330.62692656)(778.25594751,330.5441323)(778.03082363,330.37854878)
\curveto(777.80569562,330.21295529)(777.63917684,329.91991945)(777.53126679,329.49944038)
\lineto(776.54889804,329.63339975)
\curveto(776.63820362,330.05387869)(776.78518667,330.39342816)(776.98984765,330.65204917)
\curveto(777.19450658,330.91065967)(777.49033324,331.10973799)(777.87732851,331.24928472)
\curveto(778.26431996,331.38881974)(778.71271131,331.45859018)(779.22250391,331.45859624)
\curveto(779.72856889,331.45859018)(780.13974934,331.39905274)(780.45604649,331.27998374)
\curveto(780.77233465,331.16090298)(781.00490278,331.0111291)(781.15375157,330.83066167)
\curveto(781.30258998,330.65018337)(781.4067805,330.42226661)(781.46632344,330.14691069)
\curveto(781.49980775,329.9757358)(781.51655266,329.66688533)(781.51655821,329.22035835)
\lineto(781.51655821,327.88076459)
\curveto(781.51655266,326.94676851)(781.53794892,326.35604547)(781.58074708,326.10859369)
\curveto(781.623534,325.86114049)(781.70818879,325.623921)(781.83471173,325.39693451)
\lineto(780.78536329,325.39693451)
\curveto(780.68116794,325.60531555)(780.61418832,325.84904695)(780.58442422,326.12812943)
\closepath
\moveto(780.50069961,328.37194897)
\curveto(780.13602825,328.22310239)(779.58902802,328.09658533)(778.85969727,327.99239741)
\curveto(778.44665338,327.93285737)(778.15454781,327.86587775)(777.98337969,327.79145834)
\curveto(777.81220752,327.71703415)(777.68010883,327.60819226)(777.5870832,327.46493237)
\curveto(777.49405433,327.32166833)(777.4475407,327.16259173)(777.44754218,326.98770209)
\curveto(777.4475407,326.71978202)(777.5489404,326.49651662)(777.7517416,326.31790521)
\curveto(777.95453922,326.13929197)(778.25129615,326.04998581)(778.64201328,326.04998646)
\curveto(779.02900397,326.04998581)(779.37320479,326.13464061)(779.6746168,326.30395111)
\curveto(779.97602138,326.4732598)(780.19742624,326.70489766)(780.33883204,326.99886537)
\curveto(780.44673927,327.22585026)(780.50069508,327.56074837)(780.50069961,328.00356069)
\closepath
}
}
{
\newrgbcolor{curcolor}{0 0 0}
\pscustom[linestyle=none,fillstyle=solid,fillcolor=curcolor]
{
\newpath
\moveto(785.27300347,326.29557865)
\lineto(785.41812613,325.40809779)
\curveto(785.1353202,325.34856034)(784.88228607,325.31879162)(784.659023,325.31879154)
\curveto(784.29435385,325.31879162)(784.011551,325.37646851)(783.81061363,325.4918224)
\curveto(783.60967328,325.6071761)(783.46827186,325.75881052)(783.38640894,325.94672611)
\curveto(783.3045439,326.13464061)(783.26361191,326.53000643)(783.26361284,327.13282475)
\lineto(783.26361284,330.54320718)
\lineto(782.52683628,330.54320718)
\lineto(782.52683628,331.32463687)
\lineto(783.26361284,331.32463687)
\lineto(783.26361284,332.79260836)
\lineto(784.26272652,333.39542554)
\lineto(784.26272652,331.32463687)
\lineto(785.27300347,331.32463687)
\lineto(785.27300347,330.54320718)
\lineto(784.26272652,330.54320718)
\lineto(784.26272652,327.07700834)
\curveto(784.26272458,326.79048273)(784.28039976,326.60628877)(784.3157521,326.52442592)
\curveto(784.35110047,326.44256081)(784.40877737,326.37744173)(784.48878296,326.3290685)
\curveto(784.56878424,326.28069339)(784.68320776,326.25650631)(784.83205386,326.25650717)
\curveto(784.94368406,326.25650631)(785.09066711,326.26953012)(785.27300347,326.29557865)
\closepath
}
}
{
\newrgbcolor{curcolor}{0 0 0}
\pscustom[linestyle=none,fillstyle=solid,fillcolor=curcolor]
{
\newpath
\moveto(790.11786767,327.56819272)
\lineto(791.10581807,327.43981498)
\curveto(790.99790085,326.75885346)(790.72160991,326.22580732)(790.27694443,325.84067494)
\curveto(789.83226939,325.45554168)(789.28619943,325.26297527)(788.63873291,325.26297513)
\curveto(787.82753213,325.26297527)(787.17541111,325.52810293)(786.68236786,326.05835893)
\curveto(786.18932225,326.58861359)(785.94280003,327.34864623)(785.94280048,328.33845913)
\curveto(785.94280003,328.97848367)(786.0488511,329.53850772)(786.26095399,330.01853296)
\curveto(786.47305536,330.49854895)(786.79585992,330.85856441)(787.22936864,331.09858042)
\curveto(787.6628739,331.33858503)(788.13452206,331.45859018)(788.64431455,331.45859624)
\curveto(789.28805997,331.45859018)(789.81459421,331.29579249)(790.22391885,330.97020269)
\curveto(790.63323402,330.64460174)(790.89557087,330.1822563)(791.01093018,329.58316499)
\lineto(790.03414307,329.43246069)
\curveto(789.94111128,329.83061329)(789.77645304,330.13016104)(789.54016787,330.33110483)
\curveto(789.30387461,330.53203876)(789.01828095,330.63250819)(788.68338603,330.63251343)
\curveto(788.1773146,330.63250819)(787.76613415,330.45110505)(787.44984345,330.08830347)
\curveto(787.13354884,329.7254925)(786.97540252,329.15151436)(786.975404,328.36636733)
\curveto(786.97540252,327.57005109)(787.12796721,326.99142159)(787.43309853,326.63047709)
\curveto(787.73822597,326.26953012)(788.13638261,326.08905726)(788.62756962,326.08905795)
\curveto(789.02200204,326.08905726)(789.35131851,326.20999268)(789.61552002,326.45186459)
\curveto(789.87971329,326.69373439)(790.04716234,327.06584339)(790.11786767,327.56819272)
\closepath
}
}
{
\newrgbcolor{curcolor}{0 0 0}
\pscustom[linestyle=none,fillstyle=solid,fillcolor=curcolor]
{
\newpath
\moveto(791.96538977,325.39693451)
\lineto(791.96538977,333.57961969)
\lineto(792.97008509,333.57961969)
\lineto(792.97008509,330.64367671)
\curveto(793.43894067,331.18695061)(794.03059399,331.45859018)(794.74504681,331.45859624)
\curveto(795.1841319,331.45859018)(795.56554363,331.37207484)(795.88928314,331.19904995)
\curveto(796.2130133,331.02601346)(796.44465115,330.78693343)(796.5841974,330.48180913)
\curveto(796.7237329,330.17667466)(796.79350334,329.73386495)(796.79350892,329.15337866)
\lineto(796.79350892,325.39693451)
\lineto(795.78881361,325.39693451)
\lineto(795.78881361,329.15337866)
\curveto(795.78880903,329.65572206)(795.67996715,330.02131915)(795.46228763,330.25017104)
\curveto(795.24459961,330.47901323)(794.93667941,330.59343675)(794.53852611,330.59344194)
\curveto(794.24083558,330.59343675)(793.96082355,330.51622413)(793.69848919,330.36180386)
\curveto(793.43614986,330.20737366)(793.24916508,329.99806234)(793.13753431,329.73386929)
\curveto(793.02589968,329.46966756)(792.97008333,329.10500073)(792.97008509,328.63986772)
\lineto(792.97008509,325.39693451)
\closepath
}
}
{
\newrgbcolor{curcolor}{0 0 0}
\pscustom[linestyle=none,fillstyle=solid,fillcolor=curcolor]
{
\newpath
\moveto(804.44035402,325.39693451)
\lineto(804.44035402,327.35609037)
\lineto(800.89043058,327.35609037)
\lineto(800.89043058,328.27706108)
\lineto(804.62454817,333.57961969)
\lineto(805.44504934,333.57961969)
\lineto(805.44504934,328.27706108)
\lineto(806.55021419,328.27706108)
\lineto(806.55021419,327.35609037)
\lineto(805.44504934,327.35609037)
\lineto(805.44504934,325.39693451)
\closepath
\moveto(804.44035402,328.27706108)
\lineto(804.44035402,331.96652554)
\lineto(801.87838097,328.27706108)
\closepath
}
}
{
\newrgbcolor{curcolor}{0 0 0}
\pscustom[linestyle=none,fillstyle=solid,fillcolor=curcolor]
{
\newpath
\moveto(807.58840427,327.55702944)
\lineto(808.59309959,327.69098881)
\curveto(808.70845189,327.12165974)(808.90473939,326.71140957)(809.18196268,326.46023705)
\curveto(809.45918181,326.20906241)(809.79687073,326.08347562)(810.19503045,326.08347631)
\curveto(810.6676058,326.08347562)(811.06669271,326.24720358)(811.39229237,326.57466068)
\curveto(811.71788346,326.90211543)(811.88068115,327.30771424)(811.88068592,327.79145834)
\curveto(811.88068115,328.25287111)(811.72997701,328.63335257)(811.42857303,328.93290385)
\curveto(811.12716042,329.23244807)(810.74388815,329.38222194)(810.27875506,329.38222593)
\curveto(810.0889763,329.38222194)(809.85268708,329.34501104)(809.5698867,329.27059311)
\lineto(809.68151951,330.15249233)
\curveto(809.74849656,330.1450454)(809.80245237,330.14132431)(809.84338709,330.14132905)
\curveto(810.27130971,330.14132431)(810.65644253,330.25295701)(810.9987867,330.47622749)
\curveto(811.3411231,330.69948781)(811.51229324,331.04368864)(811.51229764,331.50883101)
\curveto(811.51229324,331.87721281)(811.38763672,332.18234219)(811.13832772,332.42422007)
\curveto(810.88901066,332.6660839)(810.56713637,332.78701932)(810.17270389,332.78702671)
\curveto(809.78198637,332.78701932)(809.45639099,332.66422335)(809.19591678,332.41863843)
\curveto(808.93543839,332.17303947)(808.76798934,331.80465155)(808.69356912,331.31347359)
\lineto(807.68887381,331.49208609)
\curveto(807.8116692,332.16559729)(808.09075095,332.68748017)(808.5261199,333.05773629)
\curveto(808.96148602,333.42797708)(809.50290462,333.61310131)(810.15037733,333.61310953)
\curveto(810.59690509,333.61310131)(811.00808554,333.51728324)(811.38391991,333.32565504)
\curveto(811.75974573,333.13401097)(812.04719993,332.8726044)(812.24628339,332.54143453)
\curveto(812.44535657,332.21025037)(812.54489572,331.85860736)(812.54490116,331.48650445)
\curveto(812.54489572,331.1329948)(812.45000793,330.81112051)(812.26023749,330.52088062)
\curveto(812.07045674,330.23063047)(811.78951445,329.99992289)(811.41740975,329.82875718)
\curveto(811.90114715,329.71712004)(812.27697724,329.48548219)(812.54490116,329.13384292)
\curveto(812.81281421,328.78219617)(812.94677345,328.34217727)(812.94677929,327.81378491)
\curveto(812.94677345,327.0993332)(812.68629715,326.4937258)(812.1653496,325.99696088)
\curveto(811.64439193,325.50019476)(810.985759,325.251812)(810.18944881,325.25181185)
\curveto(809.47127535,325.251812)(808.87497067,325.46577468)(808.40053299,325.89370053)
\curveto(807.92609271,326.32162538)(807.65538341,326.8760678)(807.58840427,327.55702944)
\closepath
}
}
{
\newrgbcolor{curcolor}{0 0 0}
\pscustom[linestyle=none,fillstyle=solid,fillcolor=curcolor]
{
\newpath
\moveto(817.73024261,325.39693451)
\lineto(816.7255473,325.39693451)
\lineto(816.7255473,331.79907632)
\curveto(816.48367319,331.56836234)(816.16645027,331.33765475)(815.77387757,331.10695288)
\curveto(815.38130027,330.87623959)(815.02872699,330.7032089)(814.71615667,330.5878603)
\lineto(814.71615667,331.55906577)
\curveto(815.27804002,331.823257)(815.7692239,332.14327075)(816.1897098,332.51910796)
\curveto(816.61019025,332.89493094)(816.90787746,333.25959776)(817.0827723,333.61310953)
\lineto(817.73024261,333.61310953)
\closepath
}
}
{
\newrgbcolor{curcolor}{0 0 0}
\pscustom[linestyle=none,fillstyle=solid,fillcolor=curcolor]
{
\newpath
\moveto(65.32573637,158.24954156)
\lineto(66.40857466,158.24954156)
\lineto(66.40857466,153.52189194)
\curveto(66.40856732,152.69952759)(66.31554007,152.04647628)(66.12949263,151.56273608)
\curveto(65.94343107,151.07899287)(65.60760269,150.6854876)(65.12200649,150.38221908)
\curveto(64.63639819,150.07894993)(63.99916152,149.92731551)(63.21029457,149.92731537)
\curveto(62.44374589,149.92731551)(61.81674222,150.0594142)(61.32928168,150.32361185)
\curveto(60.84181663,150.58780899)(60.49389471,150.97015099)(60.28551488,151.47063901)
\curveto(60.07713262,151.97112421)(59.9729421,152.65487451)(59.972943,153.52189194)
\lineto(59.972943,158.24954156)
\lineto(61.05578129,158.24954156)
\lineto(61.05578129,153.52747358)
\curveto(61.0557793,152.81674192)(61.12182865,152.2929985)(61.25392953,151.95624174)
\curveto(61.38602604,151.6194812)(61.61301254,151.35993517)(61.93488969,151.17760287)
\curveto(62.25676111,150.99526835)(62.65026639,150.90410164)(63.11540668,150.90410248)
\curveto(63.91171591,150.90410164)(64.47918214,151.08457451)(64.81780708,151.44552162)
\curveto(65.15642053,151.80646598)(65.32573012,152.50044927)(65.32573637,153.52747358)
\closepath
}
}
{
\newrgbcolor{curcolor}{0 0 0}
\pscustom[linestyle=none,fillstyle=solid,fillcolor=curcolor]
{
\newpath
\moveto(68.08864836,150.06685638)
\lineto(68.08864836,155.99455874)
\lineto(68.99287415,155.99455874)
\lineto(68.99287415,155.15173101)
\curveto(69.42824002,155.80291668)(70.05710424,156.12851206)(70.87946868,156.12851812)
\curveto(71.23668978,156.12851206)(71.56507598,156.06432325)(71.86462825,155.93595152)
\curveto(72.16417147,155.80756804)(72.38836715,155.63918872)(72.53721595,155.43081304)
\curveto(72.68605435,155.22242663)(72.79024487,154.97497414)(72.84978783,154.68845483)
\curveto(72.88699321,154.50239571)(72.90559866,154.17680033)(72.90560423,153.71166772)
\lineto(72.90560423,150.06685638)
\lineto(71.90090892,150.06685638)
\lineto(71.90090892,153.67259624)
\curveto(71.90090435,154.08191254)(71.86183291,154.38797219)(71.78369447,154.59077612)
\curveto(71.70554713,154.79357101)(71.56693652,154.95543842)(71.36786224,155.07637886)
\curveto(71.16877989,155.19730927)(70.93528149,155.25777699)(70.66736634,155.25778218)
\curveto(70.23943765,155.25777699)(69.87011947,155.1219572)(69.55941067,154.85032241)
\curveto(69.24869743,154.57867806)(69.09334192,154.06330709)(69.09334368,153.30420795)
\lineto(69.09334368,150.06685638)
\closepath
}
}
{
\newrgbcolor{curcolor}{0 0 0}
\pscustom[linestyle=none,fillstyle=solid,fillcolor=curcolor]
{
\newpath
\moveto(74.45730049,157.09414195)
\lineto(74.45730049,158.24954156)
\lineto(75.46199581,158.24954156)
\lineto(75.46199581,157.09414195)
\closepath
\moveto(74.45730049,150.06685638)
\lineto(74.45730049,155.99455874)
\lineto(75.46199581,155.99455874)
\lineto(75.46199581,150.06685638)
\closepath
}
}
{
\newrgbcolor{curcolor}{0 0 0}
\pscustom[linestyle=none,fillstyle=solid,fillcolor=curcolor]
{
\newpath
\moveto(76.32714998,150.06685638)
\lineto(78.49282655,153.14792202)
\lineto(76.48901756,155.99455874)
\lineto(77.7448867,155.99455874)
\lineto(78.65469413,154.60473022)
\curveto(78.82586186,154.34052829)(78.96354219,154.11912344)(79.06773553,153.94051499)
\curveto(79.23146067,154.18610306)(79.38216482,154.40378682)(79.51984842,154.59356694)
\lineto(80.5189621,155.99455874)
\lineto(81.71901484,155.99455874)
\lineto(79.67055272,153.20373842)
\lineto(81.87530077,150.06685638)
\lineto(80.64175819,150.06685638)
\lineto(79.42496053,151.9087978)
\lineto(79.10122538,152.40556381)
\lineto(77.54394764,150.06685638)
\closepath
}
}
{
\newrgbcolor{curcolor}{0 0 0}
\pscustom[linestyle=none,fillstyle=solid,fillcolor=curcolor]
{
\newpath
\moveto(84.26982556,150.06685638)
\lineto(82.09856735,158.24954156)
\lineto(83.20931384,158.24954156)
\lineto(84.4540197,152.88558491)
\curveto(84.58797645,152.32369749)(84.70333024,151.76553399)(84.80008142,151.21109272)
\curveto(85.00845962,152.08554773)(85.13125559,152.58975543)(85.1684697,152.72371733)
\lineto(86.72574744,158.24954156)
\lineto(88.03185135,158.24954156)
\lineto(89.20399589,154.10796421)
\curveto(89.49795475,153.08093932)(89.71005689,152.11531645)(89.84030292,151.21109272)
\curveto(89.94448556,151.72832309)(90.08030534,152.32183695)(90.24776268,152.99163608)
\lineto(91.53154003,158.24954156)
\lineto(92.61995996,158.24954156)
\lineto(90.37614042,150.06685638)
\lineto(89.33237362,150.06685638)
\lineto(87.60764666,156.30154898)
\curveto(87.4625185,156.82249535)(87.37693343,157.14250909)(87.35089119,157.26159117)
\curveto(87.26530073,156.88575388)(87.18529729,156.56574014)(87.11088065,156.30154898)
\lineto(85.37499041,150.06685638)
\closepath
}
}
{
\newrgbcolor{curcolor}{0 0 0}
\pscustom[linestyle=none,fillstyle=solid,fillcolor=curcolor]
{
\newpath
\moveto(97.36435434,150.79805131)
\curveto(96.99224071,150.48175792)(96.6340858,150.25849252)(96.28988852,150.12825443)
\curveto(95.94568414,149.99801622)(95.57636595,149.93289714)(95.18193285,149.93289701)
\curveto(94.53073965,149.93289714)(94.03025304,150.09197374)(93.68047152,150.41012728)
\curveto(93.33068812,150.72828014)(93.15579688,151.13480923)(93.1557973,151.62971576)
\curveto(93.15579688,151.91995922)(93.22184623,152.18508689)(93.35394554,152.42509955)
\curveto(93.48604362,152.6651075)(93.65907431,152.85767391)(93.87303812,153.00279936)
\curveto(94.08699967,153.14791894)(94.32794025,153.25769109)(94.59586058,153.33211616)
\curveto(94.7930765,153.38420815)(95.0907637,153.43444287)(95.48892308,153.48282046)
\curveto(96.30011797,153.57956538)(96.89735292,153.69491917)(97.28062973,153.82888217)
\curveto(97.28434628,153.96655874)(97.28620683,154.05400436)(97.28621137,154.09121928)
\curveto(97.28620683,154.50053516)(97.19131903,154.78891964)(97.0015477,154.95637358)
\curveto(96.74478823,155.18335519)(96.3633765,155.29684843)(95.85731137,155.29685366)
\curveto(95.38472982,155.29684843)(95.03587763,155.21405418)(94.81075375,155.04847065)
\curveto(94.58562573,154.88287717)(94.41910695,154.58984133)(94.31119691,154.16936225)
\lineto(93.32882816,154.30332163)
\curveto(93.41813373,154.72380057)(93.56511679,155.06335003)(93.76977777,155.32197105)
\curveto(93.97443669,155.58058155)(94.27026335,155.77965987)(94.65725863,155.91920659)
\curveto(95.04425008,156.05874162)(95.49264143,156.12851206)(96.00243402,156.12851812)
\curveto(96.50849901,156.12851206)(96.91967946,156.06897462)(97.2359766,155.94990562)
\curveto(97.55226476,155.83082485)(97.78483289,155.68105098)(97.93368168,155.50058355)
\curveto(98.08252009,155.32010525)(98.18671062,155.09218848)(98.24625356,154.81683257)
\curveto(98.27973787,154.64565768)(98.29648277,154.3368072)(98.29648833,153.89028022)
\lineto(98.29648833,152.55068647)
\curveto(98.29648277,151.61669039)(98.31787904,151.02596734)(98.36067719,150.77851557)
\curveto(98.40346411,150.53106237)(98.48811891,150.29384288)(98.61464184,150.06685638)
\lineto(97.5652934,150.06685638)
\curveto(97.46109806,150.27523743)(97.39411844,150.51896882)(97.36435434,150.79805131)
\closepath
\moveto(97.28062973,153.04187084)
\curveto(96.91595837,152.89302427)(96.36895813,152.76650721)(95.63962738,152.66231928)
\curveto(95.22658349,152.60277925)(94.93447792,152.53579962)(94.7633098,152.46138022)
\curveto(94.59213764,152.38695602)(94.46003894,152.27811414)(94.36701332,152.13485424)
\curveto(94.27398444,151.99159021)(94.22747081,151.83251361)(94.2274723,151.65762397)
\curveto(94.22747081,151.38970389)(94.32887052,151.16643849)(94.53167171,150.98782709)
\curveto(94.73446933,150.80921385)(95.03122626,150.71990769)(95.4219434,150.71990834)
\curveto(95.80893408,150.71990769)(96.15313491,150.80456248)(96.45454691,150.97387299)
\curveto(96.7559515,151.14318168)(96.97735635,151.37481953)(97.11876215,151.66878725)
\curveto(97.22666939,151.89577214)(97.28062519,152.23067024)(97.28062973,152.67348256)
\closepath
}
}
{
\newrgbcolor{curcolor}{0 0 0}
\pscustom[linestyle=none,fillstyle=solid,fillcolor=curcolor]
{
\newpath
\moveto(99.84818554,150.06685638)
\lineto(99.84818554,155.99455874)
\lineto(100.75241132,155.99455874)
\lineto(100.75241132,155.0959146)
\curveto(100.98311726,155.51639274)(101.19614966,155.79361395)(101.39150917,155.92757905)
\curveto(101.58686411,156.06153244)(101.80175706,156.12851206)(102.03618867,156.12851812)
\curveto(102.37480493,156.12851206)(102.71900576,156.02060045)(103.06879218,155.80478296)
\lineto(102.72273047,154.87264897)
\curveto(102.47713491,155.01776668)(102.23154296,155.09032794)(101.9859539,155.09033296)
\curveto(101.76640671,155.09032794)(101.56918894,155.02427859)(101.39429999,154.89218472)
\curveto(101.21940647,154.76008119)(101.09474996,154.57681751)(101.02033007,154.34239311)
\curveto(100.90869545,153.98516419)(100.8528791,153.59444974)(100.85288085,153.17024858)
\lineto(100.85288085,150.06685638)
\closepath
}
}
{
\newrgbcolor{curcolor}{0 0 0}
\pscustom[linestyle=none,fillstyle=solid,fillcolor=curcolor]
{
\newpath
\moveto(107.73504214,151.97577748)
\lineto(108.7732273,151.84739975)
\curveto(108.60949349,151.24086029)(108.30622465,150.7701424)(107.86341987,150.43524467)
\curveto(107.42060522,150.10034619)(106.85499953,149.93289714)(106.16660112,149.93289701)
\curveto(105.2995839,149.93289714)(104.61211251,150.19988535)(104.1041849,150.73386244)
\curveto(103.59625493,151.26783819)(103.34229054,152.01670756)(103.34229096,152.9804728)
\curveto(103.34229054,153.97772201)(103.59904575,154.75170874)(104.11255736,155.3024353)
\curveto(104.6260666,155.85315139)(105.29214172,156.12851206)(106.11078471,156.12851812)
\curveto(106.9033737,156.12851206)(107.55084337,155.85873303)(108.05319565,155.31918023)
\curveto(108.55553768,154.77961692)(108.80671126,154.02051455)(108.80671714,153.04187084)
\curveto(108.80671126,152.98233043)(108.80485071,152.89302427)(108.8011355,152.77395209)
\lineto(104.38047611,152.77395209)
\curveto(104.41768556,152.12275863)(104.60187951,151.62413257)(104.93305854,151.2780724)
\curveto(105.26423354,150.93200982)(105.67727454,150.75897913)(106.17218276,150.75897982)
\curveto(106.54056742,150.75897913)(106.85499953,150.85572747)(107.11548003,151.04922514)
\curveto(107.37595214,151.24272084)(107.58247264,151.55157131)(107.73504214,151.97577748)
\closepath
\moveto(104.43629252,153.60003491)
\lineto(107.74620542,153.60003491)
\curveto(107.70154752,154.09865744)(107.57503046,154.47262699)(107.36665386,154.72194468)
\curveto(107.04663567,155.10893339)(106.63173413,155.30243007)(106.12194799,155.3024353)
\curveto(105.66052963,155.30243007)(105.27260599,155.14800483)(104.95817592,154.83915913)
\curveto(104.64374178,154.53030388)(104.46978082,154.11726289)(104.43629252,153.60003491)
\closepath
}
}
{
\newrgbcolor{curcolor}{0 0 0}
\pscustom[linestyle=none,fillstyle=solid,fillcolor=curcolor]
{
\newpath
\moveto(112.9985305,157.17786656)
\lineto(112.9985305,158.14349039)
\lineto(118.29550747,158.14349039)
\lineto(118.29550747,157.3620607)
\curveto(117.77454902,156.80761099)(117.25824778,156.07083516)(116.74660219,155.15173101)
\curveto(116.23494802,154.23261668)(115.8395822,153.28745981)(115.56050355,152.31625756)
\curveto(115.35956159,151.63157475)(115.23118398,150.8817751)(115.17537035,150.06685638)
\lineto(114.14276683,150.06685638)
\curveto(114.15392842,150.71060496)(114.28044548,151.48831278)(114.52231839,152.39998217)
\curveto(114.76418718,153.3116469)(115.11117883,154.19075442)(115.56329437,155.03730737)
\curveto(116.01540371,155.88385039)(116.49635459,156.5973694)(117.00614848,157.17786656)
\closepath
}
}
{
\newrgbcolor{curcolor}{0 0 0}
\pscustom[linestyle=none,fillstyle=solid,fillcolor=curcolor]
{
\newpath
\moveto(119.85836414,150.06685638)
\lineto(119.85836414,151.21109272)
\lineto(121.00260048,151.21109272)
\lineto(121.00260048,150.06685638)
\closepath
}
}
{
\newrgbcolor{curcolor}{0 0 0}
\pscustom[linestyle=none,fillstyle=solid,fillcolor=curcolor]
{
\newpath
\moveto(126.2493428,150.06685638)
\lineto(125.24464749,150.06685638)
\lineto(125.24464749,156.4689982)
\curveto(125.00277338,156.23828421)(124.68555045,156.00757663)(124.29297776,155.77687476)
\curveto(123.90040046,155.54616147)(123.54782718,155.37313078)(123.23525686,155.25778218)
\lineto(123.23525686,156.22898765)
\curveto(123.79714021,156.49317888)(124.28832409,156.81319262)(124.70880999,157.18902984)
\curveto(125.12929044,157.56485281)(125.42697764,157.92951963)(125.60187249,158.2830314)
\lineto(126.2493428,158.2830314)
\closepath
}
}
{
\newrgbcolor{curcolor}{0 0 0}
\pscustom[linestyle=none,fillstyle=solid,fillcolor=curcolor]
{
\newpath
\moveto(129.39181141,150.06685638)
\lineto(129.39181141,151.21109272)
\lineto(130.53604774,151.21109272)
\lineto(130.53604774,150.06685638)
\closepath
}
}
{
\newrgbcolor{curcolor}{0 0 0}
\pscustom[linestyle=none,fillstyle=solid,fillcolor=curcolor]
{
\newpath
\moveto(135.21904436,150.06685638)
\lineto(135.21904436,152.02601225)
\lineto(131.66912092,152.02601225)
\lineto(131.66912092,152.94698295)
\lineto(135.4032385,158.24954156)
\lineto(136.22373968,158.24954156)
\lineto(136.22373968,152.94698295)
\lineto(137.32890452,152.94698295)
\lineto(137.32890452,152.02601225)
\lineto(136.22373968,152.02601225)
\lineto(136.22373968,150.06685638)
\closepath
\moveto(135.21904436,152.94698295)
\lineto(135.21904436,156.63644742)
\lineto(132.65707131,152.94698295)
\closepath
}
}
{
\newrgbcolor{curcolor}{0 0 0}
\pscustom[linestyle=none,fillstyle=solid,fillcolor=curcolor]
{
\newpath
\moveto(140.53351521,108.42591166)
\lineto(141.55495545,108.51521791)
\curveto(141.60332808,108.10589529)(141.71589106,107.77006692)(141.89264471,107.50773178)
\curveto(142.06939461,107.24539322)(142.343825,107.03329109)(142.7159367,106.87142475)
\curveto(143.08804301,106.70955626)(143.50666564,106.62862255)(143.97180584,106.62862338)
\curveto(144.38484289,106.62862255)(144.74950971,106.69002053)(145.06580741,106.81281752)
\curveto(145.38209502,106.93561248)(145.61745396,107.1039918)(145.77188495,107.317956)
\curveto(145.92630444,107.53191715)(146.00351705,107.76541555)(146.00352304,108.0184519)
\curveto(146.00351705,108.27520489)(145.92909525,108.49940056)(145.78025741,108.69103959)
\curveto(145.63140805,108.88267284)(145.38581611,109.04360998)(145.04348085,109.17385151)
\curveto(144.82393151,109.2594332)(144.33832926,109.39246217)(143.58667264,109.57293882)
\curveto(142.83500889,109.75340791)(142.30847465,109.92364778)(142.00706834,110.08365893)
\curveto(141.6163519,110.2883146)(141.3251766,110.54227899)(141.13354158,110.84555288)
\curveto(140.94190433,111.14881667)(140.84608626,111.48836614)(140.84608709,111.8642023)
\curveto(140.84608626,112.27723722)(140.9633006,112.66330032)(141.19773045,113.02239273)
\curveto(141.43215794,113.38147069)(141.77449823,113.65404054)(142.22475232,113.84010308)
\curveto(142.67500201,114.02614954)(143.17548862,114.11917679)(143.72621366,114.11918512)
\curveto(144.33274763,114.11917679)(144.86765432,114.02149818)(145.33093534,113.82614898)
\curveto(145.79420574,113.63078373)(146.15050011,113.34332952)(146.39981952,112.9637855)
\curveto(146.64912617,112.58422715)(146.78308542,112.15444125)(146.80169765,111.67442652)
\lineto(145.76351249,111.59628355)
\curveto(145.7076904,112.11350926)(145.51884508,112.50422372)(145.19697596,112.76842808)
\curveto(144.8750965,113.0326185)(144.39972725,113.1647172)(143.77086678,113.16472457)
\curveto(143.11595118,113.1647172)(142.63872139,113.04471204)(142.33917596,112.80470875)
\curveto(142.03962589,112.56469143)(141.88985202,112.27537668)(141.88985389,111.93676363)
\curveto(141.88985202,111.64279137)(141.99590308,111.40092052)(142.2080074,111.21115034)
\curveto(142.41638626,111.02136934)(142.96059567,110.82694238)(143.84063729,110.6278689)
\curveto(144.72067126,110.42878575)(145.32441812,110.25482479)(145.65187968,110.1059855)
\curveto(146.12817357,109.88643688)(146.47981658,109.60828539)(146.70680976,109.27153022)
\curveto(146.93378956,108.9347681)(147.04728281,108.54684446)(147.04728984,108.10775815)
\curveto(147.04728281,107.6723883)(146.92262629,107.26213813)(146.67331991,106.87700639)
\curveto(146.42400023,106.49187249)(146.06584531,106.19232474)(145.59885409,105.97836224)
\curveto(145.13185171,105.76439939)(144.60624774,105.65741805)(144.02204061,105.65741791)
\curveto(143.28153969,105.65741805)(142.66104793,105.76532966)(142.16056346,105.98115306)
\curveto(141.66007471,106.1969761)(141.26749971,106.52164121)(140.98283728,106.95514936)
\curveto(140.69817293,107.38865519)(140.54839906,107.8789088)(140.53351521,108.42591166)
\closepath
}
}
{
\newrgbcolor{curcolor}{0 0 0}
\pscustom[linestyle=none,fillstyle=solid,fillcolor=curcolor]
{
\newpath
\moveto(148.03524033,108.7608101)
\curveto(148.03523995,109.8585287)(148.34036933,110.67158687)(148.95062939,111.19998706)
\curveto(149.46041743,111.63907028)(150.08183947,111.8586146)(150.81489736,111.85862066)
\curveto(151.62981292,111.8586146)(152.29588804,111.59162639)(152.81312471,111.05765523)
\curveto(153.33035107,110.52367354)(153.58896683,109.78596744)(153.58897276,108.84453471)
\curveto(153.58896683,108.08170821)(153.47454331,107.48168244)(153.24570186,107.04445561)
\curveto(153.01684923,106.60722628)(152.68381168,106.26767681)(152.24658819,106.02580619)
\curveto(151.80935552,105.78393511)(151.33212572,105.66299968)(150.81489736,105.66299955)
\curveto(149.98509113,105.66299968)(149.31436465,105.92905762)(148.80271591,106.46117416)
\curveto(148.29106489,106.99328937)(148.03523995,107.75983392)(148.03524033,108.7608101)
\closepath
\moveto(149.06784384,108.7608101)
\curveto(149.06784243,108.00170477)(149.23343094,107.43330827)(149.56460986,107.05561889)
\curveto(149.89578497,106.67792699)(150.31254705,106.48908167)(150.81489736,106.48908236)
\curveto(151.31352027,106.48908167)(151.72842181,106.67885726)(152.05960323,107.05840971)
\curveto(152.39077584,107.43795963)(152.55636434,108.01658913)(152.55636924,108.79429995)
\curveto(152.55636434,109.52735169)(152.38984556,110.08272437)(152.05681241,110.46041968)
\curveto(151.72377045,110.83810565)(151.30979918,111.02695097)(150.81489736,111.0269562)
\curveto(150.31254705,111.02695097)(149.89578497,110.83903592)(149.56460986,110.4632105)
\curveto(149.23343094,110.08737574)(149.06784243,109.51990951)(149.06784384,108.7608101)
\closepath
}
}
{
\newrgbcolor{curcolor}{0 0 0}
\pscustom[linestyle=none,fillstyle=solid,fillcolor=curcolor]
{
\newpath
\moveto(154.7499537,105.79695892)
\lineto(154.7499537,113.9796441)
\lineto(155.75464901,113.9796441)
\lineto(155.75464901,105.79695892)
\closepath
}
}
{
\newrgbcolor{curcolor}{0 0 0}
\pscustom[linestyle=none,fillstyle=solid,fillcolor=curcolor]
{
\newpath
\moveto(161.18558618,106.52815385)
\curveto(160.81347256,106.21186046)(160.45531764,105.98859506)(160.11112036,105.85835697)
\curveto(159.76691598,105.72811876)(159.3975978,105.66299968)(159.00316469,105.66299955)
\curveto(158.3519715,105.66299968)(157.85148489,105.82207628)(157.50170336,106.14022982)
\curveto(157.15191996,106.45838268)(156.97702873,106.86491176)(156.97702914,107.3598183)
\curveto(156.97702873,107.65006176)(157.04307807,107.91518943)(157.17517738,108.15520209)
\curveto(157.30727547,108.39521004)(157.48030615,108.58777645)(157.69426996,108.7329019)
\curveto(157.90823151,108.87802147)(158.14917209,108.98779363)(158.41709242,109.0622187)
\curveto(158.61430834,109.11431069)(158.91199555,109.16454541)(159.31015493,109.21292299)
\curveto(160.12134981,109.30966792)(160.71858476,109.42502171)(161.10186157,109.55898471)
\curveto(161.10557812,109.69666128)(161.10743867,109.7841069)(161.10744321,109.82132182)
\curveto(161.10743867,110.2306377)(161.01255087,110.51902218)(160.82277954,110.68647612)
\curveto(160.56602007,110.91345773)(160.18460834,111.02695097)(159.67854321,111.0269562)
\curveto(159.20596166,111.02695097)(158.85710947,110.94415672)(158.63198559,110.77857319)
\curveto(158.40685757,110.61297971)(158.24033879,110.31994386)(158.13242875,109.89946479)
\lineto(157.15006,110.03342417)
\curveto(157.23936557,110.45390311)(157.38634863,110.79345257)(157.59100961,111.05207358)
\curveto(157.79566853,111.31068409)(158.09149519,111.5097624)(158.47849047,111.64930913)
\curveto(158.86548192,111.78884416)(159.31387327,111.8586146)(159.82366586,111.85862066)
\curveto(160.32973085,111.8586146)(160.7409113,111.79907715)(161.05720845,111.68000816)
\curveto(161.37349661,111.56092739)(161.60606473,111.41115352)(161.75491353,111.23068609)
\curveto(161.90375194,111.05020778)(162.00794246,110.82229102)(162.0674854,110.54693511)
\curveto(162.10096971,110.37576022)(162.11771461,110.06690974)(162.11772017,109.62038276)
\lineto(162.11772017,108.28078901)
\curveto(162.11771461,107.34679292)(162.13911088,106.75606988)(162.18190903,106.50861811)
\curveto(162.22469595,106.26116491)(162.30935075,106.02394542)(162.43587368,105.79695892)
\lineto(161.38652524,105.79695892)
\curveto(161.2823299,106.00533997)(161.21535028,106.24907136)(161.18558618,106.52815385)
\closepath
\moveto(161.10186157,108.77197338)
\curveto(160.73719021,108.62312681)(160.19018997,108.49660975)(159.46085922,108.39242182)
\curveto(159.04781533,108.33288178)(158.75570976,108.26590216)(158.58454164,108.19148276)
\curveto(158.41336948,108.11705856)(158.28127078,108.00821668)(158.18824516,107.86495678)
\curveto(158.09521628,107.72169275)(158.04870266,107.56261615)(158.04870414,107.38772651)
\curveto(158.04870266,107.11980643)(158.15010236,106.89654103)(158.35290356,106.71792963)
\curveto(158.55570117,106.53931639)(158.85245811,106.45001023)(159.24317524,106.45001088)
\curveto(159.63016592,106.45001023)(159.97436675,106.53466502)(160.27577876,106.70397553)
\curveto(160.57718334,106.87328422)(160.7985882,107.10492207)(160.93999399,107.39888979)
\curveto(161.04790123,107.62587468)(161.10185703,107.96077278)(161.10186157,108.4035851)
\closepath
}
}
{
\newrgbcolor{curcolor}{0 0 0}
\pscustom[linestyle=none,fillstyle=solid,fillcolor=curcolor]
{
\newpath
\moveto(163.66941547,105.79695892)
\lineto(163.66941547,111.72466128)
\lineto(164.57364125,111.72466128)
\lineto(164.57364125,110.82601714)
\curveto(164.80434719,111.24649528)(165.01737959,111.52371649)(165.21273911,111.65768159)
\curveto(165.40809405,111.79163497)(165.622987,111.8586146)(165.8574186,111.85862066)
\curveto(166.19603486,111.8586146)(166.54023569,111.75070298)(166.89002212,111.5348855)
\lineto(166.5439604,110.60275151)
\curveto(166.29836484,110.74786922)(166.0527729,110.82043047)(165.80718384,110.8204355)
\curveto(165.58763664,110.82043047)(165.39041887,110.75438113)(165.21552993,110.62228726)
\curveto(165.04063641,110.49018373)(164.91597989,110.30692005)(164.84156001,110.07249565)
\curveto(164.72992539,109.71526673)(164.67410904,109.32455228)(164.67411079,108.90035112)
\lineto(164.67411079,105.79695892)
\closepath
}
}
{
\newrgbcolor{curcolor}{0 0 0}
\pscustom[linestyle=none,fillstyle=solid,fillcolor=curcolor]
{
\newpath
\moveto(167.50400288,112.82424449)
\lineto(167.50400288,113.9796441)
\lineto(168.50869819,113.9796441)
\lineto(168.50869819,112.82424449)
\closepath
\moveto(167.50400288,105.79695892)
\lineto(167.50400288,111.72466128)
\lineto(168.50869819,111.72466128)
\lineto(168.50869819,105.79695892)
\closepath
}
}
{
\newrgbcolor{curcolor}{0 0 0}
\pscustom[linestyle=none,fillstyle=solid,fillcolor=curcolor]
{
\newpath
\moveto(169.64177207,107.56633901)
\lineto(170.6353041,107.72262494)
\curveto(170.69111911,107.32446638)(170.84647462,107.019337)(171.1013711,106.80723588)
\curveto(171.35626395,106.59513274)(171.71255832,106.48908167)(172.17025528,106.48908236)
\curveto(172.63166756,106.48908167)(172.97400785,106.58303919)(173.19727715,106.77095522)
\curveto(173.42053865,106.95886929)(173.53217135,107.17934387)(173.53217559,107.43237963)
\curveto(173.53217135,107.65936449)(173.43356247,107.83797681)(173.23634864,107.96821713)
\curveto(173.09866436,108.05752112)(172.75632408,108.17101437)(172.20932676,108.30869721)
\curveto(171.47254802,108.4947492)(170.96182841,108.65568635)(170.67716641,108.79150913)
\curveto(170.39250163,108.92732592)(170.17667841,109.11524096)(170.02969609,109.35525483)
\curveto(169.8827123,109.59526158)(169.80922077,109.86038924)(169.80922129,110.15063862)
\curveto(169.80922077,110.41483166)(169.86968848,110.65949333)(169.99062461,110.88462437)
\curveto(170.11155933,111.10974523)(170.27621757,111.29673)(170.4845998,111.44557925)
\curveto(170.64088439,111.56092739)(170.8539168,111.65860601)(171.12369766,111.73861538)
\curveto(171.39347485,111.81861288)(171.6827896,111.8586146)(171.99164278,111.85862066)
\curveto(172.45677633,111.8586146)(172.86516596,111.79163497)(173.2168129,111.65768159)
\curveto(173.56845198,111.52371649)(173.82799801,111.34231335)(173.99545177,111.11347163)
\curveto(174.16289611,110.88461928)(174.2782499,110.57855962)(174.34151349,110.19529175)
\lineto(173.35914473,110.06133237)
\curveto(173.31448758,110.36645749)(173.18517971,110.60460725)(172.97122071,110.77578237)
\curveto(172.75725435,110.94694754)(172.45491579,111.03253261)(172.06420411,111.03253784)
\curveto(171.60278617,111.03253261)(171.2734697,110.95625026)(171.07625371,110.80369058)
\curveto(170.87903415,110.65112088)(170.78042527,110.47250856)(170.78042676,110.26785307)
\curveto(170.78042527,110.13761045)(170.82135726,110.02039612)(170.90322285,109.91620971)
\curveto(170.98508522,109.80829398)(171.11346283,109.71898782)(171.28835606,109.64829096)
\curveto(171.38882349,109.61107621)(171.68465015,109.52549114)(172.17583692,109.39153549)
\curveto(172.88656223,109.20175631)(173.38239748,109.0464008)(173.66334415,108.9254685)
\curveto(173.94428207,108.80452995)(174.16475666,108.62870844)(174.32476856,108.39800346)
\curveto(174.4847704,108.16729328)(174.56477384,107.88076934)(174.56477911,107.5384308)
\curveto(174.56477384,107.20353096)(174.46709522,106.88816858)(174.27174298,106.59234271)
\curveto(174.07638077,106.29651526)(173.7945082,106.06766822)(173.42612442,105.90580092)
\curveto(173.05773237,105.74393339)(172.64097029,105.66299968)(172.17583692,105.66299955)
\curveto(171.40556839,105.66299968)(170.81856644,105.82300655)(170.4148293,106.14302064)
\curveto(170.0110899,106.46303404)(169.75340442,106.93747302)(169.64177207,107.56633901)
\closepath
}
}
{
\newrgbcolor{curcolor}{0 0 0}
\pscustom[linestyle=none,fillstyle=solid,fillcolor=curcolor]
{
\newpath
\moveto(182.43489349,105.79695892)
\lineto(181.43019818,105.79695892)
\lineto(181.43019818,112.19910074)
\curveto(181.18832407,111.96838675)(180.87110114,111.73767917)(180.47852845,111.5069773)
\curveto(180.08595115,111.276264)(179.73337787,111.10323332)(179.42080755,110.98788472)
\lineto(179.42080755,111.95909019)
\curveto(179.9826909,112.22328142)(180.47387478,112.54329516)(180.89436068,112.91913238)
\curveto(181.31484113,113.29495535)(181.61252833,113.65962217)(181.78742318,114.01313394)
\lineto(182.43489349,114.01313394)
\closepath
}
}
{
\newrgbcolor{curcolor}{0 0 0}
\pscustom[linestyle=none,fillstyle=solid,fillcolor=curcolor]
{
\newpath
\moveto(185.01360877,109.83248511)
\curveto(185.01360829,110.79996448)(185.11314745,111.57860257)(185.31222654,112.16840171)
\curveto(185.51130408,112.75818811)(185.80713074,113.21309137)(186.1997074,113.53311285)
\curveto(186.59228074,113.85311886)(187.08625544,114.01312573)(187.68163299,114.01313394)
\curveto(188.12071847,114.01312573)(188.50585129,113.92474984)(188.8370326,113.74800601)
\curveto(189.16820532,113.57124629)(189.44170544,113.31635162)(189.65753378,112.98332125)
\curveto(189.87335188,112.6502765)(190.04266148,112.24467769)(190.16546308,111.76652359)
\curveto(190.28825342,111.28835755)(190.34965141,110.6436787)(190.34965722,109.83248511)
\curveto(190.34965141,108.87243984)(190.25104252,108.09752284)(190.05383026,107.50773178)
\curveto(189.85660698,106.9179373)(189.56171059,106.46210377)(189.16914022,106.14022982)
\curveto(188.77656059,105.81835519)(188.28072534,105.65741805)(187.68163299,105.65741791)
\curveto(186.89275876,105.65741805)(186.27319727,105.94022089)(185.82294666,106.50582728)
\curveto(185.28338732,107.18678605)(185.01360829,108.29567088)(185.01360877,109.83248511)
\closepath
\moveto(186.04621228,109.83248511)
\curveto(186.04621078,108.48916757)(186.20342683,107.59517568)(186.51786092,107.15050678)
\curveto(186.83229105,106.70583517)(187.22021468,106.48350004)(187.68163299,106.48350072)
\curveto(188.14304501,106.48350004)(188.53096865,106.70676544)(188.84540506,107.1532976)
\curveto(189.15983287,107.59982705)(189.31704892,108.49288866)(189.3170537,109.83248511)
\curveto(189.31704892,111.17951566)(189.15983287,112.07443782)(188.84540506,112.51725425)
\curveto(188.53096865,112.96005725)(188.13932392,113.1814621)(187.67046971,113.18146949)
\curveto(187.20905141,113.1814621)(186.8406635,112.98610488)(186.56530486,112.59539722)
\curveto(186.21924146,112.09676436)(186.04621078,111.17579457)(186.04621228,109.83248511)
\closepath
}
}
{
\newrgbcolor{curcolor}{0 0 0}
\pscustom[linestyle=none,fillstyle=solid,fillcolor=curcolor]
{
\newpath
\moveto(467.79572373,105.89693451)
\lineto(467.79572373,114.07961969)
\lineto(468.90647022,114.07961969)
\lineto(473.20433351,107.65515131)
\lineto(473.20433351,114.07961969)
\lineto(474.24251867,114.07961969)
\lineto(474.24251867,105.89693451)
\lineto(473.13177218,105.89693451)
\lineto(468.83390889,112.32698453)
\lineto(468.83390889,105.89693451)
\closepath
}
}
{
\newrgbcolor{curcolor}{0 0 0}
\pscustom[linestyle=none,fillstyle=solid,fillcolor=curcolor]
{
\newpath
\moveto(479.99719004,107.80585561)
\lineto(481.0353752,107.67747787)
\curveto(480.87164139,107.07093842)(480.56837255,106.60022053)(480.12556778,106.26532279)
\curveto(479.68275312,105.93042432)(479.11714744,105.76297527)(478.42874902,105.76297513)
\curveto(477.5617318,105.76297527)(476.87426042,106.02996348)(476.36633281,106.56394057)
\curveto(475.85840284,107.09791632)(475.60443844,107.84678569)(475.60443886,108.81055092)
\curveto(475.60443844,109.80780014)(475.86119365,110.58178687)(476.37470527,111.13251343)
\curveto(476.8882145,111.68322952)(477.55428962,111.95859018)(478.37293262,111.95859624)
\curveto(479.16552161,111.95859018)(479.81299127,111.68881115)(480.31534356,111.14925835)
\curveto(480.81768558,110.60969504)(481.06885916,109.85059267)(481.06886504,108.87194897)
\curveto(481.06885916,108.81240855)(481.06699862,108.72310239)(481.0632834,108.60403022)
\lineto(476.64262402,108.60403022)
\curveto(476.67983346,107.95283676)(476.86402742,107.45421069)(477.19520644,107.10815053)
\curveto(477.52638144,106.76208794)(477.93942244,106.58905726)(478.43433066,106.58905795)
\curveto(478.80271533,106.58905726)(479.11714744,106.6858056)(479.37762793,106.87930326)
\curveto(479.63810004,107.07279896)(479.84462054,107.38164943)(479.99719004,107.80585561)
\closepath
\moveto(476.69844042,109.43011303)
\lineto(480.00835332,109.43011303)
\curveto(479.96369542,109.92873557)(479.83717836,110.30270511)(479.62880176,110.5520228)
\curveto(479.30878357,110.93901151)(478.89388203,111.13250819)(478.3840959,111.13251343)
\curveto(477.92267753,111.13250819)(477.5347539,110.97808296)(477.22032382,110.66923726)
\curveto(476.90588968,110.36038201)(476.73192872,109.94734102)(476.69844042,109.43011303)
\closepath
}
}
{
\newrgbcolor{curcolor}{0 0 0}
\pscustom[linestyle=none,fillstyle=solid,fillcolor=curcolor]
{
\newpath
\moveto(484.49599256,106.79557865)
\lineto(484.64111521,105.90809779)
\curveto(484.35830928,105.84856034)(484.10527516,105.81879162)(483.88201209,105.81879154)
\curveto(483.51734293,105.81879162)(483.23454009,105.87646851)(483.03360271,105.9918224)
\curveto(482.83266236,106.1071761)(482.69126094,106.25881052)(482.60939802,106.44672611)
\curveto(482.52753298,106.63464061)(482.48660099,107.03000643)(482.48660193,107.63282475)
\lineto(482.48660193,111.04320718)
\lineto(481.74982536,111.04320718)
\lineto(481.74982536,111.82463687)
\lineto(482.48660193,111.82463687)
\lineto(482.48660193,113.29260836)
\lineto(483.4857156,113.89542554)
\lineto(483.4857156,111.82463687)
\lineto(484.49599256,111.82463687)
\lineto(484.49599256,111.04320718)
\lineto(483.4857156,111.04320718)
\lineto(483.4857156,107.57700834)
\curveto(483.48571366,107.29048273)(483.50338884,107.10628877)(483.53874119,107.02442592)
\curveto(483.57408955,106.94256081)(483.63176645,106.87744173)(483.71177205,106.8290685)
\curveto(483.79177332,106.78069339)(483.90619684,106.75650631)(484.05504295,106.75650717)
\curveto(484.16667314,106.75650631)(484.3136562,106.76953012)(484.49599256,106.79557865)
\closepath
}
}
{
\newrgbcolor{curcolor}{0 0 0}
\pscustom[linestyle=none,fillstyle=solid,fillcolor=curcolor]
{
\newpath
\moveto(485.55650345,105.89693451)
\lineto(485.55650345,114.07961969)
\lineto(488.6264058,114.07961969)
\curveto(489.25154502,114.0796115)(489.7529619,113.99681725)(490.13065795,113.83123668)
\curveto(490.50834318,113.66564024)(490.80416984,113.41074557)(491.01813882,113.06655191)
\curveto(491.23209519,112.72234391)(491.33907653,112.36232845)(491.33908315,111.98650445)
\curveto(491.33907653,111.63671589)(491.24418874,111.30739942)(491.05441948,110.99855405)
\curveto(490.86463755,110.68969848)(490.57811362,110.44038545)(490.19484682,110.25061421)
\curveto(490.68974632,110.10548734)(491.07022778,109.85803485)(491.33629233,109.508256)
\curveto(491.60234365,109.15846993)(491.73537262,108.74542893)(491.73537964,108.26913178)
\curveto(491.73537262,107.88585713)(491.65443891,107.52956276)(491.49257827,107.2002476)
\curveto(491.33070408,106.87092983)(491.13069549,106.61696543)(490.8925519,106.43835365)
\curveto(490.65439597,106.25974079)(490.35577849,106.12485127)(489.99669858,106.03368471)
\curveto(489.63760811,105.94251786)(489.19758922,105.89693451)(488.67664057,105.89693451)
\closepath
\moveto(486.63934173,110.64132905)
\lineto(488.40872182,110.64132905)
\curveto(488.88873874,110.64132431)(489.23293957,110.67295357)(489.44132533,110.73621694)
\curveto(489.71668128,110.81807608)(489.92413205,110.95389587)(490.06367827,111.14367671)
\curveto(490.2032138,111.33344705)(490.27298424,111.57159682)(490.27298979,111.85812671)
\curveto(490.27298424,112.12976032)(490.20786516,112.36884036)(490.07763237,112.57536753)
\curveto(489.94738886,112.78188135)(489.76133436,112.92328277)(489.5194683,112.99957222)
\curveto(489.27759265,113.07584747)(488.86269111,113.11398864)(488.27476244,113.11399586)
\lineto(486.63934173,113.11399586)
\closepath
\moveto(486.63934173,106.86255834)
\lineto(488.67664057,106.86255834)
\curveto(489.02641907,106.86255737)(489.27201102,106.87558119)(489.41341713,106.90162982)
\curveto(489.66272547,106.9462819)(489.87110651,107.0207037)(490.03856088,107.12489545)
\curveto(490.20600462,107.22908474)(490.34368495,107.38071916)(490.45160229,107.57979916)
\curveto(490.55950817,107.7788758)(490.61346397,108.00865311)(490.61346987,108.26913178)
\curveto(490.61346397,108.57425879)(490.53532108,108.83938646)(490.37904096,109.06451557)
\curveto(490.22274952,109.28963835)(490.00599603,109.44778468)(489.72877983,109.53895503)
\curveto(489.45155361,109.63011809)(489.0524667,109.67570144)(488.53151791,109.67570522)
\lineto(486.63934173,109.67570522)
\closepath
}
}
{
\newrgbcolor{curcolor}{0 0 0}
\pscustom[linestyle=none,fillstyle=solid,fillcolor=curcolor]
{
\newpath
\moveto(492.86845326,108.52588725)
\lineto(493.8898935,108.6151935)
\curveto(493.93826613,108.20587088)(494.05082911,107.8700425)(494.22758276,107.60770737)
\curveto(494.40433266,107.34536881)(494.67876305,107.13326667)(495.05087475,106.97140033)
\curveto(495.42298106,106.80953184)(495.84160369,106.72859813)(496.30674389,106.72859896)
\curveto(496.71978094,106.72859813)(497.08444776,106.78999612)(497.40074546,106.91279311)
\curveto(497.71703307,107.03558806)(497.95239201,107.20396739)(498.106823,107.41793158)
\curveto(498.26124248,107.63189274)(498.3384551,107.86539114)(498.33846109,108.11842748)
\curveto(498.3384551,108.37518047)(498.2640333,108.59937615)(498.11519546,108.79101518)
\curveto(497.9663461,108.98264842)(497.72075416,109.14358557)(497.3784189,109.2738271)
\curveto(497.15886956,109.35940879)(496.67326731,109.49243776)(495.92161069,109.6729144)
\curveto(495.16994694,109.85338349)(494.6434127,110.02362336)(494.34200639,110.18363452)
\curveto(493.95128995,110.38829019)(493.66011465,110.64225458)(493.46847963,110.94552847)
\curveto(493.27684238,111.24879226)(493.18102431,111.58834172)(493.18102514,111.96417788)
\curveto(493.18102431,112.37721281)(493.29823865,112.7632759)(493.5326685,113.12236832)
\curveto(493.76709599,113.48144628)(494.10943627,113.75401612)(494.55969037,113.94007867)
\curveto(495.00994006,114.12612513)(495.51042667,114.21915238)(496.06115171,114.2191607)
\curveto(496.66768568,114.21915238)(497.20259237,114.12147377)(497.66587339,113.92612457)
\curveto(498.12914379,113.73075931)(498.48543816,113.44330511)(498.73475757,113.06376109)
\curveto(498.98406422,112.68420274)(499.11802347,112.25441684)(499.1366357,111.7744021)
\lineto(498.09845054,111.69625913)
\curveto(498.04262844,112.21348485)(497.85378313,112.6041993)(497.53191401,112.86840367)
\curveto(497.21003455,113.13259409)(496.7346653,113.26469278)(496.10580483,113.26470015)
\curveto(495.45088923,113.26469278)(494.97365944,113.14468763)(494.67411401,112.90468433)
\curveto(494.37456394,112.66466702)(494.22479007,112.37535227)(494.22479194,112.03673921)
\curveto(494.22479007,111.74276696)(494.33084113,111.50089611)(494.54294545,111.31112593)
\curveto(494.75132431,111.12134492)(495.29553372,110.92691797)(496.17557534,110.72784448)
\curveto(497.05560931,110.52876133)(497.65935617,110.35480037)(497.98681773,110.20596108)
\curveto(498.46311162,109.98641246)(498.81475463,109.70826098)(499.04174781,109.37150581)
\curveto(499.26872761,109.03474368)(499.38222086,108.64682005)(499.38222789,108.20773373)
\curveto(499.38222086,107.77236389)(499.25756434,107.36211371)(499.00825796,106.97698197)
\curveto(498.75893828,106.59184807)(498.40078336,106.29230033)(497.93379214,106.07833783)
\curveto(497.46678976,105.86437497)(496.94118579,105.75739363)(496.35697866,105.75739349)
\curveto(495.61647774,105.75739363)(494.99598598,105.86530524)(494.49550151,106.08112865)
\curveto(493.99501276,106.29695169)(493.60243776,106.62161679)(493.31777533,107.05512494)
\curveto(493.03311098,107.48863077)(492.88333711,107.97888439)(492.86845326,108.52588725)
\closepath
}
}
{
\newrgbcolor{curcolor}{0 0 0}
\pscustom[linestyle=none,fillstyle=solid,fillcolor=curcolor]
{
\newpath
\moveto(500.87252651,105.89693451)
\lineto(500.87252651,114.07961969)
\lineto(503.69125503,114.07961969)
\curveto(504.32755773,114.0796115)(504.81315998,114.04054006)(505.14806324,113.96240523)
\curveto(505.61691543,113.85448556)(506.01693261,113.65912833)(506.34811598,113.37633297)
\curveto(506.77975606,113.01165866)(507.10256062,112.54559213)(507.31653063,111.97813199)
\curveto(507.53048598,111.41065967)(507.63746732,110.76225973)(507.63747496,110.03293022)
\curveto(507.63746732,109.41150405)(507.56490606,108.86078272)(507.41979098,108.38076459)
\curveto(507.27466104,107.90074149)(507.08860654,107.50351513)(506.86162691,107.18908432)
\curveto(506.63463355,106.87465092)(506.38625079,106.62719843)(506.11647789,106.44672611)
\curveto(505.84669274,106.2662527)(505.52109736,106.12950264)(505.13969078,106.03647553)
\curveto(504.7582739,105.94344814)(504.32011555,105.89693451)(503.82521441,105.89693451)
\closepath
\moveto(501.95536479,106.86255834)
\lineto(503.70241831,106.86255834)
\curveto(504.24197266,106.86255737)(504.66524665,106.91279209)(504.97224156,107.01326264)
\curveto(505.27922651,107.11373095)(505.52388818,107.25513237)(505.7062273,107.43746733)
\curveto(505.9629768,107.694221)(506.16298539,108.0393521)(506.30625367,108.47286166)
\curveto(506.44950932,108.90636608)(506.52114031,109.43197005)(506.52114684,110.04967514)
\curveto(506.52114031,110.9055217)(506.38066916,111.56322436)(506.09973297,112.02278511)
\curveto(505.81878456,112.4823336)(505.47737455,112.7902538)(505.07550191,112.94654664)
\curveto(504.7852518,113.05817229)(504.318255,113.11398864)(503.67451011,113.11399586)
\lineto(501.95536479,113.11399586)
\closepath
}
}
{
\newrgbcolor{curcolor}{0 0 0}
\pscustom[linestyle=none,fillstyle=solid,fillcolor=curcolor]
{
\newpath
\moveto(511.9018465,108.05702944)
\lineto(512.90654182,108.19098881)
\curveto(513.02189413,107.62165974)(513.21818162,107.21140957)(513.49540491,106.96023705)
\curveto(513.77262404,106.70906241)(514.11031296,106.58347562)(514.50847268,106.58347631)
\curveto(514.98104803,106.58347562)(515.38013494,106.74720358)(515.7057346,107.07466068)
\curveto(516.03132569,107.40211543)(516.19412338,107.80771424)(516.19412815,108.29145834)
\curveto(516.19412338,108.75287111)(516.04341924,109.13335257)(515.74201526,109.43290385)
\curveto(515.44060265,109.73244807)(515.05733038,109.88222194)(514.59219729,109.88222593)
\curveto(514.40241853,109.88222194)(514.16612931,109.84501104)(513.88332893,109.77059311)
\lineto(513.99496174,110.65249233)
\curveto(514.06193879,110.6450454)(514.1158946,110.64132431)(514.15682932,110.64132905)
\curveto(514.58475194,110.64132431)(514.96988476,110.75295701)(515.31222893,110.97622749)
\curveto(515.65456533,111.19948781)(515.82573547,111.54368864)(515.82573987,112.00883101)
\curveto(515.82573547,112.37721281)(515.70107895,112.68234219)(515.45176995,112.92422007)
\curveto(515.20245289,113.1660839)(514.8805786,113.28701932)(514.48614612,113.28702671)
\curveto(514.0954286,113.28701932)(513.76983322,113.16422335)(513.50935901,112.91863843)
\curveto(513.24888062,112.67303947)(513.08143157,112.30465155)(513.00701135,111.81347359)
\lineto(512.00231604,111.99208609)
\curveto(512.12511143,112.66559729)(512.40419318,113.18748017)(512.83956213,113.55773629)
\curveto(513.27492825,113.92797708)(513.81634685,114.11310131)(514.46381956,114.11310953)
\curveto(514.91034732,114.11310131)(515.32152777,114.01728324)(515.69736214,113.82565504)
\curveto(516.07318796,113.63401097)(516.36064216,113.3726044)(516.55972562,113.04143453)
\curveto(516.7587988,112.71025037)(516.85833795,112.35860736)(516.85834339,111.98650445)
\curveto(516.85833795,111.6329948)(516.76345016,111.31112051)(516.57367972,111.02088062)
\curveto(516.38389897,110.73063047)(516.10295668,110.49992289)(515.73085198,110.32875718)
\curveto(516.21458938,110.21712004)(516.59041947,109.98548219)(516.85834339,109.63384292)
\curveto(517.12625644,109.28219617)(517.26021568,108.84217727)(517.26022152,108.31378491)
\curveto(517.26021568,107.5993332)(516.99973938,106.9937258)(516.47879183,106.49696088)
\curveto(515.95783417,106.00019476)(515.29920123,105.751812)(514.50289104,105.75181185)
\curveto(513.78471758,105.751812)(513.1884129,105.96577468)(512.71397522,106.39370053)
\curveto(512.23953494,106.82162538)(511.96882564,107.3760678)(511.9018465,108.05702944)
\closepath
}
}
{
\newrgbcolor{curcolor}{0 0 0}
\pscustom[linestyle=none,fillstyle=solid,fillcolor=curcolor]
{
\newpath
\moveto(518.82308201,105.89693451)
\lineto(518.82308201,107.04117084)
\lineto(519.96731834,107.04117084)
\lineto(519.96731834,105.89693451)
\closepath
}
}
{
\newrgbcolor{curcolor}{0 0 0}
\pscustom[linestyle=none,fillstyle=solid,fillcolor=curcolor]
{
\newpath
\moveto(521.42970831,109.93246069)
\curveto(521.42970784,110.89994007)(521.529247,111.67857816)(521.72832609,112.2683773)
\curveto(521.92740363,112.8581637)(522.22323029,113.31306695)(522.61580695,113.63308843)
\curveto(523.00838029,113.95309444)(523.50235499,114.11310131)(524.09773254,114.11310953)
\curveto(524.53681802,114.11310131)(524.92195084,114.02472542)(525.25313215,113.8479816)
\curveto(525.58430487,113.67122187)(525.85780498,113.4163272)(526.07363333,113.08329683)
\curveto(526.28945143,112.75025209)(526.45876103,112.34465327)(526.58156262,111.86649917)
\curveto(526.70435297,111.38833313)(526.76575095,110.74365428)(526.76575676,109.93246069)
\curveto(526.76575095,108.97241543)(526.66714207,108.19749843)(526.46992981,107.60770737)
\curveto(526.27270652,107.01791288)(525.97781014,106.56207935)(525.58523977,106.24020541)
\curveto(525.19266014,105.91833078)(524.69682489,105.75739363)(524.09773254,105.75739349)
\curveto(523.30885831,105.75739363)(522.68929682,106.04019648)(522.23904621,106.60580287)
\curveto(521.69948687,107.28676164)(521.42970784,108.39564647)(521.42970831,109.93246069)
\closepath
\moveto(522.46231183,109.93246069)
\curveto(522.46231033,108.58914315)(522.61952638,107.69515127)(522.93396047,107.25048236)
\curveto(523.2483906,106.80581075)(523.63631423,106.58347562)(524.09773254,106.58347631)
\curveto(524.55914456,106.58347562)(524.9470682,106.80674102)(525.26150461,107.25327318)
\curveto(525.57593241,107.69980263)(525.73314847,108.59286424)(525.73315325,109.93246069)
\curveto(525.73314847,111.27949125)(525.57593241,112.1744134)(525.26150461,112.61722984)
\curveto(524.9470682,113.06003283)(524.55542347,113.28143769)(524.08656926,113.28144507)
\curveto(523.62515096,113.28143769)(523.25676305,113.08608046)(522.98140441,112.69537281)
\curveto(522.63534101,112.19673994)(522.46231033,111.27577016)(522.46231183,109.93246069)
\closepath
}
}
{
\newrgbcolor{curcolor}{0 0 0}
\pscustom[linestyle=none,fillstyle=solid,fillcolor=curcolor]
{
\newpath
\moveto(331.74157778,68.45217221)
\curveto(331.74157723,69.81036609)(332.10624406,70.87366757)(332.83557935,71.64207984)
\curveto(333.56491135,72.41047776)(334.50634713,72.7946803)(335.65988951,72.79468863)
\curveto(336.41526632,72.7946803)(337.0962258,72.61420744)(337.70276999,72.25326949)
\curveto(338.30930115,71.89231597)(338.77164659,71.38903854)(339.08980768,70.7434357)
\curveto(339.40795298,70.0978203)(339.56702958,69.36569583)(339.56703796,68.54706011)
\curveto(339.56702958,67.71725295)(339.39958053,66.97489548)(339.0646903,66.31998549)
\curveto(338.72978432,65.66507179)(338.25534534,65.16923654)(337.64137194,64.83247826)
\curveto(337.02738563,64.49571925)(336.3650316,64.32733992)(335.65430787,64.32733978)
\curveto(334.88403777,64.32733992)(334.19563611,64.51339442)(333.58910084,64.88550385)
\curveto(332.98256076,65.25761243)(332.52300614,65.76554122)(332.2104356,66.40929174)
\curveto(331.89786301,67.05303837)(331.74157723,67.73399785)(331.74157778,68.45217221)
\closepath
\moveto(332.85790591,68.43542729)
\curveto(332.85790424,67.44933446)(333.12303191,66.67255692)(333.6532897,66.10509233)
\curveto(334.18354257,65.53762446)(334.84868741,65.25389134)(335.64872623,65.25389213)
\curveto(336.46364049,65.25389134)(337.13436697,65.54041527)(337.66090768,66.11346479)
\curveto(338.18743545,66.68651101)(338.45070257,67.49956918)(338.45070983,68.55264175)
\curveto(338.45070257,69.21871278)(338.3381396,69.8001331)(338.11302057,70.29690445)
\curveto(337.8878877,70.79366414)(337.55857123,71.17879695)(337.12507018,71.45230406)
\curveto(336.69155725,71.72579719)(336.20502473,71.86254725)(335.66547115,71.86255464)
\curveto(334.89892213,71.86254725)(334.23935892,71.59928013)(333.68677955,71.07275249)
\curveto(333.13419518,70.54621165)(332.85790424,69.66710413)(332.85790591,68.43542729)
\closepath
}
}
{
\newrgbcolor{curcolor}{0 0 0}
\pscustom[linestyle=none,fillstyle=solid,fillcolor=curcolor]
{
\newpath
\moveto(340.82848887,62.19515306)
\lineto(340.82848887,70.39458316)
\lineto(341.74387794,70.39458316)
\lineto(341.74387794,69.62431675)
\curveto(341.95969949,69.92571988)(342.20343089,70.1517761)(342.47507286,70.30248609)
\curveto(342.74671003,70.4531844)(343.0760265,70.52853647)(343.46302325,70.52854253)
\curveto(343.96908811,70.52853647)(344.41561891,70.39829832)(344.80261701,70.13782769)
\curveto(345.18960564,69.87734571)(345.48171121,69.50988807)(345.67893459,69.03545366)
\curveto(345.87614675,68.56101011)(345.97475564,68.04098778)(345.97476154,67.4753851)
\curveto(345.97475564,66.86884442)(345.86591376,66.32277446)(345.64823556,65.83717358)
\curveto(345.43054622,65.35156996)(345.11425357,64.97946095)(344.69935665,64.72084545)
\curveto(344.28445049,64.46222944)(343.84815268,64.33292156)(343.39046192,64.33292142)
\curveto(343.0555605,64.33292156)(342.75508248,64.40362227)(342.48902696,64.54502377)
\curveto(342.22296661,64.68642511)(342.00435257,64.86503743)(341.83318419,65.08086127)
\lineto(341.83318419,62.19515306)
\closepath
\moveto(341.73829629,67.39724213)
\curveto(341.73829463,66.63441575)(341.89271987,66.0706706)(342.20157247,65.70600502)
\curveto(342.51042081,65.34133696)(342.88439036,65.15900355)(343.32348224,65.15900424)
\curveto(343.77000979,65.15900355)(344.15235179,65.34784887)(344.47050939,65.72554076)
\curveto(344.78865819,66.10323014)(344.94773479,66.68837155)(344.94773966,67.48096674)
\curveto(344.94773479,68.23634501)(344.79237928,68.80195069)(344.48167267,69.1777855)
\curveto(344.17095724,69.55361088)(343.79977851,69.74152593)(343.36813536,69.7415312)
\curveto(342.94020671,69.74152593)(342.5615858,69.54151734)(342.23227149,69.14150483)
\curveto(341.90295287,68.74148298)(341.73829463,68.16006266)(341.73829629,67.39724213)
\closepath
}
}
{
\newrgbcolor{curcolor}{0 0 0}
\pscustom[linestyle=none,fillstyle=solid,fillcolor=curcolor]
{
\newpath
\moveto(351.2494121,66.3758019)
\lineto(352.28759726,66.24742416)
\curveto(352.12386345,65.64088471)(351.82059461,65.17016682)(351.37778984,64.83526908)
\curveto(350.93497518,64.50037061)(350.3693695,64.33292156)(349.68097108,64.33292142)
\curveto(348.81395386,64.33292156)(348.12648248,64.59990977)(347.61855487,65.13388685)
\curveto(347.1106249,65.66786261)(346.8566605,66.41673198)(346.85666092,67.38049721)
\curveto(346.8566605,68.37774643)(347.11341571,69.15173316)(347.62692733,69.70245972)
\curveto(348.14043656,70.25317581)(348.80651168,70.52853647)(349.62515468,70.52854253)
\curveto(350.41774367,70.52853647)(351.06521333,70.25875744)(351.56756562,69.71920464)
\curveto(352.06990764,69.17964133)(352.32108122,68.42053896)(352.32108711,67.44189526)
\curveto(352.32108122,67.38235484)(352.31922068,67.29304868)(352.31550546,67.17397651)
\lineto(347.89484608,67.17397651)
\curveto(347.93205552,66.52278304)(348.11624948,66.02415698)(348.4474285,65.67809682)
\curveto(348.77860351,65.33203423)(349.1916445,65.15900355)(349.68655272,65.15900424)
\curveto(350.05493739,65.15900355)(350.3693695,65.25575189)(350.62984999,65.44924955)
\curveto(350.8903221,65.64274525)(351.0968426,65.95159572)(351.2494121,66.3758019)
\closepath
\moveto(347.95066249,68.00005932)
\lineto(351.26057538,68.00005932)
\curveto(351.21591748,68.49868185)(351.08940042,68.8726514)(350.88102382,69.12196909)
\curveto(350.56100563,69.5089578)(350.1461041,69.70245448)(349.63631796,69.70245972)
\curveto(349.1748996,69.70245448)(348.78697596,69.54802925)(348.47254589,69.23918354)
\curveto(348.15811174,68.9303283)(347.98415078,68.5172873)(347.95066249,68.00005932)
\closepath
}
}
{
\newrgbcolor{curcolor}{0 0 0}
\pscustom[linestyle=none,fillstyle=solid,fillcolor=curcolor]
{
\newpath
\moveto(353.5546308,64.4668808)
\lineto(353.5546308,70.39458316)
\lineto(354.45885658,70.39458316)
\lineto(354.45885658,69.55175542)
\curveto(354.89422246,70.20294109)(355.52308668,70.52853647)(356.34545112,70.52854253)
\curveto(356.70267222,70.52853647)(357.03105841,70.46434767)(357.33061069,70.33597593)
\curveto(357.63015391,70.20759245)(357.85434959,70.03921313)(358.00319839,69.83083745)
\curveto(358.15203679,69.62245105)(358.25622731,69.37499856)(358.31577026,69.08847925)
\curveto(358.35297565,68.90242012)(358.3715811,68.57682475)(358.37158667,68.11169214)
\lineto(358.37158667,64.4668808)
\lineto(357.36689136,64.4668808)
\lineto(357.36689136,68.07262065)
\curveto(357.36688679,68.48193695)(357.32781535,68.78799661)(357.2496769,68.99080054)
\curveto(357.17152956,69.19359542)(357.03291896,69.35546284)(356.83384468,69.47640327)
\curveto(356.63476233,69.59733369)(356.40126393,69.6578014)(356.13334878,69.65780659)
\curveto(355.70542009,69.6578014)(355.3361019,69.52198162)(355.02539311,69.25034683)
\curveto(354.71467987,68.97870247)(354.55932436,68.4633315)(354.55932612,67.70423237)
\lineto(354.55932612,64.4668808)
\closepath
}
}
{
\newrgbcolor{curcolor}{0 0 0}
\pscustom[linestyle=none,fillstyle=solid,fillcolor=curcolor]
{
\newpath
\moveto(360.00142494,64.4668808)
\lineto(360.00142494,72.64956597)
\lineto(363.0713273,72.64956597)
\curveto(363.69646651,72.64955779)(364.1978834,72.56676354)(364.57557945,72.40118297)
\curveto(364.95326467,72.23558653)(365.24909133,71.98069186)(365.46306031,71.6364982)
\curveto(365.67701669,71.2922902)(365.78399803,70.93227474)(365.78400465,70.55645074)
\curveto(365.78399803,70.20666218)(365.68911023,69.87734571)(365.49934097,69.56850034)
\curveto(365.30955905,69.25964477)(365.02303511,69.01033173)(364.63976831,68.8205605)
\curveto(365.13466781,68.67543363)(365.51514927,68.42798114)(365.78121383,68.07820229)
\curveto(366.04726515,67.72841622)(366.18029411,67.31537522)(366.18030113,66.83907807)
\curveto(366.18029411,66.45580342)(366.09936041,66.09950905)(365.93749976,65.77019389)
\curveto(365.77562557,65.44087612)(365.57561698,65.18691172)(365.33747339,65.00829994)
\curveto(365.09931746,64.82968708)(364.80069998,64.69479756)(364.44162007,64.60363099)
\curveto(364.08252961,64.51246415)(363.64251071,64.4668808)(363.12156206,64.4668808)
\closepath
\moveto(361.08426323,69.21127534)
\lineto(362.85364331,69.21127534)
\curveto(363.33366024,69.2112706)(363.67786106,69.24289986)(363.88624683,69.30616323)
\curveto(364.16160277,69.38802237)(364.36905354,69.52384216)(364.50859976,69.713623)
\curveto(364.64813529,69.90339334)(364.71790573,70.14154311)(364.71791128,70.428073)
\curveto(364.71790573,70.69970661)(364.65278665,70.93878665)(364.52255386,71.14531382)
\curveto(364.39231035,71.35182764)(364.20625585,71.49322906)(363.9643898,71.56951851)
\curveto(363.72251414,71.64579375)(363.30761261,71.68393493)(362.71968393,71.68394214)
\lineto(361.08426323,71.68394214)
\closepath
\moveto(361.08426323,65.43250463)
\lineto(363.12156206,65.43250463)
\curveto(363.47134057,65.43250366)(363.71693251,65.44552748)(363.85833863,65.47157611)
\curveto(364.10764696,65.51622819)(364.31602801,65.59064999)(364.48348238,65.69484174)
\curveto(364.65092611,65.79903103)(364.78860644,65.95066545)(364.89652378,66.14974545)
\curveto(365.00442966,66.34882209)(365.05838547,66.5785994)(365.05839136,66.83907807)
\curveto(365.05838547,67.14420508)(364.98024258,67.40933275)(364.82396246,67.63446186)
\curveto(364.66767101,67.85958464)(364.45091752,68.01773097)(364.17370132,68.10890132)
\curveto(363.8964751,68.20006438)(363.4973882,68.24564773)(362.9764394,68.24565151)
\lineto(361.08426323,68.24565151)
\closepath
}
}
{
\newrgbcolor{curcolor}{0 0 0}
\pscustom[linestyle=none,fillstyle=solid,fillcolor=curcolor]
{
\newpath
\moveto(367.31337285,67.09583354)
\lineto(368.33481308,67.18513979)
\curveto(368.38318572,66.77581717)(368.49574869,66.43998879)(368.67250234,66.17765365)
\curveto(368.84925225,65.9153151)(369.12368264,65.70321296)(369.49579434,65.54134662)
\curveto(369.86790064,65.37947813)(370.28652327,65.29854442)(370.75166348,65.29854525)
\curveto(371.16470052,65.29854442)(371.52936735,65.35994241)(371.84566505,65.48273939)
\curveto(372.16195265,65.60553435)(372.3973116,65.77391367)(372.55174259,65.98787787)
\curveto(372.70616207,66.20183903)(372.78337469,66.43533743)(372.78338067,66.68837377)
\curveto(372.78337469,66.94512676)(372.70895289,67.16932244)(372.56011505,67.36096147)
\curveto(372.41126569,67.55259471)(372.16567374,67.71353186)(371.82333848,67.84377338)
\curveto(371.60378915,67.92935508)(371.1181869,68.06238405)(370.36653028,68.24286069)
\curveto(369.61486652,68.42332978)(369.08833228,68.59356965)(368.78692598,68.75358081)
\curveto(368.39620953,68.95823647)(368.10503424,69.21220087)(367.91339922,69.51547476)
\curveto(367.72176196,69.81873855)(367.6259439,70.15828801)(367.62594472,70.53412417)
\curveto(367.6259439,70.9471591)(367.74315823,71.33322219)(367.97758808,71.69231461)
\curveto(368.21201558,72.05139257)(368.55435586,72.32396241)(369.00460996,72.51002496)
\curveto(369.45485965,72.69607142)(369.95534626,72.78909867)(370.50607129,72.78910699)
\curveto(371.11260526,72.78909867)(371.64751195,72.69142005)(372.11079298,72.49607086)
\curveto(372.57406337,72.3007056)(372.93035775,72.0132514)(373.17967716,71.63370738)
\curveto(373.42898381,71.25414903)(373.56294305,70.82436313)(373.58155528,70.34434839)
\lineto(372.54337012,70.26620542)
\curveto(372.48754803,70.78343114)(372.29870271,71.17414559)(371.9768336,71.43834996)
\curveto(371.65495413,71.70254038)(371.17958488,71.83463907)(370.55072442,71.83464644)
\curveto(369.89580882,71.83463907)(369.41857902,71.71463392)(369.11903359,71.47463062)
\curveto(368.81948353,71.2346133)(368.66970965,70.94529855)(368.66971152,70.6066855)
\curveto(368.66970965,70.31271325)(368.77576072,70.0708424)(368.98786504,69.88107222)
\curveto(369.19624389,69.69129121)(369.74045331,69.49686426)(370.62049492,69.29779077)
\curveto(371.5005289,69.09870762)(372.10427576,68.92474666)(372.43173731,68.77590737)
\curveto(372.90803121,68.55635875)(373.25967421,68.27820727)(373.48666739,67.9414521)
\curveto(373.7136472,67.60468997)(373.82714044,67.21676634)(373.82714747,66.77768002)
\curveto(373.82714044,66.34231018)(373.70248393,65.93206)(373.45317755,65.54692826)
\curveto(373.20385786,65.16179436)(372.84570295,64.86224662)(372.37871173,64.64828412)
\curveto(371.91170935,64.43432126)(371.38610538,64.32733992)(370.80189825,64.32733978)
\curveto(370.06139733,64.32733992)(369.44090556,64.43525153)(368.94042109,64.65107494)
\curveto(368.43993234,64.86689798)(368.04735734,65.19156308)(367.76269492,65.62507123)
\curveto(367.47803057,66.05857706)(367.32825669,66.54883067)(367.31337285,67.09583354)
\closepath
}
}
{
\newrgbcolor{curcolor}{0 0 0}
\pscustom[linestyle=none,fillstyle=solid,fillcolor=curcolor]
{
\newpath
\moveto(375.3174461,64.4668808)
\lineto(375.3174461,72.64956597)
\lineto(378.13617462,72.64956597)
\curveto(378.77247731,72.64955779)(379.25807956,72.61048635)(379.59298282,72.53235152)
\curveto(380.06183501,72.42443184)(380.46185219,72.22907462)(380.79303556,71.94627925)
\curveto(381.22467565,71.58160495)(381.54748021,71.11553842)(381.76145021,70.54807827)
\curveto(381.97540556,69.98060596)(382.0823869,69.33220602)(382.08239455,68.60287651)
\curveto(382.0823869,67.98145034)(382.00982565,67.43072901)(381.86471056,66.95071088)
\curveto(381.71958062,66.47068778)(381.53352612,66.07346142)(381.3065465,65.75903061)
\curveto(381.07955314,65.44459721)(380.83117038,65.19714472)(380.56139748,65.0166724)
\curveto(380.29161232,64.83619898)(379.96601694,64.69944893)(379.58461036,64.60642181)
\curveto(379.20319349,64.51339442)(378.76503513,64.4668808)(378.27013399,64.4668808)
\closepath
\moveto(376.40028438,65.43250463)
\lineto(378.1473379,65.43250463)
\curveto(378.68689224,65.43250366)(379.11016624,65.48273838)(379.41716114,65.58320893)
\curveto(379.72414609,65.68367724)(379.96880776,65.82507866)(380.15114689,66.00741361)
\curveto(380.40789639,66.26416729)(380.60790498,66.60929839)(380.75117326,67.04280795)
\curveto(380.89442891,67.47631237)(380.96605989,68.00191633)(380.96606642,68.61962143)
\curveto(380.96605989,69.47546799)(380.82558874,70.13317065)(380.54465255,70.5927314)
\curveto(380.26370415,71.05227989)(379.92229414,71.36020009)(379.5204215,71.51649292)
\curveto(379.23017139,71.62811858)(378.76317459,71.68393493)(378.1194297,71.68394214)
\lineto(376.40028438,71.68394214)
\closepath
}
}
{
\newrgbcolor{curcolor}{0 0 0}
\pscustom[linestyle=none,fillstyle=solid,fillcolor=curcolor]
{
\newpath
\moveto(386.3467699,66.62697573)
\lineto(387.35146522,66.7609351)
\curveto(387.46681753,66.19160603)(387.66310503,65.78135585)(387.94032831,65.53018334)
\curveto(388.21754744,65.2790087)(388.55523636,65.15342191)(388.95339608,65.1534226)
\curveto(389.42597143,65.15342191)(389.82505834,65.31714987)(390.150658,65.64460697)
\curveto(390.47624909,65.97206172)(390.63904678,66.37766053)(390.63905156,66.86140463)
\curveto(390.63904678,67.3228174)(390.48834264,67.70329886)(390.18693866,68.00285014)
\curveto(389.88552605,68.30239436)(389.50225378,68.45216823)(389.03712069,68.45217221)
\curveto(388.84734193,68.45216823)(388.61105271,68.41495733)(388.32825233,68.3405394)
\lineto(388.43988514,69.22243862)
\curveto(388.50686219,69.21499169)(388.560818,69.2112706)(388.60175272,69.21127534)
\curveto(389.02967534,69.2112706)(389.41480816,69.3229033)(389.75715233,69.54617378)
\curveto(390.09948873,69.7694341)(390.27065887,70.11363493)(390.27066327,70.5787773)
\curveto(390.27065887,70.9471591)(390.14600235,71.25228848)(389.89669335,71.49416636)
\curveto(389.64737629,71.73603019)(389.325502,71.85696561)(388.93106952,71.856973)
\curveto(388.540352,71.85696561)(388.21475662,71.73416964)(387.95428241,71.48858472)
\curveto(387.69380402,71.24298576)(387.52635497,70.87459784)(387.45193475,70.38341988)
\lineto(386.44723944,70.56203238)
\curveto(386.57003483,71.23554358)(386.84911658,71.75742646)(387.28448553,72.12768257)
\curveto(387.71985165,72.49792337)(388.26127025,72.6830476)(388.90874296,72.68305582)
\curveto(389.35527072,72.6830476)(389.76645117,72.58722953)(390.14228554,72.39560133)
\curveto(390.51811136,72.20395726)(390.80556556,71.94255068)(391.00464902,71.61138082)
\curveto(391.2037222,71.28019666)(391.30326136,70.92855365)(391.30326679,70.55645074)
\curveto(391.30326136,70.20294109)(391.20837356,69.8810668)(391.01860312,69.5908269)
\curveto(390.82882238,69.30057676)(390.54788008,69.06986918)(390.17577538,68.89870347)
\curveto(390.65951278,68.78706633)(391.03534287,68.55542848)(391.30326679,68.20378921)
\curveto(391.57117984,67.85214246)(391.70513908,67.41212356)(391.70514492,66.8837312)
\curveto(391.70513908,66.16927949)(391.44466278,65.56367209)(390.92371523,65.06690717)
\curveto(390.40275757,64.57014105)(389.74412463,64.32175829)(388.94781444,64.32175814)
\curveto(388.22964098,64.32175829)(387.63333631,64.53572096)(387.15889862,64.96364682)
\curveto(386.68445835,65.39157167)(386.41374905,65.94601409)(386.3467699,66.62697573)
\closepath
}
}
{
\newrgbcolor{curcolor}{0 0 0}
\pscustom[linestyle=none,fillstyle=solid,fillcolor=curcolor]
{
\newpath
\moveto(393.2680016,64.4668808)
\lineto(393.2680016,65.61111713)
\lineto(394.41223793,65.61111713)
\lineto(394.41223793,64.4668808)
\closepath
}
}
{
\newrgbcolor{curcolor}{0 0 0}
\pscustom[linestyle=none,fillstyle=solid,fillcolor=curcolor]
{
\newpath
\moveto(396.02533601,66.35905698)
\lineto(396.99095984,66.44836323)
\curveto(397.07282223,65.99438826)(397.22910801,65.66507179)(397.45981766,65.46041283)
\curveto(397.69052318,65.25575189)(397.98634984,65.15342191)(398.34729852,65.1534226)
\curveto(398.65614604,65.15342191)(398.92685534,65.22412262)(399.15942723,65.36552494)
\curveto(399.3919916,65.50692546)(399.58269746,65.69577078)(399.7315454,65.93206147)
\curveto(399.88038467,66.16834922)(400.00504118,66.48743269)(400.10551532,66.88931284)
\curveto(400.20598004,67.29118814)(400.25621476,67.70050804)(400.25621962,68.11727378)
\curveto(400.25621476,68.16192321)(400.25435422,68.22890283)(400.25063798,68.31821284)
\curveto(400.04969426,67.99819524)(399.77526387,67.73864921)(399.42734598,67.53957397)
\curveto(399.07942004,67.34049258)(398.70265967,67.24095342)(398.29706375,67.2409562)
\curveto(397.61982247,67.24095342)(397.0467746,67.48654536)(396.57791844,67.97773276)
\curveto(396.10905991,68.46891313)(395.87463124,69.1163828)(395.87463172,69.9201437)
\curveto(395.87463124,70.74994133)(396.11929291,71.41787699)(396.60861746,71.92395269)
\curveto(397.09793959,72.43001348)(397.71098917,72.6830476)(398.44776805,72.68305582)
\curveto(398.97988088,72.6830476)(399.4664134,72.53978564)(399.90736708,72.25326949)
\curveto(400.34831174,71.96673777)(400.68320984,71.55834814)(400.91206239,71.02809937)
\curveto(401.14090392,70.49783748)(401.25532744,69.73036266)(401.25533329,68.72567261)
\curveto(401.25532744,67.68004205)(401.14183419,66.84744815)(400.91485321,66.22788842)
\curveto(400.6878612,65.60832517)(400.35017228,65.13667701)(399.90178544,64.81294252)
\curveto(399.45338958,64.48920734)(398.92778562,64.32733992)(398.32497196,64.32733978)
\curveto(397.68494154,64.32733992)(397.16212839,64.50502197)(396.75653094,64.86038646)
\curveto(396.35093077,65.21575017)(396.10719937,65.71530651)(396.02533601,66.35905698)
\closepath
\moveto(400.13900516,69.97037847)
\curveto(400.13900042,70.54714192)(399.98550546,71.004836)(399.67851981,71.34346207)
\curveto(399.3715256,71.68207438)(399.00220742,71.85138398)(398.57056414,71.85139136)
\curveto(398.12403017,71.85138398)(397.73517626,71.66905057)(397.40400125,71.30439058)
\curveto(397.07282223,70.93971692)(396.90723373,70.46713848)(396.90723523,69.88665386)
\curveto(396.90723373,69.36569583)(397.06444978,68.94242184)(397.37888387,68.61683061)
\curveto(397.693314,68.29123109)(398.08123763,68.1284334)(398.54265594,68.12843706)
\curveto(399.00778905,68.1284334)(399.39013105,68.29123109)(399.68968309,68.61683061)
\curveto(399.98922655,68.94242184)(400.13900042,69.39360401)(400.13900516,69.97037847)
\closepath
}
}
{
\newrgbcolor{curcolor}{0 0 0}
\pscustom[linestyle=none,fillstyle=solid,fillcolor=curcolor]
{
\newpath
\moveto(214.8267232,63.35689545)
\lineto(214.8267232,71.53958062)
\lineto(220.34696579,71.53958062)
\lineto(220.34696579,70.57395679)
\lineto(215.90956148,70.57395679)
\lineto(215.90956148,68.03989194)
\lineto(219.74973024,68.03989194)
\lineto(219.74973024,67.07426811)
\lineto(215.90956148,67.07426811)
\lineto(215.90956148,63.35689545)
\closepath
}
}
{
\newrgbcolor{curcolor}{0 0 0}
\pscustom[linestyle=none,fillstyle=solid,fillcolor=curcolor]
{
\newpath
\moveto(221.61957979,63.35689545)
\lineto(221.61957979,69.28459781)
\lineto(222.52380557,69.28459781)
\lineto(222.52380557,68.38595366)
\curveto(222.75451151,68.80643181)(222.96754391,69.08365302)(223.16290343,69.21761812)
\curveto(223.35825837,69.3515715)(223.57315132,69.41855112)(223.80758292,69.41855718)
\curveto(224.14619918,69.41855112)(224.49040001,69.31063951)(224.84018644,69.09482202)
\lineto(224.49412472,68.16268804)
\curveto(224.24852916,68.30780574)(224.00293722,68.380367)(223.75734815,68.38037202)
\curveto(223.53780096,68.380367)(223.34058319,68.31431765)(223.16569425,68.18222378)
\curveto(222.99080073,68.05012026)(222.86614421,67.86685657)(222.79172432,67.63243218)
\curveto(222.68008971,67.27520326)(222.62427336,66.8844888)(222.6242751,66.46028764)
\lineto(222.6242751,63.35689545)
\closepath
}
}
{
\newrgbcolor{curcolor}{0 0 0}
\pscustom[linestyle=none,fillstyle=solid,fillcolor=curcolor]
{
\newpath
\moveto(229.50643782,65.26581655)
\lineto(230.54462298,65.13743881)
\curveto(230.38088917,64.53089935)(230.07762033,64.06018146)(229.63481556,63.72528373)
\curveto(229.1920009,63.39038526)(228.62639522,63.22293621)(227.9379968,63.22293607)
\curveto(227.07097958,63.22293621)(226.3835082,63.48992442)(225.87558059,64.0239015)
\curveto(225.36765062,64.55787726)(225.11368622,65.30674663)(225.11368664,66.27051186)
\curveto(225.11368622,67.26776108)(225.37044143,68.0417478)(225.88395305,68.59247437)
\curveto(226.39746228,69.14319046)(227.0635374,69.41855112)(227.8821804,69.41855718)
\curveto(228.67476939,69.41855112)(229.32223905,69.14877209)(229.82459134,68.60921929)
\curveto(230.32693336,68.06965598)(230.57810694,67.31055361)(230.57811282,66.33190991)
\curveto(230.57810694,66.27236949)(230.5762464,66.18306333)(230.57253118,66.06399116)
\lineto(226.1518718,66.06399116)
\curveto(226.18908124,65.41279769)(226.3732752,64.91417163)(226.70445422,64.56811147)
\curveto(227.03562922,64.22204888)(227.44867022,64.04901819)(227.94357844,64.04901889)
\curveto(228.31196311,64.04901819)(228.62639522,64.14576654)(228.88687571,64.3392642)
\curveto(229.14734782,64.5327599)(229.35386832,64.84161037)(229.50643782,65.26581655)
\closepath
\moveto(226.2076882,66.89007397)
\lineto(229.5176011,66.89007397)
\curveto(229.4729432,67.3886965)(229.34642614,67.76266605)(229.13804954,68.01198374)
\curveto(228.81803135,68.39897245)(228.40312981,68.59246913)(227.89334368,68.59247437)
\curveto(227.43192531,68.59246913)(227.04400168,68.43804389)(226.7295716,68.12919819)
\curveto(226.41513746,67.82034295)(226.2411765,67.40730195)(226.2076882,66.89007397)
\closepath
}
}
{
\newrgbcolor{curcolor}{0 0 0}
\pscustom[linestyle=none,fillstyle=solid,fillcolor=curcolor]
{
\newpath
\moveto(235.86950831,65.26581655)
\lineto(236.90769347,65.13743881)
\curveto(236.74395966,64.53089935)(236.44069082,64.06018146)(235.99788604,63.72528373)
\curveto(235.55507139,63.39038526)(234.9894657,63.22293621)(234.30106729,63.22293607)
\curveto(233.43405007,63.22293621)(232.74657868,63.48992442)(232.23865107,64.0239015)
\curveto(231.7307211,64.55787726)(231.47675671,65.30674663)(231.47675713,66.27051186)
\curveto(231.47675671,67.26776108)(231.73351192,68.0417478)(232.24702353,68.59247437)
\curveto(232.76053277,69.14319046)(233.42660789,69.41855112)(234.24525088,69.41855718)
\curveto(235.03783987,69.41855112)(235.68530954,69.14877209)(236.18766183,68.60921929)
\curveto(236.69000385,68.06965598)(236.94117743,67.31055361)(236.94118331,66.33190991)
\curveto(236.94117743,66.27236949)(236.93931688,66.18306333)(236.93560167,66.06399116)
\lineto(232.51494229,66.06399116)
\curveto(232.55215173,65.41279769)(232.73634569,64.91417163)(233.06752471,64.56811147)
\curveto(233.39869971,64.22204888)(233.81174071,64.04901819)(234.30664893,64.04901889)
\curveto(234.6750336,64.04901819)(234.9894657,64.14576654)(235.2499462,64.3392642)
\curveto(235.51041831,64.5327599)(235.71693881,64.84161037)(235.86950831,65.26581655)
\closepath
\moveto(232.57075869,66.89007397)
\lineto(235.88067159,66.89007397)
\curveto(235.83601369,67.3886965)(235.70949663,67.76266605)(235.50112003,68.01198374)
\curveto(235.18110184,68.39897245)(234.7662003,68.59246913)(234.25641416,68.59247437)
\curveto(233.7949958,68.59246913)(233.40707217,68.43804389)(233.09264209,68.12919819)
\curveto(232.77820795,67.82034295)(232.60424699,67.40730195)(232.57075869,66.89007397)
\closepath
}
}
{
\newrgbcolor{curcolor}{0 0 0}
\pscustom[linestyle=none,fillstyle=solid,fillcolor=curcolor]
{
\newpath
\moveto(238.25845162,63.35689545)
\lineto(238.25845162,71.53958062)
\lineto(241.32835397,71.53958062)
\curveto(241.95349319,71.53957244)(242.45491007,71.45677819)(242.83260612,71.29119761)
\curveto(243.21029135,71.12560117)(243.50611801,70.87070651)(243.72008698,70.52651285)
\curveto(243.93404336,70.18230485)(244.0410247,69.82228939)(244.04103132,69.44646538)
\curveto(244.0410247,69.09667683)(243.9461369,68.76736036)(243.75636765,68.45851499)
\curveto(243.56658572,68.14965942)(243.28006179,67.90034638)(242.89679499,67.71057515)
\curveto(243.39169449,67.56544828)(243.77217594,67.31799579)(244.0382405,66.96821694)
\curveto(244.30429182,66.61843087)(244.43732079,66.20538987)(244.4373278,65.72909272)
\curveto(244.43732079,65.34581807)(244.35638708,64.9895237)(244.19452644,64.66020854)
\curveto(244.03265225,64.33089076)(243.83264366,64.07692637)(243.59450007,63.89831459)
\curveto(243.35634413,63.71970173)(243.05772666,63.58481221)(242.69864674,63.49364564)
\curveto(242.33955628,63.4024788)(241.89953738,63.35689545)(241.37858873,63.35689545)
\closepath
\moveto(239.3412899,68.10128999)
\lineto(241.11066998,68.10128999)
\curveto(241.59068691,68.10128525)(241.93488774,68.13291451)(242.1432735,68.19617788)
\curveto(242.41862944,68.27803702)(242.62608021,68.41385681)(242.76562643,68.60363765)
\curveto(242.90516196,68.79340799)(242.9749324,69.03155775)(242.97493796,69.31808765)
\curveto(242.9749324,69.58972126)(242.90981333,69.8288013)(242.77958053,70.03532847)
\curveto(242.64933702,70.24184229)(242.46328252,70.38324371)(242.22141647,70.45953316)
\curveto(241.97954082,70.5358084)(241.56463928,70.57394958)(240.97671061,70.57395679)
\lineto(239.3412899,70.57395679)
\closepath
\moveto(239.3412899,64.32251928)
\lineto(241.37858873,64.32251928)
\curveto(241.72836724,64.32251831)(241.97395918,64.33554213)(242.1153653,64.36159076)
\curveto(242.36467364,64.40624284)(242.57305468,64.48066464)(242.74050905,64.58485639)
\curveto(242.90795278,64.68904568)(243.04563311,64.8406801)(243.15355046,65.0397601)
\curveto(243.26145634,65.23883673)(243.31541214,65.46861404)(243.31541803,65.72909272)
\curveto(243.31541214,66.03421973)(243.23726925,66.29934739)(243.08098913,66.52447651)
\curveto(242.92469769,66.74959929)(242.70794419,66.90774562)(242.43072799,66.99891596)
\curveto(242.15350178,67.09007903)(241.75441487,67.13566238)(241.23346608,67.13566616)
\lineto(239.3412899,67.13566616)
\closepath
}
}
{
\newrgbcolor{curcolor}{0 0 0}
\pscustom[linestyle=none,fillstyle=solid,fillcolor=curcolor]
{
\newpath
\moveto(245.57039952,65.98584819)
\lineto(246.59183976,66.07515444)
\curveto(246.64021239,65.66583182)(246.75277537,65.33000344)(246.92952901,65.0676683)
\curveto(247.10627892,64.80532974)(247.38070931,64.59322761)(247.75282101,64.43136127)
\curveto(248.12492732,64.26949278)(248.54354995,64.18855907)(249.00869015,64.1885599)
\curveto(249.42172719,64.18855907)(249.78639402,64.24995706)(250.10269172,64.37275404)
\curveto(250.41897932,64.495549)(250.65433827,64.66392832)(250.80876926,64.87789252)
\curveto(250.96318874,65.09185368)(251.04040136,65.32535208)(251.04040734,65.57838842)
\curveto(251.04040136,65.83514141)(250.96597956,66.05933709)(250.81714172,66.25097612)
\curveto(250.66829236,66.44260936)(250.42270041,66.60354651)(250.08036515,66.73378803)
\curveto(249.86081582,66.81936973)(249.37521357,66.9523987)(248.62355695,67.13287534)
\curveto(247.87189319,67.31334443)(247.34535895,67.4835843)(247.04395265,67.64359546)
\curveto(246.65323621,67.84825112)(246.36206091,68.10221552)(246.17042589,68.4054894)
\curveto(245.97878864,68.70875319)(245.88297057,69.04830266)(245.88297139,69.42413882)
\curveto(245.88297057,69.83717375)(246.0001849,70.22323684)(246.23461476,70.58232925)
\curveto(246.46904225,70.94140722)(246.81138253,71.21397706)(247.26163663,71.40003961)
\curveto(247.71188632,71.58608607)(248.21237293,71.67911332)(248.76309796,71.67912164)
\curveto(249.36963193,71.67911332)(249.90453863,71.5814347)(250.36781965,71.38608551)
\curveto(250.83109005,71.19072025)(251.18738442,70.90326604)(251.43670383,70.52372203)
\curveto(251.68601048,70.14416368)(251.81996972,69.71437778)(251.83858196,69.23436304)
\lineto(250.8003968,69.15622007)
\curveto(250.7445747,69.67344579)(250.55572938,70.06416024)(250.23386027,70.3283646)
\curveto(249.91198081,70.59255503)(249.43661155,70.72465372)(248.80775109,70.72466109)
\curveto(248.15283549,70.72465372)(247.67560569,70.60464857)(247.37606027,70.36464527)
\curveto(247.0765102,70.12462795)(246.92673632,69.8353132)(246.92673819,69.49670015)
\curveto(246.92673632,69.2027279)(247.03278739,68.96085704)(247.24489171,68.77108687)
\curveto(247.45327056,68.58130586)(247.99747998,68.38687891)(248.8775216,68.18780542)
\curveto(249.75755557,67.98872227)(250.36130243,67.81476131)(250.68876398,67.66592202)
\curveto(251.16505788,67.4463734)(251.51670089,67.16822192)(251.74369407,66.83146674)
\curveto(251.97067387,66.49470462)(252.08416712,66.10678099)(252.08417414,65.66769467)
\curveto(252.08416712,65.23232483)(251.9595106,64.82207465)(251.71020422,64.43694291)
\curveto(251.46088454,64.05180901)(251.10272962,63.75226126)(250.6357384,63.53829877)
\curveto(250.16873602,63.32433591)(249.64313205,63.21735457)(249.05892492,63.21735443)
\curveto(248.318424,63.21735457)(247.69793223,63.32526618)(247.19744777,63.54108959)
\curveto(246.69695901,63.75691263)(246.30438402,64.08157773)(246.01972159,64.51508588)
\curveto(245.73505724,64.94859171)(245.58528337,65.43884532)(245.57039952,65.98584819)
\closepath
}
}
{
\newrgbcolor{curcolor}{0 0 0}
\pscustom[linestyle=none,fillstyle=solid,fillcolor=curcolor]
{
\newpath
\moveto(253.57447277,63.35689545)
\lineto(253.57447277,71.53958062)
\lineto(256.39320129,71.53958062)
\curveto(257.02950399,71.53957244)(257.51510624,71.50050099)(257.8500095,71.42236617)
\curveto(258.31886168,71.31444649)(258.71887886,71.11908927)(259.05006223,70.8362939)
\curveto(259.48170232,70.4716196)(259.80450688,70.00555307)(260.01847689,69.43809292)
\curveto(260.23243224,68.87062061)(260.33941358,68.22222067)(260.33942122,67.49289116)
\curveto(260.33941358,66.87146499)(260.26685232,66.32074366)(260.12173724,65.84072553)
\curveto(259.9766073,65.36070243)(259.79055279,64.96347607)(259.56357317,64.64904525)
\curveto(259.33657981,64.33461185)(259.08819705,64.08715937)(258.81842415,63.90668705)
\curveto(258.54863899,63.72621363)(258.22304362,63.58946357)(257.84163704,63.49643646)
\curveto(257.46022016,63.40340907)(257.02206181,63.35689545)(256.52716067,63.35689545)
\closepath
\moveto(254.65731105,64.32251928)
\lineto(256.40436457,64.32251928)
\curveto(256.94391892,64.32251831)(257.36719291,64.37275303)(257.67418782,64.47322357)
\curveto(257.98117276,64.57369189)(258.22583443,64.71509331)(258.40817356,64.89742826)
\curveto(258.66492306,65.15418194)(258.86493165,65.49931304)(259.00819993,65.9328226)
\curveto(259.15145558,66.36632702)(259.22308656,66.89193098)(259.22309309,67.50963608)
\curveto(259.22308656,68.36548264)(259.08261541,69.0231853)(258.80167923,69.48274605)
\curveto(258.52073082,69.94229454)(258.17932081,70.25021474)(257.77744817,70.40650757)
\curveto(257.48719806,70.51813322)(257.02020126,70.57394958)(256.37645637,70.57395679)
\lineto(254.65731105,70.57395679)
\closepath
}
}
{
\newrgbcolor{curcolor}{0 0 0}
\pscustom[linestyle=none,fillstyle=solid,fillcolor=curcolor]
{
\newpath
\moveto(269.81146348,69.53577163)
\lineto(268.8123498,69.45762866)
\curveto(268.72303895,69.85205811)(268.59652189,70.13858204)(268.43279824,70.31720132)
\curveto(268.16115436,70.6037183)(267.82625626,70.74698026)(267.42810293,70.74698765)
\curveto(267.10808588,70.74698026)(266.82714358,70.6576741)(266.58527519,70.4790689)
\curveto(266.26898007,70.2483542)(266.01966704,69.91159555)(265.83733534,69.46879195)
\curveto(265.65500022,69.02597612)(265.56011242,68.39525136)(265.55267167,67.57661577)
\curveto(265.79454109,67.94499946)(266.09036775,68.21849958)(266.44015253,68.39711694)
\curveto(266.78993268,68.57572423)(267.15646005,68.66503039)(267.53973574,68.66503569)
\curveto(268.20952853,68.66503039)(268.77978558,68.41850817)(269.25050859,67.92546831)
\curveto(269.72122136,67.43241931)(269.9565803,66.79518264)(269.95658613,66.01375639)
\curveto(269.9565803,65.50024331)(269.84587787,65.02301351)(269.62447852,64.58206557)
\curveto(269.40306816,64.14111517)(269.09886905,63.80342625)(268.71188027,63.56899779)
\curveto(268.32488232,63.33456891)(267.8857937,63.21735457)(267.39461308,63.21735443)
\curveto(266.55736455,63.21735457)(265.87454453,63.52527477)(265.34615097,64.14111596)
\curveto(264.81775496,64.75695557)(264.55355757,65.77188288)(264.553558,67.18590093)
\curveto(264.55355757,68.76736036)(264.84566313,69.91717718)(265.42987558,70.63535484)
\curveto(265.93966361,71.26049069)(266.62620472,71.57306225)(267.48950097,71.57307047)
\curveto(268.13324618,71.57306225)(268.6607107,71.39258938)(269.07189609,71.03165132)
\curveto(269.48307159,70.67069792)(269.72959381,70.17207185)(269.81146348,69.53577163)
\closepath
\moveto(265.70895761,66.00817475)
\curveto(265.70895602,65.66211073)(265.78244755,65.33093371)(265.92943241,65.01464272)
\curveto(266.07641366,64.69834841)(266.28200389,64.45740783)(266.5462037,64.29182025)
\curveto(266.81039867,64.12623081)(267.08761988,64.04343656)(267.37786816,64.04343725)
\curveto(267.80206917,64.04343656)(268.16673599,64.2146067)(268.47186972,64.55694818)
\curveto(268.77699476,64.89928727)(268.92955945,65.36442352)(268.92956426,65.95235834)
\curveto(268.92955945,66.51796143)(268.7788553,66.96356197)(268.47745137,67.28916128)
\curveto(268.17603872,67.61475272)(267.79648753,67.77755041)(267.33879668,67.77755483)
\curveto(266.88482048,67.77755041)(266.49968766,67.61475272)(266.18339706,67.28916128)
\curveto(265.86710235,66.96356197)(265.70895602,66.53656688)(265.70895761,66.00817475)
\closepath
}
}
{
\newrgbcolor{curcolor}{0 0 0}
\pscustom[linestyle=none,fillstyle=solid,fillcolor=curcolor]
{
\newpath
\moveto(271.52502827,63.35689545)
\lineto(271.52502827,64.50113178)
\lineto(272.6692646,64.50113178)
\lineto(272.6692646,63.35689545)
\closepath
}
}
{
\newrgbcolor{curcolor}{0 0 0}
\pscustom[linestyle=none,fillstyle=solid,fillcolor=curcolor]
{
\newpath
\moveto(277.91600693,63.35689545)
\lineto(276.91131161,63.35689545)
\lineto(276.91131161,69.75903726)
\curveto(276.6694375,69.52832327)(276.35221458,69.29761569)(275.95964188,69.06691382)
\curveto(275.56706458,68.83620053)(275.2144913,68.66316984)(274.90192098,68.54782124)
\lineto(274.90192098,69.51902671)
\curveto(275.46380433,69.78321794)(275.95498822,70.10323169)(276.37547411,70.4790689)
\curveto(276.79595457,70.85489187)(277.09364177,71.2195587)(277.26853661,71.57307047)
\lineto(277.91600693,71.57307047)
\closepath
}
}
{
\newrgbcolor{curcolor}{0 0 0}
\pscustom[linestyle=none,fillstyle=solid,fillcolor=curcolor]
{
\newpath
\moveto(620.66306368,151.17684174)
\lineto(620.66306368,153.1359976)
\lineto(617.11314023,153.1359976)
\lineto(617.11314023,154.05696831)
\lineto(620.84725782,159.35952691)
\lineto(621.667759,159.35952691)
\lineto(621.667759,154.05696831)
\lineto(622.77292384,154.05696831)
\lineto(622.77292384,153.1359976)
\lineto(621.667759,153.1359976)
\lineto(621.667759,151.17684174)
\closepath
\moveto(620.66306368,154.05696831)
\lineto(620.66306368,157.74643277)
\lineto(618.10109063,154.05696831)
\closepath
}
}
{
\newrgbcolor{curcolor}{0 0 0}
\pscustom[linestyle=none,fillstyle=solid,fillcolor=curcolor]
{
\newpath
\moveto(624.36927322,151.17684174)
\lineto(624.36927322,152.32107807)
\lineto(625.51350956,152.32107807)
\lineto(625.51350956,151.17684174)
\closepath
}
}
{
\newrgbcolor{curcolor}{0 0 0}
\pscustom[linestyle=none,fillstyle=solid,fillcolor=curcolor]
{
\newpath
\moveto(626.98148117,153.33693666)
\lineto(627.98617649,153.47089604)
\curveto(628.10152879,152.90156697)(628.29781629,152.49131679)(628.57503957,152.24014428)
\curveto(628.85225871,151.98896964)(629.18994763,151.86338285)(629.58810735,151.86338353)
\curveto(630.0606827,151.86338285)(630.4597696,152.02711081)(630.78536927,152.35456791)
\curveto(631.11096036,152.68202266)(631.27375805,153.08762147)(631.27376282,153.57136557)
\curveto(631.27375805,154.03277834)(631.1230539,154.4132598)(630.82164993,154.71281108)
\curveto(630.52023732,155.01235529)(630.13696504,155.16212917)(629.67183196,155.16213315)
\curveto(629.4820532,155.16212917)(629.24576398,155.12491827)(628.9629636,155.05050034)
\lineto(629.07459641,155.93239956)
\curveto(629.14157346,155.92495262)(629.19552926,155.92123153)(629.23646399,155.92123628)
\curveto(629.66438661,155.92123153)(630.04951943,156.03286424)(630.3918636,156.25613472)
\curveto(630.73419999,156.47939504)(630.90537014,156.82359587)(630.90537454,157.28873824)
\curveto(630.90537014,157.65712004)(630.78071362,157.96224942)(630.53140462,158.2041273)
\curveto(630.28208755,158.44599113)(629.96021327,158.56692655)(629.56578079,158.56693394)
\curveto(629.17506327,158.56692655)(628.84946789,158.44413058)(628.58899367,158.19854566)
\curveto(628.32851528,157.9529467)(628.16106623,157.58455878)(628.08664602,157.09338081)
\lineto(627.0819507,157.27199331)
\curveto(627.20474609,157.94550451)(627.48382785,158.46738739)(627.9191968,158.83764351)
\curveto(628.35456291,159.20788431)(628.89598151,159.39300854)(629.54345422,159.39301676)
\curveto(629.98998199,159.39300854)(630.40116244,159.29719047)(630.7769968,159.10556226)
\curveto(631.15282262,158.9139182)(631.44027683,158.65251162)(631.63936028,158.32134175)
\curveto(631.83843346,157.9901576)(631.93797262,157.63851459)(631.93797806,157.26641167)
\curveto(631.93797262,156.91290203)(631.84308482,156.59102774)(631.65331438,156.30078784)
\curveto(631.46353364,156.0105377)(631.18259134,155.77983011)(630.81048665,155.6086644)
\curveto(631.29422404,155.49702727)(631.67005414,155.26538942)(631.93797806,154.91375014)
\curveto(632.2058911,154.5621034)(632.33985034,154.1220845)(632.33985618,153.59369213)
\curveto(632.33985034,152.87924043)(632.07937404,152.27363302)(631.55842649,151.7768681)
\curveto(631.03746883,151.28010198)(630.37883589,151.03171922)(629.58252571,151.03171908)
\curveto(628.86435225,151.03171922)(628.26804757,151.2456819)(627.79360988,151.67360775)
\curveto(627.31916961,152.10153261)(627.04846031,152.65597503)(626.98148117,153.33693666)
\closepath
}
}
{
\newrgbcolor{curcolor}{0 0 0}
\pscustom[linestyle=none,fillstyle=solid,fillcolor=curcolor]
{
\newpath
\moveto(633.70177666,151.17684174)
\lineto(633.70177666,159.35952691)
\lineto(636.77167901,159.35952691)
\curveto(637.39681823,159.35951873)(637.89823511,159.27672448)(638.27593116,159.1111439)
\curveto(638.65361639,158.94554746)(638.94944305,158.6906528)(639.16341202,158.34645914)
\curveto(639.3773684,158.00225114)(639.48434974,157.64223568)(639.48435636,157.26641167)
\curveto(639.48434974,156.91662312)(639.38946194,156.58730665)(639.19969269,156.27846128)
\curveto(639.00991076,155.9696057)(638.72338683,155.72029267)(638.34012003,155.53052143)
\curveto(638.83501953,155.38539457)(639.21550099,155.13794208)(639.48156554,154.78816323)
\curveto(639.74761686,154.43837715)(639.88064583,154.02533616)(639.88065285,153.54903901)
\curveto(639.88064583,153.16576436)(639.79971212,152.80946999)(639.63785148,152.48015483)
\curveto(639.47597729,152.15083705)(639.2759687,151.89687266)(639.03782511,151.71826088)
\curveto(638.79966917,151.53964801)(638.5010517,151.4047585)(638.14197179,151.31359193)
\curveto(637.78288132,151.22242509)(637.34286242,151.17684174)(636.82191378,151.17684174)
\closepath
\moveto(634.78461494,155.92123628)
\lineto(636.55399503,155.92123628)
\curveto(637.03401195,155.92123153)(637.37821278,155.9528608)(637.58659854,156.01612417)
\curveto(637.86195448,156.09798331)(638.06940525,156.2338031)(638.20895147,156.42358394)
\curveto(638.34848701,156.61335428)(638.41825744,156.85150404)(638.418263,157.13803394)
\curveto(638.41825744,157.40966755)(638.35313837,157.64874758)(638.22290558,157.85527476)
\curveto(638.09266207,158.06178858)(637.90660756,158.20319)(637.66474151,158.27947945)
\curveto(637.42286586,158.35575469)(637.00796432,158.39389586)(636.42003565,158.39390308)
\lineto(634.78461494,158.39390308)
\closepath
\moveto(634.78461494,152.14246557)
\lineto(636.82191378,152.14246557)
\curveto(637.17169228,152.1424646)(637.41728422,152.15548842)(637.55869034,152.18153705)
\curveto(637.80799868,152.22618913)(638.01637972,152.30061093)(638.18383409,152.40480268)
\curveto(638.35127782,152.50899197)(638.48895816,152.66062639)(638.5968755,152.85970639)
\curveto(638.70478138,153.05878302)(638.75873718,153.28856033)(638.75874308,153.54903901)
\curveto(638.75873718,153.85416602)(638.68059429,154.11929368)(638.52431417,154.3444228)
\curveto(638.36802273,154.56954558)(638.15126923,154.7276919)(637.87405304,154.81886225)
\curveto(637.59682682,154.91002532)(637.19773991,154.95560867)(636.67679112,154.95561245)
\lineto(634.78461494,154.95561245)
\closepath
}
}
{
\newrgbcolor{curcolor}{0 0 0}
\pscustom[linestyle=none,fillstyle=solid,fillcolor=curcolor]
{
\newpath
\moveto(641.01372647,153.80579448)
\lineto(642.03516671,153.89510073)
\curveto(642.08353934,153.4857781)(642.19610232,153.14994973)(642.37285596,152.88761459)
\curveto(642.54960587,152.62527603)(642.82403626,152.4131739)(643.19614796,152.25130756)
\curveto(643.56825427,152.08943907)(643.9868769,152.00850536)(644.4520171,152.00850619)
\curveto(644.86505414,152.00850536)(645.22972097,152.06990335)(645.54601867,152.19270033)
\curveto(645.86230627,152.31549529)(646.09766522,152.48387461)(646.25209621,152.69783881)
\curveto(646.40651569,152.91179997)(646.48372831,153.14529837)(646.48373429,153.39833471)
\curveto(646.48372831,153.6550877)(646.40930651,153.87928338)(646.26046867,154.07092241)
\curveto(646.11161931,154.26255565)(645.86602736,154.42349279)(645.5236921,154.55373432)
\curveto(645.30414277,154.63931602)(644.81854052,154.77234499)(644.0668839,154.95282163)
\curveto(643.31522014,155.13329072)(642.7886859,155.30353059)(642.4872796,155.46354175)
\curveto(642.09656316,155.66819741)(641.80538786,155.92216181)(641.61375284,156.22543569)
\curveto(641.42211559,156.52869948)(641.32629752,156.86824895)(641.32629834,157.24408511)
\curveto(641.32629752,157.65712004)(641.44351185,158.04318313)(641.67794171,158.40227554)
\curveto(641.9123692,158.76135351)(642.25470948,159.03392335)(642.70496358,159.2199859)
\curveto(643.15521327,159.40603235)(643.65569988,159.49905961)(644.20642491,159.49906793)
\curveto(644.81295888,159.49905961)(645.34786558,159.40138099)(645.8111466,159.20603179)
\curveto(646.274417,159.01066654)(646.63071137,158.72321233)(646.88003078,158.34366832)
\curveto(647.12933743,157.96410997)(647.26329667,157.53432407)(647.28190891,157.05430933)
\lineto(646.24372375,156.97616636)
\curveto(646.18790165,157.49339208)(645.99905633,157.88410653)(645.67718722,158.14831089)
\curveto(645.35530776,158.41250131)(644.8799385,158.54460001)(644.25107804,158.54460738)
\curveto(643.59616244,158.54460001)(643.11893264,158.42459486)(642.81938722,158.18459156)
\curveto(642.51983715,157.94457424)(642.37006327,157.65525949)(642.37006514,157.31664644)
\curveto(642.37006327,157.02267419)(642.47611434,156.78080333)(642.68821866,156.59103316)
\curveto(642.89659751,156.40125215)(643.44080693,156.20682519)(644.32084855,156.00775171)
\curveto(645.20088252,155.80866856)(645.80462938,155.6347076)(646.13209093,155.48586831)
\curveto(646.60838483,155.26631969)(646.96002784,154.98816821)(647.18702102,154.65141303)
\curveto(647.41400082,154.31465091)(647.52749407,153.92672727)(647.52750109,153.48764096)
\curveto(647.52749407,153.05227112)(647.40283755,152.64202094)(647.15353117,152.2568892)
\curveto(646.90421149,151.8717553)(646.54605657,151.57220755)(646.07906535,151.35824506)
\curveto(645.61206297,151.1442822)(645.086459,151.03730086)(644.50225187,151.03730072)
\curveto(643.76175095,151.03730086)(643.14125918,151.14521247)(642.64077472,151.36103588)
\curveto(642.14028596,151.57685892)(641.74771097,151.90152402)(641.46304854,152.33503217)
\curveto(641.17838419,152.768538)(641.02861032,153.25879161)(641.01372647,153.80579448)
\closepath
}
}
{
\newrgbcolor{curcolor}{0 0 0}
\pscustom[linestyle=none,fillstyle=solid,fillcolor=curcolor]
{
\newpath
\moveto(649.01779781,151.17684174)
\lineto(649.01779781,159.35952691)
\lineto(651.83652633,159.35952691)
\curveto(652.47282903,159.35951873)(652.95843128,159.32044728)(653.29333454,159.24231246)
\curveto(653.76218673,159.13439278)(654.16220391,158.93903556)(654.49338728,158.65624019)
\curveto(654.92502736,158.29156589)(655.24783192,157.82549936)(655.46180193,157.25803921)
\curveto(655.67575728,156.6905669)(655.78273862,156.04216696)(655.78274626,155.31283745)
\curveto(655.78273862,154.69141128)(655.71017736,154.14068995)(655.56506228,153.66067182)
\curveto(655.41993234,153.18064872)(655.23387784,152.78342236)(655.00689822,152.46899154)
\curveto(654.77990485,152.15455814)(654.53152209,151.90710566)(654.26174919,151.72663334)
\curveto(653.99196404,151.54615992)(653.66636866,151.40940986)(653.28496208,151.31638275)
\curveto(652.9035452,151.22335536)(652.46538685,151.17684174)(651.97048571,151.17684174)
\closepath
\moveto(650.10063609,152.14246557)
\lineto(651.84768961,152.14246557)
\curveto(652.38724396,152.1424646)(652.81051795,152.19269932)(653.11751286,152.29316986)
\curveto(653.42449781,152.39363818)(653.66915948,152.5350396)(653.8514986,152.71737455)
\curveto(654.1082481,152.97412822)(654.30825669,153.31925933)(654.45152497,153.75276889)
\curveto(654.59478062,154.1862733)(654.66641161,154.71187727)(654.66641814,155.32958237)
\curveto(654.66641161,156.18542893)(654.52594046,156.84313159)(654.24500427,157.30269234)
\curveto(653.96405586,157.76224083)(653.62264585,158.07016103)(653.22077321,158.22645386)
\curveto(652.9305231,158.33807951)(652.4635263,158.39389586)(651.81978141,158.39390308)
\lineto(650.10063609,158.39390308)
\closepath
}
}
{
\newrgbcolor{curcolor}{0 0 0}
\pscustom[linestyle=none,fillstyle=solid,fillcolor=curcolor]
{
\newpath
\moveto(656.75953516,153.63276362)
\lineto(656.75953516,154.64304057)
\lineto(659.84618243,154.64304057)
\lineto(659.84618243,153.63276362)
\closepath
}
}
{
\newrgbcolor{curcolor}{0 0 0}
\pscustom[linestyle=none,fillstyle=solid,fillcolor=curcolor]
{
\newpath
\moveto(667.29767106,152.05315932)
\curveto(667.80001113,151.70709707)(668.26328684,151.45406294)(668.68749958,151.29405619)
\lineto(668.36934606,150.5405347)
\curveto(667.78140568,150.75263747)(667.195334,151.08753558)(666.61112926,151.54523002)
\curveto(666.00458519,151.20661046)(665.33478898,151.03730086)(664.60173863,151.03730072)
\curveto(663.86123733,151.03730086)(663.18958058,151.21591318)(662.58676636,151.57313822)
\curveto(661.9839474,151.93036247)(661.51974142,152.43270962)(661.19414702,153.08018119)
\curveto(660.86855067,153.72764896)(660.70575298,154.4569826)(660.70575347,155.26818432)
\curveto(660.70575298,156.07565677)(660.86948094,156.81057205)(661.19693784,157.47293238)
\curveto(661.52439279,158.13528011)(661.99045931,158.63948781)(662.59513882,158.98555699)
\curveto(663.19981357,159.33161055)(663.87612169,159.50464124)(664.6240652,159.50464957)
\curveto(665.37944206,159.50464124)(666.06040154,159.32509865)(666.66694567,158.96602125)
\curveto(667.27347689,158.60692827)(667.73582233,158.10458111)(668.05398337,157.45897827)
\curveto(668.37212873,156.81336287)(668.53120533,156.0849595)(668.53121364,155.27376596)
\curveto(668.53120533,154.60024457)(668.42887535,153.99463717)(668.22422341,153.45694194)
\curveto(668.01955545,152.91924215)(667.71070497,152.45131507)(667.29767106,152.05315932)
\closepath
\moveto(664.91431051,153.43740619)
\curveto(665.53944894,153.2625127)(666.05481991,153.0020364)(666.46042497,152.65597651)
\curveto(667.09672512,153.23646507)(667.41487832,154.10906069)(667.41488552,155.27376596)
\curveto(667.41487832,155.93611589)(667.30231534,156.5147454)(667.07719626,157.0096562)
\curveto(666.85206345,157.50455535)(666.52274698,157.88875789)(666.08924586,158.162265)
\curveto(665.655733,158.43575813)(665.16920048,158.57250819)(664.62964684,158.57251558)
\curveto(663.82216588,158.57250819)(663.15236968,158.29621725)(662.62025621,157.74364195)
\curveto(662.08813793,157.19105351)(661.82207999,156.36590179)(661.8220816,155.26818432)
\curveto(661.82207999,154.20394848)(662.08534711,153.38716922)(662.61188375,152.81784408)
\curveto(663.13841559,152.24851567)(663.81100261,151.96385228)(664.62964684,151.96385307)
\curveto(665.01663579,151.96385228)(665.38130261,152.03641353)(665.7236484,152.18153705)
\curveto(665.3850237,152.40108036)(665.02779906,152.55736614)(664.6519734,152.65039486)
\closepath
}
}
{
\newrgbcolor{curcolor}{0 0 0}
\pscustom[linestyle=none,fillstyle=solid,fillcolor=curcolor]
{
\newpath
\moveto(673.73888353,151.17684174)
\lineto(673.73888353,152.04757768)
\curveto(673.27746373,151.3777806)(672.65046006,151.04288249)(671.85787064,151.04288236)
\curveto(671.50808542,151.04288249)(671.18155976,151.10986212)(670.8782927,151.24382142)
\curveto(670.57502209,151.3777806)(670.34989614,151.54615992)(670.20291419,151.7489599)
\curveto(670.05593003,151.95175874)(669.95266978,152.2001415)(669.89313313,152.49410893)
\curveto(669.85220035,152.69132538)(669.83173435,153.00389694)(669.83173508,153.43182455)
\lineto(669.83173508,157.10454409)
\lineto(670.8364304,157.10454409)
\lineto(670.8364304,153.81695776)
\curveto(670.83642866,153.29228142)(670.85689466,152.93877787)(670.89782845,152.75644604)
\curveto(670.96108518,152.49224706)(671.09504442,152.28479629)(671.29970657,152.13409311)
\curveto(671.50436433,151.983388)(671.75739845,151.90803593)(672.0588097,151.90803666)
\curveto(672.36021503,151.90803593)(672.64301788,151.98524855)(672.90721908,152.13967475)
\curveto(673.17141266,152.29409902)(673.35839744,152.50434061)(673.46817396,152.77040014)
\curveto(673.57794175,153.03645648)(673.63282783,153.42251957)(673.63283236,153.92859057)
\lineto(673.63283236,157.10454409)
\lineto(674.63752767,157.10454409)
\lineto(674.63752767,151.17684174)
\closepath
}
}
{
\newrgbcolor{curcolor}{0 0 0}
\pscustom[linestyle=none,fillstyle=solid,fillcolor=curcolor]
{
\newpath
\moveto(680.08521005,151.90803666)
\curveto(679.71309643,151.59174328)(679.35494151,151.36847787)(679.01074423,151.23823978)
\curveto(678.66653985,151.10800157)(678.29722167,151.04288249)(677.90278856,151.04288236)
\curveto(677.25159537,151.04288249)(676.75110876,151.20195909)(676.40132723,151.52011264)
\curveto(676.05154383,151.83826549)(675.8766526,152.24479458)(675.87665301,152.73970111)
\curveto(675.8766526,153.02994457)(675.94270195,153.29507224)(676.07480125,153.53508491)
\curveto(676.20689934,153.77509286)(676.37993002,153.96765926)(676.59389383,154.11278471)
\curveto(676.80785538,154.25790429)(677.04879596,154.36767644)(677.31671629,154.44210151)
\curveto(677.51393221,154.4941935)(677.81161942,154.54442822)(678.2097788,154.59280581)
\curveto(679.02097368,154.68955073)(679.61820863,154.80490452)(680.00148544,154.93886753)
\curveto(680.00520199,155.0765441)(680.00706254,155.16398971)(680.00706708,155.20120464)
\curveto(680.00706254,155.61052052)(679.91217474,155.89890499)(679.72240341,156.06635894)
\curveto(679.46564394,156.29334054)(679.08423221,156.40683378)(678.57816708,156.40683901)
\curveto(678.10558553,156.40683378)(677.75673334,156.32403953)(677.53160946,156.15845601)
\curveto(677.30648144,155.99286252)(677.13996266,155.69982668)(677.03205262,155.27934761)
\lineto(676.04968387,155.41330698)
\curveto(676.13898944,155.83378592)(676.2859725,156.17333538)(676.49063348,156.4319564)
\curveto(676.69529241,156.6905669)(676.99111906,156.88964522)(677.37811434,157.02919195)
\curveto(677.76510579,157.16872697)(678.21349714,157.23849741)(678.72328974,157.23850347)
\curveto(679.22935472,157.23849741)(679.64053517,157.17895997)(679.95683232,157.05989097)
\curveto(680.27312048,156.9408102)(680.5056886,156.79103633)(680.6545374,156.6105689)
\curveto(680.80337581,156.4300906)(680.90756633,156.20217383)(680.96710927,155.92681792)
\curveto(681.00059358,155.75564303)(681.01733848,155.44679255)(681.01734404,155.00026557)
\lineto(681.01734404,153.66067182)
\curveto(681.01733848,152.72667574)(681.03873475,152.13595269)(681.08153291,151.88850092)
\curveto(681.12431982,151.64104772)(681.20897462,151.40382823)(681.33549755,151.17684174)
\lineto(680.28614911,151.17684174)
\curveto(680.18195377,151.38522278)(680.11497415,151.62895418)(680.08521005,151.90803666)
\closepath
\moveto(680.00148544,154.1518562)
\curveto(679.63681408,154.00300962)(679.08981384,153.87649256)(678.36048309,153.77230463)
\curveto(677.9474392,153.7127646)(677.65533364,153.64578498)(677.48416551,153.57136557)
\curveto(677.31299335,153.49694137)(677.18089466,153.38809949)(677.08786903,153.24483959)
\curveto(676.99484015,153.10157556)(676.94832653,152.94249896)(676.94832801,152.76760932)
\curveto(676.94832653,152.49968924)(677.04972623,152.27642384)(677.25252743,152.09781244)
\curveto(677.45532505,151.9191992)(677.75208198,151.82989304)(678.14279911,151.82989369)
\curveto(678.52978979,151.82989304)(678.87399062,151.91454784)(679.17540263,152.08385834)
\curveto(679.47680721,152.25316703)(679.69821207,152.48480488)(679.83961786,152.7787726)
\curveto(679.9475251,153.00575749)(680.0014809,153.34065559)(680.00148544,153.78346791)
\closepath
}
}
{
\newrgbcolor{curcolor}{0 0 0}
\pscustom[linestyle=none,fillstyle=solid,fillcolor=curcolor]
{
\newpath
\moveto(682.1783264,152.94622182)
\lineto(683.17185844,153.10250776)
\curveto(683.22767344,152.7043492)(683.38302895,152.39921981)(683.63792543,152.18711869)
\curveto(683.89281829,151.97501555)(684.24911266,151.86896448)(684.70680961,151.86896518)
\curveto(685.1682219,151.86896448)(685.51056218,151.96292201)(685.73383149,152.15083803)
\curveto(685.95709299,152.3387521)(686.06872569,152.55922669)(686.06872993,152.81226244)
\curveto(686.06872569,153.0392473)(685.9701168,153.21785962)(685.77290298,153.34809994)
\curveto(685.6352187,153.43740393)(685.29287841,153.55089718)(684.7458811,153.68858002)
\curveto(684.00910235,153.87463201)(683.49838274,154.03556916)(683.21372074,154.17139194)
\curveto(682.92905597,154.30720873)(682.71323275,154.49512378)(682.56625043,154.73513764)
\curveto(682.41926663,154.97514439)(682.3457751,155.24027206)(682.34577562,155.53052143)
\curveto(682.3457751,155.79471447)(682.40624282,156.03937614)(682.52717894,156.26450718)
\curveto(682.64811367,156.48962804)(682.8127719,156.67661281)(683.02115414,156.82546206)
\curveto(683.17743873,156.9408102)(683.39047113,157.03848882)(683.66025199,157.1184982)
\curveto(683.93002919,157.19849569)(684.21934394,157.23849741)(684.52819711,157.23850347)
\curveto(684.99333067,157.23849741)(685.4017203,157.17151779)(685.75336723,157.03756441)
\curveto(686.10500632,156.9035993)(686.36455235,156.72219617)(686.5320061,156.49335444)
\curveto(686.69945045,156.26450209)(686.81480424,155.95844243)(686.87806782,155.57517456)
\lineto(685.89569907,155.44121518)
\curveto(685.85104192,155.7463403)(685.72173404,155.98449007)(685.50777505,156.15566519)
\curveto(685.29380869,156.32683035)(684.99147012,156.41241542)(684.60075844,156.41242065)
\curveto(684.1393405,156.41241542)(683.81002403,156.33613307)(683.61280805,156.18357339)
\curveto(683.41558849,156.03100369)(683.3169796,155.85239137)(683.31698109,155.64773589)
\curveto(683.3169796,155.51749327)(683.3579116,155.40027893)(683.43977719,155.29609253)
\curveto(683.52163956,155.1881768)(683.65001716,155.09887064)(683.82491039,155.02817378)
\curveto(683.92537783,154.99095902)(684.22120448,154.90537395)(684.71239125,154.77141831)
\curveto(685.42311657,154.58163912)(685.91895181,154.42628361)(686.19989848,154.30535131)
\curveto(686.48083641,154.18441276)(686.70131099,154.00859125)(686.8613229,153.77788627)
\curveto(687.02132474,153.54717609)(687.10132817,153.26065216)(687.10133345,152.91831362)
\curveto(687.10132817,152.58341377)(687.00364956,152.26805139)(686.80829731,151.97222553)
\curveto(686.61293511,151.67639807)(686.33106254,151.44755104)(685.96267876,151.28568373)
\curveto(685.59428671,151.1238162)(685.17752462,151.04288249)(684.71239125,151.04288236)
\curveto(683.94212273,151.04288249)(683.35512078,151.20288937)(682.95138363,151.52290346)
\curveto(682.54764424,151.84291685)(682.28995875,152.31735583)(682.1783264,152.94622182)
\closepath
}
}
{
\newrgbcolor{curcolor}{0 0 0}
\pscustom[linestyle=none,fillstyle=solid,fillcolor=curcolor]
{
\newpath
\moveto(688.30138523,158.2041273)
\lineto(688.30138523,159.35952691)
\lineto(689.30608055,159.35952691)
\lineto(689.30608055,158.2041273)
\closepath
\moveto(688.30138523,151.17684174)
\lineto(688.30138523,157.10454409)
\lineto(689.30608055,157.10454409)
\lineto(689.30608055,151.17684174)
\closepath
}
}
{
\newrgbcolor{curcolor}{0 0 0}
\pscustom[linestyle=none,fillstyle=solid,fillcolor=curcolor]
{
\newpath
\moveto(690.835449,158.19296402)
\lineto(690.835449,159.35952691)
\lineto(691.84014432,159.35952691)
\lineto(691.84014432,158.19296402)
\closepath
\moveto(689.56283494,148.87720579)
\lineto(689.75261072,149.73119681)
\curveto(689.95354991,149.679103)(690.11169624,149.65305537)(690.22705017,149.65305384)
\curveto(690.43170998,149.65305537)(690.58427468,149.72096526)(690.6847447,149.85678373)
\curveto(690.78521354,149.99260483)(690.83544825,150.3321543)(690.835449,150.87543314)
\lineto(690.835449,157.10454409)
\lineto(691.84014432,157.10454409)
\lineto(691.84014432,150.85310658)
\curveto(691.84014256,150.12377326)(691.74525477,149.61584447)(691.55548064,149.32931868)
\curveto(691.31360832,148.95721153)(690.9117306,148.77115703)(690.34984627,148.77115462)
\curveto(690.07820643,148.77115703)(689.81586958,148.80650738)(689.56283494,148.87720579)
\closepath
}
}
{
\newrgbcolor{curcolor}{0 0 0}
\pscustom[linestyle=none,fillstyle=solid,fillcolor=curcolor]
{
\newpath
\moveto(697.25433465,151.90803666)
\curveto(696.88222103,151.59174328)(696.52406611,151.36847787)(696.17986883,151.23823978)
\curveto(695.83566446,151.10800157)(695.46634627,151.04288249)(695.07191317,151.04288236)
\curveto(694.42071997,151.04288249)(693.92023336,151.20195909)(693.57045183,151.52011264)
\curveto(693.22066843,151.83826549)(693.0457772,152.24479458)(693.04577761,152.73970111)
\curveto(693.0457772,153.02994457)(693.11182655,153.29507224)(693.24392586,153.53508491)
\curveto(693.37602394,153.77509286)(693.54905463,153.96765926)(693.76301844,154.11278471)
\curveto(693.97697998,154.25790429)(694.21792056,154.36767644)(694.4858409,154.44210151)
\curveto(694.68305682,154.4941935)(694.98074402,154.54442822)(695.3789034,154.59280581)
\curveto(696.19009828,154.68955073)(696.78733323,154.80490452)(697.17061005,154.93886753)
\curveto(697.1743266,155.0765441)(697.17618714,155.16398971)(697.17619169,155.20120464)
\curveto(697.17618714,155.61052052)(697.08129935,155.89890499)(696.89152801,156.06635894)
\curveto(696.63476854,156.29334054)(696.25335681,156.40683378)(695.74729168,156.40683901)
\curveto(695.27471013,156.40683378)(694.92585794,156.32403953)(694.70073406,156.15845601)
\curveto(694.47560605,155.99286252)(694.30908727,155.69982668)(694.20117723,155.27934761)
\lineto(693.21880847,155.41330698)
\curveto(693.30811405,155.83378592)(693.4550971,156.17333538)(693.65975808,156.4319564)
\curveto(693.86441701,156.6905669)(694.16024367,156.88964522)(694.54723895,157.02919195)
\curveto(694.93423039,157.16872697)(695.38262174,157.23849741)(695.89241434,157.23850347)
\curveto(696.39847932,157.23849741)(696.80965977,157.17895997)(697.12595692,157.05989097)
\curveto(697.44224508,156.9408102)(697.67481321,156.79103633)(697.823662,156.6105689)
\curveto(697.97250041,156.4300906)(698.07669093,156.20217383)(698.13623388,155.92681792)
\curveto(698.16971818,155.75564303)(698.18646309,155.44679255)(698.18646864,155.00026557)
\lineto(698.18646864,153.66067182)
\curveto(698.18646309,152.72667574)(698.20785936,152.13595269)(698.25065751,151.88850092)
\curveto(698.29344443,151.64104772)(698.37809922,151.40382823)(698.50462216,151.17684174)
\lineto(697.45527372,151.17684174)
\curveto(697.35107837,151.38522278)(697.28409875,151.62895418)(697.25433465,151.90803666)
\closepath
\moveto(697.17061005,154.1518562)
\curveto(696.80593868,154.00300962)(696.25893845,153.87649256)(695.5296077,153.77230463)
\curveto(695.11656381,153.7127646)(694.82445824,153.64578498)(694.65329012,153.57136557)
\curveto(694.48211795,153.49694137)(694.35001926,153.38809949)(694.25699363,153.24483959)
\curveto(694.16396476,153.10157556)(694.11745113,152.94249896)(694.11745262,152.76760932)
\curveto(694.11745113,152.49968924)(694.21885083,152.27642384)(694.42165203,152.09781244)
\curveto(694.62444965,151.9191992)(694.92120658,151.82989304)(695.31192371,151.82989369)
\curveto(695.6989144,151.82989304)(696.04311523,151.91454784)(696.34452723,152.08385834)
\curveto(696.64593181,152.25316703)(696.86733667,152.48480488)(697.00874247,152.7787726)
\curveto(697.1166497,153.00575749)(697.17060551,153.34065559)(697.17061005,153.78346791)
\closepath
}
}
{
\newrgbcolor{curcolor}{0 0 0}
\pscustom[linestyle=none,fillstyle=solid,fillcolor=curcolor]
{
\newpath
\moveto(699.73816967,151.17684174)
\lineto(699.73816967,157.10454409)
\lineto(700.64239545,157.10454409)
\lineto(700.64239545,156.20589995)
\curveto(700.87310139,156.6263781)(701.08613379,156.9035993)(701.2814933,157.03756441)
\curveto(701.47684825,157.17151779)(701.69174119,157.23849741)(701.9261728,157.23850347)
\curveto(702.26478906,157.23849741)(702.60898989,157.1305858)(702.95877632,156.91476831)
\lineto(702.6127146,155.98263433)
\curveto(702.36711904,156.12775203)(702.12152709,156.20031329)(701.87593803,156.20031831)
\curveto(701.65639084,156.20031329)(701.45917307,156.13426394)(701.28428412,156.00217007)
\curveto(701.1093906,155.87006655)(700.98473409,155.68680286)(700.9103142,155.45237847)
\curveto(700.79867959,155.09514955)(700.74286324,154.70443509)(700.74286498,154.28023393)
\lineto(700.74286498,151.17684174)
\closepath
}
}
{
\newrgbcolor{curcolor}{0 0 0}
\pscustom[linestyle=none,fillstyle=solid,fillcolor=curcolor]
{
\newpath
\moveto(707.45199541,151.17684174)
\lineto(707.45199541,152.04757768)
\curveto(706.99057561,151.3777806)(706.36357193,151.04288249)(705.57098251,151.04288236)
\curveto(705.22119729,151.04288249)(704.89467164,151.10986212)(704.59140458,151.24382142)
\curveto(704.28813397,151.3777806)(704.06300802,151.54615992)(703.91602606,151.7489599)
\curveto(703.76904191,151.95175874)(703.66578166,152.2001415)(703.60624501,152.49410893)
\curveto(703.56531223,152.69132538)(703.54484623,153.00389694)(703.54484696,153.43182455)
\lineto(703.54484696,157.10454409)
\lineto(704.54954228,157.10454409)
\lineto(704.54954228,153.81695776)
\curveto(704.54954054,153.29228142)(704.57000654,152.93877787)(704.61094032,152.75644604)
\curveto(704.67419706,152.49224706)(704.8081563,152.28479629)(705.01281845,152.13409311)
\curveto(705.2174762,151.983388)(705.47051033,151.90803593)(705.77192158,151.90803666)
\curveto(706.07332691,151.90803593)(706.35612975,151.98524855)(706.62033095,152.13967475)
\curveto(706.88452454,152.29409902)(707.07150931,152.50434061)(707.18128584,152.77040014)
\curveto(707.29105363,153.03645648)(707.3459397,153.42251957)(707.34594424,153.92859057)
\lineto(707.34594424,157.10454409)
\lineto(708.35063955,157.10454409)
\lineto(708.35063955,151.17684174)
\closepath
}
}
{
\newrgbcolor{curcolor}{0 0 0}
\pscustom[linestyle=none,fillstyle=solid,fillcolor=curcolor]
{
\newpath
\moveto(709.52836303,152.94622182)
\lineto(710.52189506,153.10250776)
\curveto(710.57771006,152.7043492)(710.73306557,152.39921981)(710.98796205,152.18711869)
\curveto(711.24285491,151.97501555)(711.59914928,151.86896448)(712.05684623,151.86896518)
\curveto(712.51825852,151.86896448)(712.8605988,151.96292201)(713.08386811,152.15083803)
\curveto(713.30712961,152.3387521)(713.41876231,152.55922669)(713.41876655,152.81226244)
\curveto(713.41876231,153.0392473)(713.32015342,153.21785962)(713.1229396,153.34809994)
\curveto(712.98525532,153.43740393)(712.64291504,153.55089718)(712.09591772,153.68858002)
\curveto(711.35913897,153.87463201)(710.84841937,154.03556916)(710.56375736,154.17139194)
\curveto(710.27909259,154.30720873)(710.06326937,154.49512378)(709.91628705,154.73513764)
\curveto(709.76930325,154.97514439)(709.69581173,155.24027206)(709.69581224,155.53052143)
\curveto(709.69581173,155.79471447)(709.75627944,156.03937614)(709.87721557,156.26450718)
\curveto(709.99815029,156.48962804)(710.16280853,156.67661281)(710.37119076,156.82546206)
\curveto(710.52747535,156.9408102)(710.74050775,157.03848882)(711.01028862,157.1184982)
\curveto(711.28006581,157.19849569)(711.56938056,157.23849741)(711.87823373,157.23850347)
\curveto(712.34336729,157.23849741)(712.75175692,157.17151779)(713.10340385,157.03756441)
\curveto(713.45504294,156.9035993)(713.71458897,156.72219617)(713.88204272,156.49335444)
\curveto(714.04948707,156.26450209)(714.16484086,155.95844243)(714.22810444,155.57517456)
\lineto(713.24573569,155.44121518)
\curveto(713.20107854,155.7463403)(713.07177066,155.98449007)(712.85781167,156.15566519)
\curveto(712.64384531,156.32683035)(712.34150674,156.41241542)(711.95079506,156.41242065)
\curveto(711.48937712,156.41241542)(711.16006066,156.33613307)(710.96284467,156.18357339)
\curveto(710.76562511,156.03100369)(710.66701623,155.85239137)(710.66701772,155.64773589)
\curveto(710.66701623,155.51749327)(710.70794822,155.40027893)(710.78981381,155.29609253)
\curveto(710.87167618,155.1881768)(711.00005378,155.09887064)(711.17494701,155.02817378)
\curveto(711.27541445,154.99095902)(711.5712411,154.90537395)(712.06242788,154.77141831)
\curveto(712.77315319,154.58163912)(713.26898843,154.42628361)(713.54993511,154.30535131)
\curveto(713.83087303,154.18441276)(714.05134762,154.00859125)(714.21135952,153.77788627)
\curveto(714.37136136,153.54717609)(714.45136479,153.26065216)(714.45137007,152.91831362)
\curveto(714.45136479,152.58341377)(714.35368618,152.26805139)(714.15833394,151.97222553)
\curveto(713.96297173,151.67639807)(713.68109916,151.44755104)(713.31271538,151.28568373)
\curveto(712.94432333,151.1238162)(712.52756124,151.04288249)(712.06242788,151.04288236)
\curveto(711.29215935,151.04288249)(710.7051574,151.20288937)(710.30142025,151.52290346)
\curveto(709.89768086,151.84291685)(709.63999538,152.31735583)(709.52836303,152.94622182)
\closepath
}
}
{
\newrgbcolor{curcolor}{0 0 0}
\pscustom[linestyle=none,fillstyle=solid,fillcolor=curcolor]
{
\newpath
\moveto(718.53713019,155.21236792)
\curveto(718.53712971,156.17984729)(718.63666887,156.95848538)(718.83574796,157.54828452)
\curveto(719.03482551,158.13807092)(719.33065217,158.59297418)(719.72322882,158.91299566)
\curveto(720.11580216,159.23300167)(720.60977687,159.39300854)(721.20515441,159.39301676)
\curveto(721.6442399,159.39300854)(722.02937272,159.30463265)(722.36055403,159.12788883)
\curveto(722.69172674,158.9511291)(722.96522686,158.69623443)(723.1810552,158.36320406)
\curveto(723.3968733,158.03015931)(723.5661829,157.6245605)(723.6889845,157.1464064)
\curveto(723.81177484,156.66824036)(723.87317283,156.02356151)(723.87317864,155.21236792)
\curveto(723.87317283,154.25232265)(723.77456394,153.47740565)(723.57735169,152.88761459)
\curveto(723.3801284,152.29782011)(723.08523201,151.84198658)(722.69266164,151.52011264)
\curveto(722.30008202,151.198238)(721.80424677,151.03730086)(721.20515441,151.03730072)
\curveto(720.41628018,151.03730086)(719.79671869,151.3201037)(719.34646808,151.8857101)
\curveto(718.80690874,152.56666887)(718.53712971,153.6755537)(718.53713019,155.21236792)
\closepath
\moveto(719.56973371,155.21236792)
\curveto(719.5697322,153.86905038)(719.72694825,152.9750585)(720.04138234,152.53038959)
\curveto(720.35581247,152.08571798)(720.74373611,151.86338285)(721.20515441,151.86338353)
\curveto(721.66656644,151.86338285)(722.05449007,152.08664825)(722.36892649,152.53318041)
\curveto(722.68335429,152.97970986)(722.84057034,153.87277147)(722.84057512,155.21236792)
\curveto(722.84057034,156.55939848)(722.68335429,157.45432063)(722.36892649,157.89713706)
\curveto(722.05449007,158.33994006)(721.66284535,158.56134492)(721.19399113,158.5613523)
\curveto(720.73257284,158.56134492)(720.36418492,158.36598769)(720.08882629,157.97528003)
\curveto(719.74276289,157.47664717)(719.5697322,156.55567739)(719.56973371,155.21236792)
\closepath
}
}
{
\newrgbcolor{curcolor}{0 0 0}
\pscustom[linestyle=none,fillstyle=solid,fillcolor=curcolor]
{
\newpath
\moveto(729.04736444,153.34809994)
\lineto(730.03531483,153.21972221)
\curveto(729.92739761,152.53876069)(729.65110668,152.00571454)(729.2064412,151.62058217)
\curveto(728.76176616,151.2354489)(728.21569619,151.04288249)(727.56822967,151.04288236)
\curveto(726.7570289,151.04288249)(726.10490787,151.30801016)(725.61186463,151.83826615)
\curveto(725.11881901,152.36852082)(724.8722968,153.12855346)(724.87229724,154.11836635)
\curveto(724.8722968,154.7583909)(724.97834786,155.31841495)(725.19045076,155.79844018)
\curveto(725.40255213,156.27845618)(725.72535669,156.63847164)(726.15886541,156.87848765)
\curveto(726.59237067,157.11849225)(727.06401883,157.23849741)(727.57381131,157.23850347)
\curveto(728.21755674,157.23849741)(728.74409098,157.07569972)(729.15341561,156.75010991)
\curveto(729.56273079,156.42450896)(729.82506764,155.96216352)(729.94042694,155.36307221)
\lineto(728.96363983,155.21236792)
\curveto(728.87060804,155.61052052)(728.70594981,155.91006826)(728.46966463,156.11101206)
\curveto(728.23337137,156.31194599)(727.94777771,156.41241542)(727.6128828,156.41242065)
\curveto(727.10681136,156.41241542)(726.69563091,156.23101228)(726.37934021,155.86821069)
\curveto(726.06304561,155.50539972)(725.90489928,154.93142158)(725.90490076,154.14627456)
\curveto(725.90489928,153.34995832)(726.05746397,152.77132882)(726.36259529,152.41038432)
\curveto(726.66772274,152.04943735)(727.06587937,151.86896448)(727.55706639,151.86896518)
\curveto(727.9514988,151.86896448)(728.28081527,151.98989991)(728.54501678,152.23177182)
\curveto(728.80921006,152.47364161)(728.97665911,152.84575062)(729.04736444,153.34809994)
\closepath
}
}
{
\newrgbcolor{curcolor}{0 0 0}
\pscustom[linestyle=none,fillstyle=solid,fillcolor=curcolor]
{
\newpath
\moveto(816.20839473,63.35689545)
\lineto(816.20839473,71.53958062)
\lineto(817.8382338,71.53958062)
\lineto(819.7750631,65.74583764)
\curveto(819.953671,65.2062772)(820.08390915,64.80253893)(820.16577794,64.53462162)
\curveto(820.25880039,64.83230765)(820.4039229,65.26953573)(820.60114591,65.84630717)
\lineto(822.56030178,71.53958062)
\lineto(824.01710998,71.53958062)
\lineto(824.01710998,63.35689545)
\lineto(822.97334318,63.35689545)
\lineto(822.97334318,70.20556851)
\lineto(820.59556427,63.35689545)
\lineto(819.61877716,63.35689545)
\lineto(817.25216153,70.32278296)
\lineto(817.25216153,63.35689545)
\closepath
}
}
{
\newrgbcolor{curcolor}{0 0 0}
\pscustom[linestyle=none,fillstyle=solid,fillcolor=curcolor]
{
\newpath
\moveto(825.95952121,63.35689545)
\lineto(825.95952121,71.53958062)
\lineto(827.04235949,71.53958062)
\lineto(827.04235949,63.35689545)
\closepath
}
}
{
\newrgbcolor{curcolor}{0 0 0}
\pscustom[linestyle=none,fillstyle=solid,fillcolor=curcolor]
{
\newpath
\moveto(828.9345358,63.35689545)
\lineto(828.9345358,71.53958062)
\lineto(830.04528228,71.53958062)
\lineto(834.34314558,65.11511225)
\lineto(834.34314558,71.53958062)
\lineto(835.38133074,71.53958062)
\lineto(835.38133074,63.35689545)
\lineto(834.27058425,63.35689545)
\lineto(829.97272096,69.78694546)
\lineto(829.97272096,63.35689545)
\closepath
}
}
{
\newrgbcolor{curcolor}{0 0 0}
\pscustom[linestyle=none,fillstyle=solid,fillcolor=curcolor]
{
\newpath
\moveto(837.39072028,63.35689545)
\lineto(837.39072028,71.53958062)
\lineto(838.47355857,71.53958062)
\lineto(838.47355857,63.35689545)
\closepath
}
}
{
\newrgbcolor{curcolor}{0 0 0}
\pscustom[linestyle=none,fillstyle=solid,fillcolor=curcolor]
{
\newpath
\moveto(839.5452337,63.35689545)
\lineto(842.71002394,67.62126889)
\lineto(839.91920362,71.53958062)
\lineto(841.20856261,71.53958062)
\lineto(842.69327902,69.44088374)
\curveto(843.00212629,69.00551012)(843.2216706,68.67061202)(843.35191261,68.43618843)
\curveto(843.53424217,68.73387055)(843.75006539,69.04458157)(843.99938293,69.36832241)
\lineto(845.64596692,71.53958062)
\lineto(846.82369309,71.53958062)
\lineto(843.94914816,67.68266694)
\lineto(847.04695872,63.35689545)
\lineto(845.70736496,63.35689545)
\lineto(843.64773957,66.2760935)
\curveto(843.53238162,66.44353963)(843.41330674,66.62587305)(843.29051457,66.82309428)
\curveto(843.10817736,66.52540361)(842.97793921,66.32074366)(842.89979972,66.20911381)
\lineto(840.84575597,63.35689545)
\closepath
}
}
{
\newrgbcolor{curcolor}{0 0 0}
\pscustom[linestyle=none,fillstyle=solid,fillcolor=curcolor]
{
\newpath
\moveto(850.781077,65.51699037)
\lineto(851.78577232,65.65094975)
\curveto(851.90112462,65.08162068)(852.09741212,64.6713705)(852.37463541,64.42019799)
\curveto(852.65185454,64.16902335)(852.98954346,64.04343656)(853.38770318,64.04343725)
\curveto(853.86027853,64.04343656)(854.25936544,64.20716452)(854.5849651,64.53462162)
\curveto(854.91055619,64.86207637)(855.07335388,65.26767518)(855.07335865,65.75141928)
\curveto(855.07335388,66.21283205)(854.92264974,66.59331351)(854.62124576,66.89286479)
\curveto(854.31983315,67.192409)(853.93656088,67.34218288)(853.47142779,67.34218686)
\curveto(853.28164903,67.34218288)(853.04535981,67.30497198)(852.76255943,67.23055405)
\lineto(852.87419224,68.11245327)
\curveto(852.94116929,68.10500634)(852.9951251,68.10128525)(853.03605982,68.10128999)
\curveto(853.46398244,68.10128525)(853.84911526,68.21291795)(854.19145943,68.43618843)
\curveto(854.53379583,68.65944875)(854.70496597,69.00364958)(854.70497037,69.46879195)
\curveto(854.70496597,69.83717375)(854.58030945,70.14230313)(854.33100045,70.38418101)
\curveto(854.08168339,70.62604484)(853.7598091,70.74698026)(853.36537662,70.74698765)
\curveto(852.9746591,70.74698026)(852.64906372,70.62418429)(852.38858951,70.37859937)
\curveto(852.12811112,70.13300041)(851.96066207,69.76461249)(851.88624185,69.27343452)
\lineto(850.88154653,69.45204702)
\curveto(851.00434193,70.12555823)(851.28342368,70.6474411)(851.71879263,71.01769722)
\curveto(852.15415875,71.38793802)(852.69557735,71.57306225)(853.34305006,71.57307047)
\curveto(853.78957782,71.57306225)(854.20075827,71.47724418)(854.57659264,71.28561597)
\curveto(854.95241846,71.09397191)(855.23987266,70.83256533)(855.43895612,70.50139546)
\curveto(855.63802929,70.17021131)(855.73756845,69.8185683)(855.73757389,69.44646538)
\curveto(855.73756845,69.09295574)(855.64268066,68.77108145)(855.45291022,68.48084155)
\curveto(855.26312947,68.19059141)(854.98218718,67.95988382)(854.61008248,67.78871811)
\curveto(855.09381988,67.67708098)(855.46964997,67.44544313)(855.73757389,67.09380385)
\curveto(856.00548694,66.74215711)(856.13944618,66.30213821)(856.13945202,65.77374584)
\curveto(856.13944618,65.05929414)(855.87896987,64.45368674)(855.35802233,63.95692182)
\curveto(854.83706466,63.4601557)(854.17843173,63.21177294)(853.38212154,63.21177279)
\curveto(852.66394808,63.21177294)(852.0676434,63.42573561)(851.59320572,63.85366146)
\curveto(851.11876544,64.28158632)(850.84805614,64.83602874)(850.781077,65.51699037)
\closepath
}
}
{
\newrgbcolor{curcolor}{0 0 0}
\pscustom[linestyle=none,fillstyle=solid,fillcolor=curcolor]
{
\newpath
\moveto(857.70231251,63.35689545)
\lineto(857.70231251,64.50113178)
\lineto(858.84654884,64.50113178)
\lineto(858.84654884,63.35689545)
\closepath
}
}
{
\newrgbcolor{curcolor}{0 0 0}
\pscustom[linestyle=none,fillstyle=solid,fillcolor=curcolor]
{
\newpath
\moveto(864.09329117,63.35689545)
\lineto(863.08859585,63.35689545)
\lineto(863.08859585,69.75903726)
\curveto(862.84672174,69.52832327)(862.52949882,69.29761569)(862.13692612,69.06691382)
\curveto(861.74434882,68.83620053)(861.39177554,68.66316984)(861.07920522,68.54782124)
\lineto(861.07920522,69.51902671)
\curveto(861.64108857,69.78321794)(862.13227246,70.10323169)(862.55275835,70.4790689)
\curveto(862.97323881,70.85489187)(863.27092601,71.2195587)(863.44582085,71.57307047)
\lineto(864.09329117,71.57307047)
\closepath
}
}
{
\newrgbcolor{curcolor}{0 0 0}
\pscustom[linestyle=none,fillstyle=solid,fillcolor=curcolor]
{
\newpath
\moveto(867.23575214,63.35689545)
\lineto(867.23575214,64.50113178)
\lineto(868.37998848,64.50113178)
\lineto(868.37998848,63.35689545)
\closepath
}
}
{
\newrgbcolor{curcolor}{0 0 0}
\pscustom[linestyle=none,fillstyle=solid,fillcolor=curcolor]
{
\newpath
\moveto(875.12261049,64.32251928)
\lineto(875.12261049,63.35689545)
\lineto(869.71400071,63.35689545)
\curveto(869.70655819,63.5987663)(869.74562963,63.83133443)(869.83121517,64.05460053)
\curveto(869.96889504,64.42298774)(870.18936962,64.78579402)(870.49263958,65.14302045)
\curveto(870.7959073,65.50024331)(871.23406565,65.9132843)(871.80711595,66.38214467)
\curveto(872.69645403,67.1114753)(873.29741008,67.68917452)(873.60998588,68.11524409)
\curveto(873.9225532,68.54130414)(874.07883898,68.94411214)(874.07884369,69.32366929)
\curveto(874.07883898,69.72181996)(873.93650729,70.05764833)(873.65184819,70.33115542)
\curveto(873.36718051,70.60464857)(872.99600178,70.74139863)(872.53831088,70.74140601)
\curveto(872.054566,70.74139863)(871.66757264,70.59627612)(871.37732963,70.30603804)
\curveto(871.08708259,70.01578607)(870.94009954,69.61390835)(870.93638001,69.10040366)
\lineto(869.9037765,69.20645484)
\curveto(869.97447667,69.97671462)(870.24053461,70.56371658)(870.70195111,70.96746246)
\curveto(871.16336494,71.37119312)(871.78292643,71.57306225)(872.56063744,71.57307047)
\curveto(873.34578425,71.57306225)(873.96720628,71.35537848)(874.42490541,70.92001851)
\curveto(874.88259443,70.48464341)(875.11144147,69.94508536)(875.11144721,69.30134273)
\curveto(875.11144147,68.97388086)(875.04446185,68.65200657)(874.91050815,68.3357189)
\curveto(874.77654337,68.01942126)(874.55420824,67.68638371)(874.24350209,67.33660522)
\curveto(873.9327862,66.98681878)(873.41648496,66.50679816)(872.69459682,65.89654194)
\curveto(872.0917769,65.39047115)(871.70478354,65.0472006)(871.53361556,64.86672924)
\curveto(871.36244326,64.68625486)(871.22104183,64.50485172)(871.10941087,64.32251928)
\closepath
}
}
{
\newrgbcolor{curcolor}{0 0 0}
\pscustom[linestyle=none,fillstyle=solid,fillcolor=curcolor]
{
\newpath
\moveto(912.04989396,66.9163144)
\lineto(912.04989396,67.87635659)
\lineto(915.51609279,67.88193823)
\lineto(915.51609279,64.84552572)
\curveto(914.98396874,64.42132032)(914.43510796,64.10223685)(913.8695088,63.88827435)
\curveto(913.30389659,63.6743115)(912.72340654,63.56733016)(912.12803693,63.56733002)
\curveto(911.32427669,63.56733016)(910.59401277,63.73943057)(909.93724297,64.08363178)
\curveto(909.28046799,64.42783223)(908.78463274,64.92552802)(908.44973574,65.57672065)
\curveto(908.11483653,66.22790953)(907.94738748,66.95538264)(907.94738809,67.75914214)
\curveto(907.94738748,68.55545135)(908.11390626,69.29873909)(908.44694492,69.98900757)
\curveto(908.77998137,70.67926349)(909.25907172,71.19184364)(909.88421739,71.52674957)
\curveto(910.50935797,71.86163985)(911.22938889,72.0290889)(912.04431232,72.02909722)
\curveto(912.63596093,72.0290889)(913.17086762,71.93327083)(913.649034,71.74164273)
\curveto(914.12718776,71.54999856)(914.50208758,71.28301035)(914.77373459,70.9406773)
\curveto(915.04536673,70.59832978)(915.25188722,70.15179898)(915.3932967,69.60108355)
\lineto(914.41650959,69.3331648)
\curveto(914.29370654,69.74992125)(914.14114185,70.07737718)(913.95881505,70.31553355)
\curveto(913.77647502,70.5536767)(913.51599872,70.74438257)(913.17738537,70.88765171)
\curveto(912.83876033,71.0309065)(912.46293024,71.10253748)(912.04989396,71.10254488)
\curveto(911.55498427,71.10253748)(911.12705892,71.02718541)(910.76611661,70.87648843)
\curveto(910.40516745,70.72577712)(910.11399215,70.52762907)(909.89258985,70.28204371)
\curveto(909.67118244,70.03644519)(909.49908202,69.76666616)(909.37628809,69.47270581)
\curveto(909.16790501,68.9666318)(909.06371449,68.41777102)(909.06371622,67.82612182)
\curveto(909.06371449,67.09678406)(909.18930128,66.48652529)(909.44047696,65.99534369)
\curveto(909.69164843,65.50415752)(910.05724553,65.1394907)(910.53726934,64.90134213)
\curveto(911.01728676,64.66319117)(911.5270761,64.54411629)(912.06663888,64.54411713)
\curveto(912.5354915,64.54411629)(912.99318557,64.63435272)(913.43972248,64.8148267)
\curveto(913.88624718,64.99529846)(914.22486637,65.18786487)(914.45558107,65.39252651)
\lineto(914.45558107,66.9163144)
\closepath
}
}
{
\newrgbcolor{curcolor}{0 0 0}
\pscustom[linestyle=none,fillstyle=solid,fillcolor=curcolor]
{
\newpath
\moveto(917.09569722,63.70687103)
\lineto(917.09569722,71.88955621)
\lineto(918.20644371,71.88955621)
\lineto(922.504307,65.46508783)
\lineto(922.504307,71.88955621)
\lineto(923.54249216,71.88955621)
\lineto(923.54249216,63.70687103)
\lineto(922.43174567,63.70687103)
\lineto(918.13388238,70.13692105)
\lineto(918.13388238,63.70687103)
\closepath
}
}
{
\newrgbcolor{curcolor}{0 0 0}
\pscustom[linestyle=none,fillstyle=solid,fillcolor=curcolor]
{
\newpath
\moveto(930.73722777,71.88955621)
\lineto(931.82006605,71.88955621)
\lineto(931.82006605,67.16190659)
\curveto(931.82005872,66.33954223)(931.72703147,65.68649093)(931.54098402,65.20275072)
\curveto(931.35492246,64.71900752)(931.01909409,64.32550225)(930.53349788,64.02223373)
\curveto(930.04788959,63.71896458)(929.41065292,63.56733016)(928.62178597,63.56733002)
\curveto(927.85523728,63.56733016)(927.22823361,63.69942885)(926.74077307,63.9636265)
\curveto(926.25330802,64.22782364)(925.9053861,64.61016564)(925.69700627,65.11065365)
\curveto(925.48862402,65.61113886)(925.3844335,66.29488915)(925.3844344,67.16190659)
\lineto(925.3844344,71.88955621)
\lineto(926.46727268,71.88955621)
\lineto(926.46727268,67.16748823)
\curveto(926.4672707,66.45675657)(926.53332005,65.93301315)(926.66542092,65.59625639)
\curveto(926.79751744,65.25949585)(927.02450393,64.99994982)(927.34638108,64.81761752)
\curveto(927.66825251,64.635283)(928.06175778,64.54411629)(928.52689807,64.54411713)
\curveto(929.3232073,64.54411629)(929.89067353,64.72458916)(930.22929847,65.08553627)
\curveto(930.56791192,65.44648063)(930.73722152,66.14046392)(930.73722777,67.16748823)
\closepath
}
}
{
\newrgbcolor{curcolor}{0 0 0}
\pscustom[linestyle=none,fillstyle=solid,fillcolor=curcolor]
{
\newpath
\moveto(932.74661636,63.56733002)
\lineto(935.11881363,72.02909722)
\lineto(935.92256989,72.02909722)
\lineto(933.55595426,63.56733002)
\closepath
}
}
{
\newrgbcolor{curcolor}{0 0 0}
\pscustom[linestyle=none,fillstyle=solid,fillcolor=curcolor]
{
\newpath
\moveto(936.75423447,63.70687103)
\lineto(936.75423447,71.88955621)
\lineto(937.83707275,71.88955621)
\lineto(937.83707275,64.67249486)
\lineto(941.86701729,64.67249486)
\lineto(941.86701729,63.70687103)
\closepath
}
}
{
\newrgbcolor{curcolor}{0 0 0}
\pscustom[linestyle=none,fillstyle=solid,fillcolor=curcolor]
{
\newpath
\moveto(943.03916294,70.7341566)
\lineto(943.03916294,71.88955621)
\lineto(944.04385826,71.88955621)
\lineto(944.04385826,70.7341566)
\closepath
\moveto(943.03916294,63.70687103)
\lineto(943.03916294,69.63457339)
\lineto(944.04385826,69.63457339)
\lineto(944.04385826,63.70687103)
\closepath
}
}
{
\newrgbcolor{curcolor}{0 0 0}
\pscustom[linestyle=none,fillstyle=solid,fillcolor=curcolor]
{
\newpath
\moveto(945.57880835,63.70687103)
\lineto(945.57880835,69.63457339)
\lineto(946.48303413,69.63457339)
\lineto(946.48303413,68.79174565)
\curveto(946.91840001,69.44293133)(947.54726423,69.7685267)(948.36962867,69.76853277)
\curveto(948.72684977,69.7685267)(949.05523597,69.7043379)(949.35478824,69.57596616)
\curveto(949.65433146,69.44758269)(949.87852714,69.27920336)(950.02737594,69.07082769)
\curveto(950.17621434,68.86244128)(950.28040486,68.61498879)(950.33994782,68.32846948)
\curveto(950.3771532,68.14241036)(950.39575865,67.81681498)(950.39576422,67.35168237)
\lineto(950.39576422,63.70687103)
\lineto(949.39106891,63.70687103)
\lineto(949.39106891,67.31261089)
\curveto(949.39106434,67.72192718)(949.3519929,68.02798684)(949.27385445,68.23079077)
\curveto(949.19570711,68.43358565)(949.05709651,68.59545307)(948.85802223,68.71639351)
\curveto(948.65893988,68.83732392)(948.42544148,68.89779164)(948.15752633,68.89779683)
\curveto(947.72959764,68.89779164)(947.36027945,68.76197185)(947.04957066,68.49033706)
\curveto(946.73885742,68.2186927)(946.58350191,67.70332173)(946.58350367,66.9442226)
\lineto(946.58350367,63.70687103)
\closepath
}
}
{
\newrgbcolor{curcolor}{0 0 0}
\pscustom[linestyle=none,fillstyle=solid,fillcolor=curcolor]
{
\newpath
\moveto(955.82670168,63.70687103)
\lineto(955.82670168,64.57760697)
\curveto(955.36528187,63.9078099)(954.7382782,63.57291179)(953.94568878,63.57291166)
\curveto(953.59590356,63.57291179)(953.26937791,63.63989141)(952.96611085,63.77385072)
\curveto(952.66284023,63.9078099)(952.43771429,64.07618922)(952.29073233,64.2789892)
\curveto(952.14374817,64.48178803)(952.04048793,64.73017079)(951.98095128,65.02413822)
\curveto(951.94001849,65.22135468)(951.9195525,65.53392624)(951.91955323,65.96185385)
\lineto(951.91955323,69.63457339)
\lineto(952.92424855,69.63457339)
\lineto(952.92424855,66.34698705)
\curveto(952.92424681,65.82231072)(952.9447128,65.46880717)(952.98564659,65.28647533)
\curveto(953.04890333,65.02227636)(953.18286257,64.81482559)(953.38752472,64.6641224)
\curveto(953.59218247,64.5134173)(953.84521659,64.43806523)(954.14662785,64.43806596)
\curveto(954.44803318,64.43806523)(954.73083602,64.51527784)(954.99503722,64.66970404)
\curveto(955.25923081,64.82412832)(955.44621558,65.0343699)(955.55599211,65.30042944)
\curveto(955.66575989,65.56648578)(955.72064597,65.95254887)(955.72065051,66.45861987)
\lineto(955.72065051,69.63457339)
\lineto(956.72534582,69.63457339)
\lineto(956.72534582,63.70687103)
\closepath
}
}
{
\newrgbcolor{curcolor}{0 0 0}
\pscustom[linestyle=none,fillstyle=solid,fillcolor=curcolor]
{
\newpath
\moveto(957.63515436,63.70687103)
\lineto(959.80083093,66.78793667)
\lineto(957.79702194,69.63457339)
\lineto(959.05289108,69.63457339)
\lineto(959.9626985,68.24474487)
\curveto(960.13386623,67.98054294)(960.27154657,67.75913808)(960.37573991,67.58052964)
\curveto(960.53946505,67.8261177)(960.6901692,68.04380147)(960.8278528,68.23358159)
\lineto(961.82696648,69.63457339)
\lineto(963.02701921,69.63457339)
\lineto(960.9785571,66.84375307)
\lineto(963.18330515,63.70687103)
\lineto(961.94976257,63.70687103)
\lineto(960.73296491,65.54881244)
\lineto(960.40922976,66.04557846)
\lineto(958.85195202,63.70687103)
\closepath
}
}
{
\newrgbcolor{curcolor}{0 0 0}
\pscustom[linestyle=none,fillstyle=solid,fillcolor=curcolor]
{
\newpath
\moveto(972.19207232,64.67249486)
\lineto(972.19207232,63.70687103)
\lineto(966.78346254,63.70687103)
\curveto(966.77602001,63.94874189)(966.81509146,64.18131001)(966.90067699,64.40457611)
\curveto(967.03835686,64.77296333)(967.25883144,65.13576961)(967.56210141,65.49299604)
\curveto(967.86536912,65.85021889)(968.30352747,66.26325989)(968.87657778,66.73212026)
\curveto(969.76591586,67.46145088)(970.3668719,68.03915011)(970.6794477,68.46521968)
\curveto(970.99201502,68.89127973)(971.14830081,69.29408772)(971.14830552,69.67364488)
\curveto(971.14830081,70.07179554)(971.00596911,70.40762392)(970.72131001,70.68113101)
\curveto(970.43664234,70.95462415)(970.0654636,71.09137421)(969.6077727,71.0913816)
\curveto(969.12402783,71.09137421)(968.73703446,70.9462517)(968.44679145,70.65601363)
\curveto(968.15654442,70.36576166)(968.00956136,69.96388393)(968.00584184,69.45037925)
\lineto(966.97323832,69.55643042)
\curveto(967.04393849,70.32669021)(967.30999643,70.91369216)(967.77141293,71.31743804)
\curveto(968.23282676,71.7211687)(968.85238825,71.92303784)(969.63009926,71.92304605)
\curveto(970.41524607,71.92303784)(971.0366681,71.70535407)(971.49436724,71.2699941)
\curveto(971.95205625,70.834619)(972.18090329,70.29506095)(972.18090903,69.65131831)
\curveto(972.18090329,69.32385645)(972.11392367,69.00198216)(971.97996997,68.68569448)
\curveto(971.84600519,68.36939685)(971.62367006,68.03635929)(971.31296392,67.68658081)
\curveto(971.00224802,67.33679436)(970.48594678,66.85677375)(969.76405864,66.24651752)
\curveto(969.16123873,65.74044674)(968.77424536,65.39717618)(968.60307739,65.21670483)
\curveto(968.43190508,65.03623045)(968.29050366,64.85482731)(968.1788727,64.67249486)
\closepath
}
}
{
\newrgbcolor{curcolor}{0 0 0}
\pscustom[linestyle=none,fillstyle=solid,fillcolor=curcolor]
{
\newpath
\moveto(973.8386536,63.70687103)
\lineto(973.8386536,64.85110736)
\lineto(974.98288993,64.85110736)
\lineto(974.98288993,63.70687103)
\closepath
}
}
{
\newrgbcolor{curcolor}{0 0 0}
\pscustom[linestyle=none,fillstyle=solid,fillcolor=curcolor]
{
\newpath
\moveto(981.65853226,69.88574722)
\lineto(980.65941859,69.80760425)
\curveto(980.57010774,70.20203369)(980.44359068,70.48855763)(980.27986703,70.66717691)
\curveto(980.00822315,70.95369388)(979.67332504,71.09695585)(979.27517171,71.09696324)
\curveto(978.95515466,71.09695585)(978.67421237,71.00764969)(978.43234398,70.82904449)
\curveto(978.11604886,70.59832978)(977.86673583,70.26157113)(977.68440413,69.81876753)
\curveto(977.502069,69.37595171)(977.40718121,68.74522694)(977.39974046,67.92659136)
\curveto(977.64160988,68.29497505)(977.93743654,68.56847517)(978.28722132,68.74709253)
\curveto(978.63700147,68.92569981)(979.00352883,69.01500597)(979.38680452,69.01501128)
\curveto(980.05659732,69.01500597)(980.62685436,68.76848376)(981.09757738,68.2754439)
\curveto(981.56829014,67.7823949)(981.80364909,67.14515823)(981.80365492,66.36373198)
\curveto(981.80364909,65.85021889)(981.69294666,65.3729891)(981.4715473,64.93204115)
\curveto(981.25013695,64.49109076)(980.94593783,64.15340184)(980.55894906,63.91897338)
\curveto(980.17195111,63.68454449)(979.73286248,63.56733016)(979.24168187,63.56733002)
\curveto(978.40443334,63.56733016)(977.72161332,63.87525036)(977.19321975,64.49109154)
\curveto(976.66482375,65.10693116)(976.40062635,66.12185847)(976.40062678,67.53587651)
\curveto(976.40062635,69.11733595)(976.69273192,70.26715277)(977.27694436,70.98533043)
\curveto(977.78673239,71.61046627)(978.4732735,71.92303784)(979.33656976,71.92304605)
\curveto(979.98031497,71.92303784)(980.50777948,71.74256497)(980.91896488,71.38162691)
\curveto(981.33014038,71.0206735)(981.5766626,70.52204744)(981.65853226,69.88574722)
\closepath
\moveto(977.5560264,66.35815034)
\curveto(977.55602481,66.01208631)(977.62951634,65.6809093)(977.7765012,65.3646183)
\curveto(977.92348245,65.04832399)(978.12907268,64.80738341)(978.39327249,64.64179584)
\curveto(978.65746746,64.4762064)(978.93468867,64.39341214)(979.22493695,64.39341283)
\curveto(979.64913796,64.39341214)(980.01380478,64.56458229)(980.31893851,64.90692377)
\curveto(980.62406355,65.24926285)(980.77662824,65.71439911)(980.77663304,66.30233393)
\curveto(980.77662824,66.86793702)(980.62592409,67.31353755)(980.32452015,67.63913686)
\curveto(980.02310751,67.96472831)(979.64355632,68.127526)(979.18586546,68.12753042)
\curveto(978.73188926,68.127526)(978.34675644,67.96472831)(978.03046585,67.63913686)
\curveto(977.71417114,67.31353755)(977.55602481,66.88654247)(977.5560264,66.35815034)
\closepath
}
}
{
\newrgbcolor{curcolor}{0 0 0}
\pscustom[linestyle=none,fillstyle=solid,fillcolor=curcolor]
{
\newpath
\moveto(983.37210087,63.70687103)
\lineto(983.37210087,64.85110736)
\lineto(984.5163372,64.85110736)
\lineto(984.5163372,63.70687103)
\closepath
}
}
{
\newrgbcolor{curcolor}{0 0 0}
\pscustom[linestyle=none,fillstyle=solid,fillcolor=curcolor]
{
\newpath
\moveto(989.76307953,63.70687103)
\lineto(988.75838421,63.70687103)
\lineto(988.75838421,70.10901285)
\curveto(988.51651011,69.87829886)(988.19928718,69.64759128)(987.80671448,69.41688941)
\curveto(987.41413718,69.18617611)(987.0615639,69.01314543)(986.74899358,68.89779683)
\lineto(986.74899358,69.8690023)
\curveto(987.31087693,70.13319353)(987.80206082,70.45320727)(988.22254671,70.82904449)
\curveto(988.64302717,71.20486746)(988.94071437,71.56953428)(989.11560921,71.92304605)
\lineto(989.76307953,71.92304605)
\closepath
}
}
{
\newrgbcolor{curcolor}{0 0 0}
\pscustom[linestyle=none,fillstyle=solid,fillcolor=curcolor]
{
\newpath
\moveto(997.55504716,69.88574722)
\lineto(996.55593348,69.80760425)
\curveto(996.46662263,70.20203369)(996.34010557,70.48855763)(996.17638192,70.66717691)
\curveto(995.90473804,70.95369388)(995.56983993,71.09695585)(995.1716866,71.09696324)
\curveto(994.85166956,71.09695585)(994.57072726,71.00764969)(994.32885887,70.82904449)
\curveto(994.01256375,70.59832978)(993.76325072,70.26157113)(993.58091902,69.81876753)
\curveto(993.3985839,69.37595171)(993.3036961,68.74522694)(993.29625535,67.92659136)
\curveto(993.53812477,68.29497505)(993.83395143,68.56847517)(994.18373621,68.74709253)
\curveto(994.53351636,68.92569981)(994.90004373,69.01500597)(995.28331942,69.01501128)
\curveto(995.95311221,69.01500597)(996.52336926,68.76848376)(996.99409227,68.2754439)
\curveto(997.46480504,67.7823949)(997.70016398,67.14515823)(997.70016981,66.36373198)
\curveto(997.70016398,65.85021889)(997.58946155,65.3729891)(997.3680622,64.93204115)
\curveto(997.14665184,64.49109076)(996.84245273,64.15340184)(996.45546395,63.91897338)
\curveto(996.068466,63.68454449)(995.62937737,63.56733016)(995.13819676,63.56733002)
\curveto(994.30094823,63.56733016)(993.61812821,63.87525036)(993.08973465,64.49109154)
\curveto(992.56133864,65.10693116)(992.29714125,66.12185847)(992.29714168,67.53587651)
\curveto(992.29714125,69.11733595)(992.58924681,70.26715277)(993.17345926,70.98533043)
\curveto(993.68324729,71.61046627)(994.3697884,71.92303784)(995.23308465,71.92304605)
\curveto(995.87682986,71.92303784)(996.40429438,71.74256497)(996.81547977,71.38162691)
\curveto(997.22665527,71.0206735)(997.47317749,70.52204744)(997.55504716,69.88574722)
\closepath
\moveto(993.45254129,66.35815034)
\curveto(993.4525397,66.01208631)(993.52603123,65.6809093)(993.67301609,65.3646183)
\curveto(993.81999734,65.04832399)(994.02558757,64.80738341)(994.28978738,64.64179584)
\curveto(994.55398235,64.4762064)(994.83120356,64.39341214)(995.12145184,64.39341283)
\curveto(995.54565285,64.39341214)(995.91031967,64.56458229)(996.2154534,64.90692377)
\curveto(996.52057844,65.24926285)(996.67314313,65.71439911)(996.67314794,66.30233393)
\curveto(996.67314313,66.86793702)(996.52243898,67.31353755)(996.22103504,67.63913686)
\curveto(995.9196224,67.96472831)(995.54007121,68.127526)(995.08238035,68.12753042)
\curveto(994.62840415,68.127526)(994.24327134,67.96472831)(993.92698074,67.63913686)
\curveto(993.61068603,67.31353755)(993.4525397,66.88654247)(993.45254129,66.35815034)
\closepath
}
}
{
\newrgbcolor{curcolor}{0 0 0}
\pscustom[linestyle=none,fillstyle=solid,fillcolor=curcolor]
{
\newpath
\moveto(7.48074186,1693.84930038)
\lineto(6.47604655,1693.84930038)
\lineto(6.47604655,1700.2514422)
\curveto(6.23417244,1700.02072821)(5.91694951,1699.79002063)(5.52437682,1699.55931876)
\curveto(5.13179952,1699.32860547)(4.77922624,1699.15557478)(4.46665592,1699.04022618)
\lineto(4.46665592,1700.01143165)
\curveto(5.02853927,1700.27562288)(5.51972315,1700.59563662)(5.94020905,1700.97147384)
\curveto(6.3606895,1701.34729681)(6.6583767,1701.71196363)(6.83327155,1702.0654754)
\lineto(7.48074186,1702.0654754)
\closepath
}
}
{
\newrgbcolor{curcolor}{0 0 0}
\pscustom[linestyle=none,fillstyle=solid,fillcolor=curcolor]
{
\newpath
\moveto(10.21016429,1695.74147656)
\lineto(11.17578812,1695.83078281)
\curveto(11.25765051,1695.37680785)(11.4139363,1695.04749138)(11.64464594,1694.84283242)
\curveto(11.87535146,1694.63817147)(12.17117812,1694.5358415)(12.5321268,1694.53584218)
\curveto(12.84097433,1694.5358415)(13.11168363,1694.60654221)(13.34425551,1694.74794453)
\curveto(13.57681988,1694.88934505)(13.76752574,1695.07819037)(13.91637368,1695.31448105)
\curveto(14.06521295,1695.5507688)(14.18986946,1695.86985228)(14.2903436,1696.27173242)
\curveto(14.39080833,1696.67360772)(14.44104304,1697.08292763)(14.4410479,1697.49969336)
\curveto(14.44104304,1697.54434279)(14.4391825,1697.61132241)(14.43546626,1697.70063243)
\curveto(14.23452254,1697.38061483)(13.96009215,1697.1210688)(13.61217426,1696.92199356)
\curveto(13.26424832,1696.72291217)(12.88748795,1696.62337301)(12.48189203,1696.62337578)
\curveto(11.80465075,1696.62337301)(11.23160288,1696.86896495)(10.76274672,1697.36015235)
\curveto(10.29388819,1697.85133272)(10.05945952,1698.49880239)(10.05946,1699.30256329)
\curveto(10.05945952,1700.13236091)(10.30412119,1700.80029658)(10.79344574,1701.30637228)
\curveto(11.28276787,1701.81243307)(11.89581746,1702.06546719)(12.63259633,1702.0654754)
\curveto(13.16470916,1702.06546719)(13.65124168,1701.92220522)(14.09219536,1701.63568908)
\curveto(14.53314002,1701.34915736)(14.86803812,1700.94076772)(15.09689067,1700.41051896)
\curveto(15.3257322,1699.88025706)(15.44015572,1699.11278224)(15.44016157,1698.10809219)
\curveto(15.44015572,1697.06246163)(15.32666247,1696.22986774)(15.09968149,1695.61030801)
\curveto(14.87268949,1694.99074475)(14.53500056,1694.51909659)(14.08661372,1694.1953621)
\curveto(13.63821787,1693.87162693)(13.1126139,1693.70975951)(12.50980024,1693.70975937)
\curveto(11.86976983,1693.70975951)(11.34695668,1693.88744156)(10.94135922,1694.24280605)
\curveto(10.53575905,1694.59816975)(10.29202765,1695.09772609)(10.21016429,1695.74147656)
\closepath
\moveto(14.32383344,1699.35279805)
\curveto(14.32382871,1699.92956151)(14.17033374,1700.38725558)(13.86334809,1700.72588165)
\curveto(13.55635389,1701.06449397)(13.1870357,1701.23380356)(12.75539242,1701.23381095)
\curveto(12.30885845,1701.23380356)(11.92000454,1701.05147015)(11.58882953,1700.68681017)
\curveto(11.25765051,1700.32213651)(11.09206201,1699.84955807)(11.09206351,1699.26907344)
\curveto(11.09206201,1698.74811542)(11.24927806,1698.32484143)(11.56371215,1697.9992502)
\curveto(11.87814228,1697.67365067)(12.26606591,1697.51085298)(12.72748422,1697.51085664)
\curveto(13.19261733,1697.51085298)(13.57495934,1697.67365067)(13.87451137,1697.9992502)
\curveto(14.17405483,1698.32484143)(14.32382871,1698.77602359)(14.32383344,1699.35279805)
\closepath
}
}
{
\newrgbcolor{curcolor}{0 0 0}
\pscustom[linestyle=none,fillstyle=solid,fillcolor=curcolor]
{
\newpath
\moveto(21.63578284,1700.02817657)
\lineto(20.63666917,1699.9500336)
\curveto(20.54735832,1700.34446305)(20.42084126,1700.63098698)(20.2571176,1700.80960626)
\curveto(19.98547372,1701.09612323)(19.65057562,1701.2393852)(19.25242229,1701.23939259)
\curveto(18.93240524,1701.2393852)(18.65146294,1701.15007904)(18.40959455,1700.97147384)
\curveto(18.09329944,1700.74075913)(17.8439864,1700.40400049)(17.66165471,1699.96119688)
\curveto(17.47931958,1699.51838106)(17.38443178,1698.8876563)(17.37699103,1698.06902071)
\curveto(17.61886046,1698.4374044)(17.91468712,1698.71090452)(18.2644719,1698.88952188)
\curveto(18.61425204,1699.06812916)(18.98077941,1699.15743532)(19.3640551,1699.15744063)
\curveto(20.03384789,1699.15743532)(20.60410494,1698.91091311)(21.07482796,1698.41787325)
\curveto(21.54554072,1697.92482425)(21.78089966,1697.28758758)(21.7809055,1696.50616133)
\curveto(21.78089966,1695.99264825)(21.67019724,1695.51541845)(21.44879788,1695.0744705)
\curveto(21.22738752,1694.63352011)(20.92318841,1694.29583119)(20.53619963,1694.06140273)
\curveto(20.14920168,1693.82697384)(19.71011306,1693.70975951)(19.21893244,1693.70975937)
\curveto(18.38168391,1693.70975951)(17.69886389,1694.01767971)(17.17047033,1694.63352089)
\curveto(16.64207432,1695.24936051)(16.37787693,1696.26428782)(16.37787736,1697.67830586)
\curveto(16.37787693,1699.2597653)(16.6699825,1700.40958212)(17.25419494,1701.12775978)
\curveto(17.76398297,1701.75289563)(18.45052408,1702.06546719)(19.31382034,1702.0654754)
\curveto(19.95756555,1702.06546719)(20.48503006,1701.88499432)(20.89621546,1701.52405626)
\curveto(21.30739096,1701.16310285)(21.55391317,1700.66447679)(21.63578284,1700.02817657)
\closepath
\moveto(17.53327697,1696.50057969)
\curveto(17.53327539,1696.15451566)(17.60676691,1695.82333865)(17.75375178,1695.50704765)
\curveto(17.90073303,1695.19075334)(18.10632325,1694.94981276)(18.37052307,1694.78422519)
\curveto(18.63471804,1694.61863575)(18.91193925,1694.5358415)(19.20218752,1694.53584218)
\curveto(19.62638853,1694.5358415)(19.99105536,1694.70701164)(20.29618909,1695.04935312)
\curveto(20.60131412,1695.39169221)(20.75387881,1695.85682846)(20.75388362,1696.44476328)
\curveto(20.75387881,1697.01036637)(20.60317467,1697.4559669)(20.30177073,1697.78156621)
\curveto(20.00035808,1698.10715766)(19.6208069,1698.26995535)(19.16311604,1698.26995977)
\curveto(18.70913984,1698.26995535)(18.32400702,1698.10715766)(18.00771643,1697.78156621)
\curveto(17.69142171,1697.4559669)(17.53327539,1697.02897182)(17.53327697,1696.50057969)
\closepath
}
}
{
\newrgbcolor{curcolor}{0 0 0}
\pscustom[linestyle=none,fillstyle=solid,fillcolor=curcolor]
{
\newpath
\moveto(22.93630432,1695.74147656)
\lineto(23.90192815,1695.83078281)
\curveto(23.98379054,1695.37680785)(24.14007632,1695.04749138)(24.37078596,1694.84283242)
\curveto(24.60149148,1694.63817147)(24.89731814,1694.5358415)(25.25826682,1694.53584218)
\curveto(25.56711435,1694.5358415)(25.83782365,1694.60654221)(26.07039553,1694.74794453)
\curveto(26.3029599,1694.88934505)(26.49366577,1695.07819037)(26.6425137,1695.31448105)
\curveto(26.79135297,1695.5507688)(26.91600949,1695.86985228)(27.01648362,1696.27173242)
\curveto(27.11694835,1696.67360772)(27.16718306,1697.08292763)(27.16718792,1697.49969336)
\curveto(27.16718306,1697.54434279)(27.16532252,1697.61132241)(27.16160628,1697.70063243)
\curveto(26.96066257,1697.38061483)(26.68623218,1697.1210688)(26.33831429,1696.92199356)
\curveto(25.99038834,1696.72291217)(25.61362797,1696.62337301)(25.20803206,1696.62337578)
\curveto(24.53079077,1696.62337301)(23.95774291,1696.86896495)(23.48888674,1697.36015235)
\curveto(23.02002822,1697.85133272)(22.78559954,1698.49880239)(22.78560002,1699.30256329)
\curveto(22.78559954,1700.13236091)(23.03026121,1700.80029658)(23.51958576,1701.30637228)
\curveto(24.00890789,1701.81243307)(24.62195748,1702.06546719)(25.35873635,1702.0654754)
\curveto(25.89084918,1702.06546719)(26.3773817,1701.92220522)(26.81833538,1701.63568908)
\curveto(27.25928004,1701.34915736)(27.59417815,1700.94076772)(27.82303069,1700.41051896)
\curveto(28.05187222,1699.88025706)(28.16629574,1699.11278224)(28.16630159,1698.10809219)
\curveto(28.16629574,1697.06246163)(28.05280249,1696.22986774)(27.82582152,1695.61030801)
\curveto(27.59882951,1694.99074475)(27.26114059,1694.51909659)(26.81275374,1694.1953621)
\curveto(26.36435789,1693.87162693)(25.83875392,1693.70975951)(25.23594026,1693.70975937)
\curveto(24.59590985,1693.70975951)(24.0730967,1693.88744156)(23.66749924,1694.24280605)
\curveto(23.26189907,1694.59816975)(23.01816767,1695.09772609)(22.93630432,1695.74147656)
\closepath
\moveto(27.04997347,1699.35279805)
\curveto(27.04996873,1699.92956151)(26.89647376,1700.38725558)(26.58948811,1700.72588165)
\curveto(26.28249391,1701.06449397)(25.91317572,1701.23380356)(25.48153245,1701.23381095)
\curveto(25.03499847,1701.23380356)(24.64614456,1701.05147015)(24.31496955,1700.68681017)
\curveto(23.98379054,1700.32213651)(23.81820203,1699.84955807)(23.81820354,1699.26907344)
\curveto(23.81820203,1698.74811542)(23.97541808,1698.32484143)(24.28985217,1697.9992502)
\curveto(24.6042823,1697.67365067)(24.99220594,1697.51085298)(25.45362424,1697.51085664)
\curveto(25.91875736,1697.51085298)(26.30109936,1697.67365067)(26.6006514,1697.9992502)
\curveto(26.90019485,1698.32484143)(27.04996873,1698.77602359)(27.04997347,1699.35279805)
\closepath
}
}
{
\newrgbcolor{curcolor}{0 0 0}
\pscustom[linestyle=none,fillstyle=solid,fillcolor=curcolor]
{
\newpath
\moveto(7.48074186,1650.18220139)
\lineto(6.47604655,1650.18220139)
\lineto(6.47604655,1656.5843432)
\curveto(6.23417244,1656.35362921)(5.91694951,1656.12292163)(5.52437682,1655.89221976)
\curveto(5.13179952,1655.66150647)(4.77922624,1655.48847578)(4.46665592,1655.37312718)
\lineto(4.46665592,1656.34433265)
\curveto(5.02853927,1656.60852388)(5.51972315,1656.92853762)(5.94020905,1657.30437484)
\curveto(6.3606895,1657.68019781)(6.6583767,1658.04486464)(6.83327155,1658.39837641)
\lineto(7.48074186,1658.39837641)
\closepath
}
}
{
\newrgbcolor{curcolor}{0 0 0}
\pscustom[linestyle=none,fillstyle=solid,fillcolor=curcolor]
{
\newpath
\moveto(10.21016429,1652.07437756)
\lineto(11.17578812,1652.16368381)
\curveto(11.25765051,1651.70970885)(11.4139363,1651.38039238)(11.64464594,1651.17573342)
\curveto(11.87535146,1650.97107247)(12.17117812,1650.8687425)(12.5321268,1650.86874318)
\curveto(12.84097433,1650.8687425)(13.11168363,1650.93944321)(13.34425551,1651.08084553)
\curveto(13.57681988,1651.22224605)(13.76752574,1651.41109137)(13.91637368,1651.64738205)
\curveto(14.06521295,1651.88366981)(14.18986946,1652.20275328)(14.2903436,1652.60463342)
\curveto(14.39080833,1653.00650872)(14.44104304,1653.41582863)(14.4410479,1653.83259436)
\curveto(14.44104304,1653.87724379)(14.4391825,1653.94422341)(14.43546626,1654.03353343)
\curveto(14.23452254,1653.71351583)(13.96009215,1653.4539698)(13.61217426,1653.25489456)
\curveto(13.26424832,1653.05581317)(12.88748795,1652.95627401)(12.48189203,1652.95627678)
\curveto(11.80465075,1652.95627401)(11.23160288,1653.20186595)(10.76274672,1653.69305335)
\curveto(10.29388819,1654.18423372)(10.05945952,1654.83170339)(10.05946,1655.63546429)
\curveto(10.05945952,1656.46526191)(10.30412119,1657.13319758)(10.79344574,1657.63927328)
\curveto(11.28276787,1658.14533407)(11.89581746,1658.39836819)(12.63259633,1658.39837641)
\curveto(13.16470916,1658.39836819)(13.65124168,1658.25510622)(14.09219536,1657.96859008)
\curveto(14.53314002,1657.68205836)(14.86803812,1657.27366873)(15.09689067,1656.74341996)
\curveto(15.3257322,1656.21315806)(15.44015572,1655.44568324)(15.44016157,1654.44099319)
\curveto(15.44015572,1653.39536263)(15.32666247,1652.56276874)(15.09968149,1651.94320901)
\curveto(14.87268949,1651.32364575)(14.53500056,1650.85199759)(14.08661372,1650.52826311)
\curveto(13.63821787,1650.20452793)(13.1126139,1650.04266051)(12.50980024,1650.04266037)
\curveto(11.86976983,1650.04266051)(11.34695668,1650.22034256)(10.94135922,1650.57570705)
\curveto(10.53575905,1650.93107076)(10.29202765,1651.43062709)(10.21016429,1652.07437756)
\closepath
\moveto(14.32383344,1655.68569906)
\curveto(14.32382871,1656.26246251)(14.17033374,1656.72015658)(13.86334809,1657.05878265)
\curveto(13.55635389,1657.39739497)(13.1870357,1657.56670457)(12.75539242,1657.56671195)
\curveto(12.30885845,1657.56670457)(11.92000454,1657.38437115)(11.58882953,1657.01971117)
\curveto(11.25765051,1656.65503751)(11.09206201,1656.18245907)(11.09206351,1655.60197445)
\curveto(11.09206201,1655.08101642)(11.24927806,1654.65774243)(11.56371215,1654.3321512)
\curveto(11.87814228,1654.00655167)(12.26606591,1653.84375398)(12.72748422,1653.84375764)
\curveto(13.19261733,1653.84375398)(13.57495934,1654.00655167)(13.87451137,1654.3321512)
\curveto(14.17405483,1654.65774243)(14.32382871,1655.1089246)(14.32383344,1655.68569906)
\closepath
}
}
{
\newrgbcolor{curcolor}{0 0 0}
\pscustom[linestyle=none,fillstyle=solid,fillcolor=curcolor]
{
\newpath
\moveto(16.48951017,1657.29321156)
\lineto(16.48951017,1658.25883539)
\lineto(21.78648714,1658.25883539)
\lineto(21.78648714,1657.4774057)
\curveto(21.26552869,1656.92295599)(20.74922745,1656.18618016)(20.23758186,1655.26707601)
\curveto(19.72592769,1654.34796168)(19.33056187,1653.40280481)(19.05148323,1652.43160256)
\curveto(18.85054126,1651.74691975)(18.72216365,1650.9971201)(18.66635002,1650.18220139)
\lineto(17.6337465,1650.18220139)
\curveto(17.64490809,1650.82594996)(17.77142515,1651.60365778)(18.01329807,1652.51532717)
\curveto(18.25516685,1653.4269919)(18.6021585,1654.30609942)(19.05427405,1655.15265237)
\curveto(19.50638338,1655.99919539)(19.98733427,1656.7127144)(20.49712815,1657.29321156)
\closepath
}
}
{
\newrgbcolor{curcolor}{0 0 0}
\pscustom[linestyle=none,fillstyle=solid,fillcolor=curcolor]
{
\newpath
\moveto(22.78560002,1654.21772757)
\curveto(22.78559954,1655.18520694)(22.8851387,1655.96384503)(23.08421779,1656.55364417)
\curveto(23.28329534,1657.14343057)(23.57912199,1657.59833383)(23.97169865,1657.91835531)
\curveto(24.36427199,1658.23836132)(24.8582467,1658.39836819)(25.45362424,1658.39837641)
\curveto(25.89270973,1658.39836819)(26.27784254,1658.3099923)(26.60902386,1658.13324848)
\curveto(26.94019657,1657.95648875)(27.21369669,1657.70159408)(27.42952503,1657.36856371)
\curveto(27.64534313,1657.03551896)(27.81465273,1656.62992015)(27.93745433,1656.15176605)
\curveto(28.06024467,1655.67360001)(28.12164266,1655.02892116)(28.12164847,1654.21772757)
\curveto(28.12164266,1653.2576823)(28.02303377,1652.4827653)(27.82582152,1651.89297424)
\curveto(27.62859823,1651.30317976)(27.33370184,1650.84734623)(26.94113147,1650.52547229)
\curveto(26.54855185,1650.20359765)(26.0527166,1650.04266051)(25.45362424,1650.04266037)
\curveto(24.66475001,1650.04266051)(24.04518852,1650.32546335)(23.59493791,1650.89106975)
\curveto(23.05537857,1651.57202851)(22.78559954,1652.68091335)(22.78560002,1654.21772757)
\closepath
\moveto(23.81820354,1654.21772757)
\curveto(23.81820203,1652.87441003)(23.97541808,1651.98041815)(24.28985217,1651.53574924)
\curveto(24.6042823,1651.09107763)(24.99220594,1650.8687425)(25.45362424,1650.86874318)
\curveto(25.91503627,1650.8687425)(26.3029599,1651.0920079)(26.61739632,1651.53854006)
\curveto(26.93182412,1651.98506951)(27.08904017,1652.87813112)(27.08904495,1654.21772757)
\curveto(27.08904017,1655.56475813)(26.93182412,1656.45968028)(26.61739632,1656.90249671)
\curveto(26.3029599,1657.34529971)(25.91131518,1657.56670457)(25.44246096,1657.56671195)
\curveto(24.98104267,1657.56670457)(24.61265475,1657.37134734)(24.33729612,1656.98063968)
\curveto(23.99123272,1656.48200682)(23.81820203,1655.56103704)(23.81820354,1654.21772757)
\closepath
}
}
{
\newrgbcolor{curcolor}{0 0 0}
\pscustom[linestyle=none,fillstyle=solid,fillcolor=curcolor]
{
\newpath
\moveto(7.48074186,1606.51490402)
\lineto(6.47604655,1606.51490402)
\lineto(6.47604655,1612.91704583)
\curveto(6.23417244,1612.68633185)(5.91694951,1612.45562427)(5.52437682,1612.2249224)
\curveto(5.13179952,1611.9942091)(4.77922624,1611.82117842)(4.46665592,1611.70582982)
\lineto(4.46665592,1612.67703529)
\curveto(5.02853927,1612.94122652)(5.51972315,1613.26124026)(5.94020905,1613.63707748)
\curveto(6.3606895,1614.01290045)(6.6583767,1614.37756727)(6.83327155,1614.73107904)
\lineto(7.48074186,1614.73107904)
\closepath
}
}
{
\newrgbcolor{curcolor}{0 0 0}
\pscustom[linestyle=none,fillstyle=solid,fillcolor=curcolor]
{
\newpath
\moveto(10.21016429,1608.4070802)
\lineto(11.17578812,1608.49638645)
\curveto(11.25765051,1608.04241148)(11.4139363,1607.71309501)(11.64464594,1607.50843606)
\curveto(11.87535146,1607.30377511)(12.17117812,1607.20144513)(12.5321268,1607.20144582)
\curveto(12.84097433,1607.20144513)(13.11168363,1607.27214585)(13.34425551,1607.41354817)
\curveto(13.57681988,1607.55494869)(13.76752574,1607.74379401)(13.91637368,1607.98008469)
\curveto(14.06521295,1608.21637244)(14.18986946,1608.53545591)(14.2903436,1608.93733606)
\curveto(14.39080833,1609.33921136)(14.44104304,1609.74853127)(14.4410479,1610.165297)
\curveto(14.44104304,1610.20994643)(14.4391825,1610.27692605)(14.43546626,1610.36623606)
\curveto(14.23452254,1610.04621847)(13.96009215,1609.78667244)(13.61217426,1609.58759719)
\curveto(13.26424832,1609.3885158)(12.88748795,1609.28897665)(12.48189203,1609.28897942)
\curveto(11.80465075,1609.28897665)(11.23160288,1609.53456859)(10.76274672,1610.02575598)
\curveto(10.29388819,1610.51693636)(10.05945952,1611.16440602)(10.05946,1611.96816693)
\curveto(10.05945952,1612.79796455)(10.30412119,1613.46590021)(10.79344574,1613.97197592)
\curveto(11.28276787,1614.4780367)(11.89581746,1614.73107083)(12.63259633,1614.73107904)
\curveto(13.16470916,1614.73107083)(13.65124168,1614.58780886)(14.09219536,1614.30129271)
\curveto(14.53314002,1614.01476099)(14.86803812,1613.60637136)(15.09689067,1613.07612259)
\curveto(15.3257322,1612.5458607)(15.44015572,1611.77838588)(15.44016157,1610.77369583)
\curveto(15.44015572,1609.72806527)(15.32666247,1608.89547137)(15.09968149,1608.27591164)
\curveto(14.87268949,1607.65634839)(14.53500056,1607.18470023)(14.08661372,1606.86096574)
\curveto(13.63821787,1606.53723056)(13.1126139,1606.37536315)(12.50980024,1606.37536301)
\curveto(11.86976983,1606.37536315)(11.34695668,1606.5530452)(10.94135922,1606.90840969)
\curveto(10.53575905,1607.26377339)(10.29202765,1607.76332973)(10.21016429,1608.4070802)
\closepath
\moveto(14.32383344,1612.01840169)
\curveto(14.32382871,1612.59516514)(14.17033374,1613.05285922)(13.86334809,1613.39148529)
\curveto(13.55635389,1613.73009761)(13.1870357,1613.8994072)(12.75539242,1613.89941459)
\curveto(12.30885845,1613.8994072)(11.92000454,1613.71707379)(11.58882953,1613.3524138)
\curveto(11.25765051,1612.98774014)(11.09206201,1612.51516171)(11.09206351,1611.93467708)
\curveto(11.09206201,1611.41371906)(11.24927806,1610.99044507)(11.56371215,1610.66485384)
\curveto(11.87814228,1610.33925431)(12.26606591,1610.17645662)(12.72748422,1610.17646028)
\curveto(13.19261733,1610.17645662)(13.57495934,1610.33925431)(13.87451137,1610.66485384)
\curveto(14.17405483,1610.99044507)(14.32382871,1611.44162723)(14.32383344,1612.01840169)
\closepath
}
}
{
\newrgbcolor{curcolor}{0 0 0}
\pscustom[linestyle=none,fillstyle=solid,fillcolor=curcolor]
{
\newpath
\moveto(16.48951017,1613.6259142)
\lineto(16.48951017,1614.59153803)
\lineto(21.78648714,1614.59153803)
\lineto(21.78648714,1613.81010834)
\curveto(21.26552869,1613.25565863)(20.74922745,1612.5188828)(20.23758186,1611.59977864)
\curveto(19.72592769,1610.68066432)(19.33056187,1609.73550745)(19.05148323,1608.7643052)
\curveto(18.85054126,1608.07962238)(18.72216365,1607.32982274)(18.66635002,1606.51490402)
\lineto(17.6337465,1606.51490402)
\curveto(17.64490809,1607.1586526)(17.77142515,1607.93636042)(18.01329807,1608.84802981)
\curveto(18.25516685,1609.75969454)(18.6021585,1610.63880206)(19.05427405,1611.48535501)
\curveto(19.50638338,1612.33189802)(19.98733427,1613.04541704)(20.49712815,1613.6259142)
\closepath
}
}
{
\newrgbcolor{curcolor}{0 0 0}
\pscustom[linestyle=none,fillstyle=solid,fillcolor=curcolor]
{
\newpath
\moveto(26.56995237,1606.51490402)
\lineto(25.56525706,1606.51490402)
\lineto(25.56525706,1612.91704583)
\curveto(25.32338295,1612.68633185)(25.00616002,1612.45562427)(24.61358733,1612.2249224)
\curveto(24.22101003,1611.9942091)(23.86843675,1611.82117842)(23.55586643,1611.70582982)
\lineto(23.55586643,1612.67703529)
\curveto(24.11774978,1612.94122652)(24.60893366,1613.26124026)(25.02941956,1613.63707748)
\curveto(25.44990001,1614.01290045)(25.74758721,1614.37756727)(25.92248206,1614.73107904)
\lineto(26.56995237,1614.73107904)
\closepath
}
}
{
\newrgbcolor{curcolor}{0 0 0}
\pscustom[linestyle=none,fillstyle=solid,fillcolor=curcolor]
{
\newpath
\moveto(7.48074186,1562.84789658)
\lineto(6.47604655,1562.84789658)
\lineto(6.47604655,1569.25003839)
\curveto(6.23417244,1569.0193244)(5.91694951,1568.78861682)(5.52437682,1568.55791495)
\curveto(5.13179952,1568.32720166)(4.77922624,1568.15417097)(4.46665592,1568.03882237)
\lineto(4.46665592,1569.01002784)
\curveto(5.02853927,1569.27421907)(5.51972315,1569.59423281)(5.94020905,1569.97007003)
\curveto(6.3606895,1570.345893)(6.6583767,1570.71055983)(6.83327155,1571.0640716)
\lineto(7.48074186,1571.0640716)
\closepath
}
}
{
\newrgbcolor{curcolor}{0 0 0}
\pscustom[linestyle=none,fillstyle=solid,fillcolor=curcolor]
{
\newpath
\moveto(10.21016429,1564.74007275)
\lineto(11.17578812,1564.829379)
\curveto(11.25765051,1564.37540404)(11.4139363,1564.04608757)(11.64464594,1563.84142861)
\curveto(11.87535146,1563.63676766)(12.17117812,1563.53443769)(12.5321268,1563.53443837)
\curveto(12.84097433,1563.53443769)(13.11168363,1563.6051384)(13.34425551,1563.74654072)
\curveto(13.57681988,1563.88794124)(13.76752574,1564.07678656)(13.91637368,1564.31307724)
\curveto(14.06521295,1564.549365)(14.18986946,1564.86844847)(14.2903436,1565.27032861)
\curveto(14.39080833,1565.67220391)(14.44104304,1566.08152382)(14.4410479,1566.49828955)
\curveto(14.44104304,1566.54293898)(14.4391825,1566.6099186)(14.43546626,1566.69922862)
\curveto(14.23452254,1566.37921102)(13.96009215,1566.11966499)(13.61217426,1565.92058975)
\curveto(13.26424832,1565.72150836)(12.88748795,1565.6219692)(12.48189203,1565.62197197)
\curveto(11.80465075,1565.6219692)(11.23160288,1565.86756114)(10.76274672,1566.35874854)
\curveto(10.29388819,1566.84992891)(10.05945952,1567.49739858)(10.05946,1568.30115948)
\curveto(10.05945952,1569.13095711)(10.30412119,1569.79889277)(10.79344574,1570.30496847)
\curveto(11.28276787,1570.81102926)(11.89581746,1571.06406338)(12.63259633,1571.0640716)
\curveto(13.16470916,1571.06406338)(13.65124168,1570.92080141)(14.09219536,1570.63428527)
\curveto(14.53314002,1570.34775355)(14.86803812,1569.93936392)(15.09689067,1569.40911515)
\curveto(15.3257322,1568.87885326)(15.44015572,1568.11137843)(15.44016157,1567.10668838)
\curveto(15.44015572,1566.06105782)(15.32666247,1565.22846393)(15.09968149,1564.6089042)
\curveto(14.87268949,1563.98934095)(14.53500056,1563.51769278)(14.08661372,1563.1939583)
\curveto(13.63821787,1562.87022312)(13.1126139,1562.7083557)(12.50980024,1562.70835556)
\curveto(11.86976983,1562.7083557)(11.34695668,1562.88603775)(10.94135922,1563.24140224)
\curveto(10.53575905,1563.59676595)(10.29202765,1564.09632228)(10.21016429,1564.74007275)
\closepath
\moveto(14.32383344,1568.35139425)
\curveto(14.32382871,1568.9281577)(14.17033374,1569.38585177)(13.86334809,1569.72447784)
\curveto(13.55635389,1570.06309016)(13.1870357,1570.23239976)(12.75539242,1570.23240714)
\curveto(12.30885845,1570.23239976)(11.92000454,1570.05006634)(11.58882953,1569.68540636)
\curveto(11.25765051,1569.3207327)(11.09206201,1568.84815426)(11.09206351,1568.26766964)
\curveto(11.09206201,1567.74671161)(11.24927806,1567.32343762)(11.56371215,1566.99784639)
\curveto(11.87814228,1566.67224686)(12.26606591,1566.50944917)(12.72748422,1566.50945284)
\curveto(13.19261733,1566.50944917)(13.57495934,1566.67224686)(13.87451137,1566.99784639)
\curveto(14.17405483,1567.32343762)(14.32382871,1567.77461979)(14.32383344,1568.35139425)
\closepath
}
}
{
\newrgbcolor{curcolor}{0 0 0}
\pscustom[linestyle=none,fillstyle=solid,fillcolor=curcolor]
{
\newpath
\moveto(16.48951017,1569.95890675)
\lineto(16.48951017,1570.92453058)
\lineto(21.78648714,1570.92453058)
\lineto(21.78648714,1570.14310089)
\curveto(21.26552869,1569.58865118)(20.74922745,1568.85187535)(20.23758186,1567.9327712)
\curveto(19.72592769,1567.01365687)(19.33056187,1566.0685)(19.05148323,1565.09729775)
\curveto(18.85054126,1564.41261494)(18.72216365,1563.66281529)(18.66635002,1562.84789658)
\lineto(17.6337465,1562.84789658)
\curveto(17.64490809,1563.49164515)(17.77142515,1564.26935297)(18.01329807,1565.18102236)
\curveto(18.25516685,1566.09268709)(18.6021585,1566.97179461)(19.05427405,1567.81834756)
\curveto(19.50638338,1568.66489058)(19.98733427,1569.37840959)(20.49712815,1569.95890675)
\closepath
}
}
{
\newrgbcolor{curcolor}{0 0 0}
\pscustom[linestyle=none,fillstyle=solid,fillcolor=curcolor]
{
\newpath
\moveto(28.06583206,1563.81352041)
\lineto(28.06583206,1562.84789658)
\lineto(22.65722228,1562.84789658)
\curveto(22.64977976,1563.08976743)(22.6888512,1563.32233556)(22.77443674,1563.54560166)
\curveto(22.91211661,1563.91398887)(23.13259119,1564.27679515)(23.43586115,1564.63402158)
\curveto(23.73912887,1564.99124444)(24.17728722,1565.40428543)(24.75033752,1565.8731458)
\curveto(25.6396756,1566.60247642)(26.24063164,1567.18017565)(26.55320745,1567.60624522)
\curveto(26.86577477,1568.03230527)(27.02206055,1568.43511327)(27.02206526,1568.81467042)
\curveto(27.02206055,1569.21282109)(26.87972886,1569.54864946)(26.59506975,1569.82215655)
\curveto(26.31040208,1570.0956497)(25.93922335,1570.23239976)(25.48153245,1570.23240714)
\curveto(24.99778757,1570.23239976)(24.61079421,1570.08727724)(24.32055119,1569.79703917)
\curveto(24.03030416,1569.5067872)(23.88332111,1569.10490947)(23.87960158,1568.59140479)
\lineto(22.84699807,1568.69745597)
\curveto(22.91769824,1569.46771575)(23.18375618,1570.05471771)(23.64517268,1570.45846359)
\curveto(24.10658651,1570.86219425)(24.726148,1571.06406338)(25.50385901,1571.0640716)
\curveto(26.28900581,1571.06406338)(26.91042785,1570.84637961)(27.36812698,1570.41101964)
\curveto(27.825816,1569.97564454)(28.05466304,1569.43608649)(28.05466878,1568.79234386)
\curveto(28.05466304,1568.46488199)(27.98768342,1568.1430077)(27.85372972,1567.82672003)
\curveto(27.71976493,1567.51042239)(27.4974298,1567.17738484)(27.18672366,1566.82760635)
\curveto(26.87600777,1566.47781991)(26.35970653,1565.99779929)(25.63781838,1565.38754307)
\curveto(25.03499847,1564.88147228)(24.64800511,1564.53820173)(24.47683713,1564.35773037)
\curveto(24.30566482,1564.17725599)(24.1642634,1563.99585285)(24.05263244,1563.81352041)
\closepath
}
}
{
\newrgbcolor{curcolor}{0 0 0}
\pscustom[linestyle=none,fillstyle=solid,fillcolor=curcolor]
{
\newpath
\moveto(7.48074186,1519.18090439)
\lineto(6.47604655,1519.18090439)
\lineto(6.47604655,1525.5830462)
\curveto(6.23417244,1525.35233222)(5.91694951,1525.12162463)(5.52437682,1524.89092276)
\curveto(5.13179952,1524.66020947)(4.77922624,1524.48717878)(4.46665592,1524.37183018)
\lineto(4.46665592,1525.34303565)
\curveto(5.02853927,1525.60722688)(5.51972315,1525.92724063)(5.94020905,1526.30307784)
\curveto(6.3606895,1526.67890081)(6.6583767,1527.04356764)(6.83327155,1527.39707941)
\lineto(7.48074186,1527.39707941)
\closepath
}
}
{
\newrgbcolor{curcolor}{0 0 0}
\pscustom[linestyle=none,fillstyle=solid,fillcolor=curcolor]
{
\newpath
\moveto(10.21016429,1521.07308057)
\lineto(11.17578812,1521.16238682)
\curveto(11.25765051,1520.70841185)(11.4139363,1520.37909538)(11.64464594,1520.17443642)
\curveto(11.87535146,1519.96977548)(12.17117812,1519.8674455)(12.5321268,1519.86744619)
\curveto(12.84097433,1519.8674455)(13.11168363,1519.93814621)(13.34425551,1520.07954853)
\curveto(13.57681988,1520.22094905)(13.76752574,1520.40979437)(13.91637368,1520.64608506)
\curveto(14.06521295,1520.88237281)(14.18986946,1521.20145628)(14.2903436,1521.60333643)
\curveto(14.39080833,1522.00521173)(14.44104304,1522.41453163)(14.4410479,1522.83129737)
\curveto(14.44104304,1522.8759468)(14.4391825,1522.94292642)(14.43546626,1523.03223643)
\curveto(14.23452254,1522.71221883)(13.96009215,1522.4526728)(13.61217426,1522.25359756)
\curveto(13.26424832,1522.05451617)(12.88748795,1521.95497701)(12.48189203,1521.95497979)
\curveto(11.80465075,1521.95497701)(11.23160288,1522.20056895)(10.76274672,1522.69175635)
\curveto(10.29388819,1523.18293672)(10.05945952,1523.83040639)(10.05946,1524.63416729)
\curveto(10.05945952,1525.46396492)(10.30412119,1526.13190058)(10.79344574,1526.63797628)
\curveto(11.28276787,1527.14403707)(11.89581746,1527.39707119)(12.63259633,1527.39707941)
\curveto(13.16470916,1527.39707119)(13.65124168,1527.25380923)(14.09219536,1526.96729308)
\curveto(14.53314002,1526.68076136)(14.86803812,1526.27237173)(15.09689067,1525.74212296)
\curveto(15.3257322,1525.21186107)(15.44015572,1524.44438625)(15.44016157,1523.4396962)
\curveto(15.44015572,1522.39406564)(15.32666247,1521.56147174)(15.09968149,1520.94191201)
\curveto(14.87268949,1520.32234876)(14.53500056,1519.8507006)(14.08661372,1519.52696611)
\curveto(13.63821787,1519.20323093)(13.1126139,1519.04136351)(12.50980024,1519.04136337)
\curveto(11.86976983,1519.04136351)(11.34695668,1519.21904556)(10.94135922,1519.57441005)
\curveto(10.53575905,1519.92977376)(10.29202765,1520.4293301)(10.21016429,1521.07308057)
\closepath
\moveto(14.32383344,1524.68440206)
\curveto(14.32382871,1525.26116551)(14.17033374,1525.71885959)(13.86334809,1526.05748566)
\curveto(13.55635389,1526.39609797)(13.1870357,1526.56540757)(12.75539242,1526.56541495)
\curveto(12.30885845,1526.56540757)(11.92000454,1526.38307416)(11.58882953,1526.01841417)
\curveto(11.25765051,1525.65374051)(11.09206201,1525.18116207)(11.09206351,1524.60067745)
\curveto(11.09206201,1524.07971942)(11.24927806,1523.65644543)(11.56371215,1523.3308542)
\curveto(11.87814228,1523.00525468)(12.26606591,1522.84245699)(12.72748422,1522.84246065)
\curveto(13.19261733,1522.84245699)(13.57495934,1523.00525468)(13.87451137,1523.3308542)
\curveto(14.17405483,1523.65644543)(14.32382871,1524.1076276)(14.32383344,1524.68440206)
\closepath
}
}
{
\newrgbcolor{curcolor}{0 0 0}
\pscustom[linestyle=none,fillstyle=solid,fillcolor=curcolor]
{
\newpath
\moveto(16.48951017,1526.29191456)
\lineto(16.48951017,1527.25753839)
\lineto(21.78648714,1527.25753839)
\lineto(21.78648714,1526.4761087)
\curveto(21.26552869,1525.92165899)(20.74922745,1525.18488316)(20.23758186,1524.26577901)
\curveto(19.72592769,1523.34666469)(19.33056187,1522.40150782)(19.05148323,1521.43030557)
\curveto(18.85054126,1520.74562275)(18.72216365,1519.99582311)(18.66635002,1519.18090439)
\lineto(17.6337465,1519.18090439)
\curveto(17.64490809,1519.82465297)(17.77142515,1520.60236078)(18.01329807,1521.51403018)
\curveto(18.25516685,1522.4256949)(18.6021585,1523.30480242)(19.05427405,1524.15135538)
\curveto(19.50638338,1524.99789839)(19.98733427,1525.71141741)(20.49712815,1526.29191456)
\closepath
}
}
{
\newrgbcolor{curcolor}{0 0 0}
\pscustom[linestyle=none,fillstyle=solid,fillcolor=curcolor]
{
\newpath
\moveto(22.79118166,1521.34099932)
\lineto(23.79587697,1521.47495869)
\curveto(23.91122928,1520.90562962)(24.10751678,1520.49537944)(24.38474006,1520.24420693)
\curveto(24.6619592,1519.99303229)(24.99964812,1519.8674455)(25.39780784,1519.86744619)
\curveto(25.87038319,1519.8674455)(26.26947009,1520.03117346)(26.59506975,1520.35863056)
\curveto(26.92066085,1520.68608531)(27.08345854,1521.09168412)(27.08346331,1521.57542822)
\curveto(27.08345854,1522.03684099)(26.93275439,1522.41732245)(26.63135042,1522.71687373)
\curveto(26.32993781,1523.01641795)(25.94666553,1523.16619182)(25.48153245,1523.1661958)
\curveto(25.29175368,1523.16619182)(25.05546447,1523.12898092)(24.77266409,1523.05456299)
\lineto(24.8842969,1523.93646221)
\curveto(24.95127395,1523.92901528)(25.00522975,1523.92529419)(25.04616448,1523.92529893)
\curveto(25.4740871,1523.92529419)(25.85921992,1524.03692689)(26.20156409,1524.26019737)
\curveto(26.54390048,1524.48345769)(26.71507062,1524.82765852)(26.71507503,1525.29280089)
\curveto(26.71507062,1525.66118269)(26.59041411,1525.96631207)(26.34110511,1526.20818995)
\curveto(26.09178804,1526.45005378)(25.76991375,1526.5709892)(25.37548127,1526.57099659)
\curveto(24.98476376,1526.5709892)(24.65916838,1526.44819323)(24.39869416,1526.20260831)
\curveto(24.13821577,1525.95700935)(23.97076672,1525.58862143)(23.89634651,1525.09744347)
\lineto(22.89165119,1525.27605597)
\curveto(23.01444658,1525.94956717)(23.29352833,1526.47145005)(23.72889729,1526.84170616)
\curveto(24.1642634,1527.21194696)(24.705682,1527.39707119)(25.35315471,1527.39707941)
\curveto(25.79968247,1527.39707119)(26.21086292,1527.30125312)(26.58669729,1527.10962492)
\curveto(26.96252311,1526.91798085)(27.24997732,1526.65657427)(27.44906077,1526.32540441)
\curveto(27.64813395,1525.99422025)(27.74767311,1525.64257724)(27.74767855,1525.27047433)
\curveto(27.74767311,1524.91696468)(27.65278531,1524.59509039)(27.46301487,1524.30485049)
\curveto(27.27323413,1524.01460035)(26.99229183,1523.78389277)(26.62018714,1523.61272706)
\curveto(27.10392453,1523.50108992)(27.47975463,1523.26945207)(27.74767855,1522.9178128)
\curveto(28.01559159,1522.56616605)(28.14955083,1522.12614715)(28.14955667,1521.59775479)
\curveto(28.14955083,1520.88330308)(27.88907453,1520.27769568)(27.36812698,1519.78093076)
\curveto(26.84716932,1519.28416464)(26.18853638,1519.03578188)(25.3922262,1519.03578173)
\curveto(24.67405274,1519.03578188)(24.07774806,1519.24974455)(23.60331037,1519.67767041)
\curveto(23.1288701,1520.10559526)(22.8581608,1520.66003768)(22.79118166,1521.34099932)
\closepath
}
}
{
\newrgbcolor{curcolor}{0 0 0}
\pscustom[linestyle=none,fillstyle=solid,fillcolor=curcolor]
{
\newpath
\moveto(7.48074186,1475.51389694)
\lineto(6.47604655,1475.51389694)
\lineto(6.47604655,1481.91603875)
\curveto(6.23417244,1481.68532477)(5.91694951,1481.45461719)(5.52437682,1481.22391532)
\curveto(5.13179952,1480.99320202)(4.77922624,1480.82017134)(4.46665592,1480.70482274)
\lineto(4.46665592,1481.67602821)
\curveto(5.02853927,1481.94021944)(5.51972315,1482.26023318)(5.94020905,1482.6360704)
\curveto(6.3606895,1483.01189337)(6.6583767,1483.37656019)(6.83327155,1483.73007196)
\lineto(7.48074186,1483.73007196)
\closepath
}
}
{
\newrgbcolor{curcolor}{0 0 0}
\pscustom[linestyle=none,fillstyle=solid,fillcolor=curcolor]
{
\newpath
\moveto(10.21016429,1477.40607312)
\lineto(11.17578812,1477.49537937)
\curveto(11.25765051,1477.0414044)(11.4139363,1476.71208793)(11.64464594,1476.50742898)
\curveto(11.87535146,1476.30276803)(12.17117812,1476.20043805)(12.5321268,1476.20043874)
\curveto(12.84097433,1476.20043805)(13.11168363,1476.27113877)(13.34425551,1476.41254109)
\curveto(13.57681988,1476.55394161)(13.76752574,1476.74278693)(13.91637368,1476.97907761)
\curveto(14.06521295,1477.21536536)(14.18986946,1477.53444883)(14.2903436,1477.93632898)
\curveto(14.39080833,1478.33820428)(14.44104304,1478.74752419)(14.4410479,1479.16428992)
\curveto(14.44104304,1479.20893935)(14.4391825,1479.27591897)(14.43546626,1479.36522898)
\curveto(14.23452254,1479.04521139)(13.96009215,1478.78566536)(13.61217426,1478.58659011)
\curveto(13.26424832,1478.38750872)(12.88748795,1478.28796957)(12.48189203,1478.28797234)
\curveto(11.80465075,1478.28796957)(11.23160288,1478.53356151)(10.76274672,1479.0247489)
\curveto(10.29388819,1479.51592928)(10.05945952,1480.16339894)(10.05946,1480.96715985)
\curveto(10.05945952,1481.79695747)(10.30412119,1482.46489313)(10.79344574,1482.97096884)
\curveto(11.28276787,1483.47702962)(11.89581746,1483.73006375)(12.63259633,1483.73007196)
\curveto(13.16470916,1483.73006375)(13.65124168,1483.58680178)(14.09219536,1483.30028563)
\curveto(14.53314002,1483.01375391)(14.86803812,1482.60536428)(15.09689067,1482.07511551)
\curveto(15.3257322,1481.54485362)(15.44015572,1480.7773788)(15.44016157,1479.77268875)
\curveto(15.44015572,1478.72705819)(15.32666247,1477.89446429)(15.09968149,1477.27490456)
\curveto(14.87268949,1476.65534131)(14.53500056,1476.18369315)(14.08661372,1475.85995866)
\curveto(13.63821787,1475.53622348)(13.1126139,1475.37435607)(12.50980024,1475.37435593)
\curveto(11.86976983,1475.37435607)(11.34695668,1475.55203812)(10.94135922,1475.90740261)
\curveto(10.53575905,1476.26276631)(10.29202765,1476.76232265)(10.21016429,1477.40607312)
\closepath
\moveto(14.32383344,1481.01739461)
\curveto(14.32382871,1481.59415806)(14.17033374,1482.05185214)(13.86334809,1482.39047821)
\curveto(13.55635389,1482.72909053)(13.1870357,1482.89840012)(12.75539242,1482.89840751)
\curveto(12.30885845,1482.89840012)(11.92000454,1482.71606671)(11.58882953,1482.35140672)
\curveto(11.25765051,1481.98673306)(11.09206201,1481.51415463)(11.09206351,1480.93367)
\curveto(11.09206201,1480.41271198)(11.24927806,1479.98943799)(11.56371215,1479.66384676)
\curveto(11.87814228,1479.33824723)(12.26606591,1479.17544954)(12.72748422,1479.1754532)
\curveto(13.19261733,1479.17544954)(13.57495934,1479.33824723)(13.87451137,1479.66384676)
\curveto(14.17405483,1479.98943799)(14.32382871,1480.44062015)(14.32383344,1481.01739461)
\closepath
}
}
{
\newrgbcolor{curcolor}{0 0 0}
\pscustom[linestyle=none,fillstyle=solid,fillcolor=curcolor]
{
\newpath
\moveto(16.48951017,1482.62490712)
\lineto(16.48951017,1483.59053095)
\lineto(21.78648714,1483.59053095)
\lineto(21.78648714,1482.80910126)
\curveto(21.26552869,1482.25465155)(20.74922745,1481.51787572)(20.23758186,1480.59877156)
\curveto(19.72592769,1479.67965724)(19.33056187,1478.73450037)(19.05148323,1477.76329812)
\curveto(18.85054126,1477.0786153)(18.72216365,1476.32881566)(18.66635002,1475.51389694)
\lineto(17.6337465,1475.51389694)
\curveto(17.64490809,1476.15764552)(17.77142515,1476.93535334)(18.01329807,1477.84702273)
\curveto(18.25516685,1478.75868746)(18.6021585,1479.63779498)(19.05427405,1480.48434793)
\curveto(19.50638338,1481.33089094)(19.98733427,1482.04440996)(20.49712815,1482.62490712)
\closepath
}
}
{
\newrgbcolor{curcolor}{0 0 0}
\pscustom[linestyle=none,fillstyle=solid,fillcolor=curcolor]
{
\newpath
\moveto(26.00620667,1475.51389694)
\lineto(26.00620667,1477.47305281)
\lineto(22.45628322,1477.47305281)
\lineto(22.45628322,1478.39402351)
\lineto(26.19040081,1483.69658212)
\lineto(27.01090198,1483.69658212)
\lineto(27.01090198,1478.39402351)
\lineto(28.11606683,1478.39402351)
\lineto(28.11606683,1477.47305281)
\lineto(27.01090198,1477.47305281)
\lineto(27.01090198,1475.51389694)
\closepath
\moveto(26.00620667,1478.39402351)
\lineto(26.00620667,1482.08348797)
\lineto(23.44423361,1478.39402351)
\closepath
}
}
{
\newrgbcolor{curcolor}{0 0 0}
\pscustom[linestyle=none,fillstyle=solid,fillcolor=curcolor]
{
\newpath
\moveto(7.48074186,1431.8468895)
\lineto(6.47604655,1431.8468895)
\lineto(6.47604655,1438.24903131)
\curveto(6.23417244,1438.01831732)(5.91694951,1437.78760974)(5.52437682,1437.55690787)
\curveto(5.13179952,1437.32619458)(4.77922624,1437.15316389)(4.46665592,1437.03781529)
\lineto(4.46665592,1438.00902076)
\curveto(5.02853927,1438.27321199)(5.51972315,1438.59322573)(5.94020905,1438.96906295)
\curveto(6.3606895,1439.34488592)(6.6583767,1439.70955275)(6.83327155,1440.06306452)
\lineto(7.48074186,1440.06306452)
\closepath
}
}
{
\newrgbcolor{curcolor}{0 0 0}
\pscustom[linestyle=none,fillstyle=solid,fillcolor=curcolor]
{
\newpath
\moveto(10.21016429,1433.73906567)
\lineto(11.17578812,1433.82837192)
\curveto(11.25765051,1433.37439696)(11.4139363,1433.04508049)(11.64464594,1432.84042153)
\curveto(11.87535146,1432.63576058)(12.17117812,1432.53343061)(12.5321268,1432.53343129)
\curveto(12.84097433,1432.53343061)(13.11168363,1432.60413132)(13.34425551,1432.74553364)
\curveto(13.57681988,1432.88693416)(13.76752574,1433.07577948)(13.91637368,1433.31207016)
\curveto(14.06521295,1433.54835792)(14.18986946,1433.86744139)(14.2903436,1434.26932153)
\curveto(14.39080833,1434.67119683)(14.44104304,1435.08051674)(14.4410479,1435.49728247)
\curveto(14.44104304,1435.5419319)(14.4391825,1435.60891152)(14.43546626,1435.69822154)
\curveto(14.23452254,1435.37820394)(13.96009215,1435.11865791)(13.61217426,1434.91958267)
\curveto(13.26424832,1434.72050128)(12.88748795,1434.62096212)(12.48189203,1434.62096489)
\curveto(11.80465075,1434.62096212)(11.23160288,1434.86655406)(10.76274672,1435.35774146)
\curveto(10.29388819,1435.84892183)(10.05945952,1436.4963915)(10.05946,1437.3001524)
\curveto(10.05945952,1438.12995003)(10.30412119,1438.79788569)(10.79344574,1439.30396139)
\curveto(11.28276787,1439.81002218)(11.89581746,1440.0630563)(12.63259633,1440.06306452)
\curveto(13.16470916,1440.0630563)(13.65124168,1439.91979433)(14.09219536,1439.63327819)
\curveto(14.53314002,1439.34674647)(14.86803812,1438.93835684)(15.09689067,1438.40810807)
\curveto(15.3257322,1437.87784617)(15.44015572,1437.11037135)(15.44016157,1436.1056813)
\curveto(15.44015572,1435.06005074)(15.32666247,1434.22745685)(15.09968149,1433.60789712)
\curveto(14.87268949,1432.98833387)(14.53500056,1432.5166857)(14.08661372,1432.19295122)
\curveto(13.63821787,1431.86921604)(13.1126139,1431.70734862)(12.50980024,1431.70734848)
\curveto(11.86976983,1431.70734862)(11.34695668,1431.88503067)(10.94135922,1432.24039516)
\curveto(10.53575905,1432.59575887)(10.29202765,1433.0953152)(10.21016429,1433.73906567)
\closepath
\moveto(14.32383344,1437.35038717)
\curveto(14.32382871,1437.92715062)(14.17033374,1438.38484469)(13.86334809,1438.72347076)
\curveto(13.55635389,1439.06208308)(13.1870357,1439.23139268)(12.75539242,1439.23140006)
\curveto(12.30885845,1439.23139268)(11.92000454,1439.04905926)(11.58882953,1438.68439928)
\curveto(11.25765051,1438.31972562)(11.09206201,1437.84714718)(11.09206351,1437.26666256)
\curveto(11.09206201,1436.74570453)(11.24927806,1436.32243054)(11.56371215,1435.99683931)
\curveto(11.87814228,1435.67123978)(12.26606591,1435.50844209)(12.72748422,1435.50844575)
\curveto(13.19261733,1435.50844209)(13.57495934,1435.67123978)(13.87451137,1435.99683931)
\curveto(14.17405483,1436.32243054)(14.32382871,1436.77361271)(14.32383344,1437.35038717)
\closepath
}
}
{
\newrgbcolor{curcolor}{0 0 0}
\pscustom[linestyle=none,fillstyle=solid,fillcolor=curcolor]
{
\newpath
\moveto(16.48951017,1438.95789967)
\lineto(16.48951017,1439.9235235)
\lineto(21.78648714,1439.9235235)
\lineto(21.78648714,1439.14209381)
\curveto(21.26552869,1438.5876441)(20.74922745,1437.85086827)(20.23758186,1436.93176412)
\curveto(19.72592769,1436.01264979)(19.33056187,1435.06749292)(19.05148323,1434.09629067)
\curveto(18.85054126,1433.41160786)(18.72216365,1432.66180821)(18.66635002,1431.8468895)
\lineto(17.6337465,1431.8468895)
\curveto(17.64490809,1432.49063807)(17.77142515,1433.26834589)(18.01329807,1434.18001528)
\curveto(18.25516685,1435.09168001)(18.6021585,1435.97078753)(19.05427405,1436.81734048)
\curveto(19.50638338,1437.6638835)(19.98733427,1438.37740251)(20.49712815,1438.95789967)
\closepath
}
}
{
\newrgbcolor{curcolor}{0 0 0}
\pscustom[linestyle=none,fillstyle=solid,fillcolor=curcolor]
{
\newpath
\moveto(22.78560002,1433.9902395)
\lineto(23.8405301,1434.07954575)
\curveto(23.91867146,1433.56603309)(24.1000746,1433.17997)(24.38474006,1432.92135532)
\curveto(24.66940138,1432.66273849)(25.01267193,1432.53343061)(25.41455276,1432.53343129)
\curveto(25.89829136,1432.53343061)(26.30761127,1432.71576402)(26.6425137,1433.08043208)
\curveto(26.97740747,1433.44509767)(27.14485652,1433.92883937)(27.14486136,1434.53165864)
\curveto(27.14485652,1435.10470382)(26.98391938,1435.55681626)(26.66204944,1435.88799732)
\curveto(26.3401708,1436.21917029)(25.91875736,1436.3847588)(25.39780784,1436.38476334)
\curveto(25.07406992,1436.3847588)(24.78196435,1436.31126727)(24.52149026,1436.16428853)
\curveto(24.26101174,1436.01730116)(24.05635179,1435.82659529)(23.90750979,1435.59217036)
\lineto(22.96421252,1435.71496646)
\lineto(23.75680549,1439.91794186)
\lineto(27.82582152,1439.91794186)
\lineto(27.82582152,1438.95789967)
\lineto(24.56056174,1438.95789967)
\lineto(24.11961213,1436.75873326)
\curveto(24.61079421,1437.10106863)(25.12616518,1437.27223877)(25.66572659,1437.2722442)
\curveto(26.38017252,1437.27223877)(26.98298911,1437.02478628)(27.47417815,1436.52988599)
\curveto(27.96535688,1436.03497633)(28.21094882,1435.39866994)(28.21095472,1434.62096489)
\curveto(28.21094882,1433.8804652)(27.9951256,1433.24043772)(27.56348441,1432.70088051)
\curveto(27.03880546,1432.03852563)(26.32249563,1431.70734862)(25.41455276,1431.70734848)
\curveto(24.67033165,1431.70734862)(24.0628637,1431.91572966)(23.59214709,1432.33249223)
\curveto(23.12142792,1432.74925383)(22.85257916,1433.3018357)(22.78560002,1433.9902395)
\closepath
}
}
{
\newrgbcolor{curcolor}{0 0 0}
\pscustom[linestyle=none,fillstyle=solid,fillcolor=curcolor]
{
\newpath
\moveto(7.48074186,1388.17989731)
\lineto(6.47604655,1388.17989731)
\lineto(6.47604655,1394.58203912)
\curveto(6.23417244,1394.35132514)(5.91694951,1394.12061755)(5.52437682,1393.88991568)
\curveto(5.13179952,1393.65920239)(4.77922624,1393.4861717)(4.46665592,1393.3708231)
\lineto(4.46665592,1394.34202857)
\curveto(5.02853927,1394.6062198)(5.51972315,1394.92623355)(5.94020905,1395.30207076)
\curveto(6.3606895,1395.67789373)(6.6583767,1396.04256056)(6.83327155,1396.39607233)
\lineto(7.48074186,1396.39607233)
\closepath
}
}
{
\newrgbcolor{curcolor}{0 0 0}
\pscustom[linestyle=none,fillstyle=solid,fillcolor=curcolor]
{
\newpath
\moveto(10.21016429,1390.07207349)
\lineto(11.17578812,1390.16137974)
\curveto(11.25765051,1389.70740477)(11.4139363,1389.3780883)(11.64464594,1389.17342934)
\curveto(11.87535146,1388.9687684)(12.17117812,1388.86643842)(12.5321268,1388.86643911)
\curveto(12.84097433,1388.86643842)(13.11168363,1388.93713913)(13.34425551,1389.07854145)
\curveto(13.57681988,1389.21994197)(13.76752574,1389.40878729)(13.91637368,1389.64507798)
\curveto(14.06521295,1389.88136573)(14.18986946,1390.2004492)(14.2903436,1390.60232935)
\curveto(14.39080833,1391.00420465)(14.44104304,1391.41352455)(14.4410479,1391.83029029)
\curveto(14.44104304,1391.87493972)(14.4391825,1391.94191934)(14.43546626,1392.03122935)
\curveto(14.23452254,1391.71121175)(13.96009215,1391.45166572)(13.61217426,1391.25259048)
\curveto(13.26424832,1391.05350909)(12.88748795,1390.95396993)(12.48189203,1390.95397271)
\curveto(11.80465075,1390.95396993)(11.23160288,1391.19956187)(10.76274672,1391.69074927)
\curveto(10.29388819,1392.18192964)(10.05945952,1392.82939931)(10.05946,1393.63316021)
\curveto(10.05945952,1394.46295784)(10.30412119,1395.1308935)(10.79344574,1395.6369692)
\curveto(11.28276787,1396.14302999)(11.89581746,1396.39606411)(12.63259633,1396.39607233)
\curveto(13.16470916,1396.39606411)(13.65124168,1396.25280215)(14.09219536,1395.966286)
\curveto(14.53314002,1395.67975428)(14.86803812,1395.27136465)(15.09689067,1394.74111588)
\curveto(15.3257322,1394.21085399)(15.44015572,1393.44337917)(15.44016157,1392.43868912)
\curveto(15.44015572,1391.39305856)(15.32666247,1390.56046466)(15.09968149,1389.94090493)
\curveto(14.87268949,1389.32134168)(14.53500056,1388.84969352)(14.08661372,1388.52595903)
\curveto(13.63821787,1388.20222385)(13.1126139,1388.04035643)(12.50980024,1388.04035629)
\curveto(11.86976983,1388.04035643)(11.34695668,1388.21803848)(10.94135922,1388.57340297)
\curveto(10.53575905,1388.92876668)(10.29202765,1389.42832302)(10.21016429,1390.07207349)
\closepath
\moveto(14.32383344,1393.68339498)
\curveto(14.32382871,1394.26015843)(14.17033374,1394.71785251)(13.86334809,1395.05647858)
\curveto(13.55635389,1395.39509089)(13.1870357,1395.56440049)(12.75539242,1395.56440787)
\curveto(12.30885845,1395.56440049)(11.92000454,1395.38206708)(11.58882953,1395.01740709)
\curveto(11.25765051,1394.65273343)(11.09206201,1394.18015499)(11.09206351,1393.59967037)
\curveto(11.09206201,1393.07871234)(11.24927806,1392.65543835)(11.56371215,1392.32984712)
\curveto(11.87814228,1392.0042476)(12.26606591,1391.84144991)(12.72748422,1391.84145357)
\curveto(13.19261733,1391.84144991)(13.57495934,1392.0042476)(13.87451137,1392.32984712)
\curveto(14.17405483,1392.65543835)(14.32382871,1393.10662052)(14.32383344,1393.68339498)
\closepath
}
}
{
\newrgbcolor{curcolor}{0 0 0}
\pscustom[linestyle=none,fillstyle=solid,fillcolor=curcolor]
{
\newpath
\moveto(16.48951017,1395.29090748)
\lineto(16.48951017,1396.25653131)
\lineto(21.78648714,1396.25653131)
\lineto(21.78648714,1395.47510162)
\curveto(21.26552869,1394.92065191)(20.74922745,1394.18387608)(20.23758186,1393.26477193)
\curveto(19.72592769,1392.34565761)(19.33056187,1391.40050074)(19.05148323,1390.42929849)
\curveto(18.85054126,1389.74461567)(18.72216365,1388.99481603)(18.66635002,1388.17989731)
\lineto(17.6337465,1388.17989731)
\curveto(17.64490809,1388.82364589)(17.77142515,1389.6013537)(18.01329807,1390.5130231)
\curveto(18.25516685,1391.42468782)(18.6021585,1392.30379534)(19.05427405,1393.1503483)
\curveto(19.50638338,1393.99689131)(19.98733427,1394.71041033)(20.49712815,1395.29090748)
\closepath
}
}
{
\newrgbcolor{curcolor}{0 0 0}
\pscustom[linestyle=none,fillstyle=solid,fillcolor=curcolor]
{
\newpath
\moveto(27.99885237,1394.3587735)
\lineto(26.9997387,1394.28063053)
\curveto(26.91042785,1394.67505997)(26.78391079,1394.9615839)(26.62018714,1395.14020318)
\curveto(26.34854326,1395.42672016)(26.01364515,1395.56998212)(25.61549182,1395.56998951)
\curveto(25.29547477,1395.56998212)(25.01453248,1395.48067596)(24.77266409,1395.30207076)
\curveto(24.45636897,1395.07135606)(24.20705594,1394.73459741)(24.02472424,1394.29179381)
\curveto(23.84238911,1393.84897798)(23.74750132,1393.21825322)(23.74006057,1392.39961763)
\curveto(23.98192999,1392.76800133)(24.27775665,1393.04150144)(24.62754143,1393.22011881)
\curveto(24.97732158,1393.39872609)(25.34384895,1393.48803225)(25.72712464,1393.48803756)
\curveto(26.39691743,1393.48803225)(26.96717447,1393.24151003)(27.43789749,1392.74847017)
\curveto(27.90861025,1392.25542117)(28.1439692,1391.6181845)(28.14397503,1390.83675825)
\curveto(28.1439692,1390.32324517)(28.03326677,1389.84601537)(27.81186741,1389.40506743)
\curveto(27.59045706,1388.96411703)(27.28625795,1388.62642811)(26.89926917,1388.39199965)
\curveto(26.51227122,1388.15757077)(26.07318259,1388.04035643)(25.58200198,1388.04035629)
\curveto(24.74475345,1388.04035643)(24.06193343,1388.34827663)(23.53353986,1388.96411782)
\curveto(23.00514386,1389.57995744)(22.74094646,1390.59488474)(22.74094689,1392.00890279)
\curveto(22.74094646,1393.59036222)(23.03305203,1394.74017905)(23.61726447,1395.4583567)
\curveto(24.1270525,1396.08349255)(24.81359361,1396.39606411)(25.67688987,1396.39607233)
\curveto(26.32063508,1396.39606411)(26.84809959,1396.21559125)(27.25928499,1395.85465319)
\curveto(27.67046049,1395.49369978)(27.91698271,1394.99507371)(27.99885237,1394.3587735)
\closepath
\moveto(23.89634651,1390.83117661)
\curveto(23.89634492,1390.48511259)(23.96983645,1390.15393557)(24.11682131,1389.83764458)
\curveto(24.26380256,1389.52135027)(24.46939279,1389.28040969)(24.7335926,1389.11482212)
\curveto(24.99778757,1388.94923267)(25.27500878,1388.86643842)(25.56525706,1388.86643911)
\curveto(25.98945807,1388.86643842)(26.35412489,1389.03760856)(26.65925862,1389.37995005)
\curveto(26.96438366,1389.72228913)(27.11694835,1390.18742538)(27.11695315,1390.77536021)
\curveto(27.11694835,1391.3409633)(26.9662442,1391.78656383)(26.66484026,1392.11216314)
\curveto(26.36342762,1392.43775458)(25.98387643,1392.60055227)(25.52618557,1392.60055669)
\curveto(25.07220937,1392.60055227)(24.68707655,1392.43775458)(24.37078596,1392.11216314)
\curveto(24.05449125,1391.78656383)(23.89634492,1391.35956875)(23.89634651,1390.83117661)
\closepath
}
}
{
\newrgbcolor{curcolor}{0 0 0}
\pscustom[linestyle=none,fillstyle=solid,fillcolor=curcolor]
{
\newpath
\moveto(7.48074186,1344.51290512)
\lineto(6.47604655,1344.51290512)
\lineto(6.47604655,1350.91504693)
\curveto(6.23417244,1350.68433295)(5.91694951,1350.45362537)(5.52437682,1350.22292349)
\curveto(5.13179952,1349.9922102)(4.77922624,1349.81917952)(4.46665592,1349.70383091)
\lineto(4.46665592,1350.67503639)
\curveto(5.02853927,1350.93922762)(5.51972315,1351.25924136)(5.94020905,1351.63507858)
\curveto(6.3606895,1352.01090155)(6.6583767,1352.37556837)(6.83327155,1352.72908014)
\lineto(7.48074186,1352.72908014)
\closepath
}
}
{
\newrgbcolor{curcolor}{0 0 0}
\pscustom[linestyle=none,fillstyle=solid,fillcolor=curcolor]
{
\newpath
\moveto(10.21016429,1346.4050813)
\lineto(11.17578812,1346.49438755)
\curveto(11.25765051,1346.04041258)(11.4139363,1345.71109611)(11.64464594,1345.50643715)
\curveto(11.87535146,1345.30177621)(12.17117812,1345.19944623)(12.5321268,1345.19944692)
\curveto(12.84097433,1345.19944623)(13.11168363,1345.27014694)(13.34425551,1345.41154926)
\curveto(13.57681988,1345.55294979)(13.76752574,1345.74179511)(13.91637368,1345.97808579)
\curveto(14.06521295,1346.21437354)(14.18986946,1346.53345701)(14.2903436,1346.93533716)
\curveto(14.39080833,1347.33721246)(14.44104304,1347.74653236)(14.4410479,1348.1632981)
\curveto(14.44104304,1348.20794753)(14.4391825,1348.27492715)(14.43546626,1348.36423716)
\curveto(14.23452254,1348.04421957)(13.96009215,1347.78467354)(13.61217426,1347.58559829)
\curveto(13.26424832,1347.3865169)(12.88748795,1347.28697774)(12.48189203,1347.28698052)
\curveto(11.80465075,1347.28697774)(11.23160288,1347.53256969)(10.76274672,1348.02375708)
\curveto(10.29388819,1348.51493746)(10.05945952,1349.16240712)(10.05946,1349.96616802)
\curveto(10.05945952,1350.79596565)(10.30412119,1351.46390131)(10.79344574,1351.96997701)
\curveto(11.28276787,1352.4760378)(11.89581746,1352.72907192)(12.63259633,1352.72908014)
\curveto(13.16470916,1352.72907192)(13.65124168,1352.58580996)(14.09219536,1352.29929381)
\curveto(14.53314002,1352.01276209)(14.86803812,1351.60437246)(15.09689067,1351.07412369)
\curveto(15.3257322,1350.5438618)(15.44015572,1349.77638698)(15.44016157,1348.77169693)
\curveto(15.44015572,1347.72606637)(15.32666247,1346.89347247)(15.09968149,1346.27391274)
\curveto(14.87268949,1345.65434949)(14.53500056,1345.18270133)(14.08661372,1344.85896684)
\curveto(13.63821787,1344.53523166)(13.1126139,1344.37336424)(12.50980024,1344.37336411)
\curveto(11.86976983,1344.37336424)(11.34695668,1344.55104629)(10.94135922,1344.90641079)
\curveto(10.53575905,1345.26177449)(10.29202765,1345.76133083)(10.21016429,1346.4050813)
\closepath
\moveto(14.32383344,1350.01640279)
\curveto(14.32382871,1350.59316624)(14.17033374,1351.05086032)(13.86334809,1351.38948639)
\curveto(13.55635389,1351.7280987)(13.1870357,1351.8974083)(12.75539242,1351.89741569)
\curveto(12.30885845,1351.8974083)(11.92000454,1351.71507489)(11.58882953,1351.3504149)
\curveto(11.25765051,1350.98574124)(11.09206201,1350.51316281)(11.09206351,1349.93267818)
\curveto(11.09206201,1349.41172016)(11.24927806,1348.98844616)(11.56371215,1348.66285494)
\curveto(11.87814228,1348.33725541)(12.26606591,1348.17445772)(12.72748422,1348.17446138)
\curveto(13.19261733,1348.17445772)(13.57495934,1348.33725541)(13.87451137,1348.66285494)
\curveto(14.17405483,1348.98844616)(14.32382871,1349.43962833)(14.32383344,1350.01640279)
\closepath
}
}
{
\newrgbcolor{curcolor}{0 0 0}
\pscustom[linestyle=none,fillstyle=solid,fillcolor=curcolor]
{
\newpath
\moveto(16.48951017,1351.62391529)
\lineto(16.48951017,1352.58953912)
\lineto(21.78648714,1352.58953912)
\lineto(21.78648714,1351.80810944)
\curveto(21.26552869,1351.25365972)(20.74922745,1350.5168839)(20.23758186,1349.59777974)
\curveto(19.72592769,1348.67866542)(19.33056187,1347.73350855)(19.05148323,1346.7623063)
\curveto(18.85054126,1346.07762348)(18.72216365,1345.32782384)(18.66635002,1344.51290512)
\lineto(17.6337465,1344.51290512)
\curveto(17.64490809,1345.1566537)(17.77142515,1345.93436152)(18.01329807,1346.84603091)
\curveto(18.25516685,1347.75769563)(18.6021585,1348.63680316)(19.05427405,1349.48335611)
\curveto(19.50638338,1350.32989912)(19.98733427,1351.04341814)(20.49712815,1351.62391529)
\closepath
}
}
{
\newrgbcolor{curcolor}{0 0 0}
\pscustom[linestyle=none,fillstyle=solid,fillcolor=curcolor]
{
\newpath
\moveto(22.85257971,1351.62391529)
\lineto(22.85257971,1352.58953912)
\lineto(28.14955667,1352.58953912)
\lineto(28.14955667,1351.80810944)
\curveto(27.62859823,1351.25365972)(27.11229699,1350.5168839)(26.6006514,1349.59777974)
\curveto(26.08899723,1348.67866542)(25.69363141,1347.73350855)(25.41455276,1346.7623063)
\curveto(25.21361079,1346.07762348)(25.08523319,1345.32782384)(25.02941956,1344.51290512)
\lineto(23.99681604,1344.51290512)
\curveto(24.00797762,1345.1566537)(24.13449468,1345.93436152)(24.3763676,1346.84603091)
\curveto(24.61823639,1347.75769563)(24.96522803,1348.63680316)(25.41734358,1349.48335611)
\curveto(25.86945291,1350.32989912)(26.3504038,1351.04341814)(26.86019768,1351.62391529)
\closepath
}
}
{
\newrgbcolor{curcolor}{0 0 0}
\pscustom[linestyle=none,fillstyle=solid,fillcolor=curcolor]
{
\newpath
\moveto(7.48074186,1300.84591293)
\lineto(6.47604655,1300.84591293)
\lineto(6.47604655,1307.24805475)
\curveto(6.23417244,1307.01734076)(5.91694951,1306.78663318)(5.52437682,1306.55593131)
\curveto(5.13179952,1306.32521801)(4.77922624,1306.15218733)(4.46665592,1306.03683873)
\lineto(4.46665592,1307.0080442)
\curveto(5.02853927,1307.27223543)(5.51972315,1307.59224917)(5.94020905,1307.96808639)
\curveto(6.3606895,1308.34390936)(6.6583767,1308.70857618)(6.83327155,1309.06208795)
\lineto(7.48074186,1309.06208795)
\closepath
}
}
{
\newrgbcolor{curcolor}{0 0 0}
\pscustom[linestyle=none,fillstyle=solid,fillcolor=curcolor]
{
\newpath
\moveto(10.21016429,1302.73808911)
\lineto(11.17578812,1302.82739536)
\curveto(11.25765051,1302.37342039)(11.4139363,1302.04410393)(11.64464594,1301.83944497)
\curveto(11.87535146,1301.63478402)(12.17117812,1301.53245405)(12.5321268,1301.53245473)
\curveto(12.84097433,1301.53245405)(13.11168363,1301.60315476)(13.34425551,1301.74455708)
\curveto(13.57681988,1301.8859576)(13.76752574,1302.07480292)(13.91637368,1302.3110936)
\curveto(14.06521295,1302.54738135)(14.18986946,1302.86646482)(14.2903436,1303.26834497)
\curveto(14.39080833,1303.67022027)(14.44104304,1304.07954018)(14.4410479,1304.49630591)
\curveto(14.44104304,1304.54095534)(14.4391825,1304.60793496)(14.43546626,1304.69724497)
\curveto(14.23452254,1304.37722738)(13.96009215,1304.11768135)(13.61217426,1303.91860611)
\curveto(13.26424832,1303.71952472)(12.88748795,1303.61998556)(12.48189203,1303.61998833)
\curveto(11.80465075,1303.61998556)(11.23160288,1303.8655775)(10.76274672,1304.3567649)
\curveto(10.29388819,1304.84794527)(10.05945952,1305.49541494)(10.05946,1306.29917584)
\curveto(10.05945952,1307.12897346)(10.30412119,1307.79690912)(10.79344574,1308.30298483)
\curveto(11.28276787,1308.80904561)(11.89581746,1309.06207974)(12.63259633,1309.06208795)
\curveto(13.16470916,1309.06207974)(13.65124168,1308.91881777)(14.09219536,1308.63230162)
\curveto(14.53314002,1308.3457699)(14.86803812,1307.93738027)(15.09689067,1307.4071315)
\curveto(15.3257322,1306.87686961)(15.44015572,1306.10939479)(15.44016157,1305.10470474)
\curveto(15.44015572,1304.05907418)(15.32666247,1303.22648029)(15.09968149,1302.60692056)
\curveto(14.87268949,1301.9873573)(14.53500056,1301.51570914)(14.08661372,1301.19197465)
\curveto(13.63821787,1300.86823947)(13.1126139,1300.70637206)(12.50980024,1300.70637192)
\curveto(11.86976983,1300.70637206)(11.34695668,1300.88405411)(10.94135922,1301.2394186)
\curveto(10.53575905,1301.5947823)(10.29202765,1302.09433864)(10.21016429,1302.73808911)
\closepath
\moveto(14.32383344,1306.3494106)
\curveto(14.32382871,1306.92617406)(14.17033374,1307.38386813)(13.86334809,1307.7224942)
\curveto(13.55635389,1308.06110652)(13.1870357,1308.23041611)(12.75539242,1308.2304235)
\curveto(12.30885845,1308.23041611)(11.92000454,1308.0480827)(11.58882953,1307.68342272)
\curveto(11.25765051,1307.31874905)(11.09206201,1306.84617062)(11.09206351,1306.26568599)
\curveto(11.09206201,1305.74472797)(11.24927806,1305.32145398)(11.56371215,1304.99586275)
\curveto(11.87814228,1304.67026322)(12.26606591,1304.50746553)(12.72748422,1304.50746919)
\curveto(13.19261733,1304.50746553)(13.57495934,1304.67026322)(13.87451137,1304.99586275)
\curveto(14.17405483,1305.32145398)(14.32382871,1305.77263614)(14.32383344,1306.3494106)
\closepath
}
}
{
\newrgbcolor{curcolor}{0 0 0}
\pscustom[linestyle=none,fillstyle=solid,fillcolor=curcolor]
{
\newpath
\moveto(16.48951017,1307.95692311)
\lineto(16.48951017,1308.92254694)
\lineto(21.78648714,1308.92254694)
\lineto(21.78648714,1308.14111725)
\curveto(21.26552869,1307.58666754)(20.74922745,1306.84989171)(20.23758186,1305.93078756)
\curveto(19.72592769,1305.01167323)(19.33056187,1304.06651636)(19.05148323,1303.09531411)
\curveto(18.85054126,1302.41063129)(18.72216365,1301.66083165)(18.66635002,1300.84591293)
\lineto(17.6337465,1300.84591293)
\curveto(17.64490809,1301.48966151)(17.77142515,1302.26736933)(18.01329807,1303.17903872)
\curveto(18.25516685,1304.09070345)(18.6021585,1304.96981097)(19.05427405,1305.81636392)
\curveto(19.50638338,1306.66290694)(19.98733427,1307.37642595)(20.49712815,1307.95692311)
\closepath
}
}
{
\newrgbcolor{curcolor}{0 0 0}
\pscustom[linestyle=none,fillstyle=solid,fillcolor=curcolor]
{
\newpath
\moveto(24.33171448,1305.28331724)
\curveto(23.91495037,1305.4358775)(23.6060999,1305.65356126)(23.40516213,1305.9363692)
\curveto(23.20422217,1306.21916695)(23.10375274,1306.55778614)(23.10375354,1306.95222779)
\curveto(23.10375274,1307.54759609)(23.31771542,1308.0480827)(23.74564221,1308.45368912)
\curveto(24.17356613,1308.85928033)(24.7428929,1309.06207974)(25.45362424,1309.06208795)
\curveto(26.16807039,1309.06207974)(26.7429788,1308.85462897)(27.1783512,1308.43973502)
\curveto(27.61371387,1308.02482589)(27.83139764,1307.51968792)(27.83140316,1306.92431959)
\curveto(27.83139764,1306.54476233)(27.73185848,1306.21451559)(27.53278538,1305.93357838)
\curveto(27.33370184,1305.65263099)(27.03136328,1305.4358775)(26.62576878,1305.28331724)
\curveto(27.12811162,1305.11958484)(27.51045362,1304.85538745)(27.77279593,1304.49072427)
\curveto(28.03512732,1304.1260538)(28.16629574,1303.69068627)(28.16630159,1303.18462036)
\curveto(28.16629574,1302.4850531)(27.91884325,1301.89712087)(27.42394339,1301.42082192)
\curveto(26.9290333,1300.94452182)(26.27784254,1300.70637206)(25.47036917,1300.70637192)
\curveto(24.66288947,1300.70637206)(24.01169871,1300.94545209)(23.51679494,1301.42361274)
\curveto(23.02188876,1301.90177223)(22.77443627,1302.49807691)(22.77443674,1303.21252856)
\curveto(22.77443627,1303.74464207)(22.90932579,1304.19024261)(23.17910568,1304.5493315)
\curveto(23.44888384,1304.90841298)(23.83308639,1305.15307465)(24.33171448,1305.28331724)
\closepath
\moveto(24.13077541,1306.98571764)
\curveto(24.13077359,1306.59871813)(24.25543011,1306.28242548)(24.50474534,1306.03683873)
\curveto(24.75405617,1305.79124159)(25.07779101,1305.66844562)(25.47595081,1305.66845045)
\curveto(25.86294101,1305.66844562)(26.18016393,1305.79031132)(26.42762054,1306.03404791)
\curveto(26.67506891,1306.27777412)(26.79879515,1306.57639159)(26.79879964,1306.92990123)
\curveto(26.79879515,1307.29828306)(26.67134782,1307.6080638)(26.41645725,1307.8592444)
\curveto(26.16155848,1308.11041096)(25.84433556,1308.23599775)(25.46478753,1308.23600514)
\curveto(25.0815121,1308.23599775)(24.7633589,1308.11320178)(24.51032698,1307.86761686)
\curveto(24.25729065,1307.62201789)(24.13077359,1307.32805178)(24.13077541,1306.98571764)
\closepath
\moveto(23.80704026,1303.20694692)
\curveto(23.80703876,1302.92042063)(23.87494865,1302.64319942)(24.01077014,1302.37528247)
\curveto(24.14658823,1302.10736246)(24.34845736,1301.89991169)(24.61637815,1301.75292954)
\curveto(24.88429433,1301.60594557)(25.1726788,1301.53245405)(25.48153245,1301.53245473)
\curveto(25.96154989,1301.53245405)(26.35784598,1301.68687928)(26.6704219,1301.99573091)
\curveto(26.98298911,1302.30458023)(27.13927489,1302.69715523)(27.13927972,1303.17345708)
\curveto(27.13927489,1303.65719646)(26.97833774,1304.05721364)(26.6564678,1304.37350982)
\curveto(26.33458917,1304.68979894)(25.93178117,1304.84794527)(25.4480426,1304.84794927)
\curveto(24.97546103,1304.84794527)(24.58381631,1304.69165949)(24.27310725,1304.37909146)
\curveto(23.96239427,1304.06651636)(23.80703876,1303.67580191)(23.80704026,1303.20694692)
\closepath
}
}
{
\newrgbcolor{curcolor}{0 0 0}
\pscustom[linestyle=none,fillstyle=solid,fillcolor=curcolor]
{
\newpath
\moveto(7.48074186,1257.17889023)
\lineto(6.47604655,1257.17889023)
\lineto(6.47604655,1263.58103204)
\curveto(6.23417244,1263.35031806)(5.91694951,1263.11961047)(5.52437682,1262.8889086)
\curveto(5.13179952,1262.65819531)(4.77922624,1262.48516462)(4.46665592,1262.36981602)
\lineto(4.46665592,1263.34102149)
\curveto(5.02853927,1263.60521272)(5.51972315,1263.92522647)(5.94020905,1264.30106368)
\curveto(6.3606895,1264.67688665)(6.6583767,1265.04155348)(6.83327155,1265.39506525)
\lineto(7.48074186,1265.39506525)
\closepath
}
}
{
\newrgbcolor{curcolor}{0 0 0}
\pscustom[linestyle=none,fillstyle=solid,fillcolor=curcolor]
{
\newpath
\moveto(10.21016429,1259.0710664)
\lineto(11.17578812,1259.16037266)
\curveto(11.25765051,1258.70639769)(11.4139363,1258.37708122)(11.64464594,1258.17242226)
\curveto(11.87535146,1257.96776132)(12.17117812,1257.86543134)(12.5321268,1257.86543203)
\curveto(12.84097433,1257.86543134)(13.11168363,1257.93613205)(13.34425551,1258.07753437)
\curveto(13.57681988,1258.21893489)(13.76752574,1258.40778021)(13.91637368,1258.6440709)
\curveto(14.06521295,1258.88035865)(14.18986946,1259.19944212)(14.2903436,1259.60132227)
\curveto(14.39080833,1260.00319757)(14.44104304,1260.41251747)(14.4410479,1260.82928321)
\curveto(14.44104304,1260.87393264)(14.4391825,1260.94091226)(14.43546626,1261.03022227)
\curveto(14.23452254,1260.71020467)(13.96009215,1260.45065864)(13.61217426,1260.2515834)
\curveto(13.26424832,1260.05250201)(12.88748795,1259.95296285)(12.48189203,1259.95296563)
\curveto(11.80465075,1259.95296285)(11.23160288,1260.19855479)(10.76274672,1260.68974219)
\curveto(10.29388819,1261.18092256)(10.05945952,1261.82839223)(10.05946,1262.63215313)
\curveto(10.05945952,1263.46195076)(10.30412119,1264.12988642)(10.79344574,1264.63596212)
\curveto(11.28276787,1265.14202291)(11.89581746,1265.39505703)(12.63259633,1265.39506525)
\curveto(13.16470916,1265.39505703)(13.65124168,1265.25179507)(14.09219536,1264.96527892)
\curveto(14.53314002,1264.6787472)(14.86803812,1264.27035757)(15.09689067,1263.7401088)
\curveto(15.3257322,1263.20984691)(15.44015572,1262.44237209)(15.44016157,1261.43768204)
\curveto(15.44015572,1260.39205148)(15.32666247,1259.55945758)(15.09968149,1258.93989785)
\curveto(14.87268949,1258.3203346)(14.53500056,1257.84868644)(14.08661372,1257.52495195)
\curveto(13.63821787,1257.20121677)(13.1126139,1257.03934935)(12.50980024,1257.03934921)
\curveto(11.86976983,1257.03934935)(11.34695668,1257.2170314)(10.94135922,1257.57239589)
\curveto(10.53575905,1257.9277596)(10.29202765,1258.42731594)(10.21016429,1259.0710664)
\closepath
\moveto(14.32383344,1262.6823879)
\curveto(14.32382871,1263.25915135)(14.17033374,1263.71684543)(13.86334809,1264.0554715)
\curveto(13.55635389,1264.39408381)(13.1870357,1264.56339341)(12.75539242,1264.56340079)
\curveto(12.30885845,1264.56339341)(11.92000454,1264.38106)(11.58882953,1264.01640001)
\curveto(11.25765051,1263.65172635)(11.09206201,1263.17914791)(11.09206351,1262.59866329)
\curveto(11.09206201,1262.07770526)(11.24927806,1261.65443127)(11.56371215,1261.32884004)
\curveto(11.87814228,1261.00324051)(12.26606591,1260.84044283)(12.72748422,1260.84044649)
\curveto(13.19261733,1260.84044283)(13.57495934,1261.00324051)(13.87451137,1261.32884004)
\curveto(14.17405483,1261.65443127)(14.32382871,1262.10561344)(14.32383344,1262.6823879)
\closepath
}
}
{
\newrgbcolor{curcolor}{0 0 0}
\pscustom[linestyle=none,fillstyle=solid,fillcolor=curcolor]
{
\newpath
\moveto(16.48951017,1264.2899004)
\lineto(16.48951017,1265.25552423)
\lineto(21.78648714,1265.25552423)
\lineto(21.78648714,1264.47409454)
\curveto(21.26552869,1263.91964483)(20.74922745,1263.182869)(20.23758186,1262.26376485)
\curveto(19.72592769,1261.34465053)(19.33056187,1260.39949366)(19.05148323,1259.42829141)
\curveto(18.85054126,1258.74360859)(18.72216365,1257.99380895)(18.66635002,1257.17889023)
\lineto(17.6337465,1257.17889023)
\curveto(17.64490809,1257.82263881)(17.77142515,1258.60034662)(18.01329807,1259.51201602)
\curveto(18.25516685,1260.42368074)(18.6021585,1261.30278826)(19.05427405,1262.14934122)
\curveto(19.50638338,1262.99588423)(19.98733427,1263.70940324)(20.49712815,1264.2899004)
\closepath
}
}
{
\newrgbcolor{curcolor}{0 0 0}
\pscustom[linestyle=none,fillstyle=solid,fillcolor=curcolor]
{
\newpath
\moveto(22.93630432,1259.0710664)
\lineto(23.90192815,1259.16037266)
\curveto(23.98379054,1258.70639769)(24.14007632,1258.37708122)(24.37078596,1258.17242226)
\curveto(24.60149148,1257.96776132)(24.89731814,1257.86543134)(25.25826682,1257.86543203)
\curveto(25.56711435,1257.86543134)(25.83782365,1257.93613205)(26.07039553,1258.07753437)
\curveto(26.3029599,1258.21893489)(26.49366577,1258.40778021)(26.6425137,1258.6440709)
\curveto(26.79135297,1258.88035865)(26.91600949,1259.19944212)(27.01648362,1259.60132227)
\curveto(27.11694835,1260.00319757)(27.16718306,1260.41251747)(27.16718792,1260.82928321)
\curveto(27.16718306,1260.87393264)(27.16532252,1260.94091226)(27.16160628,1261.03022227)
\curveto(26.96066257,1260.71020467)(26.68623218,1260.45065864)(26.33831429,1260.2515834)
\curveto(25.99038834,1260.05250201)(25.61362797,1259.95296285)(25.20803206,1259.95296563)
\curveto(24.53079077,1259.95296285)(23.95774291,1260.19855479)(23.48888674,1260.68974219)
\curveto(23.02002822,1261.18092256)(22.78559954,1261.82839223)(22.78560002,1262.63215313)
\curveto(22.78559954,1263.46195076)(23.03026121,1264.12988642)(23.51958576,1264.63596212)
\curveto(24.00890789,1265.14202291)(24.62195748,1265.39505703)(25.35873635,1265.39506525)
\curveto(25.89084918,1265.39505703)(26.3773817,1265.25179507)(26.81833538,1264.96527892)
\curveto(27.25928004,1264.6787472)(27.59417815,1264.27035757)(27.82303069,1263.7401088)
\curveto(28.05187222,1263.20984691)(28.16629574,1262.44237209)(28.16630159,1261.43768204)
\curveto(28.16629574,1260.39205148)(28.05280249,1259.55945758)(27.82582152,1258.93989785)
\curveto(27.59882951,1258.3203346)(27.26114059,1257.84868644)(26.81275374,1257.52495195)
\curveto(26.36435789,1257.20121677)(25.83875392,1257.03934935)(25.23594026,1257.03934921)
\curveto(24.59590985,1257.03934935)(24.0730967,1257.2170314)(23.66749924,1257.57239589)
\curveto(23.26189907,1257.9277596)(23.01816767,1258.42731594)(22.93630432,1259.0710664)
\closepath
\moveto(27.04997347,1262.6823879)
\curveto(27.04996873,1263.25915135)(26.89647376,1263.71684543)(26.58948811,1264.0554715)
\curveto(26.28249391,1264.39408381)(25.91317572,1264.56339341)(25.48153245,1264.56340079)
\curveto(25.03499847,1264.56339341)(24.64614456,1264.38106)(24.31496955,1264.01640001)
\curveto(23.98379054,1263.65172635)(23.81820203,1263.17914791)(23.81820354,1262.59866329)
\curveto(23.81820203,1262.07770526)(23.97541808,1261.65443127)(24.28985217,1261.32884004)
\curveto(24.6042823,1261.00324051)(24.99220594,1260.84044283)(25.45362424,1260.84044649)
\curveto(25.91875736,1260.84044283)(26.30109936,1261.00324051)(26.6006514,1261.32884004)
\curveto(26.90019485,1261.65443127)(27.04996873,1262.10561344)(27.04997347,1262.6823879)
\closepath
}
}
{
\newrgbcolor{curcolor}{0 0 0}
\pscustom[linestyle=none,fillstyle=solid,fillcolor=curcolor]
{
\newpath
\moveto(7.48074186,1213.51189804)
\lineto(6.47604655,1213.51189804)
\lineto(6.47604655,1219.91403985)
\curveto(6.23417244,1219.68332587)(5.91694951,1219.45261829)(5.52437682,1219.22191641)
\curveto(5.13179952,1218.99120312)(4.77922624,1218.81817244)(4.46665592,1218.70282383)
\lineto(4.46665592,1219.67402931)
\curveto(5.02853927,1219.93822054)(5.51972315,1220.25823428)(5.94020905,1220.6340715)
\curveto(6.3606895,1221.00989447)(6.6583767,1221.37456129)(6.83327155,1221.72807306)
\lineto(7.48074186,1221.72807306)
\closepath
}
}
{
\newrgbcolor{curcolor}{0 0 0}
\pscustom[linestyle=none,fillstyle=solid,fillcolor=curcolor]
{
\newpath
\moveto(10.21016429,1215.40407422)
\lineto(11.17578812,1215.49338047)
\curveto(11.25765051,1215.0394055)(11.4139363,1214.71008903)(11.64464594,1214.50543007)
\curveto(11.87535146,1214.30076913)(12.17117812,1214.19843915)(12.5321268,1214.19843984)
\curveto(12.84097433,1214.19843915)(13.11168363,1214.26913986)(13.34425551,1214.41054218)
\curveto(13.57681988,1214.55194271)(13.76752574,1214.74078803)(13.91637368,1214.97707871)
\curveto(14.06521295,1215.21336646)(14.18986946,1215.53244993)(14.2903436,1215.93433008)
\curveto(14.39080833,1216.33620538)(14.44104304,1216.74552528)(14.4410479,1217.16229102)
\curveto(14.44104304,1217.20694045)(14.4391825,1217.27392007)(14.43546626,1217.36323008)
\curveto(14.23452254,1217.04321249)(13.96009215,1216.78366646)(13.61217426,1216.58459121)
\curveto(13.26424832,1216.38550982)(12.88748795,1216.28597066)(12.48189203,1216.28597344)
\curveto(11.80465075,1216.28597066)(11.23160288,1216.53156261)(10.76274672,1217.02275)
\curveto(10.29388819,1217.51393038)(10.05945952,1218.16140004)(10.05946,1218.96516094)
\curveto(10.05945952,1219.79495857)(10.30412119,1220.46289423)(10.79344574,1220.96896993)
\curveto(11.28276787,1221.47503072)(11.89581746,1221.72806484)(12.63259633,1221.72807306)
\curveto(13.16470916,1221.72806484)(13.65124168,1221.58480288)(14.09219536,1221.29828673)
\curveto(14.53314002,1221.01175501)(14.86803812,1220.60336538)(15.09689067,1220.07311661)
\curveto(15.3257322,1219.54285472)(15.44015572,1218.7753799)(15.44016157,1217.77068985)
\curveto(15.44015572,1216.72505929)(15.32666247,1215.89246539)(15.09968149,1215.27290566)
\curveto(14.87268949,1214.65334241)(14.53500056,1214.18169425)(14.08661372,1213.85795976)
\curveto(13.63821787,1213.53422458)(13.1126139,1213.37235716)(12.50980024,1213.37235703)
\curveto(11.86976983,1213.37235716)(11.34695668,1213.55003921)(10.94135922,1213.90540371)
\curveto(10.53575905,1214.26076741)(10.29202765,1214.76032375)(10.21016429,1215.40407422)
\closepath
\moveto(14.32383344,1219.01539571)
\curveto(14.32382871,1219.59215916)(14.17033374,1220.04985324)(13.86334809,1220.38847931)
\curveto(13.55635389,1220.72709162)(13.1870357,1220.89640122)(12.75539242,1220.89640861)
\curveto(12.30885845,1220.89640122)(11.92000454,1220.71406781)(11.58882953,1220.34940782)
\curveto(11.25765051,1219.98473416)(11.09206201,1219.51215573)(11.09206351,1218.9316711)
\curveto(11.09206201,1218.41071308)(11.24927806,1217.98743908)(11.56371215,1217.66184786)
\curveto(11.87814228,1217.33624833)(12.26606591,1217.17345064)(12.72748422,1217.1734543)
\curveto(13.19261733,1217.17345064)(13.57495934,1217.33624833)(13.87451137,1217.66184786)
\curveto(14.17405483,1217.98743908)(14.32382871,1218.43862125)(14.32383344,1219.01539571)
\closepath
}
}
{
\newrgbcolor{curcolor}{0 0 0}
\pscustom[linestyle=none,fillstyle=solid,fillcolor=curcolor]
{
\newpath
\moveto(17.96864494,1217.94930235)
\curveto(17.55188084,1218.1018626)(17.24303036,1218.31954637)(17.0420926,1218.6023543)
\curveto(16.84115264,1218.88515206)(16.74068321,1219.22377125)(16.740684,1219.6182129)
\curveto(16.74068321,1220.2135812)(16.95464589,1220.71406781)(17.38257267,1221.11967423)
\curveto(17.81049659,1221.52526544)(18.37982337,1221.72806484)(19.09055471,1221.72807306)
\curveto(19.80500085,1221.72806484)(20.37990927,1221.52061408)(20.81528167,1221.10572013)
\curveto(21.25064433,1220.690811)(21.4683281,1220.18567302)(21.46833362,1219.5903047)
\curveto(21.4683281,1219.21074743)(21.36878894,1218.88050069)(21.16971585,1218.59956348)
\curveto(20.97063231,1218.3186161)(20.66829374,1218.1018626)(20.26269924,1217.94930235)
\curveto(20.76504208,1217.78556995)(21.14738409,1217.52137256)(21.40972639,1217.15670938)
\curveto(21.67205778,1216.79203891)(21.8032262,1216.35667137)(21.80323206,1215.85060547)
\curveto(21.8032262,1215.1510382)(21.55577372,1214.56310598)(21.06087385,1214.08680703)
\curveto(20.56596377,1213.61050693)(19.91477301,1213.37235716)(19.10729963,1213.37235703)
\curveto(18.29981993,1213.37235716)(17.64862918,1213.6114372)(17.15372541,1214.08959785)
\curveto(16.65881923,1214.56775734)(16.41136674,1215.16406202)(16.4113672,1215.87851367)
\curveto(16.41136674,1216.41062718)(16.54625625,1216.85622771)(16.81603615,1217.2153166)
\curveto(17.08581431,1217.57439809)(17.47001686,1217.81905976)(17.96864494,1217.94930235)
\closepath
\moveto(17.76770588,1219.65170274)
\curveto(17.76770406,1219.26470324)(17.89236057,1218.94841059)(18.1416758,1218.70282383)
\curveto(18.39098664,1218.4572267)(18.71472147,1218.33443073)(19.11288127,1218.33443555)
\curveto(19.49987147,1218.33443073)(19.8170944,1218.45629643)(20.064551,1218.70003301)
\curveto(20.31199937,1218.94375922)(20.43572562,1219.2423767)(20.4357301,1219.59588634)
\curveto(20.43572562,1219.96426817)(20.30827828,1220.27404891)(20.05338772,1220.5252295)
\curveto(19.79848895,1220.77639607)(19.48126602,1220.90198286)(19.10171799,1220.90199025)
\curveto(18.71844256,1220.90198286)(18.40028937,1220.77918688)(18.14725744,1220.53360196)
\curveto(17.89422112,1220.288003)(17.76770406,1219.99403689)(17.76770588,1219.65170274)
\closepath
\moveto(17.44397072,1215.87293203)
\curveto(17.44396923,1215.58640574)(17.51187912,1215.30918453)(17.6477006,1215.04126758)
\curveto(17.78351869,1214.77334756)(17.98538783,1214.56589679)(18.25330861,1214.41891464)
\curveto(18.52122479,1214.27193068)(18.80960927,1214.19843915)(19.11846291,1214.19843984)
\curveto(19.59848036,1214.19843915)(19.99477645,1214.35286439)(20.30735237,1214.66171601)
\curveto(20.61991957,1214.97056534)(20.77620535,1215.36314033)(20.77621018,1215.83944219)
\curveto(20.77620535,1216.32318156)(20.61526821,1216.72319874)(20.29339827,1217.03949492)
\curveto(19.97151963,1217.35578405)(19.56871164,1217.51393038)(19.08497307,1217.51393438)
\curveto(18.6123915,1217.51393038)(18.22074677,1217.3576446)(17.91003771,1217.04507657)
\curveto(17.59932473,1216.73250147)(17.44396923,1216.34178701)(17.44397072,1215.87293203)
\closepath
}
}
{
\newrgbcolor{curcolor}{0 0 0}
\pscustom[linestyle=none,fillstyle=solid,fillcolor=curcolor]
{
\newpath
\moveto(22.78560002,1217.54742422)
\curveto(22.78559954,1218.5149036)(22.8851387,1219.29354169)(23.08421779,1219.88334083)
\curveto(23.28329534,1220.47312723)(23.57912199,1220.92803049)(23.97169865,1221.24805197)
\curveto(24.36427199,1221.56805797)(24.8582467,1221.72806484)(25.45362424,1221.72807306)
\curveto(25.89270973,1221.72806484)(26.27784254,1221.63968896)(26.60902386,1221.46294513)
\curveto(26.94019657,1221.2861854)(27.21369669,1221.03129074)(27.42952503,1220.69826036)
\curveto(27.64534313,1220.36521562)(27.81465273,1219.9596168)(27.93745433,1219.4814627)
\curveto(28.06024467,1219.00329666)(28.12164266,1218.35861782)(28.12164847,1217.54742422)
\curveto(28.12164266,1216.58737896)(28.02303377,1215.81246196)(27.82582152,1215.2226709)
\curveto(27.62859823,1214.63287641)(27.33370184,1214.17704289)(26.94113147,1213.85516894)
\curveto(26.54855185,1213.53329431)(26.0527166,1213.37235716)(25.45362424,1213.37235703)
\curveto(24.66475001,1213.37235716)(24.04518852,1213.65516001)(23.59493791,1214.2207664)
\curveto(23.05537857,1214.90172517)(22.78559954,1216.01061)(22.78560002,1217.54742422)
\closepath
\moveto(23.81820354,1217.54742422)
\curveto(23.81820203,1216.20410668)(23.97541808,1215.3101148)(24.28985217,1214.8654459)
\curveto(24.6042823,1214.42077428)(24.99220594,1214.19843915)(25.45362424,1214.19843984)
\curveto(25.91503627,1214.19843915)(26.3029599,1214.42170456)(26.61739632,1214.86823672)
\curveto(26.93182412,1215.31476616)(27.08904017,1216.20782777)(27.08904495,1217.54742422)
\curveto(27.08904017,1218.89445478)(26.93182412,1219.78937693)(26.61739632,1220.23219337)
\curveto(26.3029599,1220.67499636)(25.91131518,1220.89640122)(25.44246096,1220.89640861)
\curveto(24.98104267,1220.89640122)(24.61265475,1220.70104399)(24.33729612,1220.31033634)
\curveto(23.99123272,1219.81170348)(23.81820203,1218.89073369)(23.81820354,1217.54742422)
\closepath
}
}
{
\newrgbcolor{curcolor}{0 0 0}
\pscustom[linestyle=none,fillstyle=solid,fillcolor=curcolor]
{
\newpath
\moveto(7.48074186,1169.84484482)
\lineto(6.47604655,1169.84484482)
\lineto(6.47604655,1176.24698663)
\curveto(6.23417244,1176.01627265)(5.91694951,1175.78556506)(5.52437682,1175.55486319)
\curveto(5.13179952,1175.3241499)(4.77922624,1175.15111921)(4.46665592,1175.03577061)
\lineto(4.46665592,1176.00697608)
\curveto(5.02853927,1176.27116731)(5.51972315,1176.59118106)(5.94020905,1176.96701827)
\curveto(6.3606895,1177.34284124)(6.6583767,1177.70750807)(6.83327155,1178.06101984)
\lineto(7.48074186,1178.06101984)
\closepath
}
}
{
\newrgbcolor{curcolor}{0 0 0}
\pscustom[linestyle=none,fillstyle=solid,fillcolor=curcolor]
{
\newpath
\moveto(10.21016429,1171.73702099)
\lineto(11.17578812,1171.82632725)
\curveto(11.25765051,1171.37235228)(11.4139363,1171.04303581)(11.64464594,1170.83837685)
\curveto(11.87535146,1170.63371591)(12.17117812,1170.53138593)(12.5321268,1170.53138662)
\curveto(12.84097433,1170.53138593)(13.11168363,1170.60208664)(13.34425551,1170.74348896)
\curveto(13.57681988,1170.88488948)(13.76752574,1171.0737348)(13.91637368,1171.31002549)
\curveto(14.06521295,1171.54631324)(14.18986946,1171.86539671)(14.2903436,1172.26727686)
\curveto(14.39080833,1172.66915216)(14.44104304,1173.07847206)(14.4410479,1173.4952378)
\curveto(14.44104304,1173.53988723)(14.4391825,1173.60686685)(14.43546626,1173.69617686)
\curveto(14.23452254,1173.37615926)(13.96009215,1173.11661323)(13.61217426,1172.91753799)
\curveto(13.26424832,1172.7184566)(12.88748795,1172.61891744)(12.48189203,1172.61892022)
\curveto(11.80465075,1172.61891744)(11.23160288,1172.86450938)(10.76274672,1173.35569678)
\curveto(10.29388819,1173.84687715)(10.05945952,1174.49434682)(10.05946,1175.29810772)
\curveto(10.05945952,1176.12790535)(10.30412119,1176.79584101)(10.79344574,1177.30191671)
\curveto(11.28276787,1177.8079775)(11.89581746,1178.06101162)(12.63259633,1178.06101984)
\curveto(13.16470916,1178.06101162)(13.65124168,1177.91774966)(14.09219536,1177.63123351)
\curveto(14.53314002,1177.34470179)(14.86803812,1176.93631216)(15.09689067,1176.40606339)
\curveto(15.3257322,1175.8758015)(15.44015572,1175.10832668)(15.44016157,1174.10363663)
\curveto(15.44015572,1173.05800607)(15.32666247,1172.22541217)(15.09968149,1171.60585244)
\curveto(14.87268949,1170.98628919)(14.53500056,1170.51464103)(14.08661372,1170.19090654)
\curveto(13.63821787,1169.86717136)(13.1126139,1169.70530394)(12.50980024,1169.7053038)
\curveto(11.86976983,1169.70530394)(11.34695668,1169.88298599)(10.94135922,1170.23835048)
\curveto(10.53575905,1170.59371419)(10.29202765,1171.09327053)(10.21016429,1171.73702099)
\closepath
\moveto(14.32383344,1175.34834249)
\curveto(14.32382871,1175.92510594)(14.17033374,1176.38280001)(13.86334809,1176.72142608)
\curveto(13.55635389,1177.0600384)(13.1870357,1177.229348)(12.75539242,1177.22935538)
\curveto(12.30885845,1177.229348)(11.92000454,1177.04701459)(11.58882953,1176.6823546)
\curveto(11.25765051,1176.31768094)(11.09206201,1175.8451025)(11.09206351,1175.26461788)
\curveto(11.09206201,1174.74365985)(11.24927806,1174.32038586)(11.56371215,1173.99479463)
\curveto(11.87814228,1173.6691951)(12.26606591,1173.50639742)(12.72748422,1173.50640108)
\curveto(13.19261733,1173.50639742)(13.57495934,1173.6691951)(13.87451137,1173.99479463)
\curveto(14.17405483,1174.32038586)(14.32382871,1174.77156803)(14.32383344,1175.34834249)
\closepath
}
}
{
\newrgbcolor{curcolor}{0 0 0}
\pscustom[linestyle=none,fillstyle=solid,fillcolor=curcolor]
{
\newpath
\moveto(17.96864494,1174.28224913)
\curveto(17.55188084,1174.43480938)(17.24303036,1174.65249315)(17.0420926,1174.93530108)
\curveto(16.84115264,1175.21809883)(16.74068321,1175.55671803)(16.740684,1175.95115968)
\curveto(16.74068321,1176.54652798)(16.95464589,1177.04701459)(17.38257267,1177.45262101)
\curveto(17.81049659,1177.85821221)(18.37982337,1178.06101162)(19.09055471,1178.06101984)
\curveto(19.80500085,1178.06101162)(20.37990927,1177.85356085)(20.81528167,1177.43866691)
\curveto(21.25064433,1177.02375777)(21.4683281,1176.5186198)(21.46833362,1175.92325147)
\curveto(21.4683281,1175.54369421)(21.36878894,1175.21344747)(21.16971585,1174.93251026)
\curveto(20.97063231,1174.65156287)(20.66829374,1174.43480938)(20.26269924,1174.28224913)
\curveto(20.76504208,1174.11851673)(21.14738409,1173.85431933)(21.40972639,1173.48965616)
\curveto(21.67205778,1173.12498569)(21.8032262,1172.68961815)(21.80323206,1172.18355225)
\curveto(21.8032262,1171.48398498)(21.55577372,1170.89605275)(21.06087385,1170.4197538)
\curveto(20.56596377,1169.9434537)(19.91477301,1169.70530394)(19.10729963,1169.7053038)
\curveto(18.29981993,1169.70530394)(17.64862918,1169.94438398)(17.15372541,1170.42254462)
\curveto(16.65881923,1170.90070412)(16.41136674,1171.4970088)(16.4113672,1172.21146045)
\curveto(16.41136674,1172.74357396)(16.54625625,1173.18917449)(16.81603615,1173.54826338)
\curveto(17.08581431,1173.90734487)(17.47001686,1174.15200654)(17.96864494,1174.28224913)
\closepath
\moveto(17.76770588,1175.98464952)
\curveto(17.76770406,1175.59765002)(17.89236057,1175.28135736)(18.1416758,1175.03577061)
\curveto(18.39098664,1174.79017348)(18.71472147,1174.66737751)(19.11288127,1174.66738233)
\curveto(19.49987147,1174.66737751)(19.8170944,1174.78924321)(20.064551,1175.03297979)
\curveto(20.31199937,1175.276706)(20.43572562,1175.57532348)(20.4357301,1175.92883311)
\curveto(20.43572562,1176.29721494)(20.30827828,1176.60699569)(20.05338772,1176.85817628)
\curveto(19.79848895,1177.10934284)(19.48126602,1177.23492963)(19.10171799,1177.23493702)
\curveto(18.71844256,1177.23492963)(18.40028937,1177.11213366)(18.14725744,1176.86654874)
\curveto(17.89422112,1176.62094978)(17.76770406,1176.32698366)(17.76770588,1175.98464952)
\closepath
\moveto(17.44397072,1172.20587881)
\curveto(17.44396923,1171.91935251)(17.51187912,1171.64213131)(17.6477006,1171.37421435)
\curveto(17.78351869,1171.10629434)(17.98538783,1170.89884357)(18.25330861,1170.75186142)
\curveto(18.52122479,1170.60487746)(18.80960927,1170.53138593)(19.11846291,1170.53138662)
\curveto(19.59848036,1170.53138593)(19.99477645,1170.68581117)(20.30735237,1170.99466279)
\curveto(20.61991957,1171.30351211)(20.77620535,1171.69608711)(20.77621018,1172.17238896)
\curveto(20.77620535,1172.65612834)(20.61526821,1173.05614552)(20.29339827,1173.3724417)
\curveto(19.97151963,1173.68873083)(19.56871164,1173.84687715)(19.08497307,1173.84688116)
\curveto(18.6123915,1173.84687715)(18.22074677,1173.69059137)(17.91003771,1173.37802334)
\curveto(17.59932473,1173.06544825)(17.44396923,1172.67473379)(17.44397072,1172.20587881)
\closepath
}
}
{
\newrgbcolor{curcolor}{0 0 0}
\pscustom[linestyle=none,fillstyle=solid,fillcolor=curcolor]
{
\newpath
\moveto(26.56995237,1169.84484482)
\lineto(25.56525706,1169.84484482)
\lineto(25.56525706,1176.24698663)
\curveto(25.32338295,1176.01627265)(25.00616002,1175.78556506)(24.61358733,1175.55486319)
\curveto(24.22101003,1175.3241499)(23.86843675,1175.15111921)(23.55586643,1175.03577061)
\lineto(23.55586643,1176.00697608)
\curveto(24.11774978,1176.27116731)(24.60893366,1176.59118106)(25.02941956,1176.96701827)
\curveto(25.44990001,1177.34284124)(25.74758721,1177.70750807)(25.92248206,1178.06101984)
\lineto(26.56995237,1178.06101984)
\closepath
}
}
{
\newrgbcolor{curcolor}{0 0 0}
\pscustom[linestyle=none,fillstyle=solid,fillcolor=curcolor]
{
\newpath
\moveto(7.48074186,1126.17785263)
\lineto(6.47604655,1126.17785263)
\lineto(6.47604655,1132.57999444)
\curveto(6.23417244,1132.34928046)(5.91694951,1132.11857288)(5.52437682,1131.887871)
\curveto(5.13179952,1131.65715771)(4.77922624,1131.48412702)(4.46665592,1131.36877842)
\lineto(4.46665592,1132.3399839)
\curveto(5.02853927,1132.60417513)(5.51972315,1132.92418887)(5.94020905,1133.30002609)
\curveto(6.3606895,1133.67584906)(6.6583767,1134.04051588)(6.83327155,1134.39402765)
\lineto(7.48074186,1134.39402765)
\closepath
}
}
{
\newrgbcolor{curcolor}{0 0 0}
\pscustom[linestyle=none,fillstyle=solid,fillcolor=curcolor]
{
\newpath
\moveto(10.21016429,1128.07002881)
\lineto(11.17578812,1128.15933506)
\curveto(11.25765051,1127.70536009)(11.4139363,1127.37604362)(11.64464594,1127.17138466)
\curveto(11.87535146,1126.96672372)(12.17117812,1126.86439374)(12.5321268,1126.86439443)
\curveto(12.84097433,1126.86439374)(13.11168363,1126.93509445)(13.34425551,1127.07649677)
\curveto(13.57681988,1127.2178973)(13.76752574,1127.40674262)(13.91637368,1127.6430333)
\curveto(14.06521295,1127.87932105)(14.18986946,1128.19840452)(14.2903436,1128.60028467)
\curveto(14.39080833,1129.00215997)(14.44104304,1129.41147987)(14.4410479,1129.82824561)
\curveto(14.44104304,1129.87289504)(14.4391825,1129.93987466)(14.43546626,1130.02918467)
\curveto(14.23452254,1129.70916708)(13.96009215,1129.44962105)(13.61217426,1129.2505458)
\curveto(13.26424832,1129.05146441)(12.88748795,1128.95192525)(12.48189203,1128.95192803)
\curveto(11.80465075,1128.95192525)(11.23160288,1129.1975172)(10.76274672,1129.68870459)
\curveto(10.29388819,1130.17988497)(10.05945952,1130.82735463)(10.05946,1131.63111553)
\curveto(10.05945952,1132.46091316)(10.30412119,1133.12884882)(10.79344574,1133.63492452)
\curveto(11.28276787,1134.14098531)(11.89581746,1134.39401943)(12.63259633,1134.39402765)
\curveto(13.16470916,1134.39401943)(13.65124168,1134.25075747)(14.09219536,1133.96424132)
\curveto(14.53314002,1133.6777096)(14.86803812,1133.26931997)(15.09689067,1132.7390712)
\curveto(15.3257322,1132.20880931)(15.44015572,1131.44133449)(15.44016157,1130.43664444)
\curveto(15.44015572,1129.39101388)(15.32666247,1128.55841998)(15.09968149,1127.93886025)
\curveto(14.87268949,1127.319297)(14.53500056,1126.84764884)(14.08661372,1126.52391435)
\curveto(13.63821787,1126.20017917)(13.1126139,1126.03831175)(12.50980024,1126.03831161)
\curveto(11.86976983,1126.03831175)(11.34695668,1126.2159938)(10.94135922,1126.5713583)
\curveto(10.53575905,1126.926722)(10.29202765,1127.42627834)(10.21016429,1128.07002881)
\closepath
\moveto(14.32383344,1131.6813503)
\curveto(14.32382871,1132.25811375)(14.17033374,1132.71580783)(13.86334809,1133.0544339)
\curveto(13.55635389,1133.39304621)(13.1870357,1133.56235581)(12.75539242,1133.5623632)
\curveto(12.30885845,1133.56235581)(11.92000454,1133.3800224)(11.58882953,1133.01536241)
\curveto(11.25765051,1132.65068875)(11.09206201,1132.17811032)(11.09206351,1131.59762569)
\curveto(11.09206201,1131.07666767)(11.24927806,1130.65339367)(11.56371215,1130.32780245)
\curveto(11.87814228,1130.00220292)(12.26606591,1129.83940523)(12.72748422,1129.83940889)
\curveto(13.19261733,1129.83940523)(13.57495934,1130.00220292)(13.87451137,1130.32780245)
\curveto(14.17405483,1130.65339367)(14.32382871,1131.10457584)(14.32383344,1131.6813503)
\closepath
}
}
{
\newrgbcolor{curcolor}{0 0 0}
\pscustom[linestyle=none,fillstyle=solid,fillcolor=curcolor]
{
\newpath
\moveto(17.96864494,1130.61525694)
\curveto(17.55188084,1130.76781719)(17.24303036,1130.98550096)(17.0420926,1131.26830889)
\curveto(16.84115264,1131.55110665)(16.74068321,1131.88972584)(16.740684,1132.28416749)
\curveto(16.74068321,1132.87953579)(16.95464589,1133.3800224)(17.38257267,1133.78562882)
\curveto(17.81049659,1134.19122003)(18.37982337,1134.39401943)(19.09055471,1134.39402765)
\curveto(19.80500085,1134.39401943)(20.37990927,1134.18656866)(20.81528167,1133.77167472)
\curveto(21.25064433,1133.35676559)(21.4683281,1132.85162761)(21.46833362,1132.25625929)
\curveto(21.4683281,1131.87670202)(21.36878894,1131.54645528)(21.16971585,1131.26551807)
\curveto(20.97063231,1130.98457069)(20.66829374,1130.76781719)(20.26269924,1130.61525694)
\curveto(20.76504208,1130.45152454)(21.14738409,1130.18732715)(21.40972639,1129.82266397)
\curveto(21.67205778,1129.4579935)(21.8032262,1129.02262596)(21.80323206,1128.51656006)
\curveto(21.8032262,1127.81699279)(21.55577372,1127.22906057)(21.06087385,1126.75276162)
\curveto(20.56596377,1126.27646152)(19.91477301,1126.03831175)(19.10729963,1126.03831161)
\curveto(18.29981993,1126.03831175)(17.64862918,1126.27739179)(17.15372541,1126.75555244)
\curveto(16.65881923,1127.23371193)(16.41136674,1127.83001661)(16.4113672,1128.54446826)
\curveto(16.41136674,1129.07658177)(16.54625625,1129.5221823)(16.81603615,1129.88127119)
\curveto(17.08581431,1130.24035268)(17.47001686,1130.48501435)(17.96864494,1130.61525694)
\closepath
\moveto(17.76770588,1132.31765733)
\curveto(17.76770406,1131.93065783)(17.89236057,1131.61436518)(18.1416758,1131.36877842)
\curveto(18.39098664,1131.12318129)(18.71472147,1131.00038532)(19.11288127,1131.00039014)
\curveto(19.49987147,1131.00038532)(19.8170944,1131.12225102)(20.064551,1131.3659876)
\curveto(20.31199937,1131.60971381)(20.43572562,1131.90833129)(20.4357301,1132.26184093)
\curveto(20.43572562,1132.63022276)(20.30827828,1132.9400035)(20.05338772,1133.19118409)
\curveto(19.79848895,1133.44235066)(19.48126602,1133.56793745)(19.10171799,1133.56794484)
\curveto(18.71844256,1133.56793745)(18.40028937,1133.44514147)(18.14725744,1133.19955655)
\curveto(17.89422112,1132.95395759)(17.76770406,1132.65999148)(17.76770588,1132.31765733)
\closepath
\moveto(17.44397072,1128.53888662)
\curveto(17.44396923,1128.25236033)(17.51187912,1127.97513912)(17.6477006,1127.70722217)
\curveto(17.78351869,1127.43930215)(17.98538783,1127.23185138)(18.25330861,1127.08486923)
\curveto(18.52122479,1126.93788527)(18.80960927,1126.86439374)(19.11846291,1126.86439443)
\curveto(19.59848036,1126.86439374)(19.99477645,1127.01881898)(20.30735237,1127.3276706)
\curveto(20.61991957,1127.63651993)(20.77620535,1128.02909492)(20.77621018,1128.50539678)
\curveto(20.77620535,1128.98913615)(20.61526821,1129.38915333)(20.29339827,1129.70544951)
\curveto(19.97151963,1130.02173864)(19.56871164,1130.17988497)(19.08497307,1130.17988897)
\curveto(18.6123915,1130.17988497)(18.22074677,1130.02359919)(17.91003771,1129.71103116)
\curveto(17.59932473,1129.39845606)(17.44396923,1129.0077416)(17.44397072,1128.53888662)
\closepath
}
}
{
\newrgbcolor{curcolor}{0 0 0}
\pscustom[linestyle=none,fillstyle=solid,fillcolor=curcolor]
{
\newpath
\moveto(28.06583206,1127.14347646)
\lineto(28.06583206,1126.17785263)
\lineto(22.65722228,1126.17785263)
\curveto(22.64977976,1126.41972348)(22.6888512,1126.65229161)(22.77443674,1126.87555771)
\curveto(22.91211661,1127.24394493)(23.13259119,1127.60675121)(23.43586115,1127.96397764)
\curveto(23.73912887,1128.32120049)(24.17728722,1128.73424149)(24.75033752,1129.20310186)
\curveto(25.6396756,1129.93243248)(26.24063164,1130.51013171)(26.55320745,1130.93620128)
\curveto(26.86577477,1131.36226133)(27.02206055,1131.76506932)(27.02206526,1132.14462647)
\curveto(27.02206055,1132.54277714)(26.87972886,1132.87860552)(26.59506975,1133.15211261)
\curveto(26.31040208,1133.42560575)(25.93922335,1133.56235581)(25.48153245,1133.5623632)
\curveto(24.99778757,1133.56235581)(24.61079421,1133.4172333)(24.32055119,1133.12699523)
\curveto(24.03030416,1132.83674325)(23.88332111,1132.43486553)(23.87960158,1131.92136085)
\lineto(22.84699807,1132.02741202)
\curveto(22.91769824,1132.79767181)(23.18375618,1133.38467376)(23.64517268,1133.78841964)
\curveto(24.10658651,1134.1921503)(24.726148,1134.39401943)(25.50385901,1134.39402765)
\curveto(26.28900581,1134.39401943)(26.91042785,1134.17633567)(27.36812698,1133.7409757)
\curveto(27.825816,1133.3056006)(28.05466304,1132.76604254)(28.05466878,1132.12229991)
\curveto(28.05466304,1131.79483804)(27.98768342,1131.47296375)(27.85372972,1131.15667608)
\curveto(27.71976493,1130.84037845)(27.4974298,1130.50734089)(27.18672366,1130.15756241)
\curveto(26.87600777,1129.80777596)(26.35970653,1129.32775535)(25.63781838,1128.71749912)
\curveto(25.03499847,1128.21142834)(24.64800511,1127.86815778)(24.47683713,1127.68768642)
\curveto(24.30566482,1127.50721205)(24.1642634,1127.32580891)(24.05263244,1127.14347646)
\closepath
}
}
{
\newrgbcolor{curcolor}{0 0 0}
\pscustom[linestyle=none,fillstyle=solid,fillcolor=curcolor]
{
\newpath
\moveto(7.48074186,1082.50988388)
\lineto(6.47604655,1082.50988388)
\lineto(6.47604655,1088.91202569)
\curveto(6.23417244,1088.68131171)(5.91694951,1088.45060413)(5.52437682,1088.21990225)
\curveto(5.13179952,1087.98918896)(4.77922624,1087.81615827)(4.46665592,1087.70080967)
\lineto(4.46665592,1088.67201515)
\curveto(5.02853927,1088.93620638)(5.51972315,1089.25622012)(5.94020905,1089.63205734)
\curveto(6.3606895,1090.00788031)(6.6583767,1090.37254713)(6.83327155,1090.7260589)
\lineto(7.48074186,1090.7260589)
\closepath
}
}
{
\newrgbcolor{curcolor}{0 0 0}
\pscustom[linestyle=none,fillstyle=solid,fillcolor=curcolor]
{
\newpath
\moveto(10.21016429,1084.40206006)
\lineto(11.17578812,1084.49136631)
\curveto(11.25765051,1084.03739134)(11.4139363,1083.70807487)(11.64464594,1083.50341591)
\curveto(11.87535146,1083.29875497)(12.17117812,1083.19642499)(12.5321268,1083.19642568)
\curveto(12.84097433,1083.19642499)(13.11168363,1083.2671257)(13.34425551,1083.40852802)
\curveto(13.57681988,1083.54992855)(13.76752574,1083.73877387)(13.91637368,1083.97506455)
\curveto(14.06521295,1084.2113523)(14.18986946,1084.53043577)(14.2903436,1084.93231592)
\curveto(14.39080833,1085.33419122)(14.44104304,1085.74351112)(14.4410479,1086.16027686)
\curveto(14.44104304,1086.20492629)(14.4391825,1086.27190591)(14.43546626,1086.36121592)
\curveto(14.23452254,1086.04119833)(13.96009215,1085.7816523)(13.61217426,1085.58257705)
\curveto(13.26424832,1085.38349566)(12.88748795,1085.2839565)(12.48189203,1085.28395928)
\curveto(11.80465075,1085.2839565)(11.23160288,1085.52954845)(10.76274672,1086.02073584)
\curveto(10.29388819,1086.51191622)(10.05945952,1087.15938588)(10.05946,1087.96314678)
\curveto(10.05945952,1088.79294441)(10.30412119,1089.46088007)(10.79344574,1089.96695577)
\curveto(11.28276787,1090.47301656)(11.89581746,1090.72605068)(12.63259633,1090.7260589)
\curveto(13.16470916,1090.72605068)(13.65124168,1090.58278872)(14.09219536,1090.29627257)
\curveto(14.53314002,1090.00974085)(14.86803812,1089.60135122)(15.09689067,1089.07110245)
\curveto(15.3257322,1088.54084056)(15.44015572,1087.77336574)(15.44016157,1086.76867569)
\curveto(15.44015572,1085.72304513)(15.32666247,1084.89045123)(15.09968149,1084.2708915)
\curveto(14.87268949,1083.65132825)(14.53500056,1083.17968009)(14.08661372,1082.8559456)
\curveto(13.63821787,1082.53221042)(13.1126139,1082.370343)(12.50980024,1082.37034286)
\curveto(11.86976983,1082.370343)(11.34695668,1082.54802505)(10.94135922,1082.90338955)
\curveto(10.53575905,1083.25875325)(10.29202765,1083.75830959)(10.21016429,1084.40206006)
\closepath
\moveto(14.32383344,1088.01338155)
\curveto(14.32382871,1088.590145)(14.17033374,1089.04783908)(13.86334809,1089.38646515)
\curveto(13.55635389,1089.72507746)(13.1870357,1089.89438706)(12.75539242,1089.89439445)
\curveto(12.30885845,1089.89438706)(11.92000454,1089.71205365)(11.58882953,1089.34739366)
\curveto(11.25765051,1088.98272)(11.09206201,1088.51014157)(11.09206351,1087.92965694)
\curveto(11.09206201,1087.40869892)(11.24927806,1086.98542492)(11.56371215,1086.6598337)
\curveto(11.87814228,1086.33423417)(12.26606591,1086.17143648)(12.72748422,1086.17144014)
\curveto(13.19261733,1086.17143648)(13.57495934,1086.33423417)(13.87451137,1086.6598337)
\curveto(14.17405483,1086.98542492)(14.32382871,1087.43660709)(14.32383344,1088.01338155)
\closepath
}
}
{
\newrgbcolor{curcolor}{0 0 0}
\pscustom[linestyle=none,fillstyle=solid,fillcolor=curcolor]
{
\newpath
\moveto(17.96864494,1086.94728819)
\curveto(17.55188084,1087.09984844)(17.24303036,1087.31753221)(17.0420926,1087.60034014)
\curveto(16.84115264,1087.8831379)(16.74068321,1088.22175709)(16.740684,1088.61619874)
\curveto(16.74068321,1089.21156704)(16.95464589,1089.71205365)(17.38257267,1090.11766007)
\curveto(17.81049659,1090.52325128)(18.37982337,1090.72605068)(19.09055471,1090.7260589)
\curveto(19.80500085,1090.72605068)(20.37990927,1090.51859991)(20.81528167,1090.10370597)
\curveto(21.25064433,1089.68879684)(21.4683281,1089.18365886)(21.46833362,1088.58829054)
\curveto(21.4683281,1088.20873327)(21.36878894,1087.87848653)(21.16971585,1087.59754932)
\curveto(20.97063231,1087.31660194)(20.66829374,1087.09984844)(20.26269924,1086.94728819)
\curveto(20.76504208,1086.78355579)(21.14738409,1086.5193584)(21.40972639,1086.15469522)
\curveto(21.67205778,1085.79002475)(21.8032262,1085.35465721)(21.80323206,1084.84859131)
\curveto(21.8032262,1084.14902404)(21.55577372,1083.56109182)(21.06087385,1083.08479287)
\curveto(20.56596377,1082.60849277)(19.91477301,1082.370343)(19.10729963,1082.37034286)
\curveto(18.29981993,1082.370343)(17.64862918,1082.60942304)(17.15372541,1083.08758369)
\curveto(16.65881923,1083.56574318)(16.41136674,1084.16204786)(16.4113672,1084.87649951)
\curveto(16.41136674,1085.40861302)(16.54625625,1085.85421355)(16.81603615,1086.21330244)
\curveto(17.08581431,1086.57238393)(17.47001686,1086.8170456)(17.96864494,1086.94728819)
\closepath
\moveto(17.76770588,1088.64968858)
\curveto(17.76770406,1088.26268908)(17.89236057,1087.94639643)(18.1416758,1087.70080967)
\curveto(18.39098664,1087.45521254)(18.71472147,1087.33241657)(19.11288127,1087.33242139)
\curveto(19.49987147,1087.33241657)(19.8170944,1087.45428227)(20.064551,1087.69801885)
\curveto(20.31199937,1087.94174506)(20.43572562,1088.24036254)(20.4357301,1088.59387218)
\curveto(20.43572562,1088.96225401)(20.30827828,1089.27203475)(20.05338772,1089.52321534)
\curveto(19.79848895,1089.77438191)(19.48126602,1089.8999687)(19.10171799,1089.89997609)
\curveto(18.71844256,1089.8999687)(18.40028937,1089.77717272)(18.14725744,1089.5315878)
\curveto(17.89422112,1089.28598884)(17.76770406,1088.99202273)(17.76770588,1088.64968858)
\closepath
\moveto(17.44397072,1084.87091787)
\curveto(17.44396923,1084.58439158)(17.51187912,1084.30717037)(17.6477006,1084.03925342)
\curveto(17.78351869,1083.7713334)(17.98538783,1083.56388263)(18.25330861,1083.41690048)
\curveto(18.52122479,1083.26991652)(18.80960927,1083.19642499)(19.11846291,1083.19642568)
\curveto(19.59848036,1083.19642499)(19.99477645,1083.35085023)(20.30735237,1083.65970185)
\curveto(20.61991957,1083.96855118)(20.77620535,1084.36112617)(20.77621018,1084.83742803)
\curveto(20.77620535,1085.3211674)(20.61526821,1085.72118458)(20.29339827,1086.03748076)
\curveto(19.97151963,1086.35376989)(19.56871164,1086.51191622)(19.08497307,1086.51192022)
\curveto(18.6123915,1086.51191622)(18.22074677,1086.35563044)(17.91003771,1086.04306241)
\curveto(17.59932473,1085.73048731)(17.44396923,1085.33977285)(17.44397072,1084.87091787)
\closepath
}
}
{
\newrgbcolor{curcolor}{0 0 0}
\pscustom[linestyle=none,fillstyle=solid,fillcolor=curcolor]
{
\newpath
\moveto(22.79118166,1084.66997881)
\lineto(23.79587697,1084.80393818)
\curveto(23.91122928,1084.23460911)(24.10751678,1083.82435894)(24.38474006,1083.57318642)
\curveto(24.6619592,1083.32201178)(24.99964812,1083.19642499)(25.39780784,1083.19642568)
\curveto(25.87038319,1083.19642499)(26.26947009,1083.36015295)(26.59506975,1083.68761006)
\curveto(26.92066085,1084.0150648)(27.08345854,1084.42066362)(27.08346331,1084.90440771)
\curveto(27.08345854,1085.36582048)(26.93275439,1085.74630194)(26.63135042,1086.04585323)
\curveto(26.32993781,1086.34539744)(25.94666553,1086.49517131)(25.48153245,1086.4951753)
\curveto(25.29175368,1086.49517131)(25.05546447,1086.45796041)(24.77266409,1086.38354248)
\lineto(24.8842969,1087.2654417)
\curveto(24.95127395,1087.25799477)(25.00522975,1087.25427368)(25.04616448,1087.25427842)
\curveto(25.4740871,1087.25427368)(25.85921992,1087.36590638)(26.20156409,1087.58917686)
\curveto(26.54390048,1087.81243718)(26.71507062,1088.15663801)(26.71507503,1088.62178038)
\curveto(26.71507062,1088.99016218)(26.59041411,1089.29529156)(26.34110511,1089.53716944)
\curveto(26.09178804,1089.77903327)(25.76991375,1089.8999687)(25.37548127,1089.89997609)
\curveto(24.98476376,1089.8999687)(24.65916838,1089.77717272)(24.39869416,1089.5315878)
\curveto(24.13821577,1089.28598884)(23.97076672,1088.91760093)(23.89634651,1088.42642296)
\lineto(22.89165119,1088.60503546)
\curveto(23.01444658,1089.27854666)(23.29352833,1089.80042954)(23.72889729,1090.17068566)
\curveto(24.1642634,1090.54092646)(24.705682,1090.72605068)(25.35315471,1090.7260589)
\curveto(25.79968247,1090.72605068)(26.21086292,1090.63023262)(26.58669729,1090.43860441)
\curveto(26.96252311,1090.24696034)(27.24997732,1089.98555377)(27.44906077,1089.6543839)
\curveto(27.64813395,1089.32319974)(27.74767311,1088.97155673)(27.74767855,1088.59945382)
\curveto(27.74767311,1088.24594417)(27.65278531,1087.92406989)(27.46301487,1087.63382999)
\curveto(27.27323413,1087.34357984)(26.99229183,1087.11287226)(26.62018714,1086.94170655)
\curveto(27.10392453,1086.83006941)(27.47975463,1086.59843156)(27.74767855,1086.24679229)
\curveto(28.01559159,1085.89514554)(28.14955083,1085.45512665)(28.14955667,1084.92673428)
\curveto(28.14955083,1084.21228257)(27.88907453,1083.60667517)(27.36812698,1083.10991025)
\curveto(26.84716932,1082.61314413)(26.18853638,1082.36476137)(25.3922262,1082.36476122)
\curveto(24.67405274,1082.36476137)(24.07774806,1082.57872405)(23.60331037,1083.0066499)
\curveto(23.1288701,1083.43457476)(22.8581608,1083.98901717)(22.79118166,1084.66997881)
\closepath
}
}
{
\newrgbcolor{curcolor}{0 0 0}
\pscustom[linestyle=none,fillstyle=solid,fillcolor=curcolor]
{
\newpath
\moveto(7.48074186,1038.84289169)
\lineto(6.47604655,1038.84289169)
\lineto(6.47604655,1045.24503351)
\curveto(6.23417244,1045.01431952)(5.91694951,1044.78361194)(5.52437682,1044.55291007)
\curveto(5.13179952,1044.32219677)(4.77922624,1044.14916609)(4.46665592,1044.03381749)
\lineto(4.46665592,1045.00502296)
\curveto(5.02853927,1045.26921419)(5.51972315,1045.58922793)(5.94020905,1045.96506515)
\curveto(6.3606895,1046.34088812)(6.6583767,1046.70555494)(6.83327155,1047.05906671)
\lineto(7.48074186,1047.05906671)
\closepath
}
}
{
\newrgbcolor{curcolor}{0 0 0}
\pscustom[linestyle=none,fillstyle=solid,fillcolor=curcolor]
{
\newpath
\moveto(10.21016429,1040.73506787)
\lineto(11.17578812,1040.82437412)
\curveto(11.25765051,1040.37039915)(11.4139363,1040.04108269)(11.64464594,1039.83642373)
\curveto(11.87535146,1039.63176278)(12.17117812,1039.52943281)(12.5321268,1039.52943349)
\curveto(12.84097433,1039.52943281)(13.11168363,1039.60013352)(13.34425551,1039.74153584)
\curveto(13.57681988,1039.88293636)(13.76752574,1040.07178168)(13.91637368,1040.30807236)
\curveto(14.06521295,1040.54436011)(14.18986946,1040.86344358)(14.2903436,1041.26532373)
\curveto(14.39080833,1041.66719903)(14.44104304,1042.07651894)(14.4410479,1042.49328467)
\curveto(14.44104304,1042.5379341)(14.4391825,1042.60491372)(14.43546626,1042.69422373)
\curveto(14.23452254,1042.37420614)(13.96009215,1042.11466011)(13.61217426,1041.91558486)
\curveto(13.26424832,1041.71650348)(12.88748795,1041.61696432)(12.48189203,1041.61696709)
\curveto(11.80465075,1041.61696432)(11.23160288,1041.86255626)(10.76274672,1042.35374365)
\curveto(10.29388819,1042.84492403)(10.05945952,1043.4923937)(10.05946,1044.2961546)
\curveto(10.05945952,1045.12595222)(10.30412119,1045.79388788)(10.79344574,1046.29996359)
\curveto(11.28276787,1046.80602437)(11.89581746,1047.0590585)(12.63259633,1047.05906671)
\curveto(13.16470916,1047.0590585)(13.65124168,1046.91579653)(14.09219536,1046.62928038)
\curveto(14.53314002,1046.34274866)(14.86803812,1045.93435903)(15.09689067,1045.40411026)
\curveto(15.3257322,1044.87384837)(15.44015572,1044.10637355)(15.44016157,1043.1016835)
\curveto(15.44015572,1042.05605294)(15.32666247,1041.22345905)(15.09968149,1040.60389931)
\curveto(14.87268949,1039.98433606)(14.53500056,1039.5126879)(14.08661372,1039.18895341)
\curveto(13.63821787,1038.86521823)(13.1126139,1038.70335082)(12.50980024,1038.70335068)
\curveto(11.86976983,1038.70335082)(11.34695668,1038.88103287)(10.94135922,1039.23639736)
\curveto(10.53575905,1039.59176106)(10.29202765,1040.0913174)(10.21016429,1040.73506787)
\closepath
\moveto(14.32383344,1044.34638936)
\curveto(14.32382871,1044.92315282)(14.17033374,1045.38084689)(13.86334809,1045.71947296)
\curveto(13.55635389,1046.05808528)(13.1870357,1046.22739487)(12.75539242,1046.22740226)
\curveto(12.30885845,1046.22739487)(11.92000454,1046.04506146)(11.58882953,1045.68040148)
\curveto(11.25765051,1045.31572781)(11.09206201,1044.84314938)(11.09206351,1044.26266475)
\curveto(11.09206201,1043.74170673)(11.24927806,1043.31843274)(11.56371215,1042.99284151)
\curveto(11.87814228,1042.66724198)(12.26606591,1042.50444429)(12.72748422,1042.50444795)
\curveto(13.19261733,1042.50444429)(13.57495934,1042.66724198)(13.87451137,1042.99284151)
\curveto(14.17405483,1043.31843274)(14.32382871,1043.7696149)(14.32383344,1044.34638936)
\closepath
}
}
{
\newrgbcolor{curcolor}{0 0 0}
\pscustom[linestyle=none,fillstyle=solid,fillcolor=curcolor]
{
\newpath
\moveto(17.96864494,1043.280296)
\curveto(17.55188084,1043.43285626)(17.24303036,1043.65054002)(17.0420926,1043.93334796)
\curveto(16.84115264,1044.21614571)(16.74068321,1044.5547649)(16.740684,1044.94920655)
\curveto(16.74068321,1045.54457485)(16.95464589,1046.04506146)(17.38257267,1046.45066788)
\curveto(17.81049659,1046.85625909)(18.37982337,1047.0590585)(19.09055471,1047.05906671)
\curveto(19.80500085,1047.0590585)(20.37990927,1046.85160773)(20.81528167,1046.43671378)
\curveto(21.25064433,1046.02180465)(21.4683281,1045.51666668)(21.46833362,1044.92129835)
\curveto(21.4683281,1044.54174109)(21.36878894,1044.21149435)(21.16971585,1043.93055714)
\curveto(20.97063231,1043.64960975)(20.66829374,1043.43285626)(20.26269924,1043.280296)
\curveto(20.76504208,1043.1165636)(21.14738409,1042.85236621)(21.40972639,1042.48770303)
\curveto(21.67205778,1042.12303256)(21.8032262,1041.68766503)(21.80323206,1041.18159912)
\curveto(21.8032262,1040.48203186)(21.55577372,1039.89409963)(21.06087385,1039.41780068)
\curveto(20.56596377,1038.94150058)(19.91477301,1038.70335082)(19.10729963,1038.70335068)
\curveto(18.29981993,1038.70335082)(17.64862918,1038.94243085)(17.15372541,1039.4205915)
\curveto(16.65881923,1039.89875099)(16.41136674,1040.49505567)(16.4113672,1041.20950732)
\curveto(16.41136674,1041.74162083)(16.54625625,1042.18722136)(16.81603615,1042.54631026)
\curveto(17.08581431,1042.90539174)(17.47001686,1043.15005341)(17.96864494,1043.280296)
\closepath
\moveto(17.76770588,1044.9826964)
\curveto(17.76770406,1044.59569689)(17.89236057,1044.27940424)(18.1416758,1044.03381749)
\curveto(18.39098664,1043.78822035)(18.71472147,1043.66542438)(19.11288127,1043.6654292)
\curveto(19.49987147,1043.66542438)(19.8170944,1043.78729008)(20.064551,1044.03102667)
\curveto(20.31199937,1044.27475288)(20.43572562,1044.57337035)(20.4357301,1044.92687999)
\curveto(20.43572562,1045.29526182)(20.30827828,1045.60504256)(20.05338772,1045.85622316)
\curveto(19.79848895,1046.10738972)(19.48126602,1046.23297651)(19.10171799,1046.2329839)
\curveto(18.71844256,1046.23297651)(18.40028937,1046.11018054)(18.14725744,1045.86459562)
\curveto(17.89422112,1045.61899665)(17.76770406,1045.32503054)(17.76770588,1044.9826964)
\closepath
\moveto(17.44397072,1041.20392568)
\curveto(17.44396923,1040.91739939)(17.51187912,1040.64017818)(17.6477006,1040.37226123)
\curveto(17.78351869,1040.10434122)(17.98538783,1039.89689045)(18.25330861,1039.7499083)
\curveto(18.52122479,1039.60292433)(18.80960927,1039.52943281)(19.11846291,1039.52943349)
\curveto(19.59848036,1039.52943281)(19.99477645,1039.68385804)(20.30735237,1039.99270966)
\curveto(20.61991957,1040.30155899)(20.77620535,1040.69413399)(20.77621018,1041.17043584)
\curveto(20.77620535,1041.65417522)(20.61526821,1042.0541924)(20.29339827,1042.37048858)
\curveto(19.97151963,1042.6867777)(19.56871164,1042.84492403)(19.08497307,1042.84492803)
\curveto(18.6123915,1042.84492403)(18.22074677,1042.68863825)(17.91003771,1042.37607022)
\curveto(17.59932473,1042.06349512)(17.44396923,1041.67278067)(17.44397072,1041.20392568)
\closepath
}
}
{
\newrgbcolor{curcolor}{0 0 0}
\pscustom[linestyle=none,fillstyle=solid,fillcolor=curcolor]
{
\newpath
\moveto(26.00620667,1038.84289169)
\lineto(26.00620667,1040.80204756)
\lineto(22.45628322,1040.80204756)
\lineto(22.45628322,1041.72301826)
\lineto(26.19040081,1047.02557687)
\lineto(27.01090198,1047.02557687)
\lineto(27.01090198,1041.72301826)
\lineto(28.11606683,1041.72301826)
\lineto(28.11606683,1040.80204756)
\lineto(27.01090198,1040.80204756)
\lineto(27.01090198,1038.84289169)
\closepath
\moveto(26.00620667,1041.72301826)
\lineto(26.00620667,1045.41248272)
\lineto(23.44423361,1041.72301826)
\closepath
}
}
{
\newrgbcolor{curcolor}{0 0 0}
\pscustom[linestyle=none,fillstyle=solid,fillcolor=curcolor]
{
\newpath
\moveto(7.48074186,995.17589951)
\lineto(6.47604655,995.17589951)
\lineto(6.47604655,1001.57804132)
\curveto(6.23417244,1001.34732733)(5.91694951,1001.11661975)(5.52437682,1000.88591788)
\curveto(5.13179952,1000.65520459)(4.77922624,1000.4821739)(4.46665592,1000.3668253)
\lineto(4.46665592,1001.33803077)
\curveto(5.02853927,1001.602222)(5.51972315,1001.92223574)(5.94020905,1002.29807296)
\curveto(6.3606895,1002.67389593)(6.6583767,1003.03856276)(6.83327155,1003.39207453)
\lineto(7.48074186,1003.39207453)
\closepath
}
}
{
\newrgbcolor{curcolor}{0 0 0}
\pscustom[linestyle=none,fillstyle=solid,fillcolor=curcolor]
{
\newpath
\moveto(10.21016429,997.06807568)
\lineto(11.17578812,997.15738193)
\curveto(11.25765051,996.70340697)(11.4139363,996.3740905)(11.64464594,996.16943154)
\curveto(11.87535146,995.96477059)(12.17117812,995.86244062)(12.5321268,995.8624413)
\curveto(12.84097433,995.86244062)(13.11168363,995.93314133)(13.34425551,996.07454365)
\curveto(13.57681988,996.21594417)(13.76752574,996.40478949)(13.91637368,996.64108017)
\curveto(14.06521295,996.87736793)(14.18986946,997.1964514)(14.2903436,997.59833154)
\curveto(14.39080833,998.00020684)(14.44104304,998.40952675)(14.4410479,998.82629248)
\curveto(14.44104304,998.87094191)(14.4391825,998.93792153)(14.43546626,999.02723155)
\curveto(14.23452254,998.70721395)(13.96009215,998.44766792)(13.61217426,998.24859268)
\curveto(13.26424832,998.04951129)(12.88748795,997.94997213)(12.48189203,997.9499749)
\curveto(11.80465075,997.94997213)(11.23160288,998.19556407)(10.76274672,998.68675147)
\curveto(10.29388819,999.17793184)(10.05945952,999.82540151)(10.05946,1000.62916241)
\curveto(10.05945952,1001.45896003)(10.30412119,1002.1268957)(10.79344574,1002.6329714)
\curveto(11.28276787,1003.13903219)(11.89581746,1003.39206631)(12.63259633,1003.39207453)
\curveto(13.16470916,1003.39206631)(13.65124168,1003.24880434)(14.09219536,1002.9622882)
\curveto(14.53314002,1002.67575648)(14.86803812,1002.26736685)(15.09689067,1001.73711808)
\curveto(15.3257322,1001.20685618)(15.44015572,1000.43938136)(15.44016157,999.43469131)
\curveto(15.44015572,998.38906075)(15.32666247,997.55646686)(15.09968149,996.93690713)
\curveto(14.87268949,996.31734388)(14.53500056,995.84569571)(14.08661372,995.52196123)
\curveto(13.63821787,995.19822605)(13.1126139,995.03635863)(12.50980024,995.03635849)
\curveto(11.86976983,995.03635863)(11.34695668,995.21404068)(10.94135922,995.56940517)
\curveto(10.53575905,995.92476888)(10.29202765,996.42432521)(10.21016429,997.06807568)
\closepath
\moveto(14.32383344,1000.67939718)
\curveto(14.32382871,1001.25616063)(14.17033374,1001.7138547)(13.86334809,1002.05248077)
\curveto(13.55635389,1002.39109309)(13.1870357,1002.56040269)(12.75539242,1002.56041007)
\curveto(12.30885845,1002.56040269)(11.92000454,1002.37806927)(11.58882953,1002.01340929)
\curveto(11.25765051,1001.64873563)(11.09206201,1001.17615719)(11.09206351,1000.59567257)
\curveto(11.09206201,1000.07471454)(11.24927806,999.65144055)(11.56371215,999.32584932)
\curveto(11.87814228,999.00024979)(12.26606591,998.8374521)(12.72748422,998.83745576)
\curveto(13.19261733,998.8374521)(13.57495934,999.00024979)(13.87451137,999.32584932)
\curveto(14.17405483,999.65144055)(14.32382871,1000.10262272)(14.32383344,1000.67939718)
\closepath
}
}
{
\newrgbcolor{curcolor}{0 0 0}
\pscustom[linestyle=none,fillstyle=solid,fillcolor=curcolor]
{
\newpath
\moveto(17.96864494,999.61330381)
\curveto(17.55188084,999.76586407)(17.24303036,999.98354783)(17.0420926,1000.26635577)
\curveto(16.84115264,1000.54915352)(16.74068321,1000.88777271)(16.740684,1001.28221436)
\curveto(16.74068321,1001.87758266)(16.95464589,1002.37806927)(17.38257267,1002.7836757)
\curveto(17.81049659,1003.1892669)(18.37982337,1003.39206631)(19.09055471,1003.39207453)
\curveto(19.80500085,1003.39206631)(20.37990927,1003.18461554)(20.81528167,1002.76972159)
\curveto(21.25064433,1002.35481246)(21.4683281,1001.84967449)(21.46833362,1001.25430616)
\curveto(21.4683281,1000.8747489)(21.36878894,1000.54450216)(21.16971585,1000.26356495)
\curveto(20.97063231,999.98261756)(20.66829374,999.76586407)(20.26269924,999.61330381)
\curveto(20.76504208,999.44957141)(21.14738409,999.18537402)(21.40972639,998.82071084)
\curveto(21.67205778,998.45604037)(21.8032262,998.02067284)(21.80323206,997.51460693)
\curveto(21.8032262,996.81503967)(21.55577372,996.22710744)(21.06087385,995.75080849)
\curveto(20.56596377,995.27450839)(19.91477301,995.03635863)(19.10729963,995.03635849)
\curveto(18.29981993,995.03635863)(17.64862918,995.27543866)(17.15372541,995.75359931)
\curveto(16.65881923,996.2317588)(16.41136674,996.82806348)(16.4113672,997.54251514)
\curveto(16.41136674,998.07462865)(16.54625625,998.52022918)(16.81603615,998.87931807)
\curveto(17.08581431,999.23839955)(17.47001686,999.48306122)(17.96864494,999.61330381)
\closepath
\moveto(17.76770588,1001.31570421)
\curveto(17.76770406,1000.9287047)(17.89236057,1000.61241205)(18.1416758,1000.3668253)
\curveto(18.39098664,1000.12122817)(18.71472147,999.99843219)(19.11288127,999.99843702)
\curveto(19.49987147,999.99843219)(19.8170944,1000.12029789)(20.064551,1000.36403448)
\curveto(20.31199937,1000.60776069)(20.43572562,1000.90637816)(20.4357301,1001.2598878)
\curveto(20.43572562,1001.62826963)(20.30827828,1001.93805038)(20.05338772,1002.18923097)
\curveto(19.79848895,1002.44039753)(19.48126602,1002.56598432)(19.10171799,1002.56599171)
\curveto(18.71844256,1002.56598432)(18.40028937,1002.44318835)(18.14725744,1002.19760343)
\curveto(17.89422112,1001.95200446)(17.76770406,1001.65803835)(17.76770588,1001.31570421)
\closepath
\moveto(17.44397072,997.5369335)
\curveto(17.44396923,997.2504072)(17.51187912,996.97318599)(17.6477006,996.70526904)
\curveto(17.78351869,996.43734903)(17.98538783,996.22989826)(18.25330861,996.08291611)
\curveto(18.52122479,995.93593215)(18.80960927,995.86244062)(19.11846291,995.8624413)
\curveto(19.59848036,995.86244062)(19.99477645,996.01686585)(20.30735237,996.32571748)
\curveto(20.61991957,996.6345668)(20.77620535,997.0271418)(20.77621018,997.50344365)
\curveto(20.77620535,997.98718303)(20.61526821,998.38720021)(20.29339827,998.70349639)
\curveto(19.97151963,999.01978551)(19.56871164,999.17793184)(19.08497307,999.17793584)
\curveto(18.6123915,999.17793184)(18.22074677,999.02164606)(17.91003771,998.70907803)
\curveto(17.59932473,998.39650293)(17.44396923,998.00578848)(17.44397072,997.5369335)
\closepath
}
}
{
\newrgbcolor{curcolor}{0 0 0}
\pscustom[linestyle=none,fillstyle=solid,fillcolor=curcolor]
{
\newpath
\moveto(22.78560002,997.31924951)
\lineto(23.8405301,997.40855576)
\curveto(23.91867146,996.8950431)(24.1000746,996.50898001)(24.38474006,996.25036533)
\curveto(24.66940138,995.9917485)(25.01267193,995.86244062)(25.41455276,995.8624413)
\curveto(25.89829136,995.86244062)(26.30761127,996.04477403)(26.6425137,996.40944209)
\curveto(26.97740747,996.77410768)(27.14485652,997.25784938)(27.14486136,997.86066865)
\curveto(27.14485652,998.43371383)(26.98391938,998.88582627)(26.66204944,999.21700733)
\curveto(26.3401708,999.5481803)(25.91875736,999.71376881)(25.39780784,999.71377334)
\curveto(25.07406992,999.71376881)(24.78196435,999.64027728)(24.52149026,999.49329854)
\curveto(24.26101174,999.34631117)(24.05635179,999.1556053)(23.90750979,998.92118037)
\lineto(22.96421252,999.04397647)
\lineto(23.75680549,1003.24695187)
\lineto(27.82582152,1003.24695187)
\lineto(27.82582152,1002.28690968)
\lineto(24.56056174,1002.28690968)
\lineto(24.11961213,1000.08774327)
\curveto(24.61079421,1000.43007864)(25.12616518,1000.60124878)(25.66572659,1000.60125421)
\curveto(26.38017252,1000.60124878)(26.98298911,1000.35379629)(27.47417815,999.858896)
\curveto(27.96535688,999.36398634)(28.21094882,998.72767995)(28.21095472,997.9499749)
\curveto(28.21094882,997.20947521)(27.9951256,996.56944773)(27.56348441,996.02989052)
\curveto(27.03880546,995.36753564)(26.32249563,995.03635863)(25.41455276,995.03635849)
\curveto(24.67033165,995.03635863)(24.0628637,995.24473967)(23.59214709,995.66150224)
\curveto(23.12142792,996.07826384)(22.85257916,996.63084571)(22.78560002,997.31924951)
\closepath
}
}
{
\newrgbcolor{curcolor}{0 0 0}
\pscustom[linestyle=none,fillstyle=solid,fillcolor=curcolor]
{
\newpath
\moveto(7.48074186,951.50884628)
\lineto(6.47604655,951.50884628)
\lineto(6.47604655,957.9109881)
\curveto(6.23417244,957.68027411)(5.91694951,957.44956653)(5.52437682,957.21886466)
\curveto(5.13179952,956.98815136)(4.77922624,956.81512068)(4.46665592,956.69977208)
\lineto(4.46665592,957.67097755)
\curveto(5.02853927,957.93516878)(5.51972315,958.25518252)(5.94020905,958.63101974)
\curveto(6.3606895,959.00684271)(6.6583767,959.37150953)(6.83327155,959.7250213)
\lineto(7.48074186,959.7250213)
\closepath
}
}
{
\newrgbcolor{curcolor}{0 0 0}
\pscustom[linestyle=none,fillstyle=solid,fillcolor=curcolor]
{
\newpath
\moveto(10.21016429,953.40102246)
\lineto(11.17578812,953.49032871)
\curveto(11.25765051,953.03635374)(11.4139363,952.70703728)(11.64464594,952.50237832)
\curveto(11.87535146,952.29771737)(12.17117812,952.1953874)(12.5321268,952.19538808)
\curveto(12.84097433,952.1953874)(13.11168363,952.26608811)(13.34425551,952.40749043)
\curveto(13.57681988,952.54889095)(13.76752574,952.73773627)(13.91637368,952.97402695)
\curveto(14.06521295,953.2103147)(14.18986946,953.52939817)(14.2903436,953.93127832)
\curveto(14.39080833,954.33315362)(14.44104304,954.74247353)(14.4410479,955.15923926)
\curveto(14.44104304,955.20388869)(14.4391825,955.27086831)(14.43546626,955.36017832)
\curveto(14.23452254,955.04016073)(13.96009215,954.7806147)(13.61217426,954.58153945)
\curveto(13.26424832,954.38245806)(12.88748795,954.28291891)(12.48189203,954.28292168)
\curveto(11.80465075,954.28291891)(11.23160288,954.52851085)(10.76274672,955.01969824)
\curveto(10.29388819,955.51087862)(10.05945952,956.15834829)(10.05946,956.96210919)
\curveto(10.05945952,957.79190681)(10.30412119,958.45984247)(10.79344574,958.96591818)
\curveto(11.28276787,959.47197896)(11.89581746,959.72501309)(12.63259633,959.7250213)
\curveto(13.16470916,959.72501309)(13.65124168,959.58175112)(14.09219536,959.29523497)
\curveto(14.53314002,959.00870325)(14.86803812,958.60031362)(15.09689067,958.07006485)
\curveto(15.3257322,957.53980296)(15.44015572,956.77232814)(15.44016157,955.76763809)
\curveto(15.44015572,954.72200753)(15.32666247,953.88941363)(15.09968149,953.2698539)
\curveto(14.87268949,952.65029065)(14.53500056,952.17864249)(14.08661372,951.854908)
\curveto(13.63821787,951.53117282)(13.1126139,951.36930541)(12.50980024,951.36930527)
\curveto(11.86976983,951.36930541)(11.34695668,951.54698746)(10.94135922,951.90235195)
\curveto(10.53575905,952.25771565)(10.29202765,952.75727199)(10.21016429,953.40102246)
\closepath
\moveto(14.32383344,957.01234395)
\curveto(14.32382871,957.58910741)(14.17033374,958.04680148)(13.86334809,958.38542755)
\curveto(13.55635389,958.72403987)(13.1870357,958.89334946)(12.75539242,958.89335685)
\curveto(12.30885845,958.89334946)(11.92000454,958.71101605)(11.58882953,958.34635607)
\curveto(11.25765051,957.9816824)(11.09206201,957.50910397)(11.09206351,956.92861934)
\curveto(11.09206201,956.40766132)(11.24927806,955.98438733)(11.56371215,955.6587961)
\curveto(11.87814228,955.33319657)(12.26606591,955.17039888)(12.72748422,955.17040254)
\curveto(13.19261733,955.17039888)(13.57495934,955.33319657)(13.87451137,955.6587961)
\curveto(14.17405483,955.98438733)(14.32382871,956.43556949)(14.32383344,957.01234395)
\closepath
}
}
{
\newrgbcolor{curcolor}{0 0 0}
\pscustom[linestyle=none,fillstyle=solid,fillcolor=curcolor]
{
\newpath
\moveto(17.96864494,955.94625059)
\curveto(17.55188084,956.09881084)(17.24303036,956.31649461)(17.0420926,956.59930255)
\curveto(16.84115264,956.8821003)(16.74068321,957.22071949)(16.740684,957.61516114)
\curveto(16.74068321,958.21052944)(16.95464589,958.71101605)(17.38257267,959.11662247)
\curveto(17.81049659,959.52221368)(18.37982337,959.72501309)(19.09055471,959.7250213)
\curveto(19.80500085,959.72501309)(20.37990927,959.51756232)(20.81528167,959.10266837)
\curveto(21.25064433,958.68775924)(21.4683281,958.18262127)(21.46833362,957.58725294)
\curveto(21.4683281,957.20769568)(21.36878894,956.87744894)(21.16971585,956.59651173)
\curveto(20.97063231,956.31556434)(20.66829374,956.09881084)(20.26269924,955.94625059)
\curveto(20.76504208,955.78251819)(21.14738409,955.5183208)(21.40972639,955.15365762)
\curveto(21.67205778,954.78898715)(21.8032262,954.35361962)(21.80323206,953.84755371)
\curveto(21.8032262,953.14798644)(21.55577372,952.56005422)(21.06087385,952.08375527)
\curveto(20.56596377,951.60745517)(19.91477301,951.36930541)(19.10729963,951.36930527)
\curveto(18.29981993,951.36930541)(17.64862918,951.60838544)(17.15372541,952.08654609)
\curveto(16.65881923,952.56470558)(16.41136674,953.16101026)(16.4113672,953.87546191)
\curveto(16.41136674,954.40757542)(16.54625625,954.85317595)(16.81603615,955.21226485)
\curveto(17.08581431,955.57134633)(17.47001686,955.816008)(17.96864494,955.94625059)
\closepath
\moveto(17.76770588,957.64865099)
\curveto(17.76770406,957.26165148)(17.89236057,956.94535883)(18.1416758,956.69977208)
\curveto(18.39098664,956.45417494)(18.71472147,956.33137897)(19.11288127,956.33138379)
\curveto(19.49987147,956.33137897)(19.8170944,956.45324467)(20.064551,956.69698126)
\curveto(20.31199937,956.94070747)(20.43572562,957.23932494)(20.4357301,957.59283458)
\curveto(20.43572562,957.96121641)(20.30827828,958.27099715)(20.05338772,958.52217775)
\curveto(19.79848895,958.77334431)(19.48126602,958.8989311)(19.10171799,958.89893849)
\curveto(18.71844256,958.8989311)(18.40028937,958.77613513)(18.14725744,958.53055021)
\curveto(17.89422112,958.28495124)(17.76770406,957.99098513)(17.76770588,957.64865099)
\closepath
\moveto(17.44397072,953.86988027)
\curveto(17.44396923,953.58335398)(17.51187912,953.30613277)(17.6477006,953.03821582)
\curveto(17.78351869,952.77029581)(17.98538783,952.56284504)(18.25330861,952.41586289)
\curveto(18.52122479,952.26887892)(18.80960927,952.1953874)(19.11846291,952.19538808)
\curveto(19.59848036,952.1953874)(19.99477645,952.34981263)(20.30735237,952.65866425)
\curveto(20.61991957,952.96751358)(20.77620535,953.36008858)(20.77621018,953.83639043)
\curveto(20.77620535,954.32012981)(20.61526821,954.72014699)(20.29339827,955.03644317)
\curveto(19.97151963,955.35273229)(19.56871164,955.51087862)(19.08497307,955.51088262)
\curveto(18.6123915,955.51087862)(18.22074677,955.35459284)(17.91003771,955.04202481)
\curveto(17.59932473,954.72944971)(17.44396923,954.33873526)(17.44397072,953.86988027)
\closepath
}
}
{
\newrgbcolor{curcolor}{0 0 0}
\pscustom[linestyle=none,fillstyle=solid,fillcolor=curcolor]
{
\newpath
\moveto(27.99885237,957.68772247)
\lineto(26.9997387,957.6095795)
\curveto(26.91042785,958.00400894)(26.78391079,958.29053288)(26.62018714,958.46915216)
\curveto(26.34854326,958.75566913)(26.01364515,958.8989311)(25.61549182,958.89893849)
\curveto(25.29547477,958.8989311)(25.01453248,958.80962494)(24.77266409,958.63101974)
\curveto(24.45636897,958.40030503)(24.20705594,958.06354638)(24.02472424,957.62074278)
\curveto(23.84238911,957.17792696)(23.74750132,956.54720219)(23.74006057,955.72856661)
\curveto(23.98192999,956.0969503)(24.27775665,956.37045042)(24.62754143,956.54906778)
\curveto(24.97732158,956.72767506)(25.34384895,956.81698122)(25.72712464,956.81698653)
\curveto(26.39691743,956.81698122)(26.96717447,956.57045901)(27.43789749,956.07741915)
\curveto(27.90861025,955.58437015)(28.1439692,954.94713348)(28.14397503,954.16570723)
\curveto(28.1439692,953.65219415)(28.03326677,953.17496435)(27.81186741,952.7340164)
\curveto(27.59045706,952.29306601)(27.28625795,951.95537709)(26.89926917,951.72094863)
\curveto(26.51227122,951.48651974)(26.07318259,951.36930541)(25.58200198,951.36930527)
\curveto(24.74475345,951.36930541)(24.06193343,951.67722561)(23.53353986,952.29306679)
\curveto(23.00514386,952.90890641)(22.74094646,953.92383372)(22.74094689,955.33785176)
\curveto(22.74094646,956.9193112)(23.03305203,958.06912802)(23.61726447,958.78730568)
\curveto(24.1270525,959.41244152)(24.81359361,959.72501309)(25.67688987,959.7250213)
\curveto(26.32063508,959.72501309)(26.84809959,959.54454022)(27.25928499,959.18360216)
\curveto(27.67046049,958.82264875)(27.91698271,958.32402269)(27.99885237,957.68772247)
\closepath
\moveto(23.89634651,954.16012559)
\curveto(23.89634492,953.81406156)(23.96983645,953.48288455)(24.11682131,953.16659355)
\curveto(24.26380256,952.85029924)(24.46939279,952.60935866)(24.7335926,952.44377109)
\curveto(24.99778757,952.27818165)(25.27500878,952.1953874)(25.56525706,952.19538808)
\curveto(25.98945807,952.1953874)(26.35412489,952.36655754)(26.65925862,952.70889902)
\curveto(26.96438366,953.0512381)(27.11694835,953.51637436)(27.11695315,954.10430918)
\curveto(27.11694835,954.66991227)(26.9662442,955.1155128)(26.66484026,955.44111211)
\curveto(26.36342762,955.76670356)(25.98387643,955.92950125)(25.52618557,955.92950567)
\curveto(25.07220937,955.92950125)(24.68707655,955.76670356)(24.37078596,955.44111211)
\curveto(24.05449125,955.1155128)(23.89634492,954.68851772)(23.89634651,954.16012559)
\closepath
}
}
{
\newrgbcolor{curcolor}{0 0 0}
\pscustom[linestyle=none,fillstyle=solid,fillcolor=curcolor]
{
\newpath
\moveto(7.48074186,907.8418541)
\lineto(6.47604655,907.8418541)
\lineto(6.47604655,914.24399591)
\curveto(6.23417244,914.01328192)(5.91694951,913.78257434)(5.52437682,913.55187247)
\curveto(5.13179952,913.32115918)(4.77922624,913.14812849)(4.46665592,913.03277989)
\lineto(4.46665592,914.00398536)
\curveto(5.02853927,914.26817659)(5.51972315,914.58819033)(5.94020905,914.96402755)
\curveto(6.3606895,915.33985052)(6.6583767,915.70451735)(6.83327155,916.05802912)
\lineto(7.48074186,916.05802912)
\closepath
}
}
{
\newrgbcolor{curcolor}{0 0 0}
\pscustom[linestyle=none,fillstyle=solid,fillcolor=curcolor]
{
\newpath
\moveto(10.21016429,909.73403027)
\lineto(11.17578812,909.82333652)
\curveto(11.25765051,909.36936156)(11.4139363,909.04004509)(11.64464594,908.83538613)
\curveto(11.87535146,908.63072518)(12.17117812,908.52839521)(12.5321268,908.52839589)
\curveto(12.84097433,908.52839521)(13.11168363,908.59909592)(13.34425551,908.74049824)
\curveto(13.57681988,908.88189876)(13.76752574,909.07074408)(13.91637368,909.30703476)
\curveto(14.06521295,909.54332252)(14.18986946,909.86240599)(14.2903436,910.26428613)
\curveto(14.39080833,910.66616143)(14.44104304,911.07548134)(14.4410479,911.49224707)
\curveto(14.44104304,911.5368965)(14.4391825,911.60387612)(14.43546626,911.69318614)
\curveto(14.23452254,911.37316854)(13.96009215,911.11362251)(13.61217426,910.91454727)
\curveto(13.26424832,910.71546588)(12.88748795,910.61592672)(12.48189203,910.61592949)
\curveto(11.80465075,910.61592672)(11.23160288,910.86151866)(10.76274672,911.35270606)
\curveto(10.29388819,911.84388643)(10.05945952,912.4913561)(10.05946,913.295117)
\curveto(10.05945952,914.12491462)(10.30412119,914.79285029)(10.79344574,915.29892599)
\curveto(11.28276787,915.80498678)(11.89581746,916.0580209)(12.63259633,916.05802912)
\curveto(13.16470916,916.0580209)(13.65124168,915.91475893)(14.09219536,915.62824279)
\curveto(14.53314002,915.34171107)(14.86803812,914.93332144)(15.09689067,914.40307267)
\curveto(15.3257322,913.87281077)(15.44015572,913.10533595)(15.44016157,912.1006459)
\curveto(15.44015572,911.05501534)(15.32666247,910.22242145)(15.09968149,909.60286172)
\curveto(14.87268949,908.98329846)(14.53500056,908.5116503)(14.08661372,908.18791582)
\curveto(13.63821787,907.86418064)(13.1126139,907.70231322)(12.50980024,907.70231308)
\curveto(11.86976983,907.70231322)(11.34695668,907.87999527)(10.94135922,908.23535976)
\curveto(10.53575905,908.59072347)(10.29202765,909.0902798)(10.21016429,909.73403027)
\closepath
\moveto(14.32383344,913.34535177)
\curveto(14.32382871,913.92211522)(14.17033374,914.37980929)(13.86334809,914.71843536)
\curveto(13.55635389,915.05704768)(13.1870357,915.22635728)(12.75539242,915.22636466)
\curveto(12.30885845,915.22635728)(11.92000454,915.04402386)(11.58882953,914.67936388)
\curveto(11.25765051,914.31469022)(11.09206201,913.84211178)(11.09206351,913.26162716)
\curveto(11.09206201,912.74066913)(11.24927806,912.31739514)(11.56371215,911.99180391)
\curveto(11.87814228,911.66620438)(12.26606591,911.50340669)(12.72748422,911.50341035)
\curveto(13.19261733,911.50340669)(13.57495934,911.66620438)(13.87451137,911.99180391)
\curveto(14.17405483,912.31739514)(14.32382871,912.76857731)(14.32383344,913.34535177)
\closepath
}
}
{
\newrgbcolor{curcolor}{0 0 0}
\pscustom[linestyle=none,fillstyle=solid,fillcolor=curcolor]
{
\newpath
\moveto(17.96864494,912.2792584)
\curveto(17.55188084,912.43181866)(17.24303036,912.64950242)(17.0420926,912.93231036)
\curveto(16.84115264,913.21510811)(16.74068321,913.5537273)(16.740684,913.94816895)
\curveto(16.74068321,914.54353725)(16.95464589,915.04402386)(17.38257267,915.44963029)
\curveto(17.81049659,915.85522149)(18.37982337,916.0580209)(19.09055471,916.05802912)
\curveto(19.80500085,916.0580209)(20.37990927,915.85057013)(20.81528167,915.43567618)
\curveto(21.25064433,915.02076705)(21.4683281,914.51562908)(21.46833362,913.92026075)
\curveto(21.4683281,913.54070349)(21.36878894,913.21045675)(21.16971585,912.92951954)
\curveto(20.97063231,912.64857215)(20.66829374,912.43181866)(20.26269924,912.2792584)
\curveto(20.76504208,912.115526)(21.14738409,911.85132861)(21.40972639,911.48666543)
\curveto(21.67205778,911.12199496)(21.8032262,910.68662743)(21.80323206,910.18056152)
\curveto(21.8032262,909.48099426)(21.55577372,908.89306203)(21.06087385,908.41676308)
\curveto(20.56596377,907.94046298)(19.91477301,907.70231322)(19.10729963,907.70231308)
\curveto(18.29981993,907.70231322)(17.64862918,907.94139325)(17.15372541,908.4195539)
\curveto(16.65881923,908.89771339)(16.41136674,909.49401807)(16.4113672,910.20846973)
\curveto(16.41136674,910.74058324)(16.54625625,911.18618377)(16.81603615,911.54527266)
\curveto(17.08581431,911.90435414)(17.47001686,912.14901581)(17.96864494,912.2792584)
\closepath
\moveto(17.76770588,913.9816588)
\curveto(17.76770406,913.59465929)(17.89236057,913.27836664)(18.1416758,913.03277989)
\curveto(18.39098664,912.78718276)(18.71472147,912.66438678)(19.11288127,912.66439161)
\curveto(19.49987147,912.66438678)(19.8170944,912.78625248)(20.064551,913.02998907)
\curveto(20.31199937,913.27371528)(20.43572562,913.57233275)(20.4357301,913.92584239)
\curveto(20.43572562,914.29422422)(20.30827828,914.60400497)(20.05338772,914.85518556)
\curveto(19.79848895,915.10635212)(19.48126602,915.23193891)(19.10171799,915.2319463)
\curveto(18.71844256,915.23193891)(18.40028937,915.10914294)(18.14725744,914.86355802)
\curveto(17.89422112,914.61795905)(17.76770406,914.32399294)(17.76770588,913.9816588)
\closepath
\moveto(17.44397072,910.20288809)
\curveto(17.44396923,909.91636179)(17.51187912,909.63914058)(17.6477006,909.37122363)
\curveto(17.78351869,909.10330362)(17.98538783,908.89585285)(18.25330861,908.7488707)
\curveto(18.52122479,908.60188674)(18.80960927,908.52839521)(19.11846291,908.52839589)
\curveto(19.59848036,908.52839521)(19.99477645,908.68282044)(20.30735237,908.99167207)
\curveto(20.61991957,909.30052139)(20.77620535,909.69309639)(20.77621018,910.16939824)
\curveto(20.77620535,910.65313762)(20.61526821,911.0531548)(20.29339827,911.36945098)
\curveto(19.97151963,911.6857401)(19.56871164,911.84388643)(19.08497307,911.84389043)
\curveto(18.6123915,911.84388643)(18.22074677,911.68760065)(17.91003771,911.37503262)
\curveto(17.59932473,911.06245752)(17.44396923,910.67174307)(17.44397072,910.20288809)
\closepath
}
}
{
\newrgbcolor{curcolor}{0 0 0}
\pscustom[linestyle=none,fillstyle=solid,fillcolor=curcolor]
{
\newpath
\moveto(22.85257971,914.95286427)
\lineto(22.85257971,915.9184881)
\lineto(28.14955667,915.9184881)
\lineto(28.14955667,915.13705841)
\curveto(27.62859823,914.5826087)(27.11229699,913.84583287)(26.6006514,912.92672872)
\curveto(26.08899723,912.00761439)(25.69363141,911.06245752)(25.41455276,910.09125527)
\curveto(25.21361079,909.40657246)(25.08523319,908.65677281)(25.02941956,907.8418541)
\lineto(23.99681604,907.8418541)
\curveto(24.00797762,908.48560267)(24.13449468,909.26331049)(24.3763676,910.17497988)
\curveto(24.61823639,911.08664461)(24.96522803,911.96575213)(25.41734358,912.81230508)
\curveto(25.86945291,913.6588481)(26.3504038,914.37236711)(26.86019768,914.95286427)
\closepath
}
}
{
\newrgbcolor{curcolor}{0 0 0}
\pscustom[linestyle=none,fillstyle=solid,fillcolor=curcolor]
{
\newpath
\moveto(7.48074186,864.17486191)
\lineto(6.47604655,864.17486191)
\lineto(6.47604655,870.57700372)
\curveto(6.23417244,870.34628974)(5.91694951,870.11558215)(5.52437682,869.88488028)
\curveto(5.13179952,869.65416699)(4.77922624,869.4811363)(4.46665592,869.3657877)
\lineto(4.46665592,870.33699317)
\curveto(5.02853927,870.6011844)(5.51972315,870.92119815)(5.94020905,871.29703536)
\curveto(6.3606895,871.67285833)(6.6583767,872.03752516)(6.83327155,872.39103693)
\lineto(7.48074186,872.39103693)
\closepath
}
}
{
\newrgbcolor{curcolor}{0 0 0}
\pscustom[linestyle=none,fillstyle=solid,fillcolor=curcolor]
{
\newpath
\moveto(10.21016429,866.06703808)
\lineto(11.17578812,866.15634433)
\curveto(11.25765051,865.70236937)(11.4139363,865.3730529)(11.64464594,865.16839394)
\curveto(11.87535146,864.963733)(12.17117812,864.86140302)(12.5321268,864.86140371)
\curveto(12.84097433,864.86140302)(13.11168363,864.93210373)(13.34425551,865.07350605)
\curveto(13.57681988,865.21490657)(13.76752574,865.40375189)(13.91637368,865.64004258)
\curveto(14.06521295,865.87633033)(14.18986946,866.1954138)(14.2903436,866.59729395)
\curveto(14.39080833,866.99916925)(14.44104304,867.40848915)(14.4410479,867.82525489)
\curveto(14.44104304,867.86990432)(14.4391825,867.93688394)(14.43546626,868.02619395)
\curveto(14.23452254,867.70617635)(13.96009215,867.44663032)(13.61217426,867.24755508)
\curveto(13.26424832,867.04847369)(12.88748795,866.94893453)(12.48189203,866.94893731)
\curveto(11.80465075,866.94893453)(11.23160288,867.19452647)(10.76274672,867.68571387)
\curveto(10.29388819,868.17689424)(10.05945952,868.82436391)(10.05946,869.62812481)
\curveto(10.05945952,870.45792244)(10.30412119,871.1258581)(10.79344574,871.6319338)
\curveto(11.28276787,872.13799459)(11.89581746,872.39102871)(12.63259633,872.39103693)
\curveto(13.16470916,872.39102871)(13.65124168,872.24776675)(14.09219536,871.9612506)
\curveto(14.53314002,871.67471888)(14.86803812,871.26632925)(15.09689067,870.73608048)
\curveto(15.3257322,870.20581859)(15.44015572,869.43834377)(15.44016157,868.43365372)
\curveto(15.44015572,867.38802316)(15.32666247,866.55542926)(15.09968149,865.93586953)
\curveto(14.87268949,865.31630628)(14.53500056,864.84465812)(14.08661372,864.52092363)
\curveto(13.63821787,864.19718845)(13.1126139,864.03532103)(12.50980024,864.03532089)
\curveto(11.86976983,864.03532103)(11.34695668,864.21300308)(10.94135922,864.56836757)
\curveto(10.53575905,864.92373128)(10.29202765,865.42328762)(10.21016429,866.06703808)
\closepath
\moveto(14.32383344,869.67835958)
\curveto(14.32382871,870.25512303)(14.17033374,870.7128171)(13.86334809,871.05144317)
\curveto(13.55635389,871.39005549)(13.1870357,871.55936509)(12.75539242,871.55937247)
\curveto(12.30885845,871.55936509)(11.92000454,871.37703168)(11.58882953,871.01237169)
\curveto(11.25765051,870.64769803)(11.09206201,870.17511959)(11.09206351,869.59463497)
\curveto(11.09206201,869.07367694)(11.24927806,868.65040295)(11.56371215,868.32481172)
\curveto(11.87814228,867.99921219)(12.26606591,867.83641451)(12.72748422,867.83641817)
\curveto(13.19261733,867.83641451)(13.57495934,867.99921219)(13.87451137,868.32481172)
\curveto(14.17405483,868.65040295)(14.32382871,869.10158512)(14.32383344,869.67835958)
\closepath
}
}
{
\newrgbcolor{curcolor}{0 0 0}
\pscustom[linestyle=none,fillstyle=solid,fillcolor=curcolor]
{
\newpath
\moveto(17.96864494,868.61226622)
\curveto(17.55188084,868.76482647)(17.24303036,868.98251024)(17.0420926,869.26531817)
\curveto(16.84115264,869.54811592)(16.74068321,869.88673512)(16.740684,870.28117677)
\curveto(16.74068321,870.87654507)(16.95464589,871.37703168)(17.38257267,871.7826381)
\curveto(17.81049659,872.1882293)(18.37982337,872.39102871)(19.09055471,872.39103693)
\curveto(19.80500085,872.39102871)(20.37990927,872.18357794)(20.81528167,871.768684)
\curveto(21.25064433,871.35377486)(21.4683281,870.84863689)(21.46833362,870.25326856)
\curveto(21.4683281,869.8737113)(21.36878894,869.54346456)(21.16971585,869.26252735)
\curveto(20.97063231,868.98157996)(20.66829374,868.76482647)(20.26269924,868.61226622)
\curveto(20.76504208,868.44853382)(21.14738409,868.18433642)(21.40972639,867.81967325)
\curveto(21.67205778,867.45500278)(21.8032262,867.01963524)(21.80323206,866.51356934)
\curveto(21.8032262,865.81400207)(21.55577372,865.22606984)(21.06087385,864.74977089)
\curveto(20.56596377,864.27347079)(19.91477301,864.03532103)(19.10729963,864.03532089)
\curveto(18.29981993,864.03532103)(17.64862918,864.27440107)(17.15372541,864.75256171)
\curveto(16.65881923,865.23072121)(16.41136674,865.82702589)(16.4113672,866.54147754)
\curveto(16.41136674,867.07359105)(16.54625625,867.51919158)(16.81603615,867.87828047)
\curveto(17.08581431,868.23736196)(17.47001686,868.48202363)(17.96864494,868.61226622)
\closepath
\moveto(17.76770588,870.31466661)
\curveto(17.76770406,869.92766711)(17.89236057,869.61137445)(18.1416758,869.3657877)
\curveto(18.39098664,869.12019057)(18.71472147,868.9973946)(19.11288127,868.99739942)
\curveto(19.49987147,868.9973946)(19.8170944,869.1192603)(20.064551,869.36299688)
\curveto(20.31199937,869.60672309)(20.43572562,869.90534057)(20.4357301,870.2588502)
\curveto(20.43572562,870.62723203)(20.30827828,870.93701278)(20.05338772,871.18819337)
\curveto(19.79848895,871.43935993)(19.48126602,871.56494672)(19.10171799,871.56495411)
\curveto(18.71844256,871.56494672)(18.40028937,871.44215075)(18.14725744,871.19656583)
\curveto(17.89422112,870.95096687)(17.76770406,870.65700075)(17.76770588,870.31466661)
\closepath
\moveto(17.44397072,866.5358959)
\curveto(17.44396923,866.2493696)(17.51187912,865.9721484)(17.6477006,865.70423144)
\curveto(17.78351869,865.43631143)(17.98538783,865.22886066)(18.25330861,865.08187851)
\curveto(18.52122479,864.93489455)(18.80960927,864.86140302)(19.11846291,864.86140371)
\curveto(19.59848036,864.86140302)(19.99477645,865.01582826)(20.30735237,865.32467988)
\curveto(20.61991957,865.6335292)(20.77620535,866.0261042)(20.77621018,866.50240605)
\curveto(20.77620535,866.98614543)(20.61526821,867.38616261)(20.29339827,867.70245879)
\curveto(19.97151963,868.01874792)(19.56871164,868.17689424)(19.08497307,868.17689825)
\curveto(18.6123915,868.17689424)(18.22074677,868.02060846)(17.91003771,867.70804043)
\curveto(17.59932473,867.39546534)(17.44396923,867.00475088)(17.44397072,866.5358959)
\closepath
}
}
{
\newrgbcolor{curcolor}{0 0 0}
\pscustom[linestyle=none,fillstyle=solid,fillcolor=curcolor]
{
\newpath
\moveto(24.33171448,868.61226622)
\curveto(23.91495037,868.76482647)(23.6060999,868.98251024)(23.40516213,869.26531817)
\curveto(23.20422217,869.54811592)(23.10375274,869.88673512)(23.10375354,870.28117677)
\curveto(23.10375274,870.87654507)(23.31771542,871.37703168)(23.74564221,871.7826381)
\curveto(24.17356613,872.1882293)(24.7428929,872.39102871)(25.45362424,872.39103693)
\curveto(26.16807039,872.39102871)(26.7429788,872.18357794)(27.1783512,871.768684)
\curveto(27.61371387,871.35377486)(27.83139764,870.84863689)(27.83140316,870.25326856)
\curveto(27.83139764,869.8737113)(27.73185848,869.54346456)(27.53278538,869.26252735)
\curveto(27.33370184,868.98157996)(27.03136328,868.76482647)(26.62576878,868.61226622)
\curveto(27.12811162,868.44853382)(27.51045362,868.18433642)(27.77279593,867.81967325)
\curveto(28.03512732,867.45500278)(28.16629574,867.01963524)(28.16630159,866.51356934)
\curveto(28.16629574,865.81400207)(27.91884325,865.22606984)(27.42394339,864.74977089)
\curveto(26.9290333,864.27347079)(26.27784254,864.03532103)(25.47036917,864.03532089)
\curveto(24.66288947,864.03532103)(24.01169871,864.27440107)(23.51679494,864.75256171)
\curveto(23.02188876,865.23072121)(22.77443627,865.82702589)(22.77443674,866.54147754)
\curveto(22.77443627,867.07359105)(22.90932579,867.51919158)(23.17910568,867.87828047)
\curveto(23.44888384,868.23736196)(23.83308639,868.48202363)(24.33171448,868.61226622)
\closepath
\moveto(24.13077541,870.31466661)
\curveto(24.13077359,869.92766711)(24.25543011,869.61137445)(24.50474534,869.3657877)
\curveto(24.75405617,869.12019057)(25.07779101,868.9973946)(25.47595081,868.99739942)
\curveto(25.86294101,868.9973946)(26.18016393,869.1192603)(26.42762054,869.36299688)
\curveto(26.67506891,869.60672309)(26.79879515,869.90534057)(26.79879964,870.2588502)
\curveto(26.79879515,870.62723203)(26.67134782,870.93701278)(26.41645725,871.18819337)
\curveto(26.16155848,871.43935993)(25.84433556,871.56494672)(25.46478753,871.56495411)
\curveto(25.0815121,871.56494672)(24.7633589,871.44215075)(24.51032698,871.19656583)
\curveto(24.25729065,870.95096687)(24.13077359,870.65700075)(24.13077541,870.31466661)
\closepath
\moveto(23.80704026,866.5358959)
\curveto(23.80703876,866.2493696)(23.87494865,865.9721484)(24.01077014,865.70423144)
\curveto(24.14658823,865.43631143)(24.34845736,865.22886066)(24.61637815,865.08187851)
\curveto(24.88429433,864.93489455)(25.1726788,864.86140302)(25.48153245,864.86140371)
\curveto(25.96154989,864.86140302)(26.35784598,865.01582826)(26.6704219,865.32467988)
\curveto(26.98298911,865.6335292)(27.13927489,866.0261042)(27.13927972,866.50240605)
\curveto(27.13927489,866.98614543)(26.97833774,867.38616261)(26.6564678,867.70245879)
\curveto(26.33458917,868.01874792)(25.93178117,868.17689424)(25.4480426,868.17689825)
\curveto(24.97546103,868.17689424)(24.58381631,868.02060846)(24.27310725,867.70804043)
\curveto(23.96239427,867.39546534)(23.80703876,867.00475088)(23.80704026,866.5358959)
\closepath
}
}
{
\newrgbcolor{curcolor}{0 0 0}
\pscustom[linestyle=none,fillstyle=solid,fillcolor=curcolor]
{
\newpath
\moveto(7.48074186,820.50786972)
\lineto(6.47604655,820.50786972)
\lineto(6.47604655,826.91001153)
\curveto(6.23417244,826.67929755)(5.91694951,826.44858997)(5.52437682,826.21788809)
\curveto(5.13179952,825.9871748)(4.77922624,825.81414411)(4.46665592,825.69879551)
\lineto(4.46665592,826.67000099)
\curveto(5.02853927,826.93419222)(5.51972315,827.25420596)(5.94020905,827.63004318)
\curveto(6.3606895,828.00586615)(6.6583767,828.37053297)(6.83327155,828.72404474)
\lineto(7.48074186,828.72404474)
\closepath
}
}
{
\newrgbcolor{curcolor}{0 0 0}
\pscustom[linestyle=none,fillstyle=solid,fillcolor=curcolor]
{
\newpath
\moveto(10.21016429,822.4000459)
\lineto(11.17578812,822.48935215)
\curveto(11.25765051,822.03537718)(11.4139363,821.70606071)(11.64464594,821.50140175)
\curveto(11.87535146,821.29674081)(12.17117812,821.19441083)(12.5321268,821.19441152)
\curveto(12.84097433,821.19441083)(13.11168363,821.26511154)(13.34425551,821.40651386)
\curveto(13.57681988,821.54791439)(13.76752574,821.73675971)(13.91637368,821.97305039)
\curveto(14.06521295,822.20933814)(14.18986946,822.52842161)(14.2903436,822.93030176)
\curveto(14.39080833,823.33217706)(14.44104304,823.74149696)(14.4410479,824.1582627)
\curveto(14.44104304,824.20291213)(14.4391825,824.26989175)(14.43546626,824.35920176)
\curveto(14.23452254,824.03918417)(13.96009215,823.77963814)(13.61217426,823.58056289)
\curveto(13.26424832,823.3814815)(12.88748795,823.28194234)(12.48189203,823.28194512)
\curveto(11.80465075,823.28194234)(11.23160288,823.52753429)(10.76274672,824.01872168)
\curveto(10.29388819,824.50990206)(10.05945952,825.15737172)(10.05946,825.96113262)
\curveto(10.05945952,826.79093025)(10.30412119,827.45886591)(10.79344574,827.96494161)
\curveto(11.28276787,828.4710024)(11.89581746,828.72403652)(12.63259633,828.72404474)
\curveto(13.16470916,828.72403652)(13.65124168,828.58077456)(14.09219536,828.29425841)
\curveto(14.53314002,828.00772669)(14.86803812,827.59933706)(15.09689067,827.06908829)
\curveto(15.3257322,826.5388264)(15.44015572,825.77135158)(15.44016157,824.76666153)
\curveto(15.44015572,823.72103097)(15.32666247,822.88843707)(15.09968149,822.26887734)
\curveto(14.87268949,821.64931409)(14.53500056,821.17766593)(14.08661372,820.85393144)
\curveto(13.63821787,820.53019626)(13.1126139,820.36832884)(12.50980024,820.3683287)
\curveto(11.86976983,820.36832884)(11.34695668,820.54601089)(10.94135922,820.90137539)
\curveto(10.53575905,821.25673909)(10.29202765,821.75629543)(10.21016429,822.4000459)
\closepath
\moveto(14.32383344,826.01136739)
\curveto(14.32382871,826.58813084)(14.17033374,827.04582492)(13.86334809,827.38445099)
\curveto(13.55635389,827.7230633)(13.1870357,827.8923729)(12.75539242,827.89238029)
\curveto(12.30885845,827.8923729)(11.92000454,827.71003949)(11.58882953,827.3453795)
\curveto(11.25765051,826.98070584)(11.09206201,826.50812741)(11.09206351,825.92764278)
\curveto(11.09206201,825.40668476)(11.24927806,824.98341076)(11.56371215,824.65781954)
\curveto(11.87814228,824.33222001)(12.26606591,824.16942232)(12.72748422,824.16942598)
\curveto(13.19261733,824.16942232)(13.57495934,824.33222001)(13.87451137,824.65781954)
\curveto(14.17405483,824.98341076)(14.32382871,825.43459293)(14.32383344,826.01136739)
\closepath
}
}
{
\newrgbcolor{curcolor}{0 0 0}
\pscustom[linestyle=none,fillstyle=solid,fillcolor=curcolor]
{
\newpath
\moveto(17.96864494,824.94527403)
\curveto(17.55188084,825.09783428)(17.24303036,825.31551805)(17.0420926,825.59832598)
\curveto(16.84115264,825.88112374)(16.74068321,826.21974293)(16.740684,826.61418458)
\curveto(16.74068321,827.20955288)(16.95464589,827.71003949)(17.38257267,828.11564591)
\curveto(17.81049659,828.52123712)(18.37982337,828.72403652)(19.09055471,828.72404474)
\curveto(19.80500085,828.72403652)(20.37990927,828.51658575)(20.81528167,828.10169181)
\curveto(21.25064433,827.68678268)(21.4683281,827.1816447)(21.46833362,826.58627638)
\curveto(21.4683281,826.20671911)(21.36878894,825.87647237)(21.16971585,825.59553516)
\curveto(20.97063231,825.31458778)(20.66829374,825.09783428)(20.26269924,824.94527403)
\curveto(20.76504208,824.78154163)(21.14738409,824.51734424)(21.40972639,824.15268106)
\curveto(21.67205778,823.78801059)(21.8032262,823.35264305)(21.80323206,822.84657715)
\curveto(21.8032262,822.14700988)(21.55577372,821.55907766)(21.06087385,821.08277871)
\curveto(20.56596377,820.60647861)(19.91477301,820.36832884)(19.10729963,820.3683287)
\curveto(18.29981993,820.36832884)(17.64862918,820.60740888)(17.15372541,821.08556953)
\curveto(16.65881923,821.56372902)(16.41136674,822.1600337)(16.4113672,822.87448535)
\curveto(16.41136674,823.40659886)(16.54625625,823.85219939)(16.81603615,824.21128828)
\curveto(17.08581431,824.57036977)(17.47001686,824.81503144)(17.96864494,824.94527403)
\closepath
\moveto(17.76770588,826.64767442)
\curveto(17.76770406,826.26067492)(17.89236057,825.94438227)(18.1416758,825.69879551)
\curveto(18.39098664,825.45319838)(18.71472147,825.33040241)(19.11288127,825.33040723)
\curveto(19.49987147,825.33040241)(19.8170944,825.45226811)(20.064551,825.69600469)
\curveto(20.31199937,825.9397309)(20.43572562,826.23834838)(20.4357301,826.59185802)
\curveto(20.43572562,826.96023985)(20.30827828,827.27002059)(20.05338772,827.52120118)
\curveto(19.79848895,827.77236775)(19.48126602,827.89795454)(19.10171799,827.89796193)
\curveto(18.71844256,827.89795454)(18.40028937,827.77515856)(18.14725744,827.52957364)
\curveto(17.89422112,827.28397468)(17.76770406,826.99000857)(17.76770588,826.64767442)
\closepath
\moveto(17.44397072,822.86890371)
\curveto(17.44396923,822.58237742)(17.51187912,822.30515621)(17.6477006,822.03723926)
\curveto(17.78351869,821.76931924)(17.98538783,821.56186847)(18.25330861,821.41488632)
\curveto(18.52122479,821.26790236)(18.80960927,821.19441083)(19.11846291,821.19441152)
\curveto(19.59848036,821.19441083)(19.99477645,821.34883607)(20.30735237,821.65768769)
\curveto(20.61991957,821.96653702)(20.77620535,822.35911201)(20.77621018,822.83541387)
\curveto(20.77620535,823.31915324)(20.61526821,823.71917042)(20.29339827,824.0354666)
\curveto(19.97151963,824.35175573)(19.56871164,824.50990206)(19.08497307,824.50990606)
\curveto(18.6123915,824.50990206)(18.22074677,824.35361627)(17.91003771,824.04104824)
\curveto(17.59932473,823.72847315)(17.44396923,823.33775869)(17.44397072,822.86890371)
\closepath
}
}
{
\newrgbcolor{curcolor}{0 0 0}
\pscustom[linestyle=none,fillstyle=solid,fillcolor=curcolor]
{
\newpath
\moveto(22.93630432,822.4000459)
\lineto(23.90192815,822.48935215)
\curveto(23.98379054,822.03537718)(24.14007632,821.70606071)(24.37078596,821.50140175)
\curveto(24.60149148,821.29674081)(24.89731814,821.19441083)(25.25826682,821.19441152)
\curveto(25.56711435,821.19441083)(25.83782365,821.26511154)(26.07039553,821.40651386)
\curveto(26.3029599,821.54791439)(26.49366577,821.73675971)(26.6425137,821.97305039)
\curveto(26.79135297,822.20933814)(26.91600949,822.52842161)(27.01648362,822.93030176)
\curveto(27.11694835,823.33217706)(27.16718306,823.74149696)(27.16718792,824.1582627)
\curveto(27.16718306,824.20291213)(27.16532252,824.26989175)(27.16160628,824.35920176)
\curveto(26.96066257,824.03918417)(26.68623218,823.77963814)(26.33831429,823.58056289)
\curveto(25.99038834,823.3814815)(25.61362797,823.28194234)(25.20803206,823.28194512)
\curveto(24.53079077,823.28194234)(23.95774291,823.52753429)(23.48888674,824.01872168)
\curveto(23.02002822,824.50990206)(22.78559954,825.15737172)(22.78560002,825.96113262)
\curveto(22.78559954,826.79093025)(23.03026121,827.45886591)(23.51958576,827.96494161)
\curveto(24.00890789,828.4710024)(24.62195748,828.72403652)(25.35873635,828.72404474)
\curveto(25.89084918,828.72403652)(26.3773817,828.58077456)(26.81833538,828.29425841)
\curveto(27.25928004,828.00772669)(27.59417815,827.59933706)(27.82303069,827.06908829)
\curveto(28.05187222,826.5388264)(28.16629574,825.77135158)(28.16630159,824.76666153)
\curveto(28.16629574,823.72103097)(28.05280249,822.88843707)(27.82582152,822.26887734)
\curveto(27.59882951,821.64931409)(27.26114059,821.17766593)(26.81275374,820.85393144)
\curveto(26.36435789,820.53019626)(25.83875392,820.36832884)(25.23594026,820.3683287)
\curveto(24.59590985,820.36832884)(24.0730967,820.54601089)(23.66749924,820.90137539)
\curveto(23.26189907,821.25673909)(23.01816767,821.75629543)(22.93630432,822.4000459)
\closepath
\moveto(27.04997347,826.01136739)
\curveto(27.04996873,826.58813084)(26.89647376,827.04582492)(26.58948811,827.38445099)
\curveto(26.28249391,827.7230633)(25.91317572,827.8923729)(25.48153245,827.89238029)
\curveto(25.03499847,827.8923729)(24.64614456,827.71003949)(24.31496955,827.3453795)
\curveto(23.98379054,826.98070584)(23.81820203,826.50812741)(23.81820354,825.92764278)
\curveto(23.81820203,825.40668476)(23.97541808,824.98341076)(24.28985217,824.65781954)
\curveto(24.6042823,824.33222001)(24.99220594,824.16942232)(25.45362424,824.16942598)
\curveto(25.91875736,824.16942232)(26.30109936,824.33222001)(26.6006514,824.65781954)
\curveto(26.90019485,824.98341076)(27.04996873,825.43459293)(27.04997347,826.01136739)
\closepath
}
}
{
\newrgbcolor{curcolor}{0 0 0}
\pscustom[linestyle=none,fillstyle=solid,fillcolor=curcolor]
{
\newpath
\moveto(7.48074186,776.84087753)
\lineto(6.47604655,776.84087753)
\lineto(6.47604655,783.24301935)
\curveto(6.23417244,783.01230536)(5.91694951,782.78159778)(5.52437682,782.55089591)
\curveto(5.13179952,782.32018261)(4.77922624,782.14715193)(4.46665592,782.03180333)
\lineto(4.46665592,783.0030088)
\curveto(5.02853927,783.26720003)(5.51972315,783.58721377)(5.94020905,783.96305099)
\curveto(6.3606895,784.33887396)(6.6583767,784.70354078)(6.83327155,785.05705255)
\lineto(7.48074186,785.05705255)
\closepath
}
}
{
\newrgbcolor{curcolor}{0 0 0}
\pscustom[linestyle=none,fillstyle=solid,fillcolor=curcolor]
{
\newpath
\moveto(10.21016429,778.73305371)
\lineto(11.17578812,778.82235996)
\curveto(11.25765051,778.36838499)(11.4139363,778.03906853)(11.64464594,777.83440957)
\curveto(11.87535146,777.62974862)(12.17117812,777.52741865)(12.5321268,777.52741933)
\curveto(12.84097433,777.52741865)(13.11168363,777.59811936)(13.34425551,777.73952168)
\curveto(13.57681988,777.8809222)(13.76752574,778.06976752)(13.91637368,778.3060582)
\curveto(14.06521295,778.54234595)(14.18986946,778.86142942)(14.2903436,779.26330957)
\curveto(14.39080833,779.66518487)(14.44104304,780.07450478)(14.4410479,780.49127051)
\curveto(14.44104304,780.53591994)(14.4391825,780.60289956)(14.43546626,780.69220957)
\curveto(14.23452254,780.37219198)(13.96009215,780.11264595)(13.61217426,779.9135707)
\curveto(13.26424832,779.71448931)(12.88748795,779.61495016)(12.48189203,779.61495293)
\curveto(11.80465075,779.61495016)(11.23160288,779.8605421)(10.76274672,780.35172949)
\curveto(10.29388819,780.84290987)(10.05945952,781.49037954)(10.05946,782.29414044)
\curveto(10.05945952,783.12393806)(10.30412119,783.79187372)(10.79344574,784.29794943)
\curveto(11.28276787,784.80401021)(11.89581746,785.05704434)(12.63259633,785.05705255)
\curveto(13.16470916,785.05704434)(13.65124168,784.91378237)(14.09219536,784.62726622)
\curveto(14.53314002,784.3407345)(14.86803812,783.93234487)(15.09689067,783.4020961)
\curveto(15.3257322,782.87183421)(15.44015572,782.10435939)(15.44016157,781.09966934)
\curveto(15.44015572,780.05403878)(15.32666247,779.22144488)(15.09968149,778.60188515)
\curveto(14.87268949,777.9823219)(14.53500056,777.51067374)(14.08661372,777.18693925)
\curveto(13.63821787,776.86320407)(13.1126139,776.70133666)(12.50980024,776.70133652)
\curveto(11.86976983,776.70133666)(11.34695668,776.87901871)(10.94135922,777.2343832)
\curveto(10.53575905,777.5897469)(10.29202765,778.08930324)(10.21016429,778.73305371)
\closepath
\moveto(14.32383344,782.3443752)
\curveto(14.32382871,782.92113866)(14.17033374,783.37883273)(13.86334809,783.7174588)
\curveto(13.55635389,784.05607112)(13.1870357,784.22538071)(12.75539242,784.2253881)
\curveto(12.30885845,784.22538071)(11.92000454,784.0430473)(11.58882953,783.67838732)
\curveto(11.25765051,783.31371365)(11.09206201,782.84113522)(11.09206351,782.26065059)
\curveto(11.09206201,781.73969257)(11.24927806,781.31641858)(11.56371215,780.99082735)
\curveto(11.87814228,780.66522782)(12.26606591,780.50243013)(12.72748422,780.50243379)
\curveto(13.19261733,780.50243013)(13.57495934,780.66522782)(13.87451137,780.99082735)
\curveto(14.17405483,781.31641858)(14.32382871,781.76760074)(14.32383344,782.3443752)
\closepath
}
}
{
\newrgbcolor{curcolor}{0 0 0}
\pscustom[linestyle=none,fillstyle=solid,fillcolor=curcolor]
{
\newpath
\moveto(16.57323478,778.73305371)
\lineto(17.53885861,778.82235996)
\curveto(17.620721,778.36838499)(17.77700678,778.03906853)(18.00771643,777.83440957)
\curveto(18.23842195,777.62974862)(18.53424861,777.52741865)(18.89519729,777.52741933)
\curveto(19.20404481,777.52741865)(19.47475411,777.59811936)(19.707326,777.73952168)
\curveto(19.93989037,777.8809222)(20.13059623,778.06976752)(20.27944417,778.3060582)
\curveto(20.42828344,778.54234595)(20.55293995,778.86142942)(20.65341409,779.26330957)
\curveto(20.75387881,779.66518487)(20.80411353,780.07450478)(20.80411839,780.49127051)
\curveto(20.80411353,780.53591994)(20.80225298,780.60289956)(20.79853674,780.69220957)
\curveto(20.59759303,780.37219198)(20.32316264,780.11264595)(19.97524475,779.9135707)
\curveto(19.62731881,779.71448931)(19.25055844,779.61495016)(18.84496252,779.61495293)
\curveto(18.16772124,779.61495016)(17.59467337,779.8605421)(17.1258172,780.35172949)
\curveto(16.65695868,780.84290987)(16.42253001,781.49037954)(16.42253048,782.29414044)
\curveto(16.42253001,783.12393806)(16.66719168,783.79187372)(17.15651623,784.29794943)
\curveto(17.64583836,784.80401021)(18.25888794,785.05704434)(18.99566682,785.05705255)
\curveto(19.52777965,785.05704434)(20.01431217,784.91378237)(20.45526585,784.62726622)
\curveto(20.89621051,784.3407345)(21.23110861,783.93234487)(21.45996116,783.4020961)
\curveto(21.68880269,782.87183421)(21.8032262,782.10435939)(21.80323206,781.09966934)
\curveto(21.8032262,780.05403878)(21.68973296,779.22144488)(21.46275198,778.60188515)
\curveto(21.23575997,777.9823219)(20.89807105,777.51067374)(20.4496842,777.18693925)
\curveto(20.00128835,776.86320407)(19.47568439,776.70133666)(18.87287072,776.70133652)
\curveto(18.23284031,776.70133666)(17.71002716,776.87901871)(17.30442971,777.2343832)
\curveto(16.89882953,777.5897469)(16.65509814,778.08930324)(16.57323478,778.73305371)
\closepath
\moveto(20.68690393,782.3443752)
\curveto(20.68689919,782.92113866)(20.53340423,783.37883273)(20.22641858,783.7174588)
\curveto(19.91942437,784.05607112)(19.55010619,784.22538071)(19.11846291,784.2253881)
\curveto(18.67192894,784.22538071)(18.28307503,784.0430473)(17.95190002,783.67838732)
\curveto(17.620721,783.31371365)(17.4551325,782.84113522)(17.455134,782.26065059)
\curveto(17.4551325,781.73969257)(17.61234855,781.31641858)(17.92678264,780.99082735)
\curveto(18.24121277,780.66522782)(18.6291364,780.50243013)(19.09055471,780.50243379)
\curveto(19.55568782,780.50243013)(19.93802982,780.66522782)(20.23758186,780.99082735)
\curveto(20.53712532,781.31641858)(20.68689919,781.76760074)(20.68690393,782.3443752)
\closepath
}
}
{
\newrgbcolor{curcolor}{0 0 0}
\pscustom[linestyle=none,fillstyle=solid,fillcolor=curcolor]
{
\newpath
\moveto(22.78560002,780.87640371)
\curveto(22.78559954,781.84388309)(22.8851387,782.62252118)(23.08421779,783.21232032)
\curveto(23.28329534,783.80210672)(23.57912199,784.25700998)(23.97169865,784.57703146)
\curveto(24.36427199,784.89703747)(24.8582467,785.05704434)(25.45362424,785.05705255)
\curveto(25.89270973,785.05704434)(26.27784254,784.96866845)(26.60902386,784.79192462)
\curveto(26.94019657,784.61516489)(27.21369669,784.36027023)(27.42952503,784.02723986)
\curveto(27.64534313,783.69419511)(27.81465273,783.2885963)(27.93745433,782.8104422)
\curveto(28.06024467,782.33227616)(28.12164266,781.68759731)(28.12164847,780.87640371)
\curveto(28.12164266,779.91635845)(28.02303377,779.14144145)(27.82582152,778.55165039)
\curveto(27.62859823,777.96185591)(27.33370184,777.50602238)(26.94113147,777.18414843)
\curveto(26.54855185,776.8622738)(26.0527166,776.70133666)(25.45362424,776.70133652)
\curveto(24.66475001,776.70133666)(24.04518852,776.9841395)(23.59493791,777.54974589)
\curveto(23.05537857,778.23070466)(22.78559954,779.33958949)(22.78560002,780.87640371)
\closepath
\moveto(23.81820354,780.87640371)
\curveto(23.81820203,779.53308618)(23.97541808,778.63909429)(24.28985217,778.19442539)
\curveto(24.6042823,777.74975377)(24.99220594,777.52741865)(25.45362424,777.52741933)
\curveto(25.91503627,777.52741865)(26.3029599,777.75068405)(26.61739632,778.19721621)
\curveto(26.93182412,778.64374566)(27.08904017,779.53680727)(27.08904495,780.87640371)
\curveto(27.08904017,782.22343427)(26.93182412,783.11835643)(26.61739632,783.56117286)
\curveto(26.3029599,784.00397586)(25.91131518,784.22538071)(25.44246096,784.2253881)
\curveto(24.98104267,784.22538071)(24.61265475,784.03002349)(24.33729612,783.63931583)
\curveto(23.99123272,783.14068297)(23.81820203,782.21971318)(23.81820354,780.87640371)
\closepath
}
}
{
\newrgbcolor{curcolor}{0 0 0}
\pscustom[linestyle=none,fillstyle=solid,fillcolor=curcolor]
{
\newpath
\moveto(7.48074186,733.17388535)
\lineto(6.47604655,733.17388535)
\lineto(6.47604655,739.57602716)
\curveto(6.23417244,739.34531317)(5.91694951,739.11460559)(5.52437682,738.88390372)
\curveto(5.13179952,738.65319043)(4.77922624,738.48015974)(4.46665592,738.36481114)
\lineto(4.46665592,739.33601661)
\curveto(5.02853927,739.60020784)(5.51972315,739.92022158)(5.94020905,740.2960588)
\curveto(6.3606895,740.67188177)(6.6583767,741.0365486)(6.83327155,741.39006037)
\lineto(7.48074186,741.39006037)
\closepath
}
}
{
\newrgbcolor{curcolor}{0 0 0}
\pscustom[linestyle=none,fillstyle=solid,fillcolor=curcolor]
{
\newpath
\moveto(10.21016429,735.06606152)
\lineto(11.17578812,735.15536777)
\curveto(11.25765051,734.70139281)(11.4139363,734.37207634)(11.64464594,734.16741738)
\curveto(11.87535146,733.96275643)(12.17117812,733.86042646)(12.5321268,733.86042714)
\curveto(12.84097433,733.86042646)(13.11168363,733.93112717)(13.34425551,734.07252949)
\curveto(13.57681988,734.21393001)(13.76752574,734.40277533)(13.91637368,734.63906601)
\curveto(14.06521295,734.87535377)(14.18986946,735.19443724)(14.2903436,735.59631738)
\curveto(14.39080833,735.99819268)(14.44104304,736.40751259)(14.4410479,736.82427832)
\curveto(14.44104304,736.86892775)(14.4391825,736.93590737)(14.43546626,737.02521739)
\curveto(14.23452254,736.70519979)(13.96009215,736.44565376)(13.61217426,736.24657852)
\curveto(13.26424832,736.04749713)(12.88748795,735.94795797)(12.48189203,735.94796074)
\curveto(11.80465075,735.94795797)(11.23160288,736.19354991)(10.76274672,736.68473731)
\curveto(10.29388819,737.17591768)(10.05945952,737.82338735)(10.05946,738.62714825)
\curveto(10.05945952,739.45694587)(10.30412119,740.12488154)(10.79344574,740.63095724)
\curveto(11.28276787,741.13701803)(11.89581746,741.39005215)(12.63259633,741.39006037)
\curveto(13.16470916,741.39005215)(13.65124168,741.24679018)(14.09219536,740.96027404)
\curveto(14.53314002,740.67374232)(14.86803812,740.26535269)(15.09689067,739.73510392)
\curveto(15.3257322,739.20484202)(15.44015572,738.4373672)(15.44016157,737.43267715)
\curveto(15.44015572,736.38704659)(15.32666247,735.5544527)(15.09968149,734.93489297)
\curveto(14.87268949,734.31532971)(14.53500056,733.84368155)(14.08661372,733.51994707)
\curveto(13.63821787,733.19621189)(13.1126139,733.03434447)(12.50980024,733.03434433)
\curveto(11.86976983,733.03434447)(11.34695668,733.21202652)(10.94135922,733.56739101)
\curveto(10.53575905,733.92275472)(10.29202765,734.42231105)(10.21016429,735.06606152)
\closepath
\moveto(14.32383344,738.67738302)
\curveto(14.32382871,739.25414647)(14.17033374,739.71184054)(13.86334809,740.05046661)
\curveto(13.55635389,740.38907893)(13.1870357,740.55838853)(12.75539242,740.55839591)
\curveto(12.30885845,740.55838853)(11.92000454,740.37605511)(11.58882953,740.01139513)
\curveto(11.25765051,739.64672147)(11.09206201,739.17414303)(11.09206351,738.59365841)
\curveto(11.09206201,738.07270038)(11.24927806,737.64942639)(11.56371215,737.32383516)
\curveto(11.87814228,736.99823563)(12.26606591,736.83543794)(12.72748422,736.8354416)
\curveto(13.19261733,736.83543794)(13.57495934,736.99823563)(13.87451137,737.32383516)
\curveto(14.17405483,737.64942639)(14.32382871,738.10060856)(14.32383344,738.67738302)
\closepath
}
}
{
\newrgbcolor{curcolor}{0 0 0}
\pscustom[linestyle=none,fillstyle=solid,fillcolor=curcolor]
{
\newpath
\moveto(16.57323478,735.06606152)
\lineto(17.53885861,735.15536777)
\curveto(17.620721,734.70139281)(17.77700678,734.37207634)(18.00771643,734.16741738)
\curveto(18.23842195,733.96275643)(18.53424861,733.86042646)(18.89519729,733.86042714)
\curveto(19.20404481,733.86042646)(19.47475411,733.93112717)(19.707326,734.07252949)
\curveto(19.93989037,734.21393001)(20.13059623,734.40277533)(20.27944417,734.63906601)
\curveto(20.42828344,734.87535377)(20.55293995,735.19443724)(20.65341409,735.59631738)
\curveto(20.75387881,735.99819268)(20.80411353,736.40751259)(20.80411839,736.82427832)
\curveto(20.80411353,736.86892775)(20.80225298,736.93590737)(20.79853674,737.02521739)
\curveto(20.59759303,736.70519979)(20.32316264,736.44565376)(19.97524475,736.24657852)
\curveto(19.62731881,736.04749713)(19.25055844,735.94795797)(18.84496252,735.94796074)
\curveto(18.16772124,735.94795797)(17.59467337,736.19354991)(17.1258172,736.68473731)
\curveto(16.65695868,737.17591768)(16.42253001,737.82338735)(16.42253048,738.62714825)
\curveto(16.42253001,739.45694587)(16.66719168,740.12488154)(17.15651623,740.63095724)
\curveto(17.64583836,741.13701803)(18.25888794,741.39005215)(18.99566682,741.39006037)
\curveto(19.52777965,741.39005215)(20.01431217,741.24679018)(20.45526585,740.96027404)
\curveto(20.89621051,740.67374232)(21.23110861,740.26535269)(21.45996116,739.73510392)
\curveto(21.68880269,739.20484202)(21.8032262,738.4373672)(21.80323206,737.43267715)
\curveto(21.8032262,736.38704659)(21.68973296,735.5544527)(21.46275198,734.93489297)
\curveto(21.23575997,734.31532971)(20.89807105,733.84368155)(20.4496842,733.51994707)
\curveto(20.00128835,733.19621189)(19.47568439,733.03434447)(18.87287072,733.03434433)
\curveto(18.23284031,733.03434447)(17.71002716,733.21202652)(17.30442971,733.56739101)
\curveto(16.89882953,733.92275472)(16.65509814,734.42231105)(16.57323478,735.06606152)
\closepath
\moveto(20.68690393,738.67738302)
\curveto(20.68689919,739.25414647)(20.53340423,739.71184054)(20.22641858,740.05046661)
\curveto(19.91942437,740.38907893)(19.55010619,740.55838853)(19.11846291,740.55839591)
\curveto(18.67192894,740.55838853)(18.28307503,740.37605511)(17.95190002,740.01139513)
\curveto(17.620721,739.64672147)(17.4551325,739.17414303)(17.455134,738.59365841)
\curveto(17.4551325,738.07270038)(17.61234855,737.64942639)(17.92678264,737.32383516)
\curveto(18.24121277,736.99823563)(18.6291364,736.83543794)(19.09055471,736.8354416)
\curveto(19.55568782,736.83543794)(19.93802982,736.99823563)(20.23758186,737.32383516)
\curveto(20.53712532,737.64942639)(20.68689919,738.10060856)(20.68690393,738.67738302)
\closepath
}
}
{
\newrgbcolor{curcolor}{0 0 0}
\pscustom[linestyle=none,fillstyle=solid,fillcolor=curcolor]
{
\newpath
\moveto(26.56995237,733.17388535)
\lineto(25.56525706,733.17388535)
\lineto(25.56525706,739.57602716)
\curveto(25.32338295,739.34531317)(25.00616002,739.11460559)(24.61358733,738.88390372)
\curveto(24.22101003,738.65319043)(23.86843675,738.48015974)(23.55586643,738.36481114)
\lineto(23.55586643,739.33601661)
\curveto(24.11774978,739.60020784)(24.60893366,739.92022158)(25.02941956,740.2960588)
\curveto(25.44990001,740.67188177)(25.74758721,741.0365486)(25.92248206,741.39006037)
\lineto(26.56995237,741.39006037)
\closepath
}
}
{
\newrgbcolor{curcolor}{0 0 0}
\pscustom[linestyle=none,fillstyle=solid,fillcolor=curcolor]
{
\newpath
\moveto(7.48074186,689.50689316)
\lineto(6.47604655,689.50689316)
\lineto(6.47604655,695.90903497)
\curveto(6.23417244,695.67832099)(5.91694951,695.4476134)(5.52437682,695.21691153)
\curveto(5.13179952,694.98619824)(4.77922624,694.81316755)(4.46665592,694.69781895)
\lineto(4.46665592,695.66902442)
\curveto(5.02853927,695.93321565)(5.51972315,696.2532294)(5.94020905,696.62906661)
\curveto(6.3606895,697.00488958)(6.6583767,697.36955641)(6.83327155,697.72306818)
\lineto(7.48074186,697.72306818)
\closepath
}
}
{
\newrgbcolor{curcolor}{0 0 0}
\pscustom[linestyle=none,fillstyle=solid,fillcolor=curcolor]
{
\newpath
\moveto(10.21016429,691.39906933)
\lineto(11.17578812,691.48837558)
\curveto(11.25765051,691.03440062)(11.4139363,690.70508415)(11.64464594,690.50042519)
\curveto(11.87535146,690.29576425)(12.17117812,690.19343427)(12.5321268,690.19343496)
\curveto(12.84097433,690.19343427)(13.11168363,690.26413498)(13.34425551,690.4055373)
\curveto(13.57681988,690.54693782)(13.76752574,690.73578314)(13.91637368,690.97207383)
\curveto(14.06521295,691.20836158)(14.18986946,691.52744505)(14.2903436,691.9293252)
\curveto(14.39080833,692.3312005)(14.44104304,692.7405204)(14.4410479,693.15728614)
\curveto(14.44104304,693.20193557)(14.4391825,693.26891519)(14.43546626,693.3582252)
\curveto(14.23452254,693.0382076)(13.96009215,692.77866157)(13.61217426,692.57958633)
\curveto(13.26424832,692.38050494)(12.88748795,692.28096578)(12.48189203,692.28096856)
\curveto(11.80465075,692.28096578)(11.23160288,692.52655772)(10.76274672,693.01774512)
\curveto(10.29388819,693.50892549)(10.05945952,694.15639516)(10.05946,694.96015606)
\curveto(10.05945952,695.78995369)(10.30412119,696.45788935)(10.79344574,696.96396505)
\curveto(11.28276787,697.47002584)(11.89581746,697.72305996)(12.63259633,697.72306818)
\curveto(13.16470916,697.72305996)(13.65124168,697.579798)(14.09219536,697.29328185)
\curveto(14.53314002,697.00675013)(14.86803812,696.5983605)(15.09689067,696.06811173)
\curveto(15.3257322,695.53784984)(15.44015572,694.77037502)(15.44016157,693.76568497)
\curveto(15.44015572,692.72005441)(15.32666247,691.88746051)(15.09968149,691.26790078)
\curveto(14.87268949,690.64833753)(14.53500056,690.17668937)(14.08661372,689.85295488)
\curveto(13.63821787,689.5292197)(13.1126139,689.36735228)(12.50980024,689.36735214)
\curveto(11.86976983,689.36735228)(11.34695668,689.54503433)(10.94135922,689.90039882)
\curveto(10.53575905,690.25576253)(10.29202765,690.75531887)(10.21016429,691.39906933)
\closepath
\moveto(14.32383344,695.01039083)
\curveto(14.32382871,695.58715428)(14.17033374,696.04484835)(13.86334809,696.38347442)
\curveto(13.55635389,696.72208674)(13.1870357,696.89139634)(12.75539242,696.89140372)
\curveto(12.30885845,696.89139634)(11.92000454,696.70906293)(11.58882953,696.34440294)
\curveto(11.25765051,695.97972928)(11.09206201,695.50715084)(11.09206351,694.92666622)
\curveto(11.09206201,694.40570819)(11.24927806,693.9824342)(11.56371215,693.65684297)
\curveto(11.87814228,693.33124344)(12.26606591,693.16844576)(12.72748422,693.16844942)
\curveto(13.19261733,693.16844576)(13.57495934,693.33124344)(13.87451137,693.65684297)
\curveto(14.17405483,693.9824342)(14.32382871,694.43361637)(14.32383344,695.01039083)
\closepath
}
}
{
\newrgbcolor{curcolor}{0 0 0}
\pscustom[linestyle=none,fillstyle=solid,fillcolor=curcolor]
{
\newpath
\moveto(16.57323478,691.39906933)
\lineto(17.53885861,691.48837558)
\curveto(17.620721,691.03440062)(17.77700678,690.70508415)(18.00771643,690.50042519)
\curveto(18.23842195,690.29576425)(18.53424861,690.19343427)(18.89519729,690.19343496)
\curveto(19.20404481,690.19343427)(19.47475411,690.26413498)(19.707326,690.4055373)
\curveto(19.93989037,690.54693782)(20.13059623,690.73578314)(20.27944417,690.97207383)
\curveto(20.42828344,691.20836158)(20.55293995,691.52744505)(20.65341409,691.9293252)
\curveto(20.75387881,692.3312005)(20.80411353,692.7405204)(20.80411839,693.15728614)
\curveto(20.80411353,693.20193557)(20.80225298,693.26891519)(20.79853674,693.3582252)
\curveto(20.59759303,693.0382076)(20.32316264,692.77866157)(19.97524475,692.57958633)
\curveto(19.62731881,692.38050494)(19.25055844,692.28096578)(18.84496252,692.28096856)
\curveto(18.16772124,692.28096578)(17.59467337,692.52655772)(17.1258172,693.01774512)
\curveto(16.65695868,693.50892549)(16.42253001,694.15639516)(16.42253048,694.96015606)
\curveto(16.42253001,695.78995369)(16.66719168,696.45788935)(17.15651623,696.96396505)
\curveto(17.64583836,697.47002584)(18.25888794,697.72305996)(18.99566682,697.72306818)
\curveto(19.52777965,697.72305996)(20.01431217,697.579798)(20.45526585,697.29328185)
\curveto(20.89621051,697.00675013)(21.23110861,696.5983605)(21.45996116,696.06811173)
\curveto(21.68880269,695.53784984)(21.8032262,694.77037502)(21.80323206,693.76568497)
\curveto(21.8032262,692.72005441)(21.68973296,691.88746051)(21.46275198,691.26790078)
\curveto(21.23575997,690.64833753)(20.89807105,690.17668937)(20.4496842,689.85295488)
\curveto(20.00128835,689.5292197)(19.47568439,689.36735228)(18.87287072,689.36735214)
\curveto(18.23284031,689.36735228)(17.71002716,689.54503433)(17.30442971,689.90039882)
\curveto(16.89882953,690.25576253)(16.65509814,690.75531887)(16.57323478,691.39906933)
\closepath
\moveto(20.68690393,695.01039083)
\curveto(20.68689919,695.58715428)(20.53340423,696.04484835)(20.22641858,696.38347442)
\curveto(19.91942437,696.72208674)(19.55010619,696.89139634)(19.11846291,696.89140372)
\curveto(18.67192894,696.89139634)(18.28307503,696.70906293)(17.95190002,696.34440294)
\curveto(17.620721,695.97972928)(17.4551325,695.50715084)(17.455134,694.92666622)
\curveto(17.4551325,694.40570819)(17.61234855,693.9824342)(17.92678264,693.65684297)
\curveto(18.24121277,693.33124344)(18.6291364,693.16844576)(19.09055471,693.16844942)
\curveto(19.55568782,693.16844576)(19.93802982,693.33124344)(20.23758186,693.65684297)
\curveto(20.53712532,693.9824342)(20.68689919,694.43361637)(20.68690393,695.01039083)
\closepath
}
}
{
\newrgbcolor{curcolor}{0 0 0}
\pscustom[linestyle=none,fillstyle=solid,fillcolor=curcolor]
{
\newpath
\moveto(28.06583206,690.47251699)
\lineto(28.06583206,689.50689316)
\lineto(22.65722228,689.50689316)
\curveto(22.64977976,689.74876401)(22.6888512,689.98133214)(22.77443674,690.20459824)
\curveto(22.91211661,690.57298545)(23.13259119,690.93579173)(23.43586115,691.29301816)
\curveto(23.73912887,691.65024102)(24.17728722,692.06328201)(24.75033752,692.53214238)
\curveto(25.6396756,693.26147301)(26.24063164,693.83917223)(26.55320745,694.2652418)
\curveto(26.86577477,694.69130185)(27.02206055,695.09410985)(27.02206526,695.473667)
\curveto(27.02206055,695.87181767)(26.87972886,696.20764604)(26.59506975,696.48115314)
\curveto(26.31040208,696.75464628)(25.93922335,696.89139634)(25.48153245,696.89140372)
\curveto(24.99778757,696.89139634)(24.61079421,696.74627383)(24.32055119,696.45603575)
\curveto(24.03030416,696.16578378)(23.88332111,695.76390606)(23.87960158,695.25040138)
\lineto(22.84699807,695.35645255)
\curveto(22.91769824,696.12671234)(23.18375618,696.71371429)(23.64517268,697.11746017)
\curveto(24.10658651,697.52119083)(24.726148,697.72305996)(25.50385901,697.72306818)
\curveto(26.28900581,697.72305996)(26.91042785,697.50537619)(27.36812698,697.07001622)
\curveto(27.825816,696.63464113)(28.05466304,696.09508307)(28.05466878,695.45134044)
\curveto(28.05466304,695.12387857)(27.98768342,694.80200428)(27.85372972,694.48571661)
\curveto(27.71976493,694.16941898)(27.4974298,693.83638142)(27.18672366,693.48660293)
\curveto(26.87600777,693.13681649)(26.35970653,692.65679588)(25.63781838,692.04653965)
\curveto(25.03499847,691.54046886)(24.64800511,691.19719831)(24.47683713,691.01672695)
\curveto(24.30566482,690.83625257)(24.1642634,690.65484943)(24.05263244,690.47251699)
\closepath
}
}
{
\newrgbcolor{curcolor}{0 0 0}
\pscustom[linestyle=none,fillstyle=solid,fillcolor=curcolor]
{
\newpath
\moveto(7.48074186,645.83684921)
\lineto(6.47604655,645.83684921)
\lineto(6.47604655,652.23899103)
\curveto(6.23417244,652.00827704)(5.91694951,651.77756946)(5.52437682,651.54686759)
\curveto(5.13179952,651.31615429)(4.77922624,651.14312361)(4.46665592,651.02777501)
\lineto(4.46665592,651.99898048)
\curveto(5.02853927,652.26317171)(5.51972315,652.58318545)(5.94020905,652.95902267)
\curveto(6.3606895,653.33484564)(6.6583767,653.69951246)(6.83327155,654.05302423)
\lineto(7.48074186,654.05302423)
\closepath
}
}
{
\newrgbcolor{curcolor}{0 0 0}
\pscustom[linestyle=none,fillstyle=solid,fillcolor=curcolor]
{
\newpath
\moveto(10.21016429,647.72902539)
\lineto(11.17578812,647.81833164)
\curveto(11.25765051,647.36435667)(11.4139363,647.03504021)(11.64464594,646.83038125)
\curveto(11.87535146,646.6257203)(12.17117812,646.52339032)(12.5321268,646.52339101)
\curveto(12.84097433,646.52339032)(13.11168363,646.59409104)(13.34425551,646.73549336)
\curveto(13.57681988,646.87689388)(13.76752574,647.0657392)(13.91637368,647.30202988)
\curveto(14.06521295,647.53831763)(14.18986946,647.8574011)(14.2903436,648.25928125)
\curveto(14.39080833,648.66115655)(14.44104304,649.07047646)(14.4410479,649.48724219)
\curveto(14.44104304,649.53189162)(14.4391825,649.59887124)(14.43546626,649.68818125)
\curveto(14.23452254,649.36816366)(13.96009215,649.10861763)(13.61217426,648.90954238)
\curveto(13.26424832,648.71046099)(12.88748795,648.61092184)(12.48189203,648.61092461)
\curveto(11.80465075,648.61092184)(11.23160288,648.85651378)(10.76274672,649.34770117)
\curveto(10.29388819,649.83888155)(10.05945952,650.48635122)(10.05946,651.29011212)
\curveto(10.05945952,652.11990974)(10.30412119,652.7878454)(10.79344574,653.29392111)
\curveto(11.28276787,653.79998189)(11.89581746,654.05301602)(12.63259633,654.05302423)
\curveto(13.16470916,654.05301602)(13.65124168,653.90975405)(14.09219536,653.6232379)
\curveto(14.53314002,653.33670618)(14.86803812,652.92831655)(15.09689067,652.39806778)
\curveto(15.3257322,651.86780589)(15.44015572,651.10033107)(15.44016157,650.09564102)
\curveto(15.44015572,649.05001046)(15.32666247,648.21741656)(15.09968149,647.59785683)
\curveto(14.87268949,646.97829358)(14.53500056,646.50664542)(14.08661372,646.18291093)
\curveto(13.63821787,645.85917575)(13.1126139,645.69730834)(12.50980024,645.6973082)
\curveto(11.86976983,645.69730834)(11.34695668,645.87499039)(10.94135922,646.23035488)
\curveto(10.53575905,646.58571858)(10.29202765,647.08527492)(10.21016429,647.72902539)
\closepath
\moveto(14.32383344,651.34034688)
\curveto(14.32382871,651.91711033)(14.17033374,652.37480441)(13.86334809,652.71343048)
\curveto(13.55635389,653.0520428)(13.1870357,653.22135239)(12.75539242,653.22135978)
\curveto(12.30885845,653.22135239)(11.92000454,653.03901898)(11.58882953,652.67435899)
\curveto(11.25765051,652.30968533)(11.09206201,651.8371069)(11.09206351,651.25662227)
\curveto(11.09206201,650.73566425)(11.24927806,650.31239026)(11.56371215,649.98679903)
\curveto(11.87814228,649.6611995)(12.26606591,649.49840181)(12.72748422,649.49840547)
\curveto(13.19261733,649.49840181)(13.57495934,649.6611995)(13.87451137,649.98679903)
\curveto(14.17405483,650.31239026)(14.32382871,650.76357242)(14.32383344,651.34034688)
\closepath
}
}
{
\newrgbcolor{curcolor}{0 0 0}
\pscustom[linestyle=none,fillstyle=solid,fillcolor=curcolor]
{
\newpath
\moveto(16.57323478,647.72902539)
\lineto(17.53885861,647.81833164)
\curveto(17.620721,647.36435667)(17.77700678,647.03504021)(18.00771643,646.83038125)
\curveto(18.23842195,646.6257203)(18.53424861,646.52339032)(18.89519729,646.52339101)
\curveto(19.20404481,646.52339032)(19.47475411,646.59409104)(19.707326,646.73549336)
\curveto(19.93989037,646.87689388)(20.13059623,647.0657392)(20.27944417,647.30202988)
\curveto(20.42828344,647.53831763)(20.55293995,647.8574011)(20.65341409,648.25928125)
\curveto(20.75387881,648.66115655)(20.80411353,649.07047646)(20.80411839,649.48724219)
\curveto(20.80411353,649.53189162)(20.80225298,649.59887124)(20.79853674,649.68818125)
\curveto(20.59759303,649.36816366)(20.32316264,649.10861763)(19.97524475,648.90954238)
\curveto(19.62731881,648.71046099)(19.25055844,648.61092184)(18.84496252,648.61092461)
\curveto(18.16772124,648.61092184)(17.59467337,648.85651378)(17.1258172,649.34770117)
\curveto(16.65695868,649.83888155)(16.42253001,650.48635122)(16.42253048,651.29011212)
\curveto(16.42253001,652.11990974)(16.66719168,652.7878454)(17.15651623,653.29392111)
\curveto(17.64583836,653.79998189)(18.25888794,654.05301602)(18.99566682,654.05302423)
\curveto(19.52777965,654.05301602)(20.01431217,653.90975405)(20.45526585,653.6232379)
\curveto(20.89621051,653.33670618)(21.23110861,652.92831655)(21.45996116,652.39806778)
\curveto(21.68880269,651.86780589)(21.8032262,651.10033107)(21.80323206,650.09564102)
\curveto(21.8032262,649.05001046)(21.68973296,648.21741656)(21.46275198,647.59785683)
\curveto(21.23575997,646.97829358)(20.89807105,646.50664542)(20.4496842,646.18291093)
\curveto(20.00128835,645.85917575)(19.47568439,645.69730834)(18.87287072,645.6973082)
\curveto(18.23284031,645.69730834)(17.71002716,645.87499039)(17.30442971,646.23035488)
\curveto(16.89882953,646.58571858)(16.65509814,647.08527492)(16.57323478,647.72902539)
\closepath
\moveto(20.68690393,651.34034688)
\curveto(20.68689919,651.91711033)(20.53340423,652.37480441)(20.22641858,652.71343048)
\curveto(19.91942437,653.0520428)(19.55010619,653.22135239)(19.11846291,653.22135978)
\curveto(18.67192894,653.22135239)(18.28307503,653.03901898)(17.95190002,652.67435899)
\curveto(17.620721,652.30968533)(17.4551325,651.8371069)(17.455134,651.25662227)
\curveto(17.4551325,650.73566425)(17.61234855,650.31239026)(17.92678264,649.98679903)
\curveto(18.24121277,649.6611995)(18.6291364,649.49840181)(19.09055471,649.49840547)
\curveto(19.55568782,649.49840181)(19.93802982,649.6611995)(20.23758186,649.98679903)
\curveto(20.53712532,650.31239026)(20.68689919,650.76357242)(20.68690393,651.34034688)
\closepath
}
}
{
\newrgbcolor{curcolor}{0 0 0}
\pscustom[linestyle=none,fillstyle=solid,fillcolor=curcolor]
{
\newpath
\moveto(22.79118166,647.99694414)
\lineto(23.79587697,648.13090352)
\curveto(23.91122928,647.56157445)(24.10751678,647.15132427)(24.38474006,646.90015175)
\curveto(24.6619592,646.64897711)(24.99964812,646.52339032)(25.39780784,646.52339101)
\curveto(25.87038319,646.52339032)(26.26947009,646.68711829)(26.59506975,647.01457539)
\curveto(26.92066085,647.34203013)(27.08345854,647.74762895)(27.08346331,648.23137305)
\curveto(27.08345854,648.69278582)(26.93275439,649.07326727)(26.63135042,649.37281856)
\curveto(26.32993781,649.67236277)(25.94666553,649.82213664)(25.48153245,649.82214063)
\curveto(25.29175368,649.82213664)(25.05546447,649.78492574)(24.77266409,649.71050782)
\lineto(24.8842969,650.59240704)
\curveto(24.95127395,650.5849601)(25.00522975,650.58123901)(25.04616448,650.58124376)
\curveto(25.4740871,650.58123901)(25.85921992,650.69287171)(26.20156409,650.91614219)
\curveto(26.54390048,651.13940252)(26.71507062,651.48360335)(26.71507503,651.94874571)
\curveto(26.71507062,652.31712751)(26.59041411,652.6222569)(26.34110511,652.86413478)
\curveto(26.09178804,653.1059986)(25.76991375,653.22693403)(25.37548127,653.22694142)
\curveto(24.98476376,653.22693403)(24.65916838,653.10413806)(24.39869416,652.85855314)
\curveto(24.13821577,652.61295417)(23.97076672,652.24456626)(23.89634651,651.75338829)
\lineto(22.89165119,651.93200079)
\curveto(23.01444658,652.60551199)(23.29352833,653.12739487)(23.72889729,653.49765099)
\curveto(24.1642634,653.86789179)(24.705682,654.05301602)(25.35315471,654.05302423)
\curveto(25.79968247,654.05301602)(26.21086292,653.95719795)(26.58669729,653.76556974)
\curveto(26.96252311,653.57392567)(27.24997732,653.3125191)(27.44906077,652.98134923)
\curveto(27.64813395,652.65016507)(27.74767311,652.29852206)(27.74767855,651.92641915)
\curveto(27.74767311,651.57290951)(27.65278531,651.25103522)(27.46301487,650.96079532)
\curveto(27.27323413,650.67054517)(26.99229183,650.43983759)(26.62018714,650.26867188)
\curveto(27.10392453,650.15703475)(27.47975463,649.92539689)(27.74767855,649.57375762)
\curveto(28.01559159,649.22211087)(28.14955083,648.78209198)(28.14955667,648.25369961)
\curveto(28.14955083,647.53924791)(27.88907453,646.9336405)(27.36812698,646.43687558)
\curveto(26.84716932,645.94010946)(26.18853638,645.6917267)(25.3922262,645.69172656)
\curveto(24.67405274,645.6917267)(24.07774806,645.90568938)(23.60331037,646.33361523)
\curveto(23.1288701,646.76154009)(22.8581608,647.3159825)(22.79118166,647.99694414)
\closepath
}
}
{
\newrgbcolor{curcolor}{0 0 0}
\pscustom[linestyle=none,fillstyle=solid,fillcolor=curcolor]
{
\newpath
\moveto(7.48074186,602.17681503)
\lineto(6.47604655,602.17681503)
\lineto(6.47604655,608.57895685)
\curveto(6.23417244,608.34824286)(5.91694951,608.11753528)(5.52437682,607.88683341)
\curveto(5.13179952,607.65612011)(4.77922624,607.48308943)(4.46665592,607.36774083)
\lineto(4.46665592,608.3389463)
\curveto(5.02853927,608.60313753)(5.51972315,608.92315127)(5.94020905,609.29898849)
\curveto(6.3606895,609.67481146)(6.6583767,610.03947828)(6.83327155,610.39299005)
\lineto(7.48074186,610.39299005)
\closepath
}
}
{
\newrgbcolor{curcolor}{0 0 0}
\pscustom[linestyle=none,fillstyle=solid,fillcolor=curcolor]
{
\newpath
\moveto(10.21016429,604.06899121)
\lineto(11.17578812,604.15829746)
\curveto(11.25765051,603.70432249)(11.4139363,603.37500603)(11.64464594,603.17034707)
\curveto(11.87535146,602.96568612)(12.17117812,602.86335615)(12.5321268,602.86335683)
\curveto(12.84097433,602.86335615)(13.11168363,602.93405686)(13.34425551,603.07545918)
\curveto(13.57681988,603.2168597)(13.76752574,603.40570502)(13.91637368,603.6419957)
\curveto(14.06521295,603.87828345)(14.18986946,604.19736692)(14.2903436,604.59924707)
\curveto(14.39080833,605.00112237)(14.44104304,605.41044228)(14.4410479,605.82720801)
\curveto(14.44104304,605.87185744)(14.4391825,605.93883706)(14.43546626,606.02814707)
\curveto(14.23452254,605.70812948)(13.96009215,605.44858345)(13.61217426,605.2495082)
\curveto(13.26424832,605.05042681)(12.88748795,604.95088766)(12.48189203,604.95089043)
\curveto(11.80465075,604.95088766)(11.23160288,605.1964796)(10.76274672,605.68766699)
\curveto(10.29388819,606.17884737)(10.05945952,606.82631704)(10.05946,607.63007794)
\curveto(10.05945952,608.45987556)(10.30412119,609.12781122)(10.79344574,609.63388693)
\curveto(11.28276787,610.13994771)(11.89581746,610.39298184)(12.63259633,610.39299005)
\curveto(13.16470916,610.39298184)(13.65124168,610.24971987)(14.09219536,609.96320372)
\curveto(14.53314002,609.676672)(14.86803812,609.26828237)(15.09689067,608.7380336)
\curveto(15.3257322,608.20777171)(15.44015572,607.44029689)(15.44016157,606.43560684)
\curveto(15.44015572,605.38997628)(15.32666247,604.55738238)(15.09968149,603.93782265)
\curveto(14.87268949,603.3182594)(14.53500056,602.84661124)(14.08661372,602.52287675)
\curveto(13.63821787,602.19914157)(13.1126139,602.03727416)(12.50980024,602.03727402)
\curveto(11.86976983,602.03727416)(11.34695668,602.21495621)(10.94135922,602.5703207)
\curveto(10.53575905,602.9256844)(10.29202765,603.42524074)(10.21016429,604.06899121)
\closepath
\moveto(14.32383344,607.6803127)
\curveto(14.32382871,608.25707616)(14.17033374,608.71477023)(13.86334809,609.0533963)
\curveto(13.55635389,609.39200862)(13.1870357,609.56131821)(12.75539242,609.5613256)
\curveto(12.30885845,609.56131821)(11.92000454,609.3789848)(11.58882953,609.01432482)
\curveto(11.25765051,608.64965115)(11.09206201,608.17707272)(11.09206351,607.59658809)
\curveto(11.09206201,607.07563007)(11.24927806,606.65235608)(11.56371215,606.32676485)
\curveto(11.87814228,606.00116532)(12.26606591,605.83836763)(12.72748422,605.83837129)
\curveto(13.19261733,605.83836763)(13.57495934,606.00116532)(13.87451137,606.32676485)
\curveto(14.17405483,606.65235608)(14.32382871,607.10353824)(14.32383344,607.6803127)
\closepath
}
}
{
\newrgbcolor{curcolor}{0 0 0}
\pscustom[linestyle=none,fillstyle=solid,fillcolor=curcolor]
{
\newpath
\moveto(16.57323478,604.06899121)
\lineto(17.53885861,604.15829746)
\curveto(17.620721,603.70432249)(17.77700678,603.37500603)(18.00771643,603.17034707)
\curveto(18.23842195,602.96568612)(18.53424861,602.86335615)(18.89519729,602.86335683)
\curveto(19.20404481,602.86335615)(19.47475411,602.93405686)(19.707326,603.07545918)
\curveto(19.93989037,603.2168597)(20.13059623,603.40570502)(20.27944417,603.6419957)
\curveto(20.42828344,603.87828345)(20.55293995,604.19736692)(20.65341409,604.59924707)
\curveto(20.75387881,605.00112237)(20.80411353,605.41044228)(20.80411839,605.82720801)
\curveto(20.80411353,605.87185744)(20.80225298,605.93883706)(20.79853674,606.02814707)
\curveto(20.59759303,605.70812948)(20.32316264,605.44858345)(19.97524475,605.2495082)
\curveto(19.62731881,605.05042681)(19.25055844,604.95088766)(18.84496252,604.95089043)
\curveto(18.16772124,604.95088766)(17.59467337,605.1964796)(17.1258172,605.68766699)
\curveto(16.65695868,606.17884737)(16.42253001,606.82631704)(16.42253048,607.63007794)
\curveto(16.42253001,608.45987556)(16.66719168,609.12781122)(17.15651623,609.63388693)
\curveto(17.64583836,610.13994771)(18.25888794,610.39298184)(18.99566682,610.39299005)
\curveto(19.52777965,610.39298184)(20.01431217,610.24971987)(20.45526585,609.96320372)
\curveto(20.89621051,609.676672)(21.23110861,609.26828237)(21.45996116,608.7380336)
\curveto(21.68880269,608.20777171)(21.8032262,607.44029689)(21.80323206,606.43560684)
\curveto(21.8032262,605.38997628)(21.68973296,604.55738238)(21.46275198,603.93782265)
\curveto(21.23575997,603.3182594)(20.89807105,602.84661124)(20.4496842,602.52287675)
\curveto(20.00128835,602.19914157)(19.47568439,602.03727416)(18.87287072,602.03727402)
\curveto(18.23284031,602.03727416)(17.71002716,602.21495621)(17.30442971,602.5703207)
\curveto(16.89882953,602.9256844)(16.65509814,603.42524074)(16.57323478,604.06899121)
\closepath
\moveto(20.68690393,607.6803127)
\curveto(20.68689919,608.25707616)(20.53340423,608.71477023)(20.22641858,609.0533963)
\curveto(19.91942437,609.39200862)(19.55010619,609.56131821)(19.11846291,609.5613256)
\curveto(18.67192894,609.56131821)(18.28307503,609.3789848)(17.95190002,609.01432482)
\curveto(17.620721,608.64965115)(17.4551325,608.17707272)(17.455134,607.59658809)
\curveto(17.4551325,607.07563007)(17.61234855,606.65235608)(17.92678264,606.32676485)
\curveto(18.24121277,606.00116532)(18.6291364,605.83836763)(19.09055471,605.83837129)
\curveto(19.55568782,605.83836763)(19.93802982,606.00116532)(20.23758186,606.32676485)
\curveto(20.53712532,606.65235608)(20.68689919,607.10353824)(20.68690393,607.6803127)
\closepath
}
}
{
\newrgbcolor{curcolor}{0 0 0}
\pscustom[linestyle=none,fillstyle=solid,fillcolor=curcolor]
{
\newpath
\moveto(26.00620667,602.17681503)
\lineto(26.00620667,604.1359709)
\lineto(22.45628322,604.1359709)
\lineto(22.45628322,605.0569416)
\lineto(26.19040081,610.35950021)
\lineto(27.01090198,610.35950021)
\lineto(27.01090198,605.0569416)
\lineto(28.11606683,605.0569416)
\lineto(28.11606683,604.1359709)
\lineto(27.01090198,604.1359709)
\lineto(27.01090198,602.17681503)
\closepath
\moveto(26.00620667,605.0569416)
\lineto(26.00620667,608.74640606)
\lineto(23.44423361,605.0569416)
\closepath
}
}
{
\newrgbcolor{curcolor}{0 0 0}
\pscustom[linestyle=none,fillstyle=solid,fillcolor=curcolor]
{
\newpath
\moveto(7.48074186,558.50689316)
\lineto(6.47604655,558.50689316)
\lineto(6.47604655,564.90903497)
\curveto(6.23417244,564.67832099)(5.91694951,564.4476134)(5.52437682,564.21691153)
\curveto(5.13179952,563.98619824)(4.77922624,563.81316755)(4.46665592,563.69781895)
\lineto(4.46665592,564.66902442)
\curveto(5.02853927,564.93321565)(5.51972315,565.2532294)(5.94020905,565.62906661)
\curveto(6.3606895,566.00488958)(6.6583767,566.36955641)(6.83327155,566.72306818)
\lineto(7.48074186,566.72306818)
\closepath
}
}
{
\newrgbcolor{curcolor}{0 0 0}
\pscustom[linestyle=none,fillstyle=solid,fillcolor=curcolor]
{
\newpath
\moveto(10.21016429,560.39906933)
\lineto(11.17578812,560.48837558)
\curveto(11.25765051,560.03440062)(11.4139363,559.70508415)(11.64464594,559.50042519)
\curveto(11.87535146,559.29576425)(12.17117812,559.19343427)(12.5321268,559.19343496)
\curveto(12.84097433,559.19343427)(13.11168363,559.26413498)(13.34425551,559.4055373)
\curveto(13.57681988,559.54693782)(13.76752574,559.73578314)(13.91637368,559.97207383)
\curveto(14.06521295,560.20836158)(14.18986946,560.52744505)(14.2903436,560.9293252)
\curveto(14.39080833,561.3312005)(14.44104304,561.7405204)(14.4410479,562.15728614)
\curveto(14.44104304,562.20193557)(14.4391825,562.26891519)(14.43546626,562.3582252)
\curveto(14.23452254,562.0382076)(13.96009215,561.77866157)(13.61217426,561.57958633)
\curveto(13.26424832,561.38050494)(12.88748795,561.28096578)(12.48189203,561.28096856)
\curveto(11.80465075,561.28096578)(11.23160288,561.52655772)(10.76274672,562.01774512)
\curveto(10.29388819,562.50892549)(10.05945952,563.15639516)(10.05946,563.96015606)
\curveto(10.05945952,564.78995369)(10.30412119,565.45788935)(10.79344574,565.96396505)
\curveto(11.28276787,566.47002584)(11.89581746,566.72305996)(12.63259633,566.72306818)
\curveto(13.16470916,566.72305996)(13.65124168,566.579798)(14.09219536,566.29328185)
\curveto(14.53314002,566.00675013)(14.86803812,565.5983605)(15.09689067,565.06811173)
\curveto(15.3257322,564.53784984)(15.44015572,563.77037502)(15.44016157,562.76568497)
\curveto(15.44015572,561.72005441)(15.32666247,560.88746051)(15.09968149,560.26790078)
\curveto(14.87268949,559.64833753)(14.53500056,559.17668937)(14.08661372,558.85295488)
\curveto(13.63821787,558.5292197)(13.1126139,558.36735228)(12.50980024,558.36735214)
\curveto(11.86976983,558.36735228)(11.34695668,558.54503433)(10.94135922,558.90039882)
\curveto(10.53575905,559.25576253)(10.29202765,559.75531887)(10.21016429,560.39906933)
\closepath
\moveto(14.32383344,564.01039083)
\curveto(14.32382871,564.58715428)(14.17033374,565.04484835)(13.86334809,565.38347442)
\curveto(13.55635389,565.72208674)(13.1870357,565.89139634)(12.75539242,565.89140372)
\curveto(12.30885845,565.89139634)(11.92000454,565.70906293)(11.58882953,565.34440294)
\curveto(11.25765051,564.97972928)(11.09206201,564.50715084)(11.09206351,563.92666622)
\curveto(11.09206201,563.40570819)(11.24927806,562.9824342)(11.56371215,562.65684297)
\curveto(11.87814228,562.33124344)(12.26606591,562.16844576)(12.72748422,562.16844942)
\curveto(13.19261733,562.16844576)(13.57495934,562.33124344)(13.87451137,562.65684297)
\curveto(14.17405483,562.9824342)(14.32382871,563.43361637)(14.32383344,564.01039083)
\closepath
}
}
{
\newrgbcolor{curcolor}{0 0 0}
\pscustom[linestyle=none,fillstyle=solid,fillcolor=curcolor]
{
\newpath
\moveto(16.57323478,560.39906933)
\lineto(17.53885861,560.48837558)
\curveto(17.620721,560.03440062)(17.77700678,559.70508415)(18.00771643,559.50042519)
\curveto(18.23842195,559.29576425)(18.53424861,559.19343427)(18.89519729,559.19343496)
\curveto(19.20404481,559.19343427)(19.47475411,559.26413498)(19.707326,559.4055373)
\curveto(19.93989037,559.54693782)(20.13059623,559.73578314)(20.27944417,559.97207383)
\curveto(20.42828344,560.20836158)(20.55293995,560.52744505)(20.65341409,560.9293252)
\curveto(20.75387881,561.3312005)(20.80411353,561.7405204)(20.80411839,562.15728614)
\curveto(20.80411353,562.20193557)(20.80225298,562.26891519)(20.79853674,562.3582252)
\curveto(20.59759303,562.0382076)(20.32316264,561.77866157)(19.97524475,561.57958633)
\curveto(19.62731881,561.38050494)(19.25055844,561.28096578)(18.84496252,561.28096856)
\curveto(18.16772124,561.28096578)(17.59467337,561.52655772)(17.1258172,562.01774512)
\curveto(16.65695868,562.50892549)(16.42253001,563.15639516)(16.42253048,563.96015606)
\curveto(16.42253001,564.78995369)(16.66719168,565.45788935)(17.15651623,565.96396505)
\curveto(17.64583836,566.47002584)(18.25888794,566.72305996)(18.99566682,566.72306818)
\curveto(19.52777965,566.72305996)(20.01431217,566.579798)(20.45526585,566.29328185)
\curveto(20.89621051,566.00675013)(21.23110861,565.5983605)(21.45996116,565.06811173)
\curveto(21.68880269,564.53784984)(21.8032262,563.77037502)(21.80323206,562.76568497)
\curveto(21.8032262,561.72005441)(21.68973296,560.88746051)(21.46275198,560.26790078)
\curveto(21.23575997,559.64833753)(20.89807105,559.17668937)(20.4496842,558.85295488)
\curveto(20.00128835,558.5292197)(19.47568439,558.36735228)(18.87287072,558.36735214)
\curveto(18.23284031,558.36735228)(17.71002716,558.54503433)(17.30442971,558.90039882)
\curveto(16.89882953,559.25576253)(16.65509814,559.75531887)(16.57323478,560.39906933)
\closepath
\moveto(20.68690393,564.01039083)
\curveto(20.68689919,564.58715428)(20.53340423,565.04484835)(20.22641858,565.38347442)
\curveto(19.91942437,565.72208674)(19.55010619,565.89139634)(19.11846291,565.89140372)
\curveto(18.67192894,565.89139634)(18.28307503,565.70906293)(17.95190002,565.34440294)
\curveto(17.620721,564.97972928)(17.4551325,564.50715084)(17.455134,563.92666622)
\curveto(17.4551325,563.40570819)(17.61234855,562.9824342)(17.92678264,562.65684297)
\curveto(18.24121277,562.33124344)(18.6291364,562.16844576)(19.09055471,562.16844942)
\curveto(19.55568782,562.16844576)(19.93802982,562.33124344)(20.23758186,562.65684297)
\curveto(20.53712532,562.9824342)(20.68689919,563.43361637)(20.68690393,564.01039083)
\closepath
}
}
{
\newrgbcolor{curcolor}{0 0 0}
\pscustom[linestyle=none,fillstyle=solid,fillcolor=curcolor]
{
\newpath
\moveto(22.78560002,560.65024316)
\lineto(23.8405301,560.73954941)
\curveto(23.91867146,560.22603676)(24.1000746,559.83997366)(24.38474006,559.58135898)
\curveto(24.66940138,559.32274215)(25.01267193,559.19343427)(25.41455276,559.19343496)
\curveto(25.89829136,559.19343427)(26.30761127,559.37576768)(26.6425137,559.74043574)
\curveto(26.97740747,560.10510133)(27.14485652,560.58884303)(27.14486136,561.19166231)
\curveto(27.14485652,561.76470749)(26.98391938,562.21681993)(26.66204944,562.54800098)
\curveto(26.3401708,562.87917395)(25.91875736,563.04476246)(25.39780784,563.044767)
\curveto(25.07406992,563.04476246)(24.78196435,562.97127093)(24.52149026,562.82429219)
\curveto(24.26101174,562.67730482)(24.05635179,562.48659895)(23.90750979,562.25217403)
\lineto(22.96421252,562.37497012)
\lineto(23.75680549,566.57794552)
\lineto(27.82582152,566.57794552)
\lineto(27.82582152,565.61790333)
\lineto(24.56056174,565.61790333)
\lineto(24.11961213,563.41873692)
\curveto(24.61079421,563.76107229)(25.12616518,563.93224243)(25.66572659,563.93224786)
\curveto(26.38017252,563.93224243)(26.98298911,563.68478995)(27.47417815,563.18988965)
\curveto(27.96535688,562.69498)(28.21094882,562.0586736)(28.21095472,561.28096856)
\curveto(28.21094882,560.54046886)(27.9951256,559.90044138)(27.56348441,559.36088418)
\curveto(27.03880546,558.6985293)(26.32249563,558.36735228)(25.41455276,558.36735214)
\curveto(24.67033165,558.36735228)(24.0628637,558.57573332)(23.59214709,558.99249589)
\curveto(23.12142792,559.40925749)(22.85257916,559.96183936)(22.78560002,560.65024316)
\closepath
}
}
{
\newrgbcolor{curcolor}{0 0 0}
\pscustom[linestyle=none,fillstyle=solid,fillcolor=curcolor]
{
\newpath
\moveto(7.48074186,514.83684921)
\lineto(6.47604655,514.83684921)
\lineto(6.47604655,521.23899103)
\curveto(6.23417244,521.00827704)(5.91694951,520.77756946)(5.52437682,520.54686759)
\curveto(5.13179952,520.31615429)(4.77922624,520.14312361)(4.46665592,520.02777501)
\lineto(4.46665592,520.99898048)
\curveto(5.02853927,521.26317171)(5.51972315,521.58318545)(5.94020905,521.95902267)
\curveto(6.3606895,522.33484564)(6.6583767,522.69951246)(6.83327155,523.05302423)
\lineto(7.48074186,523.05302423)
\closepath
}
}
{
\newrgbcolor{curcolor}{0 0 0}
\pscustom[linestyle=none,fillstyle=solid,fillcolor=curcolor]
{
\newpath
\moveto(10.21016429,516.72902539)
\lineto(11.17578812,516.81833164)
\curveto(11.25765051,516.36435667)(11.4139363,516.03504021)(11.64464594,515.83038125)
\curveto(11.87535146,515.6257203)(12.17117812,515.52339032)(12.5321268,515.52339101)
\curveto(12.84097433,515.52339032)(13.11168363,515.59409104)(13.34425551,515.73549336)
\curveto(13.57681988,515.87689388)(13.76752574,516.0657392)(13.91637368,516.30202988)
\curveto(14.06521295,516.53831763)(14.18986946,516.8574011)(14.2903436,517.25928125)
\curveto(14.39080833,517.66115655)(14.44104304,518.07047646)(14.4410479,518.48724219)
\curveto(14.44104304,518.53189162)(14.4391825,518.59887124)(14.43546626,518.68818125)
\curveto(14.23452254,518.36816366)(13.96009215,518.10861763)(13.61217426,517.90954238)
\curveto(13.26424832,517.71046099)(12.88748795,517.61092184)(12.48189203,517.61092461)
\curveto(11.80465075,517.61092184)(11.23160288,517.85651378)(10.76274672,518.34770117)
\curveto(10.29388819,518.83888155)(10.05945952,519.48635122)(10.05946,520.29011212)
\curveto(10.05945952,521.11990974)(10.30412119,521.7878454)(10.79344574,522.29392111)
\curveto(11.28276787,522.79998189)(11.89581746,523.05301602)(12.63259633,523.05302423)
\curveto(13.16470916,523.05301602)(13.65124168,522.90975405)(14.09219536,522.6232379)
\curveto(14.53314002,522.33670618)(14.86803812,521.92831655)(15.09689067,521.39806778)
\curveto(15.3257322,520.86780589)(15.44015572,520.10033107)(15.44016157,519.09564102)
\curveto(15.44015572,518.05001046)(15.32666247,517.21741656)(15.09968149,516.59785683)
\curveto(14.87268949,515.97829358)(14.53500056,515.50664542)(14.08661372,515.18291093)
\curveto(13.63821787,514.85917575)(13.1126139,514.69730834)(12.50980024,514.6973082)
\curveto(11.86976983,514.69730834)(11.34695668,514.87499039)(10.94135922,515.23035488)
\curveto(10.53575905,515.58571858)(10.29202765,516.08527492)(10.21016429,516.72902539)
\closepath
\moveto(14.32383344,520.34034688)
\curveto(14.32382871,520.91711033)(14.17033374,521.37480441)(13.86334809,521.71343048)
\curveto(13.55635389,522.0520428)(13.1870357,522.22135239)(12.75539242,522.22135978)
\curveto(12.30885845,522.22135239)(11.92000454,522.03901898)(11.58882953,521.67435899)
\curveto(11.25765051,521.30968533)(11.09206201,520.8371069)(11.09206351,520.25662227)
\curveto(11.09206201,519.73566425)(11.24927806,519.31239026)(11.56371215,518.98679903)
\curveto(11.87814228,518.6611995)(12.26606591,518.49840181)(12.72748422,518.49840547)
\curveto(13.19261733,518.49840181)(13.57495934,518.6611995)(13.87451137,518.98679903)
\curveto(14.17405483,519.31239026)(14.32382871,519.76357242)(14.32383344,520.34034688)
\closepath
}
}
{
\newrgbcolor{curcolor}{0 0 0}
\pscustom[linestyle=none,fillstyle=solid,fillcolor=curcolor]
{
\newpath
\moveto(16.57323478,516.72902539)
\lineto(17.53885861,516.81833164)
\curveto(17.620721,516.36435667)(17.77700678,516.03504021)(18.00771643,515.83038125)
\curveto(18.23842195,515.6257203)(18.53424861,515.52339032)(18.89519729,515.52339101)
\curveto(19.20404481,515.52339032)(19.47475411,515.59409104)(19.707326,515.73549336)
\curveto(19.93989037,515.87689388)(20.13059623,516.0657392)(20.27944417,516.30202988)
\curveto(20.42828344,516.53831763)(20.55293995,516.8574011)(20.65341409,517.25928125)
\curveto(20.75387881,517.66115655)(20.80411353,518.07047646)(20.80411839,518.48724219)
\curveto(20.80411353,518.53189162)(20.80225298,518.59887124)(20.79853674,518.68818125)
\curveto(20.59759303,518.36816366)(20.32316264,518.10861763)(19.97524475,517.90954238)
\curveto(19.62731881,517.71046099)(19.25055844,517.61092184)(18.84496252,517.61092461)
\curveto(18.16772124,517.61092184)(17.59467337,517.85651378)(17.1258172,518.34770117)
\curveto(16.65695868,518.83888155)(16.42253001,519.48635122)(16.42253048,520.29011212)
\curveto(16.42253001,521.11990974)(16.66719168,521.7878454)(17.15651623,522.29392111)
\curveto(17.64583836,522.79998189)(18.25888794,523.05301602)(18.99566682,523.05302423)
\curveto(19.52777965,523.05301602)(20.01431217,522.90975405)(20.45526585,522.6232379)
\curveto(20.89621051,522.33670618)(21.23110861,521.92831655)(21.45996116,521.39806778)
\curveto(21.68880269,520.86780589)(21.8032262,520.10033107)(21.80323206,519.09564102)
\curveto(21.8032262,518.05001046)(21.68973296,517.21741656)(21.46275198,516.59785683)
\curveto(21.23575997,515.97829358)(20.89807105,515.50664542)(20.4496842,515.18291093)
\curveto(20.00128835,514.85917575)(19.47568439,514.69730834)(18.87287072,514.6973082)
\curveto(18.23284031,514.69730834)(17.71002716,514.87499039)(17.30442971,515.23035488)
\curveto(16.89882953,515.58571858)(16.65509814,516.08527492)(16.57323478,516.72902539)
\closepath
\moveto(20.68690393,520.34034688)
\curveto(20.68689919,520.91711033)(20.53340423,521.37480441)(20.22641858,521.71343048)
\curveto(19.91942437,522.0520428)(19.55010619,522.22135239)(19.11846291,522.22135978)
\curveto(18.67192894,522.22135239)(18.28307503,522.03901898)(17.95190002,521.67435899)
\curveto(17.620721,521.30968533)(17.4551325,520.8371069)(17.455134,520.25662227)
\curveto(17.4551325,519.73566425)(17.61234855,519.31239026)(17.92678264,518.98679903)
\curveto(18.24121277,518.6611995)(18.6291364,518.49840181)(19.09055471,518.49840547)
\curveto(19.55568782,518.49840181)(19.93802982,518.6611995)(20.23758186,518.98679903)
\curveto(20.53712532,519.31239026)(20.68689919,519.76357242)(20.68690393,520.34034688)
\closepath
}
}
{
\newrgbcolor{curcolor}{0 0 0}
\pscustom[linestyle=none,fillstyle=solid,fillcolor=curcolor]
{
\newpath
\moveto(27.99885237,521.0157254)
\lineto(26.9997387,520.93758243)
\curveto(26.91042785,521.33201187)(26.78391079,521.61853581)(26.62018714,521.79715509)
\curveto(26.34854326,522.08367206)(26.01364515,522.22693403)(25.61549182,522.22694142)
\curveto(25.29547477,522.22693403)(25.01453248,522.13762787)(24.77266409,521.95902267)
\curveto(24.45636897,521.72830796)(24.20705594,521.39154931)(24.02472424,520.94874571)
\curveto(23.84238911,520.50592989)(23.74750132,519.87520512)(23.74006057,519.05656954)
\curveto(23.98192999,519.42495323)(24.27775665,519.69845335)(24.62754143,519.87707071)
\curveto(24.97732158,520.05567799)(25.34384895,520.14498415)(25.72712464,520.14498946)
\curveto(26.39691743,520.14498415)(26.96717447,519.89846194)(27.43789749,519.40542208)
\curveto(27.90861025,518.91237308)(28.1439692,518.27513641)(28.14397503,517.49371016)
\curveto(28.1439692,516.98019707)(28.03326677,516.50296728)(27.81186741,516.06201933)
\curveto(27.59045706,515.62106894)(27.28625795,515.28338002)(26.89926917,515.04895156)
\curveto(26.51227122,514.81452267)(26.07318259,514.69730834)(25.58200198,514.6973082)
\curveto(24.74475345,514.69730834)(24.06193343,515.00522854)(23.53353986,515.62106972)
\curveto(23.00514386,516.23690934)(22.74094646,517.25183665)(22.74094689,518.66585469)
\curveto(22.74094646,520.24731413)(23.03305203,521.39713095)(23.61726447,522.11530861)
\curveto(24.1270525,522.74044445)(24.81359361,523.05301602)(25.67688987,523.05302423)
\curveto(26.32063508,523.05301602)(26.84809959,522.87254315)(27.25928499,522.51160509)
\curveto(27.67046049,522.15065168)(27.91698271,521.65202562)(27.99885237,521.0157254)
\closepath
\moveto(23.89634651,517.48812852)
\curveto(23.89634492,517.14206449)(23.96983645,516.81088748)(24.11682131,516.49459648)
\curveto(24.26380256,516.17830217)(24.46939279,515.93736159)(24.7335926,515.77177402)
\curveto(24.99778757,515.60618458)(25.27500878,515.52339032)(25.56525706,515.52339101)
\curveto(25.98945807,515.52339032)(26.35412489,515.69456047)(26.65925862,516.03690195)
\curveto(26.96438366,516.37924103)(27.11694835,516.84437729)(27.11695315,517.43231211)
\curveto(27.11694835,517.9979152)(26.9662442,518.44351573)(26.66484026,518.76911504)
\curveto(26.36342762,519.09470649)(25.98387643,519.25750418)(25.52618557,519.2575086)
\curveto(25.07220937,519.25750418)(24.68707655,519.09470649)(24.37078596,518.76911504)
\curveto(24.05449125,518.44351573)(23.89634492,518.01652065)(23.89634651,517.48812852)
\closepath
}
}
{
\newrgbcolor{curcolor}{0 0 0}
\pscustom[linestyle=none,fillstyle=solid,fillcolor=curcolor]
{
\newpath
\moveto(7.48074186,471.16692734)
\lineto(6.47604655,471.16692734)
\lineto(6.47604655,477.56906915)
\curveto(6.23417244,477.33835517)(5.91694951,477.10764758)(5.52437682,476.87694571)
\curveto(5.13179952,476.64623242)(4.77922624,476.47320173)(4.46665592,476.35785313)
\lineto(4.46665592,477.3290586)
\curveto(5.02853927,477.59324983)(5.51972315,477.91326358)(5.94020905,478.28910079)
\curveto(6.3606895,478.66492376)(6.6583767,479.02959059)(6.83327155,479.38310236)
\lineto(7.48074186,479.38310236)
\closepath
}
}
{
\newrgbcolor{curcolor}{0 0 0}
\pscustom[linestyle=none,fillstyle=solid,fillcolor=curcolor]
{
\newpath
\moveto(10.21016429,473.05910351)
\lineto(11.17578812,473.14840976)
\curveto(11.25765051,472.6944348)(11.4139363,472.36511833)(11.64464594,472.16045937)
\curveto(11.87535146,471.95579843)(12.17117812,471.85346845)(12.5321268,471.85346914)
\curveto(12.84097433,471.85346845)(13.11168363,471.92416916)(13.34425551,472.06557148)
\curveto(13.57681988,472.206972)(13.76752574,472.39581732)(13.91637368,472.63210801)
\curveto(14.06521295,472.86839576)(14.18986946,473.18747923)(14.2903436,473.58935938)
\curveto(14.39080833,473.99123468)(14.44104304,474.40055458)(14.4410479,474.81732032)
\curveto(14.44104304,474.86196975)(14.4391825,474.92894937)(14.43546626,475.01825938)
\curveto(14.23452254,474.69824178)(13.96009215,474.43869575)(13.61217426,474.23962051)
\curveto(13.26424832,474.04053912)(12.88748795,473.94099996)(12.48189203,473.94100274)
\curveto(11.80465075,473.94099996)(11.23160288,474.1865919)(10.76274672,474.6777793)
\curveto(10.29388819,475.16895967)(10.05945952,475.81642934)(10.05946,476.62019024)
\curveto(10.05945952,477.44998787)(10.30412119,478.11792353)(10.79344574,478.62399923)
\curveto(11.28276787,479.13006002)(11.89581746,479.38309414)(12.63259633,479.38310236)
\curveto(13.16470916,479.38309414)(13.65124168,479.23983218)(14.09219536,478.95331603)
\curveto(14.53314002,478.66678431)(14.86803812,478.25839468)(15.09689067,477.72814591)
\curveto(15.3257322,477.19788402)(15.44015572,476.4304092)(15.44016157,475.42571915)
\curveto(15.44015572,474.38008859)(15.32666247,473.54749469)(15.09968149,472.92793496)
\curveto(14.87268949,472.30837171)(14.53500056,471.83672354)(14.08661372,471.51298906)
\curveto(13.63821787,471.18925388)(13.1126139,471.02738646)(12.50980024,471.02738632)
\curveto(11.86976983,471.02738646)(11.34695668,471.20506851)(10.94135922,471.560433)
\curveto(10.53575905,471.91579671)(10.29202765,472.41535305)(10.21016429,473.05910351)
\closepath
\moveto(14.32383344,476.67042501)
\curveto(14.32382871,477.24718846)(14.17033374,477.70488253)(13.86334809,478.0435086)
\curveto(13.55635389,478.38212092)(13.1870357,478.55143052)(12.75539242,478.5514379)
\curveto(12.30885845,478.55143052)(11.92000454,478.36909711)(11.58882953,478.00443712)
\curveto(11.25765051,477.63976346)(11.09206201,477.16718502)(11.09206351,476.5867004)
\curveto(11.09206201,476.06574237)(11.24927806,475.64246838)(11.56371215,475.31687715)
\curveto(11.87814228,474.99127762)(12.26606591,474.82847994)(12.72748422,474.8284836)
\curveto(13.19261733,474.82847994)(13.57495934,474.99127762)(13.87451137,475.31687715)
\curveto(14.17405483,475.64246838)(14.32382871,476.09365055)(14.32383344,476.67042501)
\closepath
}
}
{
\newrgbcolor{curcolor}{0 0 0}
\pscustom[linestyle=none,fillstyle=solid,fillcolor=curcolor]
{
\newpath
\moveto(16.57323478,473.05910351)
\lineto(17.53885861,473.14840976)
\curveto(17.620721,472.6944348)(17.77700678,472.36511833)(18.00771643,472.16045937)
\curveto(18.23842195,471.95579843)(18.53424861,471.85346845)(18.89519729,471.85346914)
\curveto(19.20404481,471.85346845)(19.47475411,471.92416916)(19.707326,472.06557148)
\curveto(19.93989037,472.206972)(20.13059623,472.39581732)(20.27944417,472.63210801)
\curveto(20.42828344,472.86839576)(20.55293995,473.18747923)(20.65341409,473.58935938)
\curveto(20.75387881,473.99123468)(20.80411353,474.40055458)(20.80411839,474.81732032)
\curveto(20.80411353,474.86196975)(20.80225298,474.92894937)(20.79853674,475.01825938)
\curveto(20.59759303,474.69824178)(20.32316264,474.43869575)(19.97524475,474.23962051)
\curveto(19.62731881,474.04053912)(19.25055844,473.94099996)(18.84496252,473.94100274)
\curveto(18.16772124,473.94099996)(17.59467337,474.1865919)(17.1258172,474.6777793)
\curveto(16.65695868,475.16895967)(16.42253001,475.81642934)(16.42253048,476.62019024)
\curveto(16.42253001,477.44998787)(16.66719168,478.11792353)(17.15651623,478.62399923)
\curveto(17.64583836,479.13006002)(18.25888794,479.38309414)(18.99566682,479.38310236)
\curveto(19.52777965,479.38309414)(20.01431217,479.23983218)(20.45526585,478.95331603)
\curveto(20.89621051,478.66678431)(21.23110861,478.25839468)(21.45996116,477.72814591)
\curveto(21.68880269,477.19788402)(21.8032262,476.4304092)(21.80323206,475.42571915)
\curveto(21.8032262,474.38008859)(21.68973296,473.54749469)(21.46275198,472.92793496)
\curveto(21.23575997,472.30837171)(20.89807105,471.83672354)(20.4496842,471.51298906)
\curveto(20.00128835,471.18925388)(19.47568439,471.02738646)(18.87287072,471.02738632)
\curveto(18.23284031,471.02738646)(17.71002716,471.20506851)(17.30442971,471.560433)
\curveto(16.89882953,471.91579671)(16.65509814,472.41535305)(16.57323478,473.05910351)
\closepath
\moveto(20.68690393,476.67042501)
\curveto(20.68689919,477.24718846)(20.53340423,477.70488253)(20.22641858,478.0435086)
\curveto(19.91942437,478.38212092)(19.55010619,478.55143052)(19.11846291,478.5514379)
\curveto(18.67192894,478.55143052)(18.28307503,478.36909711)(17.95190002,478.00443712)
\curveto(17.620721,477.63976346)(17.4551325,477.16718502)(17.455134,476.5867004)
\curveto(17.4551325,476.06574237)(17.61234855,475.64246838)(17.92678264,475.31687715)
\curveto(18.24121277,474.99127762)(18.6291364,474.82847994)(19.09055471,474.8284836)
\curveto(19.55568782,474.82847994)(19.93802982,474.99127762)(20.23758186,475.31687715)
\curveto(20.53712532,475.64246838)(20.68689919,476.09365055)(20.68690393,476.67042501)
\closepath
}
}
{
\newrgbcolor{curcolor}{0 0 0}
\pscustom[linestyle=none,fillstyle=solid,fillcolor=curcolor]
{
\newpath
\moveto(22.85257971,478.27793751)
\lineto(22.85257971,479.24356134)
\lineto(28.14955667,479.24356134)
\lineto(28.14955667,478.46213165)
\curveto(27.62859823,477.90768194)(27.11229699,477.17090611)(26.6006514,476.25180196)
\curveto(26.08899723,475.33268764)(25.69363141,474.38753077)(25.41455276,473.41632852)
\curveto(25.21361079,472.7316457)(25.08523319,471.98184606)(25.02941956,471.16692734)
\lineto(23.99681604,471.16692734)
\curveto(24.00797762,471.81067591)(24.13449468,472.58838373)(24.3763676,473.50005312)
\curveto(24.61823639,474.41171785)(24.96522803,475.29082537)(25.41734358,476.13737833)
\curveto(25.86945291,476.98392134)(26.3504038,477.69744035)(26.86019768,478.27793751)
\closepath
}
}
{
\newrgbcolor{curcolor}{0 0 0}
\pscustom[linestyle=none,fillstyle=solid,fillcolor=curcolor]
{
\newpath
\moveto(7.48074186,427.50689316)
\lineto(6.47604655,427.50689316)
\lineto(6.47604655,433.90903497)
\curveto(6.23417244,433.67832099)(5.91694951,433.4476134)(5.52437682,433.21691153)
\curveto(5.13179952,432.98619824)(4.77922624,432.81316755)(4.46665592,432.69781895)
\lineto(4.46665592,433.66902442)
\curveto(5.02853927,433.93321565)(5.51972315,434.2532294)(5.94020905,434.62906661)
\curveto(6.3606895,435.00488958)(6.6583767,435.36955641)(6.83327155,435.72306818)
\lineto(7.48074186,435.72306818)
\closepath
}
}
{
\newrgbcolor{curcolor}{0 0 0}
\pscustom[linestyle=none,fillstyle=solid,fillcolor=curcolor]
{
\newpath
\moveto(10.21016429,429.39906933)
\lineto(11.17578812,429.48837558)
\curveto(11.25765051,429.03440062)(11.4139363,428.70508415)(11.64464594,428.50042519)
\curveto(11.87535146,428.29576425)(12.17117812,428.19343427)(12.5321268,428.19343496)
\curveto(12.84097433,428.19343427)(13.11168363,428.26413498)(13.34425551,428.4055373)
\curveto(13.57681988,428.54693782)(13.76752574,428.73578314)(13.91637368,428.97207383)
\curveto(14.06521295,429.20836158)(14.18986946,429.52744505)(14.2903436,429.9293252)
\curveto(14.39080833,430.3312005)(14.44104304,430.7405204)(14.4410479,431.15728614)
\curveto(14.44104304,431.20193557)(14.4391825,431.26891519)(14.43546626,431.3582252)
\curveto(14.23452254,431.0382076)(13.96009215,430.77866157)(13.61217426,430.57958633)
\curveto(13.26424832,430.38050494)(12.88748795,430.28096578)(12.48189203,430.28096856)
\curveto(11.80465075,430.28096578)(11.23160288,430.52655772)(10.76274672,431.01774512)
\curveto(10.29388819,431.50892549)(10.05945952,432.15639516)(10.05946,432.96015606)
\curveto(10.05945952,433.78995369)(10.30412119,434.45788935)(10.79344574,434.96396505)
\curveto(11.28276787,435.47002584)(11.89581746,435.72305996)(12.63259633,435.72306818)
\curveto(13.16470916,435.72305996)(13.65124168,435.579798)(14.09219536,435.29328185)
\curveto(14.53314002,435.00675013)(14.86803812,434.5983605)(15.09689067,434.06811173)
\curveto(15.3257322,433.53784984)(15.44015572,432.77037502)(15.44016157,431.76568497)
\curveto(15.44015572,430.72005441)(15.32666247,429.88746051)(15.09968149,429.26790078)
\curveto(14.87268949,428.64833753)(14.53500056,428.17668937)(14.08661372,427.85295488)
\curveto(13.63821787,427.5292197)(13.1126139,427.36735228)(12.50980024,427.36735214)
\curveto(11.86976983,427.36735228)(11.34695668,427.54503433)(10.94135922,427.90039882)
\curveto(10.53575905,428.25576253)(10.29202765,428.75531887)(10.21016429,429.39906933)
\closepath
\moveto(14.32383344,433.01039083)
\curveto(14.32382871,433.58715428)(14.17033374,434.04484835)(13.86334809,434.38347442)
\curveto(13.55635389,434.72208674)(13.1870357,434.89139634)(12.75539242,434.89140372)
\curveto(12.30885845,434.89139634)(11.92000454,434.70906293)(11.58882953,434.34440294)
\curveto(11.25765051,433.97972928)(11.09206201,433.50715084)(11.09206351,432.92666622)
\curveto(11.09206201,432.40570819)(11.24927806,431.9824342)(11.56371215,431.65684297)
\curveto(11.87814228,431.33124344)(12.26606591,431.16844576)(12.72748422,431.16844942)
\curveto(13.19261733,431.16844576)(13.57495934,431.33124344)(13.87451137,431.65684297)
\curveto(14.17405483,431.9824342)(14.32382871,432.43361637)(14.32383344,433.01039083)
\closepath
}
}
{
\newrgbcolor{curcolor}{0 0 0}
\pscustom[linestyle=none,fillstyle=solid,fillcolor=curcolor]
{
\newpath
\moveto(16.57323478,429.39906933)
\lineto(17.53885861,429.48837558)
\curveto(17.620721,429.03440062)(17.77700678,428.70508415)(18.00771643,428.50042519)
\curveto(18.23842195,428.29576425)(18.53424861,428.19343427)(18.89519729,428.19343496)
\curveto(19.20404481,428.19343427)(19.47475411,428.26413498)(19.707326,428.4055373)
\curveto(19.93989037,428.54693782)(20.13059623,428.73578314)(20.27944417,428.97207383)
\curveto(20.42828344,429.20836158)(20.55293995,429.52744505)(20.65341409,429.9293252)
\curveto(20.75387881,430.3312005)(20.80411353,430.7405204)(20.80411839,431.15728614)
\curveto(20.80411353,431.20193557)(20.80225298,431.26891519)(20.79853674,431.3582252)
\curveto(20.59759303,431.0382076)(20.32316264,430.77866157)(19.97524475,430.57958633)
\curveto(19.62731881,430.38050494)(19.25055844,430.28096578)(18.84496252,430.28096856)
\curveto(18.16772124,430.28096578)(17.59467337,430.52655772)(17.1258172,431.01774512)
\curveto(16.65695868,431.50892549)(16.42253001,432.15639516)(16.42253048,432.96015606)
\curveto(16.42253001,433.78995369)(16.66719168,434.45788935)(17.15651623,434.96396505)
\curveto(17.64583836,435.47002584)(18.25888794,435.72305996)(18.99566682,435.72306818)
\curveto(19.52777965,435.72305996)(20.01431217,435.579798)(20.45526585,435.29328185)
\curveto(20.89621051,435.00675013)(21.23110861,434.5983605)(21.45996116,434.06811173)
\curveto(21.68880269,433.53784984)(21.8032262,432.77037502)(21.80323206,431.76568497)
\curveto(21.8032262,430.72005441)(21.68973296,429.88746051)(21.46275198,429.26790078)
\curveto(21.23575997,428.64833753)(20.89807105,428.17668937)(20.4496842,427.85295488)
\curveto(20.00128835,427.5292197)(19.47568439,427.36735228)(18.87287072,427.36735214)
\curveto(18.23284031,427.36735228)(17.71002716,427.54503433)(17.30442971,427.90039882)
\curveto(16.89882953,428.25576253)(16.65509814,428.75531887)(16.57323478,429.39906933)
\closepath
\moveto(20.68690393,433.01039083)
\curveto(20.68689919,433.58715428)(20.53340423,434.04484835)(20.22641858,434.38347442)
\curveto(19.91942437,434.72208674)(19.55010619,434.89139634)(19.11846291,434.89140372)
\curveto(18.67192894,434.89139634)(18.28307503,434.70906293)(17.95190002,434.34440294)
\curveto(17.620721,433.97972928)(17.4551325,433.50715084)(17.455134,432.92666622)
\curveto(17.4551325,432.40570819)(17.61234855,431.9824342)(17.92678264,431.65684297)
\curveto(18.24121277,431.33124344)(18.6291364,431.16844576)(19.09055471,431.16844942)
\curveto(19.55568782,431.16844576)(19.93802982,431.33124344)(20.23758186,431.65684297)
\curveto(20.53712532,431.9824342)(20.68689919,432.43361637)(20.68690393,433.01039083)
\closepath
}
}
{
\newrgbcolor{curcolor}{0 0 0}
\pscustom[linestyle=none,fillstyle=solid,fillcolor=curcolor]
{
\newpath
\moveto(24.33171448,431.94429747)
\curveto(23.91495037,432.09685772)(23.6060999,432.31454149)(23.40516213,432.59734942)
\curveto(23.20422217,432.88014717)(23.10375274,433.21876637)(23.10375354,433.61320802)
\curveto(23.10375274,434.20857632)(23.31771542,434.70906293)(23.74564221,435.11466935)
\curveto(24.17356613,435.52026055)(24.7428929,435.72305996)(25.45362424,435.72306818)
\curveto(26.16807039,435.72305996)(26.7429788,435.51560919)(27.1783512,435.10071525)
\curveto(27.61371387,434.68580611)(27.83139764,434.18066814)(27.83140316,433.58529981)
\curveto(27.83139764,433.20574255)(27.73185848,432.87549581)(27.53278538,432.5945586)
\curveto(27.33370184,432.31361121)(27.03136328,432.09685772)(26.62576878,431.94429747)
\curveto(27.12811162,431.78056507)(27.51045362,431.51636767)(27.77279593,431.1517045)
\curveto(28.03512732,430.78703403)(28.16629574,430.35166649)(28.16630159,429.84560059)
\curveto(28.16629574,429.14603332)(27.91884325,428.55810109)(27.42394339,428.08180214)
\curveto(26.9290333,427.60550204)(26.27784254,427.36735228)(25.47036917,427.36735214)
\curveto(24.66288947,427.36735228)(24.01169871,427.60643232)(23.51679494,428.08459296)
\curveto(23.02188876,428.56275246)(22.77443627,429.15905714)(22.77443674,429.87350879)
\curveto(22.77443627,430.4056223)(22.90932579,430.85122283)(23.17910568,431.21031172)
\curveto(23.44888384,431.56939321)(23.83308639,431.81405488)(24.33171448,431.94429747)
\closepath
\moveto(24.13077541,433.64669786)
\curveto(24.13077359,433.25969836)(24.25543011,432.9434057)(24.50474534,432.69781895)
\curveto(24.75405617,432.45222182)(25.07779101,432.32942585)(25.47595081,432.32943067)
\curveto(25.86294101,432.32942585)(26.18016393,432.45129155)(26.42762054,432.69502813)
\curveto(26.67506891,432.93875434)(26.79879515,433.23737182)(26.79879964,433.59088145)
\curveto(26.79879515,433.95926328)(26.67134782,434.26904403)(26.41645725,434.52022462)
\curveto(26.16155848,434.77139118)(25.84433556,434.89697797)(25.46478753,434.89698536)
\curveto(25.0815121,434.89697797)(24.7633589,434.774182)(24.51032698,434.52859708)
\curveto(24.25729065,434.28299812)(24.13077359,433.989032)(24.13077541,433.64669786)
\closepath
\moveto(23.80704026,429.86792715)
\curveto(23.80703876,429.58140085)(23.87494865,429.30417965)(24.01077014,429.03626269)
\curveto(24.14658823,428.76834268)(24.34845736,428.56089191)(24.61637815,428.41390976)
\curveto(24.88429433,428.2669258)(25.1726788,428.19343427)(25.48153245,428.19343496)
\curveto(25.96154989,428.19343427)(26.35784598,428.34785951)(26.6704219,428.65671113)
\curveto(26.98298911,428.96556045)(27.13927489,429.35813545)(27.13927972,429.8344373)
\curveto(27.13927489,430.31817668)(26.97833774,430.71819386)(26.6564678,431.03449004)
\curveto(26.33458917,431.35077917)(25.93178117,431.50892549)(25.4480426,431.5089295)
\curveto(24.97546103,431.50892549)(24.58381631,431.35263971)(24.27310725,431.04007168)
\curveto(23.96239427,430.72749659)(23.80703876,430.33678213)(23.80704026,429.86792715)
\closepath
}
}
{
\newrgbcolor{curcolor}{0 0 0}
\pscustom[linestyle=none,fillstyle=solid,fillcolor=curcolor]
{
\newpath
\moveto(7.48074186,383.83684921)
\lineto(6.47604655,383.83684921)
\lineto(6.47604655,390.23899103)
\curveto(6.23417244,390.00827704)(5.91694951,389.77756946)(5.52437682,389.54686759)
\curveto(5.13179952,389.31615429)(4.77922624,389.14312361)(4.46665592,389.02777501)
\lineto(4.46665592,389.99898048)
\curveto(5.02853927,390.26317171)(5.51972315,390.58318545)(5.94020905,390.95902267)
\curveto(6.3606895,391.33484564)(6.6583767,391.69951246)(6.83327155,392.05302423)
\lineto(7.48074186,392.05302423)
\closepath
}
}
{
\newrgbcolor{curcolor}{0 0 0}
\pscustom[linestyle=none,fillstyle=solid,fillcolor=curcolor]
{
\newpath
\moveto(10.21016429,385.72902539)
\lineto(11.17578812,385.81833164)
\curveto(11.25765051,385.36435667)(11.4139363,385.03504021)(11.64464594,384.83038125)
\curveto(11.87535146,384.6257203)(12.17117812,384.52339032)(12.5321268,384.52339101)
\curveto(12.84097433,384.52339032)(13.11168363,384.59409104)(13.34425551,384.73549336)
\curveto(13.57681988,384.87689388)(13.76752574,385.0657392)(13.91637368,385.30202988)
\curveto(14.06521295,385.53831763)(14.18986946,385.8574011)(14.2903436,386.25928125)
\curveto(14.39080833,386.66115655)(14.44104304,387.07047646)(14.4410479,387.48724219)
\curveto(14.44104304,387.53189162)(14.4391825,387.59887124)(14.43546626,387.68818125)
\curveto(14.23452254,387.36816366)(13.96009215,387.10861763)(13.61217426,386.90954238)
\curveto(13.26424832,386.71046099)(12.88748795,386.61092184)(12.48189203,386.61092461)
\curveto(11.80465075,386.61092184)(11.23160288,386.85651378)(10.76274672,387.34770117)
\curveto(10.29388819,387.83888155)(10.05945952,388.48635122)(10.05946,389.29011212)
\curveto(10.05945952,390.11990974)(10.30412119,390.7878454)(10.79344574,391.29392111)
\curveto(11.28276787,391.79998189)(11.89581746,392.05301602)(12.63259633,392.05302423)
\curveto(13.16470916,392.05301602)(13.65124168,391.90975405)(14.09219536,391.6232379)
\curveto(14.53314002,391.33670618)(14.86803812,390.92831655)(15.09689067,390.39806778)
\curveto(15.3257322,389.86780589)(15.44015572,389.10033107)(15.44016157,388.09564102)
\curveto(15.44015572,387.05001046)(15.32666247,386.21741656)(15.09968149,385.59785683)
\curveto(14.87268949,384.97829358)(14.53500056,384.50664542)(14.08661372,384.18291093)
\curveto(13.63821787,383.85917575)(13.1126139,383.69730834)(12.50980024,383.6973082)
\curveto(11.86976983,383.69730834)(11.34695668,383.87499039)(10.94135922,384.23035488)
\curveto(10.53575905,384.58571858)(10.29202765,385.08527492)(10.21016429,385.72902539)
\closepath
\moveto(14.32383344,389.34034688)
\curveto(14.32382871,389.91711033)(14.17033374,390.37480441)(13.86334809,390.71343048)
\curveto(13.55635389,391.0520428)(13.1870357,391.22135239)(12.75539242,391.22135978)
\curveto(12.30885845,391.22135239)(11.92000454,391.03901898)(11.58882953,390.67435899)
\curveto(11.25765051,390.30968533)(11.09206201,389.8371069)(11.09206351,389.25662227)
\curveto(11.09206201,388.73566425)(11.24927806,388.31239026)(11.56371215,387.98679903)
\curveto(11.87814228,387.6611995)(12.26606591,387.49840181)(12.72748422,387.49840547)
\curveto(13.19261733,387.49840181)(13.57495934,387.6611995)(13.87451137,387.98679903)
\curveto(14.17405483,388.31239026)(14.32382871,388.76357242)(14.32383344,389.34034688)
\closepath
}
}
{
\newrgbcolor{curcolor}{0 0 0}
\pscustom[linestyle=none,fillstyle=solid,fillcolor=curcolor]
{
\newpath
\moveto(16.57323478,385.72902539)
\lineto(17.53885861,385.81833164)
\curveto(17.620721,385.36435667)(17.77700678,385.03504021)(18.00771643,384.83038125)
\curveto(18.23842195,384.6257203)(18.53424861,384.52339032)(18.89519729,384.52339101)
\curveto(19.20404481,384.52339032)(19.47475411,384.59409104)(19.707326,384.73549336)
\curveto(19.93989037,384.87689388)(20.13059623,385.0657392)(20.27944417,385.30202988)
\curveto(20.42828344,385.53831763)(20.55293995,385.8574011)(20.65341409,386.25928125)
\curveto(20.75387881,386.66115655)(20.80411353,387.07047646)(20.80411839,387.48724219)
\curveto(20.80411353,387.53189162)(20.80225298,387.59887124)(20.79853674,387.68818125)
\curveto(20.59759303,387.36816366)(20.32316264,387.10861763)(19.97524475,386.90954238)
\curveto(19.62731881,386.71046099)(19.25055844,386.61092184)(18.84496252,386.61092461)
\curveto(18.16772124,386.61092184)(17.59467337,386.85651378)(17.1258172,387.34770117)
\curveto(16.65695868,387.83888155)(16.42253001,388.48635122)(16.42253048,389.29011212)
\curveto(16.42253001,390.11990974)(16.66719168,390.7878454)(17.15651623,391.29392111)
\curveto(17.64583836,391.79998189)(18.25888794,392.05301602)(18.99566682,392.05302423)
\curveto(19.52777965,392.05301602)(20.01431217,391.90975405)(20.45526585,391.6232379)
\curveto(20.89621051,391.33670618)(21.23110861,390.92831655)(21.45996116,390.39806778)
\curveto(21.68880269,389.86780589)(21.8032262,389.10033107)(21.80323206,388.09564102)
\curveto(21.8032262,387.05001046)(21.68973296,386.21741656)(21.46275198,385.59785683)
\curveto(21.23575997,384.97829358)(20.89807105,384.50664542)(20.4496842,384.18291093)
\curveto(20.00128835,383.85917575)(19.47568439,383.69730834)(18.87287072,383.6973082)
\curveto(18.23284031,383.69730834)(17.71002716,383.87499039)(17.30442971,384.23035488)
\curveto(16.89882953,384.58571858)(16.65509814,385.08527492)(16.57323478,385.72902539)
\closepath
\moveto(20.68690393,389.34034688)
\curveto(20.68689919,389.91711033)(20.53340423,390.37480441)(20.22641858,390.71343048)
\curveto(19.91942437,391.0520428)(19.55010619,391.22135239)(19.11846291,391.22135978)
\curveto(18.67192894,391.22135239)(18.28307503,391.03901898)(17.95190002,390.67435899)
\curveto(17.620721,390.30968533)(17.4551325,389.8371069)(17.455134,389.25662227)
\curveto(17.4551325,388.73566425)(17.61234855,388.31239026)(17.92678264,387.98679903)
\curveto(18.24121277,387.6611995)(18.6291364,387.49840181)(19.09055471,387.49840547)
\curveto(19.55568782,387.49840181)(19.93802982,387.6611995)(20.23758186,387.98679903)
\curveto(20.53712532,388.31239026)(20.68689919,388.76357242)(20.68690393,389.34034688)
\closepath
}
}
{
\newrgbcolor{curcolor}{0 0 0}
\pscustom[linestyle=none,fillstyle=solid,fillcolor=curcolor]
{
\newpath
\moveto(22.93630432,385.72902539)
\lineto(23.90192815,385.81833164)
\curveto(23.98379054,385.36435667)(24.14007632,385.03504021)(24.37078596,384.83038125)
\curveto(24.60149148,384.6257203)(24.89731814,384.52339032)(25.25826682,384.52339101)
\curveto(25.56711435,384.52339032)(25.83782365,384.59409104)(26.07039553,384.73549336)
\curveto(26.3029599,384.87689388)(26.49366577,385.0657392)(26.6425137,385.30202988)
\curveto(26.79135297,385.53831763)(26.91600949,385.8574011)(27.01648362,386.25928125)
\curveto(27.11694835,386.66115655)(27.16718306,387.07047646)(27.16718792,387.48724219)
\curveto(27.16718306,387.53189162)(27.16532252,387.59887124)(27.16160628,387.68818125)
\curveto(26.96066257,387.36816366)(26.68623218,387.10861763)(26.33831429,386.90954238)
\curveto(25.99038834,386.71046099)(25.61362797,386.61092184)(25.20803206,386.61092461)
\curveto(24.53079077,386.61092184)(23.95774291,386.85651378)(23.48888674,387.34770117)
\curveto(23.02002822,387.83888155)(22.78559954,388.48635122)(22.78560002,389.29011212)
\curveto(22.78559954,390.11990974)(23.03026121,390.7878454)(23.51958576,391.29392111)
\curveto(24.00890789,391.79998189)(24.62195748,392.05301602)(25.35873635,392.05302423)
\curveto(25.89084918,392.05301602)(26.3773817,391.90975405)(26.81833538,391.6232379)
\curveto(27.25928004,391.33670618)(27.59417815,390.92831655)(27.82303069,390.39806778)
\curveto(28.05187222,389.86780589)(28.16629574,389.10033107)(28.16630159,388.09564102)
\curveto(28.16629574,387.05001046)(28.05280249,386.21741656)(27.82582152,385.59785683)
\curveto(27.59882951,384.97829358)(27.26114059,384.50664542)(26.81275374,384.18291093)
\curveto(26.36435789,383.85917575)(25.83875392,383.69730834)(25.23594026,383.6973082)
\curveto(24.59590985,383.69730834)(24.0730967,383.87499039)(23.66749924,384.23035488)
\curveto(23.26189907,384.58571858)(23.01816767,385.08527492)(22.93630432,385.72902539)
\closepath
\moveto(27.04997347,389.34034688)
\curveto(27.04996873,389.91711033)(26.89647376,390.37480441)(26.58948811,390.71343048)
\curveto(26.28249391,391.0520428)(25.91317572,391.22135239)(25.48153245,391.22135978)
\curveto(25.03499847,391.22135239)(24.64614456,391.03901898)(24.31496955,390.67435899)
\curveto(23.98379054,390.30968533)(23.81820203,389.8371069)(23.81820354,389.25662227)
\curveto(23.81820203,388.73566425)(23.97541808,388.31239026)(24.28985217,387.98679903)
\curveto(24.6042823,387.6611995)(24.99220594,387.49840181)(25.45362424,387.49840547)
\curveto(25.91875736,387.49840181)(26.30109936,387.6611995)(26.6006514,387.98679903)
\curveto(26.90019485,388.31239026)(27.04996873,388.76357242)(27.04997347,389.34034688)
\closepath
}
}
{
\newrgbcolor{curcolor}{0 0 0}
\pscustom[linestyle=none,fillstyle=solid,fillcolor=curcolor]
{
\newpath
\moveto(8.97662155,341.13255117)
\lineto(8.97662155,340.16692734)
\lineto(3.56801177,340.16692734)
\curveto(3.56056925,340.40879819)(3.59964069,340.64136632)(3.68522623,340.86463242)
\curveto(3.8229061,341.23301963)(4.04338068,341.59582591)(4.34665064,341.95305234)
\curveto(4.64991836,342.3102752)(5.08807671,342.72331619)(5.66112701,343.19217656)
\curveto(6.55046509,343.92150719)(7.15142113,344.49920641)(7.46399694,344.92527598)
\curveto(7.77656426,345.35133603)(7.93285004,345.75414403)(7.93285475,346.13370118)
\curveto(7.93285004,346.53185185)(7.79051835,346.86768022)(7.50585924,347.14118732)
\curveto(7.22119157,347.41468046)(6.85001284,347.55143052)(6.39232194,347.5514379)
\curveto(5.90857706,347.55143052)(5.5215837,347.40630801)(5.23134068,347.11606993)
\curveto(4.94109365,346.82581796)(4.7941106,346.42394024)(4.79039107,345.91043555)
\lineto(3.75778756,346.01648673)
\curveto(3.82848773,346.78674652)(4.09454567,347.37374847)(4.55596217,347.77749435)
\curveto(5.017376,348.18122501)(5.63693749,348.38309414)(6.4146485,348.38310236)
\curveto(7.1997953,348.38309414)(7.82121734,348.16541037)(8.27891647,347.7300504)
\curveto(8.73660549,347.29467531)(8.96545253,346.75511725)(8.96545827,346.11137462)
\curveto(8.96545253,345.78391275)(8.89847291,345.46203846)(8.76451921,345.14575079)
\curveto(8.63055442,344.82945316)(8.40821929,344.4964156)(8.09751315,344.14663711)
\curveto(7.78679726,343.79685067)(7.27049602,343.31683005)(6.54860787,342.70657383)
\curveto(5.94578796,342.20050304)(5.5587946,341.85723249)(5.38762662,341.67676113)
\curveto(5.21645431,341.49628675)(5.07505289,341.31488361)(4.96342193,341.13255117)
\closepath
}
}
{
\newrgbcolor{curcolor}{0 0 0}
\pscustom[linestyle=none,fillstyle=solid,fillcolor=curcolor]
{
\newpath
\moveto(10.05946,344.20245352)
\curveto(10.05945952,345.16993289)(10.15899868,345.94857098)(10.35807777,346.53837013)
\curveto(10.55715531,347.12815653)(10.85298197,347.58305978)(11.24555863,347.90308126)
\curveto(11.63813197,348.22308727)(12.13210667,348.38309414)(12.72748422,348.38310236)
\curveto(13.1665697,348.38309414)(13.55170252,348.29471825)(13.88288383,348.11797443)
\curveto(14.21405655,347.9412147)(14.48755667,347.68632003)(14.70338501,347.35328966)
\curveto(14.91920311,347.02024492)(15.08851271,346.6146461)(15.21131431,346.136492)
\curveto(15.33410465,345.65832596)(15.39550264,345.01364711)(15.39550845,344.20245352)
\curveto(15.39550264,343.24240825)(15.29689375,342.46749125)(15.09968149,341.87770019)
\curveto(14.90245821,341.28790571)(14.60756182,340.83207218)(14.21499145,340.51019824)
\curveto(13.82241182,340.18832361)(13.32657658,340.02738646)(12.72748422,340.02738632)
\curveto(11.93860999,340.02738646)(11.3190485,340.3101893)(10.86879789,340.8757957)
\curveto(10.32923855,341.55675447)(10.05945952,342.6656393)(10.05946,344.20245352)
\closepath
\moveto(11.09206351,344.20245352)
\curveto(11.09206201,342.85913598)(11.24927806,341.9651441)(11.56371215,341.52047519)
\curveto(11.87814228,341.07580358)(12.26606591,340.85346845)(12.72748422,340.85346914)
\curveto(13.18889624,340.85346845)(13.57681988,341.07673385)(13.89125629,341.52326601)
\curveto(14.2056841,341.96979546)(14.36290015,342.86285707)(14.36290493,344.20245352)
\curveto(14.36290015,345.54948408)(14.2056841,346.44440623)(13.89125629,346.88722267)
\curveto(13.57681988,347.33002566)(13.18517515,347.55143052)(12.71632094,347.5514379)
\curveto(12.25490264,347.55143052)(11.88651473,347.35607329)(11.61115609,346.96536564)
\curveto(11.26509269,346.46673277)(11.09206201,345.54576299)(11.09206351,344.20245352)
\closepath
}
}
{
\newrgbcolor{curcolor}{0 0 0}
\pscustom[linestyle=none,fillstyle=solid,fillcolor=curcolor]
{
\newpath
\moveto(16.42253048,344.20245352)
\curveto(16.42253001,345.16993289)(16.52206917,345.94857098)(16.72114826,346.53837013)
\curveto(16.9202258,347.12815653)(17.21605246,347.58305978)(17.60862912,347.90308126)
\curveto(18.00120246,348.22308727)(18.49517716,348.38309414)(19.09055471,348.38310236)
\curveto(19.52964019,348.38309414)(19.91477301,348.29471825)(20.24595432,348.11797443)
\curveto(20.57712704,347.9412147)(20.85062715,347.68632003)(21.0664555,347.35328966)
\curveto(21.2822736,347.02024492)(21.4515832,346.6146461)(21.57438479,346.136492)
\curveto(21.69717514,345.65832596)(21.75857312,345.01364711)(21.75857893,344.20245352)
\curveto(21.75857312,343.24240825)(21.65996424,342.46749125)(21.46275198,341.87770019)
\curveto(21.26552869,341.28790571)(20.97063231,340.83207218)(20.57806194,340.51019824)
\curveto(20.18548231,340.18832361)(19.68964706,340.02738646)(19.09055471,340.02738632)
\curveto(18.30168048,340.02738646)(17.68211899,340.3101893)(17.23186838,340.8757957)
\curveto(16.69230904,341.55675447)(16.42253001,342.6656393)(16.42253048,344.20245352)
\closepath
\moveto(17.455134,344.20245352)
\curveto(17.4551325,342.85913598)(17.61234855,341.9651441)(17.92678264,341.52047519)
\curveto(18.24121277,341.07580358)(18.6291364,340.85346845)(19.09055471,340.85346914)
\curveto(19.55196673,340.85346845)(19.93989037,341.07673385)(20.25432678,341.52326601)
\curveto(20.56875458,341.96979546)(20.72597064,342.86285707)(20.72597542,344.20245352)
\curveto(20.72597064,345.54948408)(20.56875458,346.44440623)(20.25432678,346.88722267)
\curveto(19.93989037,347.33002566)(19.54824564,347.55143052)(19.07939143,347.5514379)
\curveto(18.61797313,347.55143052)(18.24958522,347.35607329)(17.97422658,346.96536564)
\curveto(17.62816318,346.46673277)(17.4551325,345.54576299)(17.455134,344.20245352)
\closepath
}
}
{
\newrgbcolor{curcolor}{0 0 0}
\pscustom[linestyle=none,fillstyle=solid,fillcolor=curcolor]
{
\newpath
\moveto(22.78560002,344.20245352)
\curveto(22.78559954,345.16993289)(22.8851387,345.94857098)(23.08421779,346.53837013)
\curveto(23.28329534,347.12815653)(23.57912199,347.58305978)(23.97169865,347.90308126)
\curveto(24.36427199,348.22308727)(24.8582467,348.38309414)(25.45362424,348.38310236)
\curveto(25.89270973,348.38309414)(26.27784254,348.29471825)(26.60902386,348.11797443)
\curveto(26.94019657,347.9412147)(27.21369669,347.68632003)(27.42952503,347.35328966)
\curveto(27.64534313,347.02024492)(27.81465273,346.6146461)(27.93745433,346.136492)
\curveto(28.06024467,345.65832596)(28.12164266,345.01364711)(28.12164847,344.20245352)
\curveto(28.12164266,343.24240825)(28.02303377,342.46749125)(27.82582152,341.87770019)
\curveto(27.62859823,341.28790571)(27.33370184,340.83207218)(26.94113147,340.51019824)
\curveto(26.54855185,340.18832361)(26.0527166,340.02738646)(25.45362424,340.02738632)
\curveto(24.66475001,340.02738646)(24.04518852,340.3101893)(23.59493791,340.8757957)
\curveto(23.05537857,341.55675447)(22.78559954,342.6656393)(22.78560002,344.20245352)
\closepath
\moveto(23.81820354,344.20245352)
\curveto(23.81820203,342.85913598)(23.97541808,341.9651441)(24.28985217,341.52047519)
\curveto(24.6042823,341.07580358)(24.99220594,340.85346845)(25.45362424,340.85346914)
\curveto(25.91503627,340.85346845)(26.3029599,341.07673385)(26.61739632,341.52326601)
\curveto(26.93182412,341.96979546)(27.08904017,342.86285707)(27.08904495,344.20245352)
\curveto(27.08904017,345.54948408)(26.93182412,346.44440623)(26.61739632,346.88722267)
\curveto(26.3029599,347.33002566)(25.91131518,347.55143052)(25.44246096,347.5514379)
\curveto(24.98104267,347.55143052)(24.61265475,347.35607329)(24.33729612,346.96536564)
\curveto(23.99123272,346.46673277)(23.81820203,345.54576299)(23.81820354,344.20245352)
\closepath
}
}
{
\newrgbcolor{curcolor}{0 0 0}
\pscustom[linestyle=none,fillstyle=solid,fillcolor=curcolor]
{
\newpath
\moveto(8.97662155,297.47251699)
\lineto(8.97662155,296.50689316)
\lineto(3.56801177,296.50689316)
\curveto(3.56056925,296.74876401)(3.59964069,296.98133214)(3.68522623,297.20459824)
\curveto(3.8229061,297.57298545)(4.04338068,297.93579173)(4.34665064,298.29301816)
\curveto(4.64991836,298.65024102)(5.08807671,299.06328201)(5.66112701,299.53214238)
\curveto(6.55046509,300.26147301)(7.15142113,300.83917223)(7.46399694,301.2652418)
\curveto(7.77656426,301.69130185)(7.93285004,302.09410985)(7.93285475,302.473667)
\curveto(7.93285004,302.87181767)(7.79051835,303.20764604)(7.50585924,303.48115314)
\curveto(7.22119157,303.75464628)(6.85001284,303.89139634)(6.39232194,303.89140372)
\curveto(5.90857706,303.89139634)(5.5215837,303.74627383)(5.23134068,303.45603575)
\curveto(4.94109365,303.16578378)(4.7941106,302.76390606)(4.79039107,302.25040138)
\lineto(3.75778756,302.35645255)
\curveto(3.82848773,303.12671234)(4.09454567,303.71371429)(4.55596217,304.11746017)
\curveto(5.017376,304.52119083)(5.63693749,304.72305996)(6.4146485,304.72306818)
\curveto(7.1997953,304.72305996)(7.82121734,304.50537619)(8.27891647,304.07001622)
\curveto(8.73660549,303.63464113)(8.96545253,303.09508307)(8.96545827,302.45134044)
\curveto(8.96545253,302.12387857)(8.89847291,301.80200428)(8.76451921,301.48571661)
\curveto(8.63055442,301.16941898)(8.40821929,300.83638142)(8.09751315,300.48660293)
\curveto(7.78679726,300.13681649)(7.27049602,299.65679588)(6.54860787,299.04653965)
\curveto(5.94578796,298.54046886)(5.5587946,298.19719831)(5.38762662,298.01672695)
\curveto(5.21645431,297.83625257)(5.07505289,297.65484943)(4.96342193,297.47251699)
\closepath
}
}
{
\newrgbcolor{curcolor}{0 0 0}
\pscustom[linestyle=none,fillstyle=solid,fillcolor=curcolor]
{
\newpath
\moveto(10.05946,300.54241934)
\curveto(10.05945952,301.50989871)(10.15899868,302.2885368)(10.35807777,302.87833595)
\curveto(10.55715531,303.46812235)(10.85298197,303.9230256)(11.24555863,304.24304708)
\curveto(11.63813197,304.56305309)(12.13210667,304.72305996)(12.72748422,304.72306818)
\curveto(13.1665697,304.72305996)(13.55170252,304.63468407)(13.88288383,304.45794025)
\curveto(14.21405655,304.28118052)(14.48755667,304.02628585)(14.70338501,303.69325548)
\curveto(14.91920311,303.36021074)(15.08851271,302.95461192)(15.21131431,302.47645782)
\curveto(15.33410465,301.99829178)(15.39550264,301.35361293)(15.39550845,300.54241934)
\curveto(15.39550264,299.58237407)(15.29689375,298.80745707)(15.09968149,298.21766601)
\curveto(14.90245821,297.62787153)(14.60756182,297.172038)(14.21499145,296.85016406)
\curveto(13.82241182,296.52828943)(13.32657658,296.36735228)(12.72748422,296.36735214)
\curveto(11.93860999,296.36735228)(11.3190485,296.65015512)(10.86879789,297.21576152)
\curveto(10.32923855,297.89672029)(10.05945952,299.00560512)(10.05946,300.54241934)
\closepath
\moveto(11.09206351,300.54241934)
\curveto(11.09206201,299.1991018)(11.24927806,298.30510992)(11.56371215,297.86044101)
\curveto(11.87814228,297.4157694)(12.26606591,297.19343427)(12.72748422,297.19343496)
\curveto(13.18889624,297.19343427)(13.57681988,297.41669967)(13.89125629,297.86323183)
\curveto(14.2056841,298.30976128)(14.36290015,299.20282289)(14.36290493,300.54241934)
\curveto(14.36290015,301.8894499)(14.2056841,302.78437205)(13.89125629,303.22718849)
\curveto(13.57681988,303.66999148)(13.18517515,303.89139634)(12.71632094,303.89140372)
\curveto(12.25490264,303.89139634)(11.88651473,303.69603911)(11.61115609,303.30533146)
\curveto(11.26509269,302.80669859)(11.09206201,301.88572881)(11.09206351,300.54241934)
\closepath
}
}
{
\newrgbcolor{curcolor}{0 0 0}
\pscustom[linestyle=none,fillstyle=solid,fillcolor=curcolor]
{
\newpath
\moveto(16.42253048,300.54241934)
\curveto(16.42253001,301.50989871)(16.52206917,302.2885368)(16.72114826,302.87833595)
\curveto(16.9202258,303.46812235)(17.21605246,303.9230256)(17.60862912,304.24304708)
\curveto(18.00120246,304.56305309)(18.49517716,304.72305996)(19.09055471,304.72306818)
\curveto(19.52964019,304.72305996)(19.91477301,304.63468407)(20.24595432,304.45794025)
\curveto(20.57712704,304.28118052)(20.85062715,304.02628585)(21.0664555,303.69325548)
\curveto(21.2822736,303.36021074)(21.4515832,302.95461192)(21.57438479,302.47645782)
\curveto(21.69717514,301.99829178)(21.75857312,301.35361293)(21.75857893,300.54241934)
\curveto(21.75857312,299.58237407)(21.65996424,298.80745707)(21.46275198,298.21766601)
\curveto(21.26552869,297.62787153)(20.97063231,297.172038)(20.57806194,296.85016406)
\curveto(20.18548231,296.52828943)(19.68964706,296.36735228)(19.09055471,296.36735214)
\curveto(18.30168048,296.36735228)(17.68211899,296.65015512)(17.23186838,297.21576152)
\curveto(16.69230904,297.89672029)(16.42253001,299.00560512)(16.42253048,300.54241934)
\closepath
\moveto(17.455134,300.54241934)
\curveto(17.4551325,299.1991018)(17.61234855,298.30510992)(17.92678264,297.86044101)
\curveto(18.24121277,297.4157694)(18.6291364,297.19343427)(19.09055471,297.19343496)
\curveto(19.55196673,297.19343427)(19.93989037,297.41669967)(20.25432678,297.86323183)
\curveto(20.56875458,298.30976128)(20.72597064,299.20282289)(20.72597542,300.54241934)
\curveto(20.72597064,301.8894499)(20.56875458,302.78437205)(20.25432678,303.22718849)
\curveto(19.93989037,303.66999148)(19.54824564,303.89139634)(19.07939143,303.89140372)
\curveto(18.61797313,303.89139634)(18.24958522,303.69603911)(17.97422658,303.30533146)
\curveto(17.62816318,302.80669859)(17.4551325,301.88572881)(17.455134,300.54241934)
\closepath
}
}
{
\newrgbcolor{curcolor}{0 0 0}
\pscustom[linestyle=none,fillstyle=solid,fillcolor=curcolor]
{
\newpath
\moveto(26.56995237,296.50689316)
\lineto(25.56525706,296.50689316)
\lineto(25.56525706,302.90903497)
\curveto(25.32338295,302.67832099)(25.00616002,302.4476134)(24.61358733,302.21691153)
\curveto(24.22101003,301.98619824)(23.86843675,301.81316755)(23.55586643,301.69781895)
\lineto(23.55586643,302.66902442)
\curveto(24.11774978,302.93321565)(24.60893366,303.2532294)(25.02941956,303.62906661)
\curveto(25.44990001,304.00488958)(25.74758721,304.36955641)(25.92248206,304.72306818)
\lineto(26.56995237,304.72306818)
\closepath
}
}
{
\newrgbcolor{curcolor}{0 0 0}
\pscustom[linestyle=none,fillstyle=solid,fillcolor=curcolor]
{
\newpath
\moveto(8.97662155,253.80247304)
\lineto(8.97662155,252.83684921)
\lineto(3.56801177,252.83684921)
\curveto(3.56056925,253.07872007)(3.59964069,253.31128819)(3.68522623,253.53455429)
\curveto(3.8229061,253.90294151)(4.04338068,254.26574779)(4.34665064,254.62297422)
\curveto(4.64991836,254.98019707)(5.08807671,255.39323807)(5.66112701,255.86209844)
\curveto(6.55046509,256.59142906)(7.15142113,257.16912829)(7.46399694,257.59519786)
\curveto(7.77656426,258.02125791)(7.93285004,258.4240659)(7.93285475,258.80362306)
\curveto(7.93285004,259.20177372)(7.79051835,259.5376021)(7.50585924,259.81110919)
\curveto(7.22119157,260.08460233)(6.85001284,260.22135239)(6.39232194,260.22135978)
\curveto(5.90857706,260.22135239)(5.5215837,260.07622988)(5.23134068,259.78599181)
\curveto(4.94109365,259.49573984)(4.7941106,259.09386211)(4.79039107,258.58035743)
\lineto(3.75778756,258.6864086)
\curveto(3.82848773,259.45666839)(4.09454567,260.04367034)(4.55596217,260.44741622)
\curveto(5.017376,260.85114688)(5.63693749,261.05301602)(6.4146485,261.05302423)
\curveto(7.1997953,261.05301602)(7.82121734,260.83533225)(8.27891647,260.39997228)
\curveto(8.73660549,259.96459718)(8.96545253,259.42503912)(8.96545827,258.78129649)
\curveto(8.96545253,258.45383463)(8.89847291,258.13196034)(8.76451921,257.81567266)
\curveto(8.63055442,257.49937503)(8.40821929,257.16633747)(8.09751315,256.81655899)
\curveto(7.78679726,256.46677254)(7.27049602,255.98675193)(6.54860787,255.3764957)
\curveto(5.94578796,254.87042492)(5.5587946,254.52715436)(5.38762662,254.34668301)
\curveto(5.21645431,254.16620863)(5.07505289,253.98480549)(4.96342193,253.80247304)
\closepath
}
}
{
\newrgbcolor{curcolor}{0 0 0}
\pscustom[linestyle=none,fillstyle=solid,fillcolor=curcolor]
{
\newpath
\moveto(10.05946,256.87237539)
\curveto(10.05945952,257.83985477)(10.15899868,258.61849286)(10.35807777,259.208292)
\curveto(10.55715531,259.7980784)(10.85298197,260.25298166)(11.24555863,260.57300314)
\curveto(11.63813197,260.89300914)(12.13210667,261.05301602)(12.72748422,261.05302423)
\curveto(13.1665697,261.05301602)(13.55170252,260.96464013)(13.88288383,260.7878963)
\curveto(14.21405655,260.61113657)(14.48755667,260.35624191)(14.70338501,260.02321153)
\curveto(14.91920311,259.69016679)(15.08851271,259.28456798)(15.21131431,258.80641388)
\curveto(15.33410465,258.32824784)(15.39550264,257.68356899)(15.39550845,256.87237539)
\curveto(15.39550264,255.91233013)(15.29689375,255.13741313)(15.09968149,254.54762207)
\curveto(14.90245821,253.95782759)(14.60756182,253.50199406)(14.21499145,253.18012011)
\curveto(13.82241182,252.85824548)(13.32657658,252.69730834)(12.72748422,252.6973082)
\curveto(11.93860999,252.69730834)(11.3190485,252.98011118)(10.86879789,253.54571757)
\curveto(10.32923855,254.22667634)(10.05945952,255.33556117)(10.05946,256.87237539)
\closepath
\moveto(11.09206351,256.87237539)
\curveto(11.09206201,255.52905786)(11.24927806,254.63506597)(11.56371215,254.19039707)
\curveto(11.87814228,253.74572545)(12.26606591,253.52339032)(12.72748422,253.52339101)
\curveto(13.18889624,253.52339032)(13.57681988,253.74665573)(13.89125629,254.19318789)
\curveto(14.2056841,254.63971734)(14.36290015,255.53277895)(14.36290493,256.87237539)
\curveto(14.36290015,258.21940595)(14.2056841,259.11432811)(13.89125629,259.55714454)
\curveto(13.57681988,259.99994754)(13.18517515,260.22135239)(12.71632094,260.22135978)
\curveto(12.25490264,260.22135239)(11.88651473,260.02599517)(11.61115609,259.63528751)
\curveto(11.26509269,259.13665465)(11.09206201,258.21568486)(11.09206351,256.87237539)
\closepath
}
}
{
\newrgbcolor{curcolor}{0 0 0}
\pscustom[linestyle=none,fillstyle=solid,fillcolor=curcolor]
{
\newpath
\moveto(16.42253048,256.87237539)
\curveto(16.42253001,257.83985477)(16.52206917,258.61849286)(16.72114826,259.208292)
\curveto(16.9202258,259.7980784)(17.21605246,260.25298166)(17.60862912,260.57300314)
\curveto(18.00120246,260.89300914)(18.49517716,261.05301602)(19.09055471,261.05302423)
\curveto(19.52964019,261.05301602)(19.91477301,260.96464013)(20.24595432,260.7878963)
\curveto(20.57712704,260.61113657)(20.85062715,260.35624191)(21.0664555,260.02321153)
\curveto(21.2822736,259.69016679)(21.4515832,259.28456798)(21.57438479,258.80641388)
\curveto(21.69717514,258.32824784)(21.75857312,257.68356899)(21.75857893,256.87237539)
\curveto(21.75857312,255.91233013)(21.65996424,255.13741313)(21.46275198,254.54762207)
\curveto(21.26552869,253.95782759)(20.97063231,253.50199406)(20.57806194,253.18012011)
\curveto(20.18548231,252.85824548)(19.68964706,252.69730834)(19.09055471,252.6973082)
\curveto(18.30168048,252.69730834)(17.68211899,252.98011118)(17.23186838,253.54571757)
\curveto(16.69230904,254.22667634)(16.42253001,255.33556117)(16.42253048,256.87237539)
\closepath
\moveto(17.455134,256.87237539)
\curveto(17.4551325,255.52905786)(17.61234855,254.63506597)(17.92678264,254.19039707)
\curveto(18.24121277,253.74572545)(18.6291364,253.52339032)(19.09055471,253.52339101)
\curveto(19.55196673,253.52339032)(19.93989037,253.74665573)(20.25432678,254.19318789)
\curveto(20.56875458,254.63971734)(20.72597064,255.53277895)(20.72597542,256.87237539)
\curveto(20.72597064,258.21940595)(20.56875458,259.11432811)(20.25432678,259.55714454)
\curveto(19.93989037,259.99994754)(19.54824564,260.22135239)(19.07939143,260.22135978)
\curveto(18.61797313,260.22135239)(18.24958522,260.02599517)(17.97422658,259.63528751)
\curveto(17.62816318,259.13665465)(17.4551325,258.21568486)(17.455134,256.87237539)
\closepath
}
}
{
\newrgbcolor{curcolor}{0 0 0}
\pscustom[linestyle=none,fillstyle=solid,fillcolor=curcolor]
{
\newpath
\moveto(28.06583206,253.80247304)
\lineto(28.06583206,252.83684921)
\lineto(22.65722228,252.83684921)
\curveto(22.64977976,253.07872007)(22.6888512,253.31128819)(22.77443674,253.53455429)
\curveto(22.91211661,253.90294151)(23.13259119,254.26574779)(23.43586115,254.62297422)
\curveto(23.73912887,254.98019707)(24.17728722,255.39323807)(24.75033752,255.86209844)
\curveto(25.6396756,256.59142906)(26.24063164,257.16912829)(26.55320745,257.59519786)
\curveto(26.86577477,258.02125791)(27.02206055,258.4240659)(27.02206526,258.80362306)
\curveto(27.02206055,259.20177372)(26.87972886,259.5376021)(26.59506975,259.81110919)
\curveto(26.31040208,260.08460233)(25.93922335,260.22135239)(25.48153245,260.22135978)
\curveto(24.99778757,260.22135239)(24.61079421,260.07622988)(24.32055119,259.78599181)
\curveto(24.03030416,259.49573984)(23.88332111,259.09386211)(23.87960158,258.58035743)
\lineto(22.84699807,258.6864086)
\curveto(22.91769824,259.45666839)(23.18375618,260.04367034)(23.64517268,260.44741622)
\curveto(24.10658651,260.85114688)(24.726148,261.05301602)(25.50385901,261.05302423)
\curveto(26.28900581,261.05301602)(26.91042785,260.83533225)(27.36812698,260.39997228)
\curveto(27.825816,259.96459718)(28.05466304,259.42503912)(28.05466878,258.78129649)
\curveto(28.05466304,258.45383463)(27.98768342,258.13196034)(27.85372972,257.81567266)
\curveto(27.71976493,257.49937503)(27.4974298,257.16633747)(27.18672366,256.81655899)
\curveto(26.87600777,256.46677254)(26.35970653,255.98675193)(25.63781838,255.3764957)
\curveto(25.03499847,254.87042492)(24.64800511,254.52715436)(24.47683713,254.34668301)
\curveto(24.30566482,254.16620863)(24.1642634,253.98480549)(24.05263244,253.80247304)
\closepath
}
}
{
\newrgbcolor{curcolor}{0 0 0}
\pscustom[linestyle=none,fillstyle=solid,fillcolor=curcolor]
{
\newpath
\moveto(8.97662155,210.13255117)
\lineto(8.97662155,209.16692734)
\lineto(3.56801177,209.16692734)
\curveto(3.56056925,209.40879819)(3.59964069,209.64136632)(3.68522623,209.86463242)
\curveto(3.8229061,210.23301963)(4.04338068,210.59582591)(4.34665064,210.95305234)
\curveto(4.64991836,211.3102752)(5.08807671,211.72331619)(5.66112701,212.19217656)
\curveto(6.55046509,212.92150719)(7.15142113,213.49920641)(7.46399694,213.92527598)
\curveto(7.77656426,214.35133603)(7.93285004,214.75414403)(7.93285475,215.13370118)
\curveto(7.93285004,215.53185185)(7.79051835,215.86768022)(7.50585924,216.14118732)
\curveto(7.22119157,216.41468046)(6.85001284,216.55143052)(6.39232194,216.5514379)
\curveto(5.90857706,216.55143052)(5.5215837,216.40630801)(5.23134068,216.11606993)
\curveto(4.94109365,215.82581796)(4.7941106,215.42394024)(4.79039107,214.91043555)
\lineto(3.75778756,215.01648673)
\curveto(3.82848773,215.78674652)(4.09454567,216.37374847)(4.55596217,216.77749435)
\curveto(5.017376,217.18122501)(5.63693749,217.38309414)(6.4146485,217.38310236)
\curveto(7.1997953,217.38309414)(7.82121734,217.16541037)(8.27891647,216.7300504)
\curveto(8.73660549,216.29467531)(8.96545253,215.75511725)(8.96545827,215.11137462)
\curveto(8.96545253,214.78391275)(8.89847291,214.46203846)(8.76451921,214.14575079)
\curveto(8.63055442,213.82945316)(8.40821929,213.4964156)(8.09751315,213.14663711)
\curveto(7.78679726,212.79685067)(7.27049602,212.31683005)(6.54860787,211.70657383)
\curveto(5.94578796,211.20050304)(5.5587946,210.85723249)(5.38762662,210.67676113)
\curveto(5.21645431,210.49628675)(5.07505289,210.31488361)(4.96342193,210.13255117)
\closepath
}
}
{
\newrgbcolor{curcolor}{0 0 0}
\pscustom[linestyle=none,fillstyle=solid,fillcolor=curcolor]
{
\newpath
\moveto(10.05946,213.20245352)
\curveto(10.05945952,214.16993289)(10.15899868,214.94857098)(10.35807777,215.53837013)
\curveto(10.55715531,216.12815653)(10.85298197,216.58305978)(11.24555863,216.90308126)
\curveto(11.63813197,217.22308727)(12.13210667,217.38309414)(12.72748422,217.38310236)
\curveto(13.1665697,217.38309414)(13.55170252,217.29471825)(13.88288383,217.11797443)
\curveto(14.21405655,216.9412147)(14.48755667,216.68632003)(14.70338501,216.35328966)
\curveto(14.91920311,216.02024492)(15.08851271,215.6146461)(15.21131431,215.136492)
\curveto(15.33410465,214.65832596)(15.39550264,214.01364711)(15.39550845,213.20245352)
\curveto(15.39550264,212.24240825)(15.29689375,211.46749125)(15.09968149,210.87770019)
\curveto(14.90245821,210.28790571)(14.60756182,209.83207218)(14.21499145,209.51019824)
\curveto(13.82241182,209.18832361)(13.32657658,209.02738646)(12.72748422,209.02738632)
\curveto(11.93860999,209.02738646)(11.3190485,209.3101893)(10.86879789,209.8757957)
\curveto(10.32923855,210.55675447)(10.05945952,211.6656393)(10.05946,213.20245352)
\closepath
\moveto(11.09206351,213.20245352)
\curveto(11.09206201,211.85913598)(11.24927806,210.9651441)(11.56371215,210.52047519)
\curveto(11.87814228,210.07580358)(12.26606591,209.85346845)(12.72748422,209.85346914)
\curveto(13.18889624,209.85346845)(13.57681988,210.07673385)(13.89125629,210.52326601)
\curveto(14.2056841,210.96979546)(14.36290015,211.86285707)(14.36290493,213.20245352)
\curveto(14.36290015,214.54948408)(14.2056841,215.44440623)(13.89125629,215.88722267)
\curveto(13.57681988,216.33002566)(13.18517515,216.55143052)(12.71632094,216.5514379)
\curveto(12.25490264,216.55143052)(11.88651473,216.35607329)(11.61115609,215.96536564)
\curveto(11.26509269,215.46673277)(11.09206201,214.54576299)(11.09206351,213.20245352)
\closepath
}
}
{
\newrgbcolor{curcolor}{0 0 0}
\pscustom[linestyle=none,fillstyle=solid,fillcolor=curcolor]
{
\newpath
\moveto(16.42253048,213.20245352)
\curveto(16.42253001,214.16993289)(16.52206917,214.94857098)(16.72114826,215.53837013)
\curveto(16.9202258,216.12815653)(17.21605246,216.58305978)(17.60862912,216.90308126)
\curveto(18.00120246,217.22308727)(18.49517716,217.38309414)(19.09055471,217.38310236)
\curveto(19.52964019,217.38309414)(19.91477301,217.29471825)(20.24595432,217.11797443)
\curveto(20.57712704,216.9412147)(20.85062715,216.68632003)(21.0664555,216.35328966)
\curveto(21.2822736,216.02024492)(21.4515832,215.6146461)(21.57438479,215.136492)
\curveto(21.69717514,214.65832596)(21.75857312,214.01364711)(21.75857893,213.20245352)
\curveto(21.75857312,212.24240825)(21.65996424,211.46749125)(21.46275198,210.87770019)
\curveto(21.26552869,210.28790571)(20.97063231,209.83207218)(20.57806194,209.51019824)
\curveto(20.18548231,209.18832361)(19.68964706,209.02738646)(19.09055471,209.02738632)
\curveto(18.30168048,209.02738646)(17.68211899,209.3101893)(17.23186838,209.8757957)
\curveto(16.69230904,210.55675447)(16.42253001,211.6656393)(16.42253048,213.20245352)
\closepath
\moveto(17.455134,213.20245352)
\curveto(17.4551325,211.85913598)(17.61234855,210.9651441)(17.92678264,210.52047519)
\curveto(18.24121277,210.07580358)(18.6291364,209.85346845)(19.09055471,209.85346914)
\curveto(19.55196673,209.85346845)(19.93989037,210.07673385)(20.25432678,210.52326601)
\curveto(20.56875458,210.96979546)(20.72597064,211.86285707)(20.72597542,213.20245352)
\curveto(20.72597064,214.54948408)(20.56875458,215.44440623)(20.25432678,215.88722267)
\curveto(19.93989037,216.33002566)(19.54824564,216.55143052)(19.07939143,216.5514379)
\curveto(18.61797313,216.55143052)(18.24958522,216.35607329)(17.97422658,215.96536564)
\curveto(17.62816318,215.46673277)(17.4551325,214.54576299)(17.455134,213.20245352)
\closepath
}
}
{
\newrgbcolor{curcolor}{0 0 0}
\pscustom[linestyle=none,fillstyle=solid,fillcolor=curcolor]
{
\newpath
\moveto(22.79118166,211.32702226)
\lineto(23.79587697,211.46098164)
\curveto(23.91122928,210.89165257)(24.10751678,210.48140239)(24.38474006,210.23022988)
\curveto(24.6619592,209.97905524)(24.99964812,209.85346845)(25.39780784,209.85346914)
\curveto(25.87038319,209.85346845)(26.26947009,210.01719641)(26.59506975,210.34465351)
\curveto(26.92066085,210.67210826)(27.08345854,211.07770707)(27.08346331,211.56145117)
\curveto(27.08345854,212.02286394)(26.93275439,212.4033454)(26.63135042,212.70289668)
\curveto(26.32993781,213.00244089)(25.94666553,213.15221477)(25.48153245,213.15221875)
\curveto(25.29175368,213.15221477)(25.05546447,213.11500387)(24.77266409,213.04058594)
\lineto(24.8842969,213.92248516)
\curveto(24.95127395,213.91503823)(25.00522975,213.91131714)(25.04616448,213.91132188)
\curveto(25.4740871,213.91131714)(25.85921992,214.02294984)(26.20156409,214.24622032)
\curveto(26.54390048,214.46948064)(26.71507062,214.81368147)(26.71507503,215.27882384)
\curveto(26.71507062,215.64720564)(26.59041411,215.95233502)(26.34110511,216.1942129)
\curveto(26.09178804,216.43607673)(25.76991375,216.55701215)(25.37548127,216.55701954)
\curveto(24.98476376,216.55701215)(24.65916838,216.43421618)(24.39869416,216.18863126)
\curveto(24.13821577,215.9430323)(23.97076672,215.57464438)(23.89634651,215.08346641)
\lineto(22.89165119,215.26207892)
\curveto(23.01444658,215.93559012)(23.29352833,216.45747299)(23.72889729,216.82772911)
\curveto(24.1642634,217.19796991)(24.705682,217.38309414)(25.35315471,217.38310236)
\curveto(25.79968247,217.38309414)(26.21086292,217.28727607)(26.58669729,217.09564786)
\curveto(26.96252311,216.9040038)(27.24997732,216.64259722)(27.44906077,216.31142736)
\curveto(27.64813395,215.9802432)(27.74767311,215.62860019)(27.74767855,215.25649727)
\curveto(27.74767311,214.90298763)(27.65278531,214.58111334)(27.46301487,214.29087344)
\curveto(27.27323413,214.0006233)(26.99229183,213.76991571)(26.62018714,213.59875)
\curveto(27.10392453,213.48711287)(27.47975463,213.25547502)(27.74767855,212.90383575)
\curveto(28.01559159,212.552189)(28.14955083,212.1121701)(28.14955667,211.58377773)
\curveto(28.14955083,210.86932603)(27.88907453,210.26371863)(27.36812698,209.76695371)
\curveto(26.84716932,209.27018759)(26.18853638,209.02180483)(25.3922262,209.02180468)
\curveto(24.67405274,209.02180483)(24.07774806,209.2357675)(23.60331037,209.66369335)
\curveto(23.1288701,210.09161821)(22.8581608,210.64606063)(22.79118166,211.32702226)
\closepath
}
}
{
\newrgbcolor{curcolor}{0 0 0}
\pscustom[linestyle=none,fillstyle=solid,fillcolor=curcolor]
{
\newpath
\moveto(8.97662155,166.47251699)
\lineto(8.97662155,165.50689316)
\lineto(3.56801177,165.50689316)
\curveto(3.56056925,165.74876401)(3.59964069,165.98133214)(3.68522623,166.20459824)
\curveto(3.8229061,166.57298545)(4.04338068,166.93579173)(4.34665064,167.29301816)
\curveto(4.64991836,167.65024102)(5.08807671,168.06328201)(5.66112701,168.53214238)
\curveto(6.55046509,169.26147301)(7.15142113,169.83917223)(7.46399694,170.2652418)
\curveto(7.77656426,170.69130185)(7.93285004,171.09410985)(7.93285475,171.473667)
\curveto(7.93285004,171.87181767)(7.79051835,172.20764604)(7.50585924,172.48115314)
\curveto(7.22119157,172.75464628)(6.85001284,172.89139634)(6.39232194,172.89140372)
\curveto(5.90857706,172.89139634)(5.5215837,172.74627383)(5.23134068,172.45603575)
\curveto(4.94109365,172.16578378)(4.7941106,171.76390606)(4.79039107,171.25040138)
\lineto(3.75778756,171.35645255)
\curveto(3.82848773,172.12671234)(4.09454567,172.71371429)(4.55596217,173.11746017)
\curveto(5.017376,173.52119083)(5.63693749,173.72305996)(6.4146485,173.72306818)
\curveto(7.1997953,173.72305996)(7.82121734,173.50537619)(8.27891647,173.07001622)
\curveto(8.73660549,172.63464113)(8.96545253,172.09508307)(8.96545827,171.45134044)
\curveto(8.96545253,171.12387857)(8.89847291,170.80200428)(8.76451921,170.48571661)
\curveto(8.63055442,170.16941898)(8.40821929,169.83638142)(8.09751315,169.48660293)
\curveto(7.78679726,169.13681649)(7.27049602,168.65679588)(6.54860787,168.04653965)
\curveto(5.94578796,167.54046886)(5.5587946,167.19719831)(5.38762662,167.01672695)
\curveto(5.21645431,166.83625257)(5.07505289,166.65484943)(4.96342193,166.47251699)
\closepath
}
}
{
\newrgbcolor{curcolor}{0 0 0}
\pscustom[linestyle=none,fillstyle=solid,fillcolor=curcolor]
{
\newpath
\moveto(10.05946,169.54241934)
\curveto(10.05945952,170.50989871)(10.15899868,171.2885368)(10.35807777,171.87833595)
\curveto(10.55715531,172.46812235)(10.85298197,172.9230256)(11.24555863,173.24304708)
\curveto(11.63813197,173.56305309)(12.13210667,173.72305996)(12.72748422,173.72306818)
\curveto(13.1665697,173.72305996)(13.55170252,173.63468407)(13.88288383,173.45794025)
\curveto(14.21405655,173.28118052)(14.48755667,173.02628585)(14.70338501,172.69325548)
\curveto(14.91920311,172.36021074)(15.08851271,171.95461192)(15.21131431,171.47645782)
\curveto(15.33410465,170.99829178)(15.39550264,170.35361293)(15.39550845,169.54241934)
\curveto(15.39550264,168.58237407)(15.29689375,167.80745707)(15.09968149,167.21766601)
\curveto(14.90245821,166.62787153)(14.60756182,166.172038)(14.21499145,165.85016406)
\curveto(13.82241182,165.52828943)(13.32657658,165.36735228)(12.72748422,165.36735214)
\curveto(11.93860999,165.36735228)(11.3190485,165.65015512)(10.86879789,166.21576152)
\curveto(10.32923855,166.89672029)(10.05945952,168.00560512)(10.05946,169.54241934)
\closepath
\moveto(11.09206351,169.54241934)
\curveto(11.09206201,168.1991018)(11.24927806,167.30510992)(11.56371215,166.86044101)
\curveto(11.87814228,166.4157694)(12.26606591,166.19343427)(12.72748422,166.19343496)
\curveto(13.18889624,166.19343427)(13.57681988,166.41669967)(13.89125629,166.86323183)
\curveto(14.2056841,167.30976128)(14.36290015,168.20282289)(14.36290493,169.54241934)
\curveto(14.36290015,170.8894499)(14.2056841,171.78437205)(13.89125629,172.22718849)
\curveto(13.57681988,172.66999148)(13.18517515,172.89139634)(12.71632094,172.89140372)
\curveto(12.25490264,172.89139634)(11.88651473,172.69603911)(11.61115609,172.30533146)
\curveto(11.26509269,171.80669859)(11.09206201,170.88572881)(11.09206351,169.54241934)
\closepath
}
}
{
\newrgbcolor{curcolor}{0 0 0}
\pscustom[linestyle=none,fillstyle=solid,fillcolor=curcolor]
{
\newpath
\moveto(16.42253048,169.54241934)
\curveto(16.42253001,170.50989871)(16.52206917,171.2885368)(16.72114826,171.87833595)
\curveto(16.9202258,172.46812235)(17.21605246,172.9230256)(17.60862912,173.24304708)
\curveto(18.00120246,173.56305309)(18.49517716,173.72305996)(19.09055471,173.72306818)
\curveto(19.52964019,173.72305996)(19.91477301,173.63468407)(20.24595432,173.45794025)
\curveto(20.57712704,173.28118052)(20.85062715,173.02628585)(21.0664555,172.69325548)
\curveto(21.2822736,172.36021074)(21.4515832,171.95461192)(21.57438479,171.47645782)
\curveto(21.69717514,170.99829178)(21.75857312,170.35361293)(21.75857893,169.54241934)
\curveto(21.75857312,168.58237407)(21.65996424,167.80745707)(21.46275198,167.21766601)
\curveto(21.26552869,166.62787153)(20.97063231,166.172038)(20.57806194,165.85016406)
\curveto(20.18548231,165.52828943)(19.68964706,165.36735228)(19.09055471,165.36735214)
\curveto(18.30168048,165.36735228)(17.68211899,165.65015512)(17.23186838,166.21576152)
\curveto(16.69230904,166.89672029)(16.42253001,168.00560512)(16.42253048,169.54241934)
\closepath
\moveto(17.455134,169.54241934)
\curveto(17.4551325,168.1991018)(17.61234855,167.30510992)(17.92678264,166.86044101)
\curveto(18.24121277,166.4157694)(18.6291364,166.19343427)(19.09055471,166.19343496)
\curveto(19.55196673,166.19343427)(19.93989037,166.41669967)(20.25432678,166.86323183)
\curveto(20.56875458,167.30976128)(20.72597064,168.20282289)(20.72597542,169.54241934)
\curveto(20.72597064,170.8894499)(20.56875458,171.78437205)(20.25432678,172.22718849)
\curveto(19.93989037,172.66999148)(19.54824564,172.89139634)(19.07939143,172.89140372)
\curveto(18.61797313,172.89139634)(18.24958522,172.69603911)(17.97422658,172.30533146)
\curveto(17.62816318,171.80669859)(17.4551325,170.88572881)(17.455134,169.54241934)
\closepath
}
}
{
\newrgbcolor{curcolor}{0 0 0}
\pscustom[linestyle=none,fillstyle=solid,fillcolor=curcolor]
{
\newpath
\moveto(26.00620667,165.50689316)
\lineto(26.00620667,167.46604902)
\lineto(22.45628322,167.46604902)
\lineto(22.45628322,168.38701973)
\lineto(26.19040081,173.68957833)
\lineto(27.01090198,173.68957833)
\lineto(27.01090198,168.38701973)
\lineto(28.11606683,168.38701973)
\lineto(28.11606683,167.46604902)
\lineto(27.01090198,167.46604902)
\lineto(27.01090198,165.50689316)
\closepath
\moveto(26.00620667,168.38701973)
\lineto(26.00620667,172.07648419)
\lineto(23.44423361,168.38701973)
\closepath
}
}
{
\newrgbcolor{curcolor}{0 0 0}
\pscustom[linestyle=none,fillstyle=solid,fillcolor=curcolor]
{
\newpath
\moveto(8.97662155,122.80247304)
\lineto(8.97662155,121.83684921)
\lineto(3.56801177,121.83684921)
\curveto(3.56056925,122.07872007)(3.59964069,122.31128819)(3.68522623,122.53455429)
\curveto(3.8229061,122.90294151)(4.04338068,123.26574779)(4.34665064,123.62297422)
\curveto(4.64991836,123.98019707)(5.08807671,124.39323807)(5.66112701,124.86209844)
\curveto(6.55046509,125.59142906)(7.15142113,126.16912829)(7.46399694,126.59519786)
\curveto(7.77656426,127.02125791)(7.93285004,127.4240659)(7.93285475,127.80362306)
\curveto(7.93285004,128.20177372)(7.79051835,128.5376021)(7.50585924,128.81110919)
\curveto(7.22119157,129.08460233)(6.85001284,129.22135239)(6.39232194,129.22135978)
\curveto(5.90857706,129.22135239)(5.5215837,129.07622988)(5.23134068,128.78599181)
\curveto(4.94109365,128.49573984)(4.7941106,128.09386211)(4.79039107,127.58035743)
\lineto(3.75778756,127.6864086)
\curveto(3.82848773,128.45666839)(4.09454567,129.04367034)(4.55596217,129.44741622)
\curveto(5.017376,129.85114688)(5.63693749,130.05301602)(6.4146485,130.05302423)
\curveto(7.1997953,130.05301602)(7.82121734,129.83533225)(8.27891647,129.39997228)
\curveto(8.73660549,128.96459718)(8.96545253,128.42503912)(8.96545827,127.78129649)
\curveto(8.96545253,127.45383463)(8.89847291,127.13196034)(8.76451921,126.81567266)
\curveto(8.63055442,126.49937503)(8.40821929,126.16633747)(8.09751315,125.81655899)
\curveto(7.78679726,125.46677254)(7.27049602,124.98675193)(6.54860787,124.3764957)
\curveto(5.94578796,123.87042492)(5.5587946,123.52715436)(5.38762662,123.34668301)
\curveto(5.21645431,123.16620863)(5.07505289,122.98480549)(4.96342193,122.80247304)
\closepath
}
}
{
\newrgbcolor{curcolor}{0 0 0}
\pscustom[linestyle=none,fillstyle=solid,fillcolor=curcolor]
{
\newpath
\moveto(10.05946,125.87237539)
\curveto(10.05945952,126.83985477)(10.15899868,127.61849286)(10.35807777,128.208292)
\curveto(10.55715531,128.7980784)(10.85298197,129.25298166)(11.24555863,129.57300314)
\curveto(11.63813197,129.89300914)(12.13210667,130.05301602)(12.72748422,130.05302423)
\curveto(13.1665697,130.05301602)(13.55170252,129.96464013)(13.88288383,129.7878963)
\curveto(14.21405655,129.61113657)(14.48755667,129.35624191)(14.70338501,129.02321153)
\curveto(14.91920311,128.69016679)(15.08851271,128.28456798)(15.21131431,127.80641388)
\curveto(15.33410465,127.32824784)(15.39550264,126.68356899)(15.39550845,125.87237539)
\curveto(15.39550264,124.91233013)(15.29689375,124.13741313)(15.09968149,123.54762207)
\curveto(14.90245821,122.95782759)(14.60756182,122.50199406)(14.21499145,122.18012011)
\curveto(13.82241182,121.85824548)(13.32657658,121.69730834)(12.72748422,121.6973082)
\curveto(11.93860999,121.69730834)(11.3190485,121.98011118)(10.86879789,122.54571757)
\curveto(10.32923855,123.22667634)(10.05945952,124.33556117)(10.05946,125.87237539)
\closepath
\moveto(11.09206351,125.87237539)
\curveto(11.09206201,124.52905786)(11.24927806,123.63506597)(11.56371215,123.19039707)
\curveto(11.87814228,122.74572545)(12.26606591,122.52339032)(12.72748422,122.52339101)
\curveto(13.18889624,122.52339032)(13.57681988,122.74665573)(13.89125629,123.19318789)
\curveto(14.2056841,123.63971734)(14.36290015,124.53277895)(14.36290493,125.87237539)
\curveto(14.36290015,127.21940595)(14.2056841,128.11432811)(13.89125629,128.55714454)
\curveto(13.57681988,128.99994754)(13.18517515,129.22135239)(12.71632094,129.22135978)
\curveto(12.25490264,129.22135239)(11.88651473,129.02599517)(11.61115609,128.63528751)
\curveto(11.26509269,128.13665465)(11.09206201,127.21568486)(11.09206351,125.87237539)
\closepath
}
}
{
\newrgbcolor{curcolor}{0 0 0}
\pscustom[linestyle=none,fillstyle=solid,fillcolor=curcolor]
{
\newpath
\moveto(16.42253048,125.87237539)
\curveto(16.42253001,126.83985477)(16.52206917,127.61849286)(16.72114826,128.208292)
\curveto(16.9202258,128.7980784)(17.21605246,129.25298166)(17.60862912,129.57300314)
\curveto(18.00120246,129.89300914)(18.49517716,130.05301602)(19.09055471,130.05302423)
\curveto(19.52964019,130.05301602)(19.91477301,129.96464013)(20.24595432,129.7878963)
\curveto(20.57712704,129.61113657)(20.85062715,129.35624191)(21.0664555,129.02321153)
\curveto(21.2822736,128.69016679)(21.4515832,128.28456798)(21.57438479,127.80641388)
\curveto(21.69717514,127.32824784)(21.75857312,126.68356899)(21.75857893,125.87237539)
\curveto(21.75857312,124.91233013)(21.65996424,124.13741313)(21.46275198,123.54762207)
\curveto(21.26552869,122.95782759)(20.97063231,122.50199406)(20.57806194,122.18012011)
\curveto(20.18548231,121.85824548)(19.68964706,121.69730834)(19.09055471,121.6973082)
\curveto(18.30168048,121.69730834)(17.68211899,121.98011118)(17.23186838,122.54571757)
\curveto(16.69230904,123.22667634)(16.42253001,124.33556117)(16.42253048,125.87237539)
\closepath
\moveto(17.455134,125.87237539)
\curveto(17.4551325,124.52905786)(17.61234855,123.63506597)(17.92678264,123.19039707)
\curveto(18.24121277,122.74572545)(18.6291364,122.52339032)(19.09055471,122.52339101)
\curveto(19.55196673,122.52339032)(19.93989037,122.74665573)(20.25432678,123.19318789)
\curveto(20.56875458,123.63971734)(20.72597064,124.53277895)(20.72597542,125.87237539)
\curveto(20.72597064,127.21940595)(20.56875458,128.11432811)(20.25432678,128.55714454)
\curveto(19.93989037,128.99994754)(19.54824564,129.22135239)(19.07939143,129.22135978)
\curveto(18.61797313,129.22135239)(18.24958522,129.02599517)(17.97422658,128.63528751)
\curveto(17.62816318,128.13665465)(17.4551325,127.21568486)(17.455134,125.87237539)
\closepath
}
}
{
\newrgbcolor{curcolor}{0 0 0}
\pscustom[linestyle=none,fillstyle=solid,fillcolor=curcolor]
{
\newpath
\moveto(22.78560002,123.98019922)
\lineto(23.8405301,124.06950547)
\curveto(23.91867146,123.55599281)(24.1000746,123.16992972)(24.38474006,122.91131504)
\curveto(24.66940138,122.6526982)(25.01267193,122.52339032)(25.41455276,122.52339101)
\curveto(25.89829136,122.52339032)(26.30761127,122.70572374)(26.6425137,123.07039179)
\curveto(26.97740747,123.43505738)(27.14485652,123.91879909)(27.14486136,124.52161836)
\curveto(27.14485652,125.09466354)(26.98391938,125.54677598)(26.66204944,125.87795704)
\curveto(26.3401708,126.20913001)(25.91875736,126.37471851)(25.39780784,126.37472305)
\curveto(25.07406992,126.37471851)(24.78196435,126.30122699)(24.52149026,126.15424825)
\curveto(24.26101174,126.00726087)(24.05635179,125.81655501)(23.90750979,125.58213008)
\lineto(22.96421252,125.70492618)
\lineto(23.75680549,129.90790158)
\lineto(27.82582152,129.90790158)
\lineto(27.82582152,128.94785939)
\lineto(24.56056174,128.94785939)
\lineto(24.11961213,126.74869297)
\curveto(24.61079421,127.09102835)(25.12616518,127.26219849)(25.66572659,127.26220391)
\curveto(26.38017252,127.26219849)(26.98298911,127.014746)(27.47417815,126.51984571)
\curveto(27.96535688,126.02493605)(28.21094882,125.38862965)(28.21095472,124.61092461)
\curveto(28.21094882,123.87042492)(27.9951256,123.23039743)(27.56348441,122.69084023)
\curveto(27.03880546,122.02848535)(26.32249563,121.69730834)(25.41455276,121.6973082)
\curveto(24.67033165,121.69730834)(24.0628637,121.90568938)(23.59214709,122.32245195)
\curveto(23.12142792,122.73921355)(22.85257916,123.29179542)(22.78560002,123.98019922)
\closepath
}
}
{
\newrgbcolor{curcolor}{0 0 0}
\pscustom[linestyle=none,fillstyle=solid,fillcolor=curcolor]
{
\newpath
\moveto(8.97662155,79.13255117)
\lineto(8.97662155,78.16692734)
\lineto(3.56801177,78.16692734)
\curveto(3.56056925,78.40879819)(3.59964069,78.64136632)(3.68522623,78.86463242)
\curveto(3.8229061,79.23301963)(4.04338068,79.59582591)(4.34665064,79.95305234)
\curveto(4.64991836,80.3102752)(5.08807671,80.72331619)(5.66112701,81.19217656)
\curveto(6.55046509,81.92150719)(7.15142113,82.49920641)(7.46399694,82.92527598)
\curveto(7.77656426,83.35133603)(7.93285004,83.75414403)(7.93285475,84.13370118)
\curveto(7.93285004,84.53185185)(7.79051835,84.86768022)(7.50585924,85.14118732)
\curveto(7.22119157,85.41468046)(6.85001284,85.55143052)(6.39232194,85.5514379)
\curveto(5.90857706,85.55143052)(5.5215837,85.40630801)(5.23134068,85.11606993)
\curveto(4.94109365,84.82581796)(4.7941106,84.42394024)(4.79039107,83.91043555)
\lineto(3.75778756,84.01648673)
\curveto(3.82848773,84.78674652)(4.09454567,85.37374847)(4.55596217,85.77749435)
\curveto(5.017376,86.18122501)(5.63693749,86.38309414)(6.4146485,86.38310236)
\curveto(7.1997953,86.38309414)(7.82121734,86.16541037)(8.27891647,85.7300504)
\curveto(8.73660549,85.29467531)(8.96545253,84.75511725)(8.96545827,84.11137462)
\curveto(8.96545253,83.78391275)(8.89847291,83.46203846)(8.76451921,83.14575079)
\curveto(8.63055442,82.82945316)(8.40821929,82.4964156)(8.09751315,82.14663711)
\curveto(7.78679726,81.79685067)(7.27049602,81.31683005)(6.54860787,80.70657383)
\curveto(5.94578796,80.20050304)(5.5587946,79.85723249)(5.38762662,79.67676113)
\curveto(5.21645431,79.49628675)(5.07505289,79.31488361)(4.96342193,79.13255117)
\closepath
}
}
{
\newrgbcolor{curcolor}{0 0 0}
\pscustom[linestyle=none,fillstyle=solid,fillcolor=curcolor]
{
\newpath
\moveto(10.05946,82.20245352)
\curveto(10.05945952,83.16993289)(10.15899868,83.94857098)(10.35807777,84.53837013)
\curveto(10.55715531,85.12815653)(10.85298197,85.58305978)(11.24555863,85.90308126)
\curveto(11.63813197,86.22308727)(12.13210667,86.38309414)(12.72748422,86.38310236)
\curveto(13.1665697,86.38309414)(13.55170252,86.29471825)(13.88288383,86.11797443)
\curveto(14.21405655,85.9412147)(14.48755667,85.68632003)(14.70338501,85.35328966)
\curveto(14.91920311,85.02024492)(15.08851271,84.6146461)(15.21131431,84.136492)
\curveto(15.33410465,83.65832596)(15.39550264,83.01364711)(15.39550845,82.20245352)
\curveto(15.39550264,81.24240825)(15.29689375,80.46749125)(15.09968149,79.87770019)
\curveto(14.90245821,79.28790571)(14.60756182,78.83207218)(14.21499145,78.51019824)
\curveto(13.82241182,78.18832361)(13.32657658,78.02738646)(12.72748422,78.02738632)
\curveto(11.93860999,78.02738646)(11.3190485,78.3101893)(10.86879789,78.8757957)
\curveto(10.32923855,79.55675447)(10.05945952,80.6656393)(10.05946,82.20245352)
\closepath
\moveto(11.09206351,82.20245352)
\curveto(11.09206201,80.85913598)(11.24927806,79.9651441)(11.56371215,79.52047519)
\curveto(11.87814228,79.07580358)(12.26606591,78.85346845)(12.72748422,78.85346914)
\curveto(13.18889624,78.85346845)(13.57681988,79.07673385)(13.89125629,79.52326601)
\curveto(14.2056841,79.96979546)(14.36290015,80.86285707)(14.36290493,82.20245352)
\curveto(14.36290015,83.54948408)(14.2056841,84.44440623)(13.89125629,84.88722267)
\curveto(13.57681988,85.33002566)(13.18517515,85.55143052)(12.71632094,85.5514379)
\curveto(12.25490264,85.55143052)(11.88651473,85.35607329)(11.61115609,84.96536564)
\curveto(11.26509269,84.46673277)(11.09206201,83.54576299)(11.09206351,82.20245352)
\closepath
}
}
{
\newrgbcolor{curcolor}{0 0 0}
\pscustom[linestyle=none,fillstyle=solid,fillcolor=curcolor]
{
\newpath
\moveto(16.42253048,82.20245352)
\curveto(16.42253001,83.16993289)(16.52206917,83.94857098)(16.72114826,84.53837013)
\curveto(16.9202258,85.12815653)(17.21605246,85.58305978)(17.60862912,85.90308126)
\curveto(18.00120246,86.22308727)(18.49517716,86.38309414)(19.09055471,86.38310236)
\curveto(19.52964019,86.38309414)(19.91477301,86.29471825)(20.24595432,86.11797443)
\curveto(20.57712704,85.9412147)(20.85062715,85.68632003)(21.0664555,85.35328966)
\curveto(21.2822736,85.02024492)(21.4515832,84.6146461)(21.57438479,84.136492)
\curveto(21.69717514,83.65832596)(21.75857312,83.01364711)(21.75857893,82.20245352)
\curveto(21.75857312,81.24240825)(21.65996424,80.46749125)(21.46275198,79.87770019)
\curveto(21.26552869,79.28790571)(20.97063231,78.83207218)(20.57806194,78.51019824)
\curveto(20.18548231,78.18832361)(19.68964706,78.02738646)(19.09055471,78.02738632)
\curveto(18.30168048,78.02738646)(17.68211899,78.3101893)(17.23186838,78.8757957)
\curveto(16.69230904,79.55675447)(16.42253001,80.6656393)(16.42253048,82.20245352)
\closepath
\moveto(17.455134,82.20245352)
\curveto(17.4551325,80.85913598)(17.61234855,79.9651441)(17.92678264,79.52047519)
\curveto(18.24121277,79.07580358)(18.6291364,78.85346845)(19.09055471,78.85346914)
\curveto(19.55196673,78.85346845)(19.93989037,79.07673385)(20.25432678,79.52326601)
\curveto(20.56875458,79.96979546)(20.72597064,80.86285707)(20.72597542,82.20245352)
\curveto(20.72597064,83.54948408)(20.56875458,84.44440623)(20.25432678,84.88722267)
\curveto(19.93989037,85.33002566)(19.54824564,85.55143052)(19.07939143,85.5514379)
\curveto(18.61797313,85.55143052)(18.24958522,85.35607329)(17.97422658,84.96536564)
\curveto(17.62816318,84.46673277)(17.4551325,83.54576299)(17.455134,82.20245352)
\closepath
}
}
{
\newrgbcolor{curcolor}{0 0 0}
\pscustom[linestyle=none,fillstyle=solid,fillcolor=curcolor]
{
\newpath
\moveto(27.99885237,84.34580352)
\lineto(26.9997387,84.26766056)
\curveto(26.91042785,84.66209)(26.78391079,84.94861393)(26.62018714,85.12723321)
\curveto(26.34854326,85.41375019)(26.01364515,85.55701215)(25.61549182,85.55701954)
\curveto(25.29547477,85.55701215)(25.01453248,85.46770599)(24.77266409,85.28910079)
\curveto(24.45636897,85.05838609)(24.20705594,84.72162744)(24.02472424,84.27882384)
\curveto(23.84238911,83.83600801)(23.74750132,83.20528325)(23.74006057,82.38664766)
\curveto(23.98192999,82.75503135)(24.27775665,83.02853147)(24.62754143,83.20714883)
\curveto(24.97732158,83.38575612)(25.34384895,83.47506228)(25.72712464,83.47506759)
\curveto(26.39691743,83.47506228)(26.96717447,83.22854006)(27.43789749,82.7355002)
\curveto(27.90861025,82.2424512)(28.1439692,81.60521453)(28.14397503,80.82378828)
\curveto(28.1439692,80.3102752)(28.03326677,79.8330454)(27.81186741,79.39209746)
\curveto(27.59045706,78.95114706)(27.28625795,78.61345814)(26.89926917,78.37902968)
\curveto(26.51227122,78.1446008)(26.07318259,78.02738646)(25.58200198,78.02738632)
\curveto(24.74475345,78.02738646)(24.06193343,78.33530666)(23.53353986,78.95114785)
\curveto(23.00514386,79.56698746)(22.74094646,80.58191477)(22.74094689,81.99593282)
\curveto(22.74094646,83.57739225)(23.03305203,84.72720907)(23.61726447,85.44538673)
\curveto(24.1270525,86.07052258)(24.81359361,86.38309414)(25.67688987,86.38310236)
\curveto(26.32063508,86.38309414)(26.84809959,86.20262127)(27.25928499,85.84168322)
\curveto(27.67046049,85.48072981)(27.91698271,84.98210374)(27.99885237,84.34580352)
\closepath
\moveto(23.89634651,80.81820664)
\curveto(23.89634492,80.47214262)(23.96983645,80.1409656)(24.11682131,79.82467461)
\curveto(24.26380256,79.5083803)(24.46939279,79.26743972)(24.7335926,79.10185214)
\curveto(24.99778757,78.9362627)(25.27500878,78.85346845)(25.56525706,78.85346914)
\curveto(25.98945807,78.85346845)(26.35412489,79.02463859)(26.65925862,79.36698008)
\curveto(26.96438366,79.70931916)(27.11694835,80.17445541)(27.11695315,80.76239023)
\curveto(27.11694835,81.32799333)(26.9662442,81.77359386)(26.66484026,82.09919317)
\curveto(26.36342762,82.42478461)(25.98387643,82.5875823)(25.52618557,82.58758672)
\curveto(25.07220937,82.5875823)(24.68707655,82.42478461)(24.37078596,82.09919317)
\curveto(24.05449125,81.77359386)(23.89634492,81.34659878)(23.89634651,80.81820664)
\closepath
}
}
{
\newrgbcolor{curcolor}{0 0 0}
\pscustom[linestyle=none,fillstyle=solid,fillcolor=curcolor]
{
\newpath
\moveto(8.97662155,35.46250722)
\lineto(8.97662155,34.49688339)
\lineto(3.56801177,34.49688339)
\curveto(3.56056925,34.73875425)(3.59964069,34.97132237)(3.68522623,35.19458847)
\curveto(3.8229061,35.56297569)(4.04338068,35.92578197)(4.34665064,36.2830084)
\curveto(4.64991836,36.64023125)(5.08807671,37.05327225)(5.66112701,37.52213262)
\curveto(6.55046509,38.25146324)(7.15142113,38.82916247)(7.46399694,39.25523204)
\curveto(7.77656426,39.68129209)(7.93285004,40.08410008)(7.93285475,40.46365724)
\curveto(7.93285004,40.8618079)(7.79051835,41.19763628)(7.50585924,41.47114337)
\curveto(7.22119157,41.74463651)(6.85001284,41.88138657)(6.39232194,41.88139396)
\curveto(5.90857706,41.88138657)(5.5215837,41.73626406)(5.23134068,41.44602599)
\curveto(4.94109365,41.15577402)(4.7941106,40.75389629)(4.79039107,40.24039161)
\lineto(3.75778756,40.34644278)
\curveto(3.82848773,41.11670257)(4.09454567,41.70370452)(4.55596217,42.1074504)
\curveto(5.017376,42.51118106)(5.63693749,42.7130502)(6.4146485,42.71305841)
\curveto(7.1997953,42.7130502)(7.82121734,42.49536643)(8.27891647,42.06000646)
\curveto(8.73660549,41.62463136)(8.96545253,41.0850733)(8.96545827,40.44133067)
\curveto(8.96545253,40.1138688)(8.89847291,39.79199452)(8.76451921,39.47570684)
\curveto(8.63055442,39.15940921)(8.40821929,38.82637165)(8.09751315,38.47659317)
\curveto(7.78679726,38.12680672)(7.27049602,37.64678611)(6.54860787,37.03652988)
\curveto(5.94578796,36.5304591)(5.5587946,36.18718854)(5.38762662,36.00671719)
\curveto(5.21645431,35.82624281)(5.07505289,35.64483967)(4.96342193,35.46250722)
\closepath
}
}
{
\newrgbcolor{curcolor}{0 0 0}
\pscustom[linestyle=none,fillstyle=solid,fillcolor=curcolor]
{
\newpath
\moveto(10.05946,38.53240957)
\curveto(10.05945952,39.49988895)(10.15899868,40.27852704)(10.35807777,40.86832618)
\curveto(10.55715531,41.45811258)(10.85298197,41.91301584)(11.24555863,42.23303732)
\curveto(11.63813197,42.55304332)(12.13210667,42.7130502)(12.72748422,42.71305841)
\curveto(13.1665697,42.7130502)(13.55170252,42.62467431)(13.88288383,42.44793048)
\curveto(14.21405655,42.27117075)(14.48755667,42.01627609)(14.70338501,41.68324571)
\curveto(14.91920311,41.35020097)(15.08851271,40.94460216)(15.21131431,40.46644806)
\curveto(15.33410465,39.98828202)(15.39550264,39.34360317)(15.39550845,38.53240957)
\curveto(15.39550264,37.57236431)(15.29689375,36.79744731)(15.09968149,36.20765625)
\curveto(14.90245821,35.61786177)(14.60756182,35.16202824)(14.21499145,34.84015429)
\curveto(13.82241182,34.51827966)(13.32657658,34.35734252)(12.72748422,34.35734238)
\curveto(11.93860999,34.35734252)(11.3190485,34.64014536)(10.86879789,35.20575175)
\curveto(10.32923855,35.88671052)(10.05945952,36.99559535)(10.05946,38.53240957)
\closepath
\moveto(11.09206351,38.53240957)
\curveto(11.09206201,37.18909204)(11.24927806,36.29510015)(11.56371215,35.85043125)
\curveto(11.87814228,35.40575963)(12.26606591,35.1834245)(12.72748422,35.18342519)
\curveto(13.18889624,35.1834245)(13.57681988,35.40668991)(13.89125629,35.85322207)
\curveto(14.2056841,36.29975152)(14.36290015,37.19281313)(14.36290493,38.53240957)
\curveto(14.36290015,39.87944013)(14.2056841,40.77436229)(13.89125629,41.21717872)
\curveto(13.57681988,41.65998172)(13.18517515,41.88138657)(12.71632094,41.88139396)
\curveto(12.25490264,41.88138657)(11.88651473,41.68602935)(11.61115609,41.29532169)
\curveto(11.26509269,40.79668883)(11.09206201,39.87571904)(11.09206351,38.53240957)
\closepath
}
}
{
\newrgbcolor{curcolor}{0 0 0}
\pscustom[linestyle=none,fillstyle=solid,fillcolor=curcolor]
{
\newpath
\moveto(16.42253048,38.53240957)
\curveto(16.42253001,39.49988895)(16.52206917,40.27852704)(16.72114826,40.86832618)
\curveto(16.9202258,41.45811258)(17.21605246,41.91301584)(17.60862912,42.23303732)
\curveto(18.00120246,42.55304332)(18.49517716,42.7130502)(19.09055471,42.71305841)
\curveto(19.52964019,42.7130502)(19.91477301,42.62467431)(20.24595432,42.44793048)
\curveto(20.57712704,42.27117075)(20.85062715,42.01627609)(21.0664555,41.68324571)
\curveto(21.2822736,41.35020097)(21.4515832,40.94460216)(21.57438479,40.46644806)
\curveto(21.69717514,39.98828202)(21.75857312,39.34360317)(21.75857893,38.53240957)
\curveto(21.75857312,37.57236431)(21.65996424,36.79744731)(21.46275198,36.20765625)
\curveto(21.26552869,35.61786177)(20.97063231,35.16202824)(20.57806194,34.84015429)
\curveto(20.18548231,34.51827966)(19.68964706,34.35734252)(19.09055471,34.35734238)
\curveto(18.30168048,34.35734252)(17.68211899,34.64014536)(17.23186838,35.20575175)
\curveto(16.69230904,35.88671052)(16.42253001,36.99559535)(16.42253048,38.53240957)
\closepath
\moveto(17.455134,38.53240957)
\curveto(17.4551325,37.18909204)(17.61234855,36.29510015)(17.92678264,35.85043125)
\curveto(18.24121277,35.40575963)(18.6291364,35.1834245)(19.09055471,35.18342519)
\curveto(19.55196673,35.1834245)(19.93989037,35.40668991)(20.25432678,35.85322207)
\curveto(20.56875458,36.29975152)(20.72597064,37.19281313)(20.72597542,38.53240957)
\curveto(20.72597064,39.87944013)(20.56875458,40.77436229)(20.25432678,41.21717872)
\curveto(19.93989037,41.65998172)(19.54824564,41.88138657)(19.07939143,41.88139396)
\curveto(18.61797313,41.88138657)(18.24958522,41.68602935)(17.97422658,41.29532169)
\curveto(17.62816318,40.79668883)(17.4551325,39.87571904)(17.455134,38.53240957)
\closepath
}
}
{
\newrgbcolor{curcolor}{0 0 0}
\pscustom[linestyle=none,fillstyle=solid,fillcolor=curcolor]
{
\newpath
\moveto(22.85257971,41.60789357)
\lineto(22.85257971,42.5735174)
\lineto(28.14955667,42.5735174)
\lineto(28.14955667,41.79208771)
\curveto(27.62859823,41.237638)(27.11229699,40.50086217)(26.6006514,39.58175801)
\curveto(26.08899723,38.66264369)(25.69363141,37.71748682)(25.41455276,36.74628457)
\curveto(25.21361079,36.06160175)(25.08523319,35.31180211)(25.02941956,34.49688339)
\lineto(23.99681604,34.49688339)
\curveto(24.00797762,35.14063197)(24.13449468,35.91833979)(24.3763676,36.83000918)
\curveto(24.61823639,37.74167391)(24.96522803,38.62078143)(25.41734358,39.46733438)
\curveto(25.86945291,40.31387739)(26.3504038,41.02739641)(26.86019768,41.60789357)
\closepath
}
}
{
\newrgbcolor{curcolor}{0 0 0}
\pscustom[linewidth=0.158766,linecolor=curcolor]
{
\newpath
\moveto(969.97258,858.64213033)
\lineto(969.97258,858.64213033)
}
}
\end{pspicture}
\end{pdfpic}

\label{fig:unix_genealogy}
\caption{Petit historique des UNIX libres (image Wikipedia)}
\end{figure}

\section{Le langage C et la cr�ation d'UNIX}

\section{POSIX et la portabilit�}

\chapter{Un petit passage par Windows}

\section{Microsoft DOS}

\paragraph{}
Anecdote sur IBM cherchant un fournisseur d'OS pour son nouveau produit : le PC

\paragraph{}
Le premier noyau de Microsoft, bas� sur DOS

\section{Les noyaux NT}

\section{Vista et 7}

\paragraph{}
Une architecture de type micro-noyau
