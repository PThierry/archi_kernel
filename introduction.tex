%%
%%
%% introduction.tex for cours in /doctorat/ece/partenariat/cours/archi_kernel
%%
%% Made by Philippe THIERRY
%% Login   <Philippe THIERRYreseau-libre.net>
%%
%% Started on  Mon Sep  6 16:16:07 2010 Philippe THIERRY
%% Last update Tue Oct  5 16:13:14 2010 Philippe THIERRY
%%

\chapter{Introduction}


\paragraph{}
Ce cours a pour but de pr�senter l'architecture et le syst�me de production du noyau Linux, afin
d'avoir les bases n�cessaires pour l'impl�mentation de modules noyau.\\
Afin de permettre aux �l�ves de facilement s'orienter dans l'architecture d'un noyau monolithique
tel que Linux, ce cours propose en premier lieu un historique des noyaux et d�crit les diff�rentes
architectures existantes.

\paragraph{}
Afin de bien comprendre les arcanes du noyau Linux et de son syst�me de production, il est conseill�
d'avoir de bonnes notions de langage C et �tre habitu� aux syst�mes de Makefile. Le cours sur les
outils GNU est un bon point de d�part pour se former � ces outils de production. Il est consid�r�
dans ce cours que les �l�ves poss�dent d�j� ces connaissances.
